% -*- TeX-master: "agda-hsp.tex" -*-
\documentclass[a4paper,UKenglish,cleveref,autoref,thm-restate]{lipics-v2021}
\usepackage{wjd}
% \usepackage{fontspec}
\usepackage{comment}
\usepackage{proof-dashed}
\usepackage{unixode}

\usepackage[dvipsnames]{xcolor}
\definecolor{britishracinggreen}{rgb}{0.0, 0.26, 0.15}
\hypersetup{
    bookmarks=true,         % show bookmarks bar?
    unicode=true,          % non-Latin characters in Acrobat’s bookmarks
    pdftoolbar=true,        % show Acrobat’s toolbar?
    pdfmenubar=true,        % show Acrobat’s menu?
    pdffitwindow=false,     % window fit to page when opened
%    pdfstartview={FitW},    % fits the width of the page to the window
    % pdftitle={A Machine-checked Proof of Birkhoff's Variety Theorem in Martin-L\"of Type Theory},    % title
    pdftitle={Birkhoff's Variety Theorem Formalized in Agda},    % title
    pdfauthor={William DeMeo},     % author
    pdfsubject={},   % subject of the document
    pdfcreator={pdflatex with hyperref},
    pdfproducer={},  % producer of the document
    pdfkeywords= {Agda} {Birkhoff’s HSP theorem} {equational logic}
    {Formalization of mathematics} {Martin-Löf Type Theory} {model theory}
    {universal algebra},
    pdfnewwindow=true,      % links in new window
    colorlinks=true,       % false: boxed links; true: colored links
    linkcolor=blue,          % color of internal links
    citecolor=black,        % color of links to bibliography
    filecolor=black,      % color of file links
    urlcolor=britishracinggreen           % color of external links
}

\usepackage[wjd,links]{agda}
\AgdaNoSpaceAroundCode{}

\newif\ifnonbold % comment this line to restore bold universe symbols

%%% HERE IS HOW WE CONTROL WHETHER LONG OR SHORT VERSION OF THE PAPER IS COMPILED
\newif\ifshort
\shorttrue % (to compile short version of the paper)
\shortfalse % (to compile long version of the paper)
% USAGE: 
% \ifshort <stuff included in short version goes here> \else <stuff included in long version goes here> \fi
\newcommand\seeunabridged{see the unabridged version of this paper,~\cite{DeMeo:2021b}, for the complete formalization\xspace}
\newcommand\seemedium{see~\cite{DeMeo:2021b} for the complete formalization\xspace}
\newcommand\seeshort{see~\cite{DeMeo:2021b}\xspace}

% \usepackage[x11names, rgb]{xcolor}
\usepackage{tikz}
\usetikzlibrary{snakes,arrows,shapes}
\usepackage{amsmath}

\crefformat{footnote}{#2\footnotemark[#1]#3}

%%%%%%%%%%%%%% TITLE %%%%%%%%%%%%%%%%%%%%%%%%%%%%%%%%%%%%%%%%%%%%%%%%%%%%%%%%%%%%%%%%
\ifshort
\title{Birkhoff's Variety Theorem formalized in Agda}
\else
\title{A Machine-checked Proof of Birkhoff's Variety Theorem in Martin-L\"of Type Theory}
\fi

%%%%%%%%%%%%%% AUTHOR %%%%%%%%%%%%%%%%%%%%%%%%%%%%%%%%%%%%%%%%%%%%%%%%%%%%%%%%%%%%%%%%
%\author{name}{affil}{email}{orcid}{funding}.
\author{William DeMeo}
       {Thmpr Research}
       {williamdemeo@gmail.com}{https://orcid.org/0000-0003-1832-5690}{supported
         by the CoCoSym project under ERC Consolidator Grant No.~771005.}
\author{Jacques Carette}{McMaster University}{carette@mcmaster.ca}{https://orcid.org/0000-0001-8993-9804}{}
\authorrunning{DeMeo and Carette}

\copyrightfooter

% \pagestyle{fancy}
% \renewcommand{\sectionmark}[1]{\markboth{#1}{}}
% \fancyhf{}
% \fancyhead[ER]{\sffamily\bfseries \leftmark}
% \fancyhead[OL]{\sffamily\bfseries The agda-algebras library}
% \fancyhead[EL,OR]{\sffamily\bfseries \thepage}

\ccsdesc[500]{Theory of computation~Logic and verification}
\ccsdesc[300]{Computing methodologies~Representation of mathematical objects}
\ccsdesc[300]{Theory of computation~Type theory}
% \ccsdesc[300]{Theory of computation~Type structures}
% \ccsdesc[300]{Theory of computation~Constructive mathematics}

\keywords{Agda, constructive mathematics, dependent types, equational logic, formalization of mathematics, Martin-L\"of type theory, model theory, universal algebra}

\category{} %optional, e.g. invited paper

% \relatedversion{hosted on arXiv}
% \relatedversiondetails[linktext={http://arxiv.org/a/demeo\_w\_1}]{Part 2, Part 3}{http://arxiv.org/a/demeo_w_1}

% \supplement{~\\ \textit{Documentation}: \ualibdotorg}%
% \supplementdetails{Software}{https://gitlab.com/ualib/ualib.gitlab.io.git}

\nolinenumbers %uncomment to disable line numbering

\hideLIPIcs  %uncomment to remove references to LIPIcs series (logo, DOI, ...), e.g. when preparing a pre-final version to be uploaded to arXiv or another public repository

%Editor-only macros:: begin (do not touch as author)%%%%%%%%%%%%%%%%%%%%%%%%%%%%%%%%%%
\EventEditors{}
\EventNoEds{2}
\EventLongTitle{}
\EventShortTitle{}
\EventAcronym{}
\EventYear{2021}
\EventDate{26 October}
\EventLocation{}
\EventLogo{}
\SeriesVolume{0}
\ArticleNo{0}
%%%%%%%%%%%%%%%%%%%%%%%%%%%%%%%%%%%%%%%%%%%%%%%%%%%%%%

% \includeonly{4Algebras} %1Introduction,2Overture,3Relations,4Algebras,5Conclusion}

\begin{document}

\maketitle

%\newpage %%%%%%%%%%%%%%%%%%%%%%%%%%%%  TOC  %%%%%%%%%%%%%%%%%%%%%%

% \setcounter{tocdepth}{2}
% \tableofcontents

\begin{abstract}
The Agda Universal Algebra Library (\ualib) is a library of types and programs (theorems and proofs) we developed to formalize the foundations of universal algebra in Martin-Löf-style dependent type theory using the \agda programming language and proof assistant. This paper describes the \ualib and demonstrates that Agda is accessible to working mathematicians (such as ourselves) as a tool for formally verifying nontrivial results in general algebra and related fields. The library includes a substantial collection of definitions, theorems, and proofs from universal algebra and equational logic and as such provides many examples that exhibit the power of inductive and dependent types for representing and reasoning about general algebraic and relational structures.

The first major milestone of the \ualib project is a complete proof of Birkhoff's HSP theorem. To the best of our knowledge, this is the first time Birkhoff's theorem has been formulated and proved in dependent type theory and verified with a proof assistant.

%% In this paper we describe the \agdaualib and the formal proof of Birkhoff's theorem, discussing some of the technically challenging parts of the proof. In so doing, we illustrate the effectiveness of dependent type theory, Agda, and the \ualib for formally verifying nontrivial theorems in universal algebra and equational logic.

\end{abstract}


\providecommand{\hypertarget}[2]{#2}

%% This is to make the typewriter font inserted by pandoc look more similar to
%% the Agda fonts used in the highlighted code.
\renewcommand\texttt[1]{\textsf{#1}}

% To support formalization in type theory of research level mathematics in universal algebra and related fields, we present the Agda Universal Algebra Library (\ualib), a software library containing formal statements and proofs of the core definitions and results of universal algebra.
The \ualib is written in \agda~\cite{Norell:2009}, a programming language and proof assistant based on Martin-L\"of Type Theory that not only supports dependent and inductive types, but also provides powerful \emph{proof tactics} for proving things about the objects that inhabit these types.


The seminal idea for the \agdaualib project was the observation that, on the one hand, a number of fundamental constructions in universal algebra can be defined recursively, and theorems about them proved by structural induction, while, on the other hand, inductive and dependent types make possible very precise formal representations of recursively defined objects, which often admit elegant constructive proofs of properties of such objects.  An important feature of such proofs in type theory is that they are total functional programs and, as such, they are computable and composable.
%% These observations suggested that there would be much to gain from implementing universal algebra in a language, such as Martin-L\"of type theory, that features dependent and inductive types.

\subsection*{Main Objectives}
The first goal of the project is to express the foundations of universal algebra constructively, in type theory, and to formally verify the foundations using the Agda proof assistant. Thus we aim to codify the edifice upon which our mathematical research stands, and demonstrate that advancements in our field can be expressed in type theory and formally verified in a way that we, and other working mathematicians, can easily understand and check the results. We hope the library inspires and encourages others to formalize and verify their own mathematics research so that we may more easily understand and verify their results.

Our field is deep and broad, so codifying all of its foundations may seem like a daunting task and a risky investment of time and resources. However, we believe the subject is well served by a new, modern, \emph{constructive} presentation of its foundations.  Finally, the mere act of reinterpreting the foundations in an alternative system offers a fresh perspective, and this often leads to deeper insights and new discoveries.

Indeed, we wish to emphasize that our ultimate objective is not merely to translate existing results into a new more modern and formal language. Rather, an important goal of the \ualib project is a system that is useful for conducting research in mathematics, and that is how we intend to use our library now that we have achieved our initial objective of implementing a substantial part of the foundations of universal algebra in \agda.

In our own mathematics research, experience has taught us that a proof assistant equipped with specialized libraries for universal algebra, as well as domain-specific tactics to automate proof idioms of our field, would be extremely valuable and powerful tool. Thus, we aim to build a library that serves as an indispensable part of our research tool kit. To this end, our intermediate-term objectives include
\begin{itemize}
\item developing domain specific ``proof tactics'' to express the idioms of universal algebra,
\item implementing automated proof search for universal algebra,
\item formalizing theorems from our own mathematics research,
\item documenting the resulting software libraries so they are accessible to other mathematicians.
\end{itemize}

\subsection*{Contributions and organization}
Apart from the library itself, we describes the formal implementation and proof of a deep result, Garrett Birkhoff's celebrated HSP theorem~\cite{Birkhoff:1935}, which was among the first major results of universal algebra.  The theorem states that a \defn{variety} (a class of algebras closed under quotients, subalgebras, and products) is an equational class (defined by the set of identities satisfied by all its members).
The fact that we now have a formal proof of this is noteworthy, not only because this is the first time the theorem has been proved in dependent type theory and verified with a proof assistant, but also because the proof is constructive. As the paper~\cite{Carlstrom:2008} of Carlstr\"om makes clear, it is a highly nontrivial exercise to take a well-known informal proof of a theorem like Birkhoff's and show that it can be formalized using only constructive logic and natural deduction, without appealing to, say, the Law of the Excluded Middle or the Axiom of Choice.

Each of the sections that follow describes the most important or noteworthy components of the \ualib. We cover just enough to keep the paper somewhat self-contained.  Of course, space does not permit us to cover every definition and theorem required to present a complete formal proof of a theorem like Birkhoff's.  We remedy this in two ways. First, throughout the paper we include pointers to places in the documentation where the omitted material can be found.  Second, we include an appendix containing \agda background, discussion of the foundational assumptions of the \ualib, and definitions of some important types of dependent type theory and how they are represented in \agda and in the \ualib.  We hope this appendix is especially useful to readers who are not already proficient users of \agda.


\subsection*{Prior art}
There have been a number of prior efforts to formalize parts of universal algebra in type theory, most notably
\begin{itemize}
  \item Capretta~\cite{Capretta:1999} (1999) formalized the basics of universal algebra in the Calculus of Inductive Constructions using the Coq proof assistant;
    \item Spitters and van der Weegen~\cite{Spitters:2011} (2011) formalized the basics of universal algebra and some classical algebraic structures, also in the Calculus of Inductive Constructions using the Coq proof assistant, promoting the use of type classes as a preferable alternative to setoids;
 \item Gunther, et al~\cite{Gunther:2018} (2018) developed what seems to be (prior to the \ualib) the most extensive library of formal universal algebra to date; in particular, this work includes a formalization of some basic equational logic; the project (like the \ualib) uses Martin-L\"of Type Theory and the Agda proof assistant.
\end{itemize}
Some other projects aimed at formalizing mathematics generally, and algebra in particular, have developed into very extensive libraries that include definitions, theorems, and proofs about algebraic structures, such as groups, rings, modules, etc.  However, goals of this prior work seem to be the formalization of special classical algebraic structures, as opposed to the general theory of (universal) algebras.

Although the \agdaualib project was initiated relatively recently (in 2018), the part of universal algebra and equational logic that it formalizes extends beyond the scope of prior efforts.  In particular, the \ualib now includes the only formal, constructive, machine-checked proof of Birkhoff's variety theorem that we know of. We remark that, with the exception of~\cite{Carlstrom:2008}, all other proofs of Birkhoff's theorem we have seen are informal and not known to be constructive.


\subsection*{Logical foundations}
The \agdaualib is based on a minimal version of Martin-Löf dependent type theory (MLTT) that is the same or very close to the type theory on which \MartinEscardo's \TypeTopology Agda library is based. This is also the type theory
that \escardo taught us in the short course \ufcourse at the Midlands Graduate School in the Foundations of Computing Science at University of Birmingham in 2019.

We won't go into great detail here because there are already other very nice resources available, such as the section \href{https://www.cs.bham.ac.uk/~mhe/HoTT-UF-in-Agda-Lecture-Notes/HoTT-UF-Agda.html\#mlttinagda}{A spartan Martin-Löf type theory} of the lecture notes by \escardo just mentioned, as well as the \href{https://ncatlab.org/nlab/show/Martin-L\%C3\%B6f+dependent+type+theory}{ncatlab entry on Martin-Löf dependent type theory}.

We will have much more to say about types and type theory as we progress. For now, suffice it to recall the handfull of objects that are assumed at the jumping-off point for MLTT: ``primitive'' \textbf{types} (𝟘, 𝟙, and ℕ, denoting the empty type, one-element type, and natural numbers), \textbf{type formers} (+, Π, Σ, Id, denoting binary sum, product, sum, and the identity type), and an infinite collection of \textbf{universes} (types of types) and universe variables to denote them (for which we will use upper-case caligraphic letters like 𝓤, 𝓥, 𝓦, etc., typically from the latter half of the English alphabet).

\subsection*{Intended audience}
This document describes \agdaualib in enough detail so that working mathematicians (and possibly some normal people, too) might be able to learn enough about Agda and its libraries to put them to use when creating, formalizing, and verifying new mathematics.

While there are no strict prerequisites, we expect anyone with an interest in this work will have been motivated by prior exposure to universal algebra, as presented in, say,~\cite{Bergman:2012}, or~\cite{McKenzie:1987}, or category theory, as presented in, say,~\cite{Riehl:2017} or \url{https://categorytheory.gitlab.io}.

Some prior exposure to type theory and \agda would be helpful, but even without this background one might still be able to get something useful out of this by referring to one or more of the resources mentioned in the references section below to fill in gaps as needed.

It is assumed that readers of this documentation are actively experimenting with \agda using Emacs with the
agda2-mode extension installed. If you don't have \agda and agda2-mode, follow the directions on the \href{https://wiki.portal.chalmers.se/agda/pmwiki.php}{main Agda website: https://wiki.portal.chalmers.se/agda/pmwiki.php} or
consult \href{https://github.com/martinescardo/HoTT-UF-Agda-Lecture-Notes/blob/master/INSTALL.md}{\MartinEscardo's installation instructions} or \href{https://gitlab.com/ualib/ualib.gitlab.io/-/blob/master/INSTALL_AGDA.md}{our
modified version of Martin's instructions}.

The main repository for the \agdaualib is gitlab.com/ualib/ualib.gitlab.io. There are more installation instructions in the \href{https://gitlab.com/ualib/ualib.gitlab.io/README.md}{README.md} file of the \href{https://gitlab.com/ualib/ualib.gitlab.io}{UALib repository}, but a summary of what's required is
\begin{itemize}
\item \href{https://www.gnu.org/software/emacs/}{Emacs}
\item \agda 2.6.1
\item \agdamode (for Emacs)
\item A copy (or ``clone'') of the \href{https://gitlab.com/ualib/ualib.gitlab.io}{\texttt{gitlab.com/ualib/ualib.gitlab.io}} repository.
\end{itemize}

Instructions for installing each of these are available in the \href{https://gitlab.com/ualib/ualib.gitlab.io/README.md}{README.md} file of the \href{https://gitlab.com/ualib/ualib.gitlab.io}{UALib repository}.

\subsection*{Attributions and citations}

Most of the mathematical results that formalized in the \ualib are well known. Regarding the Agda source code in the\agdaualib, this is mainly due to William DeMeo with one major caveat: we benefited greatly from, and the library depends upon, the lecture notes on \ufcourse and the \typetopology Agda Library by \escardo~\cite{MHE}. The author is indebted to Martin for making his library and notes available and for teaching a course on type theory in Agda at the Midlands Graduate School in the Foundations of Computing Science in Birmingham in 2019.

The following Agda documentation and tutorials helped inform and improve the \ualib, especially the first one in the list.

\begin{itemize}
\item \escardo, \ufcourse
\item Wadler, \href{https://plfa.github.io/}{Programming Languages Foundations in Agda}
\item Altenkirk, \href{http://www.cs.nott.ac.uk/~psztxa/g53cfr/}{Computer Aided Formal Reasoning}
\item Bove and Dybjer, \href{http://www.cse.chalmers.se/~peterd/papers/DependentTypesAtWork.pdf}{Dependent Types at Work}
\item Gunther, Gadea, Pagano, \href{http://www.sciencedirect.com/science/article/pii/S1571066118300768}{Formalization of Universal Algebra in Agda}
\item Norell and Chapman, \href{http://www.cse.chalmers.se/~ulfn/papers/afp08/tutorial.pdf}{Dependently Typed Programming in Agda}
\end{itemize}


\subsection*{References and resources}

The official sources of information about Agda are
\begin{itemize}
\item \href{https://wiki.portal.chalmers.se/agda/pmwiki.php}{Agda Wiki}
\item \href{https://people.inf.elte.hu/pgj/agda/tutorial/Index.html}{Agda Tutorial}
\item \href{https://agda.readthedocs.io/en/v2.6.1/}{Agda User's Manual}
\item \href{https://agda.readthedocs.io/en/v2.6.1/language}{Agda Language Reference}
\item \href{https://agda.github.io/agda-stdlib/}{Agda Standard Library}
\item \href{https://agda.readthedocs.io/en/v2.6.1/tools/}{Agda Tools}
\end{itemize}


\noindent The official sources of information about the \agdaualib are
\begin{itemize}
  \item \href{https://ualib.gitlab.io}{\texttt{ualib.org}} (the web site) includes every line of code in the library, rendered as html and accompanied by documentation, and
  \item \href{https://gitlab.com/ualib/ualib.gitlab.io}{\texttt{gitlab.com/ualib/ualib.gitlab.io}} (the source code) freely available and licensed under the \href{https://creativecommons.org/licenses/by-sa/4.0/}{Creative Commons Attribution-ShareAlike 4.0 International License}.
\end{itemize}


\subsection*{License and citation information}
This document and the \href{https://gitlab.com/ualib/ualib.gitlab.io/}{Agda Universal Algebra Library} by \href{mailto:williamdemeo@gmail.com}{William DeMeo} are licensed under a \href{http://creativecommons.org/licenses/by-sa/4.0/}{Creative Commons Attribution-ShareAlike BY-SA 4.0 International License} and is based on work at \url{https://gitlab.com/ualib/ualib.gitlab.io}.

If you use the \agdaualib and/or you wish to refer to it or its documentation (e.g., in a publication), then please use the following BibTeX data:

{\small
\begin{verbatim}
@article{DeMeo:2021,
 author        = {William DeMeo},
 title         = {The {A}gda {U}niversal {A}lgebra {L}ibrary and 
                  {B}irkhoff's {T}heorem in {M}artin-{L}\"of 
                  {D}ependent {T}ype {T}heory}, 
 journal       = {CoRR},
 volume        = {abs/2101.10166},
 year          = {2021},
 eprint        = {2101.10166},
 archivePrefix = {arXiv},
 primaryClass  = {cs.LO},
 url           = {https://arxiv.org/abs/2101.10166},
 note          = {source code: \url{https://gitlab.com/ualib/ualib.gitlab.io}}
\end{verbatim}
}

\subsection*{Acknowledgments}
The author wishes to thank Siva Somayyajula for the valuable contributions he made during the projects crucial early stages. Thanks also to Andrej Bauer, Clifford Bergman, Venanzio Capretta, Martin Escardo, Ralph Freese, Bill Lampe, Miklós Maróti, Peter Mayr, JB Nation, and Hyeyoung Shin for helpful discussions, instruction, advice, inspiration
and encouragement.

\subsection*{Contributions welcomed!}
Readers and users are encouraged to suggest improvements to the \agdaualib and/or its documentation by submitting a new issue or merge request to the main repository at \href{https://gitlab.com/ualib/ualib.gitlab.io/}{gitlab.com/ualib/ualib.gitlab.io/}.

\bigskip

\bigskip

\noindent \textit{2 February 2021} \hfill \textit{William DeMeo}

\hfill \textit{Prague}



% \input{../latex/Overture}
% \input{../latex/Overture.Preliminaries}
% \input{../latex/Overture.Inverses}
% \input{../latex/Overture.Transformers}

% \let\paragraph\subsubsection
% \let\subsubsection\subsection
% \let\subsection\section
% \let\section\chapter
% \let\chapter\part

\hypertarget{introduction}{%
\subsection{Introduction}\label{introduction}}

The Agda Universal Algebra Library
(\href{https://github.com/ualib/agda-algebras}{agda-algebras}) is a
collection of types and programs (theorems and proofs) formalizing the
foundations of universal algebra in dependent type theory using the
\href{https://wiki.portal.chalmers.se/agda/pmwiki.php}{Agda} programming
language and proof assistant. The agda-algebras library now includes a
substantial collection of definitions, theorems, and proofs from
universal algebra and equational logic and as such provides many
examples that exhibit the power of inductive and dependent types for
representing and reasoning about general algebraic and relational
structures.

The first major milestone of the
\href{https://github.com/ualib/agda-algebras}{agda-algebras} project is
a new formal proof of \emph{Birkhoff's variety theorem} (also known as
the \emph{HSP theorem}), the first version of which was completed in
\href{https://github.com/ualib/ualib.github.io/blob/b968e8af1117fc77700d3a588746cbefbd464835/sandbox/birkhoff-exp-new-new.lagda}{January
of 2021}. To the best of our knowledge, this was the first time
Birkhoff's theorem had been formulated and proved in dependent type
theory and verified with a proof assistant.

In this paper, we present a single Agda module called
{[}Demos.HSP{]}{[}{]}. This module extracts only those parts of the
library needed to prove Birkhoff's variety theorem. In order to meet
page limit guidelines, and to reduce strain on the reader, we omit
proofs of some routine or technical lemmas that do not provide much
insight into the overall development. However, a long version of this
paper, which includes all code in the {[}Demos.HSP{]}{[}{]} module, is
available on the arXiv. {[}reference needed{]}

In the course of our exposition of the proof of the HSP theorem, we
discuss some of the more challenging aspects of formalizing
\emph{universal algebra} in type theory and the issues that arise when
attempting to constructively prove some of the basic results in this
area. We demonstrate that dependent type theory and Agda, despite the
demands they place on the user, are accessible to working mathematicians
who have sufficient patience and a strong enough desire to
constructively codify their work and formally verify the correctness of
their results. Perhpas our presentation will be viewed as a sobering
glimpse of the painstaking process of doing mathematics in the languages
of dependent type theory using the Agda proof assistant. Nonetheless we
hope to make a compelling case for investing in these technologies.
Indeed, we are excited to share the gratifying rewards that come with
some mastery of type theory and interactive theorem proving.

\hypertarget{prior-art}{%
\subsubsection{Prior art}\label{prior-art}}

There have been a number of efforts to formalize parts of universal
algebra in type theory prior to ours, most notably

\begin{enumerate}
\def\labelenumi{\arabic{enumi}.}
\tightlist
\item
  Capretta {[}@Capretta:1999{]} (1999) formalized the basics of
  universal algebra in the Calculus of Inductive Constructions using the
  Coq proof assistant;
\item
  Spitters and van der Weegen {[}@Spitters:2011{]} (2011) formalized the
  basics of universal algebra and some classical algebraic structures,
  also in the Calculus of Inductive Constructions using the Coq proof
  assistant, promoting the use of type classes;
\item
  Gunther, et al {[}@Gunther:2018{]} (2018) developed what seems to be
  (prior to the
  \href{https://github.com/ualib/agda-algebras}{agda-algebras} library)
  the most extensive library of formal universal algebra to date; this
  work is based on dependent type theory and programmed in Agda; it
  treats multisorted algebras and goes beyond the basic Noether
  isomorphism theorems to include some basic equational logic.
\item
  Lynge and Spitters {[}@Lynge:2019{]} (2019) formalize basic,
  mutisorted universal algebra, up to the Noether isomorphism theorems,
  in homotopy type theory; in this setting, the authors can avoid using
  setoids by postulating a strong extensionality axiom called
  univalence.
\end{enumerate}

Some other projects aimed at formalizing mathematics generally, and
algebra in particular, have developed into very extensive libraries that
include definitions, theorems, and proofs about algebraic structures,
such as groups, rings, modules, etc. However, the goals of these efforts
seem to be the formalization of special classical algebraic structures,
as opposed to the general theory of (universal) algebras. Moreover, the
part of universal algebra and equational logic formalized in the
\href{https://github.com/ualib/agda-algebras}{agda-algebras} library
extends beyond the scope of prior efforts.

\hypertarget{preliminaries}{%
\subsection{Preliminaries}\label{preliminaries}}

\hypertarget{logical-foundations}{%
\subsubsection{Logical foundations}\label{logical-foundations}}

An Agda program typically begins by setting some language options and by
importing types from existing Agda libraries. The language options are
specified using the \texttt{OPTIONS} \emph{pragma} which affect control
the way Agda behaves by controlling the deduction rules that are
available to us and the logical axioms that are assumed when the program
is type-checked by Agda to verify its correctness. Every Agda program in
the \href{https://github.com/ualib/agda-algebras}{agda-algebras}
library, including the present module ({[}Demos.HSP{]}{[}{]}), begins
with the following line.

\begin{code}%
\>[0]\<%
\\
\>[0]\AgdaSymbol{\{-\#}\AgdaSpace{}%
\AgdaKeyword{OPTIONS}\AgdaSpace{}%
\AgdaPragma{--without-K}\AgdaSpace{}%
\AgdaPragma{--exact-split}\AgdaSpace{}%
\AgdaPragma{--safe}\AgdaSpace{}%
\AgdaSymbol{\#-\}}\<%
\\
\>[0]\<%
\end{code}

Here we provide a brief overview (along with links to more details) of
how these options affect our foundational assumptions.

\begin{itemize}
\item
  \texttt{without-K} disables
  \href{https://ncatlab.org/nlab/show/axiom+K+\%28type+theory\%29}{Streicher's
  K axiom} ; see also the
  \href{https://agda.readthedocs.io/en/v2.6.1/language/without-k.html}{section
  on axiom K} in the
  \href{https://agda.readthedocs.io/en/v2.6.1.3/language}{Agda Language
  Reference Manual}.
\item
  \texttt{exact-split} makes Agda accept only those definitions that
  behave like so-called \emph{judgmental} equalities.
  \href{https://www.cs.bham.ac.uk/~mhe}{Martín Escardó} explains this by
  saying it ``makes sure that pattern matching corresponds to Martin-Löf
  eliminators;'' see also the
  \href{https://agda.readthedocs.io/en/v2.6.1/tools/command-line-options.html\#pattern-matching-and-equality}{Pattern
  matching and equality section} of the
  \href{https://agda.readthedocs.io/en/v2.6.1.3/tools/}{Agda Tools}
  documentation.
\item
  \texttt{safe} ensures that nothing is postulated outright---every
  non-MLTT axiom has to be an explicit assumption (e.g., an argument to
  a function or module); see also
  \href{https://agda.readthedocs.io/en/v2.6.1/tools/command-line-options.html\#cmdoption-safe}{this
  section} of the
  \href{https://agda.readthedocs.io/en/v2.6.1.3/tools/}{Agda Tools}
  documentation and the
  \href{https://agda.readthedocs.io/en/v2.6.1/language/safe-agda.html\#safe-agda}{Safe
  Agda section} of the
  \href{https://agda.readthedocs.io/en/v2.6.1.3/language}{Agda Language
  Reference}.
\end{itemize}

The \texttt{OPTIONS} pragma is usually followed by the start of a module
and a list of \texttt{import} directives. We won't reproduce all of the
imports we use here. Rather we show only those imports that rename
objects from the standard library to our own notation which might be
less standard.

\begin{code}[hide]%
\>[0]\AgdaSymbol{\{-\#}\AgdaSpace{}%
\AgdaKeyword{OPTIONS}\AgdaSpace{}%
\AgdaPragma{--without-K}\AgdaSpace{}%
\AgdaPragma{--exact-split}\AgdaSpace{}%
\AgdaPragma{--safe}\AgdaSpace{}%
\AgdaSymbol{\#-\}}\<%
\\
\>[0]\AgdaKeyword{open}\AgdaSpace{}%
\AgdaKeyword{import}\AgdaSpace{}%
\AgdaModule{Algebras.Basic}\AgdaSpace{}%
\AgdaKeyword{using}\AgdaSpace{}%
\AgdaSymbol{(}\AgdaSpace{}%
\AgdaGeneralizable{𝓞}\AgdaSpace{}%
\AgdaSymbol{;}\AgdaSpace{}%
\AgdaGeneralizable{𝓥}\AgdaSpace{}%
\AgdaSymbol{;}\AgdaSpace{}%
\AgdaFunction{Signature}\AgdaSpace{}%
\AgdaSymbol{)}\<%
\\
\>[0]\AgdaKeyword{module}\AgdaSpace{}%
\AgdaModule{Demos.HSP}\AgdaSpace{}%
\AgdaSymbol{\{}\AgdaBound{𝑆}\AgdaSpace{}%
\AgdaSymbol{:}\AgdaSpace{}%
\AgdaFunction{Signature}\AgdaSpace{}%
\AgdaGeneralizable{𝓞}\AgdaSpace{}%
\AgdaGeneralizable{𝓥}\AgdaSymbol{\}}\AgdaSpace{}%
\AgdaKeyword{where}\<%
\end{code}
\begin{code}%
\>[0]\<%
\\
\>[0]\AgdaKeyword{open}\AgdaSpace{}%
\AgdaKeyword{import}\AgdaSpace{}%
\AgdaModule{Agda.Primitive}%
\>[28]\AgdaKeyword{using}\AgdaSpace{}%
\AgdaSymbol{(}\AgdaOperator{\AgdaPrimitive{\AgdaUnderscore{}⊔\AgdaUnderscore{}}}\AgdaSpace{}%
\AgdaSymbol{;}\AgdaSpace{}%
\AgdaPrimitive{lsuc}\AgdaSymbol{)}\AgdaSpace{}%
\AgdaKeyword{renaming}\AgdaSpace{}%
\AgdaSymbol{(}\AgdaPrimitive{Set}\AgdaSpace{}%
\AgdaSymbol{to}\AgdaSpace{}%
\AgdaPrimitive{Type}\AgdaSymbol{)}\<%
\\
\>[0]\AgdaKeyword{open}\AgdaSpace{}%
\AgdaKeyword{import}\AgdaSpace{}%
\AgdaModule{Data.Product}%
\>[28]\AgdaKeyword{using}\AgdaSpace{}%
\AgdaSymbol{(}\AgdaFunction{Σ-syntax}\AgdaSpace{}%
\AgdaSymbol{;}\AgdaSpace{}%
\AgdaOperator{\AgdaFunction{\AgdaUnderscore{}×\AgdaUnderscore{}}}\AgdaSpace{}%
\AgdaSymbol{;}\AgdaSpace{}%
\AgdaOperator{\AgdaInductiveConstructor{\AgdaUnderscore{},\AgdaUnderscore{}}}\AgdaSpace{}%
\AgdaSymbol{;}\AgdaSpace{}%
\AgdaRecord{Σ}\AgdaSymbol{)}\AgdaSpace{}%
\AgdaKeyword{renaming}\AgdaSpace{}%
\AgdaSymbol{(}\AgdaField{proj₁}\AgdaSpace{}%
\AgdaSymbol{to}\AgdaSpace{}%
\AgdaField{fst}\AgdaSpace{}%
\AgdaSymbol{;}\AgdaSpace{}%
\AgdaField{proj₂}\AgdaSpace{}%
\AgdaSymbol{to}\AgdaSpace{}%
\AgdaField{snd}\AgdaSymbol{)}\<%
\\
\>[0]\AgdaKeyword{open}\AgdaSpace{}%
\AgdaKeyword{import}\AgdaSpace{}%
\AgdaModule{Function}%
\>[28]\AgdaKeyword{using}\AgdaSpace{}%
\AgdaSymbol{(}\AgdaFunction{id}\AgdaSpace{}%
\AgdaSymbol{;}\AgdaSpace{}%
\AgdaOperator{\AgdaFunction{\AgdaUnderscore{}∘\AgdaUnderscore{}}}\AgdaSpace{}%
\AgdaSymbol{;}\AgdaSpace{}%
\AgdaFunction{flip}\AgdaSpace{}%
\AgdaSymbol{;}\AgdaSpace{}%
\AgdaRecord{Injection}\AgdaSpace{}%
\AgdaSymbol{;}\AgdaSpace{}%
\AgdaRecord{Surjection}\AgdaSymbol{)}\AgdaSpace{}%
\AgdaKeyword{renaming}\AgdaSpace{}%
\AgdaSymbol{(}\AgdaRecord{Func}\AgdaSpace{}%
\AgdaSymbol{to}\AgdaSpace{}%
\AgdaRecord{\AgdaUnderscore{}⟶\AgdaUnderscore{}}\AgdaSymbol{)}\<%
\\
\>[0]\AgdaKeyword{open}\AgdaSpace{}%
\AgdaModule{\AgdaUnderscore{}⟶\AgdaUnderscore{}}%
\>[28]\AgdaKeyword{using}\AgdaSpace{}%
\AgdaSymbol{(}\AgdaField{cong}\AgdaSymbol{)}\AgdaSpace{}%
\AgdaKeyword{renaming}\AgdaSpace{}%
\AgdaSymbol{(}\AgdaField{f}\AgdaSpace{}%
\AgdaSymbol{to}\AgdaSpace{}%
\AgdaField{\AgdaUnderscore{}⟨\$⟩\AgdaUnderscore{}}\AgdaSymbol{)}\<%
\\
%
\\[\AgdaEmptyExtraSkip]%
\>[0]\AgdaKeyword{open}\AgdaSpace{}%
\AgdaKeyword{import}%
\>[13]\AgdaModule{Relation.Binary.PropositionalEquality}%
\>[52]\AgdaSymbol{as}\AgdaSpace{}%
\AgdaModule{≡}\AgdaSpace{}%
\AgdaKeyword{using}\AgdaSpace{}%
\AgdaSymbol{(}\AgdaOperator{\AgdaDatatype{\AgdaUnderscore{}≡\AgdaUnderscore{}}}\AgdaSymbol{)}\<%
\\
\>[0]\AgdaKeyword{import}%
\>[13]\AgdaModule{Relation.Binary.Reasoning.Setoid}%
\>[52]\AgdaSymbol{as}\AgdaSpace{}%
\AgdaModule{SetoidReasoning}\<%
\\
\>[0]\AgdaKeyword{import}%
\>[13]\AgdaModule{Function.Definitions}%
\>[52]\AgdaSymbol{as}\AgdaSpace{}%
\AgdaModule{FD}\<%
\\
\>[0]\<%
\end{code}
\begin{code}[hide]%
\>[0]\AgdaKeyword{open}\AgdaSpace{}%
\AgdaKeyword{import}\AgdaSpace{}%
\AgdaModule{Level}\AgdaSpace{}%
\AgdaKeyword{using}\AgdaSpace{}%
\AgdaSymbol{(}\AgdaSpace{}%
\AgdaPostulate{Level}\AgdaSpace{}%
\AgdaSymbol{)}\<%
\\
\>[0]\AgdaKeyword{open}\AgdaSpace{}%
\AgdaKeyword{import}\AgdaSpace{}%
\AgdaModule{Relation.Binary}\AgdaSpace{}%
\AgdaKeyword{using}\AgdaSpace{}%
\AgdaSymbol{(}\AgdaSpace{}%
\AgdaRecord{Setoid}\AgdaSpace{}%
\AgdaSymbol{;}\AgdaSpace{}%
\AgdaFunction{Rel}\AgdaSpace{}%
\AgdaSymbol{;}\AgdaSpace{}%
\AgdaRecord{IsEquivalence}\AgdaSpace{}%
\AgdaSymbol{)}\<%
\\
\>[0]\AgdaKeyword{open}\AgdaSpace{}%
\AgdaKeyword{import}\AgdaSpace{}%
\AgdaModule{Relation.Binary.Definitions}\AgdaSpace{}%
\AgdaKeyword{using}\AgdaSpace{}%
\AgdaSymbol{(}\AgdaSpace{}%
\AgdaFunction{Reflexive}\AgdaSpace{}%
\AgdaSymbol{;}\AgdaSpace{}%
\AgdaFunction{Sym}\AgdaSpace{}%
\AgdaSymbol{;}\AgdaSpace{}%
\AgdaFunction{Trans}\AgdaSpace{}%
\AgdaSymbol{;}\AgdaSpace{}%
\AgdaFunction{Symmetric}\AgdaSpace{}%
\AgdaSymbol{;}\AgdaSpace{}%
\AgdaFunction{Transitive}\AgdaSpace{}%
\AgdaSymbol{)}\<%
\\
\>[0]\AgdaKeyword{open}\AgdaSpace{}%
\AgdaKeyword{import}\AgdaSpace{}%
\AgdaModule{Relation.Unary}\AgdaSpace{}%
\AgdaKeyword{using}\AgdaSpace{}%
\AgdaSymbol{(}\AgdaSpace{}%
\AgdaFunction{Pred}\AgdaSpace{}%
\AgdaSymbol{;}\AgdaSpace{}%
\AgdaOperator{\AgdaFunction{\AgdaUnderscore{}⊆\AgdaUnderscore{}}}\AgdaSpace{}%
\AgdaSymbol{;}\AgdaSpace{}%
\AgdaOperator{\AgdaFunction{\AgdaUnderscore{}∈\AgdaUnderscore{}}}\AgdaSpace{}%
\AgdaSymbol{)}\<%
\\
\>[0]\AgdaKeyword{open}\AgdaSpace{}%
\AgdaModule{Setoid}\AgdaSpace{}%
\AgdaKeyword{using}\AgdaSpace{}%
\AgdaSymbol{(}\AgdaSpace{}%
\AgdaField{Carrier}\AgdaSpace{}%
\AgdaSymbol{;}\AgdaSpace{}%
\AgdaField{isEquivalence}\AgdaSpace{}%
\AgdaSymbol{)}\<%
\\
\>[0]\AgdaKeyword{private}\AgdaSpace{}%
\AgdaKeyword{variable}\<%
\\
\>[0][@{}l@{\AgdaIndent{0}}]%
\>[1]\AgdaGeneralizable{α}\AgdaSpace{}%
\AgdaGeneralizable{ρᵃ}\AgdaSpace{}%
\AgdaGeneralizable{β}\AgdaSpace{}%
\AgdaGeneralizable{ρᵇ}\AgdaSpace{}%
\AgdaGeneralizable{γ}\AgdaSpace{}%
\AgdaGeneralizable{ρᶜ}\AgdaSpace{}%
\AgdaGeneralizable{δ}\AgdaSpace{}%
\AgdaGeneralizable{ρᵈ}\AgdaSpace{}%
\AgdaGeneralizable{ρ}\AgdaSpace{}%
\AgdaGeneralizable{χ}\AgdaSpace{}%
\AgdaGeneralizable{ℓ}\AgdaSpace{}%
\AgdaSymbol{:}\AgdaSpace{}%
\AgdaPostulate{Level}\<%
\\
%
\>[1]\AgdaGeneralizable{Γ}\AgdaSpace{}%
\AgdaGeneralizable{Δ}\AgdaSpace{}%
\AgdaSymbol{:}\AgdaSpace{}%
\AgdaPrimitive{Type}\AgdaSpace{}%
\AgdaGeneralizable{χ}\<%
\\
%
\>[1]\AgdaGeneralizable{𝑓}\AgdaSpace{}%
\AgdaSymbol{:}\AgdaSpace{}%
\AgdaField{fst}\AgdaSpace{}%
\AgdaBound{𝑆}\<%
\end{code}

Note, in particular, we rename the \texttt{Set} and \texttt{Func} types
of the standard library to \texttt{Type} and the infix long arrow symbol
\texttt{\_⟶\_}, respectively, and we use \texttt{fst} and \texttt{snd}
in place of \texttt{proj₁} and \texttt{proj₂} for the first and second
projections out of the product type \texttt{\_×\_}. In addition, when it
improves readability of the code, we use the alternative notation
\texttt{∣\_∣} and \texttt{∥\_∥} (defined elsewhere) for the first and
second projections.

\hypertarget{setoids}{%
\subsubsection{Setoids}\label{setoids}}

A \emph{setoid} is a type packaged with an equivalence relation on the
collection of inhabitants of that type. Setoids are useful for
representing classical (set-theory-based) mathematics in a constructive,
type-theoretic way because most mathematical structures are assumed to
come equipped with some (often implicit) equivalence relation
manifesting a notion of equality of elements, and therefore a
type-theoretic representation of such a structure should also model its
equality relation.

The \href{https://github.com/ualib/agda-algebras}{agda-algebras} library
was first developed without the use of setoids, opting instead for
specially constructed experimental quotient types. However, this
approach resulted in code that was hard to comprehend and it became
difficult to determine whether the resulting proofs were fully
constructive. In particular, our initial proof of the Birkhoff variety
theorem required postulating function extensionality, an axiom that is
not provable in pure Martin-Löf type theory.{[}reference needed{]}

In contrast, our current approach using setoids makes the equality
relation of a given type explicit and this transparency can make it
easier to determine the correctness and constructivity of the proofs.
Using setiods we need no additional axioms beyond Martin-Löf type
theory; in particular, no function extensionality axioms are postulated
in our current formalization of Birkhoff's variety theorem.

\begin{code}[hide]%
\>[0]\AgdaFunction{𝑖𝑑}\AgdaSpace{}%
\AgdaSymbol{:}\AgdaSpace{}%
\AgdaSymbol{\{}\AgdaBound{A}\AgdaSpace{}%
\AgdaSymbol{:}\AgdaSpace{}%
\AgdaRecord{Setoid}\AgdaSpace{}%
\AgdaGeneralizable{α}\AgdaSpace{}%
\AgdaGeneralizable{ρᵃ}\AgdaSymbol{\}}\AgdaSpace{}%
\AgdaSymbol{→}\AgdaSpace{}%
\AgdaBound{A}\AgdaSpace{}%
\AgdaOperator{\AgdaRecord{⟶}}\AgdaSpace{}%
\AgdaBound{A}\<%
\\
\>[0]\AgdaFunction{𝑖𝑑}\AgdaSpace{}%
\AgdaSymbol{\{}\AgdaBound{A}\AgdaSymbol{\}}\AgdaSpace{}%
\AgdaSymbol{=}\AgdaSpace{}%
\AgdaKeyword{record}\AgdaSpace{}%
\AgdaSymbol{\{}\AgdaSpace{}%
\AgdaField{f}\AgdaSpace{}%
\AgdaSymbol{=}\AgdaSpace{}%
\AgdaFunction{id}\AgdaSpace{}%
\AgdaSymbol{;}\AgdaSpace{}%
\AgdaField{cong}\AgdaSpace{}%
\AgdaSymbol{=}\AgdaSpace{}%
\AgdaFunction{id}\AgdaSpace{}%
\AgdaSymbol{\}}\<%
\\
%
\\[\AgdaEmptyExtraSkip]%
\>[0]\AgdaOperator{\AgdaFunction{\AgdaUnderscore{}⟨∘⟩\AgdaUnderscore{}}}\AgdaSpace{}%
\AgdaSymbol{:}%
\>[9]\AgdaSymbol{\{}\AgdaBound{A}\AgdaSpace{}%
\AgdaSymbol{:}\AgdaSpace{}%
\AgdaRecord{Setoid}\AgdaSpace{}%
\AgdaGeneralizable{α}\AgdaSpace{}%
\AgdaGeneralizable{ρᵃ}\AgdaSymbol{\}}\AgdaSpace{}%
\AgdaSymbol{\{}\AgdaBound{B}\AgdaSpace{}%
\AgdaSymbol{:}\AgdaSpace{}%
\AgdaRecord{Setoid}\AgdaSpace{}%
\AgdaGeneralizable{β}\AgdaSpace{}%
\AgdaGeneralizable{ρᵇ}\AgdaSymbol{\}}\AgdaSpace{}%
\AgdaSymbol{\{}\AgdaBound{C}\AgdaSpace{}%
\AgdaSymbol{:}\AgdaSpace{}%
\AgdaRecord{Setoid}\AgdaSpace{}%
\AgdaGeneralizable{γ}\AgdaSpace{}%
\AgdaGeneralizable{ρᶜ}\AgdaSymbol{\}}\<%
\\
\>[0][@{}l@{\AgdaIndent{0}}]%
\>[1]\AgdaSymbol{→}%
\>[9]\AgdaBound{B}\AgdaSpace{}%
\AgdaOperator{\AgdaRecord{⟶}}\AgdaSpace{}%
\AgdaBound{C}%
\>[16]\AgdaSymbol{→}%
\>[19]\AgdaBound{A}\AgdaSpace{}%
\AgdaOperator{\AgdaRecord{⟶}}\AgdaSpace{}%
\AgdaBound{B}%
\>[26]\AgdaSymbol{→}%
\>[29]\AgdaBound{A}\AgdaSpace{}%
\AgdaOperator{\AgdaRecord{⟶}}\AgdaSpace{}%
\AgdaBound{C}\<%
\\
%
\\[\AgdaEmptyExtraSkip]%
\>[0]\AgdaBound{f}\AgdaSpace{}%
\AgdaOperator{\AgdaFunction{⟨∘⟩}}\AgdaSpace{}%
\AgdaBound{g}\AgdaSpace{}%
\AgdaSymbol{=}\AgdaSpace{}%
\AgdaKeyword{record}%
\>[18]\AgdaSymbol{\{}\AgdaSpace{}%
\AgdaField{f}\AgdaSpace{}%
\AgdaSymbol{=}\AgdaSpace{}%
\AgdaSymbol{(}\AgdaOperator{\AgdaField{\AgdaUnderscore{}⟨\$⟩\AgdaUnderscore{}}}\AgdaSpace{}%
\AgdaBound{f}\AgdaSymbol{)}\AgdaSpace{}%
\AgdaOperator{\AgdaFunction{∘}}\AgdaSpace{}%
\AgdaSymbol{(}\AgdaOperator{\AgdaField{\AgdaUnderscore{}⟨\$⟩\AgdaUnderscore{}}}\AgdaSpace{}%
\AgdaBound{g}\AgdaSymbol{)}\<%
\\
%
\>[18]\AgdaSymbol{;}\AgdaSpace{}%
\AgdaField{cong}\AgdaSpace{}%
\AgdaSymbol{=}\AgdaSpace{}%
\AgdaSymbol{(}\AgdaField{cong}\AgdaSpace{}%
\AgdaBound{f}\AgdaSymbol{)}\AgdaSpace{}%
\AgdaOperator{\AgdaFunction{∘}}\AgdaSpace{}%
\AgdaSymbol{(}\AgdaField{cong}\AgdaSpace{}%
\AgdaBound{g}\AgdaSymbol{)}\AgdaSpace{}%
\AgdaSymbol{\}}\<%
\\
%
\\[\AgdaEmptyExtraSkip]%
\>[0]\AgdaKeyword{module}\AgdaSpace{}%
\AgdaModule{\AgdaUnderscore{}}\AgdaSpace{}%
\AgdaSymbol{\{}\AgdaBound{A}\AgdaSpace{}%
\AgdaSymbol{:}\AgdaSpace{}%
\AgdaPrimitive{Type}\AgdaSpace{}%
\AgdaGeneralizable{α}\AgdaSpace{}%
\AgdaSymbol{\}\{}\AgdaBound{B}\AgdaSpace{}%
\AgdaSymbol{:}\AgdaSpace{}%
\AgdaBound{A}\AgdaSpace{}%
\AgdaSymbol{→}\AgdaSpace{}%
\AgdaPrimitive{Type}\AgdaSpace{}%
\AgdaGeneralizable{β}\AgdaSymbol{\}}\AgdaSpace{}%
\AgdaKeyword{where}\<%
\\
%
\\[\AgdaEmptyExtraSkip]%
\>[0][@{}l@{\AgdaIndent{0}}]%
\>[1]\AgdaOperator{\AgdaFunction{∣\AgdaUnderscore{}∣}}\AgdaSpace{}%
\AgdaSymbol{:}\AgdaSpace{}%
\AgdaFunction{Σ[}\AgdaSpace{}%
\AgdaBound{x}\AgdaSpace{}%
\AgdaFunction{∈}\AgdaSpace{}%
\AgdaBound{A}\AgdaSpace{}%
\AgdaFunction{]}\AgdaSpace{}%
\AgdaBound{B}\AgdaSpace{}%
\AgdaBound{x}\AgdaSpace{}%
\AgdaSymbol{→}\AgdaSpace{}%
\AgdaBound{A}\<%
\\
%
\>[1]\AgdaOperator{\AgdaFunction{∣\AgdaUnderscore{}∣}}\AgdaSpace{}%
\AgdaSymbol{=}\AgdaSpace{}%
\AgdaField{fst}\<%
\\
%
\\[\AgdaEmptyExtraSkip]%
%
\>[1]\AgdaOperator{\AgdaFunction{∥\AgdaUnderscore{}∥}}\AgdaSpace{}%
\AgdaSymbol{:}\AgdaSpace{}%
\AgdaSymbol{(}\AgdaBound{z}\AgdaSpace{}%
\AgdaSymbol{:}\AgdaSpace{}%
\AgdaFunction{Σ[}\AgdaSpace{}%
\AgdaBound{a}\AgdaSpace{}%
\AgdaFunction{∈}\AgdaSpace{}%
\AgdaBound{A}\AgdaSpace{}%
\AgdaFunction{]}\AgdaSpace{}%
\AgdaBound{B}\AgdaSpace{}%
\AgdaBound{a}\AgdaSymbol{)}\AgdaSpace{}%
\AgdaSymbol{→}\AgdaSpace{}%
\AgdaBound{B}\AgdaSpace{}%
\AgdaOperator{\AgdaFunction{∣}}\AgdaSpace{}%
\AgdaBound{z}\AgdaSpace{}%
\AgdaOperator{\AgdaFunction{∣}}\<%
\\
%
\>[1]\AgdaOperator{\AgdaFunction{∥\AgdaUnderscore{}∥}}\AgdaSpace{}%
\AgdaSymbol{=}\AgdaSpace{}%
\AgdaField{snd}\<%
\end{code}

\hypertarget{inverses-of-setoid-functions}{%
\subsubsection{Inverses of setoid
functions}\label{inverses-of-setoid-functions}}

We define a data type that represent the semantic concept of the
\emph{image} of a function (cf.~the
\href{https://ualib.github.io/agda-algebras/Overture.Func.Inverses.html}{Overture.Func.Inverses}
module of the
\href{https://github.com/ualib/agda-algebras}{agda-algebras} library).

\begin{code}%
\>[0]\<%
\\
\>[0]\AgdaKeyword{module}\AgdaSpace{}%
\AgdaModule{\AgdaUnderscore{}}\AgdaSpace{}%
\AgdaSymbol{\{}\AgdaBound{𝑨}\AgdaSpace{}%
\AgdaSymbol{:}\AgdaSpace{}%
\AgdaRecord{Setoid}\AgdaSpace{}%
\AgdaGeneralizable{α}\AgdaSpace{}%
\AgdaGeneralizable{ρᵃ}\AgdaSymbol{\}\{}\AgdaBound{𝑩}\AgdaSpace{}%
\AgdaSymbol{:}\AgdaSpace{}%
\AgdaRecord{Setoid}\AgdaSpace{}%
\AgdaGeneralizable{β}\AgdaSpace{}%
\AgdaGeneralizable{ρᵇ}\AgdaSymbol{\}}\AgdaSpace{}%
\AgdaKeyword{where}\<%
\\
\>[0][@{}l@{\AgdaIndent{0}}]%
\>[1]\AgdaKeyword{open}\AgdaSpace{}%
\AgdaModule{Setoid}\AgdaSpace{}%
\AgdaBound{𝑩}\AgdaSpace{}%
\AgdaKeyword{using}\AgdaSpace{}%
\AgdaSymbol{(}\AgdaSpace{}%
\AgdaOperator{\AgdaField{\AgdaUnderscore{}≈\AgdaUnderscore{}}}\AgdaSpace{}%
\AgdaSymbol{;}\AgdaSpace{}%
\AgdaFunction{sym}\AgdaSpace{}%
\AgdaSymbol{)}\AgdaSpace{}%
\AgdaKeyword{renaming}\AgdaSpace{}%
\AgdaSymbol{(}\AgdaSpace{}%
\AgdaField{Carrier}\AgdaSpace{}%
\AgdaSymbol{to}\AgdaSpace{}%
\AgdaField{B}\AgdaSpace{}%
\AgdaSymbol{)}\<%
\\
%
\\[\AgdaEmptyExtraSkip]%
%
\>[1]\AgdaKeyword{data}\AgdaSpace{}%
\AgdaOperator{\AgdaDatatype{Image\AgdaUnderscore{}∋\AgdaUnderscore{}}}\AgdaSpace{}%
\AgdaSymbol{(}\AgdaBound{f}\AgdaSpace{}%
\AgdaSymbol{:}\AgdaSpace{}%
\AgdaBound{𝑨}\AgdaSpace{}%
\AgdaOperator{\AgdaRecord{⟶}}\AgdaSpace{}%
\AgdaBound{𝑩}\AgdaSymbol{)}\AgdaSpace{}%
\AgdaSymbol{:}\AgdaSpace{}%
\AgdaField{B}\AgdaSpace{}%
\AgdaSymbol{→}\AgdaSpace{}%
\AgdaPrimitive{Type}\AgdaSpace{}%
\AgdaSymbol{(}\AgdaBound{α}\AgdaSpace{}%
\AgdaOperator{\AgdaPrimitive{⊔}}\AgdaSpace{}%
\AgdaBound{β}\AgdaSpace{}%
\AgdaOperator{\AgdaPrimitive{⊔}}\AgdaSpace{}%
\AgdaBound{ρᵇ}\AgdaSymbol{)}\AgdaSpace{}%
\AgdaKeyword{where}\<%
\\
\>[1][@{}l@{\AgdaIndent{0}}]%
\>[2]\AgdaInductiveConstructor{eq}\AgdaSpace{}%
\AgdaSymbol{:}\AgdaSpace{}%
\AgdaSymbol{\{}\AgdaBound{b}\AgdaSpace{}%
\AgdaSymbol{:}\AgdaSpace{}%
\AgdaField{B}\AgdaSymbol{\}}\AgdaSpace{}%
\AgdaSymbol{→}\AgdaSpace{}%
\AgdaSymbol{∀}\AgdaSpace{}%
\AgdaBound{a}\AgdaSpace{}%
\AgdaSymbol{→}\AgdaSpace{}%
\AgdaBound{b}\AgdaSpace{}%
\AgdaOperator{\AgdaField{≈}}\AgdaSpace{}%
\AgdaSymbol{(}\AgdaBound{f}\AgdaSpace{}%
\AgdaOperator{\AgdaField{⟨\$⟩}}\AgdaSpace{}%
\AgdaBound{a}\AgdaSymbol{)}\AgdaSpace{}%
\AgdaSymbol{→}\AgdaSpace{}%
\AgdaOperator{\AgdaDatatype{Image}}\AgdaSpace{}%
\AgdaBound{f}\AgdaSpace{}%
\AgdaOperator{\AgdaDatatype{∋}}\AgdaSpace{}%
\AgdaBound{b}\<%
\\
\>[0]\<%
\end{code}

An inhabitant of \texttt{Image\ f\ ∋\ b} is a dependent pair
\texttt{(a\ ,\ p)}, where \texttt{a\ :\ A} and \texttt{p\ :\ b\ ≈\ f\ a}
is a proof that \texttt{f} maps \texttt{a} to \texttt{b}. Since the
proof that \texttt{b} belongs to the image of \texttt{f} is always
accompanied by a witness \texttt{a\ :\ A}, we can actually
\emph{compute} a (pseudo)inverse of \texttt{f}. For convenience, we
define this inverse function, which we call \texttt{Inv}, and which
takes an arbitrary \texttt{b\ :\ B} and a (witness, proof)-pair,
\texttt{(a\ ,\ p)\ :\ Image\ f\ ∋\ b}, and returns the witness
\texttt{a}.

\begin{code}%
\>[0]\<%
\\
\>[0][@{}l@{\AgdaIndent{1}}]%
\>[1]\AgdaFunction{Inv}\AgdaSpace{}%
\AgdaSymbol{:}\AgdaSpace{}%
\AgdaSymbol{(}\AgdaBound{f}\AgdaSpace{}%
\AgdaSymbol{:}\AgdaSpace{}%
\AgdaBound{𝑨}\AgdaSpace{}%
\AgdaOperator{\AgdaRecord{⟶}}\AgdaSpace{}%
\AgdaBound{𝑩}\AgdaSymbol{)\{}\AgdaBound{b}\AgdaSpace{}%
\AgdaSymbol{:}\AgdaSpace{}%
\AgdaField{B}\AgdaSymbol{\}}\AgdaSpace{}%
\AgdaSymbol{→}\AgdaSpace{}%
\AgdaOperator{\AgdaDatatype{Image}}\AgdaSpace{}%
\AgdaBound{f}\AgdaSpace{}%
\AgdaOperator{\AgdaDatatype{∋}}\AgdaSpace{}%
\AgdaBound{b}\AgdaSpace{}%
\AgdaSymbol{→}\AgdaSpace{}%
\AgdaField{Carrier}\AgdaSpace{}%
\AgdaBound{𝑨}\<%
\\
%
\>[1]\AgdaFunction{Inv}\AgdaSpace{}%
\AgdaSymbol{\AgdaUnderscore{}}\AgdaSpace{}%
\AgdaSymbol{(}\AgdaInductiveConstructor{eq}\AgdaSpace{}%
\AgdaBound{a}\AgdaSpace{}%
\AgdaSymbol{\AgdaUnderscore{})}\AgdaSpace{}%
\AgdaSymbol{=}\AgdaSpace{}%
\AgdaBound{a}\<%
\\
%
\\[\AgdaEmptyExtraSkip]%
%
\>[1]\AgdaFunction{InvIsInverseʳ}\AgdaSpace{}%
\AgdaSymbol{:}\AgdaSpace{}%
\AgdaSymbol{\{}\AgdaBound{f}\AgdaSpace{}%
\AgdaSymbol{:}\AgdaSpace{}%
\AgdaBound{𝑨}\AgdaSpace{}%
\AgdaOperator{\AgdaRecord{⟶}}\AgdaSpace{}%
\AgdaBound{𝑩}\AgdaSymbol{\}\{}\AgdaBound{b}\AgdaSpace{}%
\AgdaSymbol{:}\AgdaSpace{}%
\AgdaField{B}\AgdaSymbol{\}(}\AgdaBound{q}\AgdaSpace{}%
\AgdaSymbol{:}\AgdaSpace{}%
\AgdaOperator{\AgdaDatatype{Image}}\AgdaSpace{}%
\AgdaBound{f}\AgdaSpace{}%
\AgdaOperator{\AgdaDatatype{∋}}\AgdaSpace{}%
\AgdaBound{b}\AgdaSymbol{)}\AgdaSpace{}%
\AgdaSymbol{→}\AgdaSpace{}%
\AgdaSymbol{(}\AgdaBound{f}\AgdaSpace{}%
\AgdaOperator{\AgdaField{⟨\$⟩}}\AgdaSpace{}%
\AgdaSymbol{(}\AgdaFunction{Inv}\AgdaSpace{}%
\AgdaBound{f}\AgdaSpace{}%
\AgdaBound{q}\AgdaSymbol{))}\AgdaSpace{}%
\AgdaOperator{\AgdaField{≈}}\AgdaSpace{}%
\AgdaBound{b}\<%
\\
%
\>[1]\AgdaFunction{InvIsInverseʳ}\AgdaSpace{}%
\AgdaSymbol{(}\AgdaInductiveConstructor{eq}\AgdaSpace{}%
\AgdaSymbol{\AgdaUnderscore{}}\AgdaSpace{}%
\AgdaBound{p}\AgdaSymbol{)}\AgdaSpace{}%
\AgdaSymbol{=}\AgdaSpace{}%
\AgdaFunction{sym}\AgdaSpace{}%
\AgdaBound{p}\<%
\\
\>[0]\<%
\end{code}

\texttt{InvIsInverseʳ} proves that \texttt{Inv\ f} is the
range-restricted right-inverse of the setoid function \texttt{f}.

\hypertarget{injective-and-surjective-setoid-functions}{%
\subsubsection{Injective and surjective setoid
functions}\label{injective-and-surjective-setoid-functions}}

If \texttt{f\ :\ 𝑨\ ⟶\ 𝑩} is a setoid function from
\texttt{𝑨\ =\ (A\ ,\ ≈₀)} to \texttt{𝑩\ =\ (B\ ,\ ≈₁)}, then we call
\texttt{f} \emph{injective} provided \texttt{∀\ (a₀\ a₁\ :\ A)},
\texttt{f\ ⟨\$⟩\ a₀\ ≈₁\ f\ ⟨\$⟩\ a₁} implies \texttt{a₀\ ≈₀\ a₁}; we
call \texttt{f} \emph{surjective} provided \texttt{∀\ (b\ :\ B)},
\texttt{∃\ (a\ :\ A)} such that \texttt{f\ ⟨\$⟩\ a\ ≈₁\ b}. We codify
these definitions in Agda and prove some of their properties inside the
next submodule where we first set the stage by declaring two setoids
\texttt{𝑨} and \texttt{𝑩}, naming their equality relations, and making
some definitions from the standard library available.

\begin{code}%
\>[0]\<%
\\
\>[0]\AgdaKeyword{module}\AgdaSpace{}%
\AgdaModule{\AgdaUnderscore{}}\AgdaSpace{}%
\AgdaSymbol{\{}\AgdaBound{𝑨}\AgdaSpace{}%
\AgdaSymbol{:}\AgdaSpace{}%
\AgdaRecord{Setoid}\AgdaSpace{}%
\AgdaGeneralizable{α}\AgdaSpace{}%
\AgdaGeneralizable{ρᵃ}\AgdaSymbol{\}\{}\AgdaBound{𝑩}\AgdaSpace{}%
\AgdaSymbol{:}\AgdaSpace{}%
\AgdaRecord{Setoid}\AgdaSpace{}%
\AgdaGeneralizable{β}\AgdaSpace{}%
\AgdaGeneralizable{ρᵇ}\AgdaSymbol{\}}\AgdaSpace{}%
\AgdaKeyword{where}\<%
\\
\>[0][@{}l@{\AgdaIndent{0}}]%
\>[1]\AgdaKeyword{open}\AgdaSpace{}%
\AgdaModule{Setoid}\AgdaSpace{}%
\AgdaBound{𝑨}\AgdaSpace{}%
\AgdaKeyword{using}\AgdaSpace{}%
\AgdaSymbol{()}\AgdaSpace{}%
\AgdaKeyword{renaming}\AgdaSpace{}%
\AgdaSymbol{(}\AgdaSpace{}%
\AgdaOperator{\AgdaField{\AgdaUnderscore{}≈\AgdaUnderscore{}}}\AgdaSpace{}%
\AgdaSymbol{to}\AgdaSpace{}%
\AgdaOperator{\AgdaField{\AgdaUnderscore{}≈₁\AgdaUnderscore{}}}\AgdaSpace{}%
\AgdaSymbol{;}\AgdaSpace{}%
\AgdaField{Carrier}\AgdaSpace{}%
\AgdaSymbol{to}\AgdaSpace{}%
\AgdaField{A}\AgdaSpace{}%
\AgdaSymbol{)}\<%
\\
%
\>[1]\AgdaKeyword{open}\AgdaSpace{}%
\AgdaModule{Setoid}\AgdaSpace{}%
\AgdaBound{𝑩}\AgdaSpace{}%
\AgdaKeyword{using}\AgdaSpace{}%
\AgdaSymbol{()}\AgdaSpace{}%
\AgdaKeyword{renaming}\AgdaSpace{}%
\AgdaSymbol{(}\AgdaSpace{}%
\AgdaOperator{\AgdaField{\AgdaUnderscore{}≈\AgdaUnderscore{}}}\AgdaSpace{}%
\AgdaSymbol{to}\AgdaSpace{}%
\AgdaOperator{\AgdaField{\AgdaUnderscore{}≈₂\AgdaUnderscore{}}}\AgdaSpace{}%
\AgdaSymbol{;}\AgdaSpace{}%
\AgdaField{Carrier}\AgdaSpace{}%
\AgdaSymbol{to}\AgdaSpace{}%
\AgdaField{B}\AgdaSpace{}%
\AgdaSymbol{)}\<%
\\
%
\>[1]\AgdaKeyword{open}\AgdaSpace{}%
\AgdaModule{FD}\AgdaSpace{}%
\AgdaOperator{\AgdaFunction{\AgdaUnderscore{}≈₁\AgdaUnderscore{}}}\AgdaSpace{}%
\AgdaOperator{\AgdaField{\AgdaUnderscore{}≈₂\AgdaUnderscore{}}}\<%
\\
\>[0]\<%
\end{code}

The \href{https://agda.github.io/agda-stdlib/}{Agda Standard Library}
represents injective functions on bare types by the type
\texttt{Injective}, which we now use to define \texttt{IsInjective}
representing the property of being an injective setoid function. We then
define the type \texttt{IsSurjective} to represent the property of being
a surjective setoid function. Finally, we define \texttt{SurjInv} to
represent the \emph{right-inverse} of a surjective function. The
definitions are as follows
(cf.~\href{https://ualib.github.io/agda-algebras/Overture.Func.Injective.html}{Overture.Func.Injective}
and
\href{https://ualib.github.io/agda-algebras/Overture.Func.Surjective.html}{Overture.Func.Surjective}
in the \href{https://github.com/ualib/agda-algebras}{agda-algebras}
library).

\begin{code}%
\>[0]\<%
\\
\>[0][@{}l@{\AgdaIndent{1}}]%
\>[1]\AgdaFunction{IsInjective}\AgdaSpace{}%
\AgdaSymbol{:}\AgdaSpace{}%
\AgdaSymbol{(}\AgdaBound{𝑨}\AgdaSpace{}%
\AgdaOperator{\AgdaRecord{⟶}}\AgdaSpace{}%
\AgdaBound{𝑩}\AgdaSymbol{)}\AgdaSpace{}%
\AgdaSymbol{→}%
\>[26]\AgdaPrimitive{Type}\AgdaSpace{}%
\AgdaSymbol{(}\AgdaBound{α}\AgdaSpace{}%
\AgdaOperator{\AgdaPrimitive{⊔}}\AgdaSpace{}%
\AgdaBound{ρᵃ}\AgdaSpace{}%
\AgdaOperator{\AgdaPrimitive{⊔}}\AgdaSpace{}%
\AgdaBound{ρᵇ}\AgdaSymbol{)}\<%
\\
%
\>[1]\AgdaFunction{IsInjective}\AgdaSpace{}%
\AgdaBound{f}\AgdaSpace{}%
\AgdaSymbol{=}\AgdaSpace{}%
\AgdaFunction{Injective}\AgdaSpace{}%
\AgdaSymbol{(}\AgdaOperator{\AgdaField{\AgdaUnderscore{}⟨\$⟩\AgdaUnderscore{}}}\AgdaSpace{}%
\AgdaBound{f}\AgdaSymbol{)}\<%
\\
%
\\[\AgdaEmptyExtraSkip]%
%
\>[1]\AgdaFunction{IsSurjective}\AgdaSpace{}%
\AgdaSymbol{:}\AgdaSpace{}%
\AgdaSymbol{(}\AgdaBound{𝑨}\AgdaSpace{}%
\AgdaOperator{\AgdaRecord{⟶}}\AgdaSpace{}%
\AgdaBound{𝑩}\AgdaSymbol{)}\AgdaSpace{}%
\AgdaSymbol{→}%
\>[27]\AgdaPrimitive{Type}\AgdaSpace{}%
\AgdaSymbol{(}\AgdaBound{α}\AgdaSpace{}%
\AgdaOperator{\AgdaPrimitive{⊔}}\AgdaSpace{}%
\AgdaBound{β}\AgdaSpace{}%
\AgdaOperator{\AgdaPrimitive{⊔}}\AgdaSpace{}%
\AgdaBound{ρᵇ}\AgdaSymbol{)}\<%
\\
%
\>[1]\AgdaFunction{IsSurjective}\AgdaSpace{}%
\AgdaBound{F}\AgdaSpace{}%
\AgdaSymbol{=}\AgdaSpace{}%
\AgdaSymbol{∀}\AgdaSpace{}%
\AgdaSymbol{\{}\AgdaBound{y}\AgdaSymbol{\}}\AgdaSpace{}%
\AgdaSymbol{→}\AgdaSpace{}%
\AgdaOperator{\AgdaDatatype{Image}}\AgdaSpace{}%
\AgdaBound{F}\AgdaSpace{}%
\AgdaOperator{\AgdaDatatype{∋}}\AgdaSpace{}%
\AgdaBound{y}\<%
\\
%
\\[\AgdaEmptyExtraSkip]%
%
\>[1]\AgdaFunction{SurjInv}\AgdaSpace{}%
\AgdaSymbol{:}\AgdaSpace{}%
\AgdaSymbol{(}\AgdaBound{f}\AgdaSpace{}%
\AgdaSymbol{:}\AgdaSpace{}%
\AgdaBound{𝑨}\AgdaSpace{}%
\AgdaOperator{\AgdaRecord{⟶}}\AgdaSpace{}%
\AgdaBound{𝑩}\AgdaSymbol{)}\AgdaSpace{}%
\AgdaSymbol{→}\AgdaSpace{}%
\AgdaFunction{IsSurjective}\AgdaSpace{}%
\AgdaBound{f}\AgdaSpace{}%
\AgdaSymbol{→}\AgdaSpace{}%
\AgdaField{B}\AgdaSpace{}%
\AgdaSymbol{→}\AgdaSpace{}%
\AgdaFunction{A}\<%
\\
%
\>[1]\AgdaFunction{SurjInv}\AgdaSpace{}%
\AgdaBound{f}\AgdaSpace{}%
\AgdaBound{fonto}\AgdaSpace{}%
\AgdaBound{b}\AgdaSpace{}%
\AgdaSymbol{=}\AgdaSpace{}%
\AgdaFunction{Inv}\AgdaSpace{}%
\AgdaBound{f}\AgdaSpace{}%
\AgdaSymbol{(}\AgdaBound{fonto}\AgdaSpace{}%
\AgdaSymbol{\{}\AgdaBound{b}\AgdaSymbol{\})}\<%
\end{code}

\hypertarget{composition-of-injective-and-surjective-setoid-functions}{%
\paragraph{Composition of injective and surjective setoid
functions}\label{composition-of-injective-and-surjective-setoid-functions}}

Proving that the composition of injective setoid functions is again
injective is simply a matter of composing the two assumed witnesses to
injectivity. Proving that surjectivity is preserved under composition is
only slightly more involved.

\begin{code}%
\>[0]\<%
\\
\>[0]\AgdaKeyword{module}\AgdaSpace{}%
\AgdaModule{\AgdaUnderscore{}}%
\>[10]\AgdaSymbol{\{}\AgdaBound{𝑨}\AgdaSpace{}%
\AgdaSymbol{:}\AgdaSpace{}%
\AgdaRecord{Setoid}\AgdaSpace{}%
\AgdaGeneralizable{α}\AgdaSpace{}%
\AgdaGeneralizable{ρᵃ}\AgdaSymbol{\}\{}\AgdaBound{𝑩}\AgdaSpace{}%
\AgdaSymbol{:}\AgdaSpace{}%
\AgdaRecord{Setoid}\AgdaSpace{}%
\AgdaGeneralizable{β}\AgdaSpace{}%
\AgdaGeneralizable{ρᵇ}\AgdaSymbol{\}\{}\AgdaBound{𝑪}\AgdaSpace{}%
\AgdaSymbol{:}\AgdaSpace{}%
\AgdaRecord{Setoid}\AgdaSpace{}%
\AgdaGeneralizable{γ}\AgdaSpace{}%
\AgdaGeneralizable{ρᶜ}\AgdaSymbol{\}}\<%
\\
%
\>[10]\AgdaSymbol{(}\AgdaBound{f}\AgdaSpace{}%
\AgdaSymbol{:}\AgdaSpace{}%
\AgdaBound{𝑨}\AgdaSpace{}%
\AgdaOperator{\AgdaRecord{⟶}}\AgdaSpace{}%
\AgdaBound{𝑩}\AgdaSymbol{)(}\AgdaBound{g}\AgdaSpace{}%
\AgdaSymbol{:}\AgdaSpace{}%
\AgdaBound{𝑩}\AgdaSpace{}%
\AgdaOperator{\AgdaRecord{⟶}}\AgdaSpace{}%
\AgdaBound{𝑪}\AgdaSymbol{)}\AgdaSpace{}%
\AgdaKeyword{where}\<%
\\
%
\\[\AgdaEmptyExtraSkip]%
\>[0][@{}l@{\AgdaIndent{0}}]%
\>[1]\AgdaFunction{∘-IsInjective}\AgdaSpace{}%
\AgdaSymbol{:}\AgdaSpace{}%
\AgdaFunction{IsInjective}\AgdaSpace{}%
\AgdaBound{f}\AgdaSpace{}%
\AgdaSymbol{→}\AgdaSpace{}%
\AgdaFunction{IsInjective}\AgdaSpace{}%
\AgdaBound{g}\AgdaSpace{}%
\AgdaSymbol{→}\AgdaSpace{}%
\AgdaFunction{IsInjective}\AgdaSpace{}%
\AgdaSymbol{(}\AgdaBound{g}\AgdaSpace{}%
\AgdaOperator{\AgdaFunction{⟨∘⟩}}\AgdaSpace{}%
\AgdaBound{f}\AgdaSymbol{)}\<%
\\
%
\>[1]\AgdaFunction{∘-IsInjective}\AgdaSpace{}%
\AgdaBound{finj}\AgdaSpace{}%
\AgdaBound{ginj}\AgdaSpace{}%
\AgdaSymbol{=}\AgdaSpace{}%
\AgdaBound{finj}\AgdaSpace{}%
\AgdaOperator{\AgdaFunction{∘}}\AgdaSpace{}%
\AgdaBound{ginj}\<%
\\
%
\\[\AgdaEmptyExtraSkip]%
%
\>[1]\AgdaFunction{∘-IsSurjective}\AgdaSpace{}%
\AgdaSymbol{:}\AgdaSpace{}%
\AgdaFunction{IsSurjective}\AgdaSpace{}%
\AgdaBound{f}\AgdaSpace{}%
\AgdaSymbol{→}\AgdaSpace{}%
\AgdaFunction{IsSurjective}\AgdaSpace{}%
\AgdaBound{g}\AgdaSpace{}%
\AgdaSymbol{→}\AgdaSpace{}%
\AgdaFunction{IsSurjective}\AgdaSpace{}%
\AgdaSymbol{(}\AgdaBound{g}\AgdaSpace{}%
\AgdaOperator{\AgdaFunction{⟨∘⟩}}\AgdaSpace{}%
\AgdaBound{f}\AgdaSymbol{)}\<%
\\
%
\>[1]\AgdaFunction{∘-IsSurjective}\AgdaSpace{}%
\AgdaBound{fonto}\AgdaSpace{}%
\AgdaBound{gonto}\AgdaSpace{}%
\AgdaSymbol{\{}\AgdaBound{y}\AgdaSymbol{\}}\AgdaSpace{}%
\AgdaSymbol{=}\AgdaSpace{}%
\AgdaFunction{Goal}\<%
\\
\>[1][@{}l@{\AgdaIndent{0}}]%
\>[2]\AgdaKeyword{where}\<%
\\
%
\>[2]\AgdaFunction{mp}\AgdaSpace{}%
\AgdaSymbol{:}\AgdaSpace{}%
\AgdaOperator{\AgdaDatatype{Image}}\AgdaSpace{}%
\AgdaBound{g}\AgdaSpace{}%
\AgdaOperator{\AgdaDatatype{∋}}\AgdaSpace{}%
\AgdaBound{y}\AgdaSpace{}%
\AgdaSymbol{→}\AgdaSpace{}%
\AgdaOperator{\AgdaDatatype{Image}}\AgdaSpace{}%
\AgdaBound{g}\AgdaSpace{}%
\AgdaOperator{\AgdaFunction{⟨∘⟩}}\AgdaSpace{}%
\AgdaBound{f}\AgdaSpace{}%
\AgdaOperator{\AgdaDatatype{∋}}\AgdaSpace{}%
\AgdaBound{y}\<%
\\
%
\>[2]\AgdaFunction{mp}\AgdaSpace{}%
\AgdaSymbol{(}\AgdaInductiveConstructor{eq}\AgdaSpace{}%
\AgdaBound{c}\AgdaSpace{}%
\AgdaBound{p}\AgdaSymbol{)}\AgdaSpace{}%
\AgdaSymbol{=}\AgdaSpace{}%
\AgdaFunction{η}\AgdaSpace{}%
\AgdaBound{fonto}\<%
\\
\>[2][@{}l@{\AgdaIndent{0}}]%
\>[3]\AgdaKeyword{where}\<%
\\
%
\>[3]\AgdaKeyword{open}\AgdaSpace{}%
\AgdaModule{Setoid}\AgdaSpace{}%
\AgdaBound{𝑪}\AgdaSpace{}%
\AgdaKeyword{using}\AgdaSpace{}%
\AgdaSymbol{(}\AgdaSpace{}%
\AgdaFunction{trans}\AgdaSpace{}%
\AgdaSymbol{)}\<%
\\
%
\>[3]\AgdaFunction{η}\AgdaSpace{}%
\AgdaSymbol{:}\AgdaSpace{}%
\AgdaOperator{\AgdaDatatype{Image}}\AgdaSpace{}%
\AgdaBound{f}\AgdaSpace{}%
\AgdaOperator{\AgdaDatatype{∋}}\AgdaSpace{}%
\AgdaBound{c}\AgdaSpace{}%
\AgdaSymbol{→}\AgdaSpace{}%
\AgdaOperator{\AgdaDatatype{Image}}\AgdaSpace{}%
\AgdaBound{g}\AgdaSpace{}%
\AgdaOperator{\AgdaFunction{⟨∘⟩}}\AgdaSpace{}%
\AgdaBound{f}\AgdaSpace{}%
\AgdaOperator{\AgdaDatatype{∋}}\AgdaSpace{}%
\AgdaBound{y}\<%
\\
%
\>[3]\AgdaFunction{η}\AgdaSpace{}%
\AgdaSymbol{(}\AgdaInductiveConstructor{eq}\AgdaSpace{}%
\AgdaBound{a}\AgdaSpace{}%
\AgdaBound{q}\AgdaSymbol{)}\AgdaSpace{}%
\AgdaSymbol{=}\AgdaSpace{}%
\AgdaInductiveConstructor{eq}\AgdaSpace{}%
\AgdaBound{a}\AgdaSpace{}%
\AgdaSymbol{(}\AgdaFunction{trans}\AgdaSpace{}%
\AgdaBound{p}\AgdaSpace{}%
\AgdaSymbol{(}\AgdaField{cong}\AgdaSpace{}%
\AgdaBound{g}\AgdaSpace{}%
\AgdaBound{q}\AgdaSymbol{))}\<%
\\
%
\\[\AgdaEmptyExtraSkip]%
%
\>[2]\AgdaFunction{Goal}\AgdaSpace{}%
\AgdaSymbol{:}\AgdaSpace{}%
\AgdaOperator{\AgdaDatatype{Image}}\AgdaSpace{}%
\AgdaBound{g}\AgdaSpace{}%
\AgdaOperator{\AgdaFunction{⟨∘⟩}}\AgdaSpace{}%
\AgdaBound{f}\AgdaSpace{}%
\AgdaOperator{\AgdaDatatype{∋}}\AgdaSpace{}%
\AgdaBound{y}\<%
\\
%
\>[2]\AgdaFunction{Goal}\AgdaSpace{}%
\AgdaSymbol{=}\AgdaSpace{}%
\AgdaFunction{mp}\AgdaSpace{}%
\AgdaBound{gonto}\<%
\end{code}

\hypertarget{kernels}{%
\subsubsection{Kernels}\label{kernels}}

The \emph{kernel} of a function \texttt{f\ :\ A\ →\ B} (where \texttt{A}
and \texttt{B} are bare types) is defined informally by
\texttt{\{(x\ ,\ y)\ ∈\ A\ ×\ A\ :\ f\ x\ =\ f\ y\}}. This can be
represented in Agda in a number of ways, but for our purposes we find it
most convenient to define the kernel as an inhabitant of a (unary)
predicate over the square of the function's domain.

\begin{code}%
\>[0]\<%
\\
\>[0]\AgdaKeyword{module}\AgdaSpace{}%
\AgdaModule{\AgdaUnderscore{}}\AgdaSpace{}%
\AgdaSymbol{\{}\AgdaBound{A}\AgdaSpace{}%
\AgdaSymbol{:}\AgdaSpace{}%
\AgdaPrimitive{Type}\AgdaSpace{}%
\AgdaGeneralizable{α}\AgdaSymbol{\}\{}\AgdaBound{B}\AgdaSpace{}%
\AgdaSymbol{:}\AgdaSpace{}%
\AgdaPrimitive{Type}\AgdaSpace{}%
\AgdaGeneralizable{β}\AgdaSymbol{\}}\AgdaSpace{}%
\AgdaKeyword{where}\<%
\\
%
\\[\AgdaEmptyExtraSkip]%
\>[0][@{}l@{\AgdaIndent{0}}]%
\>[1]\AgdaFunction{kernel}\AgdaSpace{}%
\AgdaSymbol{:}\AgdaSpace{}%
\AgdaFunction{Rel}\AgdaSpace{}%
\AgdaBound{B}\AgdaSpace{}%
\AgdaGeneralizable{ρ}\AgdaSpace{}%
\AgdaSymbol{→}\AgdaSpace{}%
\AgdaSymbol{(}\AgdaBound{A}\AgdaSpace{}%
\AgdaSymbol{→}\AgdaSpace{}%
\AgdaBound{B}\AgdaSymbol{)}\AgdaSpace{}%
\AgdaSymbol{→}\AgdaSpace{}%
\AgdaFunction{Pred}\AgdaSpace{}%
\AgdaSymbol{(}\AgdaBound{A}\AgdaSpace{}%
\AgdaOperator{\AgdaFunction{×}}\AgdaSpace{}%
\AgdaBound{A}\AgdaSymbol{)}\AgdaSpace{}%
\AgdaGeneralizable{ρ}\<%
\\
%
\>[1]\AgdaFunction{kernel}\AgdaSpace{}%
\AgdaOperator{\AgdaBound{\AgdaUnderscore{}≈\AgdaUnderscore{}}}\AgdaSpace{}%
\AgdaBound{f}\AgdaSpace{}%
\AgdaSymbol{(}\AgdaBound{x}\AgdaSpace{}%
\AgdaOperator{\AgdaInductiveConstructor{,}}\AgdaSpace{}%
\AgdaBound{y}\AgdaSymbol{)}\AgdaSpace{}%
\AgdaSymbol{=}\AgdaSpace{}%
\AgdaBound{f}\AgdaSpace{}%
\AgdaBound{x}\AgdaSpace{}%
\AgdaOperator{\AgdaBound{≈}}\AgdaSpace{}%
\AgdaBound{f}\AgdaSpace{}%
\AgdaBound{y}\<%
\\
\>[0]\<%
\end{code}

The kernel of a setoid function \texttt{f\ :\ 𝐴\ ⟶\ 𝐵} is defined
informally by
\texttt{\{(x\ ,\ y)\ ∈\ A\ ×\ A\ :\ f\ ⟨\$⟩\ x\ ≈\ f\ ⟨\$⟩\ y\}}, where
\texttt{\_≈\_} denotes the equality of \texttt{𝐵}.

\begin{code}%
\>[0]\<%
\\
\>[0]\AgdaKeyword{module}\AgdaSpace{}%
\AgdaModule{\AgdaUnderscore{}}\AgdaSpace{}%
\AgdaSymbol{\{}\AgdaBound{𝐴}\AgdaSpace{}%
\AgdaSymbol{:}\AgdaSpace{}%
\AgdaRecord{Setoid}\AgdaSpace{}%
\AgdaGeneralizable{α}\AgdaSpace{}%
\AgdaGeneralizable{ρᵃ}\AgdaSymbol{\}\{}\AgdaBound{𝐵}\AgdaSpace{}%
\AgdaSymbol{:}\AgdaSpace{}%
\AgdaRecord{Setoid}\AgdaSpace{}%
\AgdaGeneralizable{β}\AgdaSpace{}%
\AgdaGeneralizable{ρᵇ}\AgdaSymbol{\}}\AgdaSpace{}%
\AgdaKeyword{where}\<%
\\
\>[0][@{}l@{\AgdaIndent{0}}]%
\>[1]\AgdaKeyword{open}\AgdaSpace{}%
\AgdaModule{Setoid}\AgdaSpace{}%
\AgdaBound{𝐴}\AgdaSpace{}%
\AgdaKeyword{using}\AgdaSpace{}%
\AgdaSymbol{()}\AgdaSpace{}%
\AgdaKeyword{renaming}\AgdaSpace{}%
\AgdaSymbol{(}\AgdaSpace{}%
\AgdaField{Carrier}\AgdaSpace{}%
\AgdaSymbol{to}\AgdaSpace{}%
\AgdaField{A}\AgdaSpace{}%
\AgdaSymbol{)}\<%
\\
%
\\[\AgdaEmptyExtraSkip]%
%
\>[1]\AgdaFunction{ker}\AgdaSpace{}%
\AgdaSymbol{:}\AgdaSpace{}%
\AgdaSymbol{(}\AgdaBound{𝐴}\AgdaSpace{}%
\AgdaOperator{\AgdaRecord{⟶}}\AgdaSpace{}%
\AgdaBound{𝐵}\AgdaSymbol{)}\AgdaSpace{}%
\AgdaSymbol{→}\AgdaSpace{}%
\AgdaFunction{Pred}\AgdaSpace{}%
\AgdaSymbol{(}\AgdaFunction{A}\AgdaSpace{}%
\AgdaOperator{\AgdaFunction{×}}\AgdaSpace{}%
\AgdaFunction{A}\AgdaSymbol{)}\AgdaSpace{}%
\AgdaBound{ρᵇ}\<%
\\
%
\>[1]\AgdaFunction{ker}\AgdaSpace{}%
\AgdaBound{g}\AgdaSpace{}%
\AgdaSymbol{(}\AgdaBound{x}\AgdaSpace{}%
\AgdaOperator{\AgdaInductiveConstructor{,}}\AgdaSpace{}%
\AgdaBound{y}\AgdaSymbol{)}\AgdaSpace{}%
\AgdaSymbol{=}\AgdaSpace{}%
\AgdaBound{g}\AgdaSpace{}%
\AgdaOperator{\AgdaField{⟨\$⟩}}\AgdaSpace{}%
\AgdaBound{x}\AgdaSpace{}%
\AgdaOperator{\AgdaFunction{≈}}\AgdaSpace{}%
\AgdaBound{g}\AgdaSpace{}%
\AgdaOperator{\AgdaField{⟨\$⟩}}\AgdaSpace{}%
\AgdaBound{y}\AgdaSpace{}%
\AgdaKeyword{where}\AgdaSpace{}%
\AgdaKeyword{open}\AgdaSpace{}%
\AgdaModule{Setoid}\AgdaSpace{}%
\AgdaBound{𝐵}\AgdaSpace{}%
\AgdaKeyword{using}\AgdaSpace{}%
\AgdaSymbol{(}\AgdaSpace{}%
\AgdaOperator{\AgdaField{\AgdaUnderscore{}≈\AgdaUnderscore{}}}\AgdaSpace{}%
\AgdaSymbol{)}\<%
\end{code}

\hypertarget{algebras}{%
\subsection{Algebras}\label{algebras}}

In this section we define the notion of an algebraic structure whose
\emph{domain} (or ``carrier'' or ``universe'') is a setoid. Our first
goal is to develop a working vocabulary and formal types for classical
(single-sorted, set-based) universal algebra.

\hypertarget{signatures-of-an-algebra}{%
\subsubsection{Signatures of an
algebra}\label{signatures-of-an-algebra}}

In \href{https://en.wikipedia.org/wiki/Model_theory}{model theory}, the
\emph{signature} \texttt{𝑆\ =\ (𝐶,\ 𝐹,\ 𝑅,\ ρ)} of a structure consists
of three (possibly empty) sets \texttt{𝐶}, \texttt{𝐹}, and
\texttt{𝑅}---called \emph{constant symbols}, \emph{function symbols},
and \emph{relation symbols}, respectively---along with a function
\texttt{ρ\ :\ 𝐶\ +\ 𝐹\ +\ 𝑅\ →\ 𝑁} that assigns an \emph{arity} to each
symbol. Often, but not always, \texttt{𝑁} is taken to be the set of
natural numbers.

As our focus here is universal algebra, we are more concerned with the
restricted notion of an \emph{algebraic signature} (or \emph{signature}
for algebraic structures), by which we mean a pair
\texttt{𝑆\ =\ (F,\ ρ)} consisting of a collection \texttt{𝐹} of
\emph{operation symbols} and an \emph{arity function}
\texttt{ρ\ :\ F\ →\ N} that maps each operation symbol to its arity;
here, 𝑁 denotes the \emph{arity type}. Heuristically, the arity
\texttt{ρ\ f} of an operation symbol \texttt{f\ ∈\ F} may be thought of
as the ``number of arguments'' that \texttt{f} takes as ``input''.

If the arity of \texttt{f} is \texttt{n}, then we call \texttt{f} an
\texttt{n}-\emph{ary} operation symbol. In case \texttt{n} is 0 (or 1 or
2 or 3, respectively) we call the function \emph{nullary} (or
\emph{unary} or \emph{binary} or \emph{ternary}, respectively).

If \texttt{A} is a set and \texttt{f} is a (\texttt{ρ\ f})-ary operation
on \texttt{A}, then the arguments of \texttt{f} form a
(\texttt{ρ\ f})-tuple, say,
\texttt{(a\ 0,\ a\ 1,\ \ldots{},\ a\ (ρf-1))}, which may be viewed as
the graph of a function, say, \texttt{a\ :\ ρf\ →\ A}. When the codomain
of \texttt{ρ} is \texttt{ℕ}, we may view \texttt{ρ\ f} as the finite set
\texttt{\{0,\ 1,\ \ldots{},\ ρf\ -\ 1\}}. Thus, by identifying the
\texttt{ρf}-th power of \texttt{A} with the type \texttt{ρ\ f\ →\ A} of
functions from \texttt{\{0,\ 1,\ \ldots{},\ ρf\ -\ 1\}} to \texttt{A},
we identify the collection of all tuples of arguments of \texttt{f} with
the function type \texttt{(ρf\ →\ A)\ →\ A}.

\hypertarget{signature-type}{%
\paragraph{Signature type}\label{signature-type}}

The \href{https://github.com/ualib/agda-algebras}{agda-algebras} library
represents a \emph{signature} as an inhabitant of the following
dependent pair type.

\begin{verbatim}
Signature : (𝓞 𝓥 : Level) → Type (lsuc (𝓞 ⊔ 𝓥))
Signature 𝓞 𝓥 = Σ[ F ∈ Type 𝓞 ] (F → Type 𝓥)
\end{verbatim}

Using special syntax for the first and second
projections---\texttt{∣\_∣} and \texttt{∥\_∥}, respectively---if
\texttt{𝑆\ :\ Signature\ 𝓞\ 𝓥} is a signature, then

\begin{itemize}
\tightlist
\item
  \texttt{∣\ 𝑆\ ∣} denotes the set of operation symbols;
\item
  \texttt{∥\ 𝑆\ ∥} denotes the arity function.
\end{itemize}

Thus, if \texttt{𝑓\ :\ ∣\ 𝑆\ ∣} is an operation symbol in the signature
\texttt{𝑆}, then \texttt{∥\ 𝑆\ ∥\ 𝑓} is the arity of \texttt{𝑓}.

We need to augment the ordinary \texttt{Signature} type so that it
supports algebras over setoid domains. To do so, we define an operator
\texttt{⟨\_⟩} which translates an ordinary signature into a setoid
signature, that is, a signature over a setoid domain. But first we must
resolve a techinical issue involving dependent types that we now
describe.

Suppose we are given two operations \texttt{f} and \texttt{g} and we
have a tuple of arguments for \texttt{f}, say,
\texttt{u\ :\ ∥\ 𝑆\ ∥\ f\ →\ A}, and a tuple of arguments for
\texttt{g}, say, \texttt{v\ :\ ∥\ 𝑆\ ∥\ g\ →\ A}. If we know that
\texttt{f} is identically equal to \texttt{g}---that is,
\texttt{f\ ≡\ g} (intensionally)---then we should be able to check
whether \texttt{u} and \texttt{v} are pointwise equal. The problem here
is that \texttt{u} and \texttt{v} ostensibly inhabit different types. To
compare \texttt{u} and \texttt{v} we must convince Agda that, from
\texttt{f\ ≡\ g} we can deduce that \texttt{u} and \texttt{v} are
actually of the same type. The type \texttt{EqArgs} (defined below, and
adapted from Andreas Abel's development {[}ref needed{]}) resolves this
technical issue nicely.

\begin{code}%
\>[0]\<%
\\
\>[0]\AgdaFunction{EqArgs}\AgdaSpace{}%
\AgdaSymbol{:}%
\>[10]\AgdaSymbol{\{}\AgdaBound{𝑆}\AgdaSpace{}%
\AgdaSymbol{:}\AgdaSpace{}%
\AgdaFunction{Signature}\AgdaSpace{}%
\AgdaBound{𝓞}\AgdaSpace{}%
\AgdaBound{𝓥}\AgdaSymbol{\}\{}\AgdaBound{ξ}\AgdaSpace{}%
\AgdaSymbol{:}\AgdaSpace{}%
\AgdaRecord{Setoid}\AgdaSpace{}%
\AgdaGeneralizable{α}\AgdaSpace{}%
\AgdaGeneralizable{ρᵃ}\AgdaSymbol{\}}\<%
\\
\>[0][@{}l@{\AgdaIndent{0}}]%
\>[1]\AgdaSymbol{→}%
\>[10]\AgdaSymbol{∀}\AgdaSpace{}%
\AgdaSymbol{\{}\AgdaBound{f}\AgdaSpace{}%
\AgdaBound{g}\AgdaSymbol{\}}\AgdaSpace{}%
\AgdaSymbol{→}\AgdaSpace{}%
\AgdaBound{f}\AgdaSpace{}%
\AgdaOperator{\AgdaDatatype{≡}}\AgdaSpace{}%
\AgdaBound{g}\AgdaSpace{}%
\AgdaSymbol{→}\AgdaSpace{}%
\AgdaSymbol{(}\AgdaOperator{\AgdaFunction{∥}}\AgdaSpace{}%
\AgdaBound{𝑆}\AgdaSpace{}%
\AgdaOperator{\AgdaFunction{∥}}\AgdaSpace{}%
\AgdaBound{f}\AgdaSpace{}%
\AgdaSymbol{→}\AgdaSpace{}%
\AgdaField{Carrier}\AgdaSpace{}%
\AgdaBound{ξ}\AgdaSymbol{)}\AgdaSpace{}%
\AgdaSymbol{→}\AgdaSpace{}%
\AgdaSymbol{(}\AgdaOperator{\AgdaFunction{∥}}\AgdaSpace{}%
\AgdaBound{𝑆}\AgdaSpace{}%
\AgdaOperator{\AgdaFunction{∥}}\AgdaSpace{}%
\AgdaBound{g}\AgdaSpace{}%
\AgdaSymbol{→}\AgdaSpace{}%
\AgdaField{Carrier}\AgdaSpace{}%
\AgdaBound{ξ}\AgdaSymbol{)}\AgdaSpace{}%
\AgdaSymbol{→}\AgdaSpace{}%
\AgdaPrimitive{Type}\AgdaSpace{}%
\AgdaSymbol{(}\AgdaBound{𝓥}\AgdaSpace{}%
\AgdaOperator{\AgdaPrimitive{⊔}}\AgdaSpace{}%
\AgdaGeneralizable{ρᵃ}\AgdaSymbol{)}\<%
\\
%
\\[\AgdaEmptyExtraSkip]%
\>[0]\AgdaFunction{EqArgs}\AgdaSpace{}%
\AgdaSymbol{\{}\AgdaArgument{ξ}\AgdaSpace{}%
\AgdaSymbol{=}\AgdaSpace{}%
\AgdaBound{ξ}\AgdaSymbol{\}}\AgdaSpace{}%
\AgdaInductiveConstructor{≡.refl}\AgdaSpace{}%
\AgdaBound{u}\AgdaSpace{}%
\AgdaBound{v}\AgdaSpace{}%
\AgdaSymbol{=}\AgdaSpace{}%
\AgdaSymbol{∀}\AgdaSpace{}%
\AgdaBound{i}\AgdaSpace{}%
\AgdaSymbol{→}\AgdaSpace{}%
\AgdaBound{u}\AgdaSpace{}%
\AgdaBound{i}\AgdaSpace{}%
\AgdaOperator{\AgdaFunction{≈}}\AgdaSpace{}%
\AgdaBound{v}\AgdaSpace{}%
\AgdaBound{i}\AgdaSpace{}%
\AgdaKeyword{where}\AgdaSpace{}%
\AgdaKeyword{open}\AgdaSpace{}%
\AgdaModule{Setoid}\AgdaSpace{}%
\AgdaBound{ξ}\AgdaSpace{}%
\AgdaKeyword{using}\AgdaSpace{}%
\AgdaSymbol{(}\AgdaSpace{}%
\AgdaOperator{\AgdaField{\AgdaUnderscore{}≈\AgdaUnderscore{}}}\AgdaSpace{}%
\AgdaSymbol{)}\<%
\\
\>[0]\<%
\end{code}

Now we are in a position to define the \texttt{⟨\_⟩} operator, which
translates an ordinary signature into a setoid signature.

\begin{code}%
\>[0]\<%
\\
\>[0]\AgdaKeyword{module}\AgdaSpace{}%
\AgdaModule{\AgdaUnderscore{}}\AgdaSpace{}%
\AgdaKeyword{where}\<%
\\
\>[0][@{}l@{\AgdaIndent{0}}]%
\>[1]\AgdaKeyword{open}\AgdaSpace{}%
\AgdaModule{Setoid}%
\>[20]\AgdaKeyword{using}\AgdaSpace{}%
\AgdaSymbol{(}\AgdaSpace{}%
\AgdaOperator{\AgdaField{\AgdaUnderscore{}≈\AgdaUnderscore{}}}\AgdaSpace{}%
\AgdaSymbol{)}\<%
\\
%
\>[1]\AgdaKeyword{open}\AgdaSpace{}%
\AgdaModule{IsEquivalence}\AgdaSpace{}%
\AgdaKeyword{using}\AgdaSpace{}%
\AgdaSymbol{(}\AgdaSpace{}%
\AgdaField{refl}\AgdaSpace{}%
\AgdaSymbol{;}\AgdaSpace{}%
\AgdaField{sym}\AgdaSpace{}%
\AgdaSymbol{;}\AgdaSpace{}%
\AgdaField{trans}\AgdaSpace{}%
\AgdaSymbol{)}\<%
\\
%
\\[\AgdaEmptyExtraSkip]%
%
\>[1]\AgdaOperator{\AgdaFunction{⟨\AgdaUnderscore{}⟩}}\AgdaSpace{}%
\AgdaSymbol{:}\AgdaSpace{}%
\AgdaFunction{Signature}\AgdaSpace{}%
\AgdaBound{𝓞}\AgdaSpace{}%
\AgdaBound{𝓥}\AgdaSpace{}%
\AgdaSymbol{→}\AgdaSpace{}%
\AgdaRecord{Setoid}\AgdaSpace{}%
\AgdaGeneralizable{α}\AgdaSpace{}%
\AgdaGeneralizable{ρᵃ}\AgdaSpace{}%
\AgdaSymbol{→}\AgdaSpace{}%
\AgdaRecord{Setoid}\AgdaSpace{}%
\AgdaSymbol{\AgdaUnderscore{}}\AgdaSpace{}%
\AgdaSymbol{\AgdaUnderscore{}}\<%
\\
%
\\[\AgdaEmptyExtraSkip]%
%
\>[1]\AgdaField{Carrier}\AgdaSpace{}%
\AgdaSymbol{(}\AgdaOperator{\AgdaFunction{⟨}}\AgdaSpace{}%
\AgdaBound{𝑆}\AgdaSpace{}%
\AgdaOperator{\AgdaFunction{⟩}}\AgdaSpace{}%
\AgdaBound{ξ}\AgdaSymbol{)}%
\>[32]\AgdaSymbol{=}\AgdaSpace{}%
\AgdaFunction{Σ[}\AgdaSpace{}%
\AgdaBound{f}\AgdaSpace{}%
\AgdaFunction{∈}\AgdaSpace{}%
\AgdaOperator{\AgdaFunction{∣}}\AgdaSpace{}%
\AgdaBound{𝑆}\AgdaSpace{}%
\AgdaOperator{\AgdaFunction{∣}}\AgdaSpace{}%
\AgdaFunction{]}\AgdaSpace{}%
\AgdaSymbol{((}\AgdaOperator{\AgdaFunction{∥}}\AgdaSpace{}%
\AgdaBound{𝑆}\AgdaSpace{}%
\AgdaOperator{\AgdaFunction{∥}}\AgdaSpace{}%
\AgdaBound{f}\AgdaSymbol{)}\AgdaSpace{}%
\AgdaSymbol{→}\AgdaSpace{}%
\AgdaBound{ξ}\AgdaSpace{}%
\AgdaSymbol{.}\AgdaField{Carrier}\AgdaSymbol{)}\<%
\\
%
\\[\AgdaEmptyExtraSkip]%
%
\>[1]\AgdaOperator{\AgdaField{\AgdaUnderscore{}≈\AgdaUnderscore{}}}\AgdaSpace{}%
\AgdaSymbol{(}\AgdaOperator{\AgdaFunction{⟨}}\AgdaSpace{}%
\AgdaBound{𝑆}\AgdaSpace{}%
\AgdaOperator{\AgdaFunction{⟩}}\AgdaSpace{}%
\AgdaBound{ξ}\AgdaSymbol{)}\AgdaSpace{}%
\AgdaSymbol{(}\AgdaBound{f}\AgdaSpace{}%
\AgdaOperator{\AgdaInductiveConstructor{,}}\AgdaSpace{}%
\AgdaBound{u}\AgdaSymbol{)}\AgdaSpace{}%
\AgdaSymbol{(}\AgdaBound{g}\AgdaSpace{}%
\AgdaOperator{\AgdaInductiveConstructor{,}}\AgdaSpace{}%
\AgdaBound{v}\AgdaSymbol{)}%
\>[32]\AgdaSymbol{=}\AgdaSpace{}%
\AgdaFunction{Σ[}\AgdaSpace{}%
\AgdaBound{eqv}\AgdaSpace{}%
\AgdaFunction{∈}\AgdaSpace{}%
\AgdaBound{f}\AgdaSpace{}%
\AgdaOperator{\AgdaDatatype{≡}}\AgdaSpace{}%
\AgdaBound{g}\AgdaSpace{}%
\AgdaFunction{]}\AgdaSpace{}%
\AgdaFunction{EqArgs}\AgdaSymbol{\{}\AgdaArgument{ξ}\AgdaSpace{}%
\AgdaSymbol{=}\AgdaSpace{}%
\AgdaBound{ξ}\AgdaSymbol{\}}\AgdaSpace{}%
\AgdaBound{eqv}\AgdaSpace{}%
\AgdaBound{u}\AgdaSpace{}%
\AgdaBound{v}\<%
\\
%
\\[\AgdaEmptyExtraSkip]%
%
\>[1]\AgdaField{refl}%
\>[8]\AgdaSymbol{(}\AgdaField{isEquivalence}\AgdaSpace{}%
\AgdaSymbol{(}\AgdaOperator{\AgdaFunction{⟨}}\AgdaSpace{}%
\AgdaBound{𝑆}\AgdaSpace{}%
\AgdaOperator{\AgdaFunction{⟩}}\AgdaSpace{}%
\AgdaBound{ξ}\AgdaSymbol{))}%
\>[60]\AgdaSymbol{=}\AgdaSpace{}%
\AgdaInductiveConstructor{≡.refl}\AgdaSpace{}%
\AgdaOperator{\AgdaInductiveConstructor{,}}\AgdaSpace{}%
\AgdaSymbol{λ}\AgdaSpace{}%
\AgdaBound{i}\AgdaSpace{}%
\AgdaSymbol{→}\AgdaSpace{}%
\AgdaFunction{Setoid.refl}%
\>[91]\AgdaBound{ξ}\<%
\\
%
\>[1]\AgdaField{sym}%
\>[8]\AgdaSymbol{(}\AgdaField{isEquivalence}\AgdaSpace{}%
\AgdaSymbol{(}\AgdaOperator{\AgdaFunction{⟨}}\AgdaSpace{}%
\AgdaBound{𝑆}\AgdaSpace{}%
\AgdaOperator{\AgdaFunction{⟩}}\AgdaSpace{}%
\AgdaBound{ξ}\AgdaSymbol{))}\AgdaSpace{}%
\AgdaSymbol{(}\AgdaInductiveConstructor{≡.refl}\AgdaSpace{}%
\AgdaOperator{\AgdaInductiveConstructor{,}}\AgdaSpace{}%
\AgdaBound{g}\AgdaSymbol{)}%
\>[60]\AgdaSymbol{=}\AgdaSpace{}%
\AgdaInductiveConstructor{≡.refl}\AgdaSpace{}%
\AgdaOperator{\AgdaInductiveConstructor{,}}\AgdaSpace{}%
\AgdaSymbol{λ}\AgdaSpace{}%
\AgdaBound{i}\AgdaSpace{}%
\AgdaSymbol{→}\AgdaSpace{}%
\AgdaFunction{Setoid.sym}%
\>[91]\AgdaBound{ξ}\AgdaSpace{}%
\AgdaSymbol{(}\AgdaBound{g}\AgdaSpace{}%
\AgdaBound{i}\AgdaSymbol{)}\<%
\\
%
\>[1]\AgdaField{trans}%
\>[8]\AgdaSymbol{(}\AgdaField{isEquivalence}\AgdaSpace{}%
\AgdaSymbol{(}\AgdaOperator{\AgdaFunction{⟨}}\AgdaSpace{}%
\AgdaBound{𝑆}\AgdaSpace{}%
\AgdaOperator{\AgdaFunction{⟩}}\AgdaSpace{}%
\AgdaBound{ξ}\AgdaSymbol{))}\AgdaSpace{}%
\AgdaSymbol{(}\AgdaInductiveConstructor{≡.refl}\AgdaSpace{}%
\AgdaOperator{\AgdaInductiveConstructor{,}}\AgdaSpace{}%
\AgdaBound{g}\AgdaSymbol{)(}\AgdaInductiveConstructor{≡.refl}\AgdaSpace{}%
\AgdaOperator{\AgdaInductiveConstructor{,}}\AgdaSpace{}%
\AgdaBound{h}\AgdaSymbol{)}%
\>[60]\AgdaSymbol{=}\AgdaSpace{}%
\AgdaInductiveConstructor{≡.refl}\AgdaSpace{}%
\AgdaOperator{\AgdaInductiveConstructor{,}}\AgdaSpace{}%
\AgdaSymbol{λ}\AgdaSpace{}%
\AgdaBound{i}\AgdaSpace{}%
\AgdaSymbol{→}\AgdaSpace{}%
\AgdaFunction{Setoid.trans}%
\>[91]\AgdaBound{ξ}\AgdaSpace{}%
\AgdaSymbol{(}\AgdaBound{g}\AgdaSpace{}%
\AgdaBound{i}\AgdaSymbol{)}\AgdaSpace{}%
\AgdaSymbol{(}\AgdaBound{h}\AgdaSpace{}%
\AgdaBound{i}\AgdaSymbol{)}\<%
\end{code}

\hypertarget{algebra-type}{%
\subsubsection{Algebra type}\label{algebra-type}}

Informally, an \emph{algebraic structure in the signature}
\texttt{𝑆\ =\ (F,\ ρ)} (or \texttt{𝑆}-\emph{algebra}) is typically
denoted by \texttt{𝑨\ =\ (A,\ Fᴬ)} and consists of

\begin{itemize}
\tightlist
\item
  \texttt{A} := a \emph{nonempty} set (or type), called the
  \emph{domain} (or \emph{carrier} or \emph{universe}) of the algebra;
\item
  \texttt{Fᴬ} :=
  \texttt{\{\ fᴬ\ ∣\ f\ ∈\ F,\ fᴬ\ :\ (ρf\ →\ A)\ →\ A\ \}}, a
  collection of \emph{operations} on \texttt{𝐴};
\item
  a (potentially empty) collection of \emph{identities} satisfied by
  elements and operations of \texttt{𝐴}.
\end{itemize}

We represent an algebra in Agda using a record type with two fields:

\begin{itemize}
\tightlist
\item
  \texttt{Domain} is a setoid denoting the underlying \emph{universe} of
  the algebra (informally, the set of elements of the algebra);
\item
  \texttt{Interp} represents the \emph{interpretation} in the algebra of
  each operation symbol of the given signature. The record type
  \texttt{Func} from the Agda Standard Library provides what we need for
  an operation on the domain setoid.
\end{itemize}

Let us present the definition of the \texttt{Algebra} type and then
discuss the definition of the \texttt{Func} type that provides the
interpretation of each operation symbol.

\begin{code}%
\>[0]\<%
\\
\>[0]\AgdaKeyword{record}\AgdaSpace{}%
\AgdaRecord{Algebra}\AgdaSpace{}%
\AgdaBound{α}\AgdaSpace{}%
\AgdaBound{ρ}\AgdaSpace{}%
\AgdaSymbol{:}\AgdaSpace{}%
\AgdaPrimitive{Type}\AgdaSpace{}%
\AgdaSymbol{(}\AgdaBound{𝓞}\AgdaSpace{}%
\AgdaOperator{\AgdaPrimitive{⊔}}\AgdaSpace{}%
\AgdaBound{𝓥}\AgdaSpace{}%
\AgdaOperator{\AgdaPrimitive{⊔}}\AgdaSpace{}%
\AgdaPrimitive{lsuc}\AgdaSpace{}%
\AgdaSymbol{(}\AgdaBound{α}\AgdaSpace{}%
\AgdaOperator{\AgdaPrimitive{⊔}}\AgdaSpace{}%
\AgdaBound{ρ}\AgdaSymbol{))}\AgdaSpace{}%
\AgdaKeyword{where}\<%
\\
\>[0][@{}l@{\AgdaIndent{0}}]%
\>[1]\AgdaKeyword{field}%
\>[8]\AgdaField{Domain}\AgdaSpace{}%
\AgdaSymbol{:}\AgdaSpace{}%
\AgdaRecord{Setoid}\AgdaSpace{}%
\AgdaBound{α}\AgdaSpace{}%
\AgdaBound{ρ}\<%
\\
%
\>[8]\AgdaField{Interp}\AgdaSpace{}%
\AgdaSymbol{:}\AgdaSpace{}%
\AgdaSymbol{(}\AgdaOperator{\AgdaFunction{⟨}}\AgdaSpace{}%
\AgdaBound{𝑆}\AgdaSpace{}%
\AgdaOperator{\AgdaFunction{⟩}}\AgdaSpace{}%
\AgdaField{Domain}\AgdaSymbol{)}\AgdaSpace{}%
\AgdaOperator{\AgdaRecord{⟶}}\AgdaSpace{}%
\AgdaField{Domain}\<%
\\
\>[0]\<%
\end{code}

The \texttt{Interp} field actually has type
\texttt{Func\ (⟨\ 𝑆\ ⟩\ Domain)\ Domain} (recall we renamed
\texttt{Func} as the infix long-arrow symbol). The \texttt{Func} type is
from the standard library and is defined as a record type with two
fields. In the present instance, the fields are

\begin{enumerate}
\def\labelenumi{\arabic{enumi}.}
\tightlist
\item
  a function
  \texttt{f\ :\ Carrier\ (⟨\ 𝑆\ ⟩\ Domain)\ →\ Carrier\ Domain}
\item
  a proof \texttt{cong\ :\ f\ Preserves\ \_≈₁\_\ ⟶\ \_≈₂\_} that
  \texttt{f} preserves the underlying setoid equalities.
\end{enumerate}

Thus \texttt{Interp} gives, for each operation symbol in the signature
\texttt{𝑆}, a setoid function \texttt{f}---namely, a function where the
domain is a power of \texttt{Domain} and the codomain is
\texttt{Domain}---along with a proof that all operations so interpreted
respect the underlying setoid equality on \texttt{Domain}.

Next we define some syntactic sugar that will make our Agda code easier
to read and comprehend. If \texttt{𝑨} is an algebra, then

\begin{itemize}
\tightlist
\item
  \texttt{f\ ̂\ 𝑨} will denote the interpretation in the algebra
  \texttt{𝑨} of the operation symbol \texttt{f},
\item
  \texttt{𝔻{[}\ 𝑨\ {]}} will denote the setoid \texttt{Domain\ 𝑨}, and
\item
  \texttt{𝕌{[}\ 𝑨\ {]}} will be the underlying carrier or ``universe''
  of the algebra \texttt{𝑨}.
\end{itemize}

\begin{code}%
\>[0]\<%
\\
\>[0]\AgdaKeyword{open}\AgdaSpace{}%
\AgdaModule{Algebra}\<%
\\
%
\\[\AgdaEmptyExtraSkip]%
\>[0]\AgdaOperator{\AgdaFunction{𝕌[\AgdaUnderscore{}]}}\AgdaSpace{}%
\AgdaSymbol{:}\AgdaSpace{}%
\AgdaRecord{Algebra}\AgdaSpace{}%
\AgdaGeneralizable{α}\AgdaSpace{}%
\AgdaGeneralizable{ρᵃ}\AgdaSpace{}%
\AgdaSymbol{→}%
\>[23]\AgdaPrimitive{Type}\AgdaSpace{}%
\AgdaGeneralizable{α}\<%
\\
\>[0]\AgdaOperator{\AgdaFunction{𝕌[}}\AgdaSpace{}%
\AgdaBound{𝑨}\AgdaSpace{}%
\AgdaOperator{\AgdaFunction{]}}\AgdaSpace{}%
\AgdaSymbol{=}\AgdaSpace{}%
\AgdaField{Carrier}\AgdaSpace{}%
\AgdaSymbol{(}\AgdaField{Domain}\AgdaSpace{}%
\AgdaBound{𝑨}\AgdaSymbol{)}\<%
\\
%
\\[\AgdaEmptyExtraSkip]%
\>[0]\AgdaOperator{\AgdaFunction{𝔻[\AgdaUnderscore{}]}}\AgdaSpace{}%
\AgdaSymbol{:}\AgdaSpace{}%
\AgdaRecord{Algebra}\AgdaSpace{}%
\AgdaGeneralizable{α}\AgdaSpace{}%
\AgdaGeneralizable{ρᵃ}\AgdaSpace{}%
\AgdaSymbol{→}%
\>[23]\AgdaRecord{Setoid}\AgdaSpace{}%
\AgdaGeneralizable{α}\AgdaSpace{}%
\AgdaGeneralizable{ρᵃ}\<%
\\
\>[0]\AgdaOperator{\AgdaFunction{𝔻[}}\AgdaSpace{}%
\AgdaBound{𝑨}\AgdaSpace{}%
\AgdaOperator{\AgdaFunction{]}}\AgdaSpace{}%
\AgdaSymbol{=}\AgdaSpace{}%
\AgdaField{Domain}\AgdaSpace{}%
\AgdaBound{𝑨}\<%
\\
%
\\[\AgdaEmptyExtraSkip]%
\>[0]\AgdaOperator{\AgdaFunction{\AgdaUnderscore{}̂\AgdaUnderscore{}}}\AgdaSpace{}%
\AgdaSymbol{:}\AgdaSpace{}%
\AgdaSymbol{(}\AgdaBound{f}\AgdaSpace{}%
\AgdaSymbol{:}\AgdaSpace{}%
\AgdaOperator{\AgdaFunction{∣}}\AgdaSpace{}%
\AgdaBound{𝑆}\AgdaSpace{}%
\AgdaOperator{\AgdaFunction{∣}}\AgdaSymbol{)(}\AgdaBound{𝑨}\AgdaSpace{}%
\AgdaSymbol{:}\AgdaSpace{}%
\AgdaRecord{Algebra}\AgdaSpace{}%
\AgdaGeneralizable{α}\AgdaSpace{}%
\AgdaGeneralizable{ρᵃ}\AgdaSymbol{)}\AgdaSpace{}%
\AgdaSymbol{→}\AgdaSpace{}%
\AgdaSymbol{(}\AgdaOperator{\AgdaFunction{∥}}\AgdaSpace{}%
\AgdaBound{𝑆}\AgdaSpace{}%
\AgdaOperator{\AgdaFunction{∥}}\AgdaSpace{}%
\AgdaBound{f}%
\>[48]\AgdaSymbol{→}%
\>[51]\AgdaOperator{\AgdaFunction{𝕌[}}\AgdaSpace{}%
\AgdaBound{𝑨}\AgdaSpace{}%
\AgdaOperator{\AgdaFunction{]}}\AgdaSymbol{)}\AgdaSpace{}%
\AgdaSymbol{→}\AgdaSpace{}%
\AgdaOperator{\AgdaFunction{𝕌[}}\AgdaSpace{}%
\AgdaBound{𝑨}\AgdaSpace{}%
\AgdaOperator{\AgdaFunction{]}}\<%
\\
%
\\[\AgdaEmptyExtraSkip]%
\>[0]\AgdaBound{f}\AgdaSpace{}%
\AgdaOperator{\AgdaFunction{̂}}\AgdaSpace{}%
\AgdaBound{𝑨}\AgdaSpace{}%
\AgdaSymbol{=}\AgdaSpace{}%
\AgdaSymbol{λ}\AgdaSpace{}%
\AgdaBound{a}\AgdaSpace{}%
\AgdaSymbol{→}\AgdaSpace{}%
\AgdaSymbol{(}\AgdaField{Interp}\AgdaSpace{}%
\AgdaBound{𝑨}\AgdaSymbol{)}\AgdaSpace{}%
\AgdaOperator{\AgdaField{⟨\$⟩}}\AgdaSpace{}%
\AgdaSymbol{(}\AgdaBound{f}\AgdaSpace{}%
\AgdaOperator{\AgdaInductiveConstructor{,}}\AgdaSpace{}%
\AgdaBound{a}\AgdaSymbol{)}\<%
\end{code}

\hypertarget{universe-lifting-of-algebra-types}{%
\subsubsection{Universe lifting of algebra
types}\label{universe-lifting-of-algebra-types}}

A technical aspect of dealing with the noncumulativity of the hierarchy
of type levels in Agda\ldots{}

\begin{code}%
\>[0]\<%
\\
\>[0]\AgdaKeyword{module}\AgdaSpace{}%
\AgdaModule{\AgdaUnderscore{}}\AgdaSpace{}%
\AgdaSymbol{(}\AgdaBound{𝑨}\AgdaSpace{}%
\AgdaSymbol{:}\AgdaSpace{}%
\AgdaRecord{Algebra}\AgdaSpace{}%
\AgdaGeneralizable{α}\AgdaSpace{}%
\AgdaGeneralizable{ρᵃ}\AgdaSymbol{)}\AgdaSpace{}%
\AgdaKeyword{where}\<%
\\
\>[0][@{}l@{\AgdaIndent{0}}]%
\>[1]\AgdaKeyword{open}\AgdaSpace{}%
\AgdaModule{Setoid}\AgdaSpace{}%
\AgdaOperator{\AgdaFunction{𝔻[}}\AgdaSpace{}%
\AgdaBound{𝑨}\AgdaSpace{}%
\AgdaOperator{\AgdaFunction{]}}\AgdaSpace{}%
\AgdaKeyword{using}\AgdaSpace{}%
\AgdaSymbol{(}\AgdaSpace{}%
\AgdaOperator{\AgdaField{\AgdaUnderscore{}≈\AgdaUnderscore{}}}\AgdaSpace{}%
\AgdaSymbol{;}\AgdaSpace{}%
\AgdaFunction{refl}\AgdaSpace{}%
\AgdaSymbol{;}\AgdaSpace{}%
\AgdaFunction{sym}\AgdaSpace{}%
\AgdaSymbol{;}\AgdaSpace{}%
\AgdaFunction{trans}\AgdaSpace{}%
\AgdaSymbol{)}\<%
\\
%
\>[1]\AgdaKeyword{open}\AgdaSpace{}%
\AgdaModule{Level}\<%
\\
%
\\[\AgdaEmptyExtraSkip]%
%
\>[1]\AgdaFunction{Lift-Algˡ}\AgdaSpace{}%
\AgdaSymbol{:}\AgdaSpace{}%
\AgdaSymbol{(}\AgdaBound{ℓ}\AgdaSpace{}%
\AgdaSymbol{:}\AgdaSpace{}%
\AgdaPostulate{Level}\AgdaSymbol{)}\AgdaSpace{}%
\AgdaSymbol{→}\AgdaSpace{}%
\AgdaRecord{Algebra}\AgdaSpace{}%
\AgdaSymbol{(}\AgdaBound{α}\AgdaSpace{}%
\AgdaOperator{\AgdaPrimitive{⊔}}\AgdaSpace{}%
\AgdaBound{ℓ}\AgdaSymbol{)}\AgdaSpace{}%
\AgdaBound{ρᵃ}\<%
\\
%
\>[1]\AgdaField{Domain}\AgdaSpace{}%
\AgdaSymbol{(}\AgdaFunction{Lift-Algˡ}\AgdaSpace{}%
\AgdaBound{ℓ}\AgdaSymbol{)}\AgdaSpace{}%
\AgdaSymbol{=}\<%
\\
\>[1][@{}l@{\AgdaIndent{0}}]%
\>[2]\AgdaKeyword{record}%
\>[11]\AgdaSymbol{\{}\AgdaSpace{}%
\AgdaField{Carrier}\AgdaSpace{}%
\AgdaSymbol{=}\AgdaSpace{}%
\AgdaRecord{Lift}\AgdaSpace{}%
\AgdaBound{ℓ}\AgdaSpace{}%
\AgdaOperator{\AgdaFunction{𝕌[}}\AgdaSpace{}%
\AgdaBound{𝑨}\AgdaSpace{}%
\AgdaOperator{\AgdaFunction{]}}\<%
\\
%
\>[11]\AgdaSymbol{;}\AgdaSpace{}%
\AgdaOperator{\AgdaField{\AgdaUnderscore{}≈\AgdaUnderscore{}}}\AgdaSpace{}%
\AgdaSymbol{=}\AgdaSpace{}%
\AgdaSymbol{λ}\AgdaSpace{}%
\AgdaBound{x}\AgdaSpace{}%
\AgdaBound{y}\AgdaSpace{}%
\AgdaSymbol{→}\AgdaSpace{}%
\AgdaField{lower}\AgdaSpace{}%
\AgdaBound{x}\AgdaSpace{}%
\AgdaOperator{\AgdaFunction{≈}}\AgdaSpace{}%
\AgdaField{lower}\AgdaSpace{}%
\AgdaBound{y}\<%
\\
%
\>[11]\AgdaSymbol{;}\AgdaSpace{}%
\AgdaField{isEquivalence}\AgdaSpace{}%
\AgdaSymbol{=}\AgdaSpace{}%
\AgdaKeyword{record}\AgdaSpace{}%
\AgdaSymbol{\{}\AgdaSpace{}%
\AgdaField{refl}\AgdaSpace{}%
\AgdaSymbol{=}\AgdaSpace{}%
\AgdaFunction{refl}\AgdaSpace{}%
\AgdaSymbol{;}\AgdaSpace{}%
\AgdaField{sym}\AgdaSpace{}%
\AgdaSymbol{=}\AgdaSpace{}%
\AgdaFunction{sym}\AgdaSpace{}%
\AgdaSymbol{;}\AgdaSpace{}%
\AgdaField{trans}\AgdaSpace{}%
\AgdaSymbol{=}\AgdaSpace{}%
\AgdaFunction{trans}\AgdaSpace{}%
\AgdaSymbol{\}\}}\<%
\\
%
\>[1]\AgdaField{Interp}\AgdaSpace{}%
\AgdaSymbol{(}\AgdaFunction{Lift-Algˡ}\AgdaSpace{}%
\AgdaBound{ℓ}\AgdaSymbol{)}\AgdaSpace{}%
\AgdaOperator{\AgdaField{⟨\$⟩}}\AgdaSpace{}%
\AgdaSymbol{(}\AgdaBound{f}\AgdaSpace{}%
\AgdaOperator{\AgdaInductiveConstructor{,}}\AgdaSpace{}%
\AgdaBound{la}\AgdaSymbol{)}\AgdaSpace{}%
\AgdaSymbol{=}\AgdaSpace{}%
\AgdaInductiveConstructor{lift}\AgdaSpace{}%
\AgdaSymbol{((}\AgdaBound{f}\AgdaSpace{}%
\AgdaOperator{\AgdaFunction{̂}}\AgdaSpace{}%
\AgdaBound{𝑨}\AgdaSymbol{)}\AgdaSpace{}%
\AgdaSymbol{(}\AgdaField{lower}\AgdaSpace{}%
\AgdaOperator{\AgdaFunction{∘}}\AgdaSpace{}%
\AgdaBound{la}\AgdaSymbol{))}\<%
\\
%
\>[1]\AgdaField{cong}\AgdaSpace{}%
\AgdaSymbol{(}\AgdaField{Interp}\AgdaSpace{}%
\AgdaSymbol{(}\AgdaFunction{Lift-Algˡ}\AgdaSpace{}%
\AgdaBound{ℓ}\AgdaSymbol{))}\AgdaSpace{}%
\AgdaSymbol{(}\AgdaInductiveConstructor{≡.refl}\AgdaSpace{}%
\AgdaOperator{\AgdaInductiveConstructor{,}}\AgdaSpace{}%
\AgdaBound{la=lb}\AgdaSymbol{)}\AgdaSpace{}%
\AgdaSymbol{=}\AgdaSpace{}%
\AgdaField{cong}\AgdaSpace{}%
\AgdaSymbol{(}\AgdaField{Interp}\AgdaSpace{}%
\AgdaBound{𝑨}\AgdaSymbol{)}\AgdaSpace{}%
\AgdaSymbol{((}\AgdaInductiveConstructor{≡.refl}\AgdaSpace{}%
\AgdaOperator{\AgdaInductiveConstructor{,}}\AgdaSpace{}%
\AgdaBound{la=lb}\AgdaSymbol{))}\<%
\\
%
\\[\AgdaEmptyExtraSkip]%
%
\>[1]\AgdaFunction{Lift-Algʳ}\AgdaSpace{}%
\AgdaSymbol{:}\AgdaSpace{}%
\AgdaSymbol{(}\AgdaBound{ℓ}\AgdaSpace{}%
\AgdaSymbol{:}\AgdaSpace{}%
\AgdaPostulate{Level}\AgdaSymbol{)}\AgdaSpace{}%
\AgdaSymbol{→}\AgdaSpace{}%
\AgdaRecord{Algebra}\AgdaSpace{}%
\AgdaBound{α}\AgdaSpace{}%
\AgdaSymbol{(}\AgdaBound{ρᵃ}\AgdaSpace{}%
\AgdaOperator{\AgdaPrimitive{⊔}}\AgdaSpace{}%
\AgdaBound{ℓ}\AgdaSymbol{)}\<%
\\
%
\>[1]\AgdaField{Domain}\AgdaSpace{}%
\AgdaSymbol{(}\AgdaFunction{Lift-Algʳ}\AgdaSpace{}%
\AgdaBound{ℓ}\AgdaSymbol{)}\AgdaSpace{}%
\AgdaSymbol{=}\<%
\\
\>[1][@{}l@{\AgdaIndent{0}}]%
\>[2]\AgdaKeyword{record}%
\>[10]\AgdaSymbol{\{}\AgdaSpace{}%
\AgdaField{Carrier}\AgdaSpace{}%
\AgdaSymbol{=}\AgdaSpace{}%
\AgdaOperator{\AgdaFunction{𝕌[}}\AgdaSpace{}%
\AgdaBound{𝑨}\AgdaSpace{}%
\AgdaOperator{\AgdaFunction{]}}\<%
\\
%
\>[10]\AgdaSymbol{;}\AgdaSpace{}%
\AgdaOperator{\AgdaField{\AgdaUnderscore{}≈\AgdaUnderscore{}}}\AgdaSpace{}%
\AgdaSymbol{=}\AgdaSpace{}%
\AgdaSymbol{λ}\AgdaSpace{}%
\AgdaBound{x}\AgdaSpace{}%
\AgdaBound{y}\AgdaSpace{}%
\AgdaSymbol{→}\AgdaSpace{}%
\AgdaRecord{Lift}\AgdaSpace{}%
\AgdaBound{ℓ}\AgdaSpace{}%
\AgdaSymbol{(}\AgdaBound{x}\AgdaSpace{}%
\AgdaOperator{\AgdaFunction{≈}}\AgdaSpace{}%
\AgdaBound{y}\AgdaSymbol{)}\<%
\\
%
\>[10]\AgdaSymbol{;}\AgdaSpace{}%
\AgdaField{isEquivalence}\AgdaSpace{}%
\AgdaSymbol{=}\AgdaSpace{}%
\AgdaKeyword{record}%
\>[36]\AgdaSymbol{\{}\AgdaSpace{}%
\AgdaField{refl}%
\>[44]\AgdaSymbol{=}\AgdaSpace{}%
\AgdaInductiveConstructor{lift}\AgdaSpace{}%
\AgdaFunction{refl}\<%
\\
%
\>[36]\AgdaSymbol{;}\AgdaSpace{}%
\AgdaField{sym}%
\>[44]\AgdaSymbol{=}\AgdaSpace{}%
\AgdaSymbol{λ}\AgdaSpace{}%
\AgdaBound{x}\AgdaSpace{}%
\AgdaSymbol{→}\AgdaSpace{}%
\AgdaInductiveConstructor{lift}\AgdaSpace{}%
\AgdaSymbol{(}\AgdaFunction{sym}\AgdaSpace{}%
\AgdaSymbol{(}\AgdaField{lower}\AgdaSpace{}%
\AgdaBound{x}\AgdaSymbol{))}\<%
\\
%
\>[36]\AgdaSymbol{;}\AgdaSpace{}%
\AgdaField{trans}\AgdaSpace{}%
\AgdaSymbol{=}\AgdaSpace{}%
\AgdaSymbol{λ}\AgdaSpace{}%
\AgdaBound{x}\AgdaSpace{}%
\AgdaBound{y}\AgdaSpace{}%
\AgdaSymbol{→}\AgdaSpace{}%
\AgdaInductiveConstructor{lift}\AgdaSpace{}%
\AgdaSymbol{(}\AgdaFunction{trans}\AgdaSpace{}%
\AgdaSymbol{(}\AgdaField{lower}\AgdaSpace{}%
\AgdaBound{x}\AgdaSymbol{)}\AgdaSpace{}%
\AgdaSymbol{(}\AgdaField{lower}\AgdaSpace{}%
\AgdaBound{y}\AgdaSymbol{))}\AgdaSpace{}%
\AgdaSymbol{\}\}}\<%
\\
%
\>[1]\AgdaField{Interp}\AgdaSpace{}%
\AgdaSymbol{(}\AgdaFunction{Lift-Algʳ}\AgdaSpace{}%
\AgdaBound{ℓ}\AgdaSpace{}%
\AgdaSymbol{)}\AgdaSpace{}%
\AgdaOperator{\AgdaField{⟨\$⟩}}\AgdaSpace{}%
\AgdaSymbol{(}\AgdaBound{f}\AgdaSpace{}%
\AgdaOperator{\AgdaInductiveConstructor{,}}\AgdaSpace{}%
\AgdaBound{la}\AgdaSymbol{)}\AgdaSpace{}%
\AgdaSymbol{=}\AgdaSpace{}%
\AgdaSymbol{(}\AgdaBound{f}\AgdaSpace{}%
\AgdaOperator{\AgdaFunction{̂}}\AgdaSpace{}%
\AgdaBound{𝑨}\AgdaSymbol{)}\AgdaSpace{}%
\AgdaBound{la}\<%
\\
%
\>[1]\AgdaField{cong}\AgdaSpace{}%
\AgdaSymbol{(}\AgdaField{Interp}\AgdaSpace{}%
\AgdaSymbol{(}\AgdaFunction{Lift-Algʳ}\AgdaSpace{}%
\AgdaBound{ℓ}\AgdaSymbol{))(}\AgdaInductiveConstructor{≡.refl}\AgdaSpace{}%
\AgdaOperator{\AgdaInductiveConstructor{,}}\AgdaSpace{}%
\AgdaBound{la≡lb}\AgdaSymbol{)}\AgdaSpace{}%
\AgdaSymbol{=}\AgdaSpace{}%
\AgdaInductiveConstructor{lift}\AgdaSymbol{(}\AgdaField{cong}\AgdaSymbol{(}\AgdaField{Interp}\AgdaSpace{}%
\AgdaBound{𝑨}\AgdaSymbol{)(}\AgdaInductiveConstructor{≡.refl}\AgdaSpace{}%
\AgdaOperator{\AgdaInductiveConstructor{,}}\AgdaSpace{}%
\AgdaSymbol{λ}\AgdaSpace{}%
\AgdaBound{i}\AgdaSpace{}%
\AgdaSymbol{→}\AgdaSpace{}%
\AgdaField{lower}\AgdaSpace{}%
\AgdaSymbol{(}\AgdaBound{la≡lb}\AgdaSpace{}%
\AgdaBound{i}\AgdaSymbol{)))}\<%
\\
%
\\[\AgdaEmptyExtraSkip]%
\>[0]\AgdaFunction{Lift-Alg}\AgdaSpace{}%
\AgdaSymbol{:}\AgdaSpace{}%
\AgdaSymbol{(}\AgdaBound{𝑨}\AgdaSpace{}%
\AgdaSymbol{:}\AgdaSpace{}%
\AgdaRecord{Algebra}\AgdaSpace{}%
\AgdaGeneralizable{α}\AgdaSpace{}%
\AgdaGeneralizable{ρᵃ}\AgdaSymbol{)(}\AgdaBound{ℓ₀}\AgdaSpace{}%
\AgdaBound{ℓ₁}\AgdaSpace{}%
\AgdaSymbol{:}\AgdaSpace{}%
\AgdaPostulate{Level}\AgdaSymbol{)}\AgdaSpace{}%
\AgdaSymbol{→}\AgdaSpace{}%
\AgdaRecord{Algebra}\AgdaSpace{}%
\AgdaSymbol{(}\AgdaGeneralizable{α}\AgdaSpace{}%
\AgdaOperator{\AgdaPrimitive{⊔}}\AgdaSpace{}%
\AgdaBound{ℓ₀}\AgdaSymbol{)}\AgdaSpace{}%
\AgdaSymbol{(}\AgdaGeneralizable{ρᵃ}\AgdaSpace{}%
\AgdaOperator{\AgdaPrimitive{⊔}}\AgdaSpace{}%
\AgdaBound{ℓ₁}\AgdaSymbol{)}\<%
\\
\>[0]\AgdaFunction{Lift-Alg}\AgdaSpace{}%
\AgdaBound{𝑨}\AgdaSpace{}%
\AgdaBound{ℓ₀}\AgdaSpace{}%
\AgdaBound{ℓ₁}\AgdaSpace{}%
\AgdaSymbol{=}\AgdaSpace{}%
\AgdaFunction{Lift-Algʳ}\AgdaSpace{}%
\AgdaSymbol{(}\AgdaFunction{Lift-Algˡ}\AgdaSpace{}%
\AgdaBound{𝑨}\AgdaSpace{}%
\AgdaBound{ℓ₀}\AgdaSymbol{)}\AgdaSpace{}%
\AgdaBound{ℓ₁}\<%
\end{code}

\hypertarget{product-algebras}{%
\subsubsection{Product Algebras}\label{product-algebras}}

(cf.~the
\href{https://ualib.github.io/agda-algebras/Algebras.Func.Products.html}{Algebras.Func.Products}
module of the \href{https://ualib.github.io/agda-algebras}{Agda
Universal Algebra Library}.)

\begin{code}%
\>[0]\<%
\\
\>[0]\AgdaKeyword{module}\AgdaSpace{}%
\AgdaModule{\AgdaUnderscore{}}\AgdaSpace{}%
\AgdaSymbol{\{}\AgdaBound{ι}\AgdaSpace{}%
\AgdaSymbol{:}\AgdaSpace{}%
\AgdaPostulate{Level}\AgdaSymbol{\}\{}\AgdaBound{I}\AgdaSpace{}%
\AgdaSymbol{:}\AgdaSpace{}%
\AgdaPrimitive{Type}\AgdaSpace{}%
\AgdaBound{ι}\AgdaSpace{}%
\AgdaSymbol{\}}\AgdaSpace{}%
\AgdaKeyword{where}\<%
\\
%
\\[\AgdaEmptyExtraSkip]%
\>[0][@{}l@{\AgdaIndent{0}}]%
\>[1]\AgdaFunction{⨅}\AgdaSpace{}%
\AgdaSymbol{:}\AgdaSpace{}%
\AgdaSymbol{(}\AgdaBound{𝒜}\AgdaSpace{}%
\AgdaSymbol{:}\AgdaSpace{}%
\AgdaBound{I}\AgdaSpace{}%
\AgdaSymbol{→}\AgdaSpace{}%
\AgdaRecord{Algebra}\AgdaSpace{}%
\AgdaGeneralizable{α}\AgdaSpace{}%
\AgdaGeneralizable{ρᵃ}\AgdaSymbol{)}\AgdaSpace{}%
\AgdaSymbol{→}\AgdaSpace{}%
\AgdaRecord{Algebra}\AgdaSpace{}%
\AgdaSymbol{(}\AgdaGeneralizable{α}\AgdaSpace{}%
\AgdaOperator{\AgdaPrimitive{⊔}}\AgdaSpace{}%
\AgdaBound{ι}\AgdaSymbol{)}\AgdaSpace{}%
\AgdaSymbol{(}\AgdaGeneralizable{ρᵃ}\AgdaSpace{}%
\AgdaOperator{\AgdaPrimitive{⊔}}\AgdaSpace{}%
\AgdaBound{ι}\AgdaSymbol{)}\<%
\\
%
\>[1]\AgdaField{Domain}\AgdaSpace{}%
\AgdaSymbol{(}\AgdaFunction{⨅}\AgdaSpace{}%
\AgdaBound{𝒜}\AgdaSymbol{)}\AgdaSpace{}%
\AgdaSymbol{=}\<%
\\
\>[1][@{}l@{\AgdaIndent{0}}]%
\>[2]\AgdaKeyword{record}%
\>[10]\AgdaSymbol{\{}\AgdaSpace{}%
\AgdaField{Carrier}\AgdaSpace{}%
\AgdaSymbol{=}\AgdaSpace{}%
\AgdaSymbol{∀}\AgdaSpace{}%
\AgdaBound{i}\AgdaSpace{}%
\AgdaSymbol{→}\AgdaSpace{}%
\AgdaOperator{\AgdaFunction{𝕌[}}\AgdaSpace{}%
\AgdaBound{𝒜}\AgdaSpace{}%
\AgdaBound{i}\AgdaSpace{}%
\AgdaOperator{\AgdaFunction{]}}\<%
\\
%
\>[10]\AgdaSymbol{;}\AgdaSpace{}%
\AgdaOperator{\AgdaField{\AgdaUnderscore{}≈\AgdaUnderscore{}}}\AgdaSpace{}%
\AgdaSymbol{=}\AgdaSpace{}%
\AgdaSymbol{λ}\AgdaSpace{}%
\AgdaBound{a}\AgdaSpace{}%
\AgdaBound{b}\AgdaSpace{}%
\AgdaSymbol{→}\AgdaSpace{}%
\AgdaSymbol{∀}\AgdaSpace{}%
\AgdaBound{i}\AgdaSpace{}%
\AgdaSymbol{→}\AgdaSpace{}%
\AgdaSymbol{(}\AgdaOperator{\AgdaField{Setoid.\AgdaUnderscore{}≈\AgdaUnderscore{}}}\AgdaSpace{}%
\AgdaOperator{\AgdaFunction{𝔻[}}\AgdaSpace{}%
\AgdaBound{𝒜}\AgdaSpace{}%
\AgdaBound{i}\AgdaSpace{}%
\AgdaOperator{\AgdaFunction{]}}\AgdaSymbol{)}\AgdaSpace{}%
\AgdaSymbol{(}\AgdaBound{a}\AgdaSpace{}%
\AgdaBound{i}\AgdaSymbol{)(}\AgdaBound{b}\AgdaSpace{}%
\AgdaBound{i}\AgdaSymbol{)}\<%
\\
%
\>[10]\AgdaSymbol{;}%
\>[1165I]\AgdaField{isEquivalence}\AgdaSpace{}%
\AgdaSymbol{=}\<%
\\
\>[.][@{}l@{}]\<[1165I]%
\>[12]\AgdaKeyword{record}%
\>[20]\AgdaSymbol{\{}\AgdaSpace{}%
\AgdaField{refl}%
\>[29]\AgdaSymbol{=}\AgdaSpace{}%
\AgdaSymbol{λ}\AgdaSpace{}%
\AgdaBound{i}\AgdaSpace{}%
\AgdaSymbol{→}%
\>[42]\AgdaField{IsEquivalence.refl}%
\>[63]\AgdaSymbol{(}\AgdaField{isEquivalence}\AgdaSpace{}%
\AgdaOperator{\AgdaFunction{𝔻[}}\AgdaSpace{}%
\AgdaBound{𝒜}\AgdaSpace{}%
\AgdaBound{i}\AgdaSpace{}%
\AgdaOperator{\AgdaFunction{]}}\AgdaSymbol{)}\<%
\\
%
\>[20]\AgdaSymbol{;}\AgdaSpace{}%
\AgdaField{sym}%
\>[29]\AgdaSymbol{=}\AgdaSpace{}%
\AgdaSymbol{λ}\AgdaSpace{}%
\AgdaBound{x}\AgdaSpace{}%
\AgdaBound{i}\AgdaSpace{}%
\AgdaSymbol{→}%
\>[42]\AgdaField{IsEquivalence.sym}%
\>[63]\AgdaSymbol{(}\AgdaField{isEquivalence}\AgdaSpace{}%
\AgdaOperator{\AgdaFunction{𝔻[}}\AgdaSpace{}%
\AgdaBound{𝒜}\AgdaSpace{}%
\AgdaBound{i}\AgdaSpace{}%
\AgdaOperator{\AgdaFunction{]}}\AgdaSymbol{)(}\AgdaBound{x}\AgdaSpace{}%
\AgdaBound{i}\AgdaSymbol{)}\<%
\\
%
\>[20]\AgdaSymbol{;}\AgdaSpace{}%
\AgdaField{trans}%
\>[29]\AgdaSymbol{=}\AgdaSpace{}%
\AgdaSymbol{λ}\AgdaSpace{}%
\AgdaBound{x}\AgdaSpace{}%
\AgdaBound{y}\AgdaSpace{}%
\AgdaBound{i}\AgdaSpace{}%
\AgdaSymbol{→}%
\>[42]\AgdaField{IsEquivalence.trans}%
\>[63]\AgdaSymbol{(}\AgdaField{isEquivalence}\AgdaSpace{}%
\AgdaOperator{\AgdaFunction{𝔻[}}\AgdaSpace{}%
\AgdaBound{𝒜}\AgdaSpace{}%
\AgdaBound{i}\AgdaSpace{}%
\AgdaOperator{\AgdaFunction{]}}\AgdaSymbol{)(}\AgdaBound{x}\AgdaSpace{}%
\AgdaBound{i}\AgdaSymbol{)(}\AgdaBound{y}\AgdaSpace{}%
\AgdaBound{i}\AgdaSymbol{)}\AgdaSpace{}%
\AgdaSymbol{\}\}}\<%
\\
%
\>[1]\AgdaField{Interp}\AgdaSpace{}%
\AgdaSymbol{(}\AgdaFunction{⨅}\AgdaSpace{}%
\AgdaBound{𝒜}\AgdaSymbol{)}\AgdaSpace{}%
\AgdaOperator{\AgdaField{⟨\$⟩}}\AgdaSpace{}%
\AgdaSymbol{(}\AgdaBound{f}\AgdaSpace{}%
\AgdaOperator{\AgdaInductiveConstructor{,}}\AgdaSpace{}%
\AgdaBound{a}\AgdaSymbol{)}\AgdaSpace{}%
\AgdaSymbol{=}\AgdaSpace{}%
\AgdaSymbol{λ}\AgdaSpace{}%
\AgdaBound{i}\AgdaSpace{}%
\AgdaSymbol{→}\AgdaSpace{}%
\AgdaSymbol{(}\AgdaBound{f}\AgdaSpace{}%
\AgdaOperator{\AgdaFunction{̂}}\AgdaSpace{}%
\AgdaSymbol{(}\AgdaBound{𝒜}\AgdaSpace{}%
\AgdaBound{i}\AgdaSymbol{))}\AgdaSpace{}%
\AgdaSymbol{(}\AgdaFunction{flip}\AgdaSpace{}%
\AgdaBound{a}\AgdaSpace{}%
\AgdaBound{i}\AgdaSymbol{)}\<%
\\
%
\>[1]\AgdaField{cong}\AgdaSpace{}%
\AgdaSymbol{(}\AgdaField{Interp}\AgdaSpace{}%
\AgdaSymbol{(}\AgdaFunction{⨅}\AgdaSpace{}%
\AgdaBound{𝒜}\AgdaSymbol{))}\AgdaSpace{}%
\AgdaSymbol{(}\AgdaInductiveConstructor{≡.refl}\AgdaSpace{}%
\AgdaOperator{\AgdaInductiveConstructor{,}}\AgdaSpace{}%
\AgdaBound{f=g}\AgdaSpace{}%
\AgdaSymbol{)}\AgdaSpace{}%
\AgdaSymbol{=}\AgdaSpace{}%
\AgdaSymbol{λ}\AgdaSpace{}%
\AgdaBound{i}\AgdaSpace{}%
\AgdaSymbol{→}\AgdaSpace{}%
\AgdaField{cong}\AgdaSpace{}%
\AgdaSymbol{(}\AgdaField{Interp}\AgdaSpace{}%
\AgdaSymbol{(}\AgdaBound{𝒜}\AgdaSpace{}%
\AgdaBound{i}\AgdaSymbol{))}\AgdaSpace{}%
\AgdaSymbol{(}\AgdaInductiveConstructor{≡.refl}\AgdaSpace{}%
\AgdaOperator{\AgdaInductiveConstructor{,}}\AgdaSpace{}%
\AgdaFunction{flip}\AgdaSpace{}%
\AgdaBound{f=g}\AgdaSpace{}%
\AgdaBound{i}\AgdaSpace{}%
\AgdaSymbol{)}\<%
\end{code}

\hypertarget{homomorphisms}{%
\subsection{Homomorphisms}\label{homomorphisms}}

\hypertarget{basic-definitions}{%
\subsubsection{Basic definitions}\label{basic-definitions}}

Here are some useful definitions and theorems extracted from the
\href{https://ualib.github.io/agda-algebras/Homomorphisms.Func.Basic.html}{Homomorphisms.Func.Basic}
module of the \href{https://ualib.github.io/agda-algebras}{Agda
Universal Algebra Library}.

\begin{code}%
\>[0]\<%
\\
\>[0]\AgdaKeyword{module}\AgdaSpace{}%
\AgdaModule{\AgdaUnderscore{}}\AgdaSpace{}%
\AgdaSymbol{(}\AgdaBound{𝑨}\AgdaSpace{}%
\AgdaSymbol{:}\AgdaSpace{}%
\AgdaRecord{Algebra}\AgdaSpace{}%
\AgdaGeneralizable{α}\AgdaSpace{}%
\AgdaGeneralizable{ρᵃ}\AgdaSymbol{)(}\AgdaBound{𝑩}\AgdaSpace{}%
\AgdaSymbol{:}\AgdaSpace{}%
\AgdaRecord{Algebra}\AgdaSpace{}%
\AgdaGeneralizable{β}\AgdaSpace{}%
\AgdaGeneralizable{ρᵇ}\AgdaSymbol{)}\AgdaSpace{}%
\AgdaKeyword{where}\<%
\\
\>[0][@{}l@{\AgdaIndent{0}}]%
\>[1]\AgdaKeyword{private}\AgdaSpace{}%
\AgdaFunction{ov}\AgdaSpace{}%
\AgdaSymbol{=}\AgdaSpace{}%
\AgdaBound{𝓞}\AgdaSpace{}%
\AgdaOperator{\AgdaPrimitive{⊔}}\AgdaSpace{}%
\AgdaBound{𝓥}\AgdaSpace{}%
\AgdaSymbol{;}\AgdaSpace{}%
\AgdaFunction{a}\AgdaSpace{}%
\AgdaSymbol{=}\AgdaSpace{}%
\AgdaBound{α}\AgdaSpace{}%
\AgdaOperator{\AgdaPrimitive{⊔}}\AgdaSpace{}%
\AgdaBound{ρᵃ}\AgdaSpace{}%
\AgdaSymbol{;}\AgdaSpace{}%
\AgdaFunction{b}\AgdaSpace{}%
\AgdaSymbol{=}\AgdaSpace{}%
\AgdaBound{β}\AgdaSpace{}%
\AgdaOperator{\AgdaPrimitive{⊔}}\AgdaSpace{}%
\AgdaBound{ρᵇ}\AgdaSpace{}%
\AgdaSymbol{;}\AgdaSpace{}%
\AgdaFunction{c}\AgdaSpace{}%
\AgdaSymbol{=}\AgdaSpace{}%
\AgdaBound{𝓞}\AgdaSpace{}%
\AgdaOperator{\AgdaPrimitive{⊔}}\AgdaSpace{}%
\AgdaBound{𝓥}\AgdaSpace{}%
\AgdaOperator{\AgdaPrimitive{⊔}}\AgdaSpace{}%
\AgdaBound{α}\AgdaSpace{}%
\AgdaOperator{\AgdaPrimitive{⊔}}\AgdaSpace{}%
\AgdaBound{ρᵃ}\AgdaSpace{}%
\AgdaOperator{\AgdaPrimitive{⊔}}\AgdaSpace{}%
\AgdaBound{β}\AgdaSpace{}%
\AgdaOperator{\AgdaPrimitive{⊔}}\AgdaSpace{}%
\AgdaBound{ρᵇ}\<%
\\
%
\\[\AgdaEmptyExtraSkip]%
%
\>[1]\AgdaFunction{compatible-map-op}\AgdaSpace{}%
\AgdaSymbol{:}\AgdaSpace{}%
\AgdaSymbol{(}\AgdaOperator{\AgdaFunction{𝔻[}}\AgdaSpace{}%
\AgdaBound{𝑨}\AgdaSpace{}%
\AgdaOperator{\AgdaFunction{]}}\AgdaSpace{}%
\AgdaOperator{\AgdaRecord{⟶}}\AgdaSpace{}%
\AgdaOperator{\AgdaFunction{𝔻[}}\AgdaSpace{}%
\AgdaBound{𝑩}\AgdaSpace{}%
\AgdaOperator{\AgdaFunction{]}}\AgdaSymbol{)}\AgdaSpace{}%
\AgdaSymbol{→}\AgdaSpace{}%
\AgdaOperator{\AgdaFunction{∣}}\AgdaSpace{}%
\AgdaBound{𝑆}\AgdaSpace{}%
\AgdaOperator{\AgdaFunction{∣}}\AgdaSpace{}%
\AgdaSymbol{→}\AgdaSpace{}%
\AgdaPrimitive{Type}\AgdaSpace{}%
\AgdaSymbol{(}\AgdaBound{𝓥}\AgdaSpace{}%
\AgdaOperator{\AgdaPrimitive{⊔}}\AgdaSpace{}%
\AgdaBound{α}\AgdaSpace{}%
\AgdaOperator{\AgdaPrimitive{⊔}}\AgdaSpace{}%
\AgdaBound{ρᵇ}\AgdaSymbol{)}\<%
\\
%
\>[1]\AgdaFunction{compatible-map-op}\AgdaSpace{}%
\AgdaBound{h}\AgdaSpace{}%
\AgdaBound{f}\AgdaSpace{}%
\AgdaSymbol{=}\AgdaSpace{}%
\AgdaSymbol{∀}\AgdaSpace{}%
\AgdaSymbol{\{}\AgdaBound{a}\AgdaSymbol{\}}\AgdaSpace{}%
\AgdaSymbol{→}\AgdaSpace{}%
\AgdaSymbol{(}\AgdaBound{h}\AgdaSpace{}%
\AgdaOperator{\AgdaField{⟨\$⟩}}\AgdaSpace{}%
\AgdaSymbol{((}\AgdaBound{f}\AgdaSpace{}%
\AgdaOperator{\AgdaFunction{̂}}\AgdaSpace{}%
\AgdaBound{𝑨}\AgdaSymbol{)}\AgdaSpace{}%
\AgdaBound{a}\AgdaSymbol{))}\AgdaSpace{}%
\AgdaOperator{\AgdaFunction{≈₂}}\AgdaSpace{}%
\AgdaSymbol{((}\AgdaBound{f}\AgdaSpace{}%
\AgdaOperator{\AgdaFunction{̂}}\AgdaSpace{}%
\AgdaBound{𝑩}\AgdaSymbol{)}\AgdaSpace{}%
\AgdaSymbol{(λ}\AgdaSpace{}%
\AgdaBound{x}\AgdaSpace{}%
\AgdaSymbol{→}\AgdaSpace{}%
\AgdaSymbol{(}\AgdaBound{h}\AgdaSpace{}%
\AgdaOperator{\AgdaField{⟨\$⟩}}\AgdaSpace{}%
\AgdaSymbol{(}\AgdaBound{a}\AgdaSpace{}%
\AgdaBound{x}\AgdaSymbol{))))}\<%
\\
\>[1][@{}l@{\AgdaIndent{0}}]%
\>[2]\AgdaKeyword{where}%
\>[9]\AgdaKeyword{open}\AgdaSpace{}%
\AgdaModule{Setoid}\AgdaSpace{}%
\AgdaOperator{\AgdaFunction{𝔻[}}\AgdaSpace{}%
\AgdaBound{𝑩}\AgdaSpace{}%
\AgdaOperator{\AgdaFunction{]}}\AgdaSpace{}%
\AgdaKeyword{using}\AgdaSpace{}%
\AgdaSymbol{()}\AgdaSpace{}%
\AgdaKeyword{renaming}\AgdaSpace{}%
\AgdaSymbol{(}\AgdaSpace{}%
\AgdaOperator{\AgdaField{\AgdaUnderscore{}≈\AgdaUnderscore{}}}\AgdaSpace{}%
\AgdaSymbol{to}\AgdaSpace{}%
\AgdaOperator{\AgdaField{\AgdaUnderscore{}≈₂\AgdaUnderscore{}}}\AgdaSpace{}%
\AgdaSymbol{)}\<%
\\
%
\\[\AgdaEmptyExtraSkip]%
%
\>[1]\AgdaFunction{compatible-map}\AgdaSpace{}%
\AgdaSymbol{:}\AgdaSpace{}%
\AgdaSymbol{(}\AgdaOperator{\AgdaFunction{𝔻[}}\AgdaSpace{}%
\AgdaBound{𝑨}\AgdaSpace{}%
\AgdaOperator{\AgdaFunction{]}}\AgdaSpace{}%
\AgdaOperator{\AgdaRecord{⟶}}\AgdaSpace{}%
\AgdaOperator{\AgdaFunction{𝔻[}}\AgdaSpace{}%
\AgdaBound{𝑩}\AgdaSpace{}%
\AgdaOperator{\AgdaFunction{]}}\AgdaSymbol{)}\AgdaSpace{}%
\AgdaSymbol{→}\AgdaSpace{}%
\AgdaPrimitive{Type}\AgdaSpace{}%
\AgdaSymbol{(}\AgdaFunction{ov}\AgdaSpace{}%
\AgdaOperator{\AgdaPrimitive{⊔}}\AgdaSpace{}%
\AgdaBound{α}\AgdaSpace{}%
\AgdaOperator{\AgdaPrimitive{⊔}}\AgdaSpace{}%
\AgdaBound{ρᵇ}\AgdaSymbol{)}\<%
\\
%
\>[1]\AgdaFunction{compatible-map}\AgdaSpace{}%
\AgdaBound{h}\AgdaSpace{}%
\AgdaSymbol{=}\AgdaSpace{}%
\AgdaSymbol{∀}\AgdaSpace{}%
\AgdaSymbol{\{}\AgdaBound{f}\AgdaSymbol{\}}\AgdaSpace{}%
\AgdaSymbol{→}\AgdaSpace{}%
\AgdaFunction{compatible-map-op}\AgdaSpace{}%
\AgdaBound{h}\AgdaSpace{}%
\AgdaBound{f}\<%
\\
%
\\[\AgdaEmptyExtraSkip]%
%
\>[1]\AgdaComment{--\ The\ property\ of\ being\ a\ homomorphism.}\<%
\\
%
\>[1]\AgdaKeyword{record}\AgdaSpace{}%
\AgdaRecord{IsHom}\AgdaSpace{}%
\AgdaSymbol{(}\AgdaBound{h}\AgdaSpace{}%
\AgdaSymbol{:}\AgdaSpace{}%
\AgdaOperator{\AgdaFunction{𝔻[}}\AgdaSpace{}%
\AgdaBound{𝑨}\AgdaSpace{}%
\AgdaOperator{\AgdaFunction{]}}\AgdaSpace{}%
\AgdaOperator{\AgdaRecord{⟶}}\AgdaSpace{}%
\AgdaOperator{\AgdaFunction{𝔻[}}\AgdaSpace{}%
\AgdaBound{𝑩}\AgdaSpace{}%
\AgdaOperator{\AgdaFunction{]}}\AgdaSymbol{)}\AgdaSpace{}%
\AgdaSymbol{:}\AgdaSpace{}%
\AgdaPrimitive{Type}\AgdaSpace{}%
\AgdaSymbol{(}\AgdaFunction{ov}\AgdaSpace{}%
\AgdaOperator{\AgdaPrimitive{⊔}}\AgdaSpace{}%
\AgdaBound{α}\AgdaSpace{}%
\AgdaOperator{\AgdaPrimitive{⊔}}\AgdaSpace{}%
\AgdaBound{ρᵇ}\AgdaSymbol{)}\AgdaSpace{}%
\AgdaKeyword{where}\<%
\\
\>[1][@{}l@{\AgdaIndent{0}}]%
\>[2]\AgdaKeyword{constructor}\AgdaSpace{}%
\AgdaInductiveConstructor{mkhom}\<%
\\
%
\>[2]\AgdaKeyword{field}\AgdaSpace{}%
\AgdaField{compatible}\AgdaSpace{}%
\AgdaSymbol{:}\AgdaSpace{}%
\AgdaFunction{compatible-map}\AgdaSpace{}%
\AgdaBound{h}\<%
\\
%
\\[\AgdaEmptyExtraSkip]%
%
\>[1]\AgdaComment{--\ The\ type\ of\ homomorphisms.}\<%
\\
%
\>[1]\AgdaFunction{hom}\AgdaSpace{}%
\AgdaSymbol{:}\AgdaSpace{}%
\AgdaPrimitive{Type}\AgdaSpace{}%
\AgdaFunction{c}\<%
\\
%
\>[1]\AgdaFunction{hom}\AgdaSpace{}%
\AgdaSymbol{=}\AgdaSpace{}%
\AgdaRecord{Σ}\AgdaSpace{}%
\AgdaSymbol{(}\AgdaOperator{\AgdaFunction{𝔻[}}\AgdaSpace{}%
\AgdaBound{𝑨}\AgdaSpace{}%
\AgdaOperator{\AgdaFunction{]}}\AgdaSpace{}%
\AgdaOperator{\AgdaRecord{⟶}}\AgdaSpace{}%
\AgdaOperator{\AgdaFunction{𝔻[}}\AgdaSpace{}%
\AgdaBound{𝑩}\AgdaSpace{}%
\AgdaOperator{\AgdaFunction{]}}\AgdaSymbol{)}\AgdaSpace{}%
\AgdaRecord{IsHom}\<%
\end{code}

\hypertarget{monomorphisms-and-epimorphisms}{%
\subsubsection{Monomorphisms and
epimorphisms}\label{monomorphisms-and-epimorphisms}}

\begin{code}%
\>[0]\<%
\\
%
\>[1]\AgdaKeyword{record}\AgdaSpace{}%
\AgdaRecord{IsMon}\AgdaSpace{}%
\AgdaSymbol{(}\AgdaBound{h}\AgdaSpace{}%
\AgdaSymbol{:}\AgdaSpace{}%
\AgdaOperator{\AgdaFunction{𝔻[}}\AgdaSpace{}%
\AgdaBound{𝑨}\AgdaSpace{}%
\AgdaOperator{\AgdaFunction{]}}\AgdaSpace{}%
\AgdaOperator{\AgdaRecord{⟶}}\AgdaSpace{}%
\AgdaOperator{\AgdaFunction{𝔻[}}\AgdaSpace{}%
\AgdaBound{𝑩}\AgdaSpace{}%
\AgdaOperator{\AgdaFunction{]}}\AgdaSymbol{)}\AgdaSpace{}%
\AgdaSymbol{:}\AgdaSpace{}%
\AgdaPrimitive{Type}\AgdaSpace{}%
\AgdaSymbol{(}\AgdaFunction{ov}\AgdaSpace{}%
\AgdaOperator{\AgdaPrimitive{⊔}}\AgdaSpace{}%
\AgdaFunction{a}\AgdaSpace{}%
\AgdaOperator{\AgdaPrimitive{⊔}}\AgdaSpace{}%
\AgdaBound{ρᵇ}\AgdaSymbol{)}\AgdaSpace{}%
\AgdaKeyword{where}\<%
\\
\>[1][@{}l@{\AgdaIndent{0}}]%
\>[2]\AgdaKeyword{field}%
\>[9]\AgdaField{isHom}\AgdaSpace{}%
\AgdaSymbol{:}\AgdaSpace{}%
\AgdaRecord{IsHom}\AgdaSpace{}%
\AgdaBound{h}\<%
\\
%
\>[9]\AgdaField{isInjective}\AgdaSpace{}%
\AgdaSymbol{:}\AgdaSpace{}%
\AgdaFunction{IsInjective}\AgdaSpace{}%
\AgdaBound{h}\<%
\\
%
\\[\AgdaEmptyExtraSkip]%
%
\>[2]\AgdaFunction{HomReduct}\AgdaSpace{}%
\AgdaSymbol{:}\AgdaSpace{}%
\AgdaFunction{hom}\<%
\\
%
\>[2]\AgdaFunction{HomReduct}\AgdaSpace{}%
\AgdaSymbol{=}\AgdaSpace{}%
\AgdaBound{h}\AgdaSpace{}%
\AgdaOperator{\AgdaInductiveConstructor{,}}\AgdaSpace{}%
\AgdaField{isHom}\<%
\\
%
\\[\AgdaEmptyExtraSkip]%
%
\>[1]\AgdaFunction{mon}\AgdaSpace{}%
\AgdaSymbol{:}\AgdaSpace{}%
\AgdaPrimitive{Type}\AgdaSpace{}%
\AgdaFunction{c}\<%
\\
%
\>[1]\AgdaFunction{mon}\AgdaSpace{}%
\AgdaSymbol{=}\AgdaSpace{}%
\AgdaRecord{Σ}\AgdaSpace{}%
\AgdaSymbol{(}\AgdaOperator{\AgdaFunction{𝔻[}}\AgdaSpace{}%
\AgdaBound{𝑨}\AgdaSpace{}%
\AgdaOperator{\AgdaFunction{]}}\AgdaSpace{}%
\AgdaOperator{\AgdaRecord{⟶}}\AgdaSpace{}%
\AgdaOperator{\AgdaFunction{𝔻[}}\AgdaSpace{}%
\AgdaBound{𝑩}\AgdaSpace{}%
\AgdaOperator{\AgdaFunction{]}}\AgdaSymbol{)}\AgdaSpace{}%
\AgdaRecord{IsMon}\<%
\\
%
\\[\AgdaEmptyExtraSkip]%
%
\>[1]\AgdaKeyword{record}\AgdaSpace{}%
\AgdaRecord{IsEpi}\AgdaSpace{}%
\AgdaSymbol{(}\AgdaBound{h}\AgdaSpace{}%
\AgdaSymbol{:}\AgdaSpace{}%
\AgdaOperator{\AgdaFunction{𝔻[}}\AgdaSpace{}%
\AgdaBound{𝑨}\AgdaSpace{}%
\AgdaOperator{\AgdaFunction{]}}\AgdaSpace{}%
\AgdaOperator{\AgdaRecord{⟶}}\AgdaSpace{}%
\AgdaOperator{\AgdaFunction{𝔻[}}\AgdaSpace{}%
\AgdaBound{𝑩}\AgdaSpace{}%
\AgdaOperator{\AgdaFunction{]}}\AgdaSymbol{)}\AgdaSpace{}%
\AgdaSymbol{:}\AgdaSpace{}%
\AgdaPrimitive{Type}\AgdaSpace{}%
\AgdaSymbol{(}\AgdaFunction{ov}\AgdaSpace{}%
\AgdaOperator{\AgdaPrimitive{⊔}}\AgdaSpace{}%
\AgdaBound{α}\AgdaSpace{}%
\AgdaOperator{\AgdaPrimitive{⊔}}\AgdaSpace{}%
\AgdaFunction{b}\AgdaSymbol{)}\AgdaSpace{}%
\AgdaKeyword{where}\<%
\\
\>[1][@{}l@{\AgdaIndent{0}}]%
\>[2]\AgdaKeyword{field}%
\>[9]\AgdaField{isHom}\AgdaSpace{}%
\AgdaSymbol{:}\AgdaSpace{}%
\AgdaRecord{IsHom}\AgdaSpace{}%
\AgdaBound{h}\<%
\\
%
\>[9]\AgdaField{isSurjective}\AgdaSpace{}%
\AgdaSymbol{:}\AgdaSpace{}%
\AgdaFunction{IsSurjective}\AgdaSpace{}%
\AgdaBound{h}\<%
\\
%
\\[\AgdaEmptyExtraSkip]%
%
\>[2]\AgdaFunction{HomReduct}\AgdaSpace{}%
\AgdaSymbol{:}\AgdaSpace{}%
\AgdaFunction{hom}\<%
\\
%
\>[2]\AgdaFunction{HomReduct}\AgdaSpace{}%
\AgdaSymbol{=}\AgdaSpace{}%
\AgdaBound{h}\AgdaSpace{}%
\AgdaOperator{\AgdaInductiveConstructor{,}}\AgdaSpace{}%
\AgdaField{isHom}\<%
\\
%
\\[\AgdaEmptyExtraSkip]%
%
\>[1]\AgdaFunction{epi}\AgdaSpace{}%
\AgdaSymbol{:}\AgdaSpace{}%
\AgdaPrimitive{Type}\AgdaSpace{}%
\AgdaFunction{c}\<%
\\
%
\>[1]\AgdaFunction{epi}\AgdaSpace{}%
\AgdaSymbol{=}\AgdaSpace{}%
\AgdaRecord{Σ}\AgdaSpace{}%
\AgdaSymbol{(}\AgdaOperator{\AgdaFunction{𝔻[}}\AgdaSpace{}%
\AgdaBound{𝑨}\AgdaSpace{}%
\AgdaOperator{\AgdaFunction{]}}\AgdaSpace{}%
\AgdaOperator{\AgdaRecord{⟶}}\AgdaSpace{}%
\AgdaOperator{\AgdaFunction{𝔻[}}\AgdaSpace{}%
\AgdaBound{𝑩}\AgdaSpace{}%
\AgdaOperator{\AgdaFunction{]}}\AgdaSymbol{)}\AgdaSpace{}%
\AgdaRecord{IsEpi}\<%
\\
%
\\[\AgdaEmptyExtraSkip]%
\>[0]\AgdaKeyword{open}\AgdaSpace{}%
\AgdaModule{IsHom}\AgdaSpace{}%
\AgdaSymbol{;}\AgdaSpace{}%
\AgdaKeyword{open}\AgdaSpace{}%
\AgdaModule{IsMon}\AgdaSpace{}%
\AgdaSymbol{;}\AgdaSpace{}%
\AgdaKeyword{open}\AgdaSpace{}%
\AgdaModule{IsEpi}\<%
\\
%
\\[\AgdaEmptyExtraSkip]%
\>[0]\AgdaKeyword{module}\AgdaSpace{}%
\AgdaModule{\AgdaUnderscore{}}\AgdaSpace{}%
\AgdaSymbol{(}\AgdaBound{𝑨}\AgdaSpace{}%
\AgdaSymbol{:}\AgdaSpace{}%
\AgdaRecord{Algebra}\AgdaSpace{}%
\AgdaGeneralizable{α}\AgdaSpace{}%
\AgdaGeneralizable{ρᵃ}\AgdaSymbol{)(}\AgdaBound{𝑩}\AgdaSpace{}%
\AgdaSymbol{:}\AgdaSpace{}%
\AgdaRecord{Algebra}\AgdaSpace{}%
\AgdaGeneralizable{β}\AgdaSpace{}%
\AgdaGeneralizable{ρᵇ}\AgdaSymbol{)}\AgdaSpace{}%
\AgdaKeyword{where}\<%
\\
%
\\[\AgdaEmptyExtraSkip]%
\>[0][@{}l@{\AgdaIndent{0}}]%
\>[1]\AgdaFunction{mon→intohom}\AgdaSpace{}%
\AgdaSymbol{:}\AgdaSpace{}%
\AgdaFunction{mon}\AgdaSpace{}%
\AgdaBound{𝑨}\AgdaSpace{}%
\AgdaBound{𝑩}\AgdaSpace{}%
\AgdaSymbol{→}\AgdaSpace{}%
\AgdaFunction{Σ[}\AgdaSpace{}%
\AgdaBound{h}\AgdaSpace{}%
\AgdaFunction{∈}\AgdaSpace{}%
\AgdaFunction{hom}\AgdaSpace{}%
\AgdaBound{𝑨}\AgdaSpace{}%
\AgdaBound{𝑩}\AgdaSpace{}%
\AgdaFunction{]}\AgdaSpace{}%
\AgdaFunction{IsInjective}\AgdaSpace{}%
\AgdaOperator{\AgdaFunction{∣}}\AgdaSpace{}%
\AgdaBound{h}\AgdaSpace{}%
\AgdaOperator{\AgdaFunction{∣}}\<%
\\
%
\>[1]\AgdaFunction{mon→intohom}\AgdaSpace{}%
\AgdaSymbol{(}\AgdaBound{hh}\AgdaSpace{}%
\AgdaOperator{\AgdaInductiveConstructor{,}}\AgdaSpace{}%
\AgdaBound{hhM}\AgdaSymbol{)}\AgdaSpace{}%
\AgdaSymbol{=}\AgdaSpace{}%
\AgdaSymbol{(}\AgdaBound{hh}\AgdaSpace{}%
\AgdaOperator{\AgdaInductiveConstructor{,}}\AgdaSpace{}%
\AgdaField{isHom}\AgdaSpace{}%
\AgdaBound{hhM}\AgdaSymbol{)}\AgdaSpace{}%
\AgdaOperator{\AgdaInductiveConstructor{,}}\AgdaSpace{}%
\AgdaField{isInjective}\AgdaSpace{}%
\AgdaBound{hhM}\<%
\\
%
\\[\AgdaEmptyExtraSkip]%
%
\>[1]\AgdaFunction{epi→ontohom}\AgdaSpace{}%
\AgdaSymbol{:}\AgdaSpace{}%
\AgdaFunction{epi}\AgdaSpace{}%
\AgdaBound{𝑨}\AgdaSpace{}%
\AgdaBound{𝑩}\AgdaSpace{}%
\AgdaSymbol{→}\AgdaSpace{}%
\AgdaFunction{Σ[}\AgdaSpace{}%
\AgdaBound{h}\AgdaSpace{}%
\AgdaFunction{∈}\AgdaSpace{}%
\AgdaFunction{hom}\AgdaSpace{}%
\AgdaBound{𝑨}\AgdaSpace{}%
\AgdaBound{𝑩}\AgdaSpace{}%
\AgdaFunction{]}\AgdaSpace{}%
\AgdaFunction{IsSurjective}\AgdaSpace{}%
\AgdaOperator{\AgdaFunction{∣}}\AgdaSpace{}%
\AgdaBound{h}\AgdaSpace{}%
\AgdaOperator{\AgdaFunction{∣}}\<%
\\
%
\>[1]\AgdaFunction{epi→ontohom}\AgdaSpace{}%
\AgdaSymbol{(}\AgdaBound{hh}\AgdaSpace{}%
\AgdaOperator{\AgdaInductiveConstructor{,}}\AgdaSpace{}%
\AgdaBound{hhE}\AgdaSymbol{)}\AgdaSpace{}%
\AgdaSymbol{=}\AgdaSpace{}%
\AgdaSymbol{(}\AgdaBound{hh}\AgdaSpace{}%
\AgdaOperator{\AgdaInductiveConstructor{,}}\AgdaSpace{}%
\AgdaField{isHom}\AgdaSpace{}%
\AgdaBound{hhE}\AgdaSymbol{)}\AgdaSpace{}%
\AgdaOperator{\AgdaInductiveConstructor{,}}\AgdaSpace{}%
\AgdaField{isSurjective}\AgdaSpace{}%
\AgdaBound{hhE}\<%
\end{code}

\hypertarget{basic-properties-of-homomorphisms}{%
\subsubsection{Basic properties of
homomorphisms}\label{basic-properties-of-homomorphisms}}

Here are some definitions and theorems extracted from the
\href{https://ualib.github.io/agda-algebras/Homomorphisms.Func.Properties.html}{Homomorphisms.Func.Properties}
module of the \href{https://ualib.github.io/agda-algebras}{Agda
Universal Algebra Library}.

\hypertarget{composition-of-homomorphisms}{%
\paragraph{Composition of
homomorphisms}\label{composition-of-homomorphisms}}

The composition of homomorphisms is again a homomorphism. Similarly, the
composition of epimorphisms is again an epimorphism.

\begin{code}%
\>[0]\<%
\\
\>[0]\AgdaKeyword{module}\AgdaSpace{}%
\AgdaModule{\AgdaUnderscore{}}%
\>[10]\AgdaSymbol{\{}\AgdaBound{𝑨}\AgdaSpace{}%
\AgdaSymbol{:}\AgdaSpace{}%
\AgdaRecord{Algebra}\AgdaSpace{}%
\AgdaGeneralizable{α}\AgdaSpace{}%
\AgdaGeneralizable{ρᵃ}\AgdaSymbol{\}}\AgdaSpace{}%
\AgdaSymbol{\{}\AgdaBound{𝑩}\AgdaSpace{}%
\AgdaSymbol{:}\AgdaSpace{}%
\AgdaRecord{Algebra}\AgdaSpace{}%
\AgdaGeneralizable{β}\AgdaSpace{}%
\AgdaGeneralizable{ρᵇ}\AgdaSymbol{\}}\AgdaSpace{}%
\AgdaSymbol{\{}\AgdaBound{𝑪}\AgdaSpace{}%
\AgdaSymbol{:}\AgdaSpace{}%
\AgdaRecord{Algebra}\AgdaSpace{}%
\AgdaGeneralizable{γ}\AgdaSpace{}%
\AgdaGeneralizable{ρᶜ}\AgdaSymbol{\}}\<%
\\
%
\>[10]\AgdaSymbol{\{}\AgdaBound{g}\AgdaSpace{}%
\AgdaSymbol{:}\AgdaSpace{}%
\AgdaOperator{\AgdaFunction{𝔻[}}\AgdaSpace{}%
\AgdaBound{𝑨}\AgdaSpace{}%
\AgdaOperator{\AgdaFunction{]}}\AgdaSpace{}%
\AgdaOperator{\AgdaRecord{⟶}}\AgdaSpace{}%
\AgdaOperator{\AgdaFunction{𝔻[}}\AgdaSpace{}%
\AgdaBound{𝑩}\AgdaSpace{}%
\AgdaOperator{\AgdaFunction{]}}\AgdaSymbol{\}\{}\AgdaBound{h}\AgdaSpace{}%
\AgdaSymbol{:}\AgdaSpace{}%
\AgdaOperator{\AgdaFunction{𝔻[}}\AgdaSpace{}%
\AgdaBound{𝑩}\AgdaSpace{}%
\AgdaOperator{\AgdaFunction{]}}\AgdaSpace{}%
\AgdaOperator{\AgdaRecord{⟶}}\AgdaSpace{}%
\AgdaOperator{\AgdaFunction{𝔻[}}\AgdaSpace{}%
\AgdaBound{𝑪}\AgdaSpace{}%
\AgdaOperator{\AgdaFunction{]}}\AgdaSymbol{\}}\AgdaSpace{}%
\AgdaKeyword{where}\<%
\\
%
\\[\AgdaEmptyExtraSkip]%
\>[0][@{}l@{\AgdaIndent{0}}]%
\>[2]\AgdaKeyword{open}\AgdaSpace{}%
\AgdaModule{Setoid}\AgdaSpace{}%
\AgdaOperator{\AgdaFunction{𝔻[}}\AgdaSpace{}%
\AgdaBound{𝑪}\AgdaSpace{}%
\AgdaOperator{\AgdaFunction{]}}\AgdaSpace{}%
\AgdaKeyword{using}\AgdaSpace{}%
\AgdaSymbol{(}\AgdaSpace{}%
\AgdaFunction{trans}\AgdaSpace{}%
\AgdaSymbol{)}\<%
\\
%
\\[\AgdaEmptyExtraSkip]%
%
\>[2]\AgdaFunction{∘-is-hom}\AgdaSpace{}%
\AgdaSymbol{:}\AgdaSpace{}%
\AgdaRecord{IsHom}\AgdaSpace{}%
\AgdaBound{𝑨}\AgdaSpace{}%
\AgdaBound{𝑩}\AgdaSpace{}%
\AgdaBound{g}\AgdaSpace{}%
\AgdaSymbol{→}\AgdaSpace{}%
\AgdaRecord{IsHom}\AgdaSpace{}%
\AgdaBound{𝑩}\AgdaSpace{}%
\AgdaBound{𝑪}\AgdaSpace{}%
\AgdaBound{h}\AgdaSpace{}%
\AgdaSymbol{→}\AgdaSpace{}%
\AgdaRecord{IsHom}\AgdaSpace{}%
\AgdaBound{𝑨}\AgdaSpace{}%
\AgdaBound{𝑪}\AgdaSpace{}%
\AgdaSymbol{(}\AgdaBound{h}\AgdaSpace{}%
\AgdaOperator{\AgdaFunction{⟨∘⟩}}\AgdaSpace{}%
\AgdaBound{g}\AgdaSymbol{)}\<%
\\
%
\>[2]\AgdaFunction{∘-is-hom}\AgdaSpace{}%
\AgdaBound{ghom}\AgdaSpace{}%
\AgdaBound{hhom}\AgdaSpace{}%
\AgdaSymbol{=}\AgdaSpace{}%
\AgdaInductiveConstructor{mkhom}\AgdaSpace{}%
\AgdaFunction{c}\<%
\\
\>[2][@{}l@{\AgdaIndent{0}}]%
\>[3]\AgdaKeyword{where}\<%
\\
%
\>[3]\AgdaFunction{c}\AgdaSpace{}%
\AgdaSymbol{:}\AgdaSpace{}%
\AgdaFunction{compatible-map}\AgdaSpace{}%
\AgdaBound{𝑨}\AgdaSpace{}%
\AgdaBound{𝑪}\AgdaSpace{}%
\AgdaSymbol{(}\AgdaBound{h}\AgdaSpace{}%
\AgdaOperator{\AgdaFunction{⟨∘⟩}}\AgdaSpace{}%
\AgdaBound{g}\AgdaSymbol{)}\<%
\\
%
\>[3]\AgdaFunction{c}\AgdaSpace{}%
\AgdaSymbol{=}\AgdaSpace{}%
\AgdaFunction{trans}\AgdaSpace{}%
\AgdaSymbol{(}\AgdaField{cong}\AgdaSpace{}%
\AgdaBound{h}\AgdaSpace{}%
\AgdaSymbol{(}\AgdaField{compatible}\AgdaSpace{}%
\AgdaBound{ghom}\AgdaSymbol{))}\AgdaSpace{}%
\AgdaSymbol{(}\AgdaField{compatible}\AgdaSpace{}%
\AgdaBound{hhom}\AgdaSymbol{)}\<%
\\
%
\\[\AgdaEmptyExtraSkip]%
%
\>[2]\AgdaFunction{∘-is-epi}\AgdaSpace{}%
\AgdaSymbol{:}\AgdaSpace{}%
\AgdaRecord{IsEpi}\AgdaSpace{}%
\AgdaBound{𝑨}\AgdaSpace{}%
\AgdaBound{𝑩}\AgdaSpace{}%
\AgdaBound{g}\AgdaSpace{}%
\AgdaSymbol{→}\AgdaSpace{}%
\AgdaRecord{IsEpi}\AgdaSpace{}%
\AgdaBound{𝑩}\AgdaSpace{}%
\AgdaBound{𝑪}\AgdaSpace{}%
\AgdaBound{h}\AgdaSpace{}%
\AgdaSymbol{→}\AgdaSpace{}%
\AgdaRecord{IsEpi}\AgdaSpace{}%
\AgdaBound{𝑨}\AgdaSpace{}%
\AgdaBound{𝑪}\AgdaSpace{}%
\AgdaSymbol{(}\AgdaBound{h}\AgdaSpace{}%
\AgdaOperator{\AgdaFunction{⟨∘⟩}}\AgdaSpace{}%
\AgdaBound{g}\AgdaSymbol{)}\<%
\\
%
\>[2]\AgdaFunction{∘-is-epi}\AgdaSpace{}%
\AgdaBound{gE}\AgdaSpace{}%
\AgdaBound{hE}\AgdaSpace{}%
\AgdaSymbol{=}\AgdaSpace{}%
\AgdaKeyword{record}%
\>[27]\AgdaSymbol{\{}\AgdaSpace{}%
\AgdaField{isHom}\AgdaSpace{}%
\AgdaSymbol{=}\AgdaSpace{}%
\AgdaFunction{∘-is-hom}\AgdaSpace{}%
\AgdaSymbol{(}\AgdaField{isHom}\AgdaSpace{}%
\AgdaBound{gE}\AgdaSymbol{)}\AgdaSpace{}%
\AgdaSymbol{(}\AgdaField{isHom}\AgdaSpace{}%
\AgdaBound{hE}\AgdaSymbol{)}\<%
\\
%
\>[27]\AgdaSymbol{;}\AgdaSpace{}%
\AgdaField{isSurjective}\AgdaSpace{}%
\AgdaSymbol{=}\AgdaSpace{}%
\AgdaFunction{∘-IsSurjective}\AgdaSpace{}%
\AgdaBound{g}\AgdaSpace{}%
\AgdaBound{h}\AgdaSpace{}%
\AgdaSymbol{(}\AgdaField{isSurjective}\AgdaSpace{}%
\AgdaBound{gE}\AgdaSymbol{)}\AgdaSpace{}%
\AgdaSymbol{(}\AgdaField{isSurjective}\AgdaSpace{}%
\AgdaBound{hE}\AgdaSymbol{)}\AgdaSpace{}%
\AgdaSymbol{\}}\<%
\\
%
\\[\AgdaEmptyExtraSkip]%
%
\\[\AgdaEmptyExtraSkip]%
\>[0]\AgdaKeyword{module}\AgdaSpace{}%
\AgdaModule{\AgdaUnderscore{}}\AgdaSpace{}%
\AgdaSymbol{\{}\AgdaBound{𝑨}\AgdaSpace{}%
\AgdaSymbol{:}\AgdaSpace{}%
\AgdaRecord{Algebra}\AgdaSpace{}%
\AgdaGeneralizable{α}\AgdaSpace{}%
\AgdaGeneralizable{ρᵃ}\AgdaSymbol{\}}\AgdaSpace{}%
\AgdaSymbol{\{}\AgdaBound{𝑩}\AgdaSpace{}%
\AgdaSymbol{:}\AgdaSpace{}%
\AgdaRecord{Algebra}\AgdaSpace{}%
\AgdaGeneralizable{β}\AgdaSpace{}%
\AgdaGeneralizable{ρᵇ}\AgdaSymbol{\}}\AgdaSpace{}%
\AgdaSymbol{\{}\AgdaBound{𝑪}\AgdaSpace{}%
\AgdaSymbol{:}\AgdaSpace{}%
\AgdaRecord{Algebra}\AgdaSpace{}%
\AgdaGeneralizable{γ}\AgdaSpace{}%
\AgdaGeneralizable{ρᶜ}\AgdaSymbol{\}}\AgdaSpace{}%
\AgdaKeyword{where}\<%
\\
%
\\[\AgdaEmptyExtraSkip]%
\>[0][@{}l@{\AgdaIndent{0}}]%
\>[2]\AgdaFunction{∘-hom}\AgdaSpace{}%
\AgdaSymbol{:}\AgdaSpace{}%
\AgdaFunction{hom}\AgdaSpace{}%
\AgdaBound{𝑨}\AgdaSpace{}%
\AgdaBound{𝑩}\AgdaSpace{}%
\AgdaSymbol{→}\AgdaSpace{}%
\AgdaFunction{hom}\AgdaSpace{}%
\AgdaBound{𝑩}\AgdaSpace{}%
\AgdaBound{𝑪}%
\>[29]\AgdaSymbol{→}\AgdaSpace{}%
\AgdaFunction{hom}\AgdaSpace{}%
\AgdaBound{𝑨}\AgdaSpace{}%
\AgdaBound{𝑪}\<%
\\
%
\>[2]\AgdaFunction{∘-hom}\AgdaSpace{}%
\AgdaSymbol{(}\AgdaBound{h}\AgdaSpace{}%
\AgdaOperator{\AgdaInductiveConstructor{,}}\AgdaSpace{}%
\AgdaBound{hhom}\AgdaSymbol{)}\AgdaSpace{}%
\AgdaSymbol{(}\AgdaBound{g}\AgdaSpace{}%
\AgdaOperator{\AgdaInductiveConstructor{,}}\AgdaSpace{}%
\AgdaBound{ghom}\AgdaSymbol{)}\AgdaSpace{}%
\AgdaSymbol{=}\AgdaSpace{}%
\AgdaSymbol{(}\AgdaBound{g}\AgdaSpace{}%
\AgdaOperator{\AgdaFunction{⟨∘⟩}}\AgdaSpace{}%
\AgdaBound{h}\AgdaSymbol{)}\AgdaSpace{}%
\AgdaOperator{\AgdaInductiveConstructor{,}}\AgdaSpace{}%
\AgdaFunction{∘-is-hom}\AgdaSpace{}%
\AgdaBound{hhom}\AgdaSpace{}%
\AgdaBound{ghom}\<%
\\
%
\\[\AgdaEmptyExtraSkip]%
%
\>[2]\AgdaFunction{∘-epi}\AgdaSpace{}%
\AgdaSymbol{:}\AgdaSpace{}%
\AgdaFunction{epi}\AgdaSpace{}%
\AgdaBound{𝑨}\AgdaSpace{}%
\AgdaBound{𝑩}\AgdaSpace{}%
\AgdaSymbol{→}\AgdaSpace{}%
\AgdaFunction{epi}\AgdaSpace{}%
\AgdaBound{𝑩}\AgdaSpace{}%
\AgdaBound{𝑪}%
\>[29]\AgdaSymbol{→}\AgdaSpace{}%
\AgdaFunction{epi}\AgdaSpace{}%
\AgdaBound{𝑨}\AgdaSpace{}%
\AgdaBound{𝑪}\<%
\\
%
\>[2]\AgdaFunction{∘-epi}\AgdaSpace{}%
\AgdaSymbol{(}\AgdaBound{h}\AgdaSpace{}%
\AgdaOperator{\AgdaInductiveConstructor{,}}\AgdaSpace{}%
\AgdaBound{hepi}\AgdaSymbol{)}\AgdaSpace{}%
\AgdaSymbol{(}\AgdaBound{g}\AgdaSpace{}%
\AgdaOperator{\AgdaInductiveConstructor{,}}\AgdaSpace{}%
\AgdaBound{gepi}\AgdaSymbol{)}\AgdaSpace{}%
\AgdaSymbol{=}\AgdaSpace{}%
\AgdaSymbol{(}\AgdaBound{g}\AgdaSpace{}%
\AgdaOperator{\AgdaFunction{⟨∘⟩}}\AgdaSpace{}%
\AgdaBound{h}\AgdaSymbol{)}\AgdaSpace{}%
\AgdaOperator{\AgdaInductiveConstructor{,}}\AgdaSpace{}%
\AgdaFunction{∘-is-epi}\AgdaSpace{}%
\AgdaBound{hepi}\AgdaSpace{}%
\AgdaBound{gepi}\<%
\end{code}

\hypertarget{universe-lifting-of-homomorphisms}{%
\paragraph{Universe lifting of
homomorphisms}\label{universe-lifting-of-homomorphisms}}

First we define the identity homomorphism for setoid algebras and then
we prove that the operations of lifting and lowering of a setoid algebra
are homomorphisms.

\begin{code}[hide]%
\>[0]\<%
\\
\>[0]\AgdaFunction{𝒾𝒹}\AgdaSpace{}%
\AgdaSymbol{:}\AgdaSpace{}%
\AgdaSymbol{\{}\AgdaBound{𝑨}\AgdaSpace{}%
\AgdaSymbol{:}\AgdaSpace{}%
\AgdaRecord{Algebra}\AgdaSpace{}%
\AgdaGeneralizable{α}\AgdaSpace{}%
\AgdaGeneralizable{ρᵃ}\AgdaSymbol{\}}\AgdaSpace{}%
\AgdaSymbol{→}\AgdaSpace{}%
\AgdaFunction{hom}\AgdaSpace{}%
\AgdaBound{𝑨}\AgdaSpace{}%
\AgdaBound{𝑨}\<%
\\
\>[0]\AgdaFunction{𝒾𝒹}\AgdaSpace{}%
\AgdaSymbol{\{}\AgdaArgument{𝑨}\AgdaSpace{}%
\AgdaSymbol{=}\AgdaSpace{}%
\AgdaBound{𝑨}\AgdaSymbol{\}}\AgdaSpace{}%
\AgdaSymbol{=}\AgdaSpace{}%
\AgdaFunction{𝑖𝑑}\AgdaSpace{}%
\AgdaOperator{\AgdaInductiveConstructor{,}}\AgdaSpace{}%
\AgdaInductiveConstructor{mkhom}\AgdaSpace{}%
\AgdaSymbol{(}\AgdaFunction{reflexive}\AgdaSpace{}%
\AgdaInductiveConstructor{≡.refl}\AgdaSymbol{)}\AgdaSpace{}%
\AgdaKeyword{where}\AgdaSpace{}%
\AgdaKeyword{open}\AgdaSpace{}%
\AgdaModule{Setoid}\AgdaSpace{}%
\AgdaSymbol{(}\AgdaSpace{}%
\AgdaField{Domain}\AgdaSpace{}%
\AgdaBound{𝑨}\AgdaSpace{}%
\AgdaSymbol{)}\AgdaSpace{}%
\AgdaKeyword{using}\AgdaSpace{}%
\AgdaSymbol{(}\AgdaSpace{}%
\AgdaFunction{reflexive}\AgdaSpace{}%
\AgdaSymbol{)}\<%
\\
%
\\[\AgdaEmptyExtraSkip]%
\>[0]\AgdaKeyword{module}\AgdaSpace{}%
\AgdaModule{\AgdaUnderscore{}}\AgdaSpace{}%
\AgdaSymbol{\{}\AgdaBound{𝑨}\AgdaSpace{}%
\AgdaSymbol{:}\AgdaSpace{}%
\AgdaRecord{Algebra}\AgdaSpace{}%
\AgdaGeneralizable{α}\AgdaSpace{}%
\AgdaGeneralizable{ρᵃ}\AgdaSymbol{\}\{}\AgdaBound{ℓ}\AgdaSpace{}%
\AgdaSymbol{:}\AgdaSpace{}%
\AgdaPostulate{Level}\AgdaSymbol{\}}\AgdaSpace{}%
\AgdaKeyword{where}\<%
\\
\>[0][@{}l@{\AgdaIndent{0}}]%
\>[1]\AgdaKeyword{open}\AgdaSpace{}%
\AgdaModule{Setoid}\AgdaSpace{}%
\AgdaOperator{\AgdaFunction{𝔻[}}\AgdaSpace{}%
\AgdaBound{𝑨}\AgdaSpace{}%
\AgdaOperator{\AgdaFunction{]}}%
\>[33]\AgdaKeyword{using}\AgdaSpace{}%
\AgdaSymbol{(}\AgdaSpace{}%
\AgdaFunction{reflexive}\AgdaSpace{}%
\AgdaSymbol{)}%
\>[54]\AgdaKeyword{renaming}\AgdaSpace{}%
\AgdaSymbol{(}\AgdaSpace{}%
\AgdaOperator{\AgdaField{\AgdaUnderscore{}≈\AgdaUnderscore{}}}\AgdaSpace{}%
\AgdaSymbol{to}\AgdaSpace{}%
\AgdaOperator{\AgdaField{\AgdaUnderscore{}≈₁\AgdaUnderscore{}}}\AgdaSpace{}%
\AgdaSymbol{;}\AgdaSpace{}%
\AgdaFunction{refl}\AgdaSpace{}%
\AgdaSymbol{to}\AgdaSpace{}%
\AgdaFunction{refl₁}\AgdaSpace{}%
\AgdaSymbol{)}\<%
\\
%
\>[1]\AgdaKeyword{open}\AgdaSpace{}%
\AgdaModule{Setoid}\AgdaSpace{}%
\AgdaOperator{\AgdaFunction{𝔻[}}\AgdaSpace{}%
\AgdaFunction{Lift-Algˡ}\AgdaSpace{}%
\AgdaBound{𝑨}\AgdaSpace{}%
\AgdaBound{ℓ}\AgdaSpace{}%
\AgdaOperator{\AgdaFunction{]}}%
\>[33]\AgdaKeyword{using}\AgdaSpace{}%
\AgdaSymbol{()}%
\>[54]\AgdaKeyword{renaming}\AgdaSpace{}%
\AgdaSymbol{(}\AgdaSpace{}%
\AgdaOperator{\AgdaField{\AgdaUnderscore{}≈\AgdaUnderscore{}}}\AgdaSpace{}%
\AgdaSymbol{to}\AgdaSpace{}%
\AgdaOperator{\AgdaField{\AgdaUnderscore{}≈ˡ\AgdaUnderscore{}}}\AgdaSpace{}%
\AgdaSymbol{;}\AgdaSpace{}%
\AgdaFunction{refl}\AgdaSpace{}%
\AgdaSymbol{to}\AgdaSpace{}%
\AgdaFunction{reflˡ}\AgdaSymbol{)}\<%
\\
%
\>[1]\AgdaKeyword{open}\AgdaSpace{}%
\AgdaModule{Setoid}\AgdaSpace{}%
\AgdaOperator{\AgdaFunction{𝔻[}}\AgdaSpace{}%
\AgdaFunction{Lift-Algʳ}\AgdaSpace{}%
\AgdaBound{𝑨}\AgdaSpace{}%
\AgdaBound{ℓ}\AgdaSpace{}%
\AgdaOperator{\AgdaFunction{]}}%
\>[33]\AgdaKeyword{using}\AgdaSpace{}%
\AgdaSymbol{()}%
\>[54]\AgdaKeyword{renaming}\AgdaSpace{}%
\AgdaSymbol{(}\AgdaSpace{}%
\AgdaOperator{\AgdaField{\AgdaUnderscore{}≈\AgdaUnderscore{}}}\AgdaSpace{}%
\AgdaSymbol{to}\AgdaSpace{}%
\AgdaOperator{\AgdaField{\AgdaUnderscore{}≈ʳ\AgdaUnderscore{}}}\AgdaSpace{}%
\AgdaSymbol{;}\AgdaSpace{}%
\AgdaFunction{refl}\AgdaSpace{}%
\AgdaSymbol{to}\AgdaSpace{}%
\AgdaFunction{reflʳ}\AgdaSymbol{)}\<%
\\
%
\>[1]\AgdaKeyword{open}\AgdaSpace{}%
\AgdaModule{Level}\<%
\\
%
\\[\AgdaEmptyExtraSkip]%
%
\>[1]\AgdaFunction{ToLiftˡ}\AgdaSpace{}%
\AgdaSymbol{:}\AgdaSpace{}%
\AgdaFunction{hom}\AgdaSpace{}%
\AgdaBound{𝑨}\AgdaSpace{}%
\AgdaSymbol{(}\AgdaFunction{Lift-Algˡ}\AgdaSpace{}%
\AgdaBound{𝑨}\AgdaSpace{}%
\AgdaBound{ℓ}\AgdaSymbol{)}\<%
\\
%
\>[1]\AgdaFunction{ToLiftˡ}\AgdaSpace{}%
\AgdaSymbol{=}\AgdaSpace{}%
\AgdaKeyword{record}\AgdaSpace{}%
\AgdaSymbol{\{}\AgdaSpace{}%
\AgdaField{f}\AgdaSpace{}%
\AgdaSymbol{=}\AgdaSpace{}%
\AgdaInductiveConstructor{lift}\AgdaSpace{}%
\AgdaSymbol{;}\AgdaSpace{}%
\AgdaField{cong}\AgdaSpace{}%
\AgdaSymbol{=}\AgdaSpace{}%
\AgdaFunction{id}\AgdaSpace{}%
\AgdaSymbol{\}}\AgdaSpace{}%
\AgdaOperator{\AgdaInductiveConstructor{,}}\AgdaSpace{}%
\AgdaInductiveConstructor{mkhom}\AgdaSpace{}%
\AgdaSymbol{(}\AgdaFunction{reflexive}\AgdaSpace{}%
\AgdaInductiveConstructor{≡.refl}\AgdaSymbol{)}\<%
\\
%
\\[\AgdaEmptyExtraSkip]%
%
\>[1]\AgdaFunction{FromLiftˡ}\AgdaSpace{}%
\AgdaSymbol{:}\AgdaSpace{}%
\AgdaFunction{hom}\AgdaSpace{}%
\AgdaSymbol{(}\AgdaFunction{Lift-Algˡ}\AgdaSpace{}%
\AgdaBound{𝑨}\AgdaSpace{}%
\AgdaBound{ℓ}\AgdaSymbol{)}\AgdaSpace{}%
\AgdaBound{𝑨}\<%
\\
%
\>[1]\AgdaFunction{FromLiftˡ}\AgdaSpace{}%
\AgdaSymbol{=}\AgdaSpace{}%
\AgdaKeyword{record}\AgdaSpace{}%
\AgdaSymbol{\{}\AgdaSpace{}%
\AgdaField{f}\AgdaSpace{}%
\AgdaSymbol{=}\AgdaSpace{}%
\AgdaField{lower}\AgdaSpace{}%
\AgdaSymbol{;}\AgdaSpace{}%
\AgdaField{cong}\AgdaSpace{}%
\AgdaSymbol{=}\AgdaSpace{}%
\AgdaFunction{id}\AgdaSpace{}%
\AgdaSymbol{\}}\AgdaSpace{}%
\AgdaOperator{\AgdaInductiveConstructor{,}}\AgdaSpace{}%
\AgdaInductiveConstructor{mkhom}\AgdaSpace{}%
\AgdaFunction{reflˡ}\<%
\\
%
\\[\AgdaEmptyExtraSkip]%
%
\>[1]\AgdaFunction{ToFromLiftˡ}\AgdaSpace{}%
\AgdaSymbol{:}\AgdaSpace{}%
\AgdaSymbol{∀}\AgdaSpace{}%
\AgdaBound{b}\AgdaSpace{}%
\AgdaSymbol{→}%
\>[22]\AgdaSymbol{(}\AgdaOperator{\AgdaFunction{∣}}\AgdaSpace{}%
\AgdaFunction{ToLiftˡ}\AgdaSpace{}%
\AgdaOperator{\AgdaFunction{∣}}\AgdaSpace{}%
\AgdaOperator{\AgdaField{⟨\$⟩}}\AgdaSpace{}%
\AgdaSymbol{(}\AgdaOperator{\AgdaFunction{∣}}\AgdaSpace{}%
\AgdaFunction{FromLiftˡ}\AgdaSpace{}%
\AgdaOperator{\AgdaFunction{∣}}\AgdaSpace{}%
\AgdaOperator{\AgdaField{⟨\$⟩}}\AgdaSpace{}%
\AgdaBound{b}\AgdaSymbol{))}\AgdaSpace{}%
\AgdaOperator{\AgdaFunction{≈ˡ}}\AgdaSpace{}%
\AgdaBound{b}\<%
\\
%
\>[1]\AgdaFunction{ToFromLiftˡ}\AgdaSpace{}%
\AgdaBound{b}\AgdaSpace{}%
\AgdaSymbol{=}\AgdaSpace{}%
\AgdaFunction{refl₁}\<%
\\
%
\\[\AgdaEmptyExtraSkip]%
%
\>[1]\AgdaFunction{FromToLiftˡ}\AgdaSpace{}%
\AgdaSymbol{:}\AgdaSpace{}%
\AgdaSymbol{∀}\AgdaSpace{}%
\AgdaBound{a}\AgdaSpace{}%
\AgdaSymbol{→}\AgdaSpace{}%
\AgdaSymbol{(}\AgdaOperator{\AgdaFunction{∣}}\AgdaSpace{}%
\AgdaFunction{FromLiftˡ}\AgdaSpace{}%
\AgdaOperator{\AgdaFunction{∣}}\AgdaSpace{}%
\AgdaOperator{\AgdaField{⟨\$⟩}}\AgdaSpace{}%
\AgdaSymbol{(}\AgdaOperator{\AgdaFunction{∣}}\AgdaSpace{}%
\AgdaFunction{ToLiftˡ}\AgdaSpace{}%
\AgdaOperator{\AgdaFunction{∣}}\AgdaSpace{}%
\AgdaOperator{\AgdaField{⟨\$⟩}}\AgdaSpace{}%
\AgdaBound{a}\AgdaSymbol{))}\AgdaSpace{}%
\AgdaOperator{\AgdaFunction{≈₁}}\AgdaSpace{}%
\AgdaBound{a}\<%
\\
%
\>[1]\AgdaFunction{FromToLiftˡ}\AgdaSpace{}%
\AgdaBound{a}\AgdaSpace{}%
\AgdaSymbol{=}\AgdaSpace{}%
\AgdaFunction{refl₁}\<%
\\
%
\\[\AgdaEmptyExtraSkip]%
%
\>[1]\AgdaFunction{ToLiftʳ}\AgdaSpace{}%
\AgdaSymbol{:}\AgdaSpace{}%
\AgdaFunction{hom}\AgdaSpace{}%
\AgdaBound{𝑨}\AgdaSpace{}%
\AgdaSymbol{(}\AgdaFunction{Lift-Algʳ}\AgdaSpace{}%
\AgdaBound{𝑨}\AgdaSpace{}%
\AgdaBound{ℓ}\AgdaSymbol{)}\<%
\\
%
\>[1]\AgdaFunction{ToLiftʳ}\AgdaSpace{}%
\AgdaSymbol{=}\AgdaSpace{}%
\AgdaKeyword{record}\AgdaSpace{}%
\AgdaSymbol{\{}\AgdaSpace{}%
\AgdaField{f}\AgdaSpace{}%
\AgdaSymbol{=}\AgdaSpace{}%
\AgdaFunction{id}\AgdaSpace{}%
\AgdaSymbol{;}\AgdaSpace{}%
\AgdaField{cong}\AgdaSpace{}%
\AgdaSymbol{=}\AgdaSpace{}%
\AgdaInductiveConstructor{lift}\AgdaSpace{}%
\AgdaSymbol{\}}\AgdaSpace{}%
\AgdaOperator{\AgdaInductiveConstructor{,}}\AgdaSpace{}%
\AgdaInductiveConstructor{mkhom}\AgdaSpace{}%
\AgdaSymbol{(}\AgdaInductiveConstructor{lift}\AgdaSpace{}%
\AgdaSymbol{(}\AgdaFunction{reflexive}\AgdaSpace{}%
\AgdaInductiveConstructor{≡.refl}\AgdaSymbol{))}\<%
\\
%
\\[\AgdaEmptyExtraSkip]%
%
\>[1]\AgdaFunction{FromLiftʳ}\AgdaSpace{}%
\AgdaSymbol{:}\AgdaSpace{}%
\AgdaFunction{hom}\AgdaSpace{}%
\AgdaSymbol{(}\AgdaFunction{Lift-Algʳ}\AgdaSpace{}%
\AgdaBound{𝑨}\AgdaSpace{}%
\AgdaBound{ℓ}\AgdaSymbol{)}\AgdaSpace{}%
\AgdaBound{𝑨}\<%
\\
%
\>[1]\AgdaFunction{FromLiftʳ}\AgdaSpace{}%
\AgdaSymbol{=}\AgdaSpace{}%
\AgdaKeyword{record}\AgdaSpace{}%
\AgdaSymbol{\{}\AgdaSpace{}%
\AgdaField{f}\AgdaSpace{}%
\AgdaSymbol{=}\AgdaSpace{}%
\AgdaFunction{id}\AgdaSpace{}%
\AgdaSymbol{;}\AgdaSpace{}%
\AgdaField{cong}\AgdaSpace{}%
\AgdaSymbol{=}\AgdaSpace{}%
\AgdaField{lower}\AgdaSpace{}%
\AgdaSymbol{\}}\AgdaSpace{}%
\AgdaOperator{\AgdaInductiveConstructor{,}}\AgdaSpace{}%
\AgdaInductiveConstructor{mkhom}\AgdaSpace{}%
\AgdaFunction{reflˡ}\<%
\\
%
\\[\AgdaEmptyExtraSkip]%
%
\>[1]\AgdaFunction{ToFromLiftʳ}\AgdaSpace{}%
\AgdaSymbol{:}\AgdaSpace{}%
\AgdaSymbol{∀}\AgdaSpace{}%
\AgdaBound{b}\AgdaSpace{}%
\AgdaSymbol{→}\AgdaSpace{}%
\AgdaSymbol{(}\AgdaOperator{\AgdaFunction{∣}}\AgdaSpace{}%
\AgdaFunction{ToLiftʳ}\AgdaSpace{}%
\AgdaOperator{\AgdaFunction{∣}}\AgdaSpace{}%
\AgdaOperator{\AgdaField{⟨\$⟩}}\AgdaSpace{}%
\AgdaSymbol{(}\AgdaOperator{\AgdaFunction{∣}}\AgdaSpace{}%
\AgdaFunction{FromLiftʳ}\AgdaSpace{}%
\AgdaOperator{\AgdaFunction{∣}}\AgdaSpace{}%
\AgdaOperator{\AgdaField{⟨\$⟩}}\AgdaSpace{}%
\AgdaBound{b}\AgdaSymbol{))}\AgdaSpace{}%
\AgdaOperator{\AgdaFunction{≈ʳ}}\AgdaSpace{}%
\AgdaBound{b}\<%
\\
%
\>[1]\AgdaFunction{ToFromLiftʳ}\AgdaSpace{}%
\AgdaBound{b}\AgdaSpace{}%
\AgdaSymbol{=}\AgdaSpace{}%
\AgdaInductiveConstructor{lift}\AgdaSpace{}%
\AgdaFunction{refl₁}\<%
\\
%
\\[\AgdaEmptyExtraSkip]%
%
\>[1]\AgdaFunction{FromToLiftʳ}\AgdaSpace{}%
\AgdaSymbol{:}\AgdaSpace{}%
\AgdaSymbol{∀}\AgdaSpace{}%
\AgdaBound{a}\AgdaSpace{}%
\AgdaSymbol{→}\AgdaSpace{}%
\AgdaSymbol{(}\AgdaOperator{\AgdaFunction{∣}}\AgdaSpace{}%
\AgdaFunction{FromLiftʳ}\AgdaSpace{}%
\AgdaOperator{\AgdaFunction{∣}}\AgdaSpace{}%
\AgdaOperator{\AgdaField{⟨\$⟩}}\AgdaSpace{}%
\AgdaSymbol{(}\AgdaOperator{\AgdaFunction{∣}}\AgdaSpace{}%
\AgdaFunction{ToLiftʳ}\AgdaSpace{}%
\AgdaOperator{\AgdaFunction{∣}}\AgdaSpace{}%
\AgdaOperator{\AgdaField{⟨\$⟩}}\AgdaSpace{}%
\AgdaBound{a}\AgdaSymbol{))}\AgdaSpace{}%
\AgdaOperator{\AgdaFunction{≈₁}}\AgdaSpace{}%
\AgdaBound{a}\<%
\\
%
\>[1]\AgdaFunction{FromToLiftʳ}\AgdaSpace{}%
\AgdaBound{a}\AgdaSpace{}%
\AgdaSymbol{=}\AgdaSpace{}%
\AgdaFunction{refl₁}\<%
\\
%
\\[\AgdaEmptyExtraSkip]%
%
\\[\AgdaEmptyExtraSkip]%
\>[0]\AgdaKeyword{module}\AgdaSpace{}%
\AgdaModule{\AgdaUnderscore{}}\AgdaSpace{}%
\AgdaSymbol{\{}\AgdaBound{𝑨}\AgdaSpace{}%
\AgdaSymbol{:}\AgdaSpace{}%
\AgdaRecord{Algebra}\AgdaSpace{}%
\AgdaGeneralizable{α}\AgdaSpace{}%
\AgdaGeneralizable{ρᵃ}\AgdaSymbol{\}\{}\AgdaBound{ℓ}\AgdaSpace{}%
\AgdaBound{r}\AgdaSpace{}%
\AgdaSymbol{:}\AgdaSpace{}%
\AgdaPostulate{Level}\AgdaSymbol{\}}\AgdaSpace{}%
\AgdaKeyword{where}\<%
\\
\>[0][@{}l@{\AgdaIndent{0}}]%
\>[1]\AgdaKeyword{open}%
\>[7]\AgdaModule{Setoid}\AgdaSpace{}%
\AgdaOperator{\AgdaFunction{𝔻[}}\AgdaSpace{}%
\AgdaBound{𝑨}\AgdaSpace{}%
\AgdaOperator{\AgdaFunction{]}}%
\>[35]\AgdaKeyword{using}\AgdaSpace{}%
\AgdaSymbol{(}\AgdaSpace{}%
\AgdaFunction{refl}\AgdaSpace{}%
\AgdaSymbol{)}\<%
\\
%
\>[1]\AgdaKeyword{open}%
\>[7]\AgdaModule{Setoid}\AgdaSpace{}%
\AgdaOperator{\AgdaFunction{𝔻[}}\AgdaSpace{}%
\AgdaFunction{Lift-Alg}\AgdaSpace{}%
\AgdaBound{𝑨}\AgdaSpace{}%
\AgdaBound{ℓ}\AgdaSpace{}%
\AgdaBound{r}\AgdaSpace{}%
\AgdaOperator{\AgdaFunction{]}}%
\>[35]\AgdaKeyword{using}\AgdaSpace{}%
\AgdaSymbol{(}\AgdaSpace{}%
\AgdaOperator{\AgdaField{\AgdaUnderscore{}≈\AgdaUnderscore{}}}\AgdaSpace{}%
\AgdaSymbol{)}\<%
\\
%
\>[1]\AgdaKeyword{open}%
\>[7]\AgdaModule{Level}\<%
\\
%
\\[\AgdaEmptyExtraSkip]%
%
\>[1]\AgdaFunction{ToLift}\AgdaSpace{}%
\AgdaSymbol{:}\AgdaSpace{}%
\AgdaFunction{hom}\AgdaSpace{}%
\AgdaBound{𝑨}\AgdaSpace{}%
\AgdaSymbol{(}\AgdaFunction{Lift-Alg}\AgdaSpace{}%
\AgdaBound{𝑨}\AgdaSpace{}%
\AgdaBound{ℓ}\AgdaSpace{}%
\AgdaBound{r}\AgdaSymbol{)}\<%
\\
%
\>[1]\AgdaFunction{ToLift}\AgdaSpace{}%
\AgdaSymbol{=}\AgdaSpace{}%
\AgdaFunction{∘-hom}\AgdaSpace{}%
\AgdaFunction{ToLiftˡ}\AgdaSpace{}%
\AgdaFunction{ToLiftʳ}\<%
\\
%
\\[\AgdaEmptyExtraSkip]%
%
\>[1]\AgdaFunction{FromLift}\AgdaSpace{}%
\AgdaSymbol{:}\AgdaSpace{}%
\AgdaFunction{hom}\AgdaSpace{}%
\AgdaSymbol{(}\AgdaFunction{Lift-Alg}\AgdaSpace{}%
\AgdaBound{𝑨}\AgdaSpace{}%
\AgdaBound{ℓ}\AgdaSpace{}%
\AgdaBound{r}\AgdaSymbol{)}\AgdaSpace{}%
\AgdaBound{𝑨}\<%
\\
%
\>[1]\AgdaFunction{FromLift}\AgdaSpace{}%
\AgdaSymbol{=}\AgdaSpace{}%
\AgdaFunction{∘-hom}\AgdaSpace{}%
\AgdaFunction{FromLiftʳ}\AgdaSpace{}%
\AgdaFunction{FromLiftˡ}\<%
\\
%
\\[\AgdaEmptyExtraSkip]%
%
\>[1]\AgdaFunction{ToFromLift}\AgdaSpace{}%
\AgdaSymbol{:}\AgdaSpace{}%
\AgdaSymbol{∀}\AgdaSpace{}%
\AgdaBound{b}\AgdaSpace{}%
\AgdaSymbol{→}\AgdaSpace{}%
\AgdaSymbol{(}\AgdaOperator{\AgdaFunction{∣}}\AgdaSpace{}%
\AgdaFunction{ToLift}\AgdaSpace{}%
\AgdaOperator{\AgdaFunction{∣}}\AgdaSpace{}%
\AgdaOperator{\AgdaField{⟨\$⟩}}\AgdaSpace{}%
\AgdaSymbol{(}\AgdaOperator{\AgdaFunction{∣}}\AgdaSpace{}%
\AgdaFunction{FromLift}\AgdaSpace{}%
\AgdaOperator{\AgdaFunction{∣}}\AgdaSpace{}%
\AgdaOperator{\AgdaField{⟨\$⟩}}\AgdaSpace{}%
\AgdaBound{b}\AgdaSymbol{))}\AgdaSpace{}%
\AgdaOperator{\AgdaFunction{≈}}\AgdaSpace{}%
\AgdaBound{b}\<%
\\
%
\>[1]\AgdaFunction{ToFromLift}\AgdaSpace{}%
\AgdaBound{b}\AgdaSpace{}%
\AgdaSymbol{=}\AgdaSpace{}%
\AgdaInductiveConstructor{lift}\AgdaSpace{}%
\AgdaFunction{refl}\<%
\\
%
\\[\AgdaEmptyExtraSkip]%
%
\>[1]\AgdaFunction{ToLift-epi}\AgdaSpace{}%
\AgdaSymbol{:}\AgdaSpace{}%
\AgdaFunction{epi}\AgdaSpace{}%
\AgdaBound{𝑨}\AgdaSpace{}%
\AgdaSymbol{(}\AgdaFunction{Lift-Alg}\AgdaSpace{}%
\AgdaBound{𝑨}\AgdaSpace{}%
\AgdaBound{ℓ}\AgdaSpace{}%
\AgdaBound{r}\AgdaSymbol{)}\<%
\\
%
\>[1]\AgdaFunction{ToLift-epi}\AgdaSpace{}%
\AgdaSymbol{=}\AgdaSpace{}%
\AgdaOperator{\AgdaFunction{∣}}\AgdaSpace{}%
\AgdaFunction{ToLift}\AgdaSpace{}%
\AgdaOperator{\AgdaFunction{∣}}\AgdaSpace{}%
\AgdaOperator{\AgdaInductiveConstructor{,}}%
\>[28]\AgdaKeyword{record}\AgdaSpace{}%
\AgdaSymbol{\{}\AgdaSpace{}%
\AgdaField{isHom}\AgdaSpace{}%
\AgdaSymbol{=}\AgdaSpace{}%
\AgdaOperator{\AgdaFunction{∥}}\AgdaSpace{}%
\AgdaFunction{ToLift}\AgdaSpace{}%
\AgdaOperator{\AgdaFunction{∥}}\<%
\\
%
\>[28]\AgdaSymbol{;}\AgdaSpace{}%
\AgdaField{isSurjective}\AgdaSpace{}%
\AgdaSymbol{=}\AgdaSpace{}%
\AgdaSymbol{λ}\AgdaSpace{}%
\AgdaSymbol{\{}\AgdaBound{y}\AgdaSymbol{\}}\AgdaSpace{}%
\AgdaSymbol{→}\AgdaSpace{}%
\AgdaInductiveConstructor{eq}\AgdaSpace{}%
\AgdaSymbol{(}\AgdaOperator{\AgdaFunction{∣}}\AgdaSpace{}%
\AgdaFunction{FromLift}\AgdaSpace{}%
\AgdaOperator{\AgdaFunction{∣}}\AgdaSpace{}%
\AgdaOperator{\AgdaField{⟨\$⟩}}\AgdaSpace{}%
\AgdaBound{y}\AgdaSymbol{)}\AgdaSpace{}%
\AgdaSymbol{(}\AgdaFunction{ToFromLift}\AgdaSpace{}%
\AgdaBound{y}\AgdaSymbol{)}\AgdaSpace{}%
\AgdaSymbol{\}}\<%
\end{code}

\begin{code}[hide]%
\>[0]\<%
\\
\>[0]\AgdaKeyword{module}\AgdaSpace{}%
\AgdaModule{\AgdaUnderscore{}}\AgdaSpace{}%
\AgdaSymbol{\{}\AgdaBound{ι}\AgdaSpace{}%
\AgdaSymbol{:}\AgdaSpace{}%
\AgdaPostulate{Level}\AgdaSymbol{\}\{}\AgdaBound{I}\AgdaSpace{}%
\AgdaSymbol{:}\AgdaSpace{}%
\AgdaPrimitive{Type}\AgdaSpace{}%
\AgdaBound{ι}\AgdaSymbol{\}\{}\AgdaBound{𝑨}\AgdaSpace{}%
\AgdaSymbol{:}\AgdaSpace{}%
\AgdaRecord{Algebra}\AgdaSpace{}%
\AgdaGeneralizable{α}\AgdaSpace{}%
\AgdaGeneralizable{ρᵃ}\AgdaSymbol{\}(}\AgdaBound{ℬ}\AgdaSpace{}%
\AgdaSymbol{:}\AgdaSpace{}%
\AgdaBound{I}\AgdaSpace{}%
\AgdaSymbol{→}\AgdaSpace{}%
\AgdaRecord{Algebra}\AgdaSpace{}%
\AgdaGeneralizable{β}\AgdaSpace{}%
\AgdaGeneralizable{ρᵇ}\AgdaSymbol{)}%
\>[74]\AgdaKeyword{where}\<%
\\
\>[0][@{}l@{\AgdaIndent{0}}]%
\>[1]\AgdaFunction{⨅-hom-co}\AgdaSpace{}%
\AgdaSymbol{:}\AgdaSpace{}%
\AgdaSymbol{(∀(}\AgdaBound{i}\AgdaSpace{}%
\AgdaSymbol{:}\AgdaSpace{}%
\AgdaBound{I}\AgdaSymbol{)}\AgdaSpace{}%
\AgdaSymbol{→}\AgdaSpace{}%
\AgdaFunction{hom}\AgdaSpace{}%
\AgdaBound{𝑨}\AgdaSpace{}%
\AgdaSymbol{(}\AgdaBound{ℬ}\AgdaSpace{}%
\AgdaBound{i}\AgdaSymbol{))}\AgdaSpace{}%
\AgdaSymbol{→}\AgdaSpace{}%
\AgdaFunction{hom}\AgdaSpace{}%
\AgdaBound{𝑨}\AgdaSpace{}%
\AgdaSymbol{(}\AgdaFunction{⨅}\AgdaSpace{}%
\AgdaBound{ℬ}\AgdaSymbol{)}\<%
\\
%
\>[1]\AgdaFunction{⨅-hom-co}\AgdaSpace{}%
\AgdaBound{𝒽}\AgdaSpace{}%
\AgdaSymbol{=}\AgdaSpace{}%
\AgdaFunction{h}\AgdaSpace{}%
\AgdaOperator{\AgdaInductiveConstructor{,}}\AgdaSpace{}%
\AgdaFunction{hhom}\<%
\\
\>[1][@{}l@{\AgdaIndent{0}}]%
\>[2]\AgdaKeyword{where}\<%
\\
%
\>[2]\AgdaFunction{h}\AgdaSpace{}%
\AgdaSymbol{:}\AgdaSpace{}%
\AgdaOperator{\AgdaFunction{𝔻[}}\AgdaSpace{}%
\AgdaBound{𝑨}\AgdaSpace{}%
\AgdaOperator{\AgdaFunction{]}}\AgdaSpace{}%
\AgdaOperator{\AgdaRecord{⟶}}\AgdaSpace{}%
\AgdaOperator{\AgdaFunction{𝔻[}}\AgdaSpace{}%
\AgdaFunction{⨅}\AgdaSpace{}%
\AgdaBound{ℬ}\AgdaSpace{}%
\AgdaOperator{\AgdaFunction{]}}\<%
\\
%
\>[2]\AgdaOperator{\AgdaField{\AgdaUnderscore{}⟨\$⟩\AgdaUnderscore{}}}\AgdaSpace{}%
\AgdaFunction{h}\AgdaSpace{}%
\AgdaSymbol{=}\AgdaSpace{}%
\AgdaSymbol{λ}\AgdaSpace{}%
\AgdaBound{a}\AgdaSpace{}%
\AgdaBound{i}\AgdaSpace{}%
\AgdaSymbol{→}\AgdaSpace{}%
\AgdaOperator{\AgdaFunction{∣}}\AgdaSpace{}%
\AgdaBound{𝒽}\AgdaSpace{}%
\AgdaBound{i}\AgdaSpace{}%
\AgdaOperator{\AgdaFunction{∣}}\AgdaSpace{}%
\AgdaOperator{\AgdaField{⟨\$⟩}}\AgdaSpace{}%
\AgdaBound{a}\<%
\\
%
\>[2]\AgdaField{cong}\AgdaSpace{}%
\AgdaFunction{h}\AgdaSpace{}%
\AgdaBound{xy}\AgdaSpace{}%
\AgdaBound{i}\AgdaSpace{}%
\AgdaSymbol{=}\AgdaSpace{}%
\AgdaField{cong}\AgdaSpace{}%
\AgdaOperator{\AgdaFunction{∣}}\AgdaSpace{}%
\AgdaBound{𝒽}\AgdaSpace{}%
\AgdaBound{i}\AgdaSpace{}%
\AgdaOperator{\AgdaFunction{∣}}\AgdaSpace{}%
\AgdaBound{xy}\<%
\\
%
\>[2]\AgdaFunction{hhom}\AgdaSpace{}%
\AgdaSymbol{:}\AgdaSpace{}%
\AgdaRecord{IsHom}\AgdaSpace{}%
\AgdaBound{𝑨}\AgdaSpace{}%
\AgdaSymbol{(}\AgdaFunction{⨅}\AgdaSpace{}%
\AgdaBound{ℬ}\AgdaSymbol{)}\AgdaSpace{}%
\AgdaFunction{h}\<%
\\
%
\>[2]\AgdaField{compatible}\AgdaSpace{}%
\AgdaFunction{hhom}\AgdaSpace{}%
\AgdaSymbol{=}\AgdaSpace{}%
\AgdaSymbol{λ}\AgdaSpace{}%
\AgdaBound{i}\AgdaSpace{}%
\AgdaSymbol{→}\AgdaSpace{}%
\AgdaField{compatible}\AgdaSpace{}%
\AgdaOperator{\AgdaFunction{∥}}\AgdaSpace{}%
\AgdaBound{𝒽}\AgdaSpace{}%
\AgdaBound{i}\AgdaSpace{}%
\AgdaOperator{\AgdaFunction{∥}}\<%
\end{code}

\hypertarget{factorization-of-homomorphisms}{%
\subsubsection{Factorization of
homomorphisms}\label{factorization-of-homomorphisms}}

If \texttt{g\ :\ hom\ 𝑨\ 𝑩}, \texttt{h\ :\ hom\ 𝑨\ 𝑪}, \texttt{h} is
surjective, and \texttt{ker\ h\ ⊆\ ker\ g}, then there exists
\texttt{φ\ :\ hom\ 𝑪\ 𝑩} such that \texttt{g\ =\ φ\ ∘\ h} (cf.~the
\href{https://ualib.github.io/agda-algebras/Homomorphisms.Func.Factor.html}{Homomorphisms.Func.Factor}
module of the \href{https://ualib.github.io/agda-algebras}{Agda
Universal Algebra Library}).

\begin{code}%
\>[0]\<%
\\
\>[0]\AgdaKeyword{module}\AgdaSpace{}%
\AgdaModule{\AgdaUnderscore{}}%
\>[2154I]\AgdaSymbol{\{}\AgdaBound{𝑨}\AgdaSpace{}%
\AgdaSymbol{:}\AgdaSpace{}%
\AgdaRecord{Algebra}\AgdaSpace{}%
\AgdaGeneralizable{α}\AgdaSpace{}%
\AgdaGeneralizable{ρᵃ}\AgdaSymbol{\}(}\AgdaBound{𝑩}\AgdaSpace{}%
\AgdaSymbol{:}\AgdaSpace{}%
\AgdaRecord{Algebra}\AgdaSpace{}%
\AgdaGeneralizable{β}\AgdaSpace{}%
\AgdaGeneralizable{ρᵇ}\AgdaSymbol{)\{}\AgdaBound{𝑪}\AgdaSpace{}%
\AgdaSymbol{:}\AgdaSpace{}%
\AgdaRecord{Algebra}\AgdaSpace{}%
\AgdaGeneralizable{γ}\AgdaSpace{}%
\AgdaGeneralizable{ρᶜ}\AgdaSymbol{\}}\<%
\\
\>[.][@{}l@{}]\<[2154I]%
\>[9]\AgdaSymbol{(}\AgdaBound{gh}\AgdaSpace{}%
\AgdaSymbol{:}\AgdaSpace{}%
\AgdaFunction{hom}\AgdaSpace{}%
\AgdaBound{𝑨}\AgdaSpace{}%
\AgdaBound{𝑩}\AgdaSymbol{)(}\AgdaBound{hh}\AgdaSpace{}%
\AgdaSymbol{:}\AgdaSpace{}%
\AgdaFunction{hom}\AgdaSpace{}%
\AgdaBound{𝑨}\AgdaSpace{}%
\AgdaBound{𝑪}\AgdaSymbol{)}\AgdaSpace{}%
\AgdaKeyword{where}\<%
\\
\>[0][@{}l@{\AgdaIndent{0}}]%
\>[1]\AgdaKeyword{open}\AgdaSpace{}%
\AgdaModule{Setoid}\AgdaSpace{}%
\AgdaOperator{\AgdaFunction{𝔻[}}\AgdaSpace{}%
\AgdaBound{𝑩}\AgdaSpace{}%
\AgdaOperator{\AgdaFunction{]}}\AgdaSpace{}%
\AgdaKeyword{using}\AgdaSpace{}%
\AgdaSymbol{()}\AgdaSpace{}%
\AgdaKeyword{renaming}\AgdaSpace{}%
\AgdaSymbol{(}\AgdaSpace{}%
\AgdaOperator{\AgdaField{\AgdaUnderscore{}≈\AgdaUnderscore{}}}\AgdaSpace{}%
\AgdaSymbol{to}\AgdaSpace{}%
\AgdaOperator{\AgdaField{\AgdaUnderscore{}≈₂\AgdaUnderscore{}}}\AgdaSpace{}%
\AgdaSymbol{;}\AgdaSpace{}%
\AgdaFunction{sym}\AgdaSpace{}%
\AgdaSymbol{to}\AgdaSpace{}%
\AgdaFunction{sym₂}\AgdaSpace{}%
\AgdaSymbol{)}\<%
\\
%
\>[1]\AgdaKeyword{open}\AgdaSpace{}%
\AgdaModule{Setoid}\AgdaSpace{}%
\AgdaOperator{\AgdaFunction{𝔻[}}\AgdaSpace{}%
\AgdaBound{𝑪}\AgdaSpace{}%
\AgdaOperator{\AgdaFunction{]}}\AgdaSpace{}%
\AgdaKeyword{using}\AgdaSpace{}%
\AgdaSymbol{(}\AgdaSpace{}%
\AgdaFunction{trans}\AgdaSpace{}%
\AgdaSymbol{)}\AgdaSpace{}%
\AgdaKeyword{renaming}\AgdaSpace{}%
\AgdaSymbol{(}\AgdaSpace{}%
\AgdaOperator{\AgdaField{\AgdaUnderscore{}≈\AgdaUnderscore{}}}\AgdaSpace{}%
\AgdaSymbol{to}\AgdaSpace{}%
\AgdaOperator{\AgdaField{\AgdaUnderscore{}≈₃\AgdaUnderscore{}}}\AgdaSpace{}%
\AgdaSymbol{;}\AgdaSpace{}%
\AgdaFunction{sym}\AgdaSpace{}%
\AgdaSymbol{to}\AgdaSpace{}%
\AgdaFunction{sym₃}\AgdaSpace{}%
\AgdaSymbol{)}\<%
\\
%
\>[1]\AgdaKeyword{open}\AgdaSpace{}%
\AgdaModule{SetoidReasoning}\AgdaSpace{}%
\AgdaOperator{\AgdaFunction{𝔻[}}\AgdaSpace{}%
\AgdaBound{𝑩}\AgdaSpace{}%
\AgdaOperator{\AgdaFunction{]}}\<%
\\
%
\>[1]\AgdaKeyword{private}\AgdaSpace{}%
\AgdaFunction{gfunc}\AgdaSpace{}%
\AgdaSymbol{=}\AgdaSpace{}%
\AgdaOperator{\AgdaFunction{∣}}\AgdaSpace{}%
\AgdaBound{gh}\AgdaSpace{}%
\AgdaOperator{\AgdaFunction{∣}}\AgdaSpace{}%
\AgdaSymbol{;}\AgdaSpace{}%
\AgdaFunction{g}\AgdaSpace{}%
\AgdaSymbol{=}\AgdaSpace{}%
\AgdaOperator{\AgdaField{\AgdaUnderscore{}⟨\$⟩\AgdaUnderscore{}}}\AgdaSpace{}%
\AgdaFunction{gfunc}\AgdaSpace{}%
\AgdaSymbol{;}\AgdaSpace{}%
\AgdaFunction{hfunc}\AgdaSpace{}%
\AgdaSymbol{=}\AgdaSpace{}%
\AgdaOperator{\AgdaFunction{∣}}\AgdaSpace{}%
\AgdaBound{hh}\AgdaSpace{}%
\AgdaOperator{\AgdaFunction{∣}}\AgdaSpace{}%
\AgdaSymbol{;}\AgdaSpace{}%
\AgdaFunction{h}\AgdaSpace{}%
\AgdaSymbol{=}\AgdaSpace{}%
\AgdaOperator{\AgdaField{\AgdaUnderscore{}⟨\$⟩\AgdaUnderscore{}}}\AgdaSpace{}%
\AgdaFunction{hfunc}\<%
\\
%
\\[\AgdaEmptyExtraSkip]%
%
\>[1]\AgdaFunction{HomFactor}\AgdaSpace{}%
\AgdaSymbol{:}%
\>[14]\AgdaFunction{kernel}\AgdaSpace{}%
\AgdaOperator{\AgdaFunction{\AgdaUnderscore{}≈₃\AgdaUnderscore{}}}\AgdaSpace{}%
\AgdaFunction{h}\AgdaSpace{}%
\AgdaOperator{\AgdaFunction{⊆}}\AgdaSpace{}%
\AgdaFunction{kernel}\AgdaSpace{}%
\AgdaOperator{\AgdaFunction{\AgdaUnderscore{}≈₂\AgdaUnderscore{}}}\AgdaSpace{}%
\AgdaFunction{g}\AgdaSpace{}%
\AgdaSymbol{→}\AgdaSpace{}%
\AgdaFunction{IsSurjective}\AgdaSpace{}%
\AgdaFunction{hfunc}\<%
\\
\>[1][@{}l@{\AgdaIndent{0}}]%
\>[2]\AgdaSymbol{→}%
\>[14]\AgdaFunction{Σ[}\AgdaSpace{}%
\AgdaBound{φ}\AgdaSpace{}%
\AgdaFunction{∈}\AgdaSpace{}%
\AgdaFunction{hom}\AgdaSpace{}%
\AgdaBound{𝑪}\AgdaSpace{}%
\AgdaBound{𝑩}\AgdaSpace{}%
\AgdaFunction{]}\AgdaSpace{}%
\AgdaSymbol{∀}\AgdaSpace{}%
\AgdaBound{a}\AgdaSpace{}%
\AgdaSymbol{→}\AgdaSpace{}%
\AgdaSymbol{(}\AgdaFunction{g}\AgdaSpace{}%
\AgdaBound{a}\AgdaSymbol{)}\AgdaSpace{}%
\AgdaOperator{\AgdaFunction{≈₂}}\AgdaSpace{}%
\AgdaOperator{\AgdaFunction{∣}}\AgdaSpace{}%
\AgdaBound{φ}\AgdaSpace{}%
\AgdaOperator{\AgdaFunction{∣}}\AgdaSpace{}%
\AgdaOperator{\AgdaField{⟨\$⟩}}\AgdaSpace{}%
\AgdaSymbol{(}\AgdaFunction{h}\AgdaSpace{}%
\AgdaBound{a}\AgdaSymbol{)}\<%
\\
%
\\[\AgdaEmptyExtraSkip]%
%
\>[1]\AgdaFunction{HomFactor}\AgdaSpace{}%
\AgdaBound{Khg}\AgdaSpace{}%
\AgdaBound{hE}\AgdaSpace{}%
\AgdaSymbol{=}\AgdaSpace{}%
\AgdaSymbol{(}\AgdaFunction{φmap}\AgdaSpace{}%
\AgdaOperator{\AgdaInductiveConstructor{,}}\AgdaSpace{}%
\AgdaFunction{φhom}\AgdaSymbol{)}\AgdaSpace{}%
\AgdaOperator{\AgdaInductiveConstructor{,}}\AgdaSpace{}%
\AgdaFunction{gφh}\<%
\\
\>[1][@{}l@{\AgdaIndent{0}}]%
\>[2]\AgdaKeyword{where}\<%
\\
%
\>[2]\AgdaFunction{kerpres}\AgdaSpace{}%
\AgdaSymbol{:}\AgdaSpace{}%
\AgdaSymbol{∀}\AgdaSpace{}%
\AgdaBound{a₀}\AgdaSpace{}%
\AgdaBound{a₁}\AgdaSpace{}%
\AgdaSymbol{→}\AgdaSpace{}%
\AgdaFunction{h}\AgdaSpace{}%
\AgdaBound{a₀}\AgdaSpace{}%
\AgdaOperator{\AgdaFunction{≈₃}}\AgdaSpace{}%
\AgdaFunction{h}\AgdaSpace{}%
\AgdaBound{a₁}\AgdaSpace{}%
\AgdaSymbol{→}\AgdaSpace{}%
\AgdaFunction{g}\AgdaSpace{}%
\AgdaBound{a₀}\AgdaSpace{}%
\AgdaOperator{\AgdaFunction{≈₂}}\AgdaSpace{}%
\AgdaFunction{g}\AgdaSpace{}%
\AgdaBound{a₁}\<%
\\
%
\>[2]\AgdaFunction{kerpres}\AgdaSpace{}%
\AgdaBound{a₀}\AgdaSpace{}%
\AgdaBound{a₁}\AgdaSpace{}%
\AgdaBound{hyp}\AgdaSpace{}%
\AgdaSymbol{=}\AgdaSpace{}%
\AgdaBound{Khg}\AgdaSpace{}%
\AgdaBound{hyp}\<%
\\
%
\\[\AgdaEmptyExtraSkip]%
%
\>[2]\AgdaFunction{h⁻¹}\AgdaSpace{}%
\AgdaSymbol{:}\AgdaSpace{}%
\AgdaOperator{\AgdaFunction{𝕌[}}\AgdaSpace{}%
\AgdaBound{𝑪}\AgdaSpace{}%
\AgdaOperator{\AgdaFunction{]}}\AgdaSpace{}%
\AgdaSymbol{→}\AgdaSpace{}%
\AgdaOperator{\AgdaFunction{𝕌[}}\AgdaSpace{}%
\AgdaBound{𝑨}\AgdaSpace{}%
\AgdaOperator{\AgdaFunction{]}}\<%
\\
%
\>[2]\AgdaFunction{h⁻¹}\AgdaSpace{}%
\AgdaSymbol{=}\AgdaSpace{}%
\AgdaFunction{SurjInv}\AgdaSpace{}%
\AgdaFunction{hfunc}\AgdaSpace{}%
\AgdaBound{hE}\<%
\\
%
\\[\AgdaEmptyExtraSkip]%
%
\>[2]\AgdaFunction{η}\AgdaSpace{}%
\AgdaSymbol{:}\AgdaSpace{}%
\AgdaSymbol{∀}\AgdaSpace{}%
\AgdaSymbol{\{}\AgdaBound{c}\AgdaSymbol{\}}\AgdaSpace{}%
\AgdaSymbol{→}\AgdaSpace{}%
\AgdaFunction{h}\AgdaSpace{}%
\AgdaSymbol{(}\AgdaFunction{h⁻¹}\AgdaSpace{}%
\AgdaBound{c}\AgdaSymbol{)}\AgdaSpace{}%
\AgdaOperator{\AgdaFunction{≈₃}}\AgdaSpace{}%
\AgdaBound{c}\<%
\\
%
\>[2]\AgdaFunction{η}\AgdaSpace{}%
\AgdaSymbol{=}\AgdaSpace{}%
\AgdaFunction{InvIsInverseʳ}\AgdaSpace{}%
\AgdaBound{hE}\<%
\\
%
\\[\AgdaEmptyExtraSkip]%
%
\>[2]\AgdaFunction{ζ}\AgdaSpace{}%
\AgdaSymbol{:}\AgdaSpace{}%
\AgdaSymbol{∀\{}\AgdaBound{x}\AgdaSpace{}%
\AgdaBound{y}\AgdaSymbol{\}}\AgdaSpace{}%
\AgdaSymbol{→}\AgdaSpace{}%
\AgdaBound{x}\AgdaSpace{}%
\AgdaOperator{\AgdaFunction{≈₃}}\AgdaSpace{}%
\AgdaBound{y}\AgdaSpace{}%
\AgdaSymbol{→}\AgdaSpace{}%
\AgdaFunction{h}\AgdaSpace{}%
\AgdaSymbol{(}\AgdaFunction{h⁻¹}\AgdaSpace{}%
\AgdaBound{x}\AgdaSymbol{)}\AgdaSpace{}%
\AgdaOperator{\AgdaFunction{≈₃}}\AgdaSpace{}%
\AgdaFunction{h}\AgdaSpace{}%
\AgdaSymbol{(}\AgdaFunction{h⁻¹}\AgdaSpace{}%
\AgdaBound{y}\AgdaSymbol{)}\<%
\\
%
\>[2]\AgdaFunction{ζ}\AgdaSpace{}%
\AgdaBound{xy}\AgdaSpace{}%
\AgdaSymbol{=}\AgdaSpace{}%
\AgdaFunction{trans}\AgdaSpace{}%
\AgdaFunction{η}\AgdaSpace{}%
\AgdaSymbol{(}\AgdaFunction{trans}\AgdaSpace{}%
\AgdaBound{xy}\AgdaSpace{}%
\AgdaSymbol{(}\AgdaFunction{sym₃}\AgdaSpace{}%
\AgdaFunction{η}\AgdaSymbol{))}\<%
\\
%
\\[\AgdaEmptyExtraSkip]%
%
\>[2]\AgdaFunction{φmap}\AgdaSpace{}%
\AgdaSymbol{:}\AgdaSpace{}%
\AgdaOperator{\AgdaFunction{𝔻[}}\AgdaSpace{}%
\AgdaBound{𝑪}\AgdaSpace{}%
\AgdaOperator{\AgdaFunction{]}}\AgdaSpace{}%
\AgdaOperator{\AgdaRecord{⟶}}\AgdaSpace{}%
\AgdaOperator{\AgdaFunction{𝔻[}}\AgdaSpace{}%
\AgdaBound{𝑩}\AgdaSpace{}%
\AgdaOperator{\AgdaFunction{]}}\<%
\\
%
\>[2]\AgdaOperator{\AgdaField{\AgdaUnderscore{}⟨\$⟩\AgdaUnderscore{}}}\AgdaSpace{}%
\AgdaFunction{φmap}\AgdaSpace{}%
\AgdaSymbol{=}\AgdaSpace{}%
\AgdaFunction{g}\AgdaSpace{}%
\AgdaOperator{\AgdaFunction{∘}}\AgdaSpace{}%
\AgdaFunction{h⁻¹}\<%
\\
%
\>[2]\AgdaField{cong}\AgdaSpace{}%
\AgdaFunction{φmap}\AgdaSpace{}%
\AgdaSymbol{=}\AgdaSpace{}%
\AgdaBound{Khg}\AgdaSpace{}%
\AgdaOperator{\AgdaFunction{∘}}\AgdaSpace{}%
\AgdaFunction{ζ}\<%
\\
%
\>[2]\AgdaKeyword{open}\AgdaSpace{}%
\AgdaModule{\AgdaUnderscore{}⟶\AgdaUnderscore{}}\AgdaSpace{}%
\AgdaFunction{φmap}\AgdaSpace{}%
\AgdaKeyword{using}\AgdaSpace{}%
\AgdaSymbol{()}\AgdaSpace{}%
\AgdaKeyword{renaming}\AgdaSpace{}%
\AgdaSymbol{(}\AgdaField{cong}\AgdaSpace{}%
\AgdaSymbol{to}\AgdaSpace{}%
\AgdaField{φcong}\AgdaSymbol{)}\<%
\\
%
\\[\AgdaEmptyExtraSkip]%
%
\>[2]\AgdaFunction{gφh}\AgdaSpace{}%
\AgdaSymbol{:}\AgdaSpace{}%
\AgdaSymbol{(}\AgdaBound{a}\AgdaSpace{}%
\AgdaSymbol{:}\AgdaSpace{}%
\AgdaOperator{\AgdaFunction{𝕌[}}\AgdaSpace{}%
\AgdaBound{𝑨}\AgdaSpace{}%
\AgdaOperator{\AgdaFunction{]}}\AgdaSymbol{)}\AgdaSpace{}%
\AgdaSymbol{→}\AgdaSpace{}%
\AgdaFunction{g}\AgdaSpace{}%
\AgdaBound{a}\AgdaSpace{}%
\AgdaOperator{\AgdaFunction{≈₂}}\AgdaSpace{}%
\AgdaFunction{φmap}\AgdaSpace{}%
\AgdaOperator{\AgdaField{⟨\$⟩}}\AgdaSpace{}%
\AgdaSymbol{(}\AgdaFunction{h}\AgdaSpace{}%
\AgdaBound{a}\AgdaSymbol{)}\<%
\\
%
\>[2]\AgdaFunction{gφh}\AgdaSpace{}%
\AgdaBound{a}\AgdaSpace{}%
\AgdaSymbol{=}\AgdaSpace{}%
\AgdaBound{Khg}\AgdaSpace{}%
\AgdaSymbol{(}\AgdaFunction{sym₃}\AgdaSpace{}%
\AgdaFunction{η}\AgdaSymbol{)}\<%
\\
%
\\[\AgdaEmptyExtraSkip]%
%
\>[2]\AgdaFunction{φcomp}\AgdaSpace{}%
\AgdaSymbol{:}\AgdaSpace{}%
\AgdaFunction{compatible-map}\AgdaSpace{}%
\AgdaBound{𝑪}\AgdaSpace{}%
\AgdaBound{𝑩}\AgdaSpace{}%
\AgdaFunction{φmap}\<%
\\
%
\>[2]\AgdaFunction{φcomp}\AgdaSpace{}%
\AgdaSymbol{\{}\AgdaBound{f}\AgdaSymbol{\}\{}\AgdaBound{c}\AgdaSymbol{\}}\AgdaSpace{}%
\AgdaSymbol{=}\<%
\\
\>[2][@{}l@{\AgdaIndent{0}}]%
\>[3]\AgdaOperator{\AgdaFunction{begin}}\<%
\\
\>[3][@{}l@{\AgdaIndent{0}}]%
\>[4]\AgdaFunction{φmap}\AgdaSpace{}%
\AgdaOperator{\AgdaField{⟨\$⟩}}\AgdaSpace{}%
\AgdaSymbol{((}\AgdaBound{f}\AgdaSpace{}%
\AgdaOperator{\AgdaFunction{̂}}\AgdaSpace{}%
\AgdaBound{𝑪}\AgdaSymbol{)}\AgdaSpace{}%
\AgdaBound{c}\AgdaSymbol{)}%
\>[36]\AgdaFunction{≈˘⟨}%
\>[41]\AgdaFunction{φcong}\AgdaSpace{}%
\AgdaSymbol{(}\AgdaField{cong}\AgdaSpace{}%
\AgdaSymbol{(}\AgdaField{Interp}\AgdaSpace{}%
\AgdaBound{𝑪}\AgdaSymbol{)}\AgdaSpace{}%
\AgdaSymbol{(}\AgdaInductiveConstructor{≡.refl}\AgdaSpace{}%
\AgdaOperator{\AgdaInductiveConstructor{,}}\AgdaSpace{}%
\AgdaSymbol{λ}\AgdaSpace{}%
\AgdaBound{\AgdaUnderscore{}}\AgdaSpace{}%
\AgdaSymbol{→}\AgdaSpace{}%
\AgdaFunction{η}\AgdaSymbol{))}%
\>[85]\AgdaFunction{⟩}\<%
\\
%
\>[4]\AgdaFunction{g}\AgdaSpace{}%
\AgdaSymbol{(}\AgdaFunction{h⁻¹}\AgdaSpace{}%
\AgdaSymbol{((}\AgdaBound{f}\AgdaSpace{}%
\AgdaOperator{\AgdaFunction{̂}}\AgdaSpace{}%
\AgdaBound{𝑪}\AgdaSymbol{)(}\AgdaFunction{h}\AgdaSpace{}%
\AgdaOperator{\AgdaFunction{∘}}\AgdaSpace{}%
\AgdaFunction{h⁻¹}\AgdaSpace{}%
\AgdaOperator{\AgdaFunction{∘}}\AgdaSpace{}%
\AgdaBound{c}\AgdaSymbol{)))}%
\>[36]\AgdaFunction{≈˘⟨}%
\>[41]\AgdaFunction{φcong}\AgdaSpace{}%
\AgdaSymbol{(}\AgdaField{compatible}\AgdaSpace{}%
\AgdaOperator{\AgdaFunction{∥}}\AgdaSpace{}%
\AgdaBound{hh}\AgdaSpace{}%
\AgdaOperator{\AgdaFunction{∥}}\AgdaSymbol{)}%
\>[85]\AgdaFunction{⟩}\<%
\\
%
\>[4]\AgdaFunction{g}\AgdaSpace{}%
\AgdaSymbol{(}\AgdaFunction{h⁻¹}\AgdaSpace{}%
\AgdaSymbol{(}\AgdaFunction{h}\AgdaSpace{}%
\AgdaSymbol{((}\AgdaBound{f}\AgdaSpace{}%
\AgdaOperator{\AgdaFunction{̂}}\AgdaSpace{}%
\AgdaBound{𝑨}\AgdaSymbol{)(}\AgdaFunction{h⁻¹}\AgdaSpace{}%
\AgdaOperator{\AgdaFunction{∘}}\AgdaSpace{}%
\AgdaBound{c}\AgdaSymbol{))))}%
\>[36]\AgdaFunction{≈˘⟨}%
\>[41]\AgdaFunction{gφh}\AgdaSpace{}%
\AgdaSymbol{((}\AgdaBound{f}\AgdaSpace{}%
\AgdaOperator{\AgdaFunction{̂}}\AgdaSpace{}%
\AgdaBound{𝑨}\AgdaSymbol{)(}\AgdaFunction{h⁻¹}\AgdaSpace{}%
\AgdaOperator{\AgdaFunction{∘}}\AgdaSpace{}%
\AgdaBound{c}\AgdaSymbol{))}%
\>[85]\AgdaFunction{⟩}\<%
\\
%
\>[4]\AgdaFunction{g}\AgdaSpace{}%
\AgdaSymbol{((}\AgdaBound{f}\AgdaSpace{}%
\AgdaOperator{\AgdaFunction{̂}}\AgdaSpace{}%
\AgdaBound{𝑨}\AgdaSymbol{)(}\AgdaFunction{h⁻¹}\AgdaSpace{}%
\AgdaOperator{\AgdaFunction{∘}}\AgdaSpace{}%
\AgdaBound{c}\AgdaSymbol{))}%
\>[36]\AgdaFunction{≈⟨}%
\>[41]\AgdaField{compatible}\AgdaSpace{}%
\AgdaOperator{\AgdaFunction{∥}}\AgdaSpace{}%
\AgdaBound{gh}\AgdaSpace{}%
\AgdaOperator{\AgdaFunction{∥}}%
\>[85]\AgdaFunction{⟩}\<%
\\
%
\>[4]\AgdaSymbol{(}\AgdaBound{f}\AgdaSpace{}%
\AgdaOperator{\AgdaFunction{̂}}\AgdaSpace{}%
\AgdaBound{𝑩}\AgdaSymbol{)(}\AgdaFunction{g}\AgdaSpace{}%
\AgdaOperator{\AgdaFunction{∘}}\AgdaSpace{}%
\AgdaSymbol{(}\AgdaFunction{h⁻¹}\AgdaSpace{}%
\AgdaOperator{\AgdaFunction{∘}}\AgdaSpace{}%
\AgdaBound{c}\AgdaSymbol{))}%
\>[36]\AgdaOperator{\AgdaFunction{∎}}\<%
\\
%
\\[\AgdaEmptyExtraSkip]%
%
\>[2]\AgdaFunction{φhom}\AgdaSpace{}%
\AgdaSymbol{:}\AgdaSpace{}%
\AgdaRecord{IsHom}\AgdaSpace{}%
\AgdaBound{𝑪}\AgdaSpace{}%
\AgdaBound{𝑩}\AgdaSpace{}%
\AgdaFunction{φmap}\<%
\\
%
\>[2]\AgdaField{compatible}\AgdaSpace{}%
\AgdaFunction{φhom}\AgdaSpace{}%
\AgdaSymbol{=}\AgdaSpace{}%
\AgdaFunction{φcomp}\<%
\end{code}

\hypertarget{isomorphisms}{%
\subsubsection{Isomorphisms}\label{isomorphisms}}

(cf.~the
\href{https://ualib.github.io/agda-algebras/Homomorphisms.Func.Isomorphisms.html}{Homomorphisms.Func.Isomorphisms}
of the \href{https://ualib.github.io/agda-algebras}{Agda Universal
Algebra Library}.)

Two structures are \emph{isomorphic} provided there are homomorphisms
going back and forth between them which compose to the identity map.

\begin{code}%
\>[0]\<%
\\
\>[0]\AgdaKeyword{module}\AgdaSpace{}%
\AgdaModule{\AgdaUnderscore{}}\AgdaSpace{}%
\AgdaSymbol{(}\AgdaBound{𝑨}\AgdaSpace{}%
\AgdaSymbol{:}\AgdaSpace{}%
\AgdaRecord{Algebra}\AgdaSpace{}%
\AgdaGeneralizable{α}\AgdaSpace{}%
\AgdaGeneralizable{ρᵃ}\AgdaSymbol{)}\AgdaSpace{}%
\AgdaSymbol{(}\AgdaBound{𝑩}\AgdaSpace{}%
\AgdaSymbol{:}\AgdaSpace{}%
\AgdaRecord{Algebra}\AgdaSpace{}%
\AgdaGeneralizable{β}\AgdaSpace{}%
\AgdaGeneralizable{ρᵇ}\AgdaSymbol{)}\AgdaSpace{}%
\AgdaKeyword{where}\<%
\\
\>[0][@{}l@{\AgdaIndent{0}}]%
\>[1]\AgdaKeyword{open}\AgdaSpace{}%
\AgdaModule{Setoid}\AgdaSpace{}%
\AgdaOperator{\AgdaFunction{𝔻[}}\AgdaSpace{}%
\AgdaBound{𝑨}\AgdaSpace{}%
\AgdaOperator{\AgdaFunction{]}}\AgdaSpace{}%
\AgdaKeyword{using}\AgdaSpace{}%
\AgdaSymbol{(}\AgdaSpace{}%
\AgdaOperator{\AgdaField{\AgdaUnderscore{}≈\AgdaUnderscore{}}}\AgdaSpace{}%
\AgdaSymbol{;}\AgdaSpace{}%
\AgdaFunction{sym}\AgdaSpace{}%
\AgdaSymbol{;}\AgdaSpace{}%
\AgdaFunction{trans}\AgdaSpace{}%
\AgdaSymbol{)}\<%
\\
%
\>[1]\AgdaKeyword{open}\AgdaSpace{}%
\AgdaModule{Setoid}\AgdaSpace{}%
\AgdaOperator{\AgdaFunction{𝔻[}}\AgdaSpace{}%
\AgdaBound{𝑩}\AgdaSpace{}%
\AgdaOperator{\AgdaFunction{]}}\AgdaSpace{}%
\AgdaKeyword{using}\AgdaSpace{}%
\AgdaSymbol{()}\AgdaSpace{}%
\AgdaKeyword{renaming}\AgdaSpace{}%
\AgdaSymbol{(}\AgdaSpace{}%
\AgdaOperator{\AgdaField{\AgdaUnderscore{}≈\AgdaUnderscore{}}}\AgdaSpace{}%
\AgdaSymbol{to}\AgdaSpace{}%
\AgdaOperator{\AgdaField{\AgdaUnderscore{}≈ᵇ\AgdaUnderscore{}}}\AgdaSpace{}%
\AgdaSymbol{;}\AgdaSpace{}%
\AgdaFunction{sym}\AgdaSpace{}%
\AgdaSymbol{to}\AgdaSpace{}%
\AgdaFunction{symᵇ}\AgdaSpace{}%
\AgdaSymbol{;}\AgdaSpace{}%
\AgdaFunction{trans}\AgdaSpace{}%
\AgdaSymbol{to}\AgdaSpace{}%
\AgdaFunction{transᵇ}\AgdaSymbol{)}\<%
\\
%
\\[\AgdaEmptyExtraSkip]%
%
\>[1]\AgdaKeyword{record}\AgdaSpace{}%
\AgdaOperator{\AgdaRecord{\AgdaUnderscore{}≅\AgdaUnderscore{}}}\AgdaSpace{}%
\AgdaSymbol{:}\AgdaSpace{}%
\AgdaPrimitive{Type}\AgdaSpace{}%
\AgdaSymbol{(}\AgdaBound{𝓞}\AgdaSpace{}%
\AgdaOperator{\AgdaPrimitive{⊔}}\AgdaSpace{}%
\AgdaBound{𝓥}\AgdaSpace{}%
\AgdaOperator{\AgdaPrimitive{⊔}}\AgdaSpace{}%
\AgdaBound{α}\AgdaSpace{}%
\AgdaOperator{\AgdaPrimitive{⊔}}\AgdaSpace{}%
\AgdaBound{β}\AgdaSpace{}%
\AgdaOperator{\AgdaPrimitive{⊔}}\AgdaSpace{}%
\AgdaBound{ρᵃ}\AgdaSpace{}%
\AgdaOperator{\AgdaPrimitive{⊔}}\AgdaSpace{}%
\AgdaBound{ρᵇ}\AgdaSpace{}%
\AgdaSymbol{)}\AgdaSpace{}%
\AgdaKeyword{where}\<%
\\
\>[1][@{}l@{\AgdaIndent{0}}]%
\>[2]\AgdaKeyword{constructor}\AgdaSpace{}%
\AgdaInductiveConstructor{mkiso}\<%
\\
%
\>[2]\AgdaKeyword{field}\<%
\\
\>[2][@{}l@{\AgdaIndent{0}}]%
\>[3]\AgdaField{to}\AgdaSpace{}%
\AgdaSymbol{:}\AgdaSpace{}%
\AgdaFunction{hom}\AgdaSpace{}%
\AgdaBound{𝑨}\AgdaSpace{}%
\AgdaBound{𝑩}\<%
\\
%
\>[3]\AgdaField{from}\AgdaSpace{}%
\AgdaSymbol{:}\AgdaSpace{}%
\AgdaFunction{hom}\AgdaSpace{}%
\AgdaBound{𝑩}\AgdaSpace{}%
\AgdaBound{𝑨}\<%
\\
%
\>[3]\AgdaField{to∼from}\AgdaSpace{}%
\AgdaSymbol{:}\AgdaSpace{}%
\AgdaSymbol{∀}\AgdaSpace{}%
\AgdaBound{b}\AgdaSpace{}%
\AgdaSymbol{→}\AgdaSpace{}%
\AgdaSymbol{(}\AgdaOperator{\AgdaFunction{∣}}\AgdaSpace{}%
\AgdaField{to}\AgdaSpace{}%
\AgdaOperator{\AgdaFunction{∣}}\AgdaSpace{}%
\AgdaOperator{\AgdaField{⟨\$⟩}}\AgdaSpace{}%
\AgdaSymbol{(}\AgdaOperator{\AgdaFunction{∣}}\AgdaSpace{}%
\AgdaField{from}\AgdaSpace{}%
\AgdaOperator{\AgdaFunction{∣}}\AgdaSpace{}%
\AgdaOperator{\AgdaField{⟨\$⟩}}\AgdaSpace{}%
\AgdaBound{b}\AgdaSymbol{))}\AgdaSpace{}%
\AgdaOperator{\AgdaFunction{≈ᵇ}}\AgdaSpace{}%
\AgdaBound{b}\<%
\\
%
\>[3]\AgdaField{from∼to}\AgdaSpace{}%
\AgdaSymbol{:}\AgdaSpace{}%
\AgdaSymbol{∀}\AgdaSpace{}%
\AgdaBound{a}\AgdaSpace{}%
\AgdaSymbol{→}\AgdaSpace{}%
\AgdaSymbol{(}\AgdaOperator{\AgdaFunction{∣}}\AgdaSpace{}%
\AgdaField{from}\AgdaSpace{}%
\AgdaOperator{\AgdaFunction{∣}}\AgdaSpace{}%
\AgdaOperator{\AgdaField{⟨\$⟩}}\AgdaSpace{}%
\AgdaSymbol{(}\AgdaOperator{\AgdaFunction{∣}}\AgdaSpace{}%
\AgdaField{to}\AgdaSpace{}%
\AgdaOperator{\AgdaFunction{∣}}\AgdaSpace{}%
\AgdaOperator{\AgdaField{⟨\$⟩}}\AgdaSpace{}%
\AgdaBound{a}\AgdaSymbol{))}\AgdaSpace{}%
\AgdaOperator{\AgdaFunction{≈}}\AgdaSpace{}%
\AgdaBound{a}\<%
\\
%
\\[\AgdaEmptyExtraSkip]%
%
\>[2]\AgdaFunction{toIsSurjective}\AgdaSpace{}%
\AgdaSymbol{:}\AgdaSpace{}%
\AgdaFunction{IsSurjective}\AgdaSpace{}%
\AgdaOperator{\AgdaFunction{∣}}\AgdaSpace{}%
\AgdaField{to}\AgdaSpace{}%
\AgdaOperator{\AgdaFunction{∣}}\<%
\\
%
\>[2]\AgdaFunction{toIsSurjective}\AgdaSpace{}%
\AgdaSymbol{\{}\AgdaBound{y}\AgdaSymbol{\}}\AgdaSpace{}%
\AgdaSymbol{=}\AgdaSpace{}%
\AgdaInductiveConstructor{eq}\AgdaSpace{}%
\AgdaSymbol{(}\AgdaOperator{\AgdaFunction{∣}}\AgdaSpace{}%
\AgdaField{from}\AgdaSpace{}%
\AgdaOperator{\AgdaFunction{∣}}\AgdaSpace{}%
\AgdaOperator{\AgdaField{⟨\$⟩}}\AgdaSpace{}%
\AgdaBound{y}\AgdaSymbol{)}\AgdaSpace{}%
\AgdaSymbol{(}\AgdaFunction{symᵇ}\AgdaSpace{}%
\AgdaSymbol{(}\AgdaField{to∼from}\AgdaSpace{}%
\AgdaBound{y}\AgdaSymbol{))}\<%
\\
%
\\[\AgdaEmptyExtraSkip]%
%
\>[2]\AgdaFunction{toIsInjective}\AgdaSpace{}%
\AgdaSymbol{:}\AgdaSpace{}%
\AgdaFunction{IsInjective}\AgdaSpace{}%
\AgdaOperator{\AgdaFunction{∣}}\AgdaSpace{}%
\AgdaField{to}\AgdaSpace{}%
\AgdaOperator{\AgdaFunction{∣}}\<%
\\
%
\>[2]\AgdaFunction{toIsInjective}\AgdaSpace{}%
\AgdaSymbol{\{}\AgdaBound{x}\AgdaSymbol{\}}\AgdaSpace{}%
\AgdaSymbol{\{}\AgdaBound{y}\AgdaSymbol{\}}\AgdaSpace{}%
\AgdaBound{xy}\AgdaSpace{}%
\AgdaSymbol{=}\AgdaSpace{}%
\AgdaFunction{Goal}\<%
\\
\>[2][@{}l@{\AgdaIndent{0}}]%
\>[3]\AgdaKeyword{where}\<%
\\
%
\>[3]\AgdaFunction{ξ}\AgdaSpace{}%
\AgdaSymbol{:}\AgdaSpace{}%
\AgdaOperator{\AgdaFunction{∣}}\AgdaSpace{}%
\AgdaField{from}\AgdaSpace{}%
\AgdaOperator{\AgdaFunction{∣}}\AgdaSpace{}%
\AgdaOperator{\AgdaField{⟨\$⟩}}\AgdaSpace{}%
\AgdaSymbol{(}\AgdaOperator{\AgdaFunction{∣}}\AgdaSpace{}%
\AgdaField{to}\AgdaSpace{}%
\AgdaOperator{\AgdaFunction{∣}}\AgdaSpace{}%
\AgdaOperator{\AgdaField{⟨\$⟩}}\AgdaSpace{}%
\AgdaBound{x}\AgdaSymbol{)}\AgdaSpace{}%
\AgdaOperator{\AgdaFunction{≈}}\AgdaSpace{}%
\AgdaOperator{\AgdaFunction{∣}}\AgdaSpace{}%
\AgdaField{from}\AgdaSpace{}%
\AgdaOperator{\AgdaFunction{∣}}\AgdaSpace{}%
\AgdaOperator{\AgdaField{⟨\$⟩}}\AgdaSpace{}%
\AgdaSymbol{(}\AgdaOperator{\AgdaFunction{∣}}\AgdaSpace{}%
\AgdaField{to}\AgdaSpace{}%
\AgdaOperator{\AgdaFunction{∣}}\AgdaSpace{}%
\AgdaOperator{\AgdaField{⟨\$⟩}}\AgdaSpace{}%
\AgdaBound{y}\AgdaSymbol{)}\<%
\\
%
\>[3]\AgdaFunction{ξ}\AgdaSpace{}%
\AgdaSymbol{=}\AgdaSpace{}%
\AgdaField{cong}\AgdaSpace{}%
\AgdaOperator{\AgdaFunction{∣}}\AgdaSpace{}%
\AgdaField{from}\AgdaSpace{}%
\AgdaOperator{\AgdaFunction{∣}}\AgdaSpace{}%
\AgdaBound{xy}\<%
\\
%
\>[3]\AgdaFunction{Goal}\AgdaSpace{}%
\AgdaSymbol{:}\AgdaSpace{}%
\AgdaBound{x}\AgdaSpace{}%
\AgdaOperator{\AgdaFunction{≈}}\AgdaSpace{}%
\AgdaBound{y}\<%
\\
%
\>[3]\AgdaFunction{Goal}\AgdaSpace{}%
\AgdaSymbol{=}\AgdaSpace{}%
\AgdaFunction{trans}\AgdaSpace{}%
\AgdaSymbol{(}\AgdaFunction{sym}\AgdaSpace{}%
\AgdaSymbol{(}\AgdaField{from∼to}\AgdaSpace{}%
\AgdaBound{x}\AgdaSymbol{))}\AgdaSpace{}%
\AgdaSymbol{(}\AgdaFunction{trans}\AgdaSpace{}%
\AgdaFunction{ξ}\AgdaSpace{}%
\AgdaSymbol{(}\AgdaField{from∼to}\AgdaSpace{}%
\AgdaBound{y}\AgdaSymbol{))}\<%
\\
%
\\[\AgdaEmptyExtraSkip]%
%
\>[2]\AgdaFunction{fromIsSurjective}\AgdaSpace{}%
\AgdaSymbol{:}\AgdaSpace{}%
\AgdaFunction{IsSurjective}\AgdaSpace{}%
\AgdaOperator{\AgdaFunction{∣}}\AgdaSpace{}%
\AgdaField{from}\AgdaSpace{}%
\AgdaOperator{\AgdaFunction{∣}}\<%
\\
%
\>[2]\AgdaFunction{fromIsSurjective}\AgdaSpace{}%
\AgdaSymbol{\{}\AgdaBound{y}\AgdaSymbol{\}}\AgdaSpace{}%
\AgdaSymbol{=}\AgdaSpace{}%
\AgdaInductiveConstructor{eq}\AgdaSpace{}%
\AgdaSymbol{(}\AgdaOperator{\AgdaFunction{∣}}\AgdaSpace{}%
\AgdaField{to}\AgdaSpace{}%
\AgdaOperator{\AgdaFunction{∣}}\AgdaSpace{}%
\AgdaOperator{\AgdaField{⟨\$⟩}}\AgdaSpace{}%
\AgdaBound{y}\AgdaSymbol{)}\AgdaSpace{}%
\AgdaSymbol{(}\AgdaFunction{sym}\AgdaSpace{}%
\AgdaSymbol{(}\AgdaField{from∼to}\AgdaSpace{}%
\AgdaBound{y}\AgdaSymbol{))}\<%
\\
%
\\[\AgdaEmptyExtraSkip]%
%
\>[2]\AgdaFunction{fromIsInjective}\AgdaSpace{}%
\AgdaSymbol{:}\AgdaSpace{}%
\AgdaFunction{IsInjective}\AgdaSpace{}%
\AgdaOperator{\AgdaFunction{∣}}\AgdaSpace{}%
\AgdaField{from}\AgdaSpace{}%
\AgdaOperator{\AgdaFunction{∣}}\<%
\\
%
\>[2]\AgdaFunction{fromIsInjective}\AgdaSpace{}%
\AgdaSymbol{\{}\AgdaBound{x}\AgdaSymbol{\}}\AgdaSpace{}%
\AgdaSymbol{\{}\AgdaBound{y}\AgdaSymbol{\}}\AgdaSpace{}%
\AgdaBound{xy}\AgdaSpace{}%
\AgdaSymbol{=}\AgdaSpace{}%
\AgdaFunction{Goal}\<%
\\
\>[2][@{}l@{\AgdaIndent{0}}]%
\>[3]\AgdaKeyword{where}\<%
\\
%
\>[3]\AgdaFunction{ξ}\AgdaSpace{}%
\AgdaSymbol{:}\AgdaSpace{}%
\AgdaOperator{\AgdaFunction{∣}}\AgdaSpace{}%
\AgdaField{to}\AgdaSpace{}%
\AgdaOperator{\AgdaFunction{∣}}\AgdaSpace{}%
\AgdaOperator{\AgdaField{⟨\$⟩}}\AgdaSpace{}%
\AgdaSymbol{(}\AgdaOperator{\AgdaFunction{∣}}\AgdaSpace{}%
\AgdaField{from}\AgdaSpace{}%
\AgdaOperator{\AgdaFunction{∣}}\AgdaSpace{}%
\AgdaOperator{\AgdaField{⟨\$⟩}}\AgdaSpace{}%
\AgdaBound{x}\AgdaSymbol{)}\AgdaSpace{}%
\AgdaOperator{\AgdaFunction{≈ᵇ}}\AgdaSpace{}%
\AgdaOperator{\AgdaFunction{∣}}\AgdaSpace{}%
\AgdaField{to}\AgdaSpace{}%
\AgdaOperator{\AgdaFunction{∣}}\AgdaSpace{}%
\AgdaOperator{\AgdaField{⟨\$⟩}}\AgdaSpace{}%
\AgdaSymbol{(}\AgdaOperator{\AgdaFunction{∣}}\AgdaSpace{}%
\AgdaField{from}\AgdaSpace{}%
\AgdaOperator{\AgdaFunction{∣}}\AgdaSpace{}%
\AgdaOperator{\AgdaField{⟨\$⟩}}\AgdaSpace{}%
\AgdaBound{y}\AgdaSymbol{)}\<%
\\
%
\>[3]\AgdaFunction{ξ}\AgdaSpace{}%
\AgdaSymbol{=}\AgdaSpace{}%
\AgdaField{cong}\AgdaSpace{}%
\AgdaOperator{\AgdaFunction{∣}}\AgdaSpace{}%
\AgdaField{to}\AgdaSpace{}%
\AgdaOperator{\AgdaFunction{∣}}\AgdaSpace{}%
\AgdaBound{xy}\<%
\\
%
\>[3]\AgdaFunction{Goal}\AgdaSpace{}%
\AgdaSymbol{:}\AgdaSpace{}%
\AgdaBound{x}\AgdaSpace{}%
\AgdaOperator{\AgdaFunction{≈ᵇ}}\AgdaSpace{}%
\AgdaBound{y}\<%
\\
%
\>[3]\AgdaFunction{Goal}\AgdaSpace{}%
\AgdaSymbol{=}\AgdaSpace{}%
\AgdaFunction{transᵇ}\AgdaSpace{}%
\AgdaSymbol{(}\AgdaFunction{symᵇ}\AgdaSpace{}%
\AgdaSymbol{(}\AgdaField{to∼from}\AgdaSpace{}%
\AgdaBound{x}\AgdaSymbol{))}\AgdaSpace{}%
\AgdaSymbol{(}\AgdaFunction{transᵇ}\AgdaSpace{}%
\AgdaFunction{ξ}\AgdaSpace{}%
\AgdaSymbol{(}\AgdaField{to∼from}\AgdaSpace{}%
\AgdaBound{y}\AgdaSymbol{))}\<%
\\
%
\\[\AgdaEmptyExtraSkip]%
\>[0]\AgdaKeyword{open}\AgdaSpace{}%
\AgdaOperator{\AgdaModule{\AgdaUnderscore{}≅\AgdaUnderscore{}}}\<%
\end{code}

\hypertarget{properties-of-isomorphisms}{%
\paragraph{Properties of
isomorphisms}\label{properties-of-isomorphisms}}

\begin{code}%
\>[0]\<%
\\
\>[0]\AgdaFunction{≅-refl}\AgdaSpace{}%
\AgdaSymbol{:}\AgdaSpace{}%
\AgdaFunction{Reflexive}\AgdaSpace{}%
\AgdaSymbol{(}\AgdaOperator{\AgdaRecord{\AgdaUnderscore{}≅\AgdaUnderscore{}}}\AgdaSpace{}%
\AgdaSymbol{\{}\AgdaGeneralizable{α}\AgdaSymbol{\}\{}\AgdaGeneralizable{ρᵃ}\AgdaSymbol{\})}\<%
\\
\>[0]\AgdaFunction{≅-refl}\AgdaSpace{}%
\AgdaSymbol{\{}\AgdaBound{α}\AgdaSymbol{\}\{}\AgdaBound{ρᵃ}\AgdaSymbol{\}\{}\AgdaBound{𝑨}\AgdaSymbol{\}}\AgdaSpace{}%
\AgdaSymbol{=}\AgdaSpace{}%
\AgdaInductiveConstructor{mkiso}\AgdaSpace{}%
\AgdaFunction{𝒾𝒹}\AgdaSpace{}%
\AgdaFunction{𝒾𝒹}\AgdaSpace{}%
\AgdaSymbol{(λ}\AgdaSpace{}%
\AgdaBound{b}\AgdaSpace{}%
\AgdaSymbol{→}\AgdaSpace{}%
\AgdaFunction{refl}\AgdaSymbol{)}\AgdaSpace{}%
\AgdaSymbol{λ}\AgdaSpace{}%
\AgdaBound{a}\AgdaSpace{}%
\AgdaSymbol{→}\AgdaSpace{}%
\AgdaFunction{refl}\<%
\\
\>[0][@{}l@{\AgdaIndent{0}}]%
\>[1]\AgdaKeyword{where}\AgdaSpace{}%
\AgdaKeyword{open}\AgdaSpace{}%
\AgdaModule{Setoid}\AgdaSpace{}%
\AgdaOperator{\AgdaFunction{𝔻[}}\AgdaSpace{}%
\AgdaBound{𝑨}\AgdaSpace{}%
\AgdaOperator{\AgdaFunction{]}}\AgdaSpace{}%
\AgdaKeyword{using}\AgdaSpace{}%
\AgdaSymbol{(}\AgdaSpace{}%
\AgdaFunction{refl}\AgdaSpace{}%
\AgdaSymbol{)}\<%
\\
%
\\[\AgdaEmptyExtraSkip]%
\>[0]\AgdaFunction{≅-sym}\AgdaSpace{}%
\AgdaSymbol{:}\AgdaSpace{}%
\AgdaFunction{Sym}\AgdaSpace{}%
\AgdaSymbol{(}\AgdaOperator{\AgdaRecord{\AgdaUnderscore{}≅\AgdaUnderscore{}}}\AgdaSymbol{\{}\AgdaGeneralizable{β}\AgdaSymbol{\}\{}\AgdaGeneralizable{ρᵇ}\AgdaSymbol{\})}\AgdaSpace{}%
\AgdaSymbol{(}\AgdaOperator{\AgdaRecord{\AgdaUnderscore{}≅\AgdaUnderscore{}}}\AgdaSymbol{\{}\AgdaGeneralizable{α}\AgdaSymbol{\}\{}\AgdaGeneralizable{ρᵃ}\AgdaSymbol{\})}\<%
\\
\>[0]\AgdaFunction{≅-sym}\AgdaSpace{}%
\AgdaBound{φ}\AgdaSpace{}%
\AgdaSymbol{=}\AgdaSpace{}%
\AgdaInductiveConstructor{mkiso}\AgdaSpace{}%
\AgdaSymbol{(}\AgdaField{from}\AgdaSpace{}%
\AgdaBound{φ}\AgdaSymbol{)}\AgdaSpace{}%
\AgdaSymbol{(}\AgdaField{to}\AgdaSpace{}%
\AgdaBound{φ}\AgdaSymbol{)}\AgdaSpace{}%
\AgdaSymbol{(}\AgdaField{from∼to}\AgdaSpace{}%
\AgdaBound{φ}\AgdaSymbol{)}\AgdaSpace{}%
\AgdaSymbol{(}\AgdaField{to∼from}\AgdaSpace{}%
\AgdaBound{φ}\AgdaSymbol{)}\<%
\\
%
\\[\AgdaEmptyExtraSkip]%
\>[0]\AgdaFunction{≅-trans}\AgdaSpace{}%
\AgdaSymbol{:}\AgdaSpace{}%
\AgdaFunction{Trans}\AgdaSpace{}%
\AgdaSymbol{(}\AgdaOperator{\AgdaRecord{\AgdaUnderscore{}≅\AgdaUnderscore{}}}\AgdaSpace{}%
\AgdaSymbol{\{}\AgdaGeneralizable{α}\AgdaSymbol{\}\{}\AgdaGeneralizable{ρᵃ}\AgdaSymbol{\})(}\AgdaOperator{\AgdaRecord{\AgdaUnderscore{}≅\AgdaUnderscore{}}}\AgdaSymbol{\{}\AgdaGeneralizable{β}\AgdaSymbol{\}\{}\AgdaGeneralizable{ρᵇ}\AgdaSymbol{\})(}\AgdaOperator{\AgdaRecord{\AgdaUnderscore{}≅\AgdaUnderscore{}}}\AgdaSymbol{\{}\AgdaGeneralizable{α}\AgdaSymbol{\}\{}\AgdaGeneralizable{ρᵃ}\AgdaSymbol{\}\{}\AgdaGeneralizable{γ}\AgdaSymbol{\}\{}\AgdaGeneralizable{ρᶜ}\AgdaSymbol{\})}\<%
\\
\>[0]\AgdaFunction{≅-trans}\AgdaSpace{}%
\AgdaSymbol{\{}\AgdaArgument{ρᶜ}\AgdaSpace{}%
\AgdaSymbol{=}\AgdaSpace{}%
\AgdaBound{ρᶜ}\AgdaSymbol{\}\{}\AgdaBound{𝑨}\AgdaSymbol{\}\{}\AgdaBound{𝑩}\AgdaSymbol{\}\{}\AgdaBound{𝑪}\AgdaSymbol{\}}\AgdaSpace{}%
\AgdaBound{ab}\AgdaSpace{}%
\AgdaBound{bc}\AgdaSpace{}%
\AgdaSymbol{=}\AgdaSpace{}%
\AgdaInductiveConstructor{mkiso}\AgdaSpace{}%
\AgdaFunction{f}\AgdaSpace{}%
\AgdaFunction{g}\AgdaSpace{}%
\AgdaFunction{τ}\AgdaSpace{}%
\AgdaFunction{ν}\<%
\\
\>[0][@{}l@{\AgdaIndent{0}}]%
\>[1]\AgdaKeyword{where}\<%
\\
\>[1][@{}l@{\AgdaIndent{0}}]%
\>[2]\AgdaKeyword{open}\AgdaSpace{}%
\AgdaModule{Setoid}\AgdaSpace{}%
\AgdaOperator{\AgdaFunction{𝔻[}}\AgdaSpace{}%
\AgdaBound{𝑨}\AgdaSpace{}%
\AgdaOperator{\AgdaFunction{]}}\AgdaSpace{}%
\AgdaKeyword{using}\AgdaSpace{}%
\AgdaSymbol{(}\AgdaSpace{}%
\AgdaOperator{\AgdaField{\AgdaUnderscore{}≈\AgdaUnderscore{}}}\AgdaSpace{}%
\AgdaSymbol{;}\AgdaSpace{}%
\AgdaFunction{trans}\AgdaSpace{}%
\AgdaSymbol{)}\<%
\\
%
\>[2]\AgdaKeyword{open}\AgdaSpace{}%
\AgdaModule{Setoid}\AgdaSpace{}%
\AgdaOperator{\AgdaFunction{𝔻[}}\AgdaSpace{}%
\AgdaBound{𝑪}\AgdaSpace{}%
\AgdaOperator{\AgdaFunction{]}}\AgdaSpace{}%
\AgdaKeyword{using}\AgdaSpace{}%
\AgdaSymbol{()}\AgdaSpace{}%
\AgdaKeyword{renaming}\AgdaSpace{}%
\AgdaSymbol{(}\AgdaSpace{}%
\AgdaOperator{\AgdaField{\AgdaUnderscore{}≈\AgdaUnderscore{}}}\AgdaSpace{}%
\AgdaSymbol{to}\AgdaSpace{}%
\AgdaOperator{\AgdaField{\AgdaUnderscore{}≈ᶜ\AgdaUnderscore{}}}\AgdaSpace{}%
\AgdaSymbol{;}\AgdaSpace{}%
\AgdaFunction{trans}\AgdaSpace{}%
\AgdaSymbol{to}\AgdaSpace{}%
\AgdaFunction{transᶜ}\AgdaSpace{}%
\AgdaSymbol{)}\<%
\\
%
\>[2]\AgdaFunction{f}\AgdaSpace{}%
\AgdaSymbol{:}\AgdaSpace{}%
\AgdaFunction{hom}\AgdaSpace{}%
\AgdaBound{𝑨}\AgdaSpace{}%
\AgdaBound{𝑪}\<%
\\
%
\>[2]\AgdaFunction{f}\AgdaSpace{}%
\AgdaSymbol{=}\AgdaSpace{}%
\AgdaFunction{∘-hom}\AgdaSpace{}%
\AgdaSymbol{(}\AgdaField{to}\AgdaSpace{}%
\AgdaBound{ab}\AgdaSymbol{)}\AgdaSpace{}%
\AgdaSymbol{(}\AgdaField{to}\AgdaSpace{}%
\AgdaBound{bc}\AgdaSymbol{)}\<%
\\
%
\>[2]\AgdaFunction{g}\AgdaSpace{}%
\AgdaSymbol{:}\AgdaSpace{}%
\AgdaFunction{hom}\AgdaSpace{}%
\AgdaBound{𝑪}\AgdaSpace{}%
\AgdaBound{𝑨}\<%
\\
%
\>[2]\AgdaFunction{g}\AgdaSpace{}%
\AgdaSymbol{=}\AgdaSpace{}%
\AgdaFunction{∘-hom}\AgdaSpace{}%
\AgdaSymbol{(}\AgdaField{from}\AgdaSpace{}%
\AgdaBound{bc}\AgdaSymbol{)}\AgdaSpace{}%
\AgdaSymbol{(}\AgdaField{from}\AgdaSpace{}%
\AgdaBound{ab}\AgdaSymbol{)}\<%
\\
%
\>[2]\AgdaFunction{τ}\AgdaSpace{}%
\AgdaSymbol{:}\AgdaSpace{}%
\AgdaSymbol{∀}\AgdaSpace{}%
\AgdaBound{b}\AgdaSpace{}%
\AgdaSymbol{→}\AgdaSpace{}%
\AgdaSymbol{(}\AgdaOperator{\AgdaFunction{∣}}\AgdaSpace{}%
\AgdaFunction{f}\AgdaSpace{}%
\AgdaOperator{\AgdaFunction{∣}}%
\>[20]\AgdaOperator{\AgdaField{⟨\$⟩}}\AgdaSpace{}%
\AgdaSymbol{(}\AgdaOperator{\AgdaFunction{∣}}\AgdaSpace{}%
\AgdaFunction{g}\AgdaSpace{}%
\AgdaOperator{\AgdaFunction{∣}}\AgdaSpace{}%
\AgdaOperator{\AgdaField{⟨\$⟩}}\AgdaSpace{}%
\AgdaBound{b}\AgdaSymbol{))}\AgdaSpace{}%
\AgdaOperator{\AgdaFunction{≈ᶜ}}\AgdaSpace{}%
\AgdaBound{b}\<%
\\
%
\>[2]\AgdaFunction{τ}\AgdaSpace{}%
\AgdaBound{b}\AgdaSpace{}%
\AgdaSymbol{=}\AgdaSpace{}%
\AgdaFunction{transᶜ}\AgdaSpace{}%
\AgdaSymbol{(}\AgdaField{cong}\AgdaSpace{}%
\AgdaOperator{\AgdaFunction{∣}}\AgdaSpace{}%
\AgdaField{to}\AgdaSpace{}%
\AgdaBound{bc}\AgdaSpace{}%
\AgdaOperator{\AgdaFunction{∣}}\AgdaSpace{}%
\AgdaSymbol{(}\AgdaField{to∼from}\AgdaSpace{}%
\AgdaBound{ab}\AgdaSpace{}%
\AgdaSymbol{(}\AgdaOperator{\AgdaFunction{∣}}\AgdaSpace{}%
\AgdaField{from}\AgdaSpace{}%
\AgdaBound{bc}\AgdaSpace{}%
\AgdaOperator{\AgdaFunction{∣}}\AgdaSpace{}%
\AgdaOperator{\AgdaField{⟨\$⟩}}\AgdaSpace{}%
\AgdaBound{b}\AgdaSymbol{)))}\AgdaSpace{}%
\AgdaSymbol{(}\AgdaField{to∼from}\AgdaSpace{}%
\AgdaBound{bc}\AgdaSpace{}%
\AgdaBound{b}\AgdaSymbol{)}\<%
\\
%
\>[2]\AgdaFunction{ν}\AgdaSpace{}%
\AgdaSymbol{:}\AgdaSpace{}%
\AgdaSymbol{∀}\AgdaSpace{}%
\AgdaBound{a}\AgdaSpace{}%
\AgdaSymbol{→}\AgdaSpace{}%
\AgdaSymbol{(}\AgdaOperator{\AgdaFunction{∣}}\AgdaSpace{}%
\AgdaFunction{g}\AgdaSpace{}%
\AgdaOperator{\AgdaFunction{∣}}\AgdaSpace{}%
\AgdaOperator{\AgdaField{⟨\$⟩}}\AgdaSpace{}%
\AgdaSymbol{(}\AgdaOperator{\AgdaFunction{∣}}\AgdaSpace{}%
\AgdaFunction{f}\AgdaSpace{}%
\AgdaOperator{\AgdaFunction{∣}}\AgdaSpace{}%
\AgdaOperator{\AgdaField{⟨\$⟩}}\AgdaSpace{}%
\AgdaBound{a}\AgdaSymbol{))}\AgdaSpace{}%
\AgdaOperator{\AgdaFunction{≈}}\AgdaSpace{}%
\AgdaBound{a}\<%
\\
%
\>[2]\AgdaFunction{ν}\AgdaSpace{}%
\AgdaBound{a}\AgdaSpace{}%
\AgdaSymbol{=}\AgdaSpace{}%
\AgdaFunction{trans}\AgdaSpace{}%
\AgdaSymbol{(}\AgdaField{cong}\AgdaSpace{}%
\AgdaOperator{\AgdaFunction{∣}}\AgdaSpace{}%
\AgdaField{from}\AgdaSpace{}%
\AgdaBound{ab}\AgdaSpace{}%
\AgdaOperator{\AgdaFunction{∣}}\AgdaSpace{}%
\AgdaSymbol{(}\AgdaField{from∼to}\AgdaSpace{}%
\AgdaBound{bc}\AgdaSpace{}%
\AgdaSymbol{(}\AgdaOperator{\AgdaFunction{∣}}\AgdaSpace{}%
\AgdaField{to}\AgdaSpace{}%
\AgdaBound{ab}\AgdaSpace{}%
\AgdaOperator{\AgdaFunction{∣}}\AgdaSpace{}%
\AgdaOperator{\AgdaField{⟨\$⟩}}\AgdaSpace{}%
\AgdaBound{a}\AgdaSymbol{)))}\AgdaSpace{}%
\AgdaSymbol{(}\AgdaField{from∼to}\AgdaSpace{}%
\AgdaBound{ab}\AgdaSpace{}%
\AgdaBound{a}\AgdaSymbol{)}\<%
\\
\>[0]\<%
\end{code}

Fortunately, the lift operation preserves isomorphism (i.e., it's an
\emph{algebraic invariant}). As our focus is universal algebra, this is
important and is what makes the lift operation a workable solution to
the technical problems that arise from the noncumulativity of Agda's
universe hierarchy.

\begin{code}[hide]%
\>[0]\AgdaKeyword{module}\AgdaSpace{}%
\AgdaModule{\AgdaUnderscore{}}\AgdaSpace{}%
\AgdaSymbol{\{}\AgdaBound{𝑨}\AgdaSpace{}%
\AgdaSymbol{:}\AgdaSpace{}%
\AgdaRecord{Algebra}\AgdaSpace{}%
\AgdaGeneralizable{α}\AgdaSpace{}%
\AgdaGeneralizable{ρᵃ}\AgdaSymbol{\}\{}\AgdaBound{ℓ}\AgdaSpace{}%
\AgdaSymbol{:}\AgdaSpace{}%
\AgdaPostulate{Level}\AgdaSymbol{\}}\AgdaSpace{}%
\AgdaKeyword{where}\<%
\\
\>[0][@{}l@{\AgdaIndent{0}}]%
\>[1]\AgdaFunction{Lift-≅ˡ}\AgdaSpace{}%
\AgdaSymbol{:}\AgdaSpace{}%
\AgdaBound{𝑨}\AgdaSpace{}%
\AgdaOperator{\AgdaRecord{≅}}\AgdaSpace{}%
\AgdaSymbol{(}\AgdaFunction{Lift-Algˡ}\AgdaSpace{}%
\AgdaBound{𝑨}\AgdaSpace{}%
\AgdaBound{ℓ}\AgdaSymbol{)}\<%
\\
%
\>[1]\AgdaFunction{Lift-≅ˡ}\AgdaSpace{}%
\AgdaSymbol{=}\AgdaSpace{}%
\AgdaInductiveConstructor{mkiso}\AgdaSpace{}%
\AgdaFunction{ToLiftˡ}\AgdaSpace{}%
\AgdaFunction{FromLiftˡ}\AgdaSpace{}%
\AgdaSymbol{(}\AgdaFunction{ToFromLiftˡ}\AgdaSymbol{\{}\AgdaArgument{𝑨}\AgdaSpace{}%
\AgdaSymbol{=}\AgdaSpace{}%
\AgdaBound{𝑨}\AgdaSymbol{\})}\AgdaSpace{}%
\AgdaSymbol{(}\AgdaFunction{FromToLiftˡ}\AgdaSymbol{\{}\AgdaArgument{𝑨}\AgdaSpace{}%
\AgdaSymbol{=}\AgdaSpace{}%
\AgdaBound{𝑨}\AgdaSymbol{\}\{}\AgdaBound{ℓ}\AgdaSymbol{\})}\<%
\\
%
\\[\AgdaEmptyExtraSkip]%
%
\>[1]\AgdaFunction{Lift-≅ʳ}\AgdaSpace{}%
\AgdaSymbol{:}\AgdaSpace{}%
\AgdaBound{𝑨}\AgdaSpace{}%
\AgdaOperator{\AgdaRecord{≅}}\AgdaSpace{}%
\AgdaSymbol{(}\AgdaFunction{Lift-Algʳ}\AgdaSpace{}%
\AgdaBound{𝑨}\AgdaSpace{}%
\AgdaBound{ℓ}\AgdaSymbol{)}\<%
\\
%
\>[1]\AgdaFunction{Lift-≅ʳ}\AgdaSpace{}%
\AgdaSymbol{=}\AgdaSpace{}%
\AgdaInductiveConstructor{mkiso}\AgdaSpace{}%
\AgdaFunction{ToLiftʳ}\AgdaSpace{}%
\AgdaFunction{FromLiftʳ}\AgdaSpace{}%
\AgdaSymbol{(}\AgdaFunction{ToFromLiftʳ}\AgdaSymbol{\{}\AgdaArgument{𝑨}\AgdaSpace{}%
\AgdaSymbol{=}\AgdaSpace{}%
\AgdaBound{𝑨}\AgdaSymbol{\})}\AgdaSpace{}%
\AgdaSymbol{(}\AgdaFunction{FromToLiftʳ}\AgdaSymbol{\{}\AgdaArgument{𝑨}\AgdaSpace{}%
\AgdaSymbol{=}\AgdaSpace{}%
\AgdaBound{𝑨}\AgdaSymbol{\}\{}\AgdaBound{ℓ}\AgdaSymbol{\})}\<%
\\
%
\\[\AgdaEmptyExtraSkip]%
\>[0]\AgdaFunction{Lift-≅}\AgdaSpace{}%
\AgdaSymbol{:}\AgdaSpace{}%
\AgdaSymbol{\{}\AgdaBound{𝑨}\AgdaSpace{}%
\AgdaSymbol{:}\AgdaSpace{}%
\AgdaRecord{Algebra}\AgdaSpace{}%
\AgdaGeneralizable{α}\AgdaSpace{}%
\AgdaGeneralizable{ρᵃ}\AgdaSymbol{\}\{}\AgdaBound{ℓ}\AgdaSpace{}%
\AgdaBound{ρ}\AgdaSpace{}%
\AgdaSymbol{:}\AgdaSpace{}%
\AgdaPostulate{Level}\AgdaSymbol{\}}\AgdaSpace{}%
\AgdaSymbol{→}\AgdaSpace{}%
\AgdaBound{𝑨}\AgdaSpace{}%
\AgdaOperator{\AgdaRecord{≅}}\AgdaSpace{}%
\AgdaSymbol{(}\AgdaFunction{Lift-Alg}\AgdaSpace{}%
\AgdaBound{𝑨}\AgdaSpace{}%
\AgdaBound{ℓ}\AgdaSpace{}%
\AgdaBound{ρ}\AgdaSymbol{)}\<%
\\
\>[0]\AgdaFunction{Lift-≅}\AgdaSpace{}%
\AgdaSymbol{=}\AgdaSpace{}%
\AgdaFunction{≅-trans}\AgdaSpace{}%
\AgdaFunction{Lift-≅ˡ}\AgdaSpace{}%
\AgdaFunction{Lift-≅ʳ}\<%
\end{code}

\hypertarget{homomorphic-images}{%
\subsubsection{Homomorphic Images}\label{homomorphic-images}}

We begin with what for our purposes is the most useful way to represent
the class of \emph{homomorphic images} of an algebra in dependent type
theory (cf.~the
\href{https://ualib.github.io/agda-algebras/Homomorphisms.Func.HomomorphicImages.html}{Homomorphisms.Func.HomomorphicImages}
module of the \href{https://ualib.github.io/agda-algebras}{Agda
Universal Algebra Library}). (The first definition is merely a
short-hand.)

\begin{code}%
\>[0]\AgdaFunction{ov}\AgdaSpace{}%
\AgdaSymbol{:}\AgdaSpace{}%
\AgdaPostulate{Level}\AgdaSpace{}%
\AgdaSymbol{→}\AgdaSpace{}%
\AgdaPostulate{Level}\<%
\\
\>[0]\AgdaFunction{ov}\AgdaSpace{}%
\AgdaBound{α}\AgdaSpace{}%
\AgdaSymbol{=}\AgdaSpace{}%
\AgdaBound{𝓞}\AgdaSpace{}%
\AgdaOperator{\AgdaPrimitive{⊔}}\AgdaSpace{}%
\AgdaBound{𝓥}\AgdaSpace{}%
\AgdaOperator{\AgdaPrimitive{⊔}}\AgdaSpace{}%
\AgdaPrimitive{lsuc}\AgdaSpace{}%
\AgdaBound{α}\<%
\\
%
\\[\AgdaEmptyExtraSkip]%
\>[0]\AgdaOperator{\AgdaFunction{\AgdaUnderscore{}IsHomImageOf\AgdaUnderscore{}}}\AgdaSpace{}%
\AgdaSymbol{:}\AgdaSpace{}%
\AgdaSymbol{(}\AgdaBound{𝑩}\AgdaSpace{}%
\AgdaSymbol{:}\AgdaSpace{}%
\AgdaRecord{Algebra}\AgdaSpace{}%
\AgdaGeneralizable{β}\AgdaSpace{}%
\AgdaGeneralizable{ρᵇ}\AgdaSymbol{)(}\AgdaBound{𝑨}\AgdaSpace{}%
\AgdaSymbol{:}\AgdaSpace{}%
\AgdaRecord{Algebra}\AgdaSpace{}%
\AgdaGeneralizable{α}\AgdaSpace{}%
\AgdaGeneralizable{ρᵃ}\AgdaSymbol{)}\AgdaSpace{}%
\AgdaSymbol{→}\AgdaSpace{}%
\AgdaPrimitive{Type}\AgdaSpace{}%
\AgdaSymbol{(}\AgdaBound{𝓞}\AgdaSpace{}%
\AgdaOperator{\AgdaPrimitive{⊔}}\AgdaSpace{}%
\AgdaBound{𝓥}\AgdaSpace{}%
\AgdaOperator{\AgdaPrimitive{⊔}}\AgdaSpace{}%
\AgdaGeneralizable{α}\AgdaSpace{}%
\AgdaOperator{\AgdaPrimitive{⊔}}\AgdaSpace{}%
\AgdaGeneralizable{β}\AgdaSpace{}%
\AgdaOperator{\AgdaPrimitive{⊔}}\AgdaSpace{}%
\AgdaGeneralizable{ρᵃ}\AgdaSpace{}%
\AgdaOperator{\AgdaPrimitive{⊔}}\AgdaSpace{}%
\AgdaGeneralizable{ρᵇ}\AgdaSymbol{)}\<%
\\
\>[0]\AgdaBound{𝑩}\AgdaSpace{}%
\AgdaOperator{\AgdaFunction{IsHomImageOf}}\AgdaSpace{}%
\AgdaBound{𝑨}\AgdaSpace{}%
\AgdaSymbol{=}\AgdaSpace{}%
\AgdaFunction{Σ[}\AgdaSpace{}%
\AgdaBound{φ}\AgdaSpace{}%
\AgdaFunction{∈}\AgdaSpace{}%
\AgdaFunction{hom}\AgdaSpace{}%
\AgdaBound{𝑨}\AgdaSpace{}%
\AgdaBound{𝑩}\AgdaSpace{}%
\AgdaFunction{]}\AgdaSpace{}%
\AgdaFunction{IsSurjective}\AgdaSpace{}%
\AgdaOperator{\AgdaFunction{∣}}\AgdaSpace{}%
\AgdaBound{φ}\AgdaSpace{}%
\AgdaOperator{\AgdaFunction{∣}}\<%
\\
%
\\[\AgdaEmptyExtraSkip]%
\>[0]\AgdaFunction{HomImages}\AgdaSpace{}%
\AgdaSymbol{:}\AgdaSpace{}%
\AgdaRecord{Algebra}\AgdaSpace{}%
\AgdaGeneralizable{α}\AgdaSpace{}%
\AgdaGeneralizable{ρᵃ}\AgdaSpace{}%
\AgdaSymbol{→}\AgdaSpace{}%
\AgdaPrimitive{Type}\AgdaSpace{}%
\AgdaSymbol{(}\AgdaGeneralizable{α}\AgdaSpace{}%
\AgdaOperator{\AgdaPrimitive{⊔}}\AgdaSpace{}%
\AgdaGeneralizable{ρᵃ}\AgdaSpace{}%
\AgdaOperator{\AgdaPrimitive{⊔}}\AgdaSpace{}%
\AgdaFunction{ov}\AgdaSpace{}%
\AgdaSymbol{(}\AgdaGeneralizable{β}\AgdaSpace{}%
\AgdaOperator{\AgdaPrimitive{⊔}}\AgdaSpace{}%
\AgdaGeneralizable{ρᵇ}\AgdaSymbol{))}\<%
\\
\>[0]\AgdaFunction{HomImages}\AgdaSpace{}%
\AgdaSymbol{\{}\AgdaArgument{β}\AgdaSpace{}%
\AgdaSymbol{=}\AgdaSpace{}%
\AgdaBound{β}\AgdaSymbol{\}\{}\AgdaArgument{ρᵇ}\AgdaSpace{}%
\AgdaSymbol{=}\AgdaSpace{}%
\AgdaBound{ρᵇ}\AgdaSymbol{\}}\AgdaSpace{}%
\AgdaBound{𝑨}\AgdaSpace{}%
\AgdaSymbol{=}\AgdaSpace{}%
\AgdaFunction{Σ[}\AgdaSpace{}%
\AgdaBound{𝑩}\AgdaSpace{}%
\AgdaFunction{∈}\AgdaSpace{}%
\AgdaRecord{Algebra}\AgdaSpace{}%
\AgdaBound{β}\AgdaSpace{}%
\AgdaBound{ρᵇ}\AgdaSpace{}%
\AgdaFunction{]}\AgdaSpace{}%
\AgdaBound{𝑩}\AgdaSpace{}%
\AgdaOperator{\AgdaFunction{IsHomImageOf}}\AgdaSpace{}%
\AgdaBound{𝑨}\<%
\\
\>[0]\<%
\end{code}

These types should be self-explanatory, but just to be sure, let's
describe the Sigma type appearing in the second definition. Given an
\texttt{𝑆}-algebra \texttt{𝑨\ :\ Algebra\ α\ ρ}, the type
\texttt{HomImages\ 𝑨} denotes the class of algebras
\texttt{𝑩\ :\ Algebra\ β\ ρ} with a map
\texttt{φ\ :\ ∣\ 𝑨\ ∣\ →\ ∣\ 𝑩\ ∣} such that \texttt{φ} is a surjective
homomorphism.

\begin{code}[hide]%
\>[0]\AgdaKeyword{module}\AgdaSpace{}%
\AgdaModule{\AgdaUnderscore{}}\AgdaSpace{}%
\AgdaSymbol{\{}\AgdaBound{𝑨}\AgdaSpace{}%
\AgdaSymbol{:}\AgdaSpace{}%
\AgdaRecord{Algebra}\AgdaSpace{}%
\AgdaGeneralizable{α}\AgdaSpace{}%
\AgdaGeneralizable{ρᵃ}\AgdaSymbol{\}\{}\AgdaBound{𝑩}\AgdaSpace{}%
\AgdaSymbol{:}\AgdaSpace{}%
\AgdaRecord{Algebra}\AgdaSpace{}%
\AgdaGeneralizable{β}\AgdaSpace{}%
\AgdaGeneralizable{ρᵇ}\AgdaSymbol{\}}\AgdaSpace{}%
\AgdaKeyword{where}\<%
\\
%
\\[\AgdaEmptyExtraSkip]%
\>[0][@{}l@{\AgdaIndent{0}}]%
\>[1]\AgdaFunction{Lift-HomImage-lemma}\AgdaSpace{}%
\AgdaSymbol{:}\AgdaSpace{}%
\AgdaSymbol{∀\{}\AgdaBound{γ}\AgdaSymbol{\}}\AgdaSpace{}%
\AgdaSymbol{→}\AgdaSpace{}%
\AgdaSymbol{(}\AgdaFunction{Lift-Alg}\AgdaSpace{}%
\AgdaBound{𝑨}\AgdaSpace{}%
\AgdaBound{γ}\AgdaSpace{}%
\AgdaBound{γ}\AgdaSymbol{)}\AgdaSpace{}%
\AgdaOperator{\AgdaFunction{IsHomImageOf}}\AgdaSpace{}%
\AgdaBound{𝑩}\AgdaSpace{}%
\AgdaSymbol{→}\AgdaSpace{}%
\AgdaBound{𝑨}\AgdaSpace{}%
\AgdaOperator{\AgdaFunction{IsHomImageOf}}\AgdaSpace{}%
\AgdaBound{𝑩}\<%
\\
%
\>[1]\AgdaFunction{Lift-HomImage-lemma}\AgdaSpace{}%
\AgdaSymbol{\{}\AgdaBound{γ}\AgdaSymbol{\}}\AgdaSpace{}%
\AgdaBound{φ}\AgdaSpace{}%
\AgdaSymbol{=}%
\>[3018I]\AgdaFunction{∘-hom}\AgdaSpace{}%
\AgdaOperator{\AgdaFunction{∣}}\AgdaSpace{}%
\AgdaBound{φ}\AgdaSpace{}%
\AgdaOperator{\AgdaFunction{∣}}\AgdaSpace{}%
\AgdaSymbol{(}\AgdaField{from}\AgdaSpace{}%
\AgdaFunction{Lift-≅}\AgdaSymbol{)}\AgdaSpace{}%
\AgdaOperator{\AgdaInductiveConstructor{,}}\<%
\\
\>[.][@{}l@{}]\<[3018I]%
\>[29]\AgdaFunction{∘-IsSurjective}\AgdaSpace{}%
\AgdaSymbol{\AgdaUnderscore{}}\AgdaSpace{}%
\AgdaSymbol{\AgdaUnderscore{}}\AgdaSpace{}%
\AgdaOperator{\AgdaFunction{∥}}\AgdaSpace{}%
\AgdaBound{φ}\AgdaSpace{}%
\AgdaOperator{\AgdaFunction{∥}}\AgdaSpace{}%
\AgdaSymbol{(}\AgdaFunction{fromIsSurjective}\AgdaSpace{}%
\AgdaSymbol{(}\AgdaFunction{Lift-≅}\AgdaSymbol{\{}\AgdaArgument{𝑨}\AgdaSpace{}%
\AgdaSymbol{=}\AgdaSpace{}%
\AgdaBound{𝑨}\AgdaSymbol{\}))}\<%
\\
%
\\[\AgdaEmptyExtraSkip]%
\>[0]\AgdaKeyword{module}\AgdaSpace{}%
\AgdaModule{\AgdaUnderscore{}}\AgdaSpace{}%
\AgdaSymbol{\{}\AgdaBound{𝑨}\AgdaSpace{}%
\AgdaBound{𝑨'}\AgdaSpace{}%
\AgdaSymbol{:}\AgdaSpace{}%
\AgdaRecord{Algebra}\AgdaSpace{}%
\AgdaGeneralizable{α}\AgdaSpace{}%
\AgdaGeneralizable{ρᵃ}\AgdaSymbol{\}\{}\AgdaBound{𝑩}\AgdaSpace{}%
\AgdaSymbol{:}\AgdaSpace{}%
\AgdaRecord{Algebra}\AgdaSpace{}%
\AgdaGeneralizable{β}\AgdaSpace{}%
\AgdaGeneralizable{ρᵇ}\AgdaSymbol{\}}\AgdaSpace{}%
\AgdaKeyword{where}\<%
\\
%
\\[\AgdaEmptyExtraSkip]%
\>[0][@{}l@{\AgdaIndent{0}}]%
\>[1]\AgdaFunction{HomImage-≅}\AgdaSpace{}%
\AgdaSymbol{:}\AgdaSpace{}%
\AgdaBound{𝑨}\AgdaSpace{}%
\AgdaOperator{\AgdaFunction{IsHomImageOf}}\AgdaSpace{}%
\AgdaBound{𝑨'}\AgdaSpace{}%
\AgdaSymbol{→}\AgdaSpace{}%
\AgdaBound{𝑨}\AgdaSpace{}%
\AgdaOperator{\AgdaRecord{≅}}\AgdaSpace{}%
\AgdaBound{𝑩}\AgdaSpace{}%
\AgdaSymbol{→}\AgdaSpace{}%
\AgdaBound{𝑩}\AgdaSpace{}%
\AgdaOperator{\AgdaFunction{IsHomImageOf}}\AgdaSpace{}%
\AgdaBound{𝑨'}\<%
\\
%
\>[1]\AgdaFunction{HomImage-≅}\AgdaSpace{}%
\AgdaBound{φ}\AgdaSpace{}%
\AgdaBound{A≅B}\AgdaSpace{}%
\AgdaSymbol{=}\AgdaSpace{}%
\AgdaFunction{∘-hom}\AgdaSpace{}%
\AgdaOperator{\AgdaFunction{∣}}\AgdaSpace{}%
\AgdaBound{φ}\AgdaSpace{}%
\AgdaOperator{\AgdaFunction{∣}}\AgdaSpace{}%
\AgdaSymbol{(}\AgdaField{to}\AgdaSpace{}%
\AgdaBound{A≅B}\AgdaSymbol{)}\AgdaSpace{}%
\AgdaOperator{\AgdaInductiveConstructor{,}}\AgdaSpace{}%
\AgdaFunction{∘-IsSurjective}\AgdaSpace{}%
\AgdaSymbol{\AgdaUnderscore{}}\AgdaSpace{}%
\AgdaSymbol{\AgdaUnderscore{}}\AgdaSpace{}%
\AgdaOperator{\AgdaFunction{∥}}\AgdaSpace{}%
\AgdaBound{φ}\AgdaSpace{}%
\AgdaOperator{\AgdaFunction{∥}}\AgdaSpace{}%
\AgdaSymbol{(}\AgdaFunction{toIsSurjective}\AgdaSpace{}%
\AgdaBound{A≅B}\AgdaSymbol{)}\<%
\end{code}

\hypertarget{subalgebras}{%
\subsection{Subalgebras}\label{subalgebras}}

\hypertarget{basic-definitions-1}{%
\subsubsection{Basic definitions}\label{basic-definitions-1}}

\begin{code}%
\>[0]\<%
\\
\>[0]\AgdaOperator{\AgdaFunction{\AgdaUnderscore{}≤\AgdaUnderscore{}}}\AgdaSpace{}%
\AgdaSymbol{:}\AgdaSpace{}%
\AgdaRecord{Algebra}\AgdaSpace{}%
\AgdaGeneralizable{α}\AgdaSpace{}%
\AgdaGeneralizable{ρᵃ}\AgdaSpace{}%
\AgdaSymbol{→}\AgdaSpace{}%
\AgdaRecord{Algebra}\AgdaSpace{}%
\AgdaGeneralizable{β}\AgdaSpace{}%
\AgdaGeneralizable{ρᵇ}\AgdaSpace{}%
\AgdaSymbol{→}\AgdaSpace{}%
\AgdaPrimitive{Type}\AgdaSpace{}%
\AgdaSymbol{(}\AgdaBound{𝓞}\AgdaSpace{}%
\AgdaOperator{\AgdaPrimitive{⊔}}\AgdaSpace{}%
\AgdaBound{𝓥}\AgdaSpace{}%
\AgdaOperator{\AgdaPrimitive{⊔}}\AgdaSpace{}%
\AgdaGeneralizable{α}\AgdaSpace{}%
\AgdaOperator{\AgdaPrimitive{⊔}}\AgdaSpace{}%
\AgdaGeneralizable{ρᵃ}\AgdaSpace{}%
\AgdaOperator{\AgdaPrimitive{⊔}}\AgdaSpace{}%
\AgdaGeneralizable{β}\AgdaSpace{}%
\AgdaOperator{\AgdaPrimitive{⊔}}\AgdaSpace{}%
\AgdaGeneralizable{ρᵇ}\AgdaSymbol{)}\<%
\\
\>[0]\AgdaBound{𝑨}\AgdaSpace{}%
\AgdaOperator{\AgdaFunction{≤}}\AgdaSpace{}%
\AgdaBound{𝑩}\AgdaSpace{}%
\AgdaSymbol{=}\AgdaSpace{}%
\AgdaFunction{Σ[}\AgdaSpace{}%
\AgdaBound{h}\AgdaSpace{}%
\AgdaFunction{∈}\AgdaSpace{}%
\AgdaFunction{hom}\AgdaSpace{}%
\AgdaBound{𝑨}\AgdaSpace{}%
\AgdaBound{𝑩}\AgdaSpace{}%
\AgdaFunction{]}\AgdaSpace{}%
\AgdaFunction{IsInjective}\AgdaSpace{}%
\AgdaOperator{\AgdaFunction{∣}}\AgdaSpace{}%
\AgdaBound{h}\AgdaSpace{}%
\AgdaOperator{\AgdaFunction{∣}}\<%
\end{code}

\hypertarget{basic-properties}{%
\subsubsection{Basic properties}\label{basic-properties}}

\begin{code}%
\>[0]\<%
\\
\>[0]\AgdaFunction{≤-reflexive}\AgdaSpace{}%
\AgdaSymbol{:}\AgdaSpace{}%
\AgdaSymbol{\{}\AgdaBound{𝑨}\AgdaSpace{}%
\AgdaSymbol{:}\AgdaSpace{}%
\AgdaRecord{Algebra}\AgdaSpace{}%
\AgdaGeneralizable{α}\AgdaSpace{}%
\AgdaGeneralizable{ρᵃ}\AgdaSymbol{\}}\AgdaSpace{}%
\AgdaSymbol{→}\AgdaSpace{}%
\AgdaBound{𝑨}\AgdaSpace{}%
\AgdaOperator{\AgdaFunction{≤}}\AgdaSpace{}%
\AgdaBound{𝑨}\<%
\\
\>[0]\AgdaFunction{≤-reflexive}\AgdaSpace{}%
\AgdaSymbol{\{}\AgdaArgument{𝑨}\AgdaSpace{}%
\AgdaSymbol{=}\AgdaSpace{}%
\AgdaBound{𝑨}\AgdaSymbol{\}}\AgdaSpace{}%
\AgdaSymbol{=}\AgdaSpace{}%
\AgdaFunction{𝒾𝒹}\AgdaSpace{}%
\AgdaOperator{\AgdaInductiveConstructor{,}}\AgdaSpace{}%
\AgdaFunction{id}\<%
\\
%
\\[\AgdaEmptyExtraSkip]%
\>[0]\AgdaFunction{mon→≤}\AgdaSpace{}%
\AgdaSymbol{:}\AgdaSpace{}%
\AgdaSymbol{\{}\AgdaBound{𝑨}\AgdaSpace{}%
\AgdaSymbol{:}\AgdaSpace{}%
\AgdaRecord{Algebra}\AgdaSpace{}%
\AgdaGeneralizable{α}\AgdaSpace{}%
\AgdaGeneralizable{ρᵃ}\AgdaSymbol{\}\{}\AgdaBound{𝑩}\AgdaSpace{}%
\AgdaSymbol{:}\AgdaSpace{}%
\AgdaRecord{Algebra}\AgdaSpace{}%
\AgdaGeneralizable{β}\AgdaSpace{}%
\AgdaGeneralizable{ρᵇ}\AgdaSymbol{\}}\AgdaSpace{}%
\AgdaSymbol{→}\AgdaSpace{}%
\AgdaFunction{mon}\AgdaSpace{}%
\AgdaBound{𝑨}\AgdaSpace{}%
\AgdaBound{𝑩}\AgdaSpace{}%
\AgdaSymbol{→}\AgdaSpace{}%
\AgdaBound{𝑨}\AgdaSpace{}%
\AgdaOperator{\AgdaFunction{≤}}\AgdaSpace{}%
\AgdaBound{𝑩}\<%
\\
\>[0]\AgdaFunction{mon→≤}\AgdaSpace{}%
\AgdaSymbol{\{}\AgdaArgument{𝑨}\AgdaSpace{}%
\AgdaSymbol{=}\AgdaSpace{}%
\AgdaBound{𝑨}\AgdaSymbol{\}\{}\AgdaBound{𝑩}\AgdaSymbol{\}}\AgdaSpace{}%
\AgdaBound{x}\AgdaSpace{}%
\AgdaSymbol{=}\AgdaSpace{}%
\AgdaFunction{mon→intohom}\AgdaSpace{}%
\AgdaBound{𝑨}\AgdaSpace{}%
\AgdaBound{𝑩}\AgdaSpace{}%
\AgdaBound{x}\<%
\\
%
\\[\AgdaEmptyExtraSkip]%
\>[0]\AgdaKeyword{module}\AgdaSpace{}%
\AgdaModule{\AgdaUnderscore{}}\AgdaSpace{}%
\AgdaSymbol{\{}\AgdaBound{𝑨}\AgdaSpace{}%
\AgdaSymbol{:}\AgdaSpace{}%
\AgdaRecord{Algebra}\AgdaSpace{}%
\AgdaGeneralizable{α}\AgdaSpace{}%
\AgdaGeneralizable{ρᵃ}\AgdaSymbol{\}\{}\AgdaBound{𝑩}\AgdaSpace{}%
\AgdaSymbol{:}\AgdaSpace{}%
\AgdaRecord{Algebra}\AgdaSpace{}%
\AgdaGeneralizable{β}\AgdaSpace{}%
\AgdaGeneralizable{ρᵇ}\AgdaSymbol{\}\{}\AgdaBound{𝑪}\AgdaSpace{}%
\AgdaSymbol{:}\AgdaSpace{}%
\AgdaRecord{Algebra}\AgdaSpace{}%
\AgdaGeneralizable{γ}\AgdaSpace{}%
\AgdaGeneralizable{ρᶜ}\AgdaSymbol{\}}\AgdaSpace{}%
\AgdaKeyword{where}\<%
\\
\>[0][@{}l@{\AgdaIndent{0}}]%
\>[1]\AgdaFunction{≤-trans}\AgdaSpace{}%
\AgdaSymbol{:}\AgdaSpace{}%
\AgdaBound{𝑨}\AgdaSpace{}%
\AgdaOperator{\AgdaFunction{≤}}\AgdaSpace{}%
\AgdaBound{𝑩}\AgdaSpace{}%
\AgdaSymbol{→}\AgdaSpace{}%
\AgdaBound{𝑩}\AgdaSpace{}%
\AgdaOperator{\AgdaFunction{≤}}\AgdaSpace{}%
\AgdaBound{𝑪}\AgdaSpace{}%
\AgdaSymbol{→}\AgdaSpace{}%
\AgdaBound{𝑨}\AgdaSpace{}%
\AgdaOperator{\AgdaFunction{≤}}\AgdaSpace{}%
\AgdaBound{𝑪}\<%
\\
%
\>[1]\AgdaFunction{≤-trans}\AgdaSpace{}%
\AgdaSymbol{(}\AgdaSpace{}%
\AgdaBound{f}\AgdaSpace{}%
\AgdaOperator{\AgdaInductiveConstructor{,}}\AgdaSpace{}%
\AgdaBound{finj}\AgdaSpace{}%
\AgdaSymbol{)}\AgdaSpace{}%
\AgdaSymbol{(}\AgdaSpace{}%
\AgdaBound{g}\AgdaSpace{}%
\AgdaOperator{\AgdaInductiveConstructor{,}}\AgdaSpace{}%
\AgdaBound{ginj}\AgdaSpace{}%
\AgdaSymbol{)}\AgdaSpace{}%
\AgdaSymbol{=}\AgdaSpace{}%
\AgdaSymbol{(}\AgdaFunction{∘-hom}\AgdaSpace{}%
\AgdaBound{f}\AgdaSpace{}%
\AgdaBound{g}\AgdaSpace{}%
\AgdaSymbol{)}\AgdaSpace{}%
\AgdaOperator{\AgdaInductiveConstructor{,}}\AgdaSpace{}%
\AgdaFunction{∘-IsInjective}\AgdaSpace{}%
\AgdaOperator{\AgdaFunction{∣}}\AgdaSpace{}%
\AgdaBound{f}\AgdaSpace{}%
\AgdaOperator{\AgdaFunction{∣}}\AgdaSpace{}%
\AgdaOperator{\AgdaFunction{∣}}\AgdaSpace{}%
\AgdaBound{g}\AgdaSpace{}%
\AgdaOperator{\AgdaFunction{∣}}\AgdaSpace{}%
\AgdaBound{finj}\AgdaSpace{}%
\AgdaBound{ginj}\<%
\\
%
\\[\AgdaEmptyExtraSkip]%
%
\>[1]\AgdaFunction{≅-trans-≤}\AgdaSpace{}%
\AgdaSymbol{:}\AgdaSpace{}%
\AgdaBound{𝑨}\AgdaSpace{}%
\AgdaOperator{\AgdaRecord{≅}}\AgdaSpace{}%
\AgdaBound{𝑩}\AgdaSpace{}%
\AgdaSymbol{→}\AgdaSpace{}%
\AgdaBound{𝑩}\AgdaSpace{}%
\AgdaOperator{\AgdaFunction{≤}}\AgdaSpace{}%
\AgdaBound{𝑪}\AgdaSpace{}%
\AgdaSymbol{→}\AgdaSpace{}%
\AgdaBound{𝑨}\AgdaSpace{}%
\AgdaOperator{\AgdaFunction{≤}}\AgdaSpace{}%
\AgdaBound{𝑪}\<%
\\
%
\>[1]\AgdaFunction{≅-trans-≤}\AgdaSpace{}%
\AgdaBound{A≅B}\AgdaSpace{}%
\AgdaSymbol{(}\AgdaBound{h}\AgdaSpace{}%
\AgdaOperator{\AgdaInductiveConstructor{,}}\AgdaSpace{}%
\AgdaBound{hinj}\AgdaSymbol{)}\AgdaSpace{}%
\AgdaSymbol{=}\AgdaSpace{}%
\AgdaSymbol{(}\AgdaFunction{∘-hom}\AgdaSpace{}%
\AgdaSymbol{(}\AgdaField{to}\AgdaSpace{}%
\AgdaBound{A≅B}\AgdaSymbol{)}\AgdaSpace{}%
\AgdaBound{h}\AgdaSymbol{)}\AgdaSpace{}%
\AgdaOperator{\AgdaInductiveConstructor{,}}\AgdaSpace{}%
\AgdaSymbol{(}\AgdaFunction{∘-IsInjective}\AgdaSpace{}%
\AgdaOperator{\AgdaFunction{∣}}\AgdaSpace{}%
\AgdaField{to}\AgdaSpace{}%
\AgdaBound{A≅B}\AgdaSpace{}%
\AgdaOperator{\AgdaFunction{∣}}\AgdaSpace{}%
\AgdaOperator{\AgdaFunction{∣}}\AgdaSpace{}%
\AgdaBound{h}\AgdaSpace{}%
\AgdaOperator{\AgdaFunction{∣}}\AgdaSpace{}%
\AgdaSymbol{(}\AgdaFunction{toIsInjective}\AgdaSpace{}%
\AgdaBound{A≅B}\AgdaSymbol{)}\AgdaSpace{}%
\AgdaBound{hinj}\AgdaSymbol{)}\<%
\end{code}

\hypertarget{products-of-subalgebras}{%
\subsubsection{Products of subalgebras}\label{products-of-subalgebras}}

\begin{code}%
\>[0]\<%
\\
\>[0]\AgdaKeyword{module}\AgdaSpace{}%
\AgdaModule{\AgdaUnderscore{}}\AgdaSpace{}%
\AgdaSymbol{\{}\AgdaBound{ι}\AgdaSpace{}%
\AgdaSymbol{:}\AgdaSpace{}%
\AgdaPostulate{Level}\AgdaSymbol{\}}\AgdaSpace{}%
\AgdaSymbol{\{}\AgdaBound{I}\AgdaSpace{}%
\AgdaSymbol{:}\AgdaSpace{}%
\AgdaPrimitive{Type}\AgdaSpace{}%
\AgdaBound{ι}\AgdaSymbol{\}\{}\AgdaBound{𝒜}\AgdaSpace{}%
\AgdaSymbol{:}\AgdaSpace{}%
\AgdaBound{I}\AgdaSpace{}%
\AgdaSymbol{→}\AgdaSpace{}%
\AgdaRecord{Algebra}\AgdaSpace{}%
\AgdaGeneralizable{α}\AgdaSpace{}%
\AgdaGeneralizable{ρᵃ}\AgdaSymbol{\}\{}\AgdaBound{ℬ}\AgdaSpace{}%
\AgdaSymbol{:}\AgdaSpace{}%
\AgdaBound{I}\AgdaSpace{}%
\AgdaSymbol{→}\AgdaSpace{}%
\AgdaRecord{Algebra}\AgdaSpace{}%
\AgdaGeneralizable{β}\AgdaSpace{}%
\AgdaGeneralizable{ρᵇ}\AgdaSymbol{\}}\AgdaSpace{}%
\AgdaKeyword{where}\<%
\\
%
\\[\AgdaEmptyExtraSkip]%
\>[0][@{}l@{\AgdaIndent{0}}]%
\>[1]\AgdaFunction{⨅-≤}\AgdaSpace{}%
\AgdaSymbol{:}\AgdaSpace{}%
\AgdaSymbol{(∀}\AgdaSpace{}%
\AgdaBound{i}\AgdaSpace{}%
\AgdaSymbol{→}\AgdaSpace{}%
\AgdaBound{ℬ}\AgdaSpace{}%
\AgdaBound{i}\AgdaSpace{}%
\AgdaOperator{\AgdaFunction{≤}}\AgdaSpace{}%
\AgdaBound{𝒜}\AgdaSpace{}%
\AgdaBound{i}\AgdaSymbol{)}\AgdaSpace{}%
\AgdaSymbol{→}\AgdaSpace{}%
\AgdaFunction{⨅}\AgdaSpace{}%
\AgdaBound{ℬ}\AgdaSpace{}%
\AgdaOperator{\AgdaFunction{≤}}\AgdaSpace{}%
\AgdaFunction{⨅}\AgdaSpace{}%
\AgdaBound{𝒜}\<%
\\
%
\>[1]\AgdaFunction{⨅-≤}\AgdaSpace{}%
\AgdaBound{B≤A}\AgdaSpace{}%
\AgdaSymbol{=}\AgdaSpace{}%
\AgdaSymbol{(}\AgdaFunction{hfunc}\AgdaSpace{}%
\AgdaOperator{\AgdaInductiveConstructor{,}}\AgdaSpace{}%
\AgdaFunction{hhom}\AgdaSymbol{)}\AgdaSpace{}%
\AgdaOperator{\AgdaInductiveConstructor{,}}\AgdaSpace{}%
\AgdaFunction{hM}\<%
\\
\>[1][@{}l@{\AgdaIndent{0}}]%
\>[2]\AgdaKeyword{where}\<%
\\
%
\>[2]\AgdaFunction{hi}\AgdaSpace{}%
\AgdaSymbol{:}\AgdaSpace{}%
\AgdaSymbol{∀}\AgdaSpace{}%
\AgdaBound{i}\AgdaSpace{}%
\AgdaSymbol{→}\AgdaSpace{}%
\AgdaFunction{hom}\AgdaSpace{}%
\AgdaSymbol{(}\AgdaBound{ℬ}\AgdaSpace{}%
\AgdaBound{i}\AgdaSymbol{)}\AgdaSpace{}%
\AgdaSymbol{(}\AgdaBound{𝒜}\AgdaSpace{}%
\AgdaBound{i}\AgdaSymbol{)}\<%
\\
%
\>[2]\AgdaFunction{hi}\AgdaSpace{}%
\AgdaBound{i}\AgdaSpace{}%
\AgdaSymbol{=}\AgdaSpace{}%
\AgdaOperator{\AgdaFunction{∣}}\AgdaSpace{}%
\AgdaBound{B≤A}\AgdaSpace{}%
\AgdaBound{i}\AgdaSpace{}%
\AgdaOperator{\AgdaFunction{∣}}\<%
\\
%
\\[\AgdaEmptyExtraSkip]%
%
\>[2]\AgdaFunction{hfunc}\AgdaSpace{}%
\AgdaSymbol{:}\AgdaSpace{}%
\AgdaOperator{\AgdaFunction{𝔻[}}\AgdaSpace{}%
\AgdaFunction{⨅}\AgdaSpace{}%
\AgdaBound{ℬ}\AgdaSpace{}%
\AgdaOperator{\AgdaFunction{]}}\AgdaSpace{}%
\AgdaOperator{\AgdaRecord{⟶}}\AgdaSpace{}%
\AgdaOperator{\AgdaFunction{𝔻[}}\AgdaSpace{}%
\AgdaFunction{⨅}\AgdaSpace{}%
\AgdaBound{𝒜}\AgdaSpace{}%
\AgdaOperator{\AgdaFunction{]}}\<%
\\
%
\>[2]\AgdaSymbol{(}\AgdaFunction{hfunc}\AgdaSpace{}%
\AgdaOperator{\AgdaField{⟨\$⟩}}\AgdaSpace{}%
\AgdaBound{x}\AgdaSymbol{)}\AgdaSpace{}%
\AgdaBound{i}\AgdaSpace{}%
\AgdaSymbol{=}\AgdaSpace{}%
\AgdaOperator{\AgdaFunction{∣}}\AgdaSpace{}%
\AgdaFunction{hi}\AgdaSpace{}%
\AgdaBound{i}\AgdaSpace{}%
\AgdaOperator{\AgdaFunction{∣}}\AgdaSpace{}%
\AgdaOperator{\AgdaField{⟨\$⟩}}\AgdaSpace{}%
\AgdaSymbol{(}\AgdaBound{x}\AgdaSpace{}%
\AgdaBound{i}\AgdaSymbol{)}\<%
\\
%
\>[2]\AgdaField{cong}\AgdaSpace{}%
\AgdaFunction{hfunc}\AgdaSpace{}%
\AgdaSymbol{=}\AgdaSpace{}%
\AgdaSymbol{λ}\AgdaSpace{}%
\AgdaBound{xy}\AgdaSpace{}%
\AgdaBound{i}\AgdaSpace{}%
\AgdaSymbol{→}\AgdaSpace{}%
\AgdaField{cong}\AgdaSpace{}%
\AgdaOperator{\AgdaFunction{∣}}\AgdaSpace{}%
\AgdaFunction{hi}\AgdaSpace{}%
\AgdaBound{i}\AgdaSpace{}%
\AgdaOperator{\AgdaFunction{∣}}\AgdaSpace{}%
\AgdaSymbol{(}\AgdaBound{xy}\AgdaSpace{}%
\AgdaBound{i}\AgdaSymbol{)}\<%
\\
%
\\[\AgdaEmptyExtraSkip]%
%
\>[2]\AgdaFunction{hhom}\AgdaSpace{}%
\AgdaSymbol{:}\AgdaSpace{}%
\AgdaRecord{IsHom}\AgdaSpace{}%
\AgdaSymbol{(}\AgdaFunction{⨅}\AgdaSpace{}%
\AgdaBound{ℬ}\AgdaSymbol{)}\AgdaSpace{}%
\AgdaSymbol{(}\AgdaFunction{⨅}\AgdaSpace{}%
\AgdaBound{𝒜}\AgdaSymbol{)}\AgdaSpace{}%
\AgdaFunction{hfunc}\<%
\\
%
\>[2]\AgdaField{compatible}\AgdaSpace{}%
\AgdaFunction{hhom}\AgdaSpace{}%
\AgdaSymbol{=}\AgdaSpace{}%
\AgdaSymbol{λ}\AgdaSpace{}%
\AgdaBound{i}\AgdaSpace{}%
\AgdaSymbol{→}\AgdaSpace{}%
\AgdaField{compatible}\AgdaSpace{}%
\AgdaOperator{\AgdaFunction{∥}}\AgdaSpace{}%
\AgdaFunction{hi}\AgdaSpace{}%
\AgdaBound{i}\AgdaSpace{}%
\AgdaOperator{\AgdaFunction{∥}}\<%
\\
%
\\[\AgdaEmptyExtraSkip]%
%
\>[2]\AgdaFunction{hM}\AgdaSpace{}%
\AgdaSymbol{:}\AgdaSpace{}%
\AgdaFunction{IsInjective}\AgdaSpace{}%
\AgdaFunction{hfunc}\<%
\\
%
\>[2]\AgdaFunction{hM}\AgdaSpace{}%
\AgdaSymbol{=}\AgdaSpace{}%
\AgdaSymbol{λ}\AgdaSpace{}%
\AgdaBound{xy}\AgdaSpace{}%
\AgdaBound{i}\AgdaSpace{}%
\AgdaSymbol{→}\AgdaSpace{}%
\AgdaOperator{\AgdaFunction{∥}}\AgdaSpace{}%
\AgdaBound{B≤A}\AgdaSpace{}%
\AgdaBound{i}\AgdaSpace{}%
\AgdaOperator{\AgdaFunction{∥}}\AgdaSpace{}%
\AgdaSymbol{(}\AgdaBound{xy}\AgdaSpace{}%
\AgdaBound{i}\AgdaSymbol{)}\<%
\end{code}

\hypertarget{terms}{%
\subsection{Terms}\label{terms}}

\hypertarget{basic-definitions-2}{%
\subsubsection{Basic definitions}\label{basic-definitions-2}}

Fix a signature \texttt{𝑆} and let \texttt{X} denote an arbitrary
nonempty collection of variable symbols. Assume the symbols in
\texttt{X} are distinct from the operation symbols of \texttt{𝑆}, that
is \texttt{X\ ∩\ ∣\ 𝑆\ ∣\ =\ ∅}.

By a \emph{word} in the language of \texttt{𝑆}, we mean a nonempty,
finite sequence of members of \texttt{X\ ∪\ ∣\ 𝑆\ ∣}. We denote the
concatenation of such sequences by simple juxtaposition.

Let \texttt{S₀} denote the set of nullary operation symbols of
\texttt{𝑆}. We define by induction on \texttt{n} the sets \texttt{𝑇ₙ} of
\emph{words} over \texttt{X\ ∪\ ∣\ 𝑆\ ∣} as follows
(cf.~\href{https://www.amazon.com/gp/product/1439851298/ref=as_li_tl?ie=UTF8\&camp=1789\&creative=9325\&creativeASIN=1439851298\&linkCode=as2\&tag=typefunc-20\&linkId=440725c9b1e60817d071c1167dff95fa}{Bergman
(2012)} Def. 4.19):

\texttt{𝑇₀\ :=\ X\ ∪\ S₀} and \texttt{𝑇ₙ₊₁\ :=\ 𝑇ₙ\ ∪\ 𝒯ₙ}

where \texttt{𝒯ₙ} is the collection of all \texttt{f\ t} such that
\texttt{f\ :\ ∣\ 𝑆\ ∣} and \texttt{t\ :\ ∥\ 𝑆\ ∥\ f\ →\ 𝑇ₙ}. (Recall,
\texttt{∥\ 𝑆\ ∥\ f} is the arity of the operation symbol f.)

We define the collection of \emph{terms} in the signature \texttt{𝑆}
over \texttt{X} by \texttt{Term\ X\ :=\ ⋃ₙ\ 𝑇ₙ}. By an 𝑆-\emph{term} we
mean a term in the language of \texttt{𝑆}.

The definition of \texttt{Term\ X} is recursive, indicating that an
inductive type could be used to represent the semantic notion of terms
in type theory. Indeed, such a representation is given by the following
inductive type.

\begin{code}%
\>[0]\<%
\\
\>[0]\AgdaKeyword{data}\AgdaSpace{}%
\AgdaDatatype{Term}\AgdaSpace{}%
\AgdaSymbol{(}\AgdaBound{X}\AgdaSpace{}%
\AgdaSymbol{:}\AgdaSpace{}%
\AgdaPrimitive{Type}\AgdaSpace{}%
\AgdaGeneralizable{χ}\AgdaSpace{}%
\AgdaSymbol{)}\AgdaSpace{}%
\AgdaSymbol{:}\AgdaSpace{}%
\AgdaPrimitive{Type}\AgdaSpace{}%
\AgdaSymbol{(}\AgdaFunction{ov}\AgdaSpace{}%
\AgdaBound{χ}\AgdaSymbol{)}%
\>[39]\AgdaKeyword{where}\<%
\\
\>[0][@{}l@{\AgdaIndent{0}}]%
\>[1]\AgdaInductiveConstructor{ℊ}\AgdaSpace{}%
\AgdaSymbol{:}\AgdaSpace{}%
\AgdaBound{X}\AgdaSpace{}%
\AgdaSymbol{→}\AgdaSpace{}%
\AgdaDatatype{Term}\AgdaSpace{}%
\AgdaBound{X}\<%
\\
%
\>[1]\AgdaInductiveConstructor{node}\AgdaSpace{}%
\AgdaSymbol{:}\AgdaSpace{}%
\AgdaSymbol{(}\AgdaBound{f}\AgdaSpace{}%
\AgdaSymbol{:}\AgdaSpace{}%
\AgdaOperator{\AgdaFunction{∣}}\AgdaSpace{}%
\AgdaBound{𝑆}\AgdaSpace{}%
\AgdaOperator{\AgdaFunction{∣}}\AgdaSymbol{)(}\AgdaBound{t}\AgdaSpace{}%
\AgdaSymbol{:}\AgdaSpace{}%
\AgdaOperator{\AgdaFunction{∥}}\AgdaSpace{}%
\AgdaBound{𝑆}\AgdaSpace{}%
\AgdaOperator{\AgdaFunction{∥}}\AgdaSpace{}%
\AgdaBound{f}\AgdaSpace{}%
\AgdaSymbol{→}\AgdaSpace{}%
\AgdaDatatype{Term}\AgdaSpace{}%
\AgdaBound{X}\AgdaSymbol{)}\AgdaSpace{}%
\AgdaSymbol{→}\AgdaSpace{}%
\AgdaDatatype{Term}\AgdaSpace{}%
\AgdaBound{X}\<%
\\
\>[0]\AgdaKeyword{open}\AgdaSpace{}%
\AgdaModule{Term}\<%
\\
\>[0]\<%
\end{code}

This is a very basic inductive type that represents each term as a tree
with an operation symbol at each \texttt{node} and a variable symbol at
each leaf (\texttt{generator}); hence the constructor names (\texttt{ℊ}
for ``generator'' and \texttt{node} for node).

\textbf{Notation}. As usual, the type \texttt{X} represents an arbitrary
collection of variable symbols. Recall, \texttt{ov\ χ} is our shorthand
notation for the universe level \texttt{𝓞\ ⊔\ 𝓥\ ⊔\ lsuc\ χ}.

\hypertarget{equality-of-terms}{%
\subsubsection{Equality of terms}\label{equality-of-terms}}

We take a different approach here, using Setoids instead of quotient
types. That is, we will define the collection of terms in a signature as
a setoid with a particular equality-of-terms relation, which we must
define. Ultimately we will use this to define the (absolutely free) term
algebra as a Algebra whose carrier is the setoid of terms.

\begin{code}%
\>[0]\<%
\\
\>[0]\AgdaKeyword{module}\AgdaSpace{}%
\AgdaModule{\AgdaUnderscore{}}\AgdaSpace{}%
\AgdaSymbol{\{}\AgdaBound{X}\AgdaSpace{}%
\AgdaSymbol{:}\AgdaSpace{}%
\AgdaPrimitive{Type}\AgdaSpace{}%
\AgdaGeneralizable{χ}\AgdaSpace{}%
\AgdaSymbol{\}}\AgdaSpace{}%
\AgdaKeyword{where}\<%
\\
%
\\[\AgdaEmptyExtraSkip]%
\>[0][@{}l@{\AgdaIndent{0}}]%
\>[1]\AgdaKeyword{data}\AgdaSpace{}%
\AgdaOperator{\AgdaDatatype{\AgdaUnderscore{}≐\AgdaUnderscore{}}}\AgdaSpace{}%
\AgdaSymbol{:}\AgdaSpace{}%
\AgdaDatatype{Term}\AgdaSpace{}%
\AgdaBound{X}\AgdaSpace{}%
\AgdaSymbol{→}\AgdaSpace{}%
\AgdaDatatype{Term}\AgdaSpace{}%
\AgdaBound{X}\AgdaSpace{}%
\AgdaSymbol{→}\AgdaSpace{}%
\AgdaPrimitive{Type}\AgdaSpace{}%
\AgdaSymbol{(}\AgdaFunction{ov}\AgdaSpace{}%
\AgdaBound{χ}\AgdaSymbol{)}\AgdaSpace{}%
\AgdaKeyword{where}\<%
\\
\>[1][@{}l@{\AgdaIndent{0}}]%
\>[2]\AgdaInductiveConstructor{rfl}\AgdaSpace{}%
\AgdaSymbol{:}\AgdaSpace{}%
\AgdaSymbol{\{}\AgdaBound{x}\AgdaSpace{}%
\AgdaBound{y}\AgdaSpace{}%
\AgdaSymbol{:}\AgdaSpace{}%
\AgdaBound{X}\AgdaSymbol{\}}\AgdaSpace{}%
\AgdaSymbol{→}\AgdaSpace{}%
\AgdaBound{x}\AgdaSpace{}%
\AgdaOperator{\AgdaDatatype{≡}}\AgdaSpace{}%
\AgdaBound{y}\AgdaSpace{}%
\AgdaSymbol{→}\AgdaSpace{}%
\AgdaSymbol{(}\AgdaInductiveConstructor{ℊ}\AgdaSpace{}%
\AgdaBound{x}\AgdaSymbol{)}\AgdaSpace{}%
\AgdaOperator{\AgdaDatatype{≐}}\AgdaSpace{}%
\AgdaSymbol{(}\AgdaInductiveConstructor{ℊ}\AgdaSpace{}%
\AgdaBound{y}\AgdaSymbol{)}\<%
\\
%
\>[2]\AgdaInductiveConstructor{gnl}\AgdaSpace{}%
\AgdaSymbol{:}\AgdaSpace{}%
\AgdaSymbol{∀}\AgdaSpace{}%
\AgdaSymbol{\{}\AgdaBound{f}\AgdaSymbol{\}\{}\AgdaBound{s}\AgdaSpace{}%
\AgdaBound{t}\AgdaSpace{}%
\AgdaSymbol{:}\AgdaSpace{}%
\AgdaOperator{\AgdaFunction{∥}}\AgdaSpace{}%
\AgdaBound{𝑆}\AgdaSpace{}%
\AgdaOperator{\AgdaFunction{∥}}\AgdaSpace{}%
\AgdaBound{f}\AgdaSpace{}%
\AgdaSymbol{→}\AgdaSpace{}%
\AgdaDatatype{Term}\AgdaSpace{}%
\AgdaBound{X}\AgdaSymbol{\}}\AgdaSpace{}%
\AgdaSymbol{→}\AgdaSpace{}%
\AgdaSymbol{(∀}\AgdaSpace{}%
\AgdaBound{i}\AgdaSpace{}%
\AgdaSymbol{→}\AgdaSpace{}%
\AgdaSymbol{(}\AgdaBound{s}\AgdaSpace{}%
\AgdaBound{i}\AgdaSymbol{)}\AgdaSpace{}%
\AgdaOperator{\AgdaDatatype{≐}}\AgdaSpace{}%
\AgdaSymbol{(}\AgdaBound{t}\AgdaSpace{}%
\AgdaBound{i}\AgdaSymbol{))}\AgdaSpace{}%
\AgdaSymbol{→}\AgdaSpace{}%
\AgdaSymbol{(}\AgdaInductiveConstructor{node}\AgdaSpace{}%
\AgdaBound{f}\AgdaSpace{}%
\AgdaBound{s}\AgdaSymbol{)}\AgdaSpace{}%
\AgdaOperator{\AgdaDatatype{≐}}\AgdaSpace{}%
\AgdaSymbol{(}\AgdaInductiveConstructor{node}\AgdaSpace{}%
\AgdaBound{f}\AgdaSpace{}%
\AgdaBound{t}\AgdaSymbol{)}\<%
\\
\>[0]\<%
\end{code}

It is easy to show that the equality-of-terms relation \texttt{\_≐\_} is
an equivalence relation, so we omit the formal proof. (See the
\href{https://ualib.github.io/agda-algebras/Terms.Func.Basic.html}{Terms.Func.Basic}
module of the
\href{https://github.com/ualib/agda-algebras}{agda-algebras} library for
details.)

\begin{code}[hide]%
\>[0][@{}l@{\AgdaIndent{1}}]%
\>[1]\AgdaFunction{≐-isRefl}\AgdaSpace{}%
\AgdaSymbol{:}\AgdaSpace{}%
\AgdaFunction{Reflexive}\AgdaSpace{}%
\AgdaOperator{\AgdaDatatype{\AgdaUnderscore{}≐\AgdaUnderscore{}}}\<%
\\
%
\>[1]\AgdaFunction{≐-isRefl}\AgdaSpace{}%
\AgdaSymbol{\{}\AgdaInductiveConstructor{ℊ}\AgdaSpace{}%
\AgdaSymbol{\AgdaUnderscore{}\}}\AgdaSpace{}%
\AgdaSymbol{=}\AgdaSpace{}%
\AgdaInductiveConstructor{rfl}\AgdaSpace{}%
\AgdaInductiveConstructor{≡.refl}\<%
\\
%
\>[1]\AgdaFunction{≐-isRefl}\AgdaSpace{}%
\AgdaSymbol{\{}\AgdaInductiveConstructor{node}\AgdaSpace{}%
\AgdaSymbol{\AgdaUnderscore{}}\AgdaSpace{}%
\AgdaSymbol{\AgdaUnderscore{}\}}\AgdaSpace{}%
\AgdaSymbol{=}\AgdaSpace{}%
\AgdaInductiveConstructor{gnl}\AgdaSpace{}%
\AgdaSymbol{(λ}\AgdaSpace{}%
\AgdaBound{\AgdaUnderscore{}}\AgdaSpace{}%
\AgdaSymbol{→}\AgdaSpace{}%
\AgdaFunction{≐-isRefl}\AgdaSymbol{)}\<%
\\
%
\\[\AgdaEmptyExtraSkip]%
%
\>[1]\AgdaFunction{≐-isSym}\AgdaSpace{}%
\AgdaSymbol{:}\AgdaSpace{}%
\AgdaFunction{Symmetric}\AgdaSpace{}%
\AgdaOperator{\AgdaDatatype{\AgdaUnderscore{}≐\AgdaUnderscore{}}}\<%
\\
%
\>[1]\AgdaFunction{≐-isSym}\AgdaSpace{}%
\AgdaSymbol{(}\AgdaInductiveConstructor{rfl}\AgdaSpace{}%
\AgdaBound{x}\AgdaSymbol{)}\AgdaSpace{}%
\AgdaSymbol{=}\AgdaSpace{}%
\AgdaInductiveConstructor{rfl}\AgdaSpace{}%
\AgdaSymbol{(}\AgdaFunction{≡.sym}\AgdaSpace{}%
\AgdaBound{x}\AgdaSymbol{)}\<%
\\
%
\>[1]\AgdaFunction{≐-isSym}\AgdaSpace{}%
\AgdaSymbol{(}\AgdaInductiveConstructor{gnl}\AgdaSpace{}%
\AgdaBound{x}\AgdaSymbol{)}\AgdaSpace{}%
\AgdaSymbol{=}\AgdaSpace{}%
\AgdaInductiveConstructor{gnl}\AgdaSpace{}%
\AgdaSymbol{(λ}\AgdaSpace{}%
\AgdaBound{i}\AgdaSpace{}%
\AgdaSymbol{→}\AgdaSpace{}%
\AgdaFunction{≐-isSym}\AgdaSpace{}%
\AgdaSymbol{(}\AgdaBound{x}\AgdaSpace{}%
\AgdaBound{i}\AgdaSymbol{))}\<%
\\
%
\\[\AgdaEmptyExtraSkip]%
%
\>[1]\AgdaFunction{≐-isTrans}\AgdaSpace{}%
\AgdaSymbol{:}\AgdaSpace{}%
\AgdaFunction{Transitive}\AgdaSpace{}%
\AgdaOperator{\AgdaDatatype{\AgdaUnderscore{}≐\AgdaUnderscore{}}}\<%
\\
%
\>[1]\AgdaFunction{≐-isTrans}\AgdaSpace{}%
\AgdaSymbol{(}\AgdaInductiveConstructor{rfl}\AgdaSpace{}%
\AgdaBound{x}\AgdaSymbol{)}\AgdaSpace{}%
\AgdaSymbol{(}\AgdaInductiveConstructor{rfl}\AgdaSpace{}%
\AgdaBound{y}\AgdaSymbol{)}\AgdaSpace{}%
\AgdaSymbol{=}\AgdaSpace{}%
\AgdaInductiveConstructor{rfl}\AgdaSpace{}%
\AgdaSymbol{(}\AgdaFunction{≡.trans}\AgdaSpace{}%
\AgdaBound{x}\AgdaSpace{}%
\AgdaBound{y}\AgdaSymbol{)}\<%
\\
%
\>[1]\AgdaFunction{≐-isTrans}\AgdaSpace{}%
\AgdaSymbol{(}\AgdaInductiveConstructor{gnl}\AgdaSpace{}%
\AgdaBound{x}\AgdaSymbol{)}\AgdaSpace{}%
\AgdaSymbol{(}\AgdaInductiveConstructor{gnl}\AgdaSpace{}%
\AgdaBound{y}\AgdaSymbol{)}\AgdaSpace{}%
\AgdaSymbol{=}\AgdaSpace{}%
\AgdaInductiveConstructor{gnl}\AgdaSpace{}%
\AgdaSymbol{(λ}\AgdaSpace{}%
\AgdaBound{i}\AgdaSpace{}%
\AgdaSymbol{→}\AgdaSpace{}%
\AgdaFunction{≐-isTrans}\AgdaSpace{}%
\AgdaSymbol{(}\AgdaBound{x}\AgdaSpace{}%
\AgdaBound{i}\AgdaSymbol{)}\AgdaSpace{}%
\AgdaSymbol{(}\AgdaBound{y}\AgdaSpace{}%
\AgdaBound{i}\AgdaSymbol{))}\<%
\\
%
\\[\AgdaEmptyExtraSkip]%
%
\>[1]\AgdaFunction{≐-isEquiv}\AgdaSpace{}%
\AgdaSymbol{:}\AgdaSpace{}%
\AgdaRecord{IsEquivalence}\AgdaSpace{}%
\AgdaOperator{\AgdaDatatype{\AgdaUnderscore{}≐\AgdaUnderscore{}}}\<%
\\
%
\>[1]\AgdaFunction{≐-isEquiv}\AgdaSpace{}%
\AgdaSymbol{=}\AgdaSpace{}%
\AgdaKeyword{record}\AgdaSpace{}%
\AgdaSymbol{\{}\AgdaSpace{}%
\AgdaField{refl}\AgdaSpace{}%
\AgdaSymbol{=}\AgdaSpace{}%
\AgdaFunction{≐-isRefl}\AgdaSpace{}%
\AgdaSymbol{;}\AgdaSpace{}%
\AgdaField{sym}\AgdaSpace{}%
\AgdaSymbol{=}\AgdaSpace{}%
\AgdaFunction{≐-isSym}\AgdaSpace{}%
\AgdaSymbol{;}\AgdaSpace{}%
\AgdaField{trans}\AgdaSpace{}%
\AgdaSymbol{=}\AgdaSpace{}%
\AgdaFunction{≐-isTrans}\AgdaSpace{}%
\AgdaSymbol{\}}\<%
\end{code}

\hypertarget{the-term-algebra}{%
\subsubsection{The term algebra}\label{the-term-algebra}}

For a given signature \texttt{𝑆}, if the type \texttt{Term\ X} is
nonempty (equivalently, if \texttt{X} or \texttt{∣\ 𝑆\ ∣} is nonempty),
then we can define an algebraic structure, denoted by \texttt{𝑻\ X} and
called the \emph{term algebra in the signature} \texttt{𝑆} \emph{over}
\texttt{X}. Terms are viewed as acting on other terms, so both the
domain and basic operations of the algebra are the terms themselves.

\begin{itemize}
\tightlist
\item
  For each operation symbol \texttt{f\ :\ ∣\ 𝑆\ ∣}, denote by
  \texttt{f\ ̂\ (𝑻\ X)} the operation on \texttt{Term\ X} that maps a
  tuple \texttt{t\ :\ ∥\ 𝑆\ ∥\ f\ →\ ∣\ 𝑻\ X\ ∣} to the formal term
  \texttt{f\ t}.
\item
  Define \texttt{𝑻\ X} to be the algebra with universe
  \texttt{∣\ 𝑻\ X\ ∣\ :=\ Term\ X} and operations \texttt{f\ ̂\ (𝑻\ X)},
  one for each symbol \texttt{f} in \texttt{∣\ 𝑆\ ∣}.
\end{itemize}

In \href{https://wiki.portal.chalmers.se/agda/pmwiki.php}{Agda} the term
algebra can be defined as simply as one might hope.

\begin{code}%
\>[0]\<%
\\
\>[0]\AgdaFunction{TermSetoid}\AgdaSpace{}%
\AgdaSymbol{:}\AgdaSpace{}%
\AgdaSymbol{(}\AgdaBound{X}\AgdaSpace{}%
\AgdaSymbol{:}\AgdaSpace{}%
\AgdaPrimitive{Type}\AgdaSpace{}%
\AgdaGeneralizable{χ}\AgdaSymbol{)}\AgdaSpace{}%
\AgdaSymbol{→}\AgdaSpace{}%
\AgdaRecord{Setoid}\AgdaSpace{}%
\AgdaSymbol{(}\AgdaFunction{ov}\AgdaSpace{}%
\AgdaGeneralizable{χ}\AgdaSymbol{)}\AgdaSpace{}%
\AgdaSymbol{(}\AgdaFunction{ov}\AgdaSpace{}%
\AgdaGeneralizable{χ}\AgdaSymbol{)}\<%
\\
\>[0]\AgdaFunction{TermSetoid}\AgdaSpace{}%
\AgdaBound{X}\AgdaSpace{}%
\AgdaSymbol{=}\AgdaSpace{}%
\AgdaKeyword{record}\AgdaSpace{}%
\AgdaSymbol{\{}\AgdaSpace{}%
\AgdaField{Carrier}\AgdaSpace{}%
\AgdaSymbol{=}\AgdaSpace{}%
\AgdaDatatype{Term}\AgdaSpace{}%
\AgdaBound{X}\AgdaSpace{}%
\AgdaSymbol{;}\AgdaSpace{}%
\AgdaOperator{\AgdaField{\AgdaUnderscore{}≈\AgdaUnderscore{}}}\AgdaSpace{}%
\AgdaSymbol{=}\AgdaSpace{}%
\AgdaOperator{\AgdaDatatype{\AgdaUnderscore{}≐\AgdaUnderscore{}}}\AgdaSpace{}%
\AgdaSymbol{;}\AgdaSpace{}%
\AgdaField{isEquivalence}\AgdaSpace{}%
\AgdaSymbol{=}\AgdaSpace{}%
\AgdaFunction{≐-isEquiv}\AgdaSpace{}%
\AgdaSymbol{\}}\<%
\\
%
\\[\AgdaEmptyExtraSkip]%
\>[0]\AgdaFunction{𝑻}\AgdaSpace{}%
\AgdaSymbol{:}\AgdaSpace{}%
\AgdaSymbol{(}\AgdaBound{X}\AgdaSpace{}%
\AgdaSymbol{:}\AgdaSpace{}%
\AgdaPrimitive{Type}\AgdaSpace{}%
\AgdaGeneralizable{χ}\AgdaSymbol{)}\AgdaSpace{}%
\AgdaSymbol{→}\AgdaSpace{}%
\AgdaRecord{Algebra}\AgdaSpace{}%
\AgdaSymbol{(}\AgdaFunction{ov}\AgdaSpace{}%
\AgdaGeneralizable{χ}\AgdaSymbol{)}\AgdaSpace{}%
\AgdaSymbol{(}\AgdaFunction{ov}\AgdaSpace{}%
\AgdaGeneralizable{χ}\AgdaSymbol{)}\<%
\\
\>[0]\AgdaField{Algebra.Domain}\AgdaSpace{}%
\AgdaSymbol{(}\AgdaFunction{𝑻}\AgdaSpace{}%
\AgdaBound{X}\AgdaSymbol{)}\AgdaSpace{}%
\AgdaSymbol{=}\AgdaSpace{}%
\AgdaFunction{TermSetoid}\AgdaSpace{}%
\AgdaBound{X}\<%
\\
\>[0]\AgdaField{Algebra.Interp}\AgdaSpace{}%
\AgdaSymbol{(}\AgdaFunction{𝑻}\AgdaSpace{}%
\AgdaBound{X}\AgdaSymbol{)}\AgdaSpace{}%
\AgdaOperator{\AgdaField{⟨\$⟩}}\AgdaSpace{}%
\AgdaSymbol{(}\AgdaBound{f}\AgdaSpace{}%
\AgdaOperator{\AgdaInductiveConstructor{,}}\AgdaSpace{}%
\AgdaBound{ts}\AgdaSymbol{)}\AgdaSpace{}%
\AgdaSymbol{=}\AgdaSpace{}%
\AgdaInductiveConstructor{node}\AgdaSpace{}%
\AgdaBound{f}\AgdaSpace{}%
\AgdaBound{ts}\<%
\\
\>[0]\AgdaField{cong}\AgdaSpace{}%
\AgdaSymbol{(}\AgdaField{Algebra.Interp}\AgdaSpace{}%
\AgdaSymbol{(}\AgdaFunction{𝑻}\AgdaSpace{}%
\AgdaBound{X}\AgdaSymbol{))}\AgdaSpace{}%
\AgdaSymbol{(}\AgdaInductiveConstructor{≡.refl}\AgdaSpace{}%
\AgdaOperator{\AgdaInductiveConstructor{,}}\AgdaSpace{}%
\AgdaBound{ss≐ts}\AgdaSymbol{)}\AgdaSpace{}%
\AgdaSymbol{=}\AgdaSpace{}%
\AgdaInductiveConstructor{gnl}\AgdaSpace{}%
\AgdaBound{ss≐ts}\<%
\end{code}

\hypertarget{interpretation-of-terms}{%
\subsubsection{Interpretation of terms}\label{interpretation-of-terms}}

The approach to terms and their interpretation in this module was
inspired by
\href{http://www.cse.chalmers.se/~abela/agda/MultiSortedAlgebra.pdf}{Andreas
Abel's formal proof of Birkhoff's completeness theorem}.

A substitution from \texttt{X} to \texttt{Y} associates a term in
\texttt{X} with each variable in \texttt{Y}. The definition of
\texttt{Sub} given here is essentially the same as the one given by
Andreas Abel, as is the recursive definition of the syntax
\texttt{t\ {[}\ σ\ {]}}, which denotes a term \texttt{t} applied to a
substitution \texttt{σ}.

\begin{code}%
\>[0]\<%
\\
\>[0]\AgdaFunction{Sub}\AgdaSpace{}%
\AgdaSymbol{:}\AgdaSpace{}%
\AgdaPrimitive{Type}\AgdaSpace{}%
\AgdaGeneralizable{χ}\AgdaSpace{}%
\AgdaSymbol{→}\AgdaSpace{}%
\AgdaPrimitive{Type}\AgdaSpace{}%
\AgdaGeneralizable{χ}\AgdaSpace{}%
\AgdaSymbol{→}\AgdaSpace{}%
\AgdaPrimitive{Type}\AgdaSpace{}%
\AgdaSymbol{(}\AgdaFunction{ov}\AgdaSpace{}%
\AgdaGeneralizable{χ}\AgdaSymbol{)}\<%
\\
\>[0]\AgdaFunction{Sub}\AgdaSpace{}%
\AgdaBound{X}\AgdaSpace{}%
\AgdaBound{Y}\AgdaSpace{}%
\AgdaSymbol{=}\AgdaSpace{}%
\AgdaSymbol{(}\AgdaBound{y}\AgdaSpace{}%
\AgdaSymbol{:}\AgdaSpace{}%
\AgdaBound{Y}\AgdaSymbol{)}\AgdaSpace{}%
\AgdaSymbol{→}\AgdaSpace{}%
\AgdaDatatype{Term}\AgdaSpace{}%
\AgdaBound{X}\<%
\\
%
\\[\AgdaEmptyExtraSkip]%
\>[0]\AgdaOperator{\AgdaFunction{\AgdaUnderscore{}[\AgdaUnderscore{}]}}\AgdaSpace{}%
\AgdaSymbol{:}\AgdaSpace{}%
\AgdaSymbol{\{}\AgdaBound{X}\AgdaSpace{}%
\AgdaBound{Y}\AgdaSpace{}%
\AgdaSymbol{:}\AgdaSpace{}%
\AgdaPrimitive{Type}\AgdaSpace{}%
\AgdaGeneralizable{χ}\AgdaSymbol{\}(}\AgdaBound{t}\AgdaSpace{}%
\AgdaSymbol{:}\AgdaSpace{}%
\AgdaDatatype{Term}\AgdaSpace{}%
\AgdaBound{Y}\AgdaSymbol{)}\AgdaSpace{}%
\AgdaSymbol{(}\AgdaBound{σ}\AgdaSpace{}%
\AgdaSymbol{:}\AgdaSpace{}%
\AgdaFunction{Sub}\AgdaSpace{}%
\AgdaBound{X}\AgdaSpace{}%
\AgdaBound{Y}\AgdaSymbol{)}\AgdaSpace{}%
\AgdaSymbol{→}\AgdaSpace{}%
\AgdaDatatype{Term}\AgdaSpace{}%
\AgdaBound{X}\<%
\\
\>[0]\AgdaSymbol{(}\AgdaInductiveConstructor{ℊ}\AgdaSpace{}%
\AgdaBound{x}\AgdaSymbol{)}\AgdaSpace{}%
\AgdaOperator{\AgdaFunction{[}}\AgdaSpace{}%
\AgdaBound{σ}\AgdaSpace{}%
\AgdaOperator{\AgdaFunction{]}}\AgdaSpace{}%
\AgdaSymbol{=}\AgdaSpace{}%
\AgdaBound{σ}\AgdaSpace{}%
\AgdaBound{x}\<%
\\
\>[0]\AgdaSymbol{(}\AgdaInductiveConstructor{node}\AgdaSpace{}%
\AgdaBound{f}\AgdaSpace{}%
\AgdaBound{ts}\AgdaSymbol{)}\AgdaSpace{}%
\AgdaOperator{\AgdaFunction{[}}\AgdaSpace{}%
\AgdaBound{σ}\AgdaSpace{}%
\AgdaOperator{\AgdaFunction{]}}\AgdaSpace{}%
\AgdaSymbol{=}\AgdaSpace{}%
\AgdaInductiveConstructor{node}\AgdaSpace{}%
\AgdaBound{f}\AgdaSpace{}%
\AgdaSymbol{(λ}\AgdaSpace{}%
\AgdaBound{i}\AgdaSpace{}%
\AgdaSymbol{→}\AgdaSpace{}%
\AgdaBound{ts}\AgdaSpace{}%
\AgdaBound{i}\AgdaSpace{}%
\AgdaOperator{\AgdaFunction{[}}\AgdaSpace{}%
\AgdaBound{σ}\AgdaSpace{}%
\AgdaOperator{\AgdaFunction{]}}\AgdaSymbol{)}\<%
\\
\>[0]\<%
\end{code}

An environment for an algebra \texttt{𝑨} in a context \texttt{X} is a
map that assigns to each variable \texttt{x\ :\ X} an element in the
domain of \texttt{𝑨}, packaged together with an equality of
environments, which is simply pointwise equality (relatively to the
setoid equality of the underlying domain of \texttt{𝑨}).

\begin{code}%
\>[0]\<%
\\
\>[0]\AgdaKeyword{module}\AgdaSpace{}%
\AgdaModule{Environment}\AgdaSpace{}%
\AgdaSymbol{(}\AgdaBound{𝑨}\AgdaSpace{}%
\AgdaSymbol{:}\AgdaSpace{}%
\AgdaRecord{Algebra}\AgdaSpace{}%
\AgdaGeneralizable{α}\AgdaSpace{}%
\AgdaGeneralizable{ℓ}\AgdaSymbol{)}\AgdaSpace{}%
\AgdaKeyword{where}\<%
\\
\>[0][@{}l@{\AgdaIndent{0}}]%
\>[1]\AgdaKeyword{open}\AgdaSpace{}%
\AgdaModule{Setoid}\AgdaSpace{}%
\AgdaOperator{\AgdaFunction{𝔻[}}\AgdaSpace{}%
\AgdaBound{𝑨}\AgdaSpace{}%
\AgdaOperator{\AgdaFunction{]}}\AgdaSpace{}%
\AgdaKeyword{using}\AgdaSpace{}%
\AgdaSymbol{(}\AgdaSpace{}%
\AgdaOperator{\AgdaField{\AgdaUnderscore{}≈\AgdaUnderscore{}}}\AgdaSpace{}%
\AgdaSymbol{;}\AgdaSpace{}%
\AgdaFunction{refl}\AgdaSpace{}%
\AgdaSymbol{;}\AgdaSpace{}%
\AgdaFunction{sym}\AgdaSpace{}%
\AgdaSymbol{;}\AgdaSpace{}%
\AgdaFunction{trans}\AgdaSpace{}%
\AgdaSymbol{)}\<%
\\
%
\>[1]\AgdaFunction{Env}\AgdaSpace{}%
\AgdaSymbol{:}\AgdaSpace{}%
\AgdaPrimitive{Type}\AgdaSpace{}%
\AgdaGeneralizable{χ}\AgdaSpace{}%
\AgdaSymbol{→}\AgdaSpace{}%
\AgdaRecord{Setoid}\AgdaSpace{}%
\AgdaSymbol{\AgdaUnderscore{}}\AgdaSpace{}%
\AgdaSymbol{\AgdaUnderscore{}}\<%
\\
%
\>[1]\AgdaFunction{Env}\AgdaSpace{}%
\AgdaBound{X}\AgdaSpace{}%
\AgdaSymbol{=}\AgdaSpace{}%
\AgdaKeyword{record}%
\>[17]\AgdaSymbol{\{}\AgdaSpace{}%
\AgdaField{Carrier}\AgdaSpace{}%
\AgdaSymbol{=}\AgdaSpace{}%
\AgdaBound{X}\AgdaSpace{}%
\AgdaSymbol{→}\AgdaSpace{}%
\AgdaOperator{\AgdaFunction{𝕌[}}\AgdaSpace{}%
\AgdaBound{𝑨}\AgdaSpace{}%
\AgdaOperator{\AgdaFunction{]}}\<%
\\
%
\>[17]\AgdaSymbol{;}\AgdaSpace{}%
\AgdaOperator{\AgdaField{\AgdaUnderscore{}≈\AgdaUnderscore{}}}\AgdaSpace{}%
\AgdaSymbol{=}\AgdaSpace{}%
\AgdaSymbol{λ}\AgdaSpace{}%
\AgdaBound{ρ}\AgdaSpace{}%
\AgdaBound{ρ'}\AgdaSpace{}%
\AgdaSymbol{→}\AgdaSpace{}%
\AgdaSymbol{(}\AgdaBound{x}\AgdaSpace{}%
\AgdaSymbol{:}\AgdaSpace{}%
\AgdaBound{X}\AgdaSymbol{)}\AgdaSpace{}%
\AgdaSymbol{→}\AgdaSpace{}%
\AgdaBound{ρ}\AgdaSpace{}%
\AgdaBound{x}\AgdaSpace{}%
\AgdaOperator{\AgdaFunction{≈}}\AgdaSpace{}%
\AgdaBound{ρ'}\AgdaSpace{}%
\AgdaBound{x}\<%
\\
%
\>[17]\AgdaSymbol{;}\AgdaSpace{}%
\AgdaField{isEquivalence}\AgdaSpace{}%
\AgdaSymbol{=}\AgdaSpace{}%
\AgdaKeyword{record}%
\>[43]\AgdaSymbol{\{}\AgdaSpace{}%
\AgdaField{refl}%
\>[52]\AgdaSymbol{=}\AgdaSpace{}%
\AgdaSymbol{λ}\AgdaSpace{}%
\AgdaBound{\AgdaUnderscore{}}\AgdaSpace{}%
\AgdaSymbol{→}\AgdaSpace{}%
\AgdaFunction{refl}\<%
\\
%
\>[43]\AgdaSymbol{;}\AgdaSpace{}%
\AgdaField{sym}%
\>[52]\AgdaSymbol{=}\AgdaSpace{}%
\AgdaSymbol{λ}\AgdaSpace{}%
\AgdaBound{h}\AgdaSpace{}%
\AgdaBound{x}\AgdaSpace{}%
\AgdaSymbol{→}\AgdaSpace{}%
\AgdaFunction{sym}\AgdaSpace{}%
\AgdaSymbol{(}\AgdaBound{h}\AgdaSpace{}%
\AgdaBound{x}\AgdaSymbol{)}\<%
\\
%
\>[43]\AgdaSymbol{;}\AgdaSpace{}%
\AgdaField{trans}%
\>[52]\AgdaSymbol{=}\AgdaSpace{}%
\AgdaSymbol{λ}\AgdaSpace{}%
\AgdaBound{g}\AgdaSpace{}%
\AgdaBound{h}\AgdaSpace{}%
\AgdaBound{x}\AgdaSpace{}%
\AgdaSymbol{→}\AgdaSpace{}%
\AgdaFunction{trans}\AgdaSpace{}%
\AgdaSymbol{(}\AgdaBound{g}\AgdaSpace{}%
\AgdaBound{x}\AgdaSymbol{)(}\AgdaBound{h}\AgdaSpace{}%
\AgdaBound{x}\AgdaSymbol{)}\AgdaSpace{}%
\AgdaSymbol{\}\}}\<%
\\
\>[0]\<%
\end{code}

Next we define \emph{evaluation of a term} in an environment \texttt{ρ},
interpreted in the algebra \texttt{𝑨}.

\begin{code}%
\>[0]\<%
\\
\>[0][@{}l@{\AgdaIndent{1}}]%
\>[1]\AgdaOperator{\AgdaFunction{⟦\AgdaUnderscore{}⟧}}\AgdaSpace{}%
\AgdaSymbol{:}\AgdaSpace{}%
\AgdaSymbol{\{}\AgdaBound{X}\AgdaSpace{}%
\AgdaSymbol{:}\AgdaSpace{}%
\AgdaPrimitive{Type}\AgdaSpace{}%
\AgdaGeneralizable{χ}\AgdaSymbol{\}(}\AgdaBound{t}\AgdaSpace{}%
\AgdaSymbol{:}\AgdaSpace{}%
\AgdaDatatype{Term}\AgdaSpace{}%
\AgdaBound{X}\AgdaSymbol{)}\AgdaSpace{}%
\AgdaSymbol{→}\AgdaSpace{}%
\AgdaSymbol{(}\AgdaFunction{Env}\AgdaSpace{}%
\AgdaBound{X}\AgdaSymbol{)}\AgdaSpace{}%
\AgdaOperator{\AgdaRecord{⟶}}\AgdaSpace{}%
\AgdaOperator{\AgdaFunction{𝔻[}}\AgdaSpace{}%
\AgdaBound{𝑨}\AgdaSpace{}%
\AgdaOperator{\AgdaFunction{]}}\<%
\\
%
\>[1]\AgdaOperator{\AgdaFunction{⟦}}\AgdaSpace{}%
\AgdaInductiveConstructor{ℊ}\AgdaSpace{}%
\AgdaBound{x}\AgdaSpace{}%
\AgdaOperator{\AgdaFunction{⟧}}\AgdaSpace{}%
\AgdaOperator{\AgdaField{⟨\$⟩}}\AgdaSpace{}%
\AgdaBound{ρ}\AgdaSpace{}%
\AgdaSymbol{=}\AgdaSpace{}%
\AgdaBound{ρ}\AgdaSpace{}%
\AgdaBound{x}\<%
\\
%
\>[1]\AgdaOperator{\AgdaFunction{⟦}}\AgdaSpace{}%
\AgdaInductiveConstructor{node}\AgdaSpace{}%
\AgdaBound{f}\AgdaSpace{}%
\AgdaBound{args}\AgdaSpace{}%
\AgdaOperator{\AgdaFunction{⟧}}\AgdaSpace{}%
\AgdaOperator{\AgdaField{⟨\$⟩}}\AgdaSpace{}%
\AgdaBound{ρ}\AgdaSpace{}%
\AgdaSymbol{=}\AgdaSpace{}%
\AgdaSymbol{(}\AgdaField{Interp}\AgdaSpace{}%
\AgdaBound{𝑨}\AgdaSymbol{)}\AgdaSpace{}%
\AgdaOperator{\AgdaField{⟨\$⟩}}\AgdaSpace{}%
\AgdaSymbol{(}\AgdaBound{f}\AgdaSpace{}%
\AgdaOperator{\AgdaInductiveConstructor{,}}\AgdaSpace{}%
\AgdaSymbol{λ}\AgdaSpace{}%
\AgdaBound{i}\AgdaSpace{}%
\AgdaSymbol{→}\AgdaSpace{}%
\AgdaOperator{\AgdaFunction{⟦}}\AgdaSpace{}%
\AgdaBound{args}\AgdaSpace{}%
\AgdaBound{i}\AgdaSpace{}%
\AgdaOperator{\AgdaFunction{⟧}}\AgdaSpace{}%
\AgdaOperator{\AgdaField{⟨\$⟩}}\AgdaSpace{}%
\AgdaBound{ρ}\AgdaSymbol{)}\<%
\\
%
\>[1]\AgdaField{cong}\AgdaSpace{}%
\AgdaOperator{\AgdaFunction{⟦}}\AgdaSpace{}%
\AgdaInductiveConstructor{ℊ}\AgdaSpace{}%
\AgdaBound{x}\AgdaSpace{}%
\AgdaOperator{\AgdaFunction{⟧}}\AgdaSpace{}%
\AgdaBound{u≈v}\AgdaSpace{}%
\AgdaSymbol{=}\AgdaSpace{}%
\AgdaBound{u≈v}\AgdaSpace{}%
\AgdaBound{x}\<%
\\
%
\>[1]\AgdaField{cong}\AgdaSpace{}%
\AgdaOperator{\AgdaFunction{⟦}}\AgdaSpace{}%
\AgdaInductiveConstructor{node}\AgdaSpace{}%
\AgdaBound{f}\AgdaSpace{}%
\AgdaBound{args}\AgdaSpace{}%
\AgdaOperator{\AgdaFunction{⟧}}\AgdaSpace{}%
\AgdaBound{x≈y}\AgdaSpace{}%
\AgdaSymbol{=}\AgdaSpace{}%
\AgdaField{cong}\AgdaSpace{}%
\AgdaSymbol{(}\AgdaField{Interp}\AgdaSpace{}%
\AgdaBound{𝑨}\AgdaSymbol{)(}\AgdaInductiveConstructor{≡.refl}\AgdaSpace{}%
\AgdaOperator{\AgdaInductiveConstructor{,}}\AgdaSpace{}%
\AgdaSymbol{λ}\AgdaSpace{}%
\AgdaBound{i}\AgdaSpace{}%
\AgdaSymbol{→}\AgdaSpace{}%
\AgdaField{cong}\AgdaSpace{}%
\AgdaOperator{\AgdaFunction{⟦}}\AgdaSpace{}%
\AgdaBound{args}\AgdaSpace{}%
\AgdaBound{i}\AgdaSpace{}%
\AgdaOperator{\AgdaFunction{⟧}}\AgdaSpace{}%
\AgdaBound{x≈y}\AgdaSpace{}%
\AgdaSymbol{)}\<%
\\
\>[0]\<%
\end{code}

An equality between two terms holds in a model if the two terms are
equal under all valuations of their free variables
(cf.~\href{http://www.cse.chalmers.se/~abela/agda/MultiSortedAlgebra.pdf}{Andreas
Abel's formal proof of Birkhoff's completeness theorem}).

\begin{code}%
\>[0]\<%
\\
\>[0][@{}l@{\AgdaIndent{1}}]%
\>[1]\AgdaFunction{Equal}\AgdaSpace{}%
\AgdaSymbol{:}\AgdaSpace{}%
\AgdaSymbol{∀}\AgdaSpace{}%
\AgdaSymbol{\{}\AgdaBound{X}\AgdaSpace{}%
\AgdaSymbol{:}\AgdaSpace{}%
\AgdaPrimitive{Type}\AgdaSpace{}%
\AgdaGeneralizable{χ}\AgdaSymbol{\}}\AgdaSpace{}%
\AgdaSymbol{(}\AgdaBound{s}\AgdaSpace{}%
\AgdaBound{t}\AgdaSpace{}%
\AgdaSymbol{:}\AgdaSpace{}%
\AgdaDatatype{Term}\AgdaSpace{}%
\AgdaBound{X}\AgdaSymbol{)}\AgdaSpace{}%
\AgdaSymbol{→}\AgdaSpace{}%
\AgdaPrimitive{Type}\AgdaSpace{}%
\AgdaSymbol{\AgdaUnderscore{}}\<%
\\
%
\>[1]\AgdaFunction{Equal}\AgdaSpace{}%
\AgdaSymbol{\{}\AgdaArgument{X}\AgdaSpace{}%
\AgdaSymbol{=}\AgdaSpace{}%
\AgdaBound{X}\AgdaSymbol{\}}\AgdaSpace{}%
\AgdaBound{s}\AgdaSpace{}%
\AgdaBound{t}\AgdaSpace{}%
\AgdaSymbol{=}\AgdaSpace{}%
\AgdaSymbol{∀}\AgdaSpace{}%
\AgdaSymbol{(}\AgdaBound{ρ}\AgdaSpace{}%
\AgdaSymbol{:}\AgdaSpace{}%
\AgdaField{Carrier}\AgdaSpace{}%
\AgdaSymbol{(}\AgdaFunction{Env}\AgdaSpace{}%
\AgdaBound{X}\AgdaSymbol{))}\AgdaSpace{}%
\AgdaSymbol{→}%
\>[48]\AgdaOperator{\AgdaFunction{⟦}}\AgdaSpace{}%
\AgdaBound{s}\AgdaSpace{}%
\AgdaOperator{\AgdaFunction{⟧}}\AgdaSpace{}%
\AgdaOperator{\AgdaField{⟨\$⟩}}\AgdaSpace{}%
\AgdaBound{ρ}\AgdaSpace{}%
\AgdaOperator{\AgdaFunction{≈}}\AgdaSpace{}%
\AgdaOperator{\AgdaFunction{⟦}}\AgdaSpace{}%
\AgdaBound{t}\AgdaSpace{}%
\AgdaOperator{\AgdaFunction{⟧}}\AgdaSpace{}%
\AgdaOperator{\AgdaField{⟨\$⟩}}\AgdaSpace{}%
\AgdaBound{ρ}\<%
\\
%
\\[\AgdaEmptyExtraSkip]%
%
\>[1]\AgdaFunction{≐→Equal}\AgdaSpace{}%
\AgdaSymbol{:}\AgdaSpace{}%
\AgdaSymbol{\{}\AgdaBound{X}\AgdaSpace{}%
\AgdaSymbol{:}\AgdaSpace{}%
\AgdaPrimitive{Type}\AgdaSpace{}%
\AgdaGeneralizable{χ}\AgdaSymbol{\}(}\AgdaBound{s}\AgdaSpace{}%
\AgdaBound{t}\AgdaSpace{}%
\AgdaSymbol{:}\AgdaSpace{}%
\AgdaDatatype{Term}\AgdaSpace{}%
\AgdaBound{X}\AgdaSymbol{)}\AgdaSpace{}%
\AgdaSymbol{→}\AgdaSpace{}%
\AgdaBound{s}\AgdaSpace{}%
\AgdaOperator{\AgdaDatatype{≐}}\AgdaSpace{}%
\AgdaBound{t}\AgdaSpace{}%
\AgdaSymbol{→}\AgdaSpace{}%
\AgdaFunction{Equal}\AgdaSpace{}%
\AgdaBound{s}\AgdaSpace{}%
\AgdaBound{t}\<%
\\
%
\>[1]\AgdaFunction{≐→Equal}\AgdaSpace{}%
\AgdaDottedPattern{\AgdaSymbol{.(}}\AgdaDottedPattern{\AgdaInductiveConstructor{ℊ}}\AgdaSpace{}%
\AgdaDottedPattern{\AgdaSymbol{\AgdaUnderscore{})}}\AgdaSpace{}%
\AgdaDottedPattern{\AgdaSymbol{.(}}\AgdaDottedPattern{\AgdaInductiveConstructor{ℊ}}\AgdaSpace{}%
\AgdaDottedPattern{\AgdaSymbol{\AgdaUnderscore{})}}\AgdaSpace{}%
\AgdaSymbol{(}\AgdaInductiveConstructor{rfl}\AgdaSpace{}%
\AgdaInductiveConstructor{≡.refl}\AgdaSymbol{)}\AgdaSpace{}%
\AgdaSymbol{=}\AgdaSpace{}%
\AgdaSymbol{λ}\AgdaSpace{}%
\AgdaBound{\AgdaUnderscore{}}\AgdaSpace{}%
\AgdaSymbol{→}\AgdaSpace{}%
\AgdaFunction{refl}\<%
\\
%
\>[1]\AgdaFunction{≐→Equal}\AgdaSpace{}%
\AgdaSymbol{(}\AgdaInductiveConstructor{node}\AgdaSpace{}%
\AgdaSymbol{\AgdaUnderscore{}}\AgdaSpace{}%
\AgdaBound{s}\AgdaSymbol{)(}\AgdaInductiveConstructor{node}\AgdaSpace{}%
\AgdaSymbol{\AgdaUnderscore{}}\AgdaSpace{}%
\AgdaBound{t}\AgdaSymbol{)(}\AgdaInductiveConstructor{gnl}\AgdaSpace{}%
\AgdaBound{x}\AgdaSymbol{)}\AgdaSpace{}%
\AgdaSymbol{=}\<%
\\
\>[1][@{}l@{\AgdaIndent{0}}]%
\>[2]\AgdaSymbol{λ}\AgdaSpace{}%
\AgdaBound{ρ}\AgdaSpace{}%
\AgdaSymbol{→}\AgdaSpace{}%
\AgdaField{cong}\AgdaSpace{}%
\AgdaSymbol{(}\AgdaField{Interp}\AgdaSpace{}%
\AgdaBound{𝑨}\AgdaSymbol{)(}\AgdaInductiveConstructor{≡.refl}\AgdaSpace{}%
\AgdaOperator{\AgdaInductiveConstructor{,}}\AgdaSpace{}%
\AgdaSymbol{λ}\AgdaSpace{}%
\AgdaBound{i}\AgdaSpace{}%
\AgdaSymbol{→}\AgdaSpace{}%
\AgdaFunction{≐→Equal}\AgdaSymbol{(}\AgdaBound{s}\AgdaSpace{}%
\AgdaBound{i}\AgdaSymbol{)(}\AgdaBound{t}\AgdaSpace{}%
\AgdaBound{i}\AgdaSymbol{)(}\AgdaBound{x}\AgdaSpace{}%
\AgdaBound{i}\AgdaSymbol{)}\AgdaBound{ρ}\AgdaSpace{}%
\AgdaSymbol{)}\<%
\\
\>[0]\<%
\end{code}

The proof that \texttt{Equal} is an equivalence relation is trivial, so
we omit it. (See the
\href{https://ualib.github.io/agda-algebras/Varieties.Func.SoundAndComplete.html}{Varieties.Func.SoundAndComplete}
module of the
\href{https://github.com/ualib/agda-algebras}{agda-algebras} library for
details.)

\begin{code}[hide]%
\>[0][@{}l@{\AgdaIndent{1}}]%
\>[1]\AgdaFunction{EqualIsEquiv}\AgdaSpace{}%
\AgdaSymbol{:}\AgdaSpace{}%
\AgdaSymbol{\{}\AgdaBound{Γ}\AgdaSpace{}%
\AgdaSymbol{:}\AgdaSpace{}%
\AgdaPrimitive{Type}\AgdaSpace{}%
\AgdaGeneralizable{χ}\AgdaSymbol{\}}\AgdaSpace{}%
\AgdaSymbol{→}\AgdaSpace{}%
\AgdaRecord{IsEquivalence}\AgdaSpace{}%
\AgdaSymbol{(}\AgdaFunction{Equal}\AgdaSpace{}%
\AgdaSymbol{\{}\AgdaArgument{X}\AgdaSpace{}%
\AgdaSymbol{=}\AgdaSpace{}%
\AgdaBound{Γ}\AgdaSymbol{\})}\<%
\\
%
\>[1]\AgdaField{IsEquivalence.refl}%
\>[21]\AgdaFunction{EqualIsEquiv}\AgdaSpace{}%
\AgdaSymbol{=}\AgdaSpace{}%
\AgdaSymbol{λ}\AgdaSpace{}%
\AgdaBound{\AgdaUnderscore{}}\AgdaSpace{}%
\AgdaSymbol{→}\AgdaSpace{}%
\AgdaFunction{refl}\<%
\\
%
\>[1]\AgdaField{IsEquivalence.sym}%
\>[21]\AgdaFunction{EqualIsEquiv}\AgdaSpace{}%
\AgdaSymbol{=}\AgdaSpace{}%
\AgdaSymbol{λ}\AgdaSpace{}%
\AgdaBound{x=y}\AgdaSpace{}%
\AgdaBound{ρ}\AgdaSpace{}%
\AgdaSymbol{→}\AgdaSpace{}%
\AgdaFunction{sym}\AgdaSpace{}%
\AgdaSymbol{(}\AgdaBound{x=y}\AgdaSpace{}%
\AgdaBound{ρ}\AgdaSymbol{)}\<%
\\
%
\>[1]\AgdaField{IsEquivalence.trans}\AgdaSpace{}%
\AgdaFunction{EqualIsEquiv}\AgdaSpace{}%
\AgdaSymbol{=}\AgdaSpace{}%
\AgdaSymbol{λ}\AgdaSpace{}%
\AgdaBound{ij}\AgdaSpace{}%
\AgdaBound{jk}\AgdaSpace{}%
\AgdaBound{ρ}\AgdaSpace{}%
\AgdaSymbol{→}\AgdaSpace{}%
\AgdaFunction{trans}\AgdaSpace{}%
\AgdaSymbol{(}\AgdaBound{ij}\AgdaSpace{}%
\AgdaBound{ρ}\AgdaSymbol{)}\AgdaSpace{}%
\AgdaSymbol{(}\AgdaBound{jk}\AgdaSpace{}%
\AgdaBound{ρ}\AgdaSymbol{)}\<%
\end{code}

Evaluation of a substitution gives an environment
(cf.~\href{http://www.cse.chalmers.se/~abela/agda/MultiSortedAlgebra.pdf}{Andreas
Abel's formal proof of Birkhoff's completeness theorem})

\begin{code}%
\>[0]\<%
\\
%
\>[1]\AgdaOperator{\AgdaFunction{⟦\AgdaUnderscore{}⟧s}}\AgdaSpace{}%
\AgdaSymbol{:}\AgdaSpace{}%
\AgdaSymbol{\{}\AgdaBound{X}\AgdaSpace{}%
\AgdaBound{Y}\AgdaSpace{}%
\AgdaSymbol{:}\AgdaSpace{}%
\AgdaPrimitive{Type}\AgdaSpace{}%
\AgdaGeneralizable{χ}\AgdaSymbol{\}}\AgdaSpace{}%
\AgdaSymbol{→}\AgdaSpace{}%
\AgdaFunction{Sub}\AgdaSpace{}%
\AgdaBound{X}\AgdaSpace{}%
\AgdaBound{Y}\AgdaSpace{}%
\AgdaSymbol{→}\AgdaSpace{}%
\AgdaField{Carrier}\AgdaSymbol{(}\AgdaFunction{Env}\AgdaSpace{}%
\AgdaBound{X}\AgdaSymbol{)}\AgdaSpace{}%
\AgdaSymbol{→}\AgdaSpace{}%
\AgdaField{Carrier}\AgdaSpace{}%
\AgdaSymbol{(}\AgdaFunction{Env}\AgdaSpace{}%
\AgdaBound{Y}\AgdaSymbol{)}\<%
\\
%
\>[1]\AgdaOperator{\AgdaFunction{⟦}}\AgdaSpace{}%
\AgdaBound{σ}\AgdaSpace{}%
\AgdaOperator{\AgdaFunction{⟧s}}\AgdaSpace{}%
\AgdaBound{ρ}\AgdaSpace{}%
\AgdaBound{x}\AgdaSpace{}%
\AgdaSymbol{=}\AgdaSpace{}%
\AgdaOperator{\AgdaFunction{⟦}}\AgdaSpace{}%
\AgdaBound{σ}\AgdaSpace{}%
\AgdaBound{x}\AgdaSpace{}%
\AgdaOperator{\AgdaFunction{⟧}}\AgdaSpace{}%
\AgdaOperator{\AgdaField{⟨\$⟩}}\AgdaSpace{}%
\AgdaBound{ρ}\<%
\\
\>[0]\<%
\end{code}

Next we prove that ⟦t{[}σ{]}⟧ρ ≃ ⟦t⟧⟦σ⟧ρ
(cf.~\href{http://www.cse.chalmers.se/~abela/agda/MultiSortedAlgebra.pdf}{Andreas
Abel's formal proof of Birkhoff's completeness theorem}).

\begin{code}%
\>[0]\<%
\\
\>[0][@{}l@{\AgdaIndent{1}}]%
\>[1]\AgdaFunction{substitution}\AgdaSpace{}%
\AgdaSymbol{:}\AgdaSpace{}%
\AgdaSymbol{\{}\AgdaBound{X}\AgdaSpace{}%
\AgdaBound{Y}\AgdaSpace{}%
\AgdaSymbol{:}\AgdaSpace{}%
\AgdaPrimitive{Type}\AgdaSpace{}%
\AgdaGeneralizable{χ}\AgdaSymbol{\}}\AgdaSpace{}%
\AgdaSymbol{→}\AgdaSpace{}%
\AgdaSymbol{(}\AgdaBound{t}\AgdaSpace{}%
\AgdaSymbol{:}\AgdaSpace{}%
\AgdaDatatype{Term}\AgdaSpace{}%
\AgdaBound{Y}\AgdaSymbol{)}\AgdaSpace{}%
\AgdaSymbol{(}\AgdaBound{σ}\AgdaSpace{}%
\AgdaSymbol{:}\AgdaSpace{}%
\AgdaFunction{Sub}\AgdaSpace{}%
\AgdaBound{X}\AgdaSpace{}%
\AgdaBound{Y}\AgdaSymbol{)}\AgdaSpace{}%
\AgdaSymbol{(}\AgdaBound{ρ}\AgdaSpace{}%
\AgdaSymbol{:}\AgdaSpace{}%
\AgdaField{Carrier}\AgdaSymbol{(}\AgdaSpace{}%
\AgdaFunction{Env}\AgdaSpace{}%
\AgdaBound{X}\AgdaSpace{}%
\AgdaSymbol{)}\AgdaSpace{}%
\AgdaSymbol{)}\<%
\\
\>[1][@{}l@{\AgdaIndent{0}}]%
\>[2]\AgdaSymbol{→}%
\>[16]\AgdaOperator{\AgdaFunction{⟦}}\AgdaSpace{}%
\AgdaBound{t}\AgdaSpace{}%
\AgdaOperator{\AgdaFunction{[}}\AgdaSpace{}%
\AgdaBound{σ}\AgdaSpace{}%
\AgdaOperator{\AgdaFunction{]}}\AgdaSpace{}%
\AgdaOperator{\AgdaFunction{⟧}}\AgdaSpace{}%
\AgdaOperator{\AgdaField{⟨\$⟩}}\AgdaSpace{}%
\AgdaBound{ρ}%
\>[35]\AgdaOperator{\AgdaFunction{≈}}%
\>[38]\AgdaOperator{\AgdaFunction{⟦}}\AgdaSpace{}%
\AgdaBound{t}\AgdaSpace{}%
\AgdaOperator{\AgdaFunction{⟧}}\AgdaSpace{}%
\AgdaOperator{\AgdaField{⟨\$⟩}}\AgdaSpace{}%
\AgdaSymbol{(}\AgdaOperator{\AgdaFunction{⟦}}\AgdaSpace{}%
\AgdaBound{σ}\AgdaSpace{}%
\AgdaOperator{\AgdaFunction{⟧s}}\AgdaSpace{}%
\AgdaBound{ρ}\AgdaSymbol{)}\<%
\\
%
\\[\AgdaEmptyExtraSkip]%
%
\>[1]\AgdaFunction{substitution}\AgdaSpace{}%
\AgdaSymbol{(}\AgdaInductiveConstructor{ℊ}\AgdaSpace{}%
\AgdaBound{x}\AgdaSymbol{)}\AgdaSpace{}%
\AgdaBound{σ}\AgdaSpace{}%
\AgdaBound{ρ}\AgdaSpace{}%
\AgdaSymbol{=}\AgdaSpace{}%
\AgdaFunction{refl}\<%
\\
%
\>[1]\AgdaFunction{substitution}\AgdaSpace{}%
\AgdaSymbol{(}\AgdaInductiveConstructor{node}\AgdaSpace{}%
\AgdaBound{f}\AgdaSpace{}%
\AgdaBound{ts}\AgdaSymbol{)}\AgdaSpace{}%
\AgdaBound{σ}\AgdaSpace{}%
\AgdaBound{ρ}\AgdaSpace{}%
\AgdaSymbol{=}\AgdaSpace{}%
\AgdaField{cong}\AgdaSpace{}%
\AgdaSymbol{(}\AgdaField{Interp}\AgdaSpace{}%
\AgdaBound{𝑨}\AgdaSymbol{)(}\AgdaInductiveConstructor{≡.refl}\AgdaSpace{}%
\AgdaOperator{\AgdaInductiveConstructor{,}}\AgdaSpace{}%
\AgdaSymbol{λ}\AgdaSpace{}%
\AgdaBound{i}\AgdaSpace{}%
\AgdaSymbol{→}\AgdaSpace{}%
\AgdaFunction{substitution}\AgdaSpace{}%
\AgdaSymbol{(}\AgdaBound{ts}\AgdaSpace{}%
\AgdaBound{i}\AgdaSymbol{)}\AgdaSpace{}%
\AgdaBound{σ}\AgdaSpace{}%
\AgdaBound{ρ}\AgdaSymbol{)}\<%
\end{code}

\hypertarget{compatibility-of-terms}{%
\subsubsection{Compatibility of terms}\label{compatibility-of-terms}}

We now prove two important facts about term operations. The first of
these, which is used very often in the sequel, asserts that every term
commutes with every homomorphism.

\begin{code}%
\>[0]\<%
\\
\>[0]\AgdaKeyword{module}\AgdaSpace{}%
\AgdaModule{\AgdaUnderscore{}}\AgdaSpace{}%
\AgdaSymbol{\{}\AgdaBound{X}\AgdaSpace{}%
\AgdaSymbol{:}\AgdaSpace{}%
\AgdaPrimitive{Type}\AgdaSpace{}%
\AgdaGeneralizable{χ}\AgdaSymbol{\}\{}\AgdaBound{𝑨}\AgdaSpace{}%
\AgdaSymbol{:}\AgdaSpace{}%
\AgdaRecord{Algebra}\AgdaSpace{}%
\AgdaGeneralizable{α}\AgdaSpace{}%
\AgdaGeneralizable{ρᵃ}\AgdaSymbol{\}\{}\AgdaBound{𝑩}\AgdaSpace{}%
\AgdaSymbol{:}\AgdaSpace{}%
\AgdaRecord{Algebra}\AgdaSpace{}%
\AgdaGeneralizable{β}\AgdaSpace{}%
\AgdaGeneralizable{ρᵇ}\AgdaSymbol{\}(}\AgdaBound{hh}\AgdaSpace{}%
\AgdaSymbol{:}\AgdaSpace{}%
\AgdaFunction{hom}\AgdaSpace{}%
\AgdaBound{𝑨}\AgdaSpace{}%
\AgdaBound{𝑩}\AgdaSymbol{)}\AgdaSpace{}%
\AgdaKeyword{where}\<%
\\
\>[0][@{}l@{\AgdaIndent{0}}]%
\>[1]\AgdaKeyword{open}\AgdaSpace{}%
\AgdaModule{Setoid}\AgdaSpace{}%
\AgdaOperator{\AgdaFunction{𝔻[}}\AgdaSpace{}%
\AgdaBound{𝑩}\AgdaSpace{}%
\AgdaOperator{\AgdaFunction{]}}\AgdaSpace{}%
\AgdaKeyword{using}\AgdaSpace{}%
\AgdaSymbol{(}\AgdaSpace{}%
\AgdaOperator{\AgdaField{\AgdaUnderscore{}≈\AgdaUnderscore{}}}\AgdaSpace{}%
\AgdaSymbol{;}\AgdaSpace{}%
\AgdaFunction{refl}\AgdaSpace{}%
\AgdaSymbol{)}\<%
\\
%
\>[1]\AgdaKeyword{open}\AgdaSpace{}%
\AgdaModule{SetoidReasoning}\AgdaSpace{}%
\AgdaOperator{\AgdaFunction{𝔻[}}\AgdaSpace{}%
\AgdaBound{𝑩}\AgdaSpace{}%
\AgdaOperator{\AgdaFunction{]}}\<%
\\
%
\>[1]\AgdaKeyword{private}\AgdaSpace{}%
\AgdaFunction{hfunc}\AgdaSpace{}%
\AgdaSymbol{=}\AgdaSpace{}%
\AgdaOperator{\AgdaFunction{∣}}\AgdaSpace{}%
\AgdaBound{hh}\AgdaSpace{}%
\AgdaOperator{\AgdaFunction{∣}}\AgdaSpace{}%
\AgdaSymbol{;}\AgdaSpace{}%
\AgdaFunction{h}\AgdaSpace{}%
\AgdaSymbol{=}\AgdaSpace{}%
\AgdaOperator{\AgdaField{\AgdaUnderscore{}⟨\$⟩\AgdaUnderscore{}}}\AgdaSpace{}%
\AgdaFunction{hfunc}\<%
\\
%
\>[1]\AgdaKeyword{open}\AgdaSpace{}%
\AgdaModule{Environment}\AgdaSpace{}%
\AgdaBound{𝑨}\AgdaSpace{}%
\AgdaKeyword{using}\AgdaSpace{}%
\AgdaSymbol{(}\AgdaSpace{}%
\AgdaOperator{\AgdaFunction{⟦\AgdaUnderscore{}⟧}}\AgdaSpace{}%
\AgdaSymbol{)}\<%
\\
%
\>[1]\AgdaKeyword{open}\AgdaSpace{}%
\AgdaModule{Environment}\AgdaSpace{}%
\AgdaBound{𝑩}\AgdaSpace{}%
\AgdaKeyword{using}\AgdaSpace{}%
\AgdaSymbol{()}\AgdaSpace{}%
\AgdaKeyword{renaming}\AgdaSpace{}%
\AgdaSymbol{(}\AgdaSpace{}%
\AgdaOperator{\AgdaFunction{⟦\AgdaUnderscore{}⟧}}\AgdaSpace{}%
\AgdaSymbol{to}\AgdaSpace{}%
\AgdaOperator{\AgdaFunction{⟦\AgdaUnderscore{}⟧ᴮ}}\AgdaSpace{}%
\AgdaSymbol{)}\<%
\\
%
\\[\AgdaEmptyExtraSkip]%
%
\>[1]\AgdaFunction{comm-hom-term}\AgdaSpace{}%
\AgdaSymbol{:}\AgdaSpace{}%
\AgdaSymbol{(}\AgdaBound{t}\AgdaSpace{}%
\AgdaSymbol{:}\AgdaSpace{}%
\AgdaDatatype{Term}\AgdaSpace{}%
\AgdaBound{X}\AgdaSymbol{)}\AgdaSpace{}%
\AgdaSymbol{(}\AgdaBound{a}\AgdaSpace{}%
\AgdaSymbol{:}\AgdaSpace{}%
\AgdaBound{X}\AgdaSpace{}%
\AgdaSymbol{→}\AgdaSpace{}%
\AgdaOperator{\AgdaFunction{𝕌[}}\AgdaSpace{}%
\AgdaBound{𝑨}\AgdaSpace{}%
\AgdaOperator{\AgdaFunction{]}}\AgdaSymbol{)}\AgdaSpace{}%
\AgdaSymbol{→}\AgdaSpace{}%
\AgdaFunction{h}\AgdaSpace{}%
\AgdaSymbol{(}\AgdaOperator{\AgdaFunction{⟦}}\AgdaSpace{}%
\AgdaBound{t}\AgdaSpace{}%
\AgdaOperator{\AgdaFunction{⟧}}\AgdaSpace{}%
\AgdaOperator{\AgdaField{⟨\$⟩}}\AgdaSpace{}%
\AgdaBound{a}\AgdaSymbol{)}\AgdaSpace{}%
\AgdaOperator{\AgdaFunction{≈}}\AgdaSpace{}%
\AgdaOperator{\AgdaFunction{⟦}}\AgdaSpace{}%
\AgdaBound{t}\AgdaSpace{}%
\AgdaOperator{\AgdaFunction{⟧ᴮ}}\AgdaSpace{}%
\AgdaOperator{\AgdaField{⟨\$⟩}}\AgdaSpace{}%
\AgdaSymbol{(}\AgdaFunction{h}\AgdaSpace{}%
\AgdaOperator{\AgdaFunction{∘}}\AgdaSpace{}%
\AgdaBound{a}\AgdaSymbol{)}\<%
\\
%
\>[1]\AgdaFunction{comm-hom-term}\AgdaSpace{}%
\AgdaSymbol{(}\AgdaInductiveConstructor{ℊ}\AgdaSpace{}%
\AgdaBound{x}\AgdaSymbol{)}\AgdaSpace{}%
\AgdaBound{a}\AgdaSpace{}%
\AgdaSymbol{=}\AgdaSpace{}%
\AgdaFunction{refl}\<%
\\
%
\>[1]\AgdaFunction{comm-hom-term}\AgdaSpace{}%
\AgdaSymbol{(}\AgdaInductiveConstructor{node}\AgdaSpace{}%
\AgdaBound{f}\AgdaSpace{}%
\AgdaBound{t}\AgdaSymbol{)}\AgdaSpace{}%
\AgdaBound{a}\AgdaSpace{}%
\AgdaSymbol{=}\AgdaSpace{}%
\AgdaFunction{goal}\<%
\\
\>[1][@{}l@{\AgdaIndent{0}}]%
\>[2]\AgdaKeyword{where}\<%
\\
%
\>[2]\AgdaFunction{goal}\AgdaSpace{}%
\AgdaSymbol{:}\AgdaSpace{}%
\AgdaFunction{h}\AgdaSpace{}%
\AgdaSymbol{(}\AgdaOperator{\AgdaFunction{⟦}}\AgdaSpace{}%
\AgdaInductiveConstructor{node}\AgdaSpace{}%
\AgdaBound{f}\AgdaSpace{}%
\AgdaBound{t}\AgdaSpace{}%
\AgdaOperator{\AgdaFunction{⟧}}\AgdaSpace{}%
\AgdaOperator{\AgdaField{⟨\$⟩}}\AgdaSpace{}%
\AgdaBound{a}\AgdaSymbol{)}\AgdaSpace{}%
\AgdaOperator{\AgdaFunction{≈}}\AgdaSpace{}%
\AgdaSymbol{(}\AgdaOperator{\AgdaFunction{⟦}}\AgdaSpace{}%
\AgdaInductiveConstructor{node}\AgdaSpace{}%
\AgdaBound{f}\AgdaSpace{}%
\AgdaBound{t}\AgdaSpace{}%
\AgdaOperator{\AgdaFunction{⟧ᴮ}}\AgdaSpace{}%
\AgdaOperator{\AgdaField{⟨\$⟩}}\AgdaSpace{}%
\AgdaSymbol{(}\AgdaFunction{h}\AgdaSpace{}%
\AgdaOperator{\AgdaFunction{∘}}\AgdaSpace{}%
\AgdaBound{a}\AgdaSymbol{))}\<%
\\
%
\>[2]\AgdaFunction{goal}\AgdaSpace{}%
\AgdaSymbol{=}%
\>[4141I]\AgdaOperator{\AgdaFunction{begin}}\<%
\\
\>[4141I][@{}l@{\AgdaIndent{0}}]%
\>[10]\AgdaFunction{h}\AgdaSpace{}%
\AgdaSymbol{(}\AgdaOperator{\AgdaFunction{⟦}}\AgdaSpace{}%
\AgdaInductiveConstructor{node}\AgdaSpace{}%
\AgdaBound{f}\AgdaSpace{}%
\AgdaBound{t}\AgdaSpace{}%
\AgdaOperator{\AgdaFunction{⟧}}\AgdaSpace{}%
\AgdaOperator{\AgdaField{⟨\$⟩}}\AgdaSpace{}%
\AgdaBound{a}\AgdaSymbol{)}%
\>[46]\AgdaFunction{≈⟨}%
\>[50]\AgdaField{compatible}\AgdaSpace{}%
\AgdaOperator{\AgdaFunction{∥}}\AgdaSpace{}%
\AgdaBound{hh}\AgdaSpace{}%
\AgdaOperator{\AgdaFunction{∥}}%
\>[104]\AgdaFunction{⟩}\<%
\\
%
\>[10]\AgdaSymbol{(}\AgdaBound{f}\AgdaSpace{}%
\AgdaOperator{\AgdaFunction{̂}}\AgdaSpace{}%
\AgdaBound{𝑩}\AgdaSymbol{)(λ}\AgdaSpace{}%
\AgdaBound{i}\AgdaSpace{}%
\AgdaSymbol{→}\AgdaSpace{}%
\AgdaFunction{h}\AgdaSpace{}%
\AgdaSymbol{(}\AgdaOperator{\AgdaFunction{⟦}}\AgdaSpace{}%
\AgdaBound{t}\AgdaSpace{}%
\AgdaBound{i}\AgdaSpace{}%
\AgdaOperator{\AgdaFunction{⟧}}\AgdaSpace{}%
\AgdaOperator{\AgdaField{⟨\$⟩}}\AgdaSpace{}%
\AgdaBound{a}\AgdaSymbol{))}%
\>[47]\AgdaFunction{≈⟨}%
\>[51]\AgdaField{cong}\AgdaSymbol{(}\AgdaField{Interp}\AgdaSpace{}%
\AgdaBound{𝑩}\AgdaSymbol{)(}\AgdaInductiveConstructor{≡.refl}\AgdaSpace{}%
\AgdaOperator{\AgdaInductiveConstructor{,}}\AgdaSpace{}%
\AgdaSymbol{λ}\AgdaSpace{}%
\AgdaBound{i}\AgdaSpace{}%
\AgdaSymbol{→}\AgdaSpace{}%
\AgdaFunction{comm-hom-term}\AgdaSpace{}%
\AgdaSymbol{(}\AgdaBound{t}\AgdaSpace{}%
\AgdaBound{i}\AgdaSymbol{)}\AgdaSpace{}%
\AgdaBound{a}\AgdaSymbol{)}%
\>[105]\AgdaFunction{⟩}\<%
\\
%
\>[10]\AgdaSymbol{(}\AgdaBound{f}\AgdaSpace{}%
\AgdaOperator{\AgdaFunction{̂}}\AgdaSpace{}%
\AgdaBound{𝑩}\AgdaSymbol{)(λ}\AgdaSpace{}%
\AgdaBound{i}\AgdaSpace{}%
\AgdaSymbol{→}\AgdaSpace{}%
\AgdaOperator{\AgdaFunction{⟦}}\AgdaSpace{}%
\AgdaBound{t}\AgdaSpace{}%
\AgdaBound{i}\AgdaSpace{}%
\AgdaOperator{\AgdaFunction{⟧ᴮ}}\AgdaSpace{}%
\AgdaOperator{\AgdaField{⟨\$⟩}}\AgdaSpace{}%
\AgdaSymbol{(}\AgdaFunction{h}\AgdaSpace{}%
\AgdaOperator{\AgdaFunction{∘}}\AgdaSpace{}%
\AgdaBound{a}\AgdaSymbol{))}%
\>[47]\AgdaFunction{≈⟨}%
\>[51]\AgdaFunction{refl}%
\>[105]\AgdaFunction{⟩}\<%
\\
%
\>[10]\AgdaSymbol{(}\AgdaOperator{\AgdaFunction{⟦}}\AgdaSpace{}%
\AgdaInductiveConstructor{node}\AgdaSpace{}%
\AgdaBound{f}\AgdaSpace{}%
\AgdaBound{t}\AgdaSpace{}%
\AgdaOperator{\AgdaFunction{⟧ᴮ}}\AgdaSpace{}%
\AgdaOperator{\AgdaField{⟨\$⟩}}\AgdaSpace{}%
\AgdaSymbol{(}\AgdaFunction{h}\AgdaSpace{}%
\AgdaOperator{\AgdaFunction{∘}}\AgdaSpace{}%
\AgdaBound{a}\AgdaSymbol{))}%
\>[46]\AgdaOperator{\AgdaFunction{∎}}\<%
\end{code}

\begin{code}[hide]%
\>[0]\AgdaKeyword{module}\AgdaSpace{}%
\AgdaModule{\AgdaUnderscore{}}\AgdaSpace{}%
\AgdaSymbol{\{}\AgdaBound{X}\AgdaSpace{}%
\AgdaSymbol{:}\AgdaSpace{}%
\AgdaPrimitive{Type}\AgdaSpace{}%
\AgdaGeneralizable{χ}\AgdaSymbol{\}\{}\AgdaBound{ι}\AgdaSpace{}%
\AgdaSymbol{:}\AgdaSpace{}%
\AgdaPostulate{Level}\AgdaSymbol{\}}\AgdaSpace{}%
\AgdaSymbol{\{}\AgdaBound{I}\AgdaSpace{}%
\AgdaSymbol{:}\AgdaSpace{}%
\AgdaPrimitive{Type}\AgdaSpace{}%
\AgdaBound{ι}\AgdaSymbol{\}}\AgdaSpace{}%
\AgdaSymbol{(}\AgdaBound{𝒜}\AgdaSpace{}%
\AgdaSymbol{:}\AgdaSpace{}%
\AgdaBound{I}\AgdaSpace{}%
\AgdaSymbol{→}\AgdaSpace{}%
\AgdaRecord{Algebra}\AgdaSpace{}%
\AgdaGeneralizable{α}\AgdaSpace{}%
\AgdaGeneralizable{ρᵃ}\AgdaSymbol{)}\AgdaSpace{}%
\AgdaKeyword{where}\<%
\\
\>[0][@{}l@{\AgdaIndent{0}}]%
\>[1]\AgdaKeyword{open}\AgdaSpace{}%
\AgdaModule{Setoid}\AgdaSpace{}%
\AgdaOperator{\AgdaFunction{𝔻[}}\AgdaSpace{}%
\AgdaFunction{⨅}\AgdaSpace{}%
\AgdaBound{𝒜}\AgdaSpace{}%
\AgdaOperator{\AgdaFunction{]}}\AgdaSpace{}%
\AgdaKeyword{using}\AgdaSpace{}%
\AgdaSymbol{(}\AgdaSpace{}%
\AgdaOperator{\AgdaField{\AgdaUnderscore{}≈\AgdaUnderscore{}}}\AgdaSpace{}%
\AgdaSymbol{)}\<%
\\
%
\>[1]\AgdaKeyword{open}\AgdaSpace{}%
\AgdaModule{Environment}\AgdaSpace{}%
\AgdaSymbol{(}\AgdaFunction{⨅}\AgdaSpace{}%
\AgdaBound{𝒜}\AgdaSymbol{)}\AgdaSpace{}%
\AgdaKeyword{using}\AgdaSpace{}%
\AgdaSymbol{()}\AgdaSpace{}%
\AgdaKeyword{renaming}\AgdaSpace{}%
\AgdaSymbol{(}\AgdaSpace{}%
\AgdaOperator{\AgdaFunction{⟦\AgdaUnderscore{}⟧}}\AgdaSpace{}%
\AgdaSymbol{to}\AgdaSpace{}%
\AgdaOperator{\AgdaFunction{⟦\AgdaUnderscore{}⟧₁}}\AgdaSpace{}%
\AgdaSymbol{)}\<%
\\
%
\>[1]\AgdaKeyword{open}\AgdaSpace{}%
\AgdaModule{Environment}\AgdaSpace{}%
\AgdaKeyword{using}\AgdaSpace{}%
\AgdaSymbol{(}\AgdaSpace{}%
\AgdaOperator{\AgdaFunction{⟦\AgdaUnderscore{}⟧}}\AgdaSpace{}%
\AgdaSymbol{;}\AgdaSpace{}%
\AgdaFunction{≐→Equal}\AgdaSpace{}%
\AgdaSymbol{)}\<%
\\
%
\\[\AgdaEmptyExtraSkip]%
%
\>[1]\AgdaFunction{interp-prod}\AgdaSpace{}%
\AgdaSymbol{:}\AgdaSpace{}%
\AgdaSymbol{(}\AgdaBound{p}\AgdaSpace{}%
\AgdaSymbol{:}\AgdaSpace{}%
\AgdaDatatype{Term}\AgdaSpace{}%
\AgdaBound{X}\AgdaSymbol{)}\AgdaSpace{}%
\AgdaSymbol{→}\AgdaSpace{}%
\AgdaSymbol{∀}\AgdaSpace{}%
\AgdaBound{ρ}\AgdaSpace{}%
\AgdaSymbol{→}\AgdaSpace{}%
\AgdaOperator{\AgdaFunction{⟦}}\AgdaSpace{}%
\AgdaBound{p}\AgdaSpace{}%
\AgdaOperator{\AgdaFunction{⟧₁}}\AgdaSpace{}%
\AgdaOperator{\AgdaField{⟨\$⟩}}\AgdaSpace{}%
\AgdaBound{ρ}\AgdaSpace{}%
\AgdaOperator{\AgdaFunction{≈}}\AgdaSpace{}%
\AgdaSymbol{(λ}\AgdaSpace{}%
\AgdaBound{i}\AgdaSpace{}%
\AgdaSymbol{→}\AgdaSpace{}%
\AgdaSymbol{(}\AgdaOperator{\AgdaFunction{⟦}}\AgdaSpace{}%
\AgdaBound{𝒜}\AgdaSpace{}%
\AgdaBound{i}\AgdaSpace{}%
\AgdaOperator{\AgdaFunction{⟧}}\AgdaSpace{}%
\AgdaBound{p}\AgdaSymbol{)}\AgdaSpace{}%
\AgdaOperator{\AgdaField{⟨\$⟩}}\AgdaSpace{}%
\AgdaSymbol{(λ}\AgdaSpace{}%
\AgdaBound{x}\AgdaSpace{}%
\AgdaSymbol{→}\AgdaSpace{}%
\AgdaSymbol{(}\AgdaBound{ρ}\AgdaSpace{}%
\AgdaBound{x}\AgdaSymbol{)}\AgdaSpace{}%
\AgdaBound{i}\AgdaSymbol{))}\<%
\\
%
\>[1]\AgdaFunction{interp-prod}\AgdaSpace{}%
\AgdaSymbol{(}\AgdaInductiveConstructor{ℊ}\AgdaSpace{}%
\AgdaBound{x}\AgdaSymbol{)}\AgdaSpace{}%
\AgdaSymbol{=}\AgdaSpace{}%
\AgdaSymbol{λ}\AgdaSpace{}%
\AgdaBound{ρ}\AgdaSpace{}%
\AgdaBound{i}\AgdaSpace{}%
\AgdaSymbol{→}\AgdaSpace{}%
\AgdaFunction{≐→Equal}\AgdaSpace{}%
\AgdaSymbol{(}\AgdaBound{𝒜}\AgdaSpace{}%
\AgdaBound{i}\AgdaSymbol{)}\AgdaSpace{}%
\AgdaSymbol{(}\AgdaInductiveConstructor{ℊ}\AgdaSpace{}%
\AgdaBound{x}\AgdaSymbol{)}\AgdaSpace{}%
\AgdaSymbol{(}\AgdaInductiveConstructor{ℊ}\AgdaSpace{}%
\AgdaBound{x}\AgdaSymbol{)}\AgdaSpace{}%
\AgdaFunction{≐-isRefl}\AgdaSpace{}%
\AgdaSymbol{λ}\AgdaSpace{}%
\AgdaBound{x'}\AgdaSpace{}%
\AgdaSymbol{→}\AgdaSpace{}%
\AgdaSymbol{(}\AgdaBound{ρ}\AgdaSpace{}%
\AgdaBound{x}\AgdaSymbol{)}\AgdaSpace{}%
\AgdaBound{i}\<%
\\
%
\>[1]\AgdaFunction{interp-prod}\AgdaSpace{}%
\AgdaSymbol{(}\AgdaInductiveConstructor{node}\AgdaSpace{}%
\AgdaBound{f}\AgdaSpace{}%
\AgdaBound{t}\AgdaSymbol{)}\AgdaSpace{}%
\AgdaSymbol{=}\AgdaSpace{}%
\AgdaSymbol{λ}\AgdaSpace{}%
\AgdaBound{ρ}\AgdaSpace{}%
\AgdaBound{i}\AgdaSpace{}%
\AgdaSymbol{→}\AgdaSpace{}%
\AgdaField{cong}\AgdaSpace{}%
\AgdaSymbol{(}\AgdaField{Interp}\AgdaSpace{}%
\AgdaSymbol{(}\AgdaFunction{⨅}\AgdaSpace{}%
\AgdaBound{𝒜}\AgdaSymbol{))(}\AgdaInductiveConstructor{≡.refl}\AgdaSpace{}%
\AgdaOperator{\AgdaInductiveConstructor{,}}\AgdaSpace{}%
\AgdaSymbol{(λ}\AgdaSpace{}%
\AgdaBound{j}\AgdaSpace{}%
\AgdaBound{k}\AgdaSpace{}%
\AgdaSymbol{→}\AgdaSpace{}%
\AgdaFunction{interp-prod}\AgdaSpace{}%
\AgdaSymbol{(}\AgdaBound{t}\AgdaSpace{}%
\AgdaBound{j}\AgdaSymbol{)}\AgdaSpace{}%
\AgdaBound{ρ}\AgdaSpace{}%
\AgdaBound{k}\AgdaSymbol{))}\AgdaSpace{}%
\AgdaBound{i}\<%
\end{code}

\hypertarget{model-theory-and-equational-logic}{%
\subsection{Model Theory and Equational
Logic}\label{model-theory-and-equational-logic}}

(cf.~the
\href{https://ualib.github.io/agda-algebras/Varieties.Func.SoundAndComplete.html}{Varieties.Func.SoundAndComplete}
module of the \href{https://ualib.github.io/agda-algebras}{Agda
Universal Algebra Library})

\hypertarget{basic-definitions-3}{%
\subsubsection{Basic definitions}\label{basic-definitions-3}}

Let \texttt{𝑆} be a signature. By an \emph{identity} or \emph{equation}
in \texttt{𝑆} we mean an ordered pair of terms in a given context. For
instance, if the context happens to be the type \texttt{X\ :\ Type\ χ},
then an equation will be a pair of inhabitants of the domain of term
algebra \texttt{𝑻\ X}.

We define an equation in Agda using the following record type with
fields denoting the left-hand and right-hand sides of the equation,
along with an equation ``context'' representing the underlying
collection of variable symbols
(cf.~\href{http://www.cse.chalmers.se/~abela/agda/MultiSortedAlgebra.pdf}{Andreas
Abel's formal proof of Birkhoff's completeness theorem}).

\begin{code}%
\>[0]\<%
\\
\>[0]\AgdaKeyword{record}\AgdaSpace{}%
\AgdaRecord{Eq}\AgdaSpace{}%
\AgdaSymbol{:}\AgdaSpace{}%
\AgdaPrimitive{Type}\AgdaSpace{}%
\AgdaSymbol{(}\AgdaFunction{ov}\AgdaSpace{}%
\AgdaBound{χ}\AgdaSymbol{)}\AgdaSpace{}%
\AgdaKeyword{where}\<%
\\
\>[0][@{}l@{\AgdaIndent{0}}]%
\>[1]\AgdaKeyword{constructor}\AgdaSpace{}%
\AgdaOperator{\AgdaInductiveConstructor{\AgdaUnderscore{}≈̇\AgdaUnderscore{}}}\<%
\\
%
\>[1]\AgdaKeyword{field}%
\>[8]\AgdaSymbol{\{}\AgdaField{cxt}\AgdaSymbol{\}}%
\>[15]\AgdaSymbol{:}\AgdaSpace{}%
\AgdaPrimitive{Type}\AgdaSpace{}%
\AgdaBound{χ}\<%
\\
%
\>[8]\AgdaField{lhs}%
\>[15]\AgdaSymbol{:}\AgdaSpace{}%
\AgdaDatatype{Term}\AgdaSpace{}%
\AgdaField{cxt}\<%
\\
%
\>[8]\AgdaField{rhs}%
\>[15]\AgdaSymbol{:}\AgdaSpace{}%
\AgdaDatatype{Term}\AgdaSpace{}%
\AgdaField{cxt}\<%
\\
%
\\[\AgdaEmptyExtraSkip]%
\>[0]\AgdaKeyword{open}\AgdaSpace{}%
\AgdaModule{Eq}\AgdaSpace{}%
\AgdaKeyword{public}\<%
\\
\>[0]\<%
\end{code}

We now define a type representing the notion of an equation
\texttt{p\ ≈̇\ q} holding (when \texttt{p} and \texttt{q} are
interpreted) in algebra \texttt{𝑨}.

If \texttt{𝑨} is an \texttt{𝑆}-algebra we say that \texttt{𝑨}
\emph{satisfies} \texttt{p\ ≈\ q} provided for all environments
\texttt{ρ\ :\ X\ →\ ∣\ 𝑨\ ∣} (assigning values in the domain of
\texttt{𝑨} to variable symbols in \texttt{X}) we have
\texttt{⟦\ p\ ⟧⟨\$⟩\ ρ\ ≈\ ⟦\ q\ ⟧\ ⟨\$⟩\ ρ}. In this situation, we
write \texttt{𝑨\ ⊧\ (p\ ≈̇\ q)} and say that \texttt{𝑨} \emph{models} the
identity \texttt{p\ ≈\ q}.

If \texttt{𝒦} is a class of algebras, all of the same signature, we
write \texttt{𝒦\ ⊫\ (p\ ≈̇\ q)\ if,\ for\ every}𝑨 ∈
𝒦\texttt{,\ we\ have}𝑨 ⊧ (p ≈̇ q)`.

Because a class of structures has a different type than a single
structure, we must use a slightly different syntax to avoid overloading
the relations \texttt{⊧} and \texttt{≈}. As a reasonable alternative to
what we would normally express informally as \texttt{𝒦\ ⊧\ p\ ≈\ q}, we
have settled on \texttt{𝒦\ ⊫\ (p\ ≈̇\ q)} to denote this relation. To
reiterate, if \texttt{𝒦} is a class of \texttt{𝑆}-algebras, we write
\texttt{𝒦\ ⊫\ (p\ ≈̇\ q)} provided every \texttt{𝑨\ ∈\ 𝒦} satisfies
\texttt{𝑨\ ⊧\ (p\ ≈̇\ q)}.

\begin{code}%
\>[0]\<%
\\
\>[0]\AgdaOperator{\AgdaFunction{\AgdaUnderscore{}⊧\AgdaUnderscore{}}}\AgdaSpace{}%
\AgdaSymbol{:}\AgdaSpace{}%
\AgdaSymbol{(}\AgdaBound{𝑨}\AgdaSpace{}%
\AgdaSymbol{:}\AgdaSpace{}%
\AgdaRecord{Algebra}\AgdaSpace{}%
\AgdaGeneralizable{α}\AgdaSpace{}%
\AgdaGeneralizable{ρᵃ}\AgdaSymbol{)(}\AgdaBound{term-identity}\AgdaSpace{}%
\AgdaSymbol{:}\AgdaSpace{}%
\AgdaRecord{Eq}\AgdaSymbol{\{}\AgdaGeneralizable{χ}\AgdaSymbol{\})}\AgdaSpace{}%
\AgdaSymbol{→}\AgdaSpace{}%
\AgdaPrimitive{Type}\AgdaSpace{}%
\AgdaSymbol{\AgdaUnderscore{}}\<%
\\
\>[0]\AgdaBound{𝑨}\AgdaSpace{}%
\AgdaOperator{\AgdaFunction{⊧}}\AgdaSpace{}%
\AgdaSymbol{(}\AgdaBound{p}\AgdaSpace{}%
\AgdaOperator{\AgdaInductiveConstructor{≈̇}}\AgdaSpace{}%
\AgdaBound{q}\AgdaSymbol{)}\AgdaSpace{}%
\AgdaSymbol{=}\AgdaSpace{}%
\AgdaFunction{Equal}\AgdaSpace{}%
\AgdaBound{p}\AgdaSpace{}%
\AgdaBound{q}\AgdaSpace{}%
\AgdaKeyword{where}\AgdaSpace{}%
\AgdaKeyword{open}\AgdaSpace{}%
\AgdaModule{Environment}\AgdaSpace{}%
\AgdaBound{𝑨}\<%
\\
%
\\[\AgdaEmptyExtraSkip]%
\>[0]\AgdaOperator{\AgdaFunction{\AgdaUnderscore{}⊫\AgdaUnderscore{}}}\AgdaSpace{}%
\AgdaSymbol{:}\AgdaSpace{}%
\AgdaFunction{Pred}\AgdaSpace{}%
\AgdaSymbol{(}\AgdaRecord{Algebra}\AgdaSpace{}%
\AgdaGeneralizable{α}\AgdaSpace{}%
\AgdaGeneralizable{ρᵃ}\AgdaSymbol{)}\AgdaSpace{}%
\AgdaGeneralizable{ℓ}\AgdaSpace{}%
\AgdaSymbol{→}\AgdaSpace{}%
\AgdaRecord{Eq}\AgdaSymbol{\{}\AgdaGeneralizable{χ}\AgdaSymbol{\}}\AgdaSpace{}%
\AgdaSymbol{→}\AgdaSpace{}%
\AgdaPrimitive{Type}\AgdaSpace{}%
\AgdaSymbol{(}\AgdaGeneralizable{ℓ}\AgdaSpace{}%
\AgdaOperator{\AgdaPrimitive{⊔}}\AgdaSpace{}%
\AgdaGeneralizable{χ}\AgdaSpace{}%
\AgdaOperator{\AgdaPrimitive{⊔}}\AgdaSpace{}%
\AgdaFunction{ov}\AgdaSymbol{(}\AgdaGeneralizable{α}\AgdaSpace{}%
\AgdaOperator{\AgdaPrimitive{⊔}}\AgdaSpace{}%
\AgdaGeneralizable{ρᵃ}\AgdaSymbol{))}\<%
\\
\>[0]\AgdaBound{𝒦}\AgdaSpace{}%
\AgdaOperator{\AgdaFunction{⊫}}\AgdaSpace{}%
\AgdaBound{equ}\AgdaSpace{}%
\AgdaSymbol{=}\AgdaSpace{}%
\AgdaSymbol{∀}\AgdaSpace{}%
\AgdaBound{𝑨}\AgdaSpace{}%
\AgdaSymbol{→}\AgdaSpace{}%
\AgdaBound{𝒦}\AgdaSpace{}%
\AgdaBound{𝑨}\AgdaSpace{}%
\AgdaSymbol{→}\AgdaSpace{}%
\AgdaBound{𝑨}\AgdaSpace{}%
\AgdaOperator{\AgdaFunction{⊧}}\AgdaSpace{}%
\AgdaBound{equ}\<%
\\
\>[0]\<%
\end{code}

We denote by \texttt{𝑨\ ⊨\ ℰ} the assertion that the algebra 𝑨 models
every equation in a collection \texttt{ℰ} of equations.

\begin{code}%
\>[0]\<%
\\
\>[0]\AgdaOperator{\AgdaFunction{\AgdaUnderscore{}⊨\AgdaUnderscore{}}}\AgdaSpace{}%
\AgdaSymbol{:}\AgdaSpace{}%
\AgdaSymbol{(}\AgdaBound{𝑨}\AgdaSpace{}%
\AgdaSymbol{:}\AgdaSpace{}%
\AgdaRecord{Algebra}\AgdaSpace{}%
\AgdaGeneralizable{α}\AgdaSpace{}%
\AgdaGeneralizable{ρᵃ}\AgdaSymbol{)}\AgdaSpace{}%
\AgdaSymbol{→}\AgdaSpace{}%
\AgdaSymbol{\{}\AgdaBound{ι}\AgdaSpace{}%
\AgdaSymbol{:}\AgdaSpace{}%
\AgdaPostulate{Level}\AgdaSymbol{\}\{}\AgdaBound{I}\AgdaSpace{}%
\AgdaSymbol{:}\AgdaSpace{}%
\AgdaPrimitive{Type}\AgdaSpace{}%
\AgdaBound{ι}\AgdaSymbol{\}}\AgdaSpace{}%
\AgdaSymbol{→}\AgdaSpace{}%
\AgdaSymbol{(}\AgdaBound{I}\AgdaSpace{}%
\AgdaSymbol{→}\AgdaSpace{}%
\AgdaRecord{Eq}\AgdaSymbol{\{}\AgdaGeneralizable{χ}\AgdaSymbol{\})}\AgdaSpace{}%
\AgdaSymbol{→}\AgdaSpace{}%
\AgdaPrimitive{Type}\AgdaSpace{}%
\AgdaSymbol{\AgdaUnderscore{}}\<%
\\
\>[0]\AgdaBound{𝑨}\AgdaSpace{}%
\AgdaOperator{\AgdaFunction{⊨}}\AgdaSpace{}%
\AgdaBound{ℰ}\AgdaSpace{}%
\AgdaSymbol{=}\AgdaSpace{}%
\AgdaSymbol{∀}\AgdaSpace{}%
\AgdaBound{i}\AgdaSpace{}%
\AgdaSymbol{→}\AgdaSpace{}%
\AgdaFunction{Equal}\AgdaSpace{}%
\AgdaSymbol{(}\AgdaField{lhs}\AgdaSpace{}%
\AgdaSymbol{(}\AgdaBound{ℰ}\AgdaSpace{}%
\AgdaBound{i}\AgdaSymbol{))(}\AgdaField{rhs}\AgdaSpace{}%
\AgdaSymbol{(}\AgdaBound{ℰ}\AgdaSpace{}%
\AgdaBound{i}\AgdaSymbol{))}\AgdaSpace{}%
\AgdaKeyword{where}\AgdaSpace{}%
\AgdaKeyword{open}\AgdaSpace{}%
\AgdaModule{Environment}\AgdaSpace{}%
\AgdaBound{𝑨}\<%
\end{code}

\hypertarget{equational-theories-and-models}{%
\subsubsection{Equational theories and
models}\label{equational-theories-and-models}}

If \texttt{𝒦} denotes a class of structures, then \texttt{Th\ 𝒦}
represents the set of identities modeled by the members of \texttt{𝒦}.

\begin{code}%
\>[0]\<%
\\
\>[0]\AgdaFunction{Th}\AgdaSpace{}%
\AgdaSymbol{:}\AgdaSpace{}%
\AgdaSymbol{\{}\AgdaBound{X}\AgdaSpace{}%
\AgdaSymbol{:}\AgdaSpace{}%
\AgdaPrimitive{Type}\AgdaSpace{}%
\AgdaGeneralizable{χ}\AgdaSymbol{\}}\AgdaSpace{}%
\AgdaSymbol{→}\AgdaSpace{}%
\AgdaFunction{Pred}\AgdaSpace{}%
\AgdaSymbol{(}\AgdaRecord{Algebra}\AgdaSpace{}%
\AgdaGeneralizable{α}\AgdaSpace{}%
\AgdaGeneralizable{ρᵃ}\AgdaSymbol{)}\AgdaSpace{}%
\AgdaGeneralizable{ℓ}\AgdaSpace{}%
\AgdaSymbol{→}\AgdaSpace{}%
\AgdaFunction{Pred}\AgdaSymbol{(}\AgdaDatatype{Term}\AgdaSpace{}%
\AgdaBound{X}\AgdaSpace{}%
\AgdaOperator{\AgdaFunction{×}}\AgdaSpace{}%
\AgdaDatatype{Term}\AgdaSpace{}%
\AgdaBound{X}\AgdaSymbol{)}\AgdaSpace{}%
\AgdaSymbol{\AgdaUnderscore{}}\<%
\\
\>[0]\AgdaFunction{Th}\AgdaSpace{}%
\AgdaBound{𝒦}\AgdaSpace{}%
\AgdaSymbol{=}\AgdaSpace{}%
\AgdaSymbol{λ}\AgdaSpace{}%
\AgdaSymbol{(}\AgdaBound{p}\AgdaSpace{}%
\AgdaOperator{\AgdaInductiveConstructor{,}}\AgdaSpace{}%
\AgdaBound{q}\AgdaSymbol{)}\AgdaSpace{}%
\AgdaSymbol{→}\AgdaSpace{}%
\AgdaBound{𝒦}\AgdaSpace{}%
\AgdaOperator{\AgdaFunction{⊫}}\AgdaSpace{}%
\AgdaSymbol{(}\AgdaBound{p}\AgdaSpace{}%
\AgdaOperator{\AgdaInductiveConstructor{≈̇}}\AgdaSpace{}%
\AgdaBound{q}\AgdaSymbol{)}\<%
\\
%
\\[\AgdaEmptyExtraSkip]%
\>[0]\AgdaFunction{Mod}\AgdaSpace{}%
\AgdaSymbol{:}\AgdaSpace{}%
\AgdaSymbol{\{}\AgdaBound{X}\AgdaSpace{}%
\AgdaSymbol{:}\AgdaSpace{}%
\AgdaPrimitive{Type}\AgdaSpace{}%
\AgdaGeneralizable{χ}\AgdaSymbol{\}}\AgdaSpace{}%
\AgdaSymbol{→}\AgdaSpace{}%
\AgdaFunction{Pred}\AgdaSymbol{(}\AgdaDatatype{Term}\AgdaSpace{}%
\AgdaBound{X}\AgdaSpace{}%
\AgdaOperator{\AgdaFunction{×}}\AgdaSpace{}%
\AgdaDatatype{Term}\AgdaSpace{}%
\AgdaBound{X}\AgdaSymbol{)}\AgdaSpace{}%
\AgdaGeneralizable{ℓ}\AgdaSpace{}%
\AgdaSymbol{→}\AgdaSpace{}%
\AgdaFunction{Pred}\AgdaSpace{}%
\AgdaSymbol{(}\AgdaRecord{Algebra}\AgdaSpace{}%
\AgdaGeneralizable{α}\AgdaSpace{}%
\AgdaGeneralizable{ρᵃ}\AgdaSymbol{)}\AgdaSpace{}%
\AgdaSymbol{\AgdaUnderscore{}}\<%
\\
\>[0]\AgdaFunction{Mod}\AgdaSpace{}%
\AgdaBound{ℰ}\AgdaSpace{}%
\AgdaBound{𝑨}\AgdaSpace{}%
\AgdaSymbol{=}\AgdaSpace{}%
\AgdaSymbol{∀}\AgdaSpace{}%
\AgdaSymbol{\{}\AgdaBound{p}\AgdaSpace{}%
\AgdaBound{q}\AgdaSymbol{\}}\AgdaSpace{}%
\AgdaSymbol{→}\AgdaSpace{}%
\AgdaSymbol{(}\AgdaBound{p}\AgdaSpace{}%
\AgdaOperator{\AgdaInductiveConstructor{,}}\AgdaSpace{}%
\AgdaBound{q}\AgdaSymbol{)}\AgdaSpace{}%
\AgdaOperator{\AgdaFunction{∈}}\AgdaSpace{}%
\AgdaBound{ℰ}\AgdaSpace{}%
\AgdaSymbol{→}\AgdaSpace{}%
\AgdaFunction{Equal}\AgdaSpace{}%
\AgdaBound{p}\AgdaSpace{}%
\AgdaBound{q}\AgdaSpace{}%
\AgdaKeyword{where}\AgdaSpace{}%
\AgdaKeyword{open}\AgdaSpace{}%
\AgdaModule{Environment}\AgdaSpace{}%
\AgdaBound{𝑨}\<%
\end{code}

\hypertarget{the-entailment-relation}{%
\subsubsection{The entailment relation}\label{the-entailment-relation}}

Based on
\href{http://www.cse.chalmers.se/~abela/agda/MultiSortedAlgebra.pdf}{Andreas
Abel's Agda formalization of Birkhoff's completeness theorem}.)

\begin{code}%
\>[0]\<%
\\
\>[0]\AgdaKeyword{module}\AgdaSpace{}%
\AgdaModule{\AgdaUnderscore{}}\AgdaSpace{}%
\AgdaSymbol{\{}\AgdaBound{χ}\AgdaSpace{}%
\AgdaBound{ι}\AgdaSpace{}%
\AgdaSymbol{:}\AgdaSpace{}%
\AgdaPostulate{Level}\AgdaSymbol{\}}\AgdaSpace{}%
\AgdaKeyword{where}\<%
\\
%
\\[\AgdaEmptyExtraSkip]%
\>[0][@{}l@{\AgdaIndent{0}}]%
\>[1]\AgdaKeyword{data}\AgdaSpace{}%
\AgdaOperator{\AgdaDatatype{\AgdaUnderscore{}⊢\AgdaUnderscore{}▹\AgdaUnderscore{}≈\AgdaUnderscore{}}}\AgdaSpace{}%
\AgdaSymbol{\{}\AgdaBound{I}\AgdaSpace{}%
\AgdaSymbol{:}\AgdaSpace{}%
\AgdaPrimitive{Type}\AgdaSpace{}%
\AgdaBound{ι}\AgdaSymbol{\}(}\AgdaBound{ℰ}\AgdaSpace{}%
\AgdaSymbol{:}\AgdaSpace{}%
\AgdaBound{I}\AgdaSpace{}%
\AgdaSymbol{→}\AgdaSpace{}%
\AgdaRecord{Eq}\AgdaSymbol{)}\AgdaSpace{}%
\AgdaSymbol{:}\AgdaSpace{}%
\AgdaSymbol{(}\AgdaBound{X}\AgdaSpace{}%
\AgdaSymbol{:}\AgdaSpace{}%
\AgdaPrimitive{Type}\AgdaSpace{}%
\AgdaBound{χ}\AgdaSymbol{)(}\AgdaBound{p}\AgdaSpace{}%
\AgdaBound{q}\AgdaSpace{}%
\AgdaSymbol{:}\AgdaSpace{}%
\AgdaDatatype{Term}\AgdaSpace{}%
\AgdaBound{X}\AgdaSymbol{)}\AgdaSpace{}%
\AgdaSymbol{→}\AgdaSpace{}%
\AgdaPrimitive{Type}\AgdaSpace{}%
\AgdaSymbol{(}\AgdaBound{ι}\AgdaSpace{}%
\AgdaOperator{\AgdaPrimitive{⊔}}\AgdaSpace{}%
\AgdaFunction{ov}\AgdaSpace{}%
\AgdaBound{χ}\AgdaSymbol{)}\AgdaSpace{}%
\AgdaKeyword{where}\<%
\\
\>[1][@{}l@{\AgdaIndent{0}}]%
\>[2]\AgdaInductiveConstructor{hyp}\AgdaSpace{}%
\AgdaSymbol{:}\AgdaSpace{}%
\AgdaSymbol{∀}\AgdaSpace{}%
\AgdaBound{i}\AgdaSpace{}%
\AgdaSymbol{→}\AgdaSpace{}%
\AgdaKeyword{let}\AgdaSpace{}%
\AgdaBound{p}\AgdaSpace{}%
\AgdaOperator{\AgdaInductiveConstructor{≈̇}}\AgdaSpace{}%
\AgdaBound{q}\AgdaSpace{}%
\AgdaSymbol{=}\AgdaSpace{}%
\AgdaBound{ℰ}\AgdaSpace{}%
\AgdaBound{i}\AgdaSpace{}%
\AgdaKeyword{in}\AgdaSpace{}%
\AgdaBound{ℰ}\AgdaSpace{}%
\AgdaOperator{\AgdaDatatype{⊢}}\AgdaSpace{}%
\AgdaSymbol{\AgdaUnderscore{}}\AgdaSpace{}%
\AgdaOperator{\AgdaDatatype{▹}}\AgdaSpace{}%
\AgdaBound{p}\AgdaSpace{}%
\AgdaOperator{\AgdaDatatype{≈}}\AgdaSpace{}%
\AgdaBound{q}\<%
\\
%
\>[2]\AgdaInductiveConstructor{app}\AgdaSpace{}%
\AgdaSymbol{:}\AgdaSpace{}%
\AgdaSymbol{∀}\AgdaSpace{}%
\AgdaSymbol{\{}\AgdaBound{ps}\AgdaSpace{}%
\AgdaBound{qs}\AgdaSymbol{\}}\AgdaSpace{}%
\AgdaSymbol{→}\AgdaSpace{}%
\AgdaSymbol{(∀}\AgdaSpace{}%
\AgdaBound{i}\AgdaSpace{}%
\AgdaSymbol{→}\AgdaSpace{}%
\AgdaBound{ℰ}\AgdaSpace{}%
\AgdaOperator{\AgdaDatatype{⊢}}\AgdaSpace{}%
\AgdaGeneralizable{Γ}\AgdaSpace{}%
\AgdaOperator{\AgdaDatatype{▹}}\AgdaSpace{}%
\AgdaBound{ps}\AgdaSpace{}%
\AgdaBound{i}\AgdaSpace{}%
\AgdaOperator{\AgdaDatatype{≈}}\AgdaSpace{}%
\AgdaBound{qs}\AgdaSpace{}%
\AgdaBound{i}\AgdaSymbol{)}\AgdaSpace{}%
\AgdaSymbol{→}\AgdaSpace{}%
\AgdaBound{ℰ}\AgdaSpace{}%
\AgdaOperator{\AgdaDatatype{⊢}}\AgdaSpace{}%
\AgdaGeneralizable{Γ}\AgdaSpace{}%
\AgdaOperator{\AgdaDatatype{▹}}\AgdaSpace{}%
\AgdaSymbol{(}\AgdaInductiveConstructor{node}\AgdaSpace{}%
\AgdaGeneralizable{𝑓}\AgdaSpace{}%
\AgdaBound{ps}\AgdaSymbol{)}\AgdaSpace{}%
\AgdaOperator{\AgdaDatatype{≈}}\AgdaSpace{}%
\AgdaSymbol{(}\AgdaInductiveConstructor{node}\AgdaSpace{}%
\AgdaGeneralizable{𝑓}\AgdaSpace{}%
\AgdaBound{qs}\AgdaSymbol{)}\<%
\\
%
\>[2]\AgdaInductiveConstructor{sub}\AgdaSpace{}%
\AgdaSymbol{:}\AgdaSpace{}%
\AgdaSymbol{∀}\AgdaSpace{}%
\AgdaSymbol{\{}\AgdaBound{p}\AgdaSpace{}%
\AgdaBound{q}\AgdaSymbol{\}}\AgdaSpace{}%
\AgdaSymbol{→}\AgdaSpace{}%
\AgdaBound{ℰ}\AgdaSpace{}%
\AgdaOperator{\AgdaDatatype{⊢}}\AgdaSpace{}%
\AgdaGeneralizable{Δ}\AgdaSpace{}%
\AgdaOperator{\AgdaDatatype{▹}}\AgdaSpace{}%
\AgdaBound{p}\AgdaSpace{}%
\AgdaOperator{\AgdaDatatype{≈}}\AgdaSpace{}%
\AgdaBound{q}\AgdaSpace{}%
\AgdaSymbol{→}\AgdaSpace{}%
\AgdaSymbol{∀}\AgdaSpace{}%
\AgdaSymbol{(}\AgdaBound{σ}\AgdaSpace{}%
\AgdaSymbol{:}\AgdaSpace{}%
\AgdaFunction{Sub}\AgdaSpace{}%
\AgdaGeneralizable{Γ}\AgdaSpace{}%
\AgdaGeneralizable{Δ}\AgdaSymbol{)}\AgdaSpace{}%
\AgdaSymbol{→}\AgdaSpace{}%
\AgdaBound{ℰ}\AgdaSpace{}%
\AgdaOperator{\AgdaDatatype{⊢}}\AgdaSpace{}%
\AgdaGeneralizable{Γ}\AgdaSpace{}%
\AgdaOperator{\AgdaDatatype{▹}}\AgdaSpace{}%
\AgdaSymbol{(}\AgdaBound{p}\AgdaSpace{}%
\AgdaOperator{\AgdaFunction{[}}\AgdaSpace{}%
\AgdaBound{σ}\AgdaSpace{}%
\AgdaOperator{\AgdaFunction{]}}\AgdaSymbol{)}\AgdaSpace{}%
\AgdaOperator{\AgdaDatatype{≈}}\AgdaSpace{}%
\AgdaSymbol{(}\AgdaBound{q}\AgdaSpace{}%
\AgdaOperator{\AgdaFunction{[}}\AgdaSpace{}%
\AgdaBound{σ}\AgdaSpace{}%
\AgdaOperator{\AgdaFunction{]}}\AgdaSymbol{)}\<%
\\
%
\\[\AgdaEmptyExtraSkip]%
%
\>[2]\AgdaInductiveConstructor{⊢refl}%
\>[10]\AgdaSymbol{:}\AgdaSpace{}%
\AgdaSymbol{∀}\AgdaSpace{}%
\AgdaSymbol{\{}\AgdaBound{p}\AgdaSymbol{\}}%
\>[32]\AgdaSymbol{→}\AgdaSpace{}%
\AgdaBound{ℰ}\AgdaSpace{}%
\AgdaOperator{\AgdaDatatype{⊢}}\AgdaSpace{}%
\AgdaGeneralizable{Γ}\AgdaSpace{}%
\AgdaOperator{\AgdaDatatype{▹}}\AgdaSpace{}%
\AgdaBound{p}\AgdaSpace{}%
\AgdaOperator{\AgdaDatatype{≈}}\AgdaSpace{}%
\AgdaBound{p}\<%
\\
%
\>[2]\AgdaInductiveConstructor{⊢sym}%
\>[10]\AgdaSymbol{:}\AgdaSpace{}%
\AgdaSymbol{∀}\AgdaSpace{}%
\AgdaSymbol{\{}\AgdaBound{p}\AgdaSpace{}%
\AgdaBound{q}\AgdaSpace{}%
\AgdaSymbol{:}\AgdaSpace{}%
\AgdaDatatype{Term}\AgdaSpace{}%
\AgdaGeneralizable{Γ}\AgdaSymbol{\}}%
\>[32]\AgdaSymbol{→}\AgdaSpace{}%
\AgdaBound{ℰ}\AgdaSpace{}%
\AgdaOperator{\AgdaDatatype{⊢}}\AgdaSpace{}%
\AgdaGeneralizable{Γ}\AgdaSpace{}%
\AgdaOperator{\AgdaDatatype{▹}}\AgdaSpace{}%
\AgdaBound{p}\AgdaSpace{}%
\AgdaOperator{\AgdaDatatype{≈}}\AgdaSpace{}%
\AgdaBound{q}\AgdaSpace{}%
\AgdaSymbol{→}\AgdaSpace{}%
\AgdaBound{ℰ}\AgdaSpace{}%
\AgdaOperator{\AgdaDatatype{⊢}}\AgdaSpace{}%
\AgdaGeneralizable{Γ}\AgdaSpace{}%
\AgdaOperator{\AgdaDatatype{▹}}\AgdaSpace{}%
\AgdaBound{q}\AgdaSpace{}%
\AgdaOperator{\AgdaDatatype{≈}}\AgdaSpace{}%
\AgdaBound{p}\<%
\\
%
\>[2]\AgdaInductiveConstructor{⊢trans}%
\>[10]\AgdaSymbol{:}\AgdaSpace{}%
\AgdaSymbol{∀}\AgdaSpace{}%
\AgdaSymbol{\{}\AgdaBound{p}\AgdaSpace{}%
\AgdaBound{q}\AgdaSpace{}%
\AgdaBound{r}\AgdaSpace{}%
\AgdaSymbol{:}\AgdaSpace{}%
\AgdaDatatype{Term}\AgdaSpace{}%
\AgdaGeneralizable{Γ}\AgdaSymbol{\}}%
\>[32]\AgdaSymbol{→}\AgdaSpace{}%
\AgdaBound{ℰ}\AgdaSpace{}%
\AgdaOperator{\AgdaDatatype{⊢}}\AgdaSpace{}%
\AgdaGeneralizable{Γ}\AgdaSpace{}%
\AgdaOperator{\AgdaDatatype{▹}}\AgdaSpace{}%
\AgdaBound{p}\AgdaSpace{}%
\AgdaOperator{\AgdaDatatype{≈}}\AgdaSpace{}%
\AgdaBound{q}\AgdaSpace{}%
\AgdaSymbol{→}\AgdaSpace{}%
\AgdaBound{ℰ}\AgdaSpace{}%
\AgdaOperator{\AgdaDatatype{⊢}}\AgdaSpace{}%
\AgdaGeneralizable{Γ}\AgdaSpace{}%
\AgdaOperator{\AgdaDatatype{▹}}\AgdaSpace{}%
\AgdaBound{q}\AgdaSpace{}%
\AgdaOperator{\AgdaDatatype{≈}}\AgdaSpace{}%
\AgdaBound{r}\AgdaSpace{}%
\AgdaSymbol{→}\AgdaSpace{}%
\AgdaBound{ℰ}\AgdaSpace{}%
\AgdaOperator{\AgdaDatatype{⊢}}\AgdaSpace{}%
\AgdaGeneralizable{Γ}\AgdaSpace{}%
\AgdaOperator{\AgdaDatatype{▹}}\AgdaSpace{}%
\AgdaBound{p}\AgdaSpace{}%
\AgdaOperator{\AgdaDatatype{≈}}\AgdaSpace{}%
\AgdaBound{r}\<%
\\
%
\\[\AgdaEmptyExtraSkip]%
%
\>[1]\AgdaFunction{⊢▹≈IsEquiv}\AgdaSpace{}%
\AgdaSymbol{:}\AgdaSpace{}%
\AgdaSymbol{\{}\AgdaBound{X}\AgdaSpace{}%
\AgdaSymbol{:}\AgdaSpace{}%
\AgdaPrimitive{Type}\AgdaSpace{}%
\AgdaBound{χ}\AgdaSymbol{\}\{}\AgdaBound{I}\AgdaSpace{}%
\AgdaSymbol{:}\AgdaSpace{}%
\AgdaPrimitive{Type}\AgdaSpace{}%
\AgdaBound{ι}\AgdaSymbol{\}\{}\AgdaBound{ℰ}\AgdaSpace{}%
\AgdaSymbol{:}\AgdaSpace{}%
\AgdaBound{I}\AgdaSpace{}%
\AgdaSymbol{→}\AgdaSpace{}%
\AgdaRecord{Eq}\AgdaSymbol{\}}\AgdaSpace{}%
\AgdaSymbol{→}\AgdaSpace{}%
\AgdaRecord{IsEquivalence}\AgdaSpace{}%
\AgdaSymbol{(}\AgdaBound{ℰ}\AgdaSpace{}%
\AgdaOperator{\AgdaDatatype{⊢}}\AgdaSpace{}%
\AgdaBound{X}\AgdaSpace{}%
\AgdaOperator{\AgdaDatatype{▹\AgdaUnderscore{}≈\AgdaUnderscore{}}}\AgdaSymbol{)}\<%
\\
%
\>[1]\AgdaFunction{⊢▹≈IsEquiv}\AgdaSpace{}%
\AgdaSymbol{=}\AgdaSpace{}%
\AgdaKeyword{record}\AgdaSpace{}%
\AgdaSymbol{\{}\AgdaSpace{}%
\AgdaField{refl}\AgdaSpace{}%
\AgdaSymbol{=}\AgdaSpace{}%
\AgdaInductiveConstructor{⊢refl}\AgdaSpace{}%
\AgdaSymbol{;}\AgdaSpace{}%
\AgdaField{sym}\AgdaSpace{}%
\AgdaSymbol{=}\AgdaSpace{}%
\AgdaInductiveConstructor{⊢sym}\AgdaSpace{}%
\AgdaSymbol{;}\AgdaSpace{}%
\AgdaField{trans}\AgdaSpace{}%
\AgdaSymbol{=}\AgdaSpace{}%
\AgdaInductiveConstructor{⊢trans}\AgdaSpace{}%
\AgdaSymbol{\}}\<%
\end{code}

\hypertarget{soundness}{%
\subsubsection{Soundness}\label{soundness}}

In any model 𝑨 that satisfies the equations ℰ, derived equality is
actual equality
(cf.~\href{http://www.cse.chalmers.se/~abela/agda/MultiSortedAlgebra.pdf}{Andreas
Abel's Agda formalization of Birkhoff's completeness theorem}.)

\begin{code}%
\>[0]\<%
\\
\>[0]\AgdaKeyword{module}\AgdaSpace{}%
\AgdaModule{Soundness}%
\>[18]\AgdaSymbol{\{}\AgdaBound{χ}\AgdaSpace{}%
\AgdaBound{α}\AgdaSpace{}%
\AgdaBound{ι}\AgdaSpace{}%
\AgdaSymbol{:}\AgdaSpace{}%
\AgdaPostulate{Level}\AgdaSymbol{\}\{}\AgdaBound{I}\AgdaSpace{}%
\AgdaSymbol{:}\AgdaSpace{}%
\AgdaPrimitive{Type}\AgdaSpace{}%
\AgdaBound{ι}\AgdaSymbol{\}}\AgdaSpace{}%
\AgdaSymbol{(}\AgdaBound{ℰ}\AgdaSpace{}%
\AgdaSymbol{:}\AgdaSpace{}%
\AgdaBound{I}\AgdaSpace{}%
\AgdaSymbol{→}\AgdaSpace{}%
\AgdaRecord{Eq}\AgdaSymbol{\{}\AgdaBound{χ}\AgdaSymbol{\})}\<%
\\
%
\>[18]\AgdaSymbol{(}\AgdaBound{𝑨}\AgdaSpace{}%
\AgdaSymbol{:}\AgdaSpace{}%
\AgdaRecord{Algebra}\AgdaSpace{}%
\AgdaBound{α}\AgdaSpace{}%
\AgdaGeneralizable{ρᵃ}\AgdaSymbol{)}%
\>[41]\AgdaComment{--\ We\ assume\ an\ algebra\ 𝑨}\<%
\\
%
\>[18]\AgdaSymbol{(}\AgdaBound{V}\AgdaSpace{}%
\AgdaSymbol{:}\AgdaSpace{}%
\AgdaBound{𝑨}\AgdaSpace{}%
\AgdaOperator{\AgdaFunction{⊨}}\AgdaSpace{}%
\AgdaBound{ℰ}\AgdaSymbol{)}%
\>[41]\AgdaComment{--\ that\ models\ all\ equations\ in\ ℰ.}\<%
\\
%
\>[18]\AgdaKeyword{where}\<%
\\
%
\\[\AgdaEmptyExtraSkip]%
\>[0][@{}l@{\AgdaIndent{0}}]%
\>[1]\AgdaKeyword{open}\AgdaSpace{}%
\AgdaModule{SetoidReasoning}\AgdaSpace{}%
\AgdaOperator{\AgdaFunction{𝔻[}}\AgdaSpace{}%
\AgdaBound{𝑨}\AgdaSpace{}%
\AgdaOperator{\AgdaFunction{]}}\<%
\\
%
\>[1]\AgdaKeyword{open}\AgdaSpace{}%
\AgdaModule{Environment}\AgdaSpace{}%
\AgdaBound{𝑨}\<%
\\
%
\>[1]\AgdaKeyword{open}\AgdaSpace{}%
\AgdaModule{IsEquivalence}\AgdaSpace{}%
\AgdaKeyword{using}\AgdaSpace{}%
\AgdaSymbol{(}\AgdaSpace{}%
\AgdaField{refl}\AgdaSpace{}%
\AgdaSymbol{;}\AgdaSpace{}%
\AgdaField{sym}\AgdaSpace{}%
\AgdaSymbol{;}\AgdaSpace{}%
\AgdaField{trans}\AgdaSpace{}%
\AgdaSymbol{)}\<%
\\
%
\\[\AgdaEmptyExtraSkip]%
%
\>[1]\AgdaFunction{sound}\AgdaSpace{}%
\AgdaSymbol{:}\AgdaSpace{}%
\AgdaSymbol{∀}\AgdaSpace{}%
\AgdaSymbol{\{}\AgdaBound{p}\AgdaSpace{}%
\AgdaBound{q}\AgdaSymbol{\}}\AgdaSpace{}%
\AgdaSymbol{→}\AgdaSpace{}%
\AgdaBound{ℰ}\AgdaSpace{}%
\AgdaOperator{\AgdaDatatype{⊢}}\AgdaSpace{}%
\AgdaGeneralizable{Γ}\AgdaSpace{}%
\AgdaOperator{\AgdaDatatype{▹}}\AgdaSpace{}%
\AgdaBound{p}\AgdaSpace{}%
\AgdaOperator{\AgdaDatatype{≈}}\AgdaSpace{}%
\AgdaBound{q}\AgdaSpace{}%
\AgdaSymbol{→}\AgdaSpace{}%
\AgdaBound{𝑨}\AgdaSpace{}%
\AgdaOperator{\AgdaFunction{⊧}}\AgdaSpace{}%
\AgdaSymbol{(}\AgdaBound{p}\AgdaSpace{}%
\AgdaOperator{\AgdaInductiveConstructor{≈̇}}\AgdaSpace{}%
\AgdaBound{q}\AgdaSymbol{)}\<%
\\
%
\>[1]\AgdaFunction{sound}\AgdaSpace{}%
\AgdaSymbol{(}\AgdaInductiveConstructor{hyp}\AgdaSpace{}%
\AgdaBound{i}\AgdaSymbol{)}\AgdaSpace{}%
\AgdaSymbol{=}\AgdaSpace{}%
\AgdaBound{V}\AgdaSpace{}%
\AgdaBound{i}\<%
\\
%
\>[1]\AgdaFunction{sound}\AgdaSpace{}%
\AgdaSymbol{(}\AgdaInductiveConstructor{app}\AgdaSpace{}%
\AgdaBound{es}\AgdaSymbol{)}\AgdaSpace{}%
\AgdaBound{ρ}\AgdaSpace{}%
\AgdaSymbol{=}\AgdaSpace{}%
\AgdaField{cong}\AgdaSpace{}%
\AgdaSymbol{(}\AgdaField{Interp}\AgdaSpace{}%
\AgdaBound{𝑨}\AgdaSymbol{)}\AgdaSpace{}%
\AgdaSymbol{(}\AgdaInductiveConstructor{≡.refl}\AgdaSpace{}%
\AgdaOperator{\AgdaInductiveConstructor{,}}\AgdaSpace{}%
\AgdaSymbol{λ}\AgdaSpace{}%
\AgdaBound{i}\AgdaSpace{}%
\AgdaSymbol{→}\AgdaSpace{}%
\AgdaFunction{sound}\AgdaSpace{}%
\AgdaSymbol{(}\AgdaBound{es}\AgdaSpace{}%
\AgdaBound{i}\AgdaSymbol{)}\AgdaSpace{}%
\AgdaBound{ρ}\AgdaSymbol{)}\<%
\\
%
\>[1]\AgdaFunction{sound}\AgdaSpace{}%
\AgdaSymbol{(}\AgdaInductiveConstructor{sub}\AgdaSpace{}%
\AgdaSymbol{\{}\AgdaArgument{p}\AgdaSpace{}%
\AgdaSymbol{=}\AgdaSpace{}%
\AgdaBound{p}\AgdaSymbol{\}\{}\AgdaBound{q}\AgdaSymbol{\}}\AgdaSpace{}%
\AgdaBound{Epq}\AgdaSpace{}%
\AgdaBound{σ}\AgdaSymbol{)}\AgdaSpace{}%
\AgdaBound{ρ}\AgdaSpace{}%
\AgdaSymbol{=}\<%
\\
\>[1][@{}l@{\AgdaIndent{0}}]%
\>[2]\AgdaOperator{\AgdaFunction{begin}}\<%
\\
\>[2][@{}l@{\AgdaIndent{0}}]%
\>[3]\AgdaOperator{\AgdaFunction{⟦}}\AgdaSpace{}%
\AgdaBound{p}\AgdaSpace{}%
\AgdaOperator{\AgdaFunction{[}}\AgdaSpace{}%
\AgdaBound{σ}\AgdaSpace{}%
\AgdaOperator{\AgdaFunction{]}}\AgdaSpace{}%
\AgdaOperator{\AgdaFunction{⟧}}\AgdaSpace{}%
\AgdaOperator{\AgdaField{⟨\$⟩}}%
\>[27]\AgdaBound{ρ}\AgdaSpace{}%
\AgdaFunction{≈⟨}%
\>[34]\AgdaFunction{substitution}\AgdaSpace{}%
\AgdaBound{p}\AgdaSpace{}%
\AgdaBound{σ}\AgdaSpace{}%
\AgdaBound{ρ}%
\>[56]\AgdaFunction{⟩}\<%
\\
%
\>[3]\AgdaOperator{\AgdaFunction{⟦}}\AgdaSpace{}%
\AgdaBound{p}%
\>[13]\AgdaOperator{\AgdaFunction{⟧}}\AgdaSpace{}%
\AgdaOperator{\AgdaField{⟨\$⟩}}\AgdaSpace{}%
\AgdaOperator{\AgdaFunction{⟦}}\AgdaSpace{}%
\AgdaBound{σ}\AgdaSpace{}%
\AgdaOperator{\AgdaFunction{⟧s}}%
\>[27]\AgdaBound{ρ}\AgdaSpace{}%
\AgdaFunction{≈⟨}%
\>[34]\AgdaFunction{sound}\AgdaSpace{}%
\AgdaBound{Epq}\AgdaSpace{}%
\AgdaSymbol{(}\AgdaOperator{\AgdaFunction{⟦}}\AgdaSpace{}%
\AgdaBound{σ}\AgdaSpace{}%
\AgdaOperator{\AgdaFunction{⟧s}}\AgdaSpace{}%
\AgdaBound{ρ}\AgdaSymbol{)}%
\>[56]\AgdaFunction{⟩}\<%
\\
%
\>[3]\AgdaOperator{\AgdaFunction{⟦}}\AgdaSpace{}%
\AgdaBound{q}%
\>[13]\AgdaOperator{\AgdaFunction{⟧}}\AgdaSpace{}%
\AgdaOperator{\AgdaField{⟨\$⟩}}\AgdaSpace{}%
\AgdaOperator{\AgdaFunction{⟦}}\AgdaSpace{}%
\AgdaBound{σ}\AgdaSpace{}%
\AgdaOperator{\AgdaFunction{⟧s}}%
\>[27]\AgdaBound{ρ}\AgdaSpace{}%
\AgdaFunction{≈˘⟨}%
\>[34]\AgdaFunction{substitution}%
\>[48]\AgdaBound{q}\AgdaSpace{}%
\AgdaBound{σ}\AgdaSpace{}%
\AgdaBound{ρ}%
\>[56]\AgdaFunction{⟩}\<%
\\
%
\>[3]\AgdaOperator{\AgdaFunction{⟦}}\AgdaSpace{}%
\AgdaBound{q}\AgdaSpace{}%
\AgdaOperator{\AgdaFunction{[}}\AgdaSpace{}%
\AgdaBound{σ}\AgdaSpace{}%
\AgdaOperator{\AgdaFunction{]}}\AgdaSpace{}%
\AgdaOperator{\AgdaFunction{⟧}}\AgdaSpace{}%
\AgdaOperator{\AgdaField{⟨\$⟩}}%
\>[27]\AgdaBound{ρ}\AgdaSpace{}%
\AgdaOperator{\AgdaFunction{∎}}\<%
\\
%
\>[1]\AgdaFunction{sound}\AgdaSpace{}%
\AgdaSymbol{(}\AgdaSpace{}%
\AgdaInductiveConstructor{⊢refl}%
\>[17]\AgdaSymbol{\{}\AgdaArgument{p}\AgdaSpace{}%
\AgdaSymbol{=}\AgdaSpace{}%
\AgdaBound{p}\AgdaSymbol{\}}%
\>[41]\AgdaSymbol{)}\AgdaSpace{}%
\AgdaSymbol{=}\AgdaSpace{}%
\AgdaField{refl}%
\>[52]\AgdaFunction{EqualIsEquiv}\AgdaSpace{}%
\AgdaSymbol{\{}\AgdaArgument{x}\AgdaSpace{}%
\AgdaSymbol{=}\AgdaSpace{}%
\AgdaBound{p}\AgdaSymbol{\}}\<%
\\
%
\>[1]\AgdaFunction{sound}\AgdaSpace{}%
\AgdaSymbol{(}\AgdaSpace{}%
\AgdaInductiveConstructor{⊢sym}%
\>[17]\AgdaSymbol{\{}\AgdaArgument{p}\AgdaSpace{}%
\AgdaSymbol{=}\AgdaSpace{}%
\AgdaBound{p}\AgdaSymbol{\}\{}\AgdaBound{q}\AgdaSymbol{\}}%
\>[32]\AgdaBound{Epq}%
\>[41]\AgdaSymbol{)}\AgdaSpace{}%
\AgdaSymbol{=}\AgdaSpace{}%
\AgdaField{sym}%
\>[52]\AgdaFunction{EqualIsEquiv}\AgdaSpace{}%
\AgdaSymbol{\{}\AgdaArgument{x}\AgdaSpace{}%
\AgdaSymbol{=}\AgdaSpace{}%
\AgdaBound{p}\AgdaSymbol{\}\{}\AgdaBound{q}\AgdaSymbol{\}}%
\>[80]\AgdaSymbol{(}\AgdaFunction{sound}\AgdaSpace{}%
\AgdaBound{Epq}\AgdaSymbol{)}\<%
\\
%
\>[1]\AgdaFunction{sound}\AgdaSpace{}%
\AgdaSymbol{(}\AgdaSpace{}%
\AgdaInductiveConstructor{⊢trans}%
\>[17]\AgdaSymbol{\{}\AgdaArgument{p}\AgdaSpace{}%
\AgdaSymbol{=}\AgdaSpace{}%
\AgdaBound{p}\AgdaSymbol{\}\{}\AgdaBound{q}\AgdaSymbol{\}\{}\AgdaBound{r}\AgdaSymbol{\}}%
\>[32]\AgdaBound{Epq}\AgdaSpace{}%
\AgdaBound{Eqr}%
\>[41]\AgdaSymbol{)}\AgdaSpace{}%
\AgdaSymbol{=}\AgdaSpace{}%
\AgdaField{trans}%
\>[52]\AgdaFunction{EqualIsEquiv}\AgdaSpace{}%
\AgdaSymbol{\{}\AgdaArgument{i}\AgdaSpace{}%
\AgdaSymbol{=}\AgdaSpace{}%
\AgdaBound{p}\AgdaSymbol{\}\{}\AgdaBound{q}\AgdaSymbol{\}\{}\AgdaBound{r}\AgdaSymbol{\}}%
\>[80]\AgdaSymbol{(}\AgdaFunction{sound}\AgdaSpace{}%
\AgdaBound{Epq}\AgdaSymbol{)(}\AgdaFunction{sound}\AgdaSpace{}%
\AgdaBound{Eqr}\AgdaSymbol{)}\<%
\end{code}

\hypertarget{the-closure-operators-h-s-p-and-v}{%
\subsubsection{The Closure Operators H, S, P and
V}\label{the-closure-operators-h-s-p-and-v}}

Fix a signature \texttt{𝑆}, let \texttt{𝒦} be a class of
\texttt{𝑆}-algebras, and define

\begin{itemize}
\tightlist
\item
  \texttt{H\ 𝒦} = algebras isomorphic to a homomorphic image of a member
  of \texttt{𝒦};
\item
  \texttt{S\ 𝒦} = algebras isomorphic to a subalgebra of a member of
  \texttt{𝒦};
\item
  \texttt{P\ 𝒦} = algebras isomorphic to a product of members of
  \texttt{𝒦}.
\end{itemize}

A straight-forward verification confirms that \texttt{H}, \texttt{S},
and \texttt{P} are \emph{closure operators} (expansive, monotone, and
idempotent). A class \texttt{𝒦} of \texttt{𝑆}-algebras is said to be
\emph{closed under the taking of homomorphic images} provided
\texttt{H\ 𝒦\ ⊆\ 𝒦}. Similarly, \texttt{𝒦} is \emph{closed under the
taking of subalgebras} (resp., \emph{arbitrary products}) provided
\texttt{S\ 𝒦\ ⊆\ 𝒦} (resp., \texttt{P\ 𝒦\ ⊆\ 𝒦}). The operators
\texttt{H}, \texttt{S}, and \texttt{P} can be composed with one another
repeatedly, forming yet more closure operators.

An algebra is a homomorphic image (resp., subalgebra; resp., product) of
every algebra to which it is isomorphic. Thus, the class \texttt{H\ 𝒦}
(resp., \texttt{S\ 𝒦}; resp., \texttt{P\ 𝒦}) is closed under
isomorphism.

A \emph{variety} is a class of \texttt{𝑆}-algebras that is closed under
the taking of homomorphic images, subalgebras, and arbitrary products.
To represent varieties we define types for the closure operators
\texttt{H}, \texttt{S}, and \texttt{P} that are composable. Separately,
we define a type \texttt{V} which represents closure under all three
operators, \texttt{H}, \texttt{S}, and \texttt{P}.

We now define the type \texttt{H} to represent classes of algebras that
include all homomorphic images of algebras in the class---i.e., classes
that are closed under the taking of homomorphic images---the type
\texttt{S} to represent classes of algebras that closed under the taking
of subalgebras, and the type \texttt{P} to represent classes of algebras
closed under the taking of arbitrary products.

\begin{code}%
\>[0]\<%
\\
\>[0]\AgdaKeyword{module}\AgdaSpace{}%
\AgdaModule{\AgdaUnderscore{}}%
\>[10]\AgdaSymbol{\{}\AgdaBound{α}\AgdaSpace{}%
\AgdaBound{ρᵃ}\AgdaSpace{}%
\AgdaBound{β}\AgdaSpace{}%
\AgdaBound{ρᵇ}\AgdaSpace{}%
\AgdaSymbol{:}\AgdaSpace{}%
\AgdaPostulate{Level}\AgdaSymbol{\}}\AgdaSpace{}%
\AgdaKeyword{where}\<%
\\
\>[0][@{}l@{\AgdaIndent{0}}]%
\>[1]\AgdaKeyword{private}\AgdaSpace{}%
\AgdaFunction{a}\AgdaSpace{}%
\AgdaSymbol{=}\AgdaSpace{}%
\AgdaBound{α}\AgdaSpace{}%
\AgdaOperator{\AgdaPrimitive{⊔}}\AgdaSpace{}%
\AgdaBound{ρᵃ}\AgdaSpace{}%
\AgdaSymbol{;}\AgdaSpace{}%
\AgdaFunction{b}\AgdaSpace{}%
\AgdaSymbol{=}\AgdaSpace{}%
\AgdaBound{β}\AgdaSpace{}%
\AgdaOperator{\AgdaPrimitive{⊔}}\AgdaSpace{}%
\AgdaBound{ρᵇ}\<%
\\
%
\\[\AgdaEmptyExtraSkip]%
%
\>[1]\AgdaFunction{H}\AgdaSpace{}%
\AgdaSymbol{:}\AgdaSpace{}%
\AgdaSymbol{∀}\AgdaSpace{}%
\AgdaBound{ℓ}\AgdaSpace{}%
\AgdaSymbol{→}\AgdaSpace{}%
\AgdaFunction{Pred}\AgdaSymbol{(}\AgdaRecord{Algebra}\AgdaSpace{}%
\AgdaBound{α}\AgdaSpace{}%
\AgdaBound{ρᵃ}\AgdaSymbol{)}\AgdaSpace{}%
\AgdaSymbol{(}\AgdaFunction{a}\AgdaSpace{}%
\AgdaOperator{\AgdaPrimitive{⊔}}\AgdaSpace{}%
\AgdaFunction{ov}\AgdaSpace{}%
\AgdaBound{ℓ}\AgdaSymbol{)}\AgdaSpace{}%
\AgdaSymbol{→}\AgdaSpace{}%
\AgdaFunction{Pred}\AgdaSymbol{(}\AgdaRecord{Algebra}\AgdaSpace{}%
\AgdaBound{β}\AgdaSpace{}%
\AgdaBound{ρᵇ}\AgdaSymbol{)}\AgdaSpace{}%
\AgdaSymbol{(}\AgdaFunction{b}\AgdaSpace{}%
\AgdaOperator{\AgdaPrimitive{⊔}}\AgdaSpace{}%
\AgdaFunction{ov}\AgdaSymbol{(}\AgdaFunction{a}\AgdaSpace{}%
\AgdaOperator{\AgdaPrimitive{⊔}}\AgdaSpace{}%
\AgdaBound{ℓ}\AgdaSymbol{))}\<%
\\
%
\>[1]\AgdaFunction{H}\AgdaSpace{}%
\AgdaSymbol{\AgdaUnderscore{}}\AgdaSpace{}%
\AgdaBound{𝒦}\AgdaSpace{}%
\AgdaBound{𝑩}\AgdaSpace{}%
\AgdaSymbol{=}\AgdaSpace{}%
\AgdaFunction{Σ[}\AgdaSpace{}%
\AgdaBound{𝑨}\AgdaSpace{}%
\AgdaFunction{∈}\AgdaSpace{}%
\AgdaRecord{Algebra}\AgdaSpace{}%
\AgdaBound{α}\AgdaSpace{}%
\AgdaBound{ρᵃ}\AgdaSpace{}%
\AgdaFunction{]}\AgdaSpace{}%
\AgdaBound{𝑨}\AgdaSpace{}%
\AgdaOperator{\AgdaFunction{∈}}\AgdaSpace{}%
\AgdaBound{𝒦}\AgdaSpace{}%
\AgdaOperator{\AgdaFunction{×}}\AgdaSpace{}%
\AgdaBound{𝑩}\AgdaSpace{}%
\AgdaOperator{\AgdaFunction{IsHomImageOf}}\AgdaSpace{}%
\AgdaBound{𝑨}\<%
\\
%
\\[\AgdaEmptyExtraSkip]%
%
\>[1]\AgdaFunction{S}\AgdaSpace{}%
\AgdaSymbol{:}\AgdaSpace{}%
\AgdaSymbol{∀}\AgdaSpace{}%
\AgdaBound{ℓ}\AgdaSpace{}%
\AgdaSymbol{→}\AgdaSpace{}%
\AgdaFunction{Pred}\AgdaSymbol{(}\AgdaRecord{Algebra}\AgdaSpace{}%
\AgdaBound{α}\AgdaSpace{}%
\AgdaBound{ρᵃ}\AgdaSymbol{)}\AgdaSpace{}%
\AgdaSymbol{(}\AgdaFunction{a}\AgdaSpace{}%
\AgdaOperator{\AgdaPrimitive{⊔}}\AgdaSpace{}%
\AgdaFunction{ov}\AgdaSpace{}%
\AgdaBound{ℓ}\AgdaSymbol{)}\AgdaSpace{}%
\AgdaSymbol{→}\AgdaSpace{}%
\AgdaFunction{Pred}\AgdaSymbol{(}\AgdaRecord{Algebra}\AgdaSpace{}%
\AgdaBound{β}\AgdaSpace{}%
\AgdaBound{ρᵇ}\AgdaSymbol{)}\AgdaSpace{}%
\AgdaSymbol{(}\AgdaFunction{b}\AgdaSpace{}%
\AgdaOperator{\AgdaPrimitive{⊔}}\AgdaSpace{}%
\AgdaFunction{ov}\AgdaSymbol{(}\AgdaFunction{a}\AgdaSpace{}%
\AgdaOperator{\AgdaPrimitive{⊔}}\AgdaSpace{}%
\AgdaBound{ℓ}\AgdaSymbol{))}\<%
\\
%
\>[1]\AgdaFunction{S}\AgdaSpace{}%
\AgdaSymbol{\AgdaUnderscore{}}\AgdaSpace{}%
\AgdaBound{𝒦}\AgdaSpace{}%
\AgdaBound{𝑩}\AgdaSpace{}%
\AgdaSymbol{=}\AgdaSpace{}%
\AgdaFunction{Σ[}\AgdaSpace{}%
\AgdaBound{𝑨}\AgdaSpace{}%
\AgdaFunction{∈}\AgdaSpace{}%
\AgdaRecord{Algebra}\AgdaSpace{}%
\AgdaBound{α}\AgdaSpace{}%
\AgdaBound{ρᵃ}\AgdaSpace{}%
\AgdaFunction{]}\AgdaSpace{}%
\AgdaBound{𝑨}\AgdaSpace{}%
\AgdaOperator{\AgdaFunction{∈}}\AgdaSpace{}%
\AgdaBound{𝒦}\AgdaSpace{}%
\AgdaOperator{\AgdaFunction{×}}\AgdaSpace{}%
\AgdaBound{𝑩}\AgdaSpace{}%
\AgdaOperator{\AgdaFunction{≤}}\AgdaSpace{}%
\AgdaBound{𝑨}\<%
\\
%
\\[\AgdaEmptyExtraSkip]%
%
\>[1]\AgdaFunction{P}\AgdaSpace{}%
\AgdaSymbol{:}\AgdaSpace{}%
\AgdaSymbol{∀}\AgdaSpace{}%
\AgdaBound{ℓ}\AgdaSpace{}%
\AgdaBound{ι}\AgdaSpace{}%
\AgdaSymbol{→}\AgdaSpace{}%
\AgdaFunction{Pred}\AgdaSymbol{(}\AgdaRecord{Algebra}\AgdaSpace{}%
\AgdaBound{α}\AgdaSpace{}%
\AgdaBound{ρᵃ}\AgdaSymbol{)}\AgdaSpace{}%
\AgdaSymbol{(}\AgdaFunction{a}\AgdaSpace{}%
\AgdaOperator{\AgdaPrimitive{⊔}}\AgdaSpace{}%
\AgdaFunction{ov}\AgdaSpace{}%
\AgdaBound{ℓ}\AgdaSymbol{)}\AgdaSpace{}%
\AgdaSymbol{→}\AgdaSpace{}%
\AgdaFunction{Pred}\AgdaSymbol{(}\AgdaRecord{Algebra}\AgdaSpace{}%
\AgdaBound{β}\AgdaSpace{}%
\AgdaBound{ρᵇ}\AgdaSymbol{)}\AgdaSpace{}%
\AgdaSymbol{(}\AgdaFunction{b}\AgdaSpace{}%
\AgdaOperator{\AgdaPrimitive{⊔}}\AgdaSpace{}%
\AgdaFunction{ov}\AgdaSymbol{(}\AgdaFunction{a}\AgdaSpace{}%
\AgdaOperator{\AgdaPrimitive{⊔}}\AgdaSpace{}%
\AgdaBound{ℓ}\AgdaSpace{}%
\AgdaOperator{\AgdaPrimitive{⊔}}\AgdaSpace{}%
\AgdaBound{ι}\AgdaSymbol{))}\<%
\\
%
\>[1]\AgdaFunction{P}\AgdaSpace{}%
\AgdaSymbol{\AgdaUnderscore{}}\AgdaSpace{}%
\AgdaBound{ι}\AgdaSpace{}%
\AgdaBound{𝒦}\AgdaSpace{}%
\AgdaBound{𝑩}\AgdaSpace{}%
\AgdaSymbol{=}\AgdaSpace{}%
\AgdaFunction{Σ[}\AgdaSpace{}%
\AgdaBound{I}\AgdaSpace{}%
\AgdaFunction{∈}\AgdaSpace{}%
\AgdaPrimitive{Type}\AgdaSpace{}%
\AgdaBound{ι}\AgdaSpace{}%
\AgdaFunction{]}\AgdaSpace{}%
\AgdaSymbol{(}\AgdaFunction{Σ[}\AgdaSpace{}%
\AgdaBound{𝒜}\AgdaSpace{}%
\AgdaFunction{∈}\AgdaSpace{}%
\AgdaSymbol{(}\AgdaBound{I}\AgdaSpace{}%
\AgdaSymbol{→}\AgdaSpace{}%
\AgdaRecord{Algebra}\AgdaSpace{}%
\AgdaBound{α}\AgdaSpace{}%
\AgdaBound{ρᵃ}\AgdaSymbol{)}\AgdaSpace{}%
\AgdaFunction{]}\AgdaSpace{}%
\AgdaSymbol{(∀}\AgdaSpace{}%
\AgdaBound{i}\AgdaSpace{}%
\AgdaSymbol{→}\AgdaSpace{}%
\AgdaBound{𝒜}\AgdaSpace{}%
\AgdaBound{i}\AgdaSpace{}%
\AgdaOperator{\AgdaFunction{∈}}\AgdaSpace{}%
\AgdaBound{𝒦}\AgdaSymbol{)}\AgdaSpace{}%
\AgdaOperator{\AgdaFunction{×}}\AgdaSpace{}%
\AgdaSymbol{(}\AgdaBound{𝑩}\AgdaSpace{}%
\AgdaOperator{\AgdaRecord{≅}}\AgdaSpace{}%
\AgdaFunction{⨅}\AgdaSpace{}%
\AgdaBound{𝒜}\AgdaSymbol{))}\<%
\\
\>[0]\<%
\end{code}

A class \texttt{𝒦} of \texttt{𝑆}-algebras is called a \emph{variety} if
it is closed under each of the closure operators \texttt{H}, \texttt{S},
and \texttt{P} defined above. The corresponding closure operator is
often denoted \texttt{𝕍} or \texttt{𝒱}, but we will denote it by
\texttt{V}.

\begin{code}%
\>[0]\<%
\\
\>[0]\AgdaKeyword{module}\AgdaSpace{}%
\AgdaModule{\AgdaUnderscore{}}%
\>[10]\AgdaSymbol{\{}\AgdaBound{α}\AgdaSpace{}%
\AgdaBound{ρᵃ}\AgdaSpace{}%
\AgdaBound{β}\AgdaSpace{}%
\AgdaBound{ρᵇ}\AgdaSpace{}%
\AgdaBound{γ}\AgdaSpace{}%
\AgdaBound{ρᶜ}\AgdaSpace{}%
\AgdaBound{δ}\AgdaSpace{}%
\AgdaBound{ρᵈ}\AgdaSpace{}%
\AgdaSymbol{:}\AgdaSpace{}%
\AgdaPostulate{Level}\AgdaSymbol{\}}\AgdaSpace{}%
\AgdaKeyword{where}\<%
\\
\>[0][@{}l@{\AgdaIndent{0}}]%
\>[1]\AgdaKeyword{private}\AgdaSpace{}%
\AgdaFunction{a}\AgdaSpace{}%
\AgdaSymbol{=}\AgdaSpace{}%
\AgdaBound{α}\AgdaSpace{}%
\AgdaOperator{\AgdaPrimitive{⊔}}\AgdaSpace{}%
\AgdaBound{ρᵃ}\AgdaSpace{}%
\AgdaSymbol{;}\AgdaSpace{}%
\AgdaFunction{b}\AgdaSpace{}%
\AgdaSymbol{=}\AgdaSpace{}%
\AgdaBound{β}\AgdaSpace{}%
\AgdaOperator{\AgdaPrimitive{⊔}}\AgdaSpace{}%
\AgdaBound{ρᵇ}\AgdaSpace{}%
\AgdaSymbol{;}\AgdaSpace{}%
\AgdaFunction{c}\AgdaSpace{}%
\AgdaSymbol{=}\AgdaSpace{}%
\AgdaBound{γ}\AgdaSpace{}%
\AgdaOperator{\AgdaPrimitive{⊔}}\AgdaSpace{}%
\AgdaBound{ρᶜ}\AgdaSpace{}%
\AgdaSymbol{;}\AgdaSpace{}%
\AgdaFunction{d}\AgdaSpace{}%
\AgdaSymbol{=}\AgdaSpace{}%
\AgdaBound{δ}\AgdaSpace{}%
\AgdaOperator{\AgdaPrimitive{⊔}}\AgdaSpace{}%
\AgdaBound{ρᵈ}\<%
\\
%
\\[\AgdaEmptyExtraSkip]%
%
\>[1]\AgdaFunction{V}\AgdaSpace{}%
\AgdaSymbol{:}\AgdaSpace{}%
\AgdaSymbol{∀}\AgdaSpace{}%
\AgdaBound{ℓ}\AgdaSpace{}%
\AgdaBound{ι}\AgdaSpace{}%
\AgdaSymbol{→}\AgdaSpace{}%
\AgdaFunction{Pred}\AgdaSymbol{(}\AgdaRecord{Algebra}\AgdaSpace{}%
\AgdaBound{α}\AgdaSpace{}%
\AgdaBound{ρᵃ}\AgdaSymbol{)}\AgdaSpace{}%
\AgdaSymbol{(}\AgdaFunction{a}\AgdaSpace{}%
\AgdaOperator{\AgdaPrimitive{⊔}}\AgdaSpace{}%
\AgdaFunction{ov}\AgdaSpace{}%
\AgdaBound{ℓ}\AgdaSymbol{)}\AgdaSpace{}%
\AgdaSymbol{→}%
\>[46]\AgdaFunction{Pred}\AgdaSymbol{(}\AgdaRecord{Algebra}\AgdaSpace{}%
\AgdaBound{δ}\AgdaSpace{}%
\AgdaBound{ρᵈ}\AgdaSymbol{)}\AgdaSpace{}%
\AgdaSymbol{(}\AgdaFunction{d}\AgdaSpace{}%
\AgdaOperator{\AgdaPrimitive{⊔}}\AgdaSpace{}%
\AgdaFunction{ov}\AgdaSymbol{(}\AgdaFunction{a}\AgdaSpace{}%
\AgdaOperator{\AgdaPrimitive{⊔}}\AgdaSpace{}%
\AgdaFunction{b}\AgdaSpace{}%
\AgdaOperator{\AgdaPrimitive{⊔}}\AgdaSpace{}%
\AgdaFunction{c}\AgdaSpace{}%
\AgdaOperator{\AgdaPrimitive{⊔}}\AgdaSpace{}%
\AgdaBound{ℓ}\AgdaSpace{}%
\AgdaOperator{\AgdaPrimitive{⊔}}\AgdaSpace{}%
\AgdaBound{ι}\AgdaSymbol{))}\<%
\\
%
\>[1]\AgdaFunction{V}\AgdaSpace{}%
\AgdaBound{ℓ}\AgdaSpace{}%
\AgdaBound{ι}\AgdaSpace{}%
\AgdaBound{𝒦}\AgdaSpace{}%
\AgdaSymbol{=}\AgdaSpace{}%
\AgdaFunction{H}\AgdaSymbol{\{}\AgdaBound{γ}\AgdaSymbol{\}\{}\AgdaBound{ρᶜ}\AgdaSymbol{\}\{}\AgdaBound{δ}\AgdaSymbol{\}\{}\AgdaBound{ρᵈ}\AgdaSymbol{\}}\AgdaSpace{}%
\AgdaSymbol{(}\AgdaFunction{a}\AgdaSpace{}%
\AgdaOperator{\AgdaPrimitive{⊔}}\AgdaSpace{}%
\AgdaFunction{b}\AgdaSpace{}%
\AgdaOperator{\AgdaPrimitive{⊔}}\AgdaSpace{}%
\AgdaBound{ℓ}\AgdaSpace{}%
\AgdaOperator{\AgdaPrimitive{⊔}}\AgdaSpace{}%
\AgdaBound{ι}\AgdaSymbol{)}\AgdaSpace{}%
\AgdaSymbol{(}\AgdaFunction{S}\AgdaSymbol{\{}\AgdaBound{β}\AgdaSymbol{\}\{}\AgdaBound{ρᵇ}\AgdaSymbol{\}}\AgdaSpace{}%
\AgdaSymbol{(}\AgdaFunction{a}\AgdaSpace{}%
\AgdaOperator{\AgdaPrimitive{⊔}}\AgdaSpace{}%
\AgdaBound{ℓ}\AgdaSpace{}%
\AgdaOperator{\AgdaPrimitive{⊔}}\AgdaSpace{}%
\AgdaBound{ι}\AgdaSymbol{)}\AgdaSpace{}%
\AgdaSymbol{(}\AgdaFunction{P}\AgdaSpace{}%
\AgdaBound{ℓ}\AgdaSpace{}%
\AgdaBound{ι}\AgdaSpace{}%
\AgdaBound{𝒦}\AgdaSymbol{))}\<%
\\
%
\\[\AgdaEmptyExtraSkip]%
\>[0]\AgdaKeyword{module}\AgdaSpace{}%
\AgdaModule{\AgdaUnderscore{}}%
\>[5070I]\AgdaSymbol{\{}\AgdaBound{α}\AgdaSpace{}%
\AgdaBound{ρᵃ}\AgdaSpace{}%
\AgdaBound{ℓ}\AgdaSpace{}%
\AgdaSymbol{:}\AgdaSpace{}%
\AgdaPostulate{Level}\AgdaSymbol{\}(}\AgdaBound{𝒦}\AgdaSpace{}%
\AgdaSymbol{:}\AgdaSpace{}%
\AgdaFunction{Pred}\AgdaSymbol{(}\AgdaRecord{Algebra}\AgdaSpace{}%
\AgdaBound{α}\AgdaSpace{}%
\AgdaBound{ρᵃ}\AgdaSymbol{)}\AgdaSpace{}%
\AgdaSymbol{(}\AgdaBound{α}\AgdaSpace{}%
\AgdaOperator{\AgdaPrimitive{⊔}}\AgdaSpace{}%
\AgdaBound{ρᵃ}\AgdaSpace{}%
\AgdaOperator{\AgdaPrimitive{⊔}}\AgdaSpace{}%
\AgdaFunction{ov}\AgdaSpace{}%
\AgdaBound{ℓ}\AgdaSymbol{))}\<%
\\
\>[.][@{}l@{}]\<[5070I]%
\>[9]\AgdaSymbol{(}\AgdaBound{𝑨}\AgdaSpace{}%
\AgdaSymbol{:}\AgdaSpace{}%
\AgdaRecord{Algebra}\AgdaSpace{}%
\AgdaSymbol{(}\AgdaBound{α}\AgdaSpace{}%
\AgdaOperator{\AgdaPrimitive{⊔}}\AgdaSpace{}%
\AgdaBound{ρᵃ}\AgdaSpace{}%
\AgdaOperator{\AgdaPrimitive{⊔}}\AgdaSpace{}%
\AgdaBound{ℓ}\AgdaSymbol{)}\AgdaSpace{}%
\AgdaSymbol{(}\AgdaBound{α}\AgdaSpace{}%
\AgdaOperator{\AgdaPrimitive{⊔}}\AgdaSpace{}%
\AgdaBound{ρᵃ}\AgdaSpace{}%
\AgdaOperator{\AgdaPrimitive{⊔}}\AgdaSpace{}%
\AgdaBound{ℓ}\AgdaSymbol{))}\AgdaSpace{}%
\AgdaKeyword{where}\<%
\\
\>[0][@{}l@{\AgdaIndent{0}}]%
\>[1]\AgdaKeyword{private}\AgdaSpace{}%
\AgdaFunction{ι}\AgdaSpace{}%
\AgdaSymbol{=}\AgdaSpace{}%
\AgdaFunction{ov}\AgdaSymbol{(}\AgdaBound{α}\AgdaSpace{}%
\AgdaOperator{\AgdaPrimitive{⊔}}\AgdaSpace{}%
\AgdaBound{ρᵃ}\AgdaSpace{}%
\AgdaOperator{\AgdaPrimitive{⊔}}\AgdaSpace{}%
\AgdaBound{ℓ}\AgdaSymbol{)}\<%
\\
%
\\[\AgdaEmptyExtraSkip]%
%
\>[1]\AgdaFunction{V-≅-lc}\AgdaSpace{}%
\AgdaSymbol{:}\AgdaSpace{}%
\AgdaFunction{Lift-Alg}\AgdaSpace{}%
\AgdaBound{𝑨}\AgdaSpace{}%
\AgdaFunction{ι}\AgdaSpace{}%
\AgdaFunction{ι}\AgdaSpace{}%
\AgdaOperator{\AgdaFunction{∈}}\AgdaSpace{}%
\AgdaFunction{V}\AgdaSymbol{\{}\AgdaArgument{β}\AgdaSpace{}%
\AgdaSymbol{=}\AgdaSpace{}%
\AgdaFunction{ι}\AgdaSymbol{\}\{}\AgdaFunction{ι}\AgdaSymbol{\}}\AgdaSpace{}%
\AgdaBound{ℓ}\AgdaSpace{}%
\AgdaFunction{ι}\AgdaSpace{}%
\AgdaBound{𝒦}\AgdaSpace{}%
\AgdaSymbol{→}\AgdaSpace{}%
\AgdaBound{𝑨}\AgdaSpace{}%
\AgdaOperator{\AgdaFunction{∈}}\AgdaSpace{}%
\AgdaFunction{V}\AgdaSymbol{\{}\AgdaArgument{γ}\AgdaSpace{}%
\AgdaSymbol{=}\AgdaSpace{}%
\AgdaFunction{ι}\AgdaSymbol{\}\{}\AgdaFunction{ι}\AgdaSymbol{\}}\AgdaSpace{}%
\AgdaBound{ℓ}\AgdaSpace{}%
\AgdaFunction{ι}\AgdaSpace{}%
\AgdaBound{𝒦}\<%
\\
%
\>[1]\AgdaFunction{V-≅-lc}\AgdaSpace{}%
\AgdaSymbol{(}\AgdaBound{𝑨'}\AgdaSpace{}%
\AgdaOperator{\AgdaInductiveConstructor{,}}\AgdaSpace{}%
\AgdaBound{spA'}\AgdaSpace{}%
\AgdaOperator{\AgdaInductiveConstructor{,}}\AgdaSpace{}%
\AgdaBound{lAimgA'}\AgdaSymbol{)}\AgdaSpace{}%
\AgdaSymbol{=}\AgdaSpace{}%
\AgdaBound{𝑨'}\AgdaSpace{}%
\AgdaOperator{\AgdaInductiveConstructor{,}}\AgdaSpace{}%
\AgdaSymbol{(}\AgdaBound{spA'}\AgdaSpace{}%
\AgdaOperator{\AgdaInductiveConstructor{,}}\AgdaSpace{}%
\AgdaFunction{AimgA'}\AgdaSymbol{)}\<%
\\
\>[1][@{}l@{\AgdaIndent{0}}]%
\>[2]\AgdaKeyword{where}\<%
\\
%
\>[2]\AgdaFunction{AimgA'}\AgdaSpace{}%
\AgdaSymbol{:}\AgdaSpace{}%
\AgdaBound{𝑨}\AgdaSpace{}%
\AgdaOperator{\AgdaFunction{IsHomImageOf}}\AgdaSpace{}%
\AgdaBound{𝑨'}\<%
\\
%
\>[2]\AgdaFunction{AimgA'}\AgdaSpace{}%
\AgdaSymbol{=}\AgdaSpace{}%
\AgdaFunction{Lift-HomImage-lemma}\AgdaSpace{}%
\AgdaBound{lAimgA'}\<%
\end{code}

\hypertarget{idempotence-of-s}{%
\paragraph{Idempotence of S}\label{idempotence-of-s}}

\texttt{S} is a closure operator. The facts that S is monotone and
expansive won't be needed, so we omit the proof of these facts. However,
we will make use of idempotence of \texttt{S}, so we prove that property
as follows.

\begin{code}%
\>[0]\<%
\\
\>[0]\AgdaFunction{S-idem}\AgdaSpace{}%
\AgdaSymbol{:}\AgdaSpace{}%
\AgdaSymbol{\{}\AgdaBound{𝒦}\AgdaSpace{}%
\AgdaSymbol{:}\AgdaSpace{}%
\AgdaFunction{Pred}\AgdaSpace{}%
\AgdaSymbol{(}\AgdaRecord{Algebra}\AgdaSpace{}%
\AgdaGeneralizable{α}\AgdaSpace{}%
\AgdaGeneralizable{ρᵃ}\AgdaSymbol{)(}\AgdaGeneralizable{α}\AgdaSpace{}%
\AgdaOperator{\AgdaPrimitive{⊔}}\AgdaSpace{}%
\AgdaGeneralizable{ρᵃ}\AgdaSpace{}%
\AgdaOperator{\AgdaPrimitive{⊔}}\AgdaSpace{}%
\AgdaFunction{ov}\AgdaSpace{}%
\AgdaGeneralizable{ℓ}\AgdaSymbol{)\}}\<%
\\
\>[0][@{}l@{\AgdaIndent{0}}]%
\>[1]\AgdaSymbol{→}%
\>[9]\AgdaFunction{S}\AgdaSymbol{\{}\AgdaArgument{β}\AgdaSpace{}%
\AgdaSymbol{=}\AgdaSpace{}%
\AgdaGeneralizable{γ}\AgdaSymbol{\}\{}\AgdaGeneralizable{ρᶜ}\AgdaSymbol{\}}\AgdaSpace{}%
\AgdaSymbol{(}\AgdaGeneralizable{α}\AgdaSpace{}%
\AgdaOperator{\AgdaPrimitive{⊔}}\AgdaSpace{}%
\AgdaGeneralizable{ρᵃ}%
\>[31]\AgdaOperator{\AgdaPrimitive{⊔}}\AgdaSpace{}%
\AgdaGeneralizable{ℓ}\AgdaSymbol{)}\AgdaSpace{}%
\AgdaSymbol{(}\AgdaFunction{S}\AgdaSymbol{\{}\AgdaArgument{β}\AgdaSpace{}%
\AgdaSymbol{=}\AgdaSpace{}%
\AgdaGeneralizable{β}\AgdaSymbol{\}\{}\AgdaGeneralizable{ρᵇ}\AgdaSymbol{\}}\AgdaSpace{}%
\AgdaGeneralizable{ℓ}\AgdaSpace{}%
\AgdaBound{𝒦}\AgdaSymbol{)}\AgdaSpace{}%
\AgdaOperator{\AgdaFunction{⊆}}\AgdaSpace{}%
\AgdaFunction{S}\AgdaSymbol{\{}\AgdaArgument{β}\AgdaSpace{}%
\AgdaSymbol{=}\AgdaSpace{}%
\AgdaGeneralizable{γ}\AgdaSymbol{\}\{}\AgdaGeneralizable{ρᶜ}\AgdaSymbol{\}}\AgdaSpace{}%
\AgdaGeneralizable{ℓ}\AgdaSpace{}%
\AgdaBound{𝒦}\<%
\\
%
\\[\AgdaEmptyExtraSkip]%
\>[0]\AgdaFunction{S-idem}\AgdaSpace{}%
\AgdaSymbol{(}\AgdaBound{𝑨}\AgdaSpace{}%
\AgdaOperator{\AgdaInductiveConstructor{,}}\AgdaSpace{}%
\AgdaSymbol{(}\AgdaBound{𝑩}\AgdaSpace{}%
\AgdaOperator{\AgdaInductiveConstructor{,}}\AgdaSpace{}%
\AgdaBound{sB}\AgdaSpace{}%
\AgdaOperator{\AgdaInductiveConstructor{,}}\AgdaSpace{}%
\AgdaBound{A≤B}\AgdaSymbol{)}\AgdaSpace{}%
\AgdaOperator{\AgdaInductiveConstructor{,}}\AgdaSpace{}%
\AgdaBound{x≤A}\AgdaSymbol{)}\AgdaSpace{}%
\AgdaSymbol{=}\AgdaSpace{}%
\AgdaBound{𝑩}\AgdaSpace{}%
\AgdaOperator{\AgdaInductiveConstructor{,}}\AgdaSpace{}%
\AgdaSymbol{(}\AgdaBound{sB}\AgdaSpace{}%
\AgdaOperator{\AgdaInductiveConstructor{,}}\AgdaSpace{}%
\AgdaFunction{≤-trans}\AgdaSpace{}%
\AgdaBound{x≤A}\AgdaSpace{}%
\AgdaBound{A≤B}\AgdaSymbol{)}\<%
\end{code}

\hypertarget{algebraic-invariance-of}{%
\paragraph{Algebraic invariance of ⊧}\label{algebraic-invariance-of}}

The binary relation \texttt{⊧} would be practically useless if it were
not an \emph{algebraic invariant} (i.e., invariant under isomorphism).
Let us now verify that the models relation we defined above has this
essential property.

\begin{code}%
\>[0]\<%
\\
\>[0]\AgdaKeyword{module}\AgdaSpace{}%
\AgdaModule{\AgdaUnderscore{}}\AgdaSpace{}%
\AgdaSymbol{\{}\AgdaBound{X}\AgdaSpace{}%
\AgdaSymbol{:}\AgdaSpace{}%
\AgdaPrimitive{Type}\AgdaSpace{}%
\AgdaGeneralizable{χ}\AgdaSymbol{\}\{}\AgdaBound{𝑨}\AgdaSpace{}%
\AgdaSymbol{:}\AgdaSpace{}%
\AgdaRecord{Algebra}\AgdaSpace{}%
\AgdaGeneralizable{α}\AgdaSpace{}%
\AgdaGeneralizable{ρᵃ}\AgdaSymbol{\}(}\AgdaBound{𝑩}\AgdaSpace{}%
\AgdaSymbol{:}\AgdaSpace{}%
\AgdaRecord{Algebra}\AgdaSpace{}%
\AgdaGeneralizable{β}\AgdaSpace{}%
\AgdaGeneralizable{ρᵇ}\AgdaSymbol{)(}\AgdaBound{p}\AgdaSpace{}%
\AgdaBound{q}\AgdaSpace{}%
\AgdaSymbol{:}\AgdaSpace{}%
\AgdaDatatype{Term}\AgdaSpace{}%
\AgdaBound{X}\AgdaSymbol{)}\AgdaSpace{}%
\AgdaKeyword{where}\<%
\\
%
\\[\AgdaEmptyExtraSkip]%
\>[0][@{}l@{\AgdaIndent{0}}]%
\>[1]\AgdaFunction{⊧-I-invar}\AgdaSpace{}%
\AgdaSymbol{:}\AgdaSpace{}%
\AgdaBound{𝑨}\AgdaSpace{}%
\AgdaOperator{\AgdaFunction{⊧}}\AgdaSpace{}%
\AgdaSymbol{(}\AgdaBound{p}\AgdaSpace{}%
\AgdaOperator{\AgdaInductiveConstructor{≈̇}}\AgdaSpace{}%
\AgdaBound{q}\AgdaSymbol{)}%
\>[27]\AgdaSymbol{→}%
\>[30]\AgdaBound{𝑨}\AgdaSpace{}%
\AgdaOperator{\AgdaRecord{≅}}\AgdaSpace{}%
\AgdaBound{𝑩}%
\>[37]\AgdaSymbol{→}%
\>[40]\AgdaBound{𝑩}\AgdaSpace{}%
\AgdaOperator{\AgdaFunction{⊧}}\AgdaSpace{}%
\AgdaSymbol{(}\AgdaBound{p}\AgdaSpace{}%
\AgdaOperator{\AgdaInductiveConstructor{≈̇}}\AgdaSpace{}%
\AgdaBound{q}\AgdaSymbol{)}\<%
\\
%
\>[1]\AgdaFunction{⊧-I-invar}\AgdaSpace{}%
\AgdaBound{Apq}\AgdaSpace{}%
\AgdaSymbol{(}\AgdaInductiveConstructor{mkiso}\AgdaSpace{}%
\AgdaBound{fh}\AgdaSpace{}%
\AgdaBound{gh}\AgdaSpace{}%
\AgdaBound{f∼g}\AgdaSpace{}%
\AgdaBound{g∼f}\AgdaSymbol{)}\AgdaSpace{}%
\AgdaBound{ρ}\AgdaSpace{}%
\AgdaSymbol{=}\<%
\\
\>[1][@{}l@{\AgdaIndent{0}}]%
\>[2]\AgdaOperator{\AgdaFunction{begin}}\<%
\\
\>[2][@{}l@{\AgdaIndent{0}}]%
\>[3]\AgdaOperator{\AgdaFunction{⟦}}\AgdaSpace{}%
\AgdaBound{p}\AgdaSpace{}%
\AgdaOperator{\AgdaFunction{⟧₂}}\AgdaSpace{}%
\AgdaOperator{\AgdaField{⟨\$⟩}}\AgdaSpace{}%
\AgdaBound{ρ}%
\>[29]\AgdaFunction{≈˘⟨}%
\>[34]\AgdaField{cong}\AgdaSpace{}%
\AgdaOperator{\AgdaFunction{⟦}}\AgdaSpace{}%
\AgdaBound{p}\AgdaSpace{}%
\AgdaOperator{\AgdaFunction{⟧₂}}\AgdaSpace{}%
\AgdaSymbol{(λ}\AgdaSpace{}%
\AgdaBound{x}\AgdaSpace{}%
\AgdaSymbol{→}\AgdaSpace{}%
\AgdaBound{f∼g}\AgdaSpace{}%
\AgdaSymbol{(}\AgdaBound{ρ}\AgdaSpace{}%
\AgdaBound{x}\AgdaSymbol{))}%
\>[65]\AgdaFunction{⟩}\<%
\\
%
\>[3]\AgdaOperator{\AgdaFunction{⟦}}\AgdaSpace{}%
\AgdaBound{p}\AgdaSpace{}%
\AgdaOperator{\AgdaFunction{⟧₂}}\AgdaSpace{}%
\AgdaOperator{\AgdaField{⟨\$⟩}}\AgdaSpace{}%
\AgdaSymbol{(}\AgdaFunction{f}\AgdaSpace{}%
\AgdaOperator{\AgdaFunction{∘}}\AgdaSpace{}%
\AgdaSymbol{(}\AgdaFunction{g}\AgdaSpace{}%
\AgdaOperator{\AgdaFunction{∘}}\AgdaSpace{}%
\AgdaBound{ρ}\AgdaSymbol{))}%
\>[29]\AgdaFunction{≈˘⟨}%
\>[34]\AgdaFunction{comm-hom-term}\AgdaSpace{}%
\AgdaBound{fh}\AgdaSpace{}%
\AgdaBound{p}\AgdaSpace{}%
\AgdaSymbol{(}\AgdaFunction{g}\AgdaSpace{}%
\AgdaOperator{\AgdaFunction{∘}}\AgdaSpace{}%
\AgdaBound{ρ}\AgdaSymbol{)}%
\>[65]\AgdaFunction{⟩}\<%
\\
%
\>[3]\AgdaFunction{f}\AgdaSpace{}%
\AgdaSymbol{(}\AgdaOperator{\AgdaFunction{⟦}}\AgdaSpace{}%
\AgdaBound{p}\AgdaSpace{}%
\AgdaOperator{\AgdaFunction{⟧₁}}\AgdaSpace{}%
\AgdaOperator{\AgdaField{⟨\$⟩}}\AgdaSpace{}%
\AgdaSymbol{(}\AgdaFunction{g}\AgdaSpace{}%
\AgdaOperator{\AgdaFunction{∘}}\AgdaSpace{}%
\AgdaBound{ρ}\AgdaSymbol{))}%
\>[29]\AgdaFunction{≈⟨}%
\>[34]\AgdaField{cong}\AgdaSpace{}%
\AgdaOperator{\AgdaFunction{∣}}\AgdaSpace{}%
\AgdaBound{fh}\AgdaSpace{}%
\AgdaOperator{\AgdaFunction{∣}}\AgdaSpace{}%
\AgdaSymbol{(}\AgdaBound{Apq}\AgdaSpace{}%
\AgdaSymbol{(}\AgdaFunction{g}\AgdaSpace{}%
\AgdaOperator{\AgdaFunction{∘}}\AgdaSpace{}%
\AgdaBound{ρ}\AgdaSymbol{))}%
\>[65]\AgdaFunction{⟩}\<%
\\
%
\>[3]\AgdaFunction{f}\AgdaSpace{}%
\AgdaSymbol{(}\AgdaOperator{\AgdaFunction{⟦}}\AgdaSpace{}%
\AgdaBound{q}\AgdaSpace{}%
\AgdaOperator{\AgdaFunction{⟧₁}}\AgdaSpace{}%
\AgdaOperator{\AgdaField{⟨\$⟩}}\AgdaSpace{}%
\AgdaSymbol{(}\AgdaFunction{g}\AgdaSpace{}%
\AgdaOperator{\AgdaFunction{∘}}\AgdaSpace{}%
\AgdaBound{ρ}\AgdaSymbol{))}%
\>[29]\AgdaFunction{≈⟨}%
\>[34]\AgdaFunction{comm-hom-term}\AgdaSpace{}%
\AgdaBound{fh}\AgdaSpace{}%
\AgdaBound{q}\AgdaSpace{}%
\AgdaSymbol{(}\AgdaFunction{g}\AgdaSpace{}%
\AgdaOperator{\AgdaFunction{∘}}\AgdaSpace{}%
\AgdaBound{ρ}\AgdaSymbol{)}%
\>[65]\AgdaFunction{⟩}\<%
\\
%
\>[3]\AgdaOperator{\AgdaFunction{⟦}}\AgdaSpace{}%
\AgdaBound{q}\AgdaSpace{}%
\AgdaOperator{\AgdaFunction{⟧₂}}\AgdaSpace{}%
\AgdaOperator{\AgdaField{⟨\$⟩}}\AgdaSpace{}%
\AgdaSymbol{(}\AgdaFunction{f}\AgdaSpace{}%
\AgdaOperator{\AgdaFunction{∘}}\AgdaSpace{}%
\AgdaSymbol{(}\AgdaFunction{g}\AgdaSpace{}%
\AgdaOperator{\AgdaFunction{∘}}\AgdaSpace{}%
\AgdaBound{ρ}\AgdaSymbol{))}%
\>[29]\AgdaFunction{≈⟨}%
\>[34]\AgdaField{cong}\AgdaSpace{}%
\AgdaOperator{\AgdaFunction{⟦}}\AgdaSpace{}%
\AgdaBound{q}\AgdaSpace{}%
\AgdaOperator{\AgdaFunction{⟧₂}}\AgdaSpace{}%
\AgdaSymbol{(λ}\AgdaSpace{}%
\AgdaBound{x}\AgdaSpace{}%
\AgdaSymbol{→}\AgdaSpace{}%
\AgdaBound{f∼g}\AgdaSpace{}%
\AgdaSymbol{(}\AgdaBound{ρ}\AgdaSpace{}%
\AgdaBound{x}\AgdaSymbol{))}%
\>[65]\AgdaFunction{⟩}\<%
\\
%
\>[3]\AgdaOperator{\AgdaFunction{⟦}}\AgdaSpace{}%
\AgdaBound{q}\AgdaSpace{}%
\AgdaOperator{\AgdaFunction{⟧₂}}\AgdaSpace{}%
\AgdaOperator{\AgdaField{⟨\$⟩}}\AgdaSpace{}%
\AgdaBound{ρ}%
\>[29]\AgdaOperator{\AgdaFunction{∎}}\<%
\\
%
\>[2]\AgdaKeyword{where}\<%
\\
%
\>[2]\AgdaKeyword{open}\AgdaSpace{}%
\AgdaModule{Environment}\AgdaSpace{}%
\AgdaBound{𝑨}%
\>[25]\AgdaKeyword{using}\AgdaSpace{}%
\AgdaSymbol{()}\AgdaSpace{}%
\AgdaKeyword{renaming}\AgdaSpace{}%
\AgdaSymbol{(}\AgdaSpace{}%
\AgdaOperator{\AgdaFunction{⟦\AgdaUnderscore{}⟧}}\AgdaSpace{}%
\AgdaSymbol{to}\AgdaSpace{}%
\AgdaOperator{\AgdaFunction{⟦\AgdaUnderscore{}⟧₁}}\AgdaSpace{}%
\AgdaSymbol{)}\<%
\\
%
\>[2]\AgdaKeyword{open}\AgdaSpace{}%
\AgdaModule{Environment}\AgdaSpace{}%
\AgdaBound{𝑩}%
\>[25]\AgdaKeyword{using}\AgdaSpace{}%
\AgdaSymbol{()}\AgdaSpace{}%
\AgdaKeyword{renaming}\AgdaSpace{}%
\AgdaSymbol{(}\AgdaSpace{}%
\AgdaOperator{\AgdaFunction{⟦\AgdaUnderscore{}⟧}}\AgdaSpace{}%
\AgdaSymbol{to}\AgdaSpace{}%
\AgdaOperator{\AgdaFunction{⟦\AgdaUnderscore{}⟧₂}}\AgdaSpace{}%
\AgdaSymbol{)}\<%
\\
%
\>[2]\AgdaKeyword{open}\AgdaSpace{}%
\AgdaModule{SetoidReasoning}\AgdaSpace{}%
\AgdaOperator{\AgdaFunction{𝔻[}}\AgdaSpace{}%
\AgdaBound{𝑩}\AgdaSpace{}%
\AgdaOperator{\AgdaFunction{]}}\<%
\\
%
\>[2]\AgdaKeyword{private}\AgdaSpace{}%
\AgdaFunction{f}\AgdaSpace{}%
\AgdaSymbol{=}\AgdaSpace{}%
\AgdaOperator{\AgdaField{\AgdaUnderscore{}⟨\$⟩\AgdaUnderscore{}}}\AgdaSpace{}%
\AgdaOperator{\AgdaFunction{∣}}\AgdaSpace{}%
\AgdaBound{fh}\AgdaSpace{}%
\AgdaOperator{\AgdaFunction{∣}}\AgdaSpace{}%
\AgdaSymbol{;}\AgdaSpace{}%
\AgdaFunction{g}\AgdaSpace{}%
\AgdaSymbol{=}\AgdaSpace{}%
\AgdaOperator{\AgdaField{\AgdaUnderscore{}⟨\$⟩\AgdaUnderscore{}}}\AgdaSpace{}%
\AgdaOperator{\AgdaFunction{∣}}\AgdaSpace{}%
\AgdaBound{gh}\AgdaSpace{}%
\AgdaOperator{\AgdaFunction{∣}}\<%
\end{code}

\hypertarget{subalgebraic-invariance-of}{%
\paragraph{Subalgebraic invariance of
⊧}\label{subalgebraic-invariance-of}}

Identities modeled by an algebra \texttt{𝑨} are also modeled by every
subalgebra of \texttt{𝑨}, which fact can be formalized as follows.

\begin{code}%
\>[0]\<%
\\
\>[0]\AgdaKeyword{module}\AgdaSpace{}%
\AgdaModule{\AgdaUnderscore{}}\AgdaSpace{}%
\AgdaSymbol{\{}\AgdaBound{X}\AgdaSpace{}%
\AgdaSymbol{:}\AgdaSpace{}%
\AgdaPrimitive{Type}\AgdaSpace{}%
\AgdaGeneralizable{χ}\AgdaSymbol{\}\{}\AgdaBound{𝑨}\AgdaSpace{}%
\AgdaSymbol{:}\AgdaSpace{}%
\AgdaRecord{Algebra}\AgdaSpace{}%
\AgdaGeneralizable{α}\AgdaSpace{}%
\AgdaGeneralizable{ρᵃ}\AgdaSymbol{\}\{}\AgdaBound{𝑩}\AgdaSpace{}%
\AgdaSymbol{:}\AgdaSpace{}%
\AgdaRecord{Algebra}\AgdaSpace{}%
\AgdaGeneralizable{β}\AgdaSpace{}%
\AgdaGeneralizable{ρᵇ}\AgdaSymbol{\}\{}\AgdaBound{p}\AgdaSpace{}%
\AgdaBound{q}\AgdaSpace{}%
\AgdaSymbol{:}\AgdaSpace{}%
\AgdaDatatype{Term}\AgdaSpace{}%
\AgdaBound{X}\AgdaSymbol{\}}\AgdaSpace{}%
\AgdaKeyword{where}\<%
\\
%
\\[\AgdaEmptyExtraSkip]%
\>[0][@{}l@{\AgdaIndent{0}}]%
\>[1]\AgdaFunction{⊧-S-invar}\AgdaSpace{}%
\AgdaSymbol{:}\AgdaSpace{}%
\AgdaBound{𝑨}\AgdaSpace{}%
\AgdaOperator{\AgdaFunction{⊧}}\AgdaSpace{}%
\AgdaSymbol{(}\AgdaBound{p}\AgdaSpace{}%
\AgdaOperator{\AgdaInductiveConstructor{≈̇}}\AgdaSpace{}%
\AgdaBound{q}\AgdaSymbol{)}\AgdaSpace{}%
\AgdaSymbol{→}%
\>[29]\AgdaBound{𝑩}\AgdaSpace{}%
\AgdaOperator{\AgdaFunction{≤}}\AgdaSpace{}%
\AgdaBound{𝑨}%
\>[36]\AgdaSymbol{→}%
\>[39]\AgdaBound{𝑩}\AgdaSpace{}%
\AgdaOperator{\AgdaFunction{⊧}}\AgdaSpace{}%
\AgdaSymbol{(}\AgdaBound{p}\AgdaSpace{}%
\AgdaOperator{\AgdaInductiveConstructor{≈̇}}\AgdaSpace{}%
\AgdaBound{q}\AgdaSymbol{)}\<%
\\
%
\>[1]\AgdaFunction{⊧-S-invar}\AgdaSpace{}%
\AgdaBound{Apq}\AgdaSpace{}%
\AgdaBound{B≤A}\AgdaSpace{}%
\AgdaBound{b}\AgdaSpace{}%
\AgdaSymbol{=}\AgdaSpace{}%
\AgdaFunction{goal}\<%
\\
\>[1][@{}l@{\AgdaIndent{0}}]%
\>[2]\AgdaKeyword{where}\<%
\\
%
\>[2]\AgdaKeyword{open}\AgdaSpace{}%
\AgdaModule{Setoid}\AgdaSpace{}%
\AgdaOperator{\AgdaFunction{𝔻[}}\AgdaSpace{}%
\AgdaBound{𝑨}\AgdaSpace{}%
\AgdaOperator{\AgdaFunction{]}}\AgdaSpace{}%
\AgdaKeyword{using}\AgdaSpace{}%
\AgdaSymbol{(}\AgdaSpace{}%
\AgdaOperator{\AgdaField{\AgdaUnderscore{}≈\AgdaUnderscore{}}}\AgdaSpace{}%
\AgdaSymbol{)}\<%
\\
%
\>[2]\AgdaKeyword{open}\AgdaSpace{}%
\AgdaModule{Environment}\AgdaSpace{}%
\AgdaBound{𝑨}\AgdaSpace{}%
\AgdaKeyword{using}\AgdaSpace{}%
\AgdaSymbol{()}\AgdaSpace{}%
\AgdaKeyword{renaming}\AgdaSpace{}%
\AgdaSymbol{(}\AgdaSpace{}%
\AgdaOperator{\AgdaFunction{⟦\AgdaUnderscore{}⟧}}\AgdaSpace{}%
\AgdaSymbol{to}\AgdaSpace{}%
\AgdaOperator{\AgdaFunction{⟦\AgdaUnderscore{}⟧ᵃ}}\AgdaSpace{}%
\AgdaSymbol{)}\<%
\\
%
\>[2]\AgdaKeyword{open}\AgdaSpace{}%
\AgdaModule{Setoid}\AgdaSpace{}%
\AgdaOperator{\AgdaFunction{𝔻[}}\AgdaSpace{}%
\AgdaBound{𝑩}\AgdaSpace{}%
\AgdaOperator{\AgdaFunction{]}}\AgdaSpace{}%
\AgdaKeyword{using}\AgdaSpace{}%
\AgdaSymbol{()}\AgdaSpace{}%
\AgdaKeyword{renaming}\AgdaSpace{}%
\AgdaSymbol{(}\AgdaSpace{}%
\AgdaOperator{\AgdaField{\AgdaUnderscore{}≈\AgdaUnderscore{}}}\AgdaSpace{}%
\AgdaSymbol{to}\AgdaSpace{}%
\AgdaOperator{\AgdaField{\AgdaUnderscore{}≈ᵇ\AgdaUnderscore{}}}\AgdaSpace{}%
\AgdaSymbol{)}\<%
\\
%
\>[2]\AgdaKeyword{open}\AgdaSpace{}%
\AgdaModule{Environment}\AgdaSpace{}%
\AgdaBound{𝑩}\AgdaSpace{}%
\AgdaKeyword{using}\AgdaSpace{}%
\AgdaSymbol{(}\AgdaSpace{}%
\AgdaOperator{\AgdaFunction{⟦\AgdaUnderscore{}⟧}}\AgdaSpace{}%
\AgdaSymbol{)}\<%
\\
%
\>[2]\AgdaKeyword{open}\AgdaSpace{}%
\AgdaModule{SetoidReasoning}\AgdaSpace{}%
\AgdaOperator{\AgdaFunction{𝔻[}}\AgdaSpace{}%
\AgdaBound{𝑨}\AgdaSpace{}%
\AgdaOperator{\AgdaFunction{]}}\<%
\\
%
\>[2]\AgdaFunction{hh}\AgdaSpace{}%
\AgdaSymbol{:}\AgdaSpace{}%
\AgdaFunction{hom}\AgdaSpace{}%
\AgdaBound{𝑩}\AgdaSpace{}%
\AgdaBound{𝑨}\<%
\\
%
\>[2]\AgdaFunction{hh}\AgdaSpace{}%
\AgdaSymbol{=}\AgdaSpace{}%
\AgdaOperator{\AgdaFunction{∣}}\AgdaSpace{}%
\AgdaBound{B≤A}\AgdaSpace{}%
\AgdaOperator{\AgdaFunction{∣}}\<%
\\
%
\>[2]\AgdaFunction{h}\AgdaSpace{}%
\AgdaSymbol{=}\AgdaSpace{}%
\AgdaOperator{\AgdaField{\AgdaUnderscore{}⟨\$⟩\AgdaUnderscore{}}}\AgdaSpace{}%
\AgdaOperator{\AgdaFunction{∣}}\AgdaSpace{}%
\AgdaFunction{hh}\AgdaSpace{}%
\AgdaOperator{\AgdaFunction{∣}}\<%
\\
%
\>[2]\AgdaFunction{ξ}\AgdaSpace{}%
\AgdaSymbol{:}\AgdaSpace{}%
\AgdaSymbol{∀}\AgdaSpace{}%
\AgdaBound{b}\AgdaSpace{}%
\AgdaSymbol{→}\AgdaSpace{}%
\AgdaFunction{h}\AgdaSpace{}%
\AgdaSymbol{(}\AgdaOperator{\AgdaFunction{⟦}}\AgdaSpace{}%
\AgdaBound{p}\AgdaSpace{}%
\AgdaOperator{\AgdaFunction{⟧}}\AgdaSpace{}%
\AgdaOperator{\AgdaField{⟨\$⟩}}\AgdaSpace{}%
\AgdaBound{b}\AgdaSymbol{)}\AgdaSpace{}%
\AgdaOperator{\AgdaFunction{≈}}\AgdaSpace{}%
\AgdaFunction{h}\AgdaSpace{}%
\AgdaSymbol{(}\AgdaOperator{\AgdaFunction{⟦}}\AgdaSpace{}%
\AgdaBound{q}\AgdaSpace{}%
\AgdaOperator{\AgdaFunction{⟧}}\AgdaSpace{}%
\AgdaOperator{\AgdaField{⟨\$⟩}}\AgdaSpace{}%
\AgdaBound{b}\AgdaSymbol{)}\<%
\\
%
\>[2]\AgdaFunction{ξ}\AgdaSpace{}%
\AgdaBound{b}\AgdaSpace{}%
\AgdaSymbol{=}%
\>[5444I]\AgdaOperator{\AgdaFunction{begin}}\<%
\\
\>[5444I][@{}l@{\AgdaIndent{0}}]%
\>[9]\AgdaFunction{h}\AgdaSpace{}%
\AgdaSymbol{(}\AgdaOperator{\AgdaFunction{⟦}}\AgdaSpace{}%
\AgdaBound{p}\AgdaSpace{}%
\AgdaOperator{\AgdaFunction{⟧}}\AgdaSpace{}%
\AgdaOperator{\AgdaField{⟨\$⟩}}\AgdaSpace{}%
\AgdaBound{b}\AgdaSymbol{)}%
\>[29]\AgdaFunction{≈⟨}\AgdaSpace{}%
\AgdaFunction{comm-hom-term}\AgdaSpace{}%
\AgdaFunction{hh}\AgdaSpace{}%
\AgdaBound{p}\AgdaSpace{}%
\AgdaBound{b}%
\>[55]\AgdaFunction{⟩}\<%
\\
%
\>[9]\AgdaOperator{\AgdaFunction{⟦}}\AgdaSpace{}%
\AgdaBound{p}\AgdaSpace{}%
\AgdaOperator{\AgdaFunction{⟧ᵃ}}\AgdaSpace{}%
\AgdaOperator{\AgdaField{⟨\$⟩}}\AgdaSpace{}%
\AgdaSymbol{(}\AgdaFunction{h}\AgdaSpace{}%
\AgdaOperator{\AgdaFunction{∘}}\AgdaSpace{}%
\AgdaBound{b}\AgdaSymbol{)}%
\>[29]\AgdaFunction{≈⟨}\AgdaSpace{}%
\AgdaBound{Apq}\AgdaSpace{}%
\AgdaSymbol{(}\AgdaFunction{h}\AgdaSpace{}%
\AgdaOperator{\AgdaFunction{∘}}\AgdaSpace{}%
\AgdaBound{b}\AgdaSymbol{)}%
\>[55]\AgdaFunction{⟩}\<%
\\
%
\>[9]\AgdaOperator{\AgdaFunction{⟦}}\AgdaSpace{}%
\AgdaBound{q}\AgdaSpace{}%
\AgdaOperator{\AgdaFunction{⟧ᵃ}}\AgdaSpace{}%
\AgdaOperator{\AgdaField{⟨\$⟩}}\AgdaSpace{}%
\AgdaSymbol{(}\AgdaFunction{h}\AgdaSpace{}%
\AgdaOperator{\AgdaFunction{∘}}\AgdaSpace{}%
\AgdaBound{b}\AgdaSymbol{)}%
\>[29]\AgdaFunction{≈˘⟨}\AgdaSpace{}%
\AgdaFunction{comm-hom-term}\AgdaSpace{}%
\AgdaFunction{hh}\AgdaSpace{}%
\AgdaBound{q}\AgdaSpace{}%
\AgdaBound{b}%
\>[55]\AgdaFunction{⟩}\<%
\\
%
\>[9]\AgdaFunction{h}\AgdaSpace{}%
\AgdaSymbol{(}\AgdaOperator{\AgdaFunction{⟦}}\AgdaSpace{}%
\AgdaBound{q}\AgdaSpace{}%
\AgdaOperator{\AgdaFunction{⟧}}\AgdaSpace{}%
\AgdaOperator{\AgdaField{⟨\$⟩}}\AgdaSpace{}%
\AgdaBound{b}\AgdaSymbol{)}%
\>[29]\AgdaOperator{\AgdaFunction{∎}}\<%
\\
%
\>[2]\AgdaFunction{goal}\AgdaSpace{}%
\AgdaSymbol{:}\AgdaSpace{}%
\AgdaOperator{\AgdaFunction{⟦}}\AgdaSpace{}%
\AgdaBound{p}\AgdaSpace{}%
\AgdaOperator{\AgdaFunction{⟧}}\AgdaSpace{}%
\AgdaOperator{\AgdaField{⟨\$⟩}}\AgdaSpace{}%
\AgdaBound{b}\AgdaSpace{}%
\AgdaOperator{\AgdaFunction{≈ᵇ}}\AgdaSpace{}%
\AgdaOperator{\AgdaFunction{⟦}}\AgdaSpace{}%
\AgdaBound{q}\AgdaSpace{}%
\AgdaOperator{\AgdaFunction{⟧}}\AgdaSpace{}%
\AgdaOperator{\AgdaField{⟨\$⟩}}\AgdaSpace{}%
\AgdaBound{b}\<%
\\
%
\>[2]\AgdaFunction{goal}\AgdaSpace{}%
\AgdaSymbol{=}\AgdaSpace{}%
\AgdaOperator{\AgdaFunction{∥}}\AgdaSpace{}%
\AgdaBound{B≤A}\AgdaSpace{}%
\AgdaOperator{\AgdaFunction{∥}}\AgdaSpace{}%
\AgdaSymbol{(}\AgdaFunction{ξ}\AgdaSpace{}%
\AgdaBound{b}\AgdaSymbol{)}\<%
\end{code}

\hypertarget{product-invariance-of}{%
\paragraph{Product invariance of ⊧}\label{product-invariance-of}}

An identity satisfied by all algebras in an indexed collection is also
satisfied by the product of algebras in that collection.

\begin{code}%
\>[0]\<%
\\
%
\\[\AgdaEmptyExtraSkip]%
\>[0]\AgdaKeyword{module}\AgdaSpace{}%
\AgdaModule{\AgdaUnderscore{}}\AgdaSpace{}%
\AgdaSymbol{\{}\AgdaBound{X}\AgdaSpace{}%
\AgdaSymbol{:}\AgdaSpace{}%
\AgdaPrimitive{Type}\AgdaSpace{}%
\AgdaGeneralizable{χ}\AgdaSymbol{\}\{}\AgdaBound{I}\AgdaSpace{}%
\AgdaSymbol{:}\AgdaSpace{}%
\AgdaPrimitive{Type}\AgdaSpace{}%
\AgdaGeneralizable{ℓ}\AgdaSymbol{\}(}\AgdaBound{𝒜}\AgdaSpace{}%
\AgdaSymbol{:}\AgdaSpace{}%
\AgdaBound{I}\AgdaSpace{}%
\AgdaSymbol{→}\AgdaSpace{}%
\AgdaRecord{Algebra}\AgdaSpace{}%
\AgdaGeneralizable{α}\AgdaSpace{}%
\AgdaGeneralizable{ρᵃ}\AgdaSymbol{)\{}\AgdaBound{p}\AgdaSpace{}%
\AgdaBound{q}\AgdaSpace{}%
\AgdaSymbol{:}\AgdaSpace{}%
\AgdaDatatype{Term}\AgdaSpace{}%
\AgdaBound{X}\AgdaSymbol{\}}\AgdaSpace{}%
\AgdaKeyword{where}\<%
\\
%
\\[\AgdaEmptyExtraSkip]%
\>[0][@{}l@{\AgdaIndent{0}}]%
\>[1]\AgdaFunction{⊧-P-invar}\AgdaSpace{}%
\AgdaSymbol{:}\AgdaSpace{}%
\AgdaSymbol{(∀}\AgdaSpace{}%
\AgdaBound{i}\AgdaSpace{}%
\AgdaSymbol{→}\AgdaSpace{}%
\AgdaBound{𝒜}\AgdaSpace{}%
\AgdaBound{i}\AgdaSpace{}%
\AgdaOperator{\AgdaFunction{⊧}}\AgdaSpace{}%
\AgdaSymbol{(}\AgdaBound{p}\AgdaSpace{}%
\AgdaOperator{\AgdaInductiveConstructor{≈̇}}\AgdaSpace{}%
\AgdaBound{q}\AgdaSymbol{))}\AgdaSpace{}%
\AgdaSymbol{→}\AgdaSpace{}%
\AgdaFunction{⨅}\AgdaSpace{}%
\AgdaBound{𝒜}\AgdaSpace{}%
\AgdaOperator{\AgdaFunction{⊧}}\AgdaSpace{}%
\AgdaSymbol{(}\AgdaBound{p}\AgdaSpace{}%
\AgdaOperator{\AgdaInductiveConstructor{≈̇}}\AgdaSpace{}%
\AgdaBound{q}\AgdaSymbol{)}\<%
\\
%
\>[1]\AgdaFunction{⊧-P-invar}\AgdaSpace{}%
\AgdaBound{𝒜pq}\AgdaSpace{}%
\AgdaBound{a}\AgdaSpace{}%
\AgdaSymbol{=}\AgdaSpace{}%
\AgdaFunction{goal}\<%
\\
\>[1][@{}l@{\AgdaIndent{0}}]%
\>[2]\AgdaKeyword{where}\<%
\\
%
\>[2]\AgdaKeyword{open}\AgdaSpace{}%
\AgdaModule{Environment}\AgdaSpace{}%
\AgdaSymbol{(}\AgdaFunction{⨅}\AgdaSpace{}%
\AgdaBound{𝒜}\AgdaSymbol{)}\AgdaSpace{}%
\AgdaKeyword{using}\AgdaSpace{}%
\AgdaSymbol{()}\AgdaSpace{}%
\AgdaKeyword{renaming}\AgdaSpace{}%
\AgdaSymbol{(}\AgdaSpace{}%
\AgdaOperator{\AgdaFunction{⟦\AgdaUnderscore{}⟧}}\AgdaSpace{}%
\AgdaSymbol{to}\AgdaSpace{}%
\AgdaOperator{\AgdaFunction{⟦\AgdaUnderscore{}⟧₁}}\AgdaSpace{}%
\AgdaSymbol{)}\<%
\\
%
\>[2]\AgdaKeyword{open}\AgdaSpace{}%
\AgdaModule{Environment}\AgdaSpace{}%
\AgdaKeyword{using}\AgdaSpace{}%
\AgdaSymbol{(}\AgdaSpace{}%
\AgdaOperator{\AgdaFunction{⟦\AgdaUnderscore{}⟧}}\AgdaSpace{}%
\AgdaSymbol{)}\<%
\\
%
\>[2]\AgdaKeyword{open}\AgdaSpace{}%
\AgdaModule{Setoid}\AgdaSpace{}%
\AgdaOperator{\AgdaFunction{𝔻[}}\AgdaSpace{}%
\AgdaFunction{⨅}\AgdaSpace{}%
\AgdaBound{𝒜}\AgdaSpace{}%
\AgdaOperator{\AgdaFunction{]}}\AgdaSpace{}%
\AgdaKeyword{using}\AgdaSpace{}%
\AgdaSymbol{(}\AgdaSpace{}%
\AgdaOperator{\AgdaField{\AgdaUnderscore{}≈\AgdaUnderscore{}}}\AgdaSpace{}%
\AgdaSymbol{)}\<%
\\
%
\>[2]\AgdaKeyword{open}\AgdaSpace{}%
\AgdaModule{SetoidReasoning}\AgdaSpace{}%
\AgdaOperator{\AgdaFunction{𝔻[}}\AgdaSpace{}%
\AgdaFunction{⨅}\AgdaSpace{}%
\AgdaBound{𝒜}\AgdaSpace{}%
\AgdaOperator{\AgdaFunction{]}}\<%
\\
%
\>[2]\AgdaFunction{ξ}\AgdaSpace{}%
\AgdaSymbol{:}\AgdaSpace{}%
\AgdaSymbol{(λ}\AgdaSpace{}%
\AgdaBound{i}\AgdaSpace{}%
\AgdaSymbol{→}\AgdaSpace{}%
\AgdaSymbol{(}\AgdaOperator{\AgdaFunction{⟦}}\AgdaSpace{}%
\AgdaBound{𝒜}\AgdaSpace{}%
\AgdaBound{i}\AgdaSpace{}%
\AgdaOperator{\AgdaFunction{⟧}}\AgdaSpace{}%
\AgdaBound{p}\AgdaSymbol{)}\AgdaSpace{}%
\AgdaOperator{\AgdaField{⟨\$⟩}}\AgdaSpace{}%
\AgdaSymbol{(λ}\AgdaSpace{}%
\AgdaBound{x}\AgdaSpace{}%
\AgdaSymbol{→}\AgdaSpace{}%
\AgdaSymbol{(}\AgdaBound{a}\AgdaSpace{}%
\AgdaBound{x}\AgdaSymbol{)}\AgdaSpace{}%
\AgdaBound{i}\AgdaSymbol{))}\AgdaSpace{}%
\AgdaOperator{\AgdaFunction{≈}}\AgdaSpace{}%
\AgdaSymbol{(λ}\AgdaSpace{}%
\AgdaBound{i}\AgdaSpace{}%
\AgdaSymbol{→}\AgdaSpace{}%
\AgdaSymbol{(}\AgdaOperator{\AgdaFunction{⟦}}\AgdaSpace{}%
\AgdaBound{𝒜}\AgdaSpace{}%
\AgdaBound{i}\AgdaSpace{}%
\AgdaOperator{\AgdaFunction{⟧}}\AgdaSpace{}%
\AgdaBound{q}\AgdaSymbol{)}\AgdaSpace{}%
\AgdaOperator{\AgdaField{⟨\$⟩}}\AgdaSpace{}%
\AgdaSymbol{(λ}\AgdaSpace{}%
\AgdaBound{x}\AgdaSpace{}%
\AgdaSymbol{→}\AgdaSpace{}%
\AgdaSymbol{(}\AgdaBound{a}\AgdaSpace{}%
\AgdaBound{x}\AgdaSymbol{)}\AgdaSpace{}%
\AgdaBound{i}\AgdaSymbol{))}\<%
\\
%
\>[2]\AgdaFunction{ξ}\AgdaSpace{}%
\AgdaSymbol{=}\AgdaSpace{}%
\AgdaSymbol{λ}\AgdaSpace{}%
\AgdaBound{i}\AgdaSpace{}%
\AgdaSymbol{→}\AgdaSpace{}%
\AgdaBound{𝒜pq}\AgdaSpace{}%
\AgdaBound{i}\AgdaSpace{}%
\AgdaSymbol{(λ}\AgdaSpace{}%
\AgdaBound{x}\AgdaSpace{}%
\AgdaSymbol{→}\AgdaSpace{}%
\AgdaSymbol{(}\AgdaBound{a}\AgdaSpace{}%
\AgdaBound{x}\AgdaSymbol{)}\AgdaSpace{}%
\AgdaBound{i}\AgdaSymbol{)}\<%
\\
%
\>[2]\AgdaFunction{goal}\AgdaSpace{}%
\AgdaSymbol{:}\AgdaSpace{}%
\AgdaOperator{\AgdaFunction{⟦}}\AgdaSpace{}%
\AgdaBound{p}\AgdaSpace{}%
\AgdaOperator{\AgdaFunction{⟧₁}}\AgdaSpace{}%
\AgdaOperator{\AgdaField{⟨\$⟩}}\AgdaSpace{}%
\AgdaBound{a}\AgdaSpace{}%
\AgdaOperator{\AgdaFunction{≈}}\AgdaSpace{}%
\AgdaOperator{\AgdaFunction{⟦}}\AgdaSpace{}%
\AgdaBound{q}\AgdaSpace{}%
\AgdaOperator{\AgdaFunction{⟧₁}}\AgdaSpace{}%
\AgdaOperator{\AgdaField{⟨\$⟩}}\AgdaSpace{}%
\AgdaBound{a}\<%
\\
%
\>[2]\AgdaFunction{goal}\AgdaSpace{}%
\AgdaSymbol{=}%
\>[5624I]\AgdaOperator{\AgdaFunction{begin}}\<%
\\
\>[5624I][@{}l@{\AgdaIndent{0}}]%
\>[10]\AgdaOperator{\AgdaFunction{⟦}}\AgdaSpace{}%
\AgdaBound{p}\AgdaSpace{}%
\AgdaOperator{\AgdaFunction{⟧₁}}\AgdaSpace{}%
\AgdaOperator{\AgdaField{⟨\$⟩}}\AgdaSpace{}%
\AgdaBound{a}%
\>[51]\AgdaFunction{≈⟨}\AgdaSpace{}%
\AgdaFunction{interp-prod}\AgdaSpace{}%
\AgdaBound{𝒜}\AgdaSpace{}%
\AgdaBound{p}\AgdaSpace{}%
\AgdaBound{a}%
\>[74]\AgdaFunction{⟩}\<%
\\
%
\>[10]\AgdaSymbol{(λ}\AgdaSpace{}%
\AgdaBound{i}\AgdaSpace{}%
\AgdaSymbol{→}\AgdaSpace{}%
\AgdaSymbol{(}\AgdaOperator{\AgdaFunction{⟦}}\AgdaSpace{}%
\AgdaBound{𝒜}\AgdaSpace{}%
\AgdaBound{i}\AgdaSpace{}%
\AgdaOperator{\AgdaFunction{⟧}}\AgdaSpace{}%
\AgdaBound{p}\AgdaSymbol{)}\AgdaSpace{}%
\AgdaOperator{\AgdaField{⟨\$⟩}}\AgdaSpace{}%
\AgdaSymbol{(λ}\AgdaSpace{}%
\AgdaBound{x}\AgdaSpace{}%
\AgdaSymbol{→}\AgdaSpace{}%
\AgdaSymbol{(}\AgdaBound{a}\AgdaSpace{}%
\AgdaBound{x}\AgdaSymbol{)}\AgdaSpace{}%
\AgdaBound{i}\AgdaSymbol{))}%
\>[51]\AgdaFunction{≈⟨}\AgdaSpace{}%
\AgdaFunction{ξ}%
\>[74]\AgdaFunction{⟩}\<%
\\
%
\>[10]\AgdaSymbol{(λ}\AgdaSpace{}%
\AgdaBound{i}\AgdaSpace{}%
\AgdaSymbol{→}\AgdaSpace{}%
\AgdaSymbol{(}\AgdaOperator{\AgdaFunction{⟦}}\AgdaSpace{}%
\AgdaBound{𝒜}\AgdaSpace{}%
\AgdaBound{i}\AgdaSpace{}%
\AgdaOperator{\AgdaFunction{⟧}}\AgdaSpace{}%
\AgdaBound{q}\AgdaSymbol{)}\AgdaSpace{}%
\AgdaOperator{\AgdaField{⟨\$⟩}}\AgdaSpace{}%
\AgdaSymbol{(λ}\AgdaSpace{}%
\AgdaBound{x}\AgdaSpace{}%
\AgdaSymbol{→}\AgdaSpace{}%
\AgdaSymbol{(}\AgdaBound{a}\AgdaSpace{}%
\AgdaBound{x}\AgdaSymbol{)}\AgdaSpace{}%
\AgdaBound{i}\AgdaSymbol{))}%
\>[51]\AgdaFunction{≈˘⟨}\AgdaSpace{}%
\AgdaFunction{interp-prod}\AgdaSpace{}%
\AgdaBound{𝒜}\AgdaSpace{}%
\AgdaBound{q}\AgdaSpace{}%
\AgdaBound{a}%
\>[74]\AgdaFunction{⟩}\<%
\\
%
\>[10]\AgdaOperator{\AgdaFunction{⟦}}\AgdaSpace{}%
\AgdaBound{q}\AgdaSpace{}%
\AgdaOperator{\AgdaFunction{⟧₁}}\AgdaSpace{}%
\AgdaOperator{\AgdaField{⟨\$⟩}}\AgdaSpace{}%
\AgdaBound{a}%
\>[51]\AgdaOperator{\AgdaFunction{∎}}\<%
\end{code}

\hypertarget{ps-sp}{%
\paragraph{PS ⊆ SP}\label{ps-sp}}

Another important fact we will need about the operators \texttt{S} and
\texttt{P} is that a product of subalgebras of algebras in a class
\texttt{𝒦} is a subalgebra of a product of algebras in \texttt{𝒦}. We
denote this inclusion by \texttt{PS⊆SP}, which we state and prove as
follows.

\begin{code}%
\>[0]\<%
\\
\>[0]\AgdaKeyword{module}\AgdaSpace{}%
\AgdaModule{\AgdaUnderscore{}}%
\>[10]\AgdaSymbol{\{}\AgdaBound{𝒦}\AgdaSpace{}%
\AgdaSymbol{:}\AgdaSpace{}%
\AgdaFunction{Pred}\AgdaSymbol{(}\AgdaRecord{Algebra}\AgdaSpace{}%
\AgdaGeneralizable{α}\AgdaSpace{}%
\AgdaGeneralizable{ρᵃ}\AgdaSymbol{)}\AgdaSpace{}%
\AgdaSymbol{(}\AgdaGeneralizable{α}\AgdaSpace{}%
\AgdaOperator{\AgdaPrimitive{⊔}}\AgdaSpace{}%
\AgdaGeneralizable{ρᵃ}\AgdaSpace{}%
\AgdaOperator{\AgdaPrimitive{⊔}}\AgdaSpace{}%
\AgdaFunction{ov}\AgdaSpace{}%
\AgdaGeneralizable{ℓ}\AgdaSymbol{)\}}\AgdaSpace{}%
\AgdaKeyword{where}\<%
\\
\>[0][@{}l@{\AgdaIndent{0}}]%
\>[1]\AgdaKeyword{private}\<%
\\
\>[1][@{}l@{\AgdaIndent{0}}]%
\>[2]\AgdaFunction{a}\AgdaSpace{}%
\AgdaSymbol{=}\AgdaSpace{}%
\AgdaBound{α}\AgdaSpace{}%
\AgdaOperator{\AgdaPrimitive{⊔}}\AgdaSpace{}%
\AgdaBound{ρᵃ}\<%
\\
%
\>[2]\AgdaFunction{oaℓ}\AgdaSpace{}%
\AgdaSymbol{=}\AgdaSpace{}%
\AgdaFunction{ov}\AgdaSpace{}%
\AgdaSymbol{(}\AgdaFunction{a}\AgdaSpace{}%
\AgdaOperator{\AgdaPrimitive{⊔}}\AgdaSpace{}%
\AgdaBound{ℓ}\AgdaSymbol{)}\<%
\\
%
\\[\AgdaEmptyExtraSkip]%
%
\>[1]\AgdaFunction{PS⊆SP}\AgdaSpace{}%
\AgdaSymbol{:}\AgdaSpace{}%
\AgdaFunction{P}\AgdaSpace{}%
\AgdaSymbol{(}\AgdaFunction{a}\AgdaSpace{}%
\AgdaOperator{\AgdaPrimitive{⊔}}\AgdaSpace{}%
\AgdaBound{ℓ}\AgdaSymbol{)}\AgdaSpace{}%
\AgdaFunction{oaℓ}\AgdaSpace{}%
\AgdaSymbol{(}\AgdaFunction{S}\AgdaSymbol{\{}\AgdaArgument{β}\AgdaSpace{}%
\AgdaSymbol{=}\AgdaSpace{}%
\AgdaBound{α}\AgdaSymbol{\}\{}\AgdaBound{ρᵃ}\AgdaSymbol{\}}\AgdaSpace{}%
\AgdaBound{ℓ}\AgdaSpace{}%
\AgdaBound{𝒦}\AgdaSymbol{)}\AgdaSpace{}%
\AgdaOperator{\AgdaFunction{⊆}}\AgdaSpace{}%
\AgdaFunction{S}\AgdaSpace{}%
\AgdaFunction{oaℓ}\AgdaSpace{}%
\AgdaSymbol{(}\AgdaFunction{P}\AgdaSpace{}%
\AgdaBound{ℓ}\AgdaSpace{}%
\AgdaFunction{oaℓ}\AgdaSpace{}%
\AgdaBound{𝒦}\AgdaSymbol{)}\<%
\\
%
\>[1]\AgdaFunction{PS⊆SP}\AgdaSpace{}%
\AgdaSymbol{\{}\AgdaBound{𝑩}\AgdaSymbol{\}}\AgdaSpace{}%
\AgdaSymbol{(}\AgdaBound{I}\AgdaSpace{}%
\AgdaOperator{\AgdaInductiveConstructor{,}}\AgdaSpace{}%
\AgdaSymbol{(}\AgdaSpace{}%
\AgdaBound{𝒜}\AgdaSpace{}%
\AgdaOperator{\AgdaInductiveConstructor{,}}\AgdaSpace{}%
\AgdaBound{sA}\AgdaSpace{}%
\AgdaOperator{\AgdaInductiveConstructor{,}}\AgdaSpace{}%
\AgdaBound{B≅⨅A}\AgdaSpace{}%
\AgdaSymbol{))}\AgdaSpace{}%
\AgdaSymbol{=}\AgdaSpace{}%
\AgdaFunction{Goal}\<%
\\
\>[1][@{}l@{\AgdaIndent{0}}]%
\>[2]\AgdaKeyword{where}\<%
\\
%
\>[2]\AgdaFunction{ℬ}\AgdaSpace{}%
\AgdaSymbol{:}\AgdaSpace{}%
\AgdaBound{I}\AgdaSpace{}%
\AgdaSymbol{→}\AgdaSpace{}%
\AgdaRecord{Algebra}\AgdaSpace{}%
\AgdaBound{α}\AgdaSpace{}%
\AgdaBound{ρᵃ}\<%
\\
%
\>[2]\AgdaFunction{ℬ}\AgdaSpace{}%
\AgdaBound{i}\AgdaSpace{}%
\AgdaSymbol{=}\AgdaSpace{}%
\AgdaOperator{\AgdaFunction{∣}}\AgdaSpace{}%
\AgdaBound{sA}\AgdaSpace{}%
\AgdaBound{i}\AgdaSpace{}%
\AgdaOperator{\AgdaFunction{∣}}\<%
\\
%
\>[2]\AgdaFunction{kB}\AgdaSpace{}%
\AgdaSymbol{:}\AgdaSpace{}%
\AgdaSymbol{(}\AgdaBound{i}\AgdaSpace{}%
\AgdaSymbol{:}\AgdaSpace{}%
\AgdaBound{I}\AgdaSymbol{)}\AgdaSpace{}%
\AgdaSymbol{→}\AgdaSpace{}%
\AgdaFunction{ℬ}\AgdaSpace{}%
\AgdaBound{i}\AgdaSpace{}%
\AgdaOperator{\AgdaFunction{∈}}\AgdaSpace{}%
\AgdaBound{𝒦}\<%
\\
%
\>[2]\AgdaFunction{kB}\AgdaSpace{}%
\AgdaBound{i}\AgdaSpace{}%
\AgdaSymbol{=}%
\>[10]\AgdaField{fst}\AgdaSpace{}%
\AgdaOperator{\AgdaFunction{∥}}\AgdaSpace{}%
\AgdaBound{sA}\AgdaSpace{}%
\AgdaBound{i}\AgdaSpace{}%
\AgdaOperator{\AgdaFunction{∥}}\<%
\\
%
\>[2]\AgdaFunction{⨅A≤⨅B}\AgdaSpace{}%
\AgdaSymbol{:}\AgdaSpace{}%
\AgdaFunction{⨅}\AgdaSpace{}%
\AgdaBound{𝒜}\AgdaSpace{}%
\AgdaOperator{\AgdaFunction{≤}}\AgdaSpace{}%
\AgdaFunction{⨅}\AgdaSpace{}%
\AgdaFunction{ℬ}\<%
\\
%
\>[2]\AgdaFunction{⨅A≤⨅B}\AgdaSpace{}%
\AgdaSymbol{=}\AgdaSpace{}%
\AgdaFunction{⨅-≤}\AgdaSpace{}%
\AgdaSymbol{λ}\AgdaSpace{}%
\AgdaBound{i}\AgdaSpace{}%
\AgdaSymbol{→}\AgdaSpace{}%
\AgdaField{snd}\AgdaSpace{}%
\AgdaOperator{\AgdaFunction{∥}}\AgdaSpace{}%
\AgdaBound{sA}\AgdaSpace{}%
\AgdaBound{i}\AgdaSpace{}%
\AgdaOperator{\AgdaFunction{∥}}\<%
\\
%
\>[2]\AgdaFunction{Goal}\AgdaSpace{}%
\AgdaSymbol{:}\AgdaSpace{}%
\AgdaBound{𝑩}\AgdaSpace{}%
\AgdaOperator{\AgdaFunction{∈}}\AgdaSpace{}%
\AgdaFunction{S}\AgdaSymbol{\{}\AgdaArgument{β}\AgdaSpace{}%
\AgdaSymbol{=}\AgdaSpace{}%
\AgdaFunction{oaℓ}\AgdaSymbol{\}\{}\AgdaFunction{oaℓ}\AgdaSymbol{\}}\AgdaFunction{oaℓ}\AgdaSpace{}%
\AgdaSymbol{(}\AgdaFunction{P}\AgdaSpace{}%
\AgdaSymbol{\{}\AgdaArgument{β}\AgdaSpace{}%
\AgdaSymbol{=}\AgdaSpace{}%
\AgdaFunction{oaℓ}\AgdaSymbol{\}\{}\AgdaFunction{oaℓ}\AgdaSymbol{\}}\AgdaSpace{}%
\AgdaBound{ℓ}\AgdaSpace{}%
\AgdaFunction{oaℓ}\AgdaSpace{}%
\AgdaBound{𝒦}\AgdaSymbol{)}\<%
\\
%
\>[2]\AgdaFunction{Goal}\AgdaSpace{}%
\AgdaSymbol{=}\AgdaSpace{}%
\AgdaFunction{⨅}\AgdaSpace{}%
\AgdaFunction{ℬ}\AgdaSpace{}%
\AgdaOperator{\AgdaInductiveConstructor{,}}\AgdaSpace{}%
\AgdaSymbol{(}\AgdaBound{I}\AgdaSpace{}%
\AgdaOperator{\AgdaInductiveConstructor{,}}\AgdaSpace{}%
\AgdaSymbol{(}\AgdaFunction{ℬ}\AgdaSpace{}%
\AgdaOperator{\AgdaInductiveConstructor{,}}\AgdaSpace{}%
\AgdaSymbol{(}\AgdaFunction{kB}\AgdaSpace{}%
\AgdaOperator{\AgdaInductiveConstructor{,}}\AgdaSpace{}%
\AgdaFunction{≅-refl}\AgdaSymbol{)))}\AgdaSpace{}%
\AgdaOperator{\AgdaInductiveConstructor{,}}\AgdaSpace{}%
\AgdaSymbol{(}\AgdaFunction{≅-trans-≤}\AgdaSpace{}%
\AgdaBound{B≅⨅A}\AgdaSpace{}%
\AgdaFunction{⨅A≤⨅B}\AgdaSymbol{)}\<%
\end{code}

\hypertarget{identity-preservation}{%
\paragraph{Identity preservation}\label{identity-preservation}}

The classes \texttt{H\ 𝒦}, \texttt{S\ 𝒦}, \texttt{P\ 𝒦}, and
\texttt{V\ 𝒦} all satisfy the same set of equations. We will only use a
subset of the inclusions used to prove this fact. (For a complete proof,
see the
\href{https://ualib.github.io/agda-algebras/Varieties.Func.Preservation.html}{Varieties.Func.Preservation}
module of the \href{https://ualib.github.io/agda-algebras}{Agda
Universal Algebra Library}.)

\hypertarget{h-preserves-identities}{%
\subparagraph{H preserves identities}\label{h-preserves-identities}}

First we prove that the closure operator H is compatible with identities
that hold in the given class.

\begin{code}%
\>[0]\<%
\\
\>[0]\AgdaKeyword{module}\AgdaSpace{}%
\AgdaModule{\AgdaUnderscore{}}%
\>[10]\AgdaSymbol{\{}\AgdaBound{X}\AgdaSpace{}%
\AgdaSymbol{:}\AgdaSpace{}%
\AgdaPrimitive{Type}\AgdaSpace{}%
\AgdaGeneralizable{χ}\AgdaSymbol{\}\{}\AgdaBound{𝒦}\AgdaSpace{}%
\AgdaSymbol{:}\AgdaSpace{}%
\AgdaFunction{Pred}\AgdaSymbol{(}\AgdaRecord{Algebra}\AgdaSpace{}%
\AgdaGeneralizable{α}\AgdaSpace{}%
\AgdaGeneralizable{ρᵃ}\AgdaSymbol{)}\AgdaSpace{}%
\AgdaSymbol{(}\AgdaGeneralizable{α}\AgdaSpace{}%
\AgdaOperator{\AgdaPrimitive{⊔}}\AgdaSpace{}%
\AgdaGeneralizable{ρᵃ}\AgdaSpace{}%
\AgdaOperator{\AgdaPrimitive{⊔}}\AgdaSpace{}%
\AgdaFunction{ov}\AgdaSpace{}%
\AgdaGeneralizable{ℓ}\AgdaSymbol{)\}\{}\AgdaBound{p}\AgdaSpace{}%
\AgdaBound{q}\AgdaSpace{}%
\AgdaSymbol{:}\AgdaSpace{}%
\AgdaDatatype{Term}\AgdaSpace{}%
\AgdaBound{X}\AgdaSymbol{\}}\AgdaSpace{}%
\AgdaKeyword{where}\<%
\\
%
\\[\AgdaEmptyExtraSkip]%
\>[0][@{}l@{\AgdaIndent{0}}]%
\>[1]\AgdaFunction{H-id1}\AgdaSpace{}%
\AgdaSymbol{:}\AgdaSpace{}%
\AgdaBound{𝒦}\AgdaSpace{}%
\AgdaOperator{\AgdaFunction{⊫}}\AgdaSpace{}%
\AgdaSymbol{(}\AgdaBound{p}\AgdaSpace{}%
\AgdaOperator{\AgdaInductiveConstructor{≈̇}}\AgdaSpace{}%
\AgdaBound{q}\AgdaSymbol{)}\AgdaSpace{}%
\AgdaSymbol{→}\AgdaSpace{}%
\AgdaSymbol{(}\AgdaFunction{H}\AgdaSpace{}%
\AgdaSymbol{\{}\AgdaArgument{β}\AgdaSpace{}%
\AgdaSymbol{=}\AgdaSpace{}%
\AgdaBound{α}\AgdaSymbol{\}\{}\AgdaBound{ρᵃ}\AgdaSymbol{\}}\AgdaBound{ℓ}\AgdaSpace{}%
\AgdaBound{𝒦}\AgdaSymbol{)}\AgdaSpace{}%
\AgdaOperator{\AgdaFunction{⊫}}\AgdaSpace{}%
\AgdaSymbol{(}\AgdaBound{p}\AgdaSpace{}%
\AgdaOperator{\AgdaInductiveConstructor{≈̇}}\AgdaSpace{}%
\AgdaBound{q}\AgdaSymbol{)}\<%
\\
%
\>[1]\AgdaFunction{H-id1}\AgdaSpace{}%
\AgdaBound{σ}\AgdaSpace{}%
\AgdaBound{𝑩}\AgdaSpace{}%
\AgdaSymbol{(}\AgdaBound{𝑨}\AgdaSpace{}%
\AgdaOperator{\AgdaInductiveConstructor{,}}\AgdaSpace{}%
\AgdaBound{kA}\AgdaSpace{}%
\AgdaOperator{\AgdaInductiveConstructor{,}}\AgdaSpace{}%
\AgdaBound{BimgOfA}\AgdaSymbol{)}\AgdaSpace{}%
\AgdaBound{ρ}\AgdaSpace{}%
\AgdaSymbol{=}\AgdaSpace{}%
\AgdaFunction{B⊧pq}\<%
\\
\>[1][@{}l@{\AgdaIndent{0}}]%
\>[2]\AgdaKeyword{where}\<%
\\
%
\>[2]\AgdaFunction{IH}\AgdaSpace{}%
\AgdaSymbol{:}\AgdaSpace{}%
\AgdaBound{𝑨}\AgdaSpace{}%
\AgdaOperator{\AgdaFunction{⊧}}\AgdaSpace{}%
\AgdaSymbol{(}\AgdaBound{p}\AgdaSpace{}%
\AgdaOperator{\AgdaInductiveConstructor{≈̇}}\AgdaSpace{}%
\AgdaBound{q}\AgdaSymbol{)}\<%
\\
%
\>[2]\AgdaFunction{IH}\AgdaSpace{}%
\AgdaSymbol{=}\AgdaSpace{}%
\AgdaBound{σ}\AgdaSpace{}%
\AgdaBound{𝑨}\AgdaSpace{}%
\AgdaBound{kA}\<%
\\
%
\>[2]\AgdaKeyword{open}\AgdaSpace{}%
\AgdaModule{Environment}\AgdaSpace{}%
\AgdaBound{𝑨}\AgdaSpace{}%
\AgdaKeyword{using}\AgdaSpace{}%
\AgdaSymbol{()}\AgdaSpace{}%
\AgdaKeyword{renaming}\AgdaSpace{}%
\AgdaSymbol{(}\AgdaSpace{}%
\AgdaOperator{\AgdaFunction{⟦\AgdaUnderscore{}⟧}}\AgdaSpace{}%
\AgdaSymbol{to}\AgdaSpace{}%
\AgdaOperator{\AgdaFunction{⟦\AgdaUnderscore{}⟧₁}}\AgdaSymbol{)}\<%
\\
%
\>[2]\AgdaKeyword{open}\AgdaSpace{}%
\AgdaModule{Environment}\AgdaSpace{}%
\AgdaBound{𝑩}\AgdaSpace{}%
\AgdaKeyword{using}\AgdaSpace{}%
\AgdaSymbol{(}\AgdaSpace{}%
\AgdaOperator{\AgdaFunction{⟦\AgdaUnderscore{}⟧}}\AgdaSpace{}%
\AgdaSymbol{)}\<%
\\
%
\>[2]\AgdaKeyword{open}\AgdaSpace{}%
\AgdaModule{Setoid}\AgdaSpace{}%
\AgdaOperator{\AgdaFunction{𝔻[}}\AgdaSpace{}%
\AgdaBound{𝑩}\AgdaSpace{}%
\AgdaOperator{\AgdaFunction{]}}\AgdaSpace{}%
\AgdaKeyword{using}\AgdaSpace{}%
\AgdaSymbol{(}\AgdaSpace{}%
\AgdaOperator{\AgdaField{\AgdaUnderscore{}≈\AgdaUnderscore{}}}\AgdaSpace{}%
\AgdaSymbol{)}\<%
\\
%
\>[2]\AgdaKeyword{open}\AgdaSpace{}%
\AgdaModule{SetoidReasoning}\AgdaSpace{}%
\AgdaOperator{\AgdaFunction{𝔻[}}\AgdaSpace{}%
\AgdaBound{𝑩}\AgdaSpace{}%
\AgdaOperator{\AgdaFunction{]}}\<%
\\
%
\\[\AgdaEmptyExtraSkip]%
%
\>[2]\AgdaFunction{φ}\AgdaSpace{}%
\AgdaSymbol{:}\AgdaSpace{}%
\AgdaFunction{hom}\AgdaSpace{}%
\AgdaBound{𝑨}\AgdaSpace{}%
\AgdaBound{𝑩}\<%
\\
%
\>[2]\AgdaFunction{φ}\AgdaSpace{}%
\AgdaSymbol{=}\AgdaSpace{}%
\AgdaOperator{\AgdaFunction{∣}}\AgdaSpace{}%
\AgdaBound{BimgOfA}\AgdaSpace{}%
\AgdaOperator{\AgdaFunction{∣}}\<%
\\
%
\>[2]\AgdaFunction{φE}\AgdaSpace{}%
\AgdaSymbol{:}\AgdaSpace{}%
\AgdaFunction{IsSurjective}\AgdaSpace{}%
\AgdaOperator{\AgdaFunction{∣}}\AgdaSpace{}%
\AgdaFunction{φ}\AgdaSpace{}%
\AgdaOperator{\AgdaFunction{∣}}\<%
\\
%
\>[2]\AgdaFunction{φE}\AgdaSpace{}%
\AgdaSymbol{=}\AgdaSpace{}%
\AgdaOperator{\AgdaFunction{∥}}\AgdaSpace{}%
\AgdaBound{BimgOfA}\AgdaSpace{}%
\AgdaOperator{\AgdaFunction{∥}}\<%
\\
%
\>[2]\AgdaFunction{φ⁻¹}\AgdaSpace{}%
\AgdaSymbol{:}\AgdaSpace{}%
\AgdaOperator{\AgdaFunction{𝕌[}}\AgdaSpace{}%
\AgdaBound{𝑩}\AgdaSpace{}%
\AgdaOperator{\AgdaFunction{]}}\AgdaSpace{}%
\AgdaSymbol{→}\AgdaSpace{}%
\AgdaOperator{\AgdaFunction{𝕌[}}\AgdaSpace{}%
\AgdaBound{𝑨}\AgdaSpace{}%
\AgdaOperator{\AgdaFunction{]}}\<%
\\
%
\>[2]\AgdaFunction{φ⁻¹}\AgdaSpace{}%
\AgdaSymbol{=}\AgdaSpace{}%
\AgdaFunction{SurjInv}\AgdaSpace{}%
\AgdaOperator{\AgdaFunction{∣}}\AgdaSpace{}%
\AgdaFunction{φ}\AgdaSpace{}%
\AgdaOperator{\AgdaFunction{∣}}\AgdaSpace{}%
\AgdaFunction{φE}\<%
\\
%
\\[\AgdaEmptyExtraSkip]%
%
\>[2]\AgdaFunction{ζ}\AgdaSpace{}%
\AgdaSymbol{:}\AgdaSpace{}%
\AgdaSymbol{∀}\AgdaSpace{}%
\AgdaBound{x}\AgdaSpace{}%
\AgdaSymbol{→}\AgdaSpace{}%
\AgdaSymbol{(}\AgdaOperator{\AgdaFunction{∣}}\AgdaSpace{}%
\AgdaFunction{φ}\AgdaSpace{}%
\AgdaOperator{\AgdaFunction{∣}}\AgdaSpace{}%
\AgdaOperator{\AgdaField{⟨\$⟩}}\AgdaSpace{}%
\AgdaSymbol{(}\AgdaFunction{φ⁻¹}\AgdaSpace{}%
\AgdaOperator{\AgdaFunction{∘}}\AgdaSpace{}%
\AgdaBound{ρ}\AgdaSymbol{)}\AgdaSpace{}%
\AgdaBound{x}\AgdaSpace{}%
\AgdaSymbol{)}\AgdaSpace{}%
\AgdaOperator{\AgdaFunction{≈}}\AgdaSpace{}%
\AgdaBound{ρ}\AgdaSpace{}%
\AgdaBound{x}\<%
\\
%
\>[2]\AgdaFunction{ζ}\AgdaSpace{}%
\AgdaSymbol{=}\AgdaSpace{}%
\AgdaSymbol{λ}\AgdaSpace{}%
\AgdaBound{\AgdaUnderscore{}}\AgdaSpace{}%
\AgdaSymbol{→}\AgdaSpace{}%
\AgdaFunction{InvIsInverseʳ}\AgdaSpace{}%
\AgdaFunction{φE}\<%
\\
%
\\[\AgdaEmptyExtraSkip]%
%
\>[2]\AgdaFunction{B⊧pq}\AgdaSpace{}%
\AgdaSymbol{:}\AgdaSpace{}%
\AgdaSymbol{(}\AgdaOperator{\AgdaFunction{⟦}}\AgdaSpace{}%
\AgdaBound{p}\AgdaSpace{}%
\AgdaOperator{\AgdaFunction{⟧}}\AgdaSpace{}%
\AgdaOperator{\AgdaField{⟨\$⟩}}\AgdaSpace{}%
\AgdaBound{ρ}\AgdaSymbol{)}\AgdaSpace{}%
\AgdaOperator{\AgdaFunction{≈}}\AgdaSpace{}%
\AgdaSymbol{(}\AgdaOperator{\AgdaFunction{⟦}}\AgdaSpace{}%
\AgdaBound{q}\AgdaSpace{}%
\AgdaOperator{\AgdaFunction{⟧}}\AgdaSpace{}%
\AgdaOperator{\AgdaField{⟨\$⟩}}\AgdaSpace{}%
\AgdaBound{ρ}\AgdaSymbol{)}\<%
\\
%
\>[2]\AgdaFunction{B⊧pq}\AgdaSpace{}%
\AgdaSymbol{=}%
\>[5940I]\AgdaOperator{\AgdaFunction{begin}}\<%
\\
\>[.][@{}l@{}]\<[5940I]%
\>[9]\AgdaOperator{\AgdaFunction{⟦}}\AgdaSpace{}%
\AgdaBound{p}\AgdaSpace{}%
\AgdaOperator{\AgdaFunction{⟧}}\AgdaSpace{}%
\AgdaOperator{\AgdaField{⟨\$⟩}}\AgdaSpace{}%
\AgdaBound{ρ}%
\>[52]\AgdaFunction{≈˘⟨}\AgdaSpace{}%
\AgdaField{cong}\AgdaSpace{}%
\AgdaOperator{\AgdaFunction{⟦}}\AgdaSpace{}%
\AgdaBound{p}\AgdaSpace{}%
\AgdaOperator{\AgdaFunction{⟧}}\AgdaSpace{}%
\AgdaFunction{ζ}%
\>[85]\AgdaFunction{⟩}\<%
\\
%
\>[9]\AgdaOperator{\AgdaFunction{⟦}}\AgdaSpace{}%
\AgdaBound{p}\AgdaSpace{}%
\AgdaOperator{\AgdaFunction{⟧}}\AgdaSpace{}%
\AgdaOperator{\AgdaField{⟨\$⟩}}\AgdaSpace{}%
\AgdaSymbol{(λ}\AgdaSpace{}%
\AgdaBound{x}\AgdaSpace{}%
\AgdaSymbol{→}\AgdaSpace{}%
\AgdaSymbol{(}\AgdaOperator{\AgdaFunction{∣}}\AgdaSpace{}%
\AgdaFunction{φ}\AgdaSpace{}%
\AgdaOperator{\AgdaFunction{∣}}\AgdaSpace{}%
\AgdaOperator{\AgdaField{⟨\$⟩}}\AgdaSpace{}%
\AgdaSymbol{(}\AgdaFunction{φ⁻¹}\AgdaSpace{}%
\AgdaOperator{\AgdaFunction{∘}}\AgdaSpace{}%
\AgdaBound{ρ}\AgdaSymbol{)}\AgdaSpace{}%
\AgdaBound{x}\AgdaSymbol{))}%
\>[52]\AgdaFunction{≈˘⟨}\AgdaSpace{}%
\AgdaFunction{comm-hom-term}\AgdaSpace{}%
\AgdaFunction{φ}\AgdaSpace{}%
\AgdaBound{p}\AgdaSpace{}%
\AgdaSymbol{(}\AgdaFunction{φ⁻¹}\AgdaSpace{}%
\AgdaOperator{\AgdaFunction{∘}}\AgdaSpace{}%
\AgdaBound{ρ}\AgdaSymbol{)}%
\>[85]\AgdaFunction{⟩}\<%
\\
%
\>[9]\AgdaOperator{\AgdaFunction{∣}}\AgdaSpace{}%
\AgdaFunction{φ}\AgdaSpace{}%
\AgdaOperator{\AgdaFunction{∣}}\AgdaSpace{}%
\AgdaOperator{\AgdaField{⟨\$⟩}}\AgdaSpace{}%
\AgdaSymbol{(}\AgdaOperator{\AgdaFunction{⟦}}\AgdaSpace{}%
\AgdaBound{p}\AgdaSpace{}%
\AgdaOperator{\AgdaFunction{⟧₁}}\AgdaSpace{}%
\AgdaOperator{\AgdaField{⟨\$⟩}}\AgdaSpace{}%
\AgdaSymbol{(}\AgdaFunction{φ⁻¹}\AgdaSpace{}%
\AgdaOperator{\AgdaFunction{∘}}\AgdaSpace{}%
\AgdaBound{ρ}\AgdaSymbol{))}%
\>[52]\AgdaFunction{≈⟨}\AgdaSpace{}%
\AgdaField{cong}\AgdaSpace{}%
\AgdaOperator{\AgdaFunction{∣}}\AgdaSpace{}%
\AgdaFunction{φ}\AgdaSpace{}%
\AgdaOperator{\AgdaFunction{∣}}\AgdaSpace{}%
\AgdaSymbol{(}\AgdaFunction{IH}\AgdaSpace{}%
\AgdaSymbol{(}\AgdaFunction{φ⁻¹}\AgdaSpace{}%
\AgdaOperator{\AgdaFunction{∘}}\AgdaSpace{}%
\AgdaBound{ρ}\AgdaSymbol{))}%
\>[85]\AgdaFunction{⟩}\<%
\\
%
\>[9]\AgdaOperator{\AgdaFunction{∣}}\AgdaSpace{}%
\AgdaFunction{φ}\AgdaSpace{}%
\AgdaOperator{\AgdaFunction{∣}}\AgdaSpace{}%
\AgdaOperator{\AgdaField{⟨\$⟩}}\AgdaSpace{}%
\AgdaSymbol{(}\AgdaOperator{\AgdaFunction{⟦}}\AgdaSpace{}%
\AgdaBound{q}\AgdaSpace{}%
\AgdaOperator{\AgdaFunction{⟧₁}}\AgdaSpace{}%
\AgdaOperator{\AgdaField{⟨\$⟩}}\AgdaSpace{}%
\AgdaSymbol{(}\AgdaFunction{φ⁻¹}\AgdaSpace{}%
\AgdaOperator{\AgdaFunction{∘}}\AgdaSpace{}%
\AgdaBound{ρ}\AgdaSymbol{))}%
\>[52]\AgdaFunction{≈⟨}\AgdaSpace{}%
\AgdaFunction{comm-hom-term}\AgdaSpace{}%
\AgdaFunction{φ}\AgdaSpace{}%
\AgdaBound{q}\AgdaSpace{}%
\AgdaSymbol{(}\AgdaFunction{φ⁻¹}\AgdaSpace{}%
\AgdaOperator{\AgdaFunction{∘}}\AgdaSpace{}%
\AgdaBound{ρ}\AgdaSymbol{)}%
\>[85]\AgdaFunction{⟩}\<%
\\
%
\>[9]\AgdaOperator{\AgdaFunction{⟦}}\AgdaSpace{}%
\AgdaBound{q}\AgdaSpace{}%
\AgdaOperator{\AgdaFunction{⟧}}\AgdaSpace{}%
\AgdaOperator{\AgdaField{⟨\$⟩}}\AgdaSpace{}%
\AgdaSymbol{(λ}\AgdaSpace{}%
\AgdaBound{x}\AgdaSpace{}%
\AgdaSymbol{→}\AgdaSpace{}%
\AgdaSymbol{(}\AgdaOperator{\AgdaFunction{∣}}\AgdaSpace{}%
\AgdaFunction{φ}\AgdaSpace{}%
\AgdaOperator{\AgdaFunction{∣}}\AgdaSpace{}%
\AgdaOperator{\AgdaField{⟨\$⟩}}\AgdaSpace{}%
\AgdaSymbol{(}\AgdaFunction{φ⁻¹}\AgdaSpace{}%
\AgdaOperator{\AgdaFunction{∘}}\AgdaSpace{}%
\AgdaBound{ρ}\AgdaSymbol{)}\AgdaSpace{}%
\AgdaBound{x}\AgdaSymbol{))}%
\>[52]\AgdaFunction{≈⟨}\AgdaSpace{}%
\AgdaField{cong}\AgdaSpace{}%
\AgdaOperator{\AgdaFunction{⟦}}\AgdaSpace{}%
\AgdaBound{q}\AgdaSpace{}%
\AgdaOperator{\AgdaFunction{⟧}}\AgdaSpace{}%
\AgdaFunction{ζ}%
\>[85]\AgdaFunction{⟩}\<%
\\
%
\>[9]\AgdaOperator{\AgdaFunction{⟦}}\AgdaSpace{}%
\AgdaBound{q}\AgdaSpace{}%
\AgdaOperator{\AgdaFunction{⟧}}\AgdaSpace{}%
\AgdaOperator{\AgdaField{⟨\$⟩}}\AgdaSpace{}%
\AgdaBound{ρ}%
\>[52]\AgdaOperator{\AgdaFunction{∎}}\<%
\end{code}

\hypertarget{s-preserves-identities}{%
\subparagraph{S preserves identities}\label{s-preserves-identities}}

\begin{code}%
\>[0]\<%
\\
%
\>[1]\AgdaFunction{S-id1}\AgdaSpace{}%
\AgdaSymbol{:}\AgdaSpace{}%
\AgdaBound{𝒦}\AgdaSpace{}%
\AgdaOperator{\AgdaFunction{⊫}}\AgdaSpace{}%
\AgdaSymbol{(}\AgdaBound{p}\AgdaSpace{}%
\AgdaOperator{\AgdaInductiveConstructor{≈̇}}\AgdaSpace{}%
\AgdaBound{q}\AgdaSymbol{)}\AgdaSpace{}%
\AgdaSymbol{→}\AgdaSpace{}%
\AgdaSymbol{(}\AgdaFunction{S}\AgdaSpace{}%
\AgdaSymbol{\{}\AgdaArgument{β}\AgdaSpace{}%
\AgdaSymbol{=}\AgdaSpace{}%
\AgdaBound{α}\AgdaSymbol{\}\{}\AgdaBound{ρᵃ}\AgdaSymbol{\}}\AgdaSpace{}%
\AgdaBound{ℓ}\AgdaSpace{}%
\AgdaBound{𝒦}\AgdaSymbol{)}\AgdaSpace{}%
\AgdaOperator{\AgdaFunction{⊫}}\AgdaSpace{}%
\AgdaSymbol{(}\AgdaBound{p}\AgdaSpace{}%
\AgdaOperator{\AgdaInductiveConstructor{≈̇}}\AgdaSpace{}%
\AgdaBound{q}\AgdaSymbol{)}\<%
\\
%
\>[1]\AgdaFunction{S-id1}\AgdaSpace{}%
\AgdaBound{σ}\AgdaSpace{}%
\AgdaBound{𝑩}\AgdaSpace{}%
\AgdaSymbol{(}\AgdaBound{𝑨}\AgdaSpace{}%
\AgdaOperator{\AgdaInductiveConstructor{,}}\AgdaSpace{}%
\AgdaBound{kA}\AgdaSpace{}%
\AgdaOperator{\AgdaInductiveConstructor{,}}\AgdaSpace{}%
\AgdaBound{B≤A}\AgdaSymbol{)}\AgdaSpace{}%
\AgdaSymbol{=}\AgdaSpace{}%
\AgdaFunction{⊧-S-invar}\AgdaSymbol{\{}\AgdaArgument{p}\AgdaSpace{}%
\AgdaSymbol{=}\AgdaSpace{}%
\AgdaBound{p}\AgdaSymbol{\}\{}\AgdaBound{q}\AgdaSymbol{\}}\AgdaSpace{}%
\AgdaSymbol{(}\AgdaBound{σ}\AgdaSpace{}%
\AgdaBound{𝑨}\AgdaSpace{}%
\AgdaBound{kA}\AgdaSymbol{)}\AgdaSpace{}%
\AgdaBound{B≤A}\<%
\\
\>[0]\<%
\end{code}

The obvious converse is barely worth the bits needed to formalize it,
but we will use it below, so let's prove it now.

\begin{code}%
\>[0]\<%
\\
\>[0][@{}l@{\AgdaIndent{1}}]%
\>[1]\AgdaFunction{S-id2}\AgdaSpace{}%
\AgdaSymbol{:}\AgdaSpace{}%
\AgdaFunction{S}\AgdaSpace{}%
\AgdaBound{ℓ}\AgdaSpace{}%
\AgdaBound{𝒦}\AgdaSpace{}%
\AgdaOperator{\AgdaFunction{⊫}}\AgdaSpace{}%
\AgdaSymbol{(}\AgdaBound{p}\AgdaSpace{}%
\AgdaOperator{\AgdaInductiveConstructor{≈̇}}\AgdaSpace{}%
\AgdaBound{q}\AgdaSymbol{)}\AgdaSpace{}%
\AgdaSymbol{→}\AgdaSpace{}%
\AgdaBound{𝒦}\AgdaSpace{}%
\AgdaOperator{\AgdaFunction{⊫}}\AgdaSpace{}%
\AgdaSymbol{(}\AgdaBound{p}\AgdaSpace{}%
\AgdaOperator{\AgdaInductiveConstructor{≈̇}}\AgdaSpace{}%
\AgdaBound{q}\AgdaSymbol{)}\<%
\\
%
\>[1]\AgdaFunction{S-id2}\AgdaSpace{}%
\AgdaBound{Spq}\AgdaSpace{}%
\AgdaBound{𝑨}\AgdaSpace{}%
\AgdaBound{kA}\AgdaSpace{}%
\AgdaSymbol{=}\AgdaSpace{}%
\AgdaBound{Spq}\AgdaSpace{}%
\AgdaBound{𝑨}\AgdaSpace{}%
\AgdaSymbol{(}\AgdaBound{𝑨}\AgdaSpace{}%
\AgdaOperator{\AgdaInductiveConstructor{,}}\AgdaSpace{}%
\AgdaSymbol{(}\AgdaBound{kA}\AgdaSpace{}%
\AgdaOperator{\AgdaInductiveConstructor{,}}\AgdaSpace{}%
\AgdaFunction{≤-reflexive}\AgdaSymbol{))}\<%
\end{code}

\hypertarget{p-preserves-identities}{%
\subparagraph{P preserves identities}\label{p-preserves-identities}}

\begin{code}%
\>[0]\<%
\\
%
\>[1]\AgdaFunction{P-id1}\AgdaSpace{}%
\AgdaSymbol{:}\AgdaSpace{}%
\AgdaSymbol{∀\{}\AgdaBound{ι}\AgdaSymbol{\}}\AgdaSpace{}%
\AgdaSymbol{→}\AgdaSpace{}%
\AgdaBound{𝒦}\AgdaSpace{}%
\AgdaOperator{\AgdaFunction{⊫}}\AgdaSpace{}%
\AgdaSymbol{(}\AgdaBound{p}\AgdaSpace{}%
\AgdaOperator{\AgdaInductiveConstructor{≈̇}}\AgdaSpace{}%
\AgdaBound{q}\AgdaSymbol{)}\AgdaSpace{}%
\AgdaSymbol{→}\AgdaSpace{}%
\AgdaFunction{P}\AgdaSpace{}%
\AgdaSymbol{\{}\AgdaArgument{β}\AgdaSpace{}%
\AgdaSymbol{=}\AgdaSpace{}%
\AgdaBound{α}\AgdaSymbol{\}\{}\AgdaBound{ρᵃ}\AgdaSymbol{\}}\AgdaBound{ℓ}\AgdaSpace{}%
\AgdaBound{ι}\AgdaSpace{}%
\AgdaBound{𝒦}\AgdaSpace{}%
\AgdaOperator{\AgdaFunction{⊫}}\AgdaSpace{}%
\AgdaSymbol{(}\AgdaBound{p}\AgdaSpace{}%
\AgdaOperator{\AgdaInductiveConstructor{≈̇}}\AgdaSpace{}%
\AgdaBound{q}\AgdaSymbol{)}\<%
\\
%
\>[1]\AgdaFunction{P-id1}\AgdaSpace{}%
\AgdaBound{σ}\AgdaSpace{}%
\AgdaBound{𝑨}\AgdaSpace{}%
\AgdaSymbol{(}\AgdaBound{I}\AgdaSpace{}%
\AgdaOperator{\AgdaInductiveConstructor{,}}\AgdaSpace{}%
\AgdaBound{𝒜}\AgdaSpace{}%
\AgdaOperator{\AgdaInductiveConstructor{,}}\AgdaSpace{}%
\AgdaBound{kA}\AgdaSpace{}%
\AgdaOperator{\AgdaInductiveConstructor{,}}\AgdaSpace{}%
\AgdaBound{A≅⨅A}\AgdaSymbol{)}\AgdaSpace{}%
\AgdaSymbol{=}\AgdaSpace{}%
\AgdaFunction{⊧-I-invar}\AgdaSpace{}%
\AgdaBound{𝑨}\AgdaSpace{}%
\AgdaBound{p}\AgdaSpace{}%
\AgdaBound{q}\AgdaSpace{}%
\AgdaFunction{IH}\AgdaSpace{}%
\AgdaSymbol{(}\AgdaFunction{≅-sym}\AgdaSpace{}%
\AgdaBound{A≅⨅A}\AgdaSymbol{)}\<%
\\
\>[1][@{}l@{\AgdaIndent{0}}]%
\>[2]\AgdaKeyword{where}\<%
\\
%
\>[2]\AgdaFunction{ih}\AgdaSpace{}%
\AgdaSymbol{:}\AgdaSpace{}%
\AgdaSymbol{∀}\AgdaSpace{}%
\AgdaBound{i}\AgdaSpace{}%
\AgdaSymbol{→}\AgdaSpace{}%
\AgdaBound{𝒜}\AgdaSpace{}%
\AgdaBound{i}\AgdaSpace{}%
\AgdaOperator{\AgdaFunction{⊧}}\AgdaSpace{}%
\AgdaSymbol{(}\AgdaBound{p}\AgdaSpace{}%
\AgdaOperator{\AgdaInductiveConstructor{≈̇}}\AgdaSpace{}%
\AgdaBound{q}\AgdaSymbol{)}\<%
\\
%
\>[2]\AgdaFunction{ih}\AgdaSpace{}%
\AgdaBound{i}\AgdaSpace{}%
\AgdaSymbol{=}\AgdaSpace{}%
\AgdaBound{σ}\AgdaSpace{}%
\AgdaSymbol{(}\AgdaBound{𝒜}\AgdaSpace{}%
\AgdaBound{i}\AgdaSymbol{)}\AgdaSpace{}%
\AgdaSymbol{(}\AgdaBound{kA}\AgdaSpace{}%
\AgdaBound{i}\AgdaSymbol{)}\<%
\\
%
\>[2]\AgdaFunction{IH}\AgdaSpace{}%
\AgdaSymbol{:}\AgdaSpace{}%
\AgdaFunction{⨅}\AgdaSpace{}%
\AgdaBound{𝒜}\AgdaSpace{}%
\AgdaOperator{\AgdaFunction{⊧}}\AgdaSpace{}%
\AgdaSymbol{(}\AgdaBound{p}\AgdaSpace{}%
\AgdaOperator{\AgdaInductiveConstructor{≈̇}}\AgdaSpace{}%
\AgdaBound{q}\AgdaSymbol{)}\<%
\\
%
\>[2]\AgdaFunction{IH}\AgdaSpace{}%
\AgdaSymbol{=}\AgdaSpace{}%
\AgdaFunction{⊧-P-invar}\AgdaSpace{}%
\AgdaBound{𝒜}\AgdaSpace{}%
\AgdaSymbol{\{}\AgdaBound{p}\AgdaSymbol{\}\{}\AgdaBound{q}\AgdaSymbol{\}}\AgdaSpace{}%
\AgdaFunction{ih}\<%
\end{code}

\hypertarget{v-preserves-identities}{%
\subparagraph{V preserves identities}\label{v-preserves-identities}}

Finally, we prove the analogous preservation lemmas for the closure
operator \texttt{V}.

\begin{code}%
\>[0]\<%
\\
\>[0]\AgdaKeyword{module}\AgdaSpace{}%
\AgdaModule{\AgdaUnderscore{}}\AgdaSpace{}%
\AgdaSymbol{\{}\AgdaBound{X}\AgdaSpace{}%
\AgdaSymbol{:}\AgdaSpace{}%
\AgdaPrimitive{Type}\AgdaSpace{}%
\AgdaGeneralizable{χ}\AgdaSymbol{\}\{}\AgdaBound{ι}\AgdaSpace{}%
\AgdaSymbol{:}\AgdaSpace{}%
\AgdaPostulate{Level}\AgdaSymbol{\}\{}\AgdaBound{𝒦}\AgdaSpace{}%
\AgdaSymbol{:}\AgdaSpace{}%
\AgdaFunction{Pred}\AgdaSymbol{(}\AgdaRecord{Algebra}\AgdaSpace{}%
\AgdaGeneralizable{α}\AgdaSpace{}%
\AgdaGeneralizable{ρᵃ}\AgdaSymbol{)(}\AgdaGeneralizable{α}\AgdaSpace{}%
\AgdaOperator{\AgdaPrimitive{⊔}}\AgdaSpace{}%
\AgdaGeneralizable{ρᵃ}\AgdaSpace{}%
\AgdaOperator{\AgdaPrimitive{⊔}}\AgdaSpace{}%
\AgdaFunction{ov}\AgdaSpace{}%
\AgdaGeneralizable{ℓ}\AgdaSymbol{)\}\{}\AgdaBound{p}\AgdaSpace{}%
\AgdaBound{q}\AgdaSpace{}%
\AgdaSymbol{:}\AgdaSpace{}%
\AgdaDatatype{Term}\AgdaSpace{}%
\AgdaBound{X}\AgdaSymbol{\}}\AgdaSpace{}%
\AgdaKeyword{where}\<%
\\
\>[0][@{}l@{\AgdaIndent{0}}]%
\>[1]\AgdaKeyword{private}\<%
\\
\>[1][@{}l@{\AgdaIndent{0}}]%
\>[2]\AgdaFunction{aℓι}\AgdaSpace{}%
\AgdaSymbol{=}\AgdaSpace{}%
\AgdaBound{α}\AgdaSpace{}%
\AgdaOperator{\AgdaPrimitive{⊔}}\AgdaSpace{}%
\AgdaBound{ρᵃ}\AgdaSpace{}%
\AgdaOperator{\AgdaPrimitive{⊔}}\AgdaSpace{}%
\AgdaBound{ℓ}\AgdaSpace{}%
\AgdaOperator{\AgdaPrimitive{⊔}}\AgdaSpace{}%
\AgdaBound{ι}\<%
\\
%
\\[\AgdaEmptyExtraSkip]%
%
\>[1]\AgdaFunction{V-id1}\AgdaSpace{}%
\AgdaSymbol{:}\AgdaSpace{}%
\AgdaBound{𝒦}\AgdaSpace{}%
\AgdaOperator{\AgdaFunction{⊫}}\AgdaSpace{}%
\AgdaSymbol{(}\AgdaBound{p}\AgdaSpace{}%
\AgdaOperator{\AgdaInductiveConstructor{≈̇}}\AgdaSpace{}%
\AgdaBound{q}\AgdaSymbol{)}\AgdaSpace{}%
\AgdaSymbol{→}\AgdaSpace{}%
\AgdaFunction{V}\AgdaSpace{}%
\AgdaBound{ℓ}\AgdaSpace{}%
\AgdaBound{ι}\AgdaSpace{}%
\AgdaBound{𝒦}\AgdaSpace{}%
\AgdaOperator{\AgdaFunction{⊫}}\AgdaSpace{}%
\AgdaSymbol{(}\AgdaBound{p}\AgdaSpace{}%
\AgdaOperator{\AgdaInductiveConstructor{≈̇}}\AgdaSpace{}%
\AgdaBound{q}\AgdaSymbol{)}\<%
\\
%
\>[1]\AgdaFunction{V-id1}\AgdaSpace{}%
\AgdaBound{σ}\AgdaSpace{}%
\AgdaBound{𝑩}\AgdaSpace{}%
\AgdaSymbol{(}\AgdaBound{𝑨}\AgdaSpace{}%
\AgdaOperator{\AgdaInductiveConstructor{,}}\AgdaSpace{}%
\AgdaSymbol{(}\AgdaBound{⨅A}\AgdaSpace{}%
\AgdaOperator{\AgdaInductiveConstructor{,}}\AgdaSpace{}%
\AgdaBound{p⨅A}\AgdaSpace{}%
\AgdaOperator{\AgdaInductiveConstructor{,}}\AgdaSpace{}%
\AgdaBound{A≤⨅A}\AgdaSymbol{)}\AgdaSpace{}%
\AgdaOperator{\AgdaInductiveConstructor{,}}\AgdaSpace{}%
\AgdaBound{BimgA}\AgdaSymbol{)}\AgdaSpace{}%
\AgdaSymbol{=}\<%
\\
\>[1][@{}l@{\AgdaIndent{0}}]%
\>[2]\AgdaFunction{H-id1}\AgdaSymbol{\{}\AgdaArgument{ℓ}\AgdaSpace{}%
\AgdaSymbol{=}\AgdaSpace{}%
\AgdaFunction{aℓι}\AgdaSymbol{\}\{}\AgdaArgument{𝒦}\AgdaSpace{}%
\AgdaSymbol{=}\AgdaSpace{}%
\AgdaFunction{S}\AgdaSpace{}%
\AgdaFunction{aℓι}\AgdaSpace{}%
\AgdaSymbol{(}\AgdaFunction{P}\AgdaSpace{}%
\AgdaSymbol{\{}\AgdaArgument{β}\AgdaSpace{}%
\AgdaSymbol{=}\AgdaSpace{}%
\AgdaBound{α}\AgdaSymbol{\}\{}\AgdaBound{ρᵃ}\AgdaSymbol{\}}\AgdaBound{ℓ}\AgdaSpace{}%
\AgdaBound{ι}\AgdaSpace{}%
\AgdaBound{𝒦}\AgdaSymbol{)\}\{}\AgdaArgument{p}\AgdaSpace{}%
\AgdaSymbol{=}\AgdaSpace{}%
\AgdaBound{p}\AgdaSymbol{\}\{}\AgdaBound{q}\AgdaSymbol{\}}\AgdaSpace{}%
\AgdaFunction{spK⊧pq}\AgdaSpace{}%
\AgdaBound{𝑩}\AgdaSpace{}%
\AgdaSymbol{(}\AgdaBound{𝑨}\AgdaSpace{}%
\AgdaOperator{\AgdaInductiveConstructor{,}}\AgdaSpace{}%
\AgdaSymbol{(}\AgdaFunction{spA}\AgdaSpace{}%
\AgdaOperator{\AgdaInductiveConstructor{,}}\AgdaSpace{}%
\AgdaBound{BimgA}\AgdaSymbol{))}\<%
\\
\>[2][@{}l@{\AgdaIndent{0}}]%
\>[3]\AgdaKeyword{where}\<%
\\
%
\>[3]\AgdaFunction{spA}\AgdaSpace{}%
\AgdaSymbol{:}\AgdaSpace{}%
\AgdaBound{𝑨}\AgdaSpace{}%
\AgdaOperator{\AgdaFunction{∈}}\AgdaSpace{}%
\AgdaFunction{S}\AgdaSpace{}%
\AgdaFunction{aℓι}\AgdaSpace{}%
\AgdaSymbol{(}\AgdaFunction{P}\AgdaSpace{}%
\AgdaSymbol{\{}\AgdaArgument{β}\AgdaSpace{}%
\AgdaSymbol{=}\AgdaSpace{}%
\AgdaBound{α}\AgdaSymbol{\}\{}\AgdaBound{ρᵃ}\AgdaSymbol{\}}\AgdaBound{ℓ}\AgdaSpace{}%
\AgdaBound{ι}\AgdaSpace{}%
\AgdaBound{𝒦}\AgdaSymbol{)}\<%
\\
%
\>[3]\AgdaFunction{spA}\AgdaSpace{}%
\AgdaSymbol{=}\AgdaSpace{}%
\AgdaBound{⨅A}\AgdaSpace{}%
\AgdaOperator{\AgdaInductiveConstructor{,}}\AgdaSpace{}%
\AgdaSymbol{(}\AgdaBound{p⨅A}\AgdaSpace{}%
\AgdaOperator{\AgdaInductiveConstructor{,}}\AgdaSpace{}%
\AgdaBound{A≤⨅A}\AgdaSymbol{)}\<%
\\
%
\>[3]\AgdaFunction{spK⊧pq}\AgdaSpace{}%
\AgdaSymbol{:}\AgdaSpace{}%
\AgdaFunction{S}\AgdaSpace{}%
\AgdaFunction{aℓι}\AgdaSpace{}%
\AgdaSymbol{(}\AgdaFunction{P}\AgdaSpace{}%
\AgdaBound{ℓ}\AgdaSpace{}%
\AgdaBound{ι}\AgdaSpace{}%
\AgdaBound{𝒦}\AgdaSymbol{)}\AgdaSpace{}%
\AgdaOperator{\AgdaFunction{⊫}}\AgdaSpace{}%
\AgdaSymbol{(}\AgdaBound{p}\AgdaSpace{}%
\AgdaOperator{\AgdaInductiveConstructor{≈̇}}\AgdaSpace{}%
\AgdaBound{q}\AgdaSymbol{)}\<%
\\
%
\>[3]\AgdaFunction{spK⊧pq}\AgdaSpace{}%
\AgdaSymbol{=}\AgdaSpace{}%
\AgdaFunction{S-id1}\AgdaSymbol{\{}\AgdaArgument{ℓ}\AgdaSpace{}%
\AgdaSymbol{=}\AgdaSpace{}%
\AgdaFunction{aℓι}\AgdaSymbol{\}\{}\AgdaArgument{p}\AgdaSpace{}%
\AgdaSymbol{=}\AgdaSpace{}%
\AgdaBound{p}\AgdaSymbol{\}\{}\AgdaBound{q}\AgdaSymbol{\}}\AgdaSpace{}%
\AgdaSymbol{(}\AgdaFunction{P-id1}\AgdaSymbol{\{}\AgdaArgument{ℓ}\AgdaSpace{}%
\AgdaSymbol{=}\AgdaSpace{}%
\AgdaBound{ℓ}\AgdaSymbol{\}}\AgdaSpace{}%
\AgdaSymbol{\{}\AgdaArgument{𝒦}\AgdaSpace{}%
\AgdaSymbol{=}\AgdaSpace{}%
\AgdaBound{𝒦}\AgdaSymbol{\}\{}\AgdaArgument{p}\AgdaSpace{}%
\AgdaSymbol{=}\AgdaSpace{}%
\AgdaBound{p}\AgdaSymbol{\}\{}\AgdaBound{q}\AgdaSymbol{\}}\AgdaSpace{}%
\AgdaBound{σ}\AgdaSymbol{)}\<%
\end{code}

\hypertarget{th-ux1d4a6-th-v-ux1d4a6}{%
\paragraph{Th 𝒦 ⊆ Th (V 𝒦)}\label{th-ux1d4a6-th-v-ux1d4a6}}

From \texttt{V-id1} it follows that if 𝒦 is a class of algebras, then
the set of identities modeled by the algebras in \texttt{𝒦} is contained
in the set of identities modeled by the algebras in \texttt{V\ 𝒦}. In
other terms, \texttt{Th\ 𝒦\ ⊆\ Th\ (V\ 𝒦)}. We formalize this
observation as follows.

\begin{code}%
\>[0]\<%
\\
%
\>[1]\AgdaFunction{classIds-⊆-VIds}\AgdaSpace{}%
\AgdaSymbol{:}\AgdaSpace{}%
\AgdaBound{𝒦}\AgdaSpace{}%
\AgdaOperator{\AgdaFunction{⊫}}\AgdaSpace{}%
\AgdaSymbol{(}\AgdaBound{p}\AgdaSpace{}%
\AgdaOperator{\AgdaInductiveConstructor{≈̇}}\AgdaSpace{}%
\AgdaBound{q}\AgdaSymbol{)}%
\>[33]\AgdaSymbol{→}\AgdaSpace{}%
\AgdaSymbol{(}\AgdaBound{p}\AgdaSpace{}%
\AgdaOperator{\AgdaInductiveConstructor{,}}\AgdaSpace{}%
\AgdaBound{q}\AgdaSymbol{)}\AgdaSpace{}%
\AgdaOperator{\AgdaFunction{∈}}\AgdaSpace{}%
\AgdaFunction{Th}\AgdaSpace{}%
\AgdaSymbol{(}\AgdaFunction{V}\AgdaSpace{}%
\AgdaBound{ℓ}\AgdaSpace{}%
\AgdaBound{ι}\AgdaSpace{}%
\AgdaBound{𝒦}\AgdaSymbol{)}\<%
\\
%
\>[1]\AgdaFunction{classIds-⊆-VIds}\AgdaSpace{}%
\AgdaBound{pKq}\AgdaSpace{}%
\AgdaBound{𝑨}\AgdaSpace{}%
\AgdaSymbol{=}\AgdaSpace{}%
\AgdaFunction{V-id1}\AgdaSpace{}%
\AgdaBound{pKq}\AgdaSpace{}%
\AgdaBound{𝑨}\<%
\end{code}

\hypertarget{free-algebras}{%
\subsection{Free Algebras}\label{free-algebras}}

\hypertarget{the-absolutely-free-algebra-ux1d47b-x}{%
\subsubsection{The absolutely free algebra 𝑻
X}\label{the-absolutely-free-algebra-ux1d47b-x}}

The term algebra \texttt{𝑻\ X} is \emph{absolutely free} (or
\emph{universal}, or \emph{initial}) for algebras in the signature
\texttt{𝑆}. That is, for every 𝑆-algebra \texttt{𝑨}, the following hold.

\begin{enumerate}
\def\labelenumi{\arabic{enumi}.}
\tightlist
\item
  Every function from \texttt{𝑋} to \texttt{∣\ 𝑨\ ∣} lifts to a
  homomorphism from \texttt{𝑻\ X} to \texttt{𝑨}.
\item
  The homomorphism that exists by item 1 is unique.
\end{enumerate}

We now prove this in
\href{https://wiki.portal.chalmers.se/agda/pmwiki.php}{Agda}, starting
with the fact that every map from \texttt{X} to \texttt{∣\ 𝑨\ ∣} lifts
to a map from \texttt{∣\ 𝑻\ X\ ∣} to \texttt{∣\ 𝑨\ ∣} in a natural way,
by induction on the structure of the given term.

\begin{code}%
\>[0]\<%
\\
\>[0]\AgdaKeyword{module}\AgdaSpace{}%
\AgdaModule{\AgdaUnderscore{}}\AgdaSpace{}%
\AgdaSymbol{\{}\AgdaBound{X}\AgdaSpace{}%
\AgdaSymbol{:}\AgdaSpace{}%
\AgdaPrimitive{Type}\AgdaSpace{}%
\AgdaGeneralizable{χ}\AgdaSymbol{\}\{}\AgdaBound{𝑨}\AgdaSpace{}%
\AgdaSymbol{:}\AgdaSpace{}%
\AgdaRecord{Algebra}\AgdaSpace{}%
\AgdaGeneralizable{α}\AgdaSpace{}%
\AgdaGeneralizable{ρᵃ}\AgdaSymbol{\}(}\AgdaBound{h}\AgdaSpace{}%
\AgdaSymbol{:}\AgdaSpace{}%
\AgdaBound{X}\AgdaSpace{}%
\AgdaSymbol{→}\AgdaSpace{}%
\AgdaOperator{\AgdaFunction{𝕌[}}\AgdaSpace{}%
\AgdaBound{𝑨}\AgdaSpace{}%
\AgdaOperator{\AgdaFunction{]}}\AgdaSymbol{)}\AgdaSpace{}%
\AgdaKeyword{where}\<%
\\
\>[0][@{}l@{\AgdaIndent{0}}]%
\>[1]\AgdaKeyword{open}\AgdaSpace{}%
\AgdaModule{Setoid}\AgdaSpace{}%
\AgdaOperator{\AgdaFunction{𝔻[}}\AgdaSpace{}%
\AgdaBound{𝑨}\AgdaSpace{}%
\AgdaOperator{\AgdaFunction{]}}\AgdaSpace{}%
\AgdaKeyword{using}\AgdaSpace{}%
\AgdaSymbol{(}\AgdaSpace{}%
\AgdaOperator{\AgdaField{\AgdaUnderscore{}≈\AgdaUnderscore{}}}\AgdaSpace{}%
\AgdaSymbol{;}\AgdaSpace{}%
\AgdaFunction{reflexive}\AgdaSpace{}%
\AgdaSymbol{;}\AgdaSpace{}%
\AgdaFunction{refl}\AgdaSpace{}%
\AgdaSymbol{;}\AgdaSpace{}%
\AgdaFunction{trans}\AgdaSpace{}%
\AgdaSymbol{)}\<%
\\
%
\\[\AgdaEmptyExtraSkip]%
%
\>[1]\AgdaFunction{free-lift}\AgdaSpace{}%
\AgdaSymbol{:}\AgdaSpace{}%
\AgdaOperator{\AgdaFunction{𝕌[}}\AgdaSpace{}%
\AgdaFunction{𝑻}\AgdaSpace{}%
\AgdaBound{X}\AgdaSpace{}%
\AgdaOperator{\AgdaFunction{]}}\AgdaSpace{}%
\AgdaSymbol{→}\AgdaSpace{}%
\AgdaOperator{\AgdaFunction{𝕌[}}\AgdaSpace{}%
\AgdaBound{𝑨}\AgdaSpace{}%
\AgdaOperator{\AgdaFunction{]}}\<%
\\
%
\>[1]\AgdaFunction{free-lift}\AgdaSpace{}%
\AgdaSymbol{(}\AgdaInductiveConstructor{ℊ}\AgdaSpace{}%
\AgdaBound{x}\AgdaSymbol{)}\AgdaSpace{}%
\AgdaSymbol{=}\AgdaSpace{}%
\AgdaBound{h}\AgdaSpace{}%
\AgdaBound{x}\<%
\\
%
\>[1]\AgdaFunction{free-lift}\AgdaSpace{}%
\AgdaSymbol{(}\AgdaInductiveConstructor{node}\AgdaSpace{}%
\AgdaBound{f}\AgdaSpace{}%
\AgdaBound{t}\AgdaSymbol{)}\AgdaSpace{}%
\AgdaSymbol{=}\AgdaSpace{}%
\AgdaSymbol{(}\AgdaBound{f}\AgdaSpace{}%
\AgdaOperator{\AgdaFunction{̂}}\AgdaSpace{}%
\AgdaBound{𝑨}\AgdaSymbol{)}\AgdaSpace{}%
\AgdaSymbol{(λ}\AgdaSpace{}%
\AgdaBound{i}\AgdaSpace{}%
\AgdaSymbol{→}\AgdaSpace{}%
\AgdaFunction{free-lift}\AgdaSpace{}%
\AgdaSymbol{(}\AgdaBound{t}\AgdaSpace{}%
\AgdaBound{i}\AgdaSymbol{))}\<%
\\
%
\\[\AgdaEmptyExtraSkip]%
%
\>[1]\AgdaFunction{free-lift-func}\AgdaSpace{}%
\AgdaSymbol{:}\AgdaSpace{}%
\AgdaOperator{\AgdaFunction{𝔻[}}\AgdaSpace{}%
\AgdaFunction{𝑻}\AgdaSpace{}%
\AgdaBound{X}\AgdaSpace{}%
\AgdaOperator{\AgdaFunction{]}}\AgdaSpace{}%
\AgdaOperator{\AgdaRecord{⟶}}\AgdaSpace{}%
\AgdaOperator{\AgdaFunction{𝔻[}}\AgdaSpace{}%
\AgdaBound{𝑨}\AgdaSpace{}%
\AgdaOperator{\AgdaFunction{]}}\<%
\\
%
\>[1]\AgdaFunction{free-lift-func}\AgdaSpace{}%
\AgdaOperator{\AgdaField{⟨\$⟩}}\AgdaSpace{}%
\AgdaBound{x}\AgdaSpace{}%
\AgdaSymbol{=}\AgdaSpace{}%
\AgdaFunction{free-lift}\AgdaSpace{}%
\AgdaBound{x}\<%
\\
%
\>[1]\AgdaField{cong}\AgdaSpace{}%
\AgdaFunction{free-lift-func}\AgdaSpace{}%
\AgdaSymbol{=}\AgdaSpace{}%
\AgdaFunction{flcong}\<%
\\
\>[1][@{}l@{\AgdaIndent{0}}]%
\>[2]\AgdaKeyword{where}\<%
\\
%
\>[2]\AgdaFunction{flcong}\AgdaSpace{}%
\AgdaSymbol{:}\AgdaSpace{}%
\AgdaSymbol{∀}\AgdaSpace{}%
\AgdaSymbol{\{}\AgdaBound{s}\AgdaSpace{}%
\AgdaBound{t}\AgdaSymbol{\}}\AgdaSpace{}%
\AgdaSymbol{→}\AgdaSpace{}%
\AgdaBound{s}\AgdaSpace{}%
\AgdaOperator{\AgdaDatatype{≐}}\AgdaSpace{}%
\AgdaBound{t}\AgdaSpace{}%
\AgdaSymbol{→}%
\>[30]\AgdaFunction{free-lift}\AgdaSpace{}%
\AgdaBound{s}\AgdaSpace{}%
\AgdaOperator{\AgdaFunction{≈}}\AgdaSpace{}%
\AgdaFunction{free-lift}\AgdaSpace{}%
\AgdaBound{t}\<%
\\
%
\>[2]\AgdaFunction{flcong}\AgdaSpace{}%
\AgdaSymbol{(}\AgdaInductiveConstructor{\AgdaUnderscore{}≐\AgdaUnderscore{}.rfl}\AgdaSpace{}%
\AgdaBound{x}\AgdaSymbol{)}\AgdaSpace{}%
\AgdaSymbol{=}\AgdaSpace{}%
\AgdaFunction{reflexive}\AgdaSpace{}%
\AgdaSymbol{(}\AgdaFunction{≡.cong}\AgdaSpace{}%
\AgdaBound{h}\AgdaSpace{}%
\AgdaBound{x}\AgdaSymbol{)}\<%
\\
%
\>[2]\AgdaFunction{flcong}\AgdaSpace{}%
\AgdaSymbol{(}\AgdaInductiveConstructor{\AgdaUnderscore{}≐\AgdaUnderscore{}.gnl}\AgdaSpace{}%
\AgdaBound{x}\AgdaSymbol{)}\AgdaSpace{}%
\AgdaSymbol{=}\AgdaSpace{}%
\AgdaField{cong}\AgdaSpace{}%
\AgdaSymbol{(}\AgdaField{Interp}\AgdaSpace{}%
\AgdaBound{𝑨}\AgdaSymbol{)}\AgdaSpace{}%
\AgdaSymbol{(}\AgdaInductiveConstructor{≡.refl}\AgdaSpace{}%
\AgdaOperator{\AgdaInductiveConstructor{,}}\AgdaSpace{}%
\AgdaSymbol{(λ}\AgdaSpace{}%
\AgdaBound{i}\AgdaSpace{}%
\AgdaSymbol{→}\AgdaSpace{}%
\AgdaFunction{flcong}\AgdaSpace{}%
\AgdaSymbol{(}\AgdaBound{x}\AgdaSpace{}%
\AgdaBound{i}\AgdaSymbol{)))}\<%
\\
\>[0]\<%
\end{code}

Naturally, at the base step of the induction, when the term has the form
\texttt{generator} x, the free lift of \texttt{h} agrees with
\texttt{h}. For the inductive step, when the given term has the form
\texttt{node\ f\ t}, the free lift is defined as follows: Assuming (the
induction hypothesis) that we know the image of each subterm
\texttt{t\ i} under the free lift of \texttt{h}, define the free lift at
the full term by applying \texttt{f\ ̂\ 𝑨} to the images of the subterms.

The free lift so defined is a homomorphism by construction. Indeed, here
is the trivial proof.

\begin{code}%
\>[0]\<%
\\
\>[0][@{}l@{\AgdaIndent{1}}]%
\>[1]\AgdaFunction{lift-hom}\AgdaSpace{}%
\AgdaSymbol{:}\AgdaSpace{}%
\AgdaFunction{hom}\AgdaSpace{}%
\AgdaSymbol{(}\AgdaFunction{𝑻}\AgdaSpace{}%
\AgdaBound{X}\AgdaSymbol{)}\AgdaSpace{}%
\AgdaBound{𝑨}\<%
\\
%
\>[1]\AgdaFunction{lift-hom}\AgdaSpace{}%
\AgdaSymbol{=}\AgdaSpace{}%
\AgdaFunction{free-lift-func}\AgdaSpace{}%
\AgdaOperator{\AgdaInductiveConstructor{,}}\AgdaSpace{}%
\AgdaFunction{hhom}\<%
\\
\>[1][@{}l@{\AgdaIndent{0}}]%
\>[2]\AgdaKeyword{where}\<%
\\
%
\>[2]\AgdaFunction{hfunc}\AgdaSpace{}%
\AgdaSymbol{:}\AgdaSpace{}%
\AgdaOperator{\AgdaFunction{𝔻[}}\AgdaSpace{}%
\AgdaFunction{𝑻}\AgdaSpace{}%
\AgdaBound{X}\AgdaSpace{}%
\AgdaOperator{\AgdaFunction{]}}\AgdaSpace{}%
\AgdaOperator{\AgdaRecord{⟶}}\AgdaSpace{}%
\AgdaOperator{\AgdaFunction{𝔻[}}\AgdaSpace{}%
\AgdaBound{𝑨}\AgdaSpace{}%
\AgdaOperator{\AgdaFunction{]}}\<%
\\
%
\>[2]\AgdaFunction{hfunc}\AgdaSpace{}%
\AgdaSymbol{=}\AgdaSpace{}%
\AgdaFunction{free-lift-func}\<%
\\
%
\\[\AgdaEmptyExtraSkip]%
%
\>[2]\AgdaFunction{hcomp}\AgdaSpace{}%
\AgdaSymbol{:}\AgdaSpace{}%
\AgdaFunction{compatible-map}\AgdaSpace{}%
\AgdaSymbol{(}\AgdaFunction{𝑻}\AgdaSpace{}%
\AgdaBound{X}\AgdaSymbol{)}\AgdaSpace{}%
\AgdaBound{𝑨}\AgdaSpace{}%
\AgdaFunction{free-lift-func}\<%
\\
%
\>[2]\AgdaFunction{hcomp}\AgdaSpace{}%
\AgdaSymbol{\{}\AgdaBound{f}\AgdaSymbol{\}\{}\AgdaBound{a}\AgdaSymbol{\}}\AgdaSpace{}%
\AgdaSymbol{=}\AgdaSpace{}%
\AgdaField{cong}\AgdaSpace{}%
\AgdaSymbol{(}\AgdaField{Interp}\AgdaSpace{}%
\AgdaBound{𝑨}\AgdaSymbol{)}\AgdaSpace{}%
\AgdaSymbol{(}\AgdaInductiveConstructor{≡.refl}\AgdaSpace{}%
\AgdaOperator{\AgdaInductiveConstructor{,}}\AgdaSpace{}%
\AgdaSymbol{(λ}\AgdaSpace{}%
\AgdaBound{i}\AgdaSpace{}%
\AgdaSymbol{→}\AgdaSpace{}%
\AgdaSymbol{(}\AgdaField{cong}\AgdaSpace{}%
\AgdaFunction{free-lift-func}\AgdaSymbol{)\{}\AgdaBound{a}\AgdaSpace{}%
\AgdaBound{i}\AgdaSymbol{\}}\AgdaSpace{}%
\AgdaFunction{≐-isRefl}\AgdaSymbol{))}\<%
\\
%
\\[\AgdaEmptyExtraSkip]%
%
\>[2]\AgdaFunction{hhom}\AgdaSpace{}%
\AgdaSymbol{:}\AgdaSpace{}%
\AgdaRecord{IsHom}\AgdaSpace{}%
\AgdaSymbol{(}\AgdaFunction{𝑻}\AgdaSpace{}%
\AgdaBound{X}\AgdaSymbol{)}\AgdaSpace{}%
\AgdaBound{𝑨}\AgdaSpace{}%
\AgdaFunction{hfunc}\<%
\\
%
\>[2]\AgdaFunction{hhom}\AgdaSpace{}%
\AgdaSymbol{=}\AgdaSpace{}%
\AgdaInductiveConstructor{mkhom}\AgdaSpace{}%
\AgdaSymbol{(λ\{}\AgdaBound{f}\AgdaSymbol{\}\{}\AgdaBound{a}\AgdaSymbol{\}}\AgdaSpace{}%
\AgdaSymbol{→}\AgdaSpace{}%
\AgdaFunction{hcomp}\AgdaSymbol{\{}\AgdaBound{f}\AgdaSymbol{\}\{}\AgdaBound{a}\AgdaSymbol{\})}\<%
\\
%
\\[\AgdaEmptyExtraSkip]%
%
\\[\AgdaEmptyExtraSkip]%
\>[0]\AgdaKeyword{module}\AgdaSpace{}%
\AgdaModule{\AgdaUnderscore{}}\AgdaSpace{}%
\AgdaSymbol{\{}\AgdaBound{X}\AgdaSpace{}%
\AgdaSymbol{:}\AgdaSpace{}%
\AgdaPrimitive{Type}\AgdaSpace{}%
\AgdaGeneralizable{χ}\AgdaSymbol{\}\{}\AgdaBound{𝑨}\AgdaSpace{}%
\AgdaSymbol{:}\AgdaSpace{}%
\AgdaRecord{Algebra}\AgdaSpace{}%
\AgdaGeneralizable{α}\AgdaSpace{}%
\AgdaGeneralizable{ρᵃ}\AgdaSymbol{\}}\AgdaSpace{}%
\AgdaKeyword{where}\<%
\\
\>[0][@{}l@{\AgdaIndent{0}}]%
\>[1]\AgdaKeyword{open}\AgdaSpace{}%
\AgdaModule{Setoid}\AgdaSpace{}%
\AgdaOperator{\AgdaFunction{𝔻[}}\AgdaSpace{}%
\AgdaBound{𝑨}\AgdaSpace{}%
\AgdaOperator{\AgdaFunction{]}}\AgdaSpace{}%
\AgdaKeyword{using}\AgdaSpace{}%
\AgdaSymbol{(}\AgdaSpace{}%
\AgdaOperator{\AgdaField{\AgdaUnderscore{}≈\AgdaUnderscore{}}}\AgdaSpace{}%
\AgdaSymbol{;}\AgdaSpace{}%
\AgdaFunction{refl}\AgdaSpace{}%
\AgdaSymbol{)}\<%
\\
%
\>[1]\AgdaKeyword{open}\AgdaSpace{}%
\AgdaModule{Environment}\AgdaSpace{}%
\AgdaBound{𝑨}\AgdaSpace{}%
\AgdaKeyword{using}\AgdaSpace{}%
\AgdaSymbol{(}\AgdaSpace{}%
\AgdaOperator{\AgdaFunction{⟦\AgdaUnderscore{}⟧}}\AgdaSpace{}%
\AgdaSymbol{)}\<%
\\
%
\\[\AgdaEmptyExtraSkip]%
%
\>[1]\AgdaFunction{free-lift-interp}\AgdaSpace{}%
\AgdaSymbol{:}\AgdaSpace{}%
\AgdaSymbol{(}\AgdaBound{η}\AgdaSpace{}%
\AgdaSymbol{:}\AgdaSpace{}%
\AgdaBound{X}\AgdaSpace{}%
\AgdaSymbol{→}\AgdaSpace{}%
\AgdaOperator{\AgdaFunction{𝕌[}}\AgdaSpace{}%
\AgdaBound{𝑨}\AgdaSpace{}%
\AgdaOperator{\AgdaFunction{]}}\AgdaSymbol{)(}\AgdaBound{p}\AgdaSpace{}%
\AgdaSymbol{:}\AgdaSpace{}%
\AgdaDatatype{Term}\AgdaSpace{}%
\AgdaBound{X}\AgdaSymbol{)}\AgdaSpace{}%
\AgdaSymbol{→}\AgdaSpace{}%
\AgdaOperator{\AgdaFunction{⟦}}\AgdaSpace{}%
\AgdaBound{p}\AgdaSpace{}%
\AgdaOperator{\AgdaFunction{⟧}}\AgdaSpace{}%
\AgdaOperator{\AgdaField{⟨\$⟩}}\AgdaSpace{}%
\AgdaBound{η}\AgdaSpace{}%
\AgdaOperator{\AgdaFunction{≈}}\AgdaSpace{}%
\AgdaSymbol{(}\AgdaFunction{free-lift}\AgdaSpace{}%
\AgdaSymbol{\{}\AgdaArgument{𝑨}\AgdaSpace{}%
\AgdaSymbol{=}\AgdaSpace{}%
\AgdaBound{𝑨}\AgdaSymbol{\}}\AgdaSpace{}%
\AgdaBound{η}\AgdaSymbol{)}\AgdaSpace{}%
\AgdaBound{p}\<%
\\
%
\>[1]\AgdaFunction{free-lift-interp}\AgdaSpace{}%
\AgdaBound{η}\AgdaSpace{}%
\AgdaSymbol{(}\AgdaInductiveConstructor{ℊ}\AgdaSpace{}%
\AgdaBound{x}\AgdaSymbol{)}\AgdaSpace{}%
\AgdaSymbol{=}\AgdaSpace{}%
\AgdaFunction{refl}\<%
\\
%
\>[1]\AgdaFunction{free-lift-interp}\AgdaSpace{}%
\AgdaBound{η}\AgdaSpace{}%
\AgdaSymbol{(}\AgdaInductiveConstructor{node}\AgdaSpace{}%
\AgdaBound{f}\AgdaSpace{}%
\AgdaBound{t}\AgdaSymbol{)}\AgdaSpace{}%
\AgdaSymbol{=}\AgdaSpace{}%
\AgdaField{cong}\AgdaSpace{}%
\AgdaSymbol{(}\AgdaField{Interp}\AgdaSpace{}%
\AgdaBound{𝑨}\AgdaSymbol{)}\AgdaSpace{}%
\AgdaSymbol{(}\AgdaInductiveConstructor{≡.refl}\AgdaSpace{}%
\AgdaOperator{\AgdaInductiveConstructor{,}}\AgdaSpace{}%
\AgdaSymbol{(}\AgdaFunction{free-lift-interp}\AgdaSpace{}%
\AgdaBound{η}\AgdaSymbol{)}\AgdaSpace{}%
\AgdaOperator{\AgdaFunction{∘}}\AgdaSpace{}%
\AgdaBound{t}\AgdaSymbol{)}\<%
\end{code}

\hypertarget{the-relatively-free-algebra-ux1d53d}{%
\subsubsection{The relatively free algebra
𝔽}\label{the-relatively-free-algebra-ux1d53d}}

We now define the algebra \texttt{𝔽{[}\ X\ {]}}, which plays the role of
the relatively free algebra, along with the natural epimorphism
\texttt{epi𝔽\ :\ epi\ (𝑻\ X)\ 𝔽{[}\ X\ {]}} from \texttt{𝑻\ X} to
\texttt{𝔽{[}\ X\ {]}}.

\begin{code}%
\>[0]\<%
\\
\>[0]\AgdaKeyword{module}\AgdaSpace{}%
\AgdaModule{FreeAlgebra}\AgdaSpace{}%
\AgdaSymbol{\{}\AgdaBound{χ}\AgdaSpace{}%
\AgdaSymbol{:}\AgdaSpace{}%
\AgdaPostulate{Level}\AgdaSymbol{\}\{}\AgdaBound{ι}\AgdaSpace{}%
\AgdaSymbol{:}\AgdaSpace{}%
\AgdaPostulate{Level}\AgdaSymbol{\}\{}\AgdaBound{I}\AgdaSpace{}%
\AgdaSymbol{:}\AgdaSpace{}%
\AgdaPrimitive{Type}\AgdaSpace{}%
\AgdaBound{ι}\AgdaSymbol{\}(}\AgdaBound{ℰ}\AgdaSpace{}%
\AgdaSymbol{:}\AgdaSpace{}%
\AgdaBound{I}\AgdaSpace{}%
\AgdaSymbol{→}\AgdaSpace{}%
\AgdaRecord{Eq}\AgdaSymbol{)}\AgdaSpace{}%
\AgdaKeyword{where}\<%
\\
\>[0][@{}l@{\AgdaIndent{0}}]%
\>[1]\AgdaKeyword{open}\AgdaSpace{}%
\AgdaModule{Algebra}\<%
\\
%
\\[\AgdaEmptyExtraSkip]%
%
\>[1]\AgdaFunction{FreeDomain}\AgdaSpace{}%
\AgdaSymbol{:}\AgdaSpace{}%
\AgdaPrimitive{Type}\AgdaSpace{}%
\AgdaBound{χ}\AgdaSpace{}%
\AgdaSymbol{→}\AgdaSpace{}%
\AgdaRecord{Setoid}\AgdaSpace{}%
\AgdaSymbol{\AgdaUnderscore{}}\AgdaSpace{}%
\AgdaSymbol{\AgdaUnderscore{}}\<%
\\
%
\>[1]\AgdaFunction{FreeDomain}\AgdaSpace{}%
\AgdaBound{X}\AgdaSpace{}%
\AgdaSymbol{=}\AgdaSpace{}%
\AgdaKeyword{record}%
\>[24]\AgdaSymbol{\{}\AgdaSpace{}%
\AgdaField{Carrier}%
\>[41]\AgdaSymbol{=}\AgdaSpace{}%
\AgdaDatatype{Term}\AgdaSpace{}%
\AgdaBound{X}\<%
\\
%
\>[24]\AgdaSymbol{;}\AgdaSpace{}%
\AgdaOperator{\AgdaField{\AgdaUnderscore{}≈\AgdaUnderscore{}}}%
\>[41]\AgdaSymbol{=}\AgdaSpace{}%
\AgdaBound{ℰ}\AgdaSpace{}%
\AgdaOperator{\AgdaDatatype{⊢}}\AgdaSpace{}%
\AgdaBound{X}\AgdaSpace{}%
\AgdaOperator{\AgdaDatatype{▹\AgdaUnderscore{}≈\AgdaUnderscore{}}}\<%
\\
%
\>[24]\AgdaSymbol{;}\AgdaSpace{}%
\AgdaField{isEquivalence}%
\>[41]\AgdaSymbol{=}\AgdaSpace{}%
\AgdaFunction{⊢▹≈IsEquiv}\AgdaSpace{}%
\AgdaSymbol{\}}\<%
\end{code}

The interpretation of an operation is simply the operation itself. This
works since \texttt{ℰ\ ⊢\ X\ ▹\_≈\_} is a congruence.

\begin{code}%
\>[0]\<%
\\
%
\>[1]\AgdaFunction{FreeInterp}\AgdaSpace{}%
\AgdaSymbol{:}\AgdaSpace{}%
\AgdaSymbol{∀}\AgdaSpace{}%
\AgdaSymbol{\{}\AgdaBound{X}\AgdaSymbol{\}}\AgdaSpace{}%
\AgdaSymbol{→}\AgdaSpace{}%
\AgdaOperator{\AgdaFunction{⟨}}\AgdaSpace{}%
\AgdaBound{𝑆}\AgdaSpace{}%
\AgdaOperator{\AgdaFunction{⟩}}\AgdaSpace{}%
\AgdaSymbol{(}\AgdaFunction{FreeDomain}\AgdaSpace{}%
\AgdaBound{X}\AgdaSymbol{)}\AgdaSpace{}%
\AgdaOperator{\AgdaRecord{⟶}}\AgdaSpace{}%
\AgdaFunction{FreeDomain}\AgdaSpace{}%
\AgdaBound{X}\<%
\\
%
\>[1]\AgdaFunction{FreeInterp}\AgdaSpace{}%
\AgdaOperator{\AgdaField{⟨\$⟩}}\AgdaSpace{}%
\AgdaSymbol{(}\AgdaBound{f}\AgdaSpace{}%
\AgdaOperator{\AgdaInductiveConstructor{,}}\AgdaSpace{}%
\AgdaBound{ts}\AgdaSymbol{)}\AgdaSpace{}%
\AgdaSymbol{=}\AgdaSpace{}%
\AgdaInductiveConstructor{node}\AgdaSpace{}%
\AgdaBound{f}\AgdaSpace{}%
\AgdaBound{ts}\<%
\\
%
\>[1]\AgdaField{cong}\AgdaSpace{}%
\AgdaFunction{FreeInterp}\AgdaSpace{}%
\AgdaSymbol{(}\AgdaInductiveConstructor{≡.refl}\AgdaSpace{}%
\AgdaOperator{\AgdaInductiveConstructor{,}}\AgdaSpace{}%
\AgdaBound{h}\AgdaSymbol{)}\AgdaSpace{}%
\AgdaSymbol{=}\AgdaSpace{}%
\AgdaInductiveConstructor{app}\AgdaSpace{}%
\AgdaBound{h}\<%
\\
%
\\[\AgdaEmptyExtraSkip]%
%
\>[1]\AgdaOperator{\AgdaFunction{𝔽[\AgdaUnderscore{}]}}\AgdaSpace{}%
\AgdaSymbol{:}\AgdaSpace{}%
\AgdaPrimitive{Type}\AgdaSpace{}%
\AgdaBound{χ}\AgdaSpace{}%
\AgdaSymbol{→}\AgdaSpace{}%
\AgdaRecord{Algebra}\AgdaSpace{}%
\AgdaSymbol{(}\AgdaFunction{ov}\AgdaSpace{}%
\AgdaBound{χ}\AgdaSymbol{)}\AgdaSpace{}%
\AgdaSymbol{(}\AgdaBound{ι}\AgdaSpace{}%
\AgdaOperator{\AgdaPrimitive{⊔}}\AgdaSpace{}%
\AgdaFunction{ov}\AgdaSpace{}%
\AgdaBound{χ}\AgdaSymbol{)}\<%
\\
%
\>[1]\AgdaField{Domain}\AgdaSpace{}%
\AgdaOperator{\AgdaFunction{𝔽[}}\AgdaSpace{}%
\AgdaBound{X}\AgdaSpace{}%
\AgdaOperator{\AgdaFunction{]}}\AgdaSpace{}%
\AgdaSymbol{=}\AgdaSpace{}%
\AgdaFunction{FreeDomain}\AgdaSpace{}%
\AgdaBound{X}\<%
\\
%
\>[1]\AgdaField{Interp}\AgdaSpace{}%
\AgdaOperator{\AgdaFunction{𝔽[}}\AgdaSpace{}%
\AgdaBound{X}\AgdaSpace{}%
\AgdaOperator{\AgdaFunction{]}}\AgdaSpace{}%
\AgdaSymbol{=}\AgdaSpace{}%
\AgdaFunction{FreeInterp}\<%
\end{code}

\hypertarget{basic-properties-of-free-algebras}{%
\subsubsection{Basic properties of free
algebras}\label{basic-properties-of-free-algebras}}

In the code below, \texttt{X} will play the role of an arbitrary
collection of variables; it would suffice to take \texttt{X} to be the
cardinality of the largest algebra in 𝒦, but since we don't know that
cardinality, we leave \texttt{X} aribtrary for now.

\begin{code}%
\>[0]\<%
\\
\>[0]\AgdaKeyword{module}\AgdaSpace{}%
\AgdaModule{FreeHom}\AgdaSpace{}%
\AgdaSymbol{(}\AgdaBound{χ}\AgdaSpace{}%
\AgdaSymbol{:}\AgdaSpace{}%
\AgdaPostulate{Level}\AgdaSymbol{)}\AgdaSpace{}%
\AgdaSymbol{\{}\AgdaBound{𝒦}\AgdaSpace{}%
\AgdaSymbol{:}\AgdaSpace{}%
\AgdaFunction{Pred}\AgdaSymbol{(}\AgdaRecord{Algebra}\AgdaSpace{}%
\AgdaGeneralizable{α}\AgdaSpace{}%
\AgdaGeneralizable{ρᵃ}\AgdaSymbol{)}\AgdaSpace{}%
\AgdaSymbol{(}\AgdaGeneralizable{α}\AgdaSpace{}%
\AgdaOperator{\AgdaPrimitive{⊔}}\AgdaSpace{}%
\AgdaGeneralizable{ρᵃ}\AgdaSpace{}%
\AgdaOperator{\AgdaPrimitive{⊔}}\AgdaSpace{}%
\AgdaFunction{ov}\AgdaSpace{}%
\AgdaGeneralizable{ℓ}\AgdaSymbol{)\}}\AgdaSpace{}%
\AgdaKeyword{where}\<%
\\
\>[0][@{}l@{\AgdaIndent{0}}]%
\>[1]\AgdaKeyword{private}\AgdaSpace{}%
\AgdaFunction{ι}\AgdaSpace{}%
\AgdaSymbol{=}\AgdaSpace{}%
\AgdaFunction{ov}\AgdaSymbol{(}\AgdaBound{χ}\AgdaSpace{}%
\AgdaOperator{\AgdaPrimitive{⊔}}\AgdaSpace{}%
\AgdaBound{α}\AgdaSpace{}%
\AgdaOperator{\AgdaPrimitive{⊔}}\AgdaSpace{}%
\AgdaBound{ρᵃ}\AgdaSpace{}%
\AgdaOperator{\AgdaPrimitive{⊔}}\AgdaSpace{}%
\AgdaBound{ℓ}\AgdaSymbol{)}\<%
\\
%
\>[1]\AgdaKeyword{open}\AgdaSpace{}%
\AgdaModule{Eq}\<%
\\
%
\\[\AgdaEmptyExtraSkip]%
%
\>[1]\AgdaFunction{ℐ}\AgdaSpace{}%
\AgdaSymbol{:}\AgdaSpace{}%
\AgdaPrimitive{Type}\AgdaSpace{}%
\AgdaFunction{ι}\AgdaSpace{}%
\AgdaComment{--\ indexes\ the\ collection\ of\ equations\ modeled\ by\ 𝒦}\<%
\\
%
\>[1]\AgdaFunction{ℐ}\AgdaSpace{}%
\AgdaSymbol{=}\AgdaSpace{}%
\AgdaFunction{Σ[}\AgdaSpace{}%
\AgdaBound{eq}\AgdaSpace{}%
\AgdaFunction{∈}\AgdaSpace{}%
\AgdaRecord{Eq}\AgdaSymbol{\{}\AgdaBound{χ}\AgdaSymbol{\}}\AgdaSpace{}%
\AgdaFunction{]}\AgdaSpace{}%
\AgdaBound{𝒦}\AgdaSpace{}%
\AgdaOperator{\AgdaFunction{⊫}}\AgdaSpace{}%
\AgdaSymbol{((}\AgdaField{lhs}\AgdaSpace{}%
\AgdaBound{eq}\AgdaSymbol{)}\AgdaSpace{}%
\AgdaOperator{\AgdaInductiveConstructor{≈̇}}\AgdaSpace{}%
\AgdaSymbol{(}\AgdaField{rhs}\AgdaSpace{}%
\AgdaBound{eq}\AgdaSymbol{))}\<%
\\
%
\\[\AgdaEmptyExtraSkip]%
%
\>[1]\AgdaFunction{ℰ}\AgdaSpace{}%
\AgdaSymbol{:}\AgdaSpace{}%
\AgdaFunction{ℐ}\AgdaSpace{}%
\AgdaSymbol{→}\AgdaSpace{}%
\AgdaRecord{Eq}\<%
\\
%
\>[1]\AgdaFunction{ℰ}\AgdaSpace{}%
\AgdaSymbol{(}\AgdaBound{eqv}\AgdaSpace{}%
\AgdaOperator{\AgdaInductiveConstructor{,}}\AgdaSpace{}%
\AgdaBound{p}\AgdaSymbol{)}\AgdaSpace{}%
\AgdaSymbol{=}\AgdaSpace{}%
\AgdaBound{eqv}\<%
\\
%
\\[\AgdaEmptyExtraSkip]%
%
\>[1]\AgdaOperator{\AgdaFunction{ℰ⊢[\AgdaUnderscore{}]▹Th𝒦}}\AgdaSpace{}%
\AgdaSymbol{:}\AgdaSpace{}%
\AgdaSymbol{(}\AgdaBound{X}\AgdaSpace{}%
\AgdaSymbol{:}\AgdaSpace{}%
\AgdaPrimitive{Type}\AgdaSpace{}%
\AgdaBound{χ}\AgdaSymbol{)}\AgdaSpace{}%
\AgdaSymbol{→}\AgdaSpace{}%
\AgdaSymbol{∀\{}\AgdaBound{p}\AgdaSpace{}%
\AgdaBound{q}\AgdaSymbol{\}}\AgdaSpace{}%
\AgdaSymbol{→}\AgdaSpace{}%
\AgdaFunction{ℰ}\AgdaSpace{}%
\AgdaOperator{\AgdaDatatype{⊢}}\AgdaSpace{}%
\AgdaBound{X}\AgdaSpace{}%
\AgdaOperator{\AgdaDatatype{▹}}\AgdaSpace{}%
\AgdaBound{p}\AgdaSpace{}%
\AgdaOperator{\AgdaDatatype{≈}}\AgdaSpace{}%
\AgdaBound{q}\AgdaSpace{}%
\AgdaSymbol{→}\AgdaSpace{}%
\AgdaBound{𝒦}\AgdaSpace{}%
\AgdaOperator{\AgdaFunction{⊫}}\AgdaSpace{}%
\AgdaSymbol{(}\AgdaBound{p}\AgdaSpace{}%
\AgdaOperator{\AgdaInductiveConstructor{≈̇}}\AgdaSpace{}%
\AgdaBound{q}\AgdaSymbol{)}\<%
\\
%
\>[1]\AgdaOperator{\AgdaFunction{ℰ⊢[}}\AgdaSpace{}%
\AgdaBound{X}\AgdaSpace{}%
\AgdaOperator{\AgdaFunction{]▹Th𝒦}}\AgdaSpace{}%
\AgdaBound{x}\AgdaSpace{}%
\AgdaBound{𝑨}\AgdaSpace{}%
\AgdaBound{kA}\AgdaSpace{}%
\AgdaSymbol{=}\AgdaSpace{}%
\AgdaFunction{sound}\AgdaSpace{}%
\AgdaSymbol{(λ}\AgdaSpace{}%
\AgdaBound{i}\AgdaSpace{}%
\AgdaBound{ρ}\AgdaSpace{}%
\AgdaSymbol{→}\AgdaSpace{}%
\AgdaOperator{\AgdaFunction{∥}}\AgdaSpace{}%
\AgdaBound{i}\AgdaSpace{}%
\AgdaOperator{\AgdaFunction{∥}}\AgdaSpace{}%
\AgdaBound{𝑨}\AgdaSpace{}%
\AgdaBound{kA}\AgdaSpace{}%
\AgdaBound{ρ}\AgdaSymbol{)}\AgdaSpace{}%
\AgdaBound{x}\<%
\\
\>[1][@{}l@{\AgdaIndent{0}}]%
\>[2]\AgdaKeyword{where}\AgdaSpace{}%
\AgdaKeyword{open}\AgdaSpace{}%
\AgdaModule{Soundness}\AgdaSpace{}%
\AgdaFunction{ℰ}\AgdaSpace{}%
\AgdaBound{𝑨}\<%
\\
%
\>[1]\AgdaKeyword{open}\AgdaSpace{}%
\AgdaModule{FreeAlgebra}\AgdaSpace{}%
\AgdaSymbol{\{}\AgdaArgument{ι}\AgdaSpace{}%
\AgdaSymbol{=}\AgdaSpace{}%
\AgdaFunction{ι}\AgdaSymbol{\}\{}\AgdaArgument{I}\AgdaSpace{}%
\AgdaSymbol{=}\AgdaSpace{}%
\AgdaFunction{ℐ}\AgdaSymbol{\}}\AgdaSpace{}%
\AgdaFunction{ℰ}\AgdaSpace{}%
\AgdaKeyword{using}\AgdaSpace{}%
\AgdaSymbol{(}\AgdaSpace{}%
\AgdaOperator{\AgdaFunction{𝔽[\AgdaUnderscore{}]}}\AgdaSpace{}%
\AgdaSymbol{)}\<%
\end{code}

\hypertarget{the-natural-epimorphism-from-ux1d47b-x-to-ux1d53d-x}{%
\paragraph{The natural epimorphism from 𝑻 X to 𝔽{[} X
{]}}\label{the-natural-epimorphism-from-ux1d47b-x-to-ux1d53d-x}}

Next we define an epimorphism from \texttt{𝑻\ X} onto the relatively
free algebra \texttt{𝔽{[}\ X\ {]}}. Of course, the kernel of this
epimorphism will be the congruence of \texttt{𝑻\ X} defined by
identities modeled by (\texttt{S\ 𝒦}, hence) \texttt{𝒦}.

\begin{code}%
\>[0]\<%
\\
%
\>[1]\AgdaOperator{\AgdaFunction{epi𝔽[\AgdaUnderscore{}]}}\AgdaSpace{}%
\AgdaSymbol{:}\AgdaSpace{}%
\AgdaSymbol{(}\AgdaBound{X}\AgdaSpace{}%
\AgdaSymbol{:}\AgdaSpace{}%
\AgdaPrimitive{Type}\AgdaSpace{}%
\AgdaBound{χ}\AgdaSymbol{)}\AgdaSpace{}%
\AgdaSymbol{→}\AgdaSpace{}%
\AgdaFunction{epi}\AgdaSpace{}%
\AgdaSymbol{(}\AgdaFunction{𝑻}\AgdaSpace{}%
\AgdaBound{X}\AgdaSymbol{)}\AgdaSpace{}%
\AgdaOperator{\AgdaFunction{𝔽[}}\AgdaSpace{}%
\AgdaBound{X}\AgdaSpace{}%
\AgdaOperator{\AgdaFunction{]}}\<%
\\
%
\>[1]\AgdaOperator{\AgdaFunction{epi𝔽[}}\AgdaSpace{}%
\AgdaBound{X}\AgdaSpace{}%
\AgdaOperator{\AgdaFunction{]}}\AgdaSpace{}%
\AgdaSymbol{=}\AgdaSpace{}%
\AgdaFunction{h}\AgdaSpace{}%
\AgdaOperator{\AgdaInductiveConstructor{,}}\AgdaSpace{}%
\AgdaFunction{hepi}\<%
\\
\>[1][@{}l@{\AgdaIndent{0}}]%
\>[2]\AgdaKeyword{where}\<%
\\
%
\>[2]\AgdaKeyword{open}\AgdaSpace{}%
\AgdaModule{Algebra}\AgdaSpace{}%
\AgdaSymbol{(}\AgdaFunction{𝑻}\AgdaSpace{}%
\AgdaBound{X}\AgdaSymbol{)}\AgdaSpace{}%
\AgdaKeyword{using}\AgdaSpace{}%
\AgdaSymbol{()}\AgdaSpace{}%
\AgdaKeyword{renaming}\AgdaSpace{}%
\AgdaSymbol{(}\AgdaField{Domain}\AgdaSpace{}%
\AgdaSymbol{to}\AgdaSpace{}%
\AgdaField{TX}\AgdaSymbol{)}\<%
\\
%
\>[2]\AgdaKeyword{open}\AgdaSpace{}%
\AgdaModule{Setoid}\AgdaSpace{}%
\AgdaFunction{TX}\AgdaSpace{}%
\AgdaKeyword{using}\AgdaSpace{}%
\AgdaSymbol{()}\AgdaSpace{}%
\AgdaKeyword{renaming}\AgdaSpace{}%
\AgdaSymbol{(}\AgdaSpace{}%
\AgdaOperator{\AgdaField{\AgdaUnderscore{}≈\AgdaUnderscore{}}}\AgdaSpace{}%
\AgdaSymbol{to}\AgdaSpace{}%
\AgdaOperator{\AgdaField{\AgdaUnderscore{}≈₀\AgdaUnderscore{}}}\AgdaSpace{}%
\AgdaSymbol{;}\AgdaSpace{}%
\AgdaFunction{refl}\AgdaSpace{}%
\AgdaSymbol{to}\AgdaSpace{}%
\AgdaFunction{refl₀}\AgdaSpace{}%
\AgdaSymbol{)}\<%
\\
%
\>[2]\AgdaKeyword{open}\AgdaSpace{}%
\AgdaModule{Algebra}\AgdaSpace{}%
\AgdaOperator{\AgdaFunction{𝔽[}}\AgdaSpace{}%
\AgdaBound{X}\AgdaSpace{}%
\AgdaOperator{\AgdaFunction{]}}\AgdaSpace{}%
\AgdaKeyword{using}\AgdaSpace{}%
\AgdaSymbol{()}\AgdaSpace{}%
\AgdaKeyword{renaming}\AgdaSpace{}%
\AgdaSymbol{(}\AgdaSpace{}%
\AgdaField{Domain}\AgdaSpace{}%
\AgdaSymbol{to}\AgdaSpace{}%
\AgdaField{F}\AgdaSpace{}%
\AgdaSymbol{)}\<%
\\
%
\>[2]\AgdaKeyword{open}\AgdaSpace{}%
\AgdaModule{Setoid}\AgdaSpace{}%
\AgdaFunction{F}\AgdaSpace{}%
\AgdaKeyword{using}\AgdaSpace{}%
\AgdaSymbol{()}\AgdaSpace{}%
\AgdaKeyword{renaming}\AgdaSpace{}%
\AgdaSymbol{(}\AgdaSpace{}%
\AgdaOperator{\AgdaField{\AgdaUnderscore{}≈\AgdaUnderscore{}}}%
\>[41]\AgdaSymbol{to}\AgdaSpace{}%
\AgdaOperator{\AgdaField{\AgdaUnderscore{}≈₁\AgdaUnderscore{}}}\AgdaSpace{}%
\AgdaSymbol{;}\AgdaSpace{}%
\AgdaFunction{refl}\AgdaSpace{}%
\AgdaSymbol{to}\AgdaSpace{}%
\AgdaFunction{refl₁}\AgdaSpace{}%
\AgdaSymbol{)}\<%
\\
%
\>[2]\AgdaKeyword{open}\AgdaSpace{}%
\AgdaOperator{\AgdaModule{\AgdaUnderscore{}≐\AgdaUnderscore{}}}\<%
\\
%
\\[\AgdaEmptyExtraSkip]%
%
\>[2]\AgdaFunction{c}\AgdaSpace{}%
\AgdaSymbol{:}\AgdaSpace{}%
\AgdaSymbol{∀}\AgdaSpace{}%
\AgdaSymbol{\{}\AgdaBound{x}\AgdaSpace{}%
\AgdaBound{y}\AgdaSymbol{\}}\AgdaSpace{}%
\AgdaSymbol{→}\AgdaSpace{}%
\AgdaBound{x}\AgdaSpace{}%
\AgdaOperator{\AgdaFunction{≈₀}}\AgdaSpace{}%
\AgdaBound{y}\AgdaSpace{}%
\AgdaSymbol{→}\AgdaSpace{}%
\AgdaBound{x}\AgdaSpace{}%
\AgdaOperator{\AgdaFunction{≈₁}}\AgdaSpace{}%
\AgdaBound{y}\<%
\\
%
\>[2]\AgdaFunction{c}\AgdaSpace{}%
\AgdaSymbol{(}\AgdaInductiveConstructor{rfl}\AgdaSpace{}%
\AgdaSymbol{\{}\AgdaBound{x}\AgdaSymbol{\}\{}\AgdaBound{y}\AgdaSymbol{\}}\AgdaSpace{}%
\AgdaInductiveConstructor{≡.refl}\AgdaSymbol{)}\AgdaSpace{}%
\AgdaSymbol{=}\AgdaSpace{}%
\AgdaFunction{refl₁}\<%
\\
%
\>[2]\AgdaFunction{c}\AgdaSpace{}%
\AgdaSymbol{(}\AgdaInductiveConstructor{gnl}\AgdaSpace{}%
\AgdaSymbol{\{}\AgdaBound{f}\AgdaSymbol{\}\{}\AgdaBound{s}\AgdaSymbol{\}\{}\AgdaBound{t}\AgdaSymbol{\}}\AgdaSpace{}%
\AgdaBound{x}\AgdaSymbol{)}\AgdaSpace{}%
\AgdaSymbol{=}\AgdaSpace{}%
\AgdaField{cong}\AgdaSpace{}%
\AgdaSymbol{(}\AgdaField{Interp}\AgdaSpace{}%
\AgdaOperator{\AgdaFunction{𝔽[}}\AgdaSpace{}%
\AgdaBound{X}\AgdaSpace{}%
\AgdaOperator{\AgdaFunction{]}}\AgdaSymbol{)}\AgdaSpace{}%
\AgdaSymbol{(}\AgdaInductiveConstructor{≡.refl}\AgdaSpace{}%
\AgdaOperator{\AgdaInductiveConstructor{,}}\AgdaSpace{}%
\AgdaFunction{c}\AgdaSpace{}%
\AgdaOperator{\AgdaFunction{∘}}\AgdaSpace{}%
\AgdaBound{x}\AgdaSymbol{)}\<%
\\
%
\\[\AgdaEmptyExtraSkip]%
%
\>[2]\AgdaFunction{h}\AgdaSpace{}%
\AgdaSymbol{:}\AgdaSpace{}%
\AgdaFunction{TX}\AgdaSpace{}%
\AgdaOperator{\AgdaRecord{⟶}}\AgdaSpace{}%
\AgdaFunction{F}\<%
\\
%
\>[2]\AgdaFunction{h}\AgdaSpace{}%
\AgdaSymbol{=}\AgdaSpace{}%
\AgdaKeyword{record}\AgdaSpace{}%
\AgdaSymbol{\{}\AgdaSpace{}%
\AgdaField{f}\AgdaSpace{}%
\AgdaSymbol{=}\AgdaSpace{}%
\AgdaFunction{id}\AgdaSpace{}%
\AgdaSymbol{;}\AgdaSpace{}%
\AgdaField{cong}\AgdaSpace{}%
\AgdaSymbol{=}\AgdaSpace{}%
\AgdaFunction{c}\AgdaSpace{}%
\AgdaSymbol{\}}\<%
\\
%
\\[\AgdaEmptyExtraSkip]%
%
\>[2]\AgdaFunction{hepi}\AgdaSpace{}%
\AgdaSymbol{:}\AgdaSpace{}%
\AgdaRecord{IsEpi}\AgdaSpace{}%
\AgdaSymbol{(}\AgdaFunction{𝑻}\AgdaSpace{}%
\AgdaBound{X}\AgdaSymbol{)}\AgdaSpace{}%
\AgdaOperator{\AgdaFunction{𝔽[}}\AgdaSpace{}%
\AgdaBound{X}\AgdaSpace{}%
\AgdaOperator{\AgdaFunction{]}}\AgdaSpace{}%
\AgdaFunction{h}\<%
\\
%
\>[2]\AgdaField{compatible}\AgdaSpace{}%
\AgdaSymbol{(}\AgdaField{isHom}\AgdaSpace{}%
\AgdaFunction{hepi}\AgdaSymbol{)}\AgdaSpace{}%
\AgdaSymbol{=}\AgdaSpace{}%
\AgdaField{cong}\AgdaSpace{}%
\AgdaFunction{h}\AgdaSpace{}%
\AgdaFunction{refl₀}\<%
\\
%
\>[2]\AgdaField{isSurjective}\AgdaSpace{}%
\AgdaFunction{hepi}\AgdaSpace{}%
\AgdaSymbol{\{}\AgdaBound{y}\AgdaSymbol{\}}\AgdaSpace{}%
\AgdaSymbol{=}\AgdaSpace{}%
\AgdaInductiveConstructor{eq}\AgdaSpace{}%
\AgdaBound{y}\AgdaSpace{}%
\AgdaFunction{refl₁}\<%
\\
%
\\[\AgdaEmptyExtraSkip]%
%
\>[1]\AgdaOperator{\AgdaFunction{hom𝔽[\AgdaUnderscore{}]}}\AgdaSpace{}%
\AgdaSymbol{:}\AgdaSpace{}%
\AgdaSymbol{(}\AgdaBound{X}\AgdaSpace{}%
\AgdaSymbol{:}\AgdaSpace{}%
\AgdaPrimitive{Type}\AgdaSpace{}%
\AgdaBound{χ}\AgdaSymbol{)}\AgdaSpace{}%
\AgdaSymbol{→}\AgdaSpace{}%
\AgdaFunction{hom}\AgdaSpace{}%
\AgdaSymbol{(}\AgdaFunction{𝑻}\AgdaSpace{}%
\AgdaBound{X}\AgdaSymbol{)}\AgdaSpace{}%
\AgdaOperator{\AgdaFunction{𝔽[}}\AgdaSpace{}%
\AgdaBound{X}\AgdaSpace{}%
\AgdaOperator{\AgdaFunction{]}}\<%
\\
%
\>[1]\AgdaOperator{\AgdaFunction{hom𝔽[}}\AgdaSpace{}%
\AgdaBound{X}\AgdaSpace{}%
\AgdaOperator{\AgdaFunction{]}}\AgdaSpace{}%
\AgdaSymbol{=}\AgdaSpace{}%
\AgdaFunction{IsEpi.HomReduct}\AgdaSpace{}%
\AgdaOperator{\AgdaFunction{∥}}\AgdaSpace{}%
\AgdaOperator{\AgdaFunction{epi𝔽[}}\AgdaSpace{}%
\AgdaBound{X}\AgdaSpace{}%
\AgdaOperator{\AgdaFunction{]}}\AgdaSpace{}%
\AgdaOperator{\AgdaFunction{∥}}\<%
\\
%
\\[\AgdaEmptyExtraSkip]%
%
\>[1]\AgdaOperator{\AgdaFunction{hom𝔽[\AgdaUnderscore{}]-is-epic}}\AgdaSpace{}%
\AgdaSymbol{:}\AgdaSpace{}%
\AgdaSymbol{(}\AgdaBound{X}\AgdaSpace{}%
\AgdaSymbol{:}\AgdaSpace{}%
\AgdaPrimitive{Type}\AgdaSpace{}%
\AgdaBound{χ}\AgdaSymbol{)}\AgdaSpace{}%
\AgdaSymbol{→}\AgdaSpace{}%
\AgdaFunction{IsSurjective}\AgdaSpace{}%
\AgdaOperator{\AgdaFunction{∣}}\AgdaSpace{}%
\AgdaOperator{\AgdaFunction{hom𝔽[}}\AgdaSpace{}%
\AgdaBound{X}\AgdaSpace{}%
\AgdaOperator{\AgdaFunction{]}}\AgdaSpace{}%
\AgdaOperator{\AgdaFunction{∣}}\<%
\\
%
\>[1]\AgdaOperator{\AgdaFunction{hom𝔽[}}\AgdaSpace{}%
\AgdaBound{X}\AgdaSpace{}%
\AgdaOperator{\AgdaFunction{]-is-epic}}\AgdaSpace{}%
\AgdaSymbol{=}\AgdaSpace{}%
\AgdaField{IsEpi.isSurjective}\AgdaSpace{}%
\AgdaSymbol{(}\AgdaField{snd}\AgdaSpace{}%
\AgdaSymbol{(}\AgdaOperator{\AgdaFunction{epi𝔽[}}\AgdaSpace{}%
\AgdaBound{X}\AgdaSpace{}%
\AgdaOperator{\AgdaFunction{]}}\AgdaSymbol{))}\<%
\end{code}

\hypertarget{the-kernel-of-the-natural-epimorphism}{%
\paragraph{The kernel of the natural
epimorphism}\label{the-kernel-of-the-natural-epimorphism}}

\begin{code}%
\>[0]\<%
\\
%
\>[1]\AgdaFunction{class-models-kernel}\AgdaSpace{}%
\AgdaSymbol{:}\AgdaSpace{}%
\AgdaSymbol{∀\{}\AgdaBound{X}\AgdaSpace{}%
\AgdaBound{p}\AgdaSpace{}%
\AgdaBound{q}\AgdaSymbol{\}}\AgdaSpace{}%
\AgdaSymbol{→}\AgdaSpace{}%
\AgdaSymbol{(}\AgdaBound{p}\AgdaSpace{}%
\AgdaOperator{\AgdaInductiveConstructor{,}}\AgdaSpace{}%
\AgdaBound{q}\AgdaSymbol{)}\AgdaSpace{}%
\AgdaOperator{\AgdaFunction{∈}}\AgdaSpace{}%
\AgdaFunction{ker}\AgdaSpace{}%
\AgdaOperator{\AgdaFunction{∣}}\AgdaSpace{}%
\AgdaOperator{\AgdaFunction{hom𝔽[}}\AgdaSpace{}%
\AgdaBound{X}\AgdaSpace{}%
\AgdaOperator{\AgdaFunction{]}}\AgdaSpace{}%
\AgdaOperator{\AgdaFunction{∣}}\AgdaSpace{}%
\AgdaSymbol{→}\AgdaSpace{}%
\AgdaBound{𝒦}\AgdaSpace{}%
\AgdaOperator{\AgdaFunction{⊫}}\AgdaSpace{}%
\AgdaSymbol{(}\AgdaBound{p}\AgdaSpace{}%
\AgdaOperator{\AgdaInductiveConstructor{≈̇}}\AgdaSpace{}%
\AgdaBound{q}\AgdaSymbol{)}\<%
\\
%
\>[1]\AgdaFunction{class-models-kernel}\AgdaSpace{}%
\AgdaSymbol{\{}\AgdaArgument{X}\AgdaSpace{}%
\AgdaSymbol{=}\AgdaSpace{}%
\AgdaBound{X}\AgdaSymbol{\}\{}\AgdaBound{p}\AgdaSymbol{\}\{}\AgdaBound{q}\AgdaSymbol{\}}\AgdaSpace{}%
\AgdaBound{pKq}\AgdaSpace{}%
\AgdaSymbol{=}\AgdaSpace{}%
\AgdaOperator{\AgdaFunction{ℰ⊢[}}\AgdaSpace{}%
\AgdaBound{X}\AgdaSpace{}%
\AgdaOperator{\AgdaFunction{]▹Th𝒦}}\AgdaSpace{}%
\AgdaBound{pKq}\<%
\\
%
\\[\AgdaEmptyExtraSkip]%
%
\>[1]\AgdaFunction{kernel-in-theory}\AgdaSpace{}%
\AgdaSymbol{:}\AgdaSpace{}%
\AgdaSymbol{\{}\AgdaBound{X}\AgdaSpace{}%
\AgdaSymbol{:}\AgdaSpace{}%
\AgdaPrimitive{Type}\AgdaSpace{}%
\AgdaBound{χ}\AgdaSymbol{\}}\AgdaSpace{}%
\AgdaSymbol{→}\AgdaSpace{}%
\AgdaFunction{ker}\AgdaSpace{}%
\AgdaOperator{\AgdaFunction{∣}}\AgdaSpace{}%
\AgdaOperator{\AgdaFunction{hom𝔽[}}\AgdaSpace{}%
\AgdaBound{X}\AgdaSpace{}%
\AgdaOperator{\AgdaFunction{]}}\AgdaSpace{}%
\AgdaOperator{\AgdaFunction{∣}}\AgdaSpace{}%
\AgdaOperator{\AgdaFunction{⊆}}\AgdaSpace{}%
\AgdaFunction{Th}\AgdaSpace{}%
\AgdaSymbol{(}\AgdaFunction{V}\AgdaSpace{}%
\AgdaBound{ℓ}\AgdaSpace{}%
\AgdaFunction{ι}\AgdaSpace{}%
\AgdaBound{𝒦}\AgdaSymbol{)}\<%
\\
%
\>[1]\AgdaFunction{kernel-in-theory}\AgdaSpace{}%
\AgdaSymbol{\{}\AgdaArgument{X}\AgdaSpace{}%
\AgdaSymbol{=}\AgdaSpace{}%
\AgdaBound{X}\AgdaSymbol{\}}\AgdaSpace{}%
\AgdaSymbol{\{}\AgdaBound{p}\AgdaSpace{}%
\AgdaOperator{\AgdaInductiveConstructor{,}}\AgdaSpace{}%
\AgdaBound{q}\AgdaSymbol{\}}\AgdaSpace{}%
\AgdaBound{pKq}\AgdaSpace{}%
\AgdaBound{vkA}\AgdaSpace{}%
\AgdaBound{x}\AgdaSpace{}%
\AgdaSymbol{=}%
\>[6941I]\AgdaFunction{classIds-⊆-VIds}\AgdaSpace{}%
\AgdaSymbol{\{}\AgdaArgument{ℓ}\AgdaSpace{}%
\AgdaSymbol{=}\AgdaSpace{}%
\AgdaBound{ℓ}\AgdaSymbol{\}\{}\AgdaArgument{p}\AgdaSpace{}%
\AgdaSymbol{=}\AgdaSpace{}%
\AgdaBound{p}\AgdaSymbol{\}\{}\AgdaBound{q}\AgdaSymbol{\}}\<%
\\
\>[6941I][@{}l@{\AgdaIndent{0}}]%
\>[47]\AgdaSymbol{(}\AgdaFunction{class-models-kernel}\AgdaSpace{}%
\AgdaBound{pKq}\AgdaSymbol{)}\AgdaSpace{}%
\AgdaBound{vkA}\AgdaSpace{}%
\AgdaBound{x}\<%
\\
%
\\[\AgdaEmptyExtraSkip]%
%
\>[1]\AgdaKeyword{module}\AgdaSpace{}%
\AgdaModule{\AgdaUnderscore{}}\AgdaSpace{}%
\AgdaSymbol{\{}\AgdaBound{X}\AgdaSpace{}%
\AgdaSymbol{:}\AgdaSpace{}%
\AgdaPrimitive{Type}\AgdaSpace{}%
\AgdaBound{χ}\AgdaSymbol{\}}\AgdaSpace{}%
\AgdaSymbol{\{}\AgdaBound{𝑨}\AgdaSpace{}%
\AgdaSymbol{:}\AgdaSpace{}%
\AgdaRecord{Algebra}\AgdaSpace{}%
\AgdaBound{α}\AgdaSpace{}%
\AgdaBound{ρᵃ}\AgdaSymbol{\}\{}\AgdaBound{sA}\AgdaSpace{}%
\AgdaSymbol{:}\AgdaSpace{}%
\AgdaBound{𝑨}\AgdaSpace{}%
\AgdaOperator{\AgdaFunction{∈}}\AgdaSpace{}%
\AgdaFunction{S}\AgdaSpace{}%
\AgdaSymbol{\{}\AgdaArgument{β}\AgdaSpace{}%
\AgdaSymbol{=}\AgdaSpace{}%
\AgdaBound{α}\AgdaSymbol{\}\{}\AgdaBound{ρᵃ}\AgdaSymbol{\}}\AgdaSpace{}%
\AgdaBound{ℓ}\AgdaSpace{}%
\AgdaBound{𝒦}\AgdaSymbol{\}}\AgdaSpace{}%
\AgdaKeyword{where}\<%
\\
\>[1][@{}l@{\AgdaIndent{0}}]%
\>[2]\AgdaKeyword{open}\AgdaSpace{}%
\AgdaModule{Environment}\AgdaSpace{}%
\AgdaBound{𝑨}\AgdaSpace{}%
\AgdaKeyword{using}\AgdaSpace{}%
\AgdaSymbol{(}\AgdaSpace{}%
\AgdaFunction{Equal}\AgdaSpace{}%
\AgdaSymbol{)}\<%
\\
%
\>[2]\AgdaFunction{ker𝔽⊆Equal}\AgdaSpace{}%
\AgdaSymbol{:}\AgdaSpace{}%
\AgdaSymbol{∀\{}\AgdaBound{p}\AgdaSpace{}%
\AgdaBound{q}\AgdaSymbol{\}}\AgdaSpace{}%
\AgdaSymbol{→}\AgdaSpace{}%
\AgdaSymbol{(}\AgdaBound{p}\AgdaSpace{}%
\AgdaOperator{\AgdaInductiveConstructor{,}}\AgdaSpace{}%
\AgdaBound{q}\AgdaSymbol{)}\AgdaSpace{}%
\AgdaOperator{\AgdaFunction{∈}}\AgdaSpace{}%
\AgdaFunction{ker}\AgdaSpace{}%
\AgdaOperator{\AgdaFunction{∣}}\AgdaSpace{}%
\AgdaOperator{\AgdaFunction{hom𝔽[}}\AgdaSpace{}%
\AgdaBound{X}\AgdaSpace{}%
\AgdaOperator{\AgdaFunction{]}}\AgdaSpace{}%
\AgdaOperator{\AgdaFunction{∣}}\AgdaSpace{}%
\AgdaSymbol{→}\AgdaSpace{}%
\AgdaFunction{Equal}\AgdaSpace{}%
\AgdaBound{p}\AgdaSpace{}%
\AgdaBound{q}\<%
\\
%
\>[2]\AgdaFunction{ker𝔽⊆Equal}\AgdaSymbol{\{}\AgdaArgument{p}\AgdaSpace{}%
\AgdaSymbol{=}\AgdaSpace{}%
\AgdaBound{p}\AgdaSymbol{\}\{}\AgdaBound{q}\AgdaSymbol{\}}\AgdaSpace{}%
\AgdaBound{x}\AgdaSpace{}%
\AgdaSymbol{=}\AgdaSpace{}%
\AgdaFunction{S-id1}\AgdaSymbol{\{}\AgdaArgument{ℓ}\AgdaSpace{}%
\AgdaSymbol{=}\AgdaSpace{}%
\AgdaBound{ℓ}\AgdaSymbol{\}\{}\AgdaArgument{p}\AgdaSpace{}%
\AgdaSymbol{=}\AgdaSpace{}%
\AgdaBound{p}\AgdaSymbol{\}\{}\AgdaBound{q}\AgdaSymbol{\}}\AgdaSpace{}%
\AgdaSymbol{(}\AgdaOperator{\AgdaFunction{ℰ⊢[}}\AgdaSpace{}%
\AgdaBound{X}\AgdaSpace{}%
\AgdaOperator{\AgdaFunction{]▹Th𝒦}}\AgdaSpace{}%
\AgdaBound{x}\AgdaSymbol{)}\AgdaSpace{}%
\AgdaBound{𝑨}\AgdaSpace{}%
\AgdaBound{sA}\<%
\\
%
\\[\AgdaEmptyExtraSkip]%
%
\>[1]\AgdaFunction{𝒦⊫→ℰ⊢}\AgdaSpace{}%
\AgdaSymbol{:}\AgdaSpace{}%
\AgdaSymbol{\{}\AgdaBound{X}\AgdaSpace{}%
\AgdaSymbol{:}\AgdaSpace{}%
\AgdaPrimitive{Type}\AgdaSpace{}%
\AgdaBound{χ}\AgdaSymbol{\}}\AgdaSpace{}%
\AgdaSymbol{→}\AgdaSpace{}%
\AgdaSymbol{∀\{}\AgdaBound{p}\AgdaSpace{}%
\AgdaBound{q}\AgdaSymbol{\}}\AgdaSpace{}%
\AgdaSymbol{→}\AgdaSpace{}%
\AgdaBound{𝒦}\AgdaSpace{}%
\AgdaOperator{\AgdaFunction{⊫}}\AgdaSpace{}%
\AgdaSymbol{(}\AgdaBound{p}\AgdaSpace{}%
\AgdaOperator{\AgdaInductiveConstructor{≈̇}}\AgdaSpace{}%
\AgdaBound{q}\AgdaSymbol{)}\AgdaSpace{}%
\AgdaSymbol{→}\AgdaSpace{}%
\AgdaFunction{ℰ}\AgdaSpace{}%
\AgdaOperator{\AgdaDatatype{⊢}}\AgdaSpace{}%
\AgdaBound{X}\AgdaSpace{}%
\AgdaOperator{\AgdaDatatype{▹}}\AgdaSpace{}%
\AgdaBound{p}\AgdaSpace{}%
\AgdaOperator{\AgdaDatatype{≈}}\AgdaSpace{}%
\AgdaBound{q}\<%
\\
%
\>[1]\AgdaFunction{𝒦⊫→ℰ⊢}\AgdaSpace{}%
\AgdaSymbol{\{}\AgdaArgument{p}\AgdaSpace{}%
\AgdaSymbol{=}\AgdaSpace{}%
\AgdaBound{p}\AgdaSymbol{\}}\AgdaSpace{}%
\AgdaSymbol{\{}\AgdaBound{q}\AgdaSymbol{\}}\AgdaSpace{}%
\AgdaBound{pKq}\AgdaSpace{}%
\AgdaSymbol{=}\AgdaSpace{}%
\AgdaInductiveConstructor{hyp}\AgdaSpace{}%
\AgdaSymbol{((}\AgdaBound{p}\AgdaSpace{}%
\AgdaOperator{\AgdaInductiveConstructor{≈̇}}\AgdaSpace{}%
\AgdaBound{q}\AgdaSymbol{)}\AgdaSpace{}%
\AgdaOperator{\AgdaInductiveConstructor{,}}\AgdaSpace{}%
\AgdaBound{pKq}\AgdaSymbol{)}\AgdaSpace{}%
\AgdaKeyword{where}\AgdaSpace{}%
\AgdaKeyword{open}\AgdaSpace{}%
\AgdaOperator{\AgdaModule{\AgdaUnderscore{}⊢\AgdaUnderscore{}▹\AgdaUnderscore{}≈\AgdaUnderscore{}}}\AgdaSpace{}%
\AgdaKeyword{using}\AgdaSpace{}%
\AgdaSymbol{(}\AgdaInductiveConstructor{hyp}\AgdaSymbol{)}\<%
\end{code}

\hypertarget{the-universal-property}{%
\paragraph{The universal property}\label{the-universal-property}}

\begin{code}%
\>[0]\<%
\\
\>[0]\AgdaKeyword{module}\AgdaSpace{}%
\AgdaModule{\AgdaUnderscore{}}%
\>[10]\AgdaSymbol{\{}\AgdaBound{𝑨}\AgdaSpace{}%
\AgdaSymbol{:}\AgdaSpace{}%
\AgdaRecord{Algebra}\AgdaSpace{}%
\AgdaSymbol{(}\AgdaGeneralizable{α}\AgdaSpace{}%
\AgdaOperator{\AgdaPrimitive{⊔}}\AgdaSpace{}%
\AgdaGeneralizable{ρᵃ}\AgdaSpace{}%
\AgdaOperator{\AgdaPrimitive{⊔}}\AgdaSpace{}%
\AgdaGeneralizable{ℓ}\AgdaSymbol{)}\AgdaSpace{}%
\AgdaSymbol{(}\AgdaGeneralizable{α}\AgdaSpace{}%
\AgdaOperator{\AgdaPrimitive{⊔}}\AgdaSpace{}%
\AgdaGeneralizable{ρᵃ}\AgdaSpace{}%
\AgdaOperator{\AgdaPrimitive{⊔}}\AgdaSpace{}%
\AgdaGeneralizable{ℓ}\AgdaSymbol{)\}}\AgdaSpace{}%
\AgdaSymbol{\{}\AgdaBound{𝒦}\AgdaSpace{}%
\AgdaSymbol{:}\AgdaSpace{}%
\AgdaFunction{Pred}\AgdaSymbol{(}\AgdaRecord{Algebra}\AgdaSpace{}%
\AgdaGeneralizable{α}\AgdaSpace{}%
\AgdaGeneralizable{ρᵃ}\AgdaSymbol{)}\AgdaSpace{}%
\AgdaSymbol{(}\AgdaGeneralizable{α}\AgdaSpace{}%
\AgdaOperator{\AgdaPrimitive{⊔}}\AgdaSpace{}%
\AgdaGeneralizable{ρᵃ}\AgdaSpace{}%
\AgdaOperator{\AgdaPrimitive{⊔}}\AgdaSpace{}%
\AgdaFunction{ov}\AgdaSpace{}%
\AgdaGeneralizable{ℓ}\AgdaSymbol{)\}}\AgdaSpace{}%
\AgdaKeyword{where}\<%
\\
\>[0][@{}l@{\AgdaIndent{0}}]%
\>[1]\AgdaKeyword{private}\AgdaSpace{}%
\AgdaFunction{ι}\AgdaSpace{}%
\AgdaSymbol{=}\AgdaSpace{}%
\AgdaFunction{ov}\AgdaSymbol{(}\AgdaBound{α}\AgdaSpace{}%
\AgdaOperator{\AgdaPrimitive{⊔}}\AgdaSpace{}%
\AgdaBound{ρᵃ}\AgdaSpace{}%
\AgdaOperator{\AgdaPrimitive{⊔}}\AgdaSpace{}%
\AgdaBound{ℓ}\AgdaSymbol{)}\<%
\\
%
\\[\AgdaEmptyExtraSkip]%
%
\>[1]\AgdaKeyword{open}\AgdaSpace{}%
\AgdaModule{FreeHom}\AgdaSpace{}%
\AgdaSymbol{\{}\AgdaArgument{ℓ}\AgdaSpace{}%
\AgdaSymbol{=}\AgdaSpace{}%
\AgdaBound{ℓ}\AgdaSymbol{\}(}\AgdaBound{α}\AgdaSpace{}%
\AgdaOperator{\AgdaPrimitive{⊔}}\AgdaSpace{}%
\AgdaBound{ρᵃ}\AgdaSpace{}%
\AgdaOperator{\AgdaPrimitive{⊔}}\AgdaSpace{}%
\AgdaBound{ℓ}\AgdaSymbol{)}\AgdaSpace{}%
\AgdaSymbol{\{}\AgdaBound{𝒦}\AgdaSymbol{\}}\<%
\\
%
\>[1]\AgdaKeyword{open}\AgdaSpace{}%
\AgdaModule{FreeAlgebra}\AgdaSpace{}%
\AgdaSymbol{\{}\AgdaArgument{ι}\AgdaSpace{}%
\AgdaSymbol{=}\AgdaSpace{}%
\AgdaFunction{ι}\AgdaSymbol{\}\{}\AgdaArgument{I}\AgdaSpace{}%
\AgdaSymbol{=}\AgdaSpace{}%
\AgdaFunction{ℐ}\AgdaSymbol{\}}\AgdaSpace{}%
\AgdaFunction{ℰ}%
\>[36]\AgdaKeyword{using}\AgdaSpace{}%
\AgdaSymbol{(}\AgdaSpace{}%
\AgdaOperator{\AgdaFunction{𝔽[\AgdaUnderscore{}]}}\AgdaSpace{}%
\AgdaSymbol{)}\<%
\\
%
\>[1]\AgdaKeyword{open}\AgdaSpace{}%
\AgdaModule{Setoid}\AgdaSpace{}%
\AgdaOperator{\AgdaFunction{𝔻[}}\AgdaSpace{}%
\AgdaBound{𝑨}\AgdaSpace{}%
\AgdaOperator{\AgdaFunction{]}}%
\>[36]\AgdaKeyword{using}\AgdaSpace{}%
\AgdaSymbol{(}\AgdaSpace{}%
\AgdaFunction{trans}\AgdaSpace{}%
\AgdaSymbol{;}\AgdaSpace{}%
\AgdaFunction{sym}\AgdaSpace{}%
\AgdaSymbol{;}\AgdaSpace{}%
\AgdaFunction{refl}\AgdaSpace{}%
\AgdaSymbol{)}\AgdaSpace{}%
\AgdaKeyword{renaming}\AgdaSpace{}%
\AgdaSymbol{(}\AgdaSpace{}%
\AgdaField{Carrier}\AgdaSpace{}%
\AgdaSymbol{to}\AgdaSpace{}%
\AgdaField{A}\AgdaSpace{}%
\AgdaSymbol{)}\<%
\\
%
\\[\AgdaEmptyExtraSkip]%
%
\\[\AgdaEmptyExtraSkip]%
%
\>[1]\AgdaFunction{𝔽-ModTh-epi}\AgdaSpace{}%
\AgdaSymbol{:}\AgdaSpace{}%
\AgdaBound{𝑨}\AgdaSpace{}%
\AgdaOperator{\AgdaFunction{∈}}\AgdaSpace{}%
\AgdaFunction{Mod}\AgdaSpace{}%
\AgdaSymbol{(}\AgdaFunction{Th}\AgdaSpace{}%
\AgdaSymbol{(}\AgdaFunction{V}\AgdaSpace{}%
\AgdaBound{ℓ}\AgdaSpace{}%
\AgdaFunction{ι}\AgdaSpace{}%
\AgdaBound{𝒦}\AgdaSymbol{))}\<%
\\
\>[1][@{}l@{\AgdaIndent{0}}]%
\>[2]\AgdaSymbol{→}%
\>[15]\AgdaFunction{epi}\AgdaSpace{}%
\AgdaOperator{\AgdaFunction{𝔽[}}\AgdaSpace{}%
\AgdaFunction{A}\AgdaSpace{}%
\AgdaOperator{\AgdaFunction{]}}\AgdaSpace{}%
\AgdaBound{𝑨}\<%
\\
%
\>[1]\AgdaFunction{𝔽-ModTh-epi}\AgdaSpace{}%
\AgdaBound{A∈ModThK}\AgdaSpace{}%
\AgdaSymbol{=}\AgdaSpace{}%
\AgdaFunction{φ}\AgdaSpace{}%
\AgdaOperator{\AgdaInductiveConstructor{,}}\AgdaSpace{}%
\AgdaFunction{isEpi}\<%
\\
\>[1][@{}l@{\AgdaIndent{0}}]%
\>[2]\AgdaKeyword{where}\<%
\\
\>[2][@{}l@{\AgdaIndent{0}}]%
\>[3]\AgdaFunction{φ}\AgdaSpace{}%
\AgdaSymbol{:}\AgdaSpace{}%
\AgdaOperator{\AgdaFunction{𝔻[}}\AgdaSpace{}%
\AgdaOperator{\AgdaFunction{𝔽[}}\AgdaSpace{}%
\AgdaFunction{A}\AgdaSpace{}%
\AgdaOperator{\AgdaFunction{]}}\AgdaSpace{}%
\AgdaOperator{\AgdaFunction{]}}\AgdaSpace{}%
\AgdaOperator{\AgdaRecord{⟶}}\AgdaSpace{}%
\AgdaOperator{\AgdaFunction{𝔻[}}\AgdaSpace{}%
\AgdaBound{𝑨}\AgdaSpace{}%
\AgdaOperator{\AgdaFunction{]}}\<%
\\
%
\>[3]\AgdaOperator{\AgdaField{\AgdaUnderscore{}⟨\$⟩\AgdaUnderscore{}}}\AgdaSpace{}%
\AgdaFunction{φ}\AgdaSpace{}%
\AgdaSymbol{=}\AgdaSpace{}%
\AgdaFunction{free-lift}\AgdaSymbol{\{}\AgdaArgument{𝑨}\AgdaSpace{}%
\AgdaSymbol{=}\AgdaSpace{}%
\AgdaBound{𝑨}\AgdaSymbol{\}}\AgdaSpace{}%
\AgdaFunction{id}\<%
\\
%
\>[3]\AgdaField{cong}\AgdaSpace{}%
\AgdaFunction{φ}\AgdaSpace{}%
\AgdaSymbol{\{}\AgdaBound{p}\AgdaSymbol{\}}\AgdaSpace{}%
\AgdaSymbol{\{}\AgdaBound{q}\AgdaSymbol{\}}\AgdaSpace{}%
\AgdaBound{pq}%
\>[22]\AgdaSymbol{=}\AgdaSpace{}%
\AgdaFunction{trans}%
\>[31]\AgdaSymbol{(}\AgdaSpace{}%
\AgdaFunction{sym}\AgdaSpace{}%
\AgdaSymbol{(}\AgdaFunction{free-lift-interp}\AgdaSymbol{\{}\AgdaArgument{𝑨}\AgdaSpace{}%
\AgdaSymbol{=}\AgdaSpace{}%
\AgdaBound{𝑨}\AgdaSymbol{\}}\AgdaSpace{}%
\AgdaFunction{id}\AgdaSpace{}%
\AgdaBound{p}\AgdaSymbol{)}\AgdaSpace{}%
\AgdaSymbol{)}\<%
\\
%
\>[22]\AgdaSymbol{(}\AgdaSpace{}%
\AgdaFunction{trans}%
\>[31]\AgdaSymbol{(}\AgdaSpace{}%
\AgdaBound{A∈ModThK}\AgdaSymbol{\{}\AgdaArgument{p}\AgdaSpace{}%
\AgdaSymbol{=}\AgdaSpace{}%
\AgdaBound{p}\AgdaSymbol{\}\{}\AgdaBound{q}\AgdaSymbol{\}}\AgdaSpace{}%
\AgdaSymbol{(}\AgdaFunction{kernel-in-theory}\AgdaSpace{}%
\AgdaBound{pq}\AgdaSymbol{)}\AgdaSpace{}%
\AgdaFunction{id}\AgdaSpace{}%
\AgdaSymbol{)}\<%
\\
%
\>[31]\AgdaSymbol{(}\AgdaSpace{}%
\AgdaFunction{free-lift-interp}\AgdaSymbol{\{}\AgdaArgument{𝑨}\AgdaSpace{}%
\AgdaSymbol{=}\AgdaSpace{}%
\AgdaBound{𝑨}\AgdaSymbol{\}}\AgdaSpace{}%
\AgdaFunction{id}\AgdaSpace{}%
\AgdaBound{q}\AgdaSpace{}%
\AgdaSymbol{)}\AgdaSpace{}%
\AgdaSymbol{)}\<%
\\
%
\>[3]\AgdaFunction{isEpi}\AgdaSpace{}%
\AgdaSymbol{:}\AgdaSpace{}%
\AgdaRecord{IsEpi}\AgdaSpace{}%
\AgdaOperator{\AgdaFunction{𝔽[}}\AgdaSpace{}%
\AgdaFunction{A}\AgdaSpace{}%
\AgdaOperator{\AgdaFunction{]}}\AgdaSpace{}%
\AgdaBound{𝑨}\AgdaSpace{}%
\AgdaFunction{φ}\<%
\\
%
\>[3]\AgdaField{compatible}\AgdaSpace{}%
\AgdaSymbol{(}\AgdaField{isHom}\AgdaSpace{}%
\AgdaFunction{isEpi}\AgdaSymbol{)}\AgdaSpace{}%
\AgdaSymbol{=}\AgdaSpace{}%
\AgdaField{cong}\AgdaSpace{}%
\AgdaSymbol{(}\AgdaField{Interp}\AgdaSpace{}%
\AgdaBound{𝑨}\AgdaSymbol{)}\AgdaSpace{}%
\AgdaSymbol{(}\AgdaInductiveConstructor{≡.refl}\AgdaSpace{}%
\AgdaOperator{\AgdaInductiveConstructor{,}}\AgdaSpace{}%
\AgdaSymbol{(λ}\AgdaSpace{}%
\AgdaBound{\AgdaUnderscore{}}\AgdaSpace{}%
\AgdaSymbol{→}\AgdaSpace{}%
\AgdaFunction{refl}\AgdaSymbol{))}\<%
\\
%
\>[3]\AgdaField{isSurjective}\AgdaSpace{}%
\AgdaFunction{isEpi}\AgdaSpace{}%
\AgdaSymbol{\{}\AgdaBound{y}\AgdaSymbol{\}}\AgdaSpace{}%
\AgdaSymbol{=}\AgdaSpace{}%
\AgdaInductiveConstructor{eq}\AgdaSpace{}%
\AgdaSymbol{(}\AgdaInductiveConstructor{ℊ}\AgdaSpace{}%
\AgdaBound{y}\AgdaSymbol{)}\AgdaSpace{}%
\AgdaFunction{refl}\<%
\\
%
\\[\AgdaEmptyExtraSkip]%
%
\>[1]\AgdaFunction{𝔽-ModTh-epi-lift}\AgdaSpace{}%
\AgdaSymbol{:}\AgdaSpace{}%
\AgdaBound{𝑨}\AgdaSpace{}%
\AgdaOperator{\AgdaFunction{∈}}\AgdaSpace{}%
\AgdaFunction{Mod}\AgdaSpace{}%
\AgdaSymbol{(}\AgdaFunction{Th}\AgdaSpace{}%
\AgdaSymbol{(}\AgdaFunction{V}\AgdaSpace{}%
\AgdaBound{ℓ}\AgdaSpace{}%
\AgdaFunction{ι}\AgdaSpace{}%
\AgdaBound{𝒦}\AgdaSymbol{))}\AgdaSpace{}%
\AgdaSymbol{→}\AgdaSpace{}%
\AgdaFunction{epi}\AgdaSpace{}%
\AgdaOperator{\AgdaFunction{𝔽[}}\AgdaSpace{}%
\AgdaFunction{A}\AgdaSpace{}%
\AgdaOperator{\AgdaFunction{]}}\AgdaSpace{}%
\AgdaSymbol{(}\AgdaFunction{Lift-Alg}\AgdaSpace{}%
\AgdaBound{𝑨}\AgdaSpace{}%
\AgdaFunction{ι}\AgdaSpace{}%
\AgdaFunction{ι}\AgdaSymbol{)}\<%
\\
%
\>[1]\AgdaFunction{𝔽-ModTh-epi-lift}\AgdaSpace{}%
\AgdaBound{A∈ModThK}\AgdaSpace{}%
\AgdaSymbol{=}\AgdaSpace{}%
\AgdaFunction{∘-epi}\AgdaSpace{}%
\AgdaSymbol{(}\AgdaFunction{𝔽-ModTh-epi}\AgdaSpace{}%
\AgdaSymbol{(λ}\AgdaSpace{}%
\AgdaSymbol{\{}\AgdaBound{p}\AgdaSpace{}%
\AgdaBound{q}\AgdaSymbol{\}}\AgdaSpace{}%
\AgdaSymbol{→}\AgdaSpace{}%
\AgdaBound{A∈ModThK}\AgdaSymbol{\{}\AgdaArgument{p}\AgdaSpace{}%
\AgdaSymbol{=}\AgdaSpace{}%
\AgdaBound{p}\AgdaSymbol{\}\{}\AgdaBound{q}\AgdaSymbol{\}))}\AgdaSpace{}%
\AgdaFunction{ToLift-epi}\<%
\end{code}

\hypertarget{products-of-classes-of-algebras}{%
\subsection{Products of classes of
algebras}\label{products-of-classes-of-algebras}}

We want to pair each \texttt{(𝑨\ ,\ p)} (where p : 𝑨 ∈ S 𝒦) with an
environment \texttt{ρ\ :\ X\ →\ ∣\ 𝑨\ ∣} so that we can quantify over
all algebras \emph{and} all assignments of values in the domain
\texttt{∣\ 𝑨\ ∣} to variables in \texttt{X}.

\begin{code}%
\>[0]\<%
\\
\>[0]\AgdaKeyword{module}\AgdaSpace{}%
\AgdaModule{\AgdaUnderscore{}}\AgdaSpace{}%
\AgdaSymbol{(}\AgdaBound{𝒦}\AgdaSpace{}%
\AgdaSymbol{:}\AgdaSpace{}%
\AgdaFunction{Pred}\AgdaSymbol{(}\AgdaRecord{Algebra}\AgdaSpace{}%
\AgdaGeneralizable{α}\AgdaSpace{}%
\AgdaGeneralizable{ρᵃ}\AgdaSymbol{)}\AgdaSpace{}%
\AgdaSymbol{(}\AgdaGeneralizable{α}\AgdaSpace{}%
\AgdaOperator{\AgdaPrimitive{⊔}}\AgdaSpace{}%
\AgdaGeneralizable{ρᵃ}\AgdaSpace{}%
\AgdaOperator{\AgdaPrimitive{⊔}}\AgdaSpace{}%
\AgdaFunction{ov}\AgdaSpace{}%
\AgdaGeneralizable{ℓ}\AgdaSymbol{))\{}\AgdaBound{X}\AgdaSpace{}%
\AgdaSymbol{:}\AgdaSpace{}%
\AgdaPrimitive{Type}\AgdaSpace{}%
\AgdaSymbol{(}\AgdaGeneralizable{α}\AgdaSpace{}%
\AgdaOperator{\AgdaPrimitive{⊔}}\AgdaSpace{}%
\AgdaGeneralizable{ρᵃ}\AgdaSpace{}%
\AgdaOperator{\AgdaPrimitive{⊔}}\AgdaSpace{}%
\AgdaGeneralizable{ℓ}\AgdaSymbol{)\}}\AgdaSpace{}%
\AgdaKeyword{where}\<%
\\
\>[0][@{}l@{\AgdaIndent{0}}]%
\>[1]\AgdaKeyword{private}\AgdaSpace{}%
\AgdaFunction{ι}\AgdaSpace{}%
\AgdaSymbol{=}\AgdaSpace{}%
\AgdaFunction{ov}\AgdaSymbol{(}\AgdaBound{α}\AgdaSpace{}%
\AgdaOperator{\AgdaPrimitive{⊔}}\AgdaSpace{}%
\AgdaBound{ρᵃ}\AgdaSpace{}%
\AgdaOperator{\AgdaPrimitive{⊔}}\AgdaSpace{}%
\AgdaBound{ℓ}\AgdaSymbol{)}\<%
\\
%
\>[1]\AgdaKeyword{open}\AgdaSpace{}%
\AgdaModule{FreeHom}\AgdaSpace{}%
\AgdaSymbol{\{}\AgdaArgument{ℓ}\AgdaSpace{}%
\AgdaSymbol{=}\AgdaSpace{}%
\AgdaBound{ℓ}\AgdaSymbol{\}}\AgdaSpace{}%
\AgdaSymbol{(}\AgdaBound{α}\AgdaSpace{}%
\AgdaOperator{\AgdaPrimitive{⊔}}\AgdaSpace{}%
\AgdaBound{ρᵃ}\AgdaSpace{}%
\AgdaOperator{\AgdaPrimitive{⊔}}\AgdaSpace{}%
\AgdaBound{ℓ}\AgdaSymbol{)\{}\AgdaBound{𝒦}\AgdaSymbol{\}}\<%
\\
%
\>[1]\AgdaKeyword{open}\AgdaSpace{}%
\AgdaModule{FreeAlgebra}\AgdaSpace{}%
\AgdaSymbol{\{}\AgdaArgument{ι}\AgdaSpace{}%
\AgdaSymbol{=}\AgdaSpace{}%
\AgdaFunction{ι}\AgdaSymbol{\}\{}\AgdaArgument{I}\AgdaSpace{}%
\AgdaSymbol{=}\AgdaSpace{}%
\AgdaFunction{ℐ}\AgdaSymbol{\}}\AgdaSpace{}%
\AgdaFunction{ℰ}\AgdaSpace{}%
\AgdaKeyword{using}\AgdaSpace{}%
\AgdaSymbol{(}\AgdaSpace{}%
\AgdaOperator{\AgdaFunction{𝔽[\AgdaUnderscore{}]}}\AgdaSpace{}%
\AgdaSymbol{)}\<%
\\
%
\>[1]\AgdaKeyword{open}\AgdaSpace{}%
\AgdaModule{Environment}%
\>[20]\AgdaKeyword{using}\AgdaSpace{}%
\AgdaSymbol{(}\AgdaSpace{}%
\AgdaFunction{Env}\AgdaSpace{}%
\AgdaSymbol{)}\<%
\\
%
\\[\AgdaEmptyExtraSkip]%
%
\>[1]\AgdaFunction{ℑ⁺}\AgdaSpace{}%
\AgdaSymbol{:}\AgdaSpace{}%
\AgdaPrimitive{Type}\AgdaSpace{}%
\AgdaFunction{ι}\<%
\\
%
\>[1]\AgdaFunction{ℑ⁺}\AgdaSpace{}%
\AgdaSymbol{=}\AgdaSpace{}%
\AgdaFunction{Σ[}\AgdaSpace{}%
\AgdaBound{𝑨}\AgdaSpace{}%
\AgdaFunction{∈}\AgdaSpace{}%
\AgdaSymbol{(}\AgdaRecord{Algebra}\AgdaSpace{}%
\AgdaBound{α}\AgdaSpace{}%
\AgdaBound{ρᵃ}\AgdaSymbol{)}\AgdaSpace{}%
\AgdaFunction{]}\AgdaSpace{}%
\AgdaSymbol{(}\AgdaBound{𝑨}\AgdaSpace{}%
\AgdaOperator{\AgdaFunction{∈}}\AgdaSpace{}%
\AgdaFunction{S}\AgdaSpace{}%
\AgdaBound{ℓ}\AgdaSpace{}%
\AgdaBound{𝒦}\AgdaSymbol{)}\AgdaSpace{}%
\AgdaOperator{\AgdaFunction{×}}\AgdaSpace{}%
\AgdaSymbol{(}\AgdaField{Carrier}\AgdaSpace{}%
\AgdaSymbol{(}\AgdaFunction{Env}\AgdaSpace{}%
\AgdaBound{𝑨}\AgdaSpace{}%
\AgdaBound{X}\AgdaSymbol{))}\<%
\\
%
\\[\AgdaEmptyExtraSkip]%
%
\>[1]\AgdaFunction{𝔄⁺}\AgdaSpace{}%
\AgdaSymbol{:}\AgdaSpace{}%
\AgdaFunction{ℑ⁺}\AgdaSpace{}%
\AgdaSymbol{→}\AgdaSpace{}%
\AgdaRecord{Algebra}\AgdaSpace{}%
\AgdaBound{α}\AgdaSpace{}%
\AgdaBound{ρᵃ}\<%
\\
%
\>[1]\AgdaFunction{𝔄⁺}\AgdaSpace{}%
\AgdaBound{i}\AgdaSpace{}%
\AgdaSymbol{=}\AgdaSpace{}%
\AgdaOperator{\AgdaFunction{∣}}\AgdaSpace{}%
\AgdaBound{i}\AgdaSpace{}%
\AgdaOperator{\AgdaFunction{∣}}\<%
\\
%
\\[\AgdaEmptyExtraSkip]%
%
\>[1]\AgdaFunction{ℭ}\AgdaSpace{}%
\AgdaSymbol{:}\AgdaSpace{}%
\AgdaRecord{Algebra}\AgdaSpace{}%
\AgdaFunction{ι}\AgdaSpace{}%
\AgdaFunction{ι}\<%
\\
%
\>[1]\AgdaFunction{ℭ}\AgdaSpace{}%
\AgdaSymbol{=}\AgdaSpace{}%
\AgdaFunction{⨅}\AgdaSpace{}%
\AgdaFunction{𝔄⁺}\<%
\\
\>[0]\<%
\end{code}

Next we define a useful type, \texttt{skEqual}, which we use to
represent a term identity \texttt{p\ ≈\ q} for any given
\texttt{i\ =\ (𝑨\ ,\ sA\ ,\ ρ)} (where \texttt{𝑨} is an algebra,
\texttt{sA\ :\ 𝑨\ ∈\ S\ 𝒦} is a proof that \texttt{𝑨} belongs to
\texttt{S\ 𝒦}, and \texttt{ρ} is a mapping from \texttt{X} to the domain
of \texttt{𝑨}). Then we prove \texttt{AllEqual⊆ker𝔽} which asserts that
if the identity \texttt{p\ ≈\ q} holds in all \texttt{𝑨\ ∈\ S\ 𝒦} (for
all environments), then \texttt{p\ ≈\ q} holds in the relatively free
algebra \texttt{𝔽{[}\ X\ {]}}; equivalently, the pair \texttt{(p\ ,\ q)}
belongs to the kernel of the natural homomorphism from \texttt{𝑻\ X}
onto \texttt{𝔽{[}\ X\ {]}}. We will use this fact below to prove that
there is a monomorphism from \texttt{𝔽{[}\ X\ {]}} into \texttt{ℭ}, and
thus \texttt{𝔽{[}\ X\ {]}} is a subalgebra of ℭ, so belongs to
\texttt{S\ (P\ 𝒦)}.

\begin{code}%
\>[0]\<%
\\
\>[0][@{}l@{\AgdaIndent{1}}]%
\>[1]\AgdaFunction{skEqual}\AgdaSpace{}%
\AgdaSymbol{:}\AgdaSpace{}%
\AgdaSymbol{(}\AgdaBound{i}\AgdaSpace{}%
\AgdaSymbol{:}\AgdaSpace{}%
\AgdaFunction{ℑ⁺}\AgdaSymbol{)}\AgdaSpace{}%
\AgdaSymbol{→}\AgdaSpace{}%
\AgdaSymbol{∀\{}\AgdaBound{p}\AgdaSpace{}%
\AgdaBound{q}\AgdaSymbol{\}}\AgdaSpace{}%
\AgdaSymbol{→}\AgdaSpace{}%
\AgdaPrimitive{Type}\AgdaSpace{}%
\AgdaBound{ρᵃ}\<%
\\
%
\>[1]\AgdaFunction{skEqual}\AgdaSpace{}%
\AgdaBound{i}\AgdaSpace{}%
\AgdaSymbol{\{}\AgdaBound{p}\AgdaSymbol{\}\{}\AgdaBound{q}\AgdaSymbol{\}}\AgdaSpace{}%
\AgdaSymbol{=}\AgdaSpace{}%
\AgdaOperator{\AgdaFunction{⟦}}\AgdaSpace{}%
\AgdaBound{p}\AgdaSpace{}%
\AgdaOperator{\AgdaFunction{⟧}}\AgdaSpace{}%
\AgdaOperator{\AgdaField{⟨\$⟩}}\AgdaSpace{}%
\AgdaField{snd}\AgdaSpace{}%
\AgdaOperator{\AgdaFunction{∥}}\AgdaSpace{}%
\AgdaBound{i}\AgdaSpace{}%
\AgdaOperator{\AgdaFunction{∥}}\AgdaSpace{}%
\AgdaOperator{\AgdaFunction{≈}}\AgdaSpace{}%
\AgdaOperator{\AgdaFunction{⟦}}\AgdaSpace{}%
\AgdaBound{q}\AgdaSpace{}%
\AgdaOperator{\AgdaFunction{⟧}}\AgdaSpace{}%
\AgdaOperator{\AgdaField{⟨\$⟩}}\AgdaSpace{}%
\AgdaField{snd}\AgdaSpace{}%
\AgdaOperator{\AgdaFunction{∥}}\AgdaSpace{}%
\AgdaBound{i}\AgdaSpace{}%
\AgdaOperator{\AgdaFunction{∥}}\<%
\\
\>[1][@{}l@{\AgdaIndent{0}}]%
\>[2]\AgdaKeyword{where}\<%
\\
%
\>[2]\AgdaKeyword{open}\AgdaSpace{}%
\AgdaModule{Setoid}\AgdaSpace{}%
\AgdaOperator{\AgdaFunction{𝔻[}}\AgdaSpace{}%
\AgdaFunction{𝔄⁺}\AgdaSpace{}%
\AgdaBound{i}\AgdaSpace{}%
\AgdaOperator{\AgdaFunction{]}}\AgdaSpace{}%
\AgdaKeyword{using}\AgdaSpace{}%
\AgdaSymbol{(}\AgdaSpace{}%
\AgdaOperator{\AgdaField{\AgdaUnderscore{}≈\AgdaUnderscore{}}}\AgdaSpace{}%
\AgdaSymbol{)}\<%
\\
%
\>[2]\AgdaKeyword{open}\AgdaSpace{}%
\AgdaModule{Environment}\AgdaSpace{}%
\AgdaSymbol{(}\AgdaFunction{𝔄⁺}\AgdaSpace{}%
\AgdaBound{i}\AgdaSymbol{)}\AgdaSpace{}%
\AgdaKeyword{using}\AgdaSpace{}%
\AgdaSymbol{(}\AgdaSpace{}%
\AgdaOperator{\AgdaFunction{⟦\AgdaUnderscore{}⟧}}\AgdaSpace{}%
\AgdaSymbol{)}\<%
\\
%
\\[\AgdaEmptyExtraSkip]%
%
\>[1]\AgdaFunction{AllEqual⊆ker𝔽}\AgdaSpace{}%
\AgdaSymbol{:}\AgdaSpace{}%
\AgdaSymbol{∀}\AgdaSpace{}%
\AgdaSymbol{\{}\AgdaBound{p}\AgdaSpace{}%
\AgdaBound{q}\AgdaSymbol{\}}\AgdaSpace{}%
\AgdaSymbol{→}\AgdaSpace{}%
\AgdaSymbol{(∀}\AgdaSpace{}%
\AgdaBound{i}\AgdaSpace{}%
\AgdaSymbol{→}\AgdaSpace{}%
\AgdaFunction{skEqual}\AgdaSpace{}%
\AgdaBound{i}\AgdaSpace{}%
\AgdaSymbol{\{}\AgdaBound{p}\AgdaSymbol{\}\{}\AgdaBound{q}\AgdaSymbol{\})}\AgdaSpace{}%
\AgdaSymbol{→}\AgdaSpace{}%
\AgdaSymbol{(}\AgdaBound{p}\AgdaSpace{}%
\AgdaOperator{\AgdaInductiveConstructor{,}}\AgdaSpace{}%
\AgdaBound{q}\AgdaSymbol{)}\AgdaSpace{}%
\AgdaOperator{\AgdaFunction{∈}}\AgdaSpace{}%
\AgdaFunction{ker}\AgdaSpace{}%
\AgdaOperator{\AgdaFunction{∣}}\AgdaSpace{}%
\AgdaOperator{\AgdaFunction{hom𝔽[}}\AgdaSpace{}%
\AgdaBound{X}\AgdaSpace{}%
\AgdaOperator{\AgdaFunction{]}}\AgdaSpace{}%
\AgdaOperator{\AgdaFunction{∣}}\<%
\\
%
\>[1]\AgdaFunction{AllEqual⊆ker𝔽}\AgdaSpace{}%
\AgdaSymbol{\{}\AgdaBound{p}\AgdaSymbol{\}}\AgdaSpace{}%
\AgdaSymbol{\{}\AgdaBound{q}\AgdaSymbol{\}}\AgdaSpace{}%
\AgdaBound{x}\AgdaSpace{}%
\AgdaSymbol{=}\AgdaSpace{}%
\AgdaFunction{Goal}\<%
\\
\>[1][@{}l@{\AgdaIndent{0}}]%
\>[2]\AgdaKeyword{where}\<%
\\
%
\>[2]\AgdaKeyword{open}\AgdaSpace{}%
\AgdaModule{Setoid}\AgdaSpace{}%
\AgdaOperator{\AgdaFunction{𝔻[}}\AgdaSpace{}%
\AgdaOperator{\AgdaFunction{𝔽[}}\AgdaSpace{}%
\AgdaBound{X}\AgdaSpace{}%
\AgdaOperator{\AgdaFunction{]}}\AgdaSpace{}%
\AgdaOperator{\AgdaFunction{]}}\AgdaSpace{}%
\AgdaKeyword{using}\AgdaSpace{}%
\AgdaSymbol{(}\AgdaSpace{}%
\AgdaOperator{\AgdaField{\AgdaUnderscore{}≈\AgdaUnderscore{}}}\AgdaSpace{}%
\AgdaSymbol{)}\<%
\\
%
\>[2]\AgdaFunction{S𝒦⊫pq}\AgdaSpace{}%
\AgdaSymbol{:}\AgdaSpace{}%
\AgdaFunction{S}\AgdaSymbol{\{}\AgdaArgument{β}\AgdaSpace{}%
\AgdaSymbol{=}\AgdaSpace{}%
\AgdaBound{α}\AgdaSymbol{\}\{}\AgdaBound{ρᵃ}\AgdaSymbol{\}}\AgdaSpace{}%
\AgdaBound{ℓ}\AgdaSpace{}%
\AgdaBound{𝒦}\AgdaSpace{}%
\AgdaOperator{\AgdaFunction{⊫}}\AgdaSpace{}%
\AgdaSymbol{(}\AgdaBound{p}\AgdaSpace{}%
\AgdaOperator{\AgdaInductiveConstructor{≈̇}}\AgdaSpace{}%
\AgdaBound{q}\AgdaSymbol{)}\<%
\\
%
\>[2]\AgdaFunction{S𝒦⊫pq}\AgdaSpace{}%
\AgdaBound{𝑨}\AgdaSpace{}%
\AgdaBound{sA}\AgdaSpace{}%
\AgdaBound{ρ}\AgdaSpace{}%
\AgdaSymbol{=}\AgdaSpace{}%
\AgdaBound{x}\AgdaSpace{}%
\AgdaSymbol{(}\AgdaBound{𝑨}\AgdaSpace{}%
\AgdaOperator{\AgdaInductiveConstructor{,}}\AgdaSpace{}%
\AgdaBound{sA}\AgdaSpace{}%
\AgdaOperator{\AgdaInductiveConstructor{,}}\AgdaSpace{}%
\AgdaBound{ρ}\AgdaSymbol{)}\<%
\\
%
\>[2]\AgdaFunction{Goal}\AgdaSpace{}%
\AgdaSymbol{:}\AgdaSpace{}%
\AgdaBound{p}\AgdaSpace{}%
\AgdaOperator{\AgdaFunction{≈}}\AgdaSpace{}%
\AgdaBound{q}\<%
\\
%
\>[2]\AgdaFunction{Goal}\AgdaSpace{}%
\AgdaSymbol{=}\AgdaSpace{}%
\AgdaFunction{𝒦⊫→ℰ⊢}\AgdaSpace{}%
\AgdaSymbol{(}\AgdaFunction{S-id2}\AgdaSymbol{\{}\AgdaArgument{ℓ}\AgdaSpace{}%
\AgdaSymbol{=}\AgdaSpace{}%
\AgdaBound{ℓ}\AgdaSymbol{\}\{}\AgdaArgument{p}\AgdaSpace{}%
\AgdaSymbol{=}\AgdaSpace{}%
\AgdaBound{p}\AgdaSymbol{\}\{}\AgdaBound{q}\AgdaSymbol{\}}\AgdaSpace{}%
\AgdaFunction{S𝒦⊫pq}\AgdaSymbol{)}\<%
\\
%
\\[\AgdaEmptyExtraSkip]%
%
\>[1]\AgdaFunction{homℭ}\AgdaSpace{}%
\AgdaSymbol{:}\AgdaSpace{}%
\AgdaFunction{hom}\AgdaSpace{}%
\AgdaSymbol{(}\AgdaFunction{𝑻}\AgdaSpace{}%
\AgdaBound{X}\AgdaSymbol{)}\AgdaSpace{}%
\AgdaFunction{ℭ}\<%
\\
%
\>[1]\AgdaFunction{homℭ}\AgdaSpace{}%
\AgdaSymbol{=}\AgdaSpace{}%
\AgdaFunction{⨅-hom-co}\AgdaSpace{}%
\AgdaFunction{𝔄⁺}\AgdaSpace{}%
\AgdaFunction{h}\<%
\\
\>[1][@{}l@{\AgdaIndent{0}}]%
\>[2]\AgdaKeyword{where}\<%
\\
%
\>[2]\AgdaFunction{h}\AgdaSpace{}%
\AgdaSymbol{:}\AgdaSpace{}%
\AgdaSymbol{∀}\AgdaSpace{}%
\AgdaBound{i}\AgdaSpace{}%
\AgdaSymbol{→}\AgdaSpace{}%
\AgdaFunction{hom}\AgdaSpace{}%
\AgdaSymbol{(}\AgdaFunction{𝑻}\AgdaSpace{}%
\AgdaBound{X}\AgdaSymbol{)}\AgdaSpace{}%
\AgdaSymbol{(}\AgdaFunction{𝔄⁺}\AgdaSpace{}%
\AgdaBound{i}\AgdaSymbol{)}\<%
\\
%
\>[2]\AgdaFunction{h}\AgdaSpace{}%
\AgdaBound{i}\AgdaSpace{}%
\AgdaSymbol{=}\AgdaSpace{}%
\AgdaFunction{lift-hom}\AgdaSpace{}%
\AgdaSymbol{(}\AgdaField{snd}\AgdaSpace{}%
\AgdaOperator{\AgdaFunction{∥}}\AgdaSpace{}%
\AgdaBound{i}\AgdaSpace{}%
\AgdaOperator{\AgdaFunction{∥}}\AgdaSymbol{)}\<%
\\
%
\\[\AgdaEmptyExtraSkip]%
%
\>[1]\AgdaFunction{ker𝔽⊆kerℭ}\AgdaSpace{}%
\AgdaSymbol{:}\AgdaSpace{}%
\AgdaFunction{ker}\AgdaSpace{}%
\AgdaOperator{\AgdaFunction{∣}}\AgdaSpace{}%
\AgdaOperator{\AgdaFunction{hom𝔽[}}\AgdaSpace{}%
\AgdaBound{X}\AgdaSpace{}%
\AgdaOperator{\AgdaFunction{]}}\AgdaSpace{}%
\AgdaOperator{\AgdaFunction{∣}}\AgdaSpace{}%
\AgdaOperator{\AgdaFunction{⊆}}\AgdaSpace{}%
\AgdaFunction{ker}\AgdaSpace{}%
\AgdaOperator{\AgdaFunction{∣}}\AgdaSpace{}%
\AgdaFunction{homℭ}\AgdaSpace{}%
\AgdaOperator{\AgdaFunction{∣}}\<%
\\
%
\>[1]\AgdaFunction{ker𝔽⊆kerℭ}\AgdaSpace{}%
\AgdaSymbol{\{}\AgdaBound{p}\AgdaSpace{}%
\AgdaOperator{\AgdaInductiveConstructor{,}}\AgdaSpace{}%
\AgdaBound{q}\AgdaSymbol{\}}\AgdaSpace{}%
\AgdaBound{pKq}\AgdaSpace{}%
\AgdaSymbol{(}\AgdaBound{𝑨}\AgdaSpace{}%
\AgdaOperator{\AgdaInductiveConstructor{,}}\AgdaSpace{}%
\AgdaBound{sA}\AgdaSpace{}%
\AgdaOperator{\AgdaInductiveConstructor{,}}\AgdaSpace{}%
\AgdaBound{ρ}\AgdaSymbol{)}\AgdaSpace{}%
\AgdaSymbol{=}\AgdaSpace{}%
\AgdaFunction{Goal}\<%
\\
\>[1][@{}l@{\AgdaIndent{0}}]%
\>[2]\AgdaKeyword{where}\<%
\\
%
\>[2]\AgdaKeyword{open}\AgdaSpace{}%
\AgdaModule{Setoid}\AgdaSpace{}%
\AgdaOperator{\AgdaFunction{𝔻[}}\AgdaSpace{}%
\AgdaBound{𝑨}\AgdaSpace{}%
\AgdaOperator{\AgdaFunction{]}}\AgdaSpace{}%
\AgdaKeyword{using}\AgdaSpace{}%
\AgdaSymbol{(}\AgdaSpace{}%
\AgdaOperator{\AgdaField{\AgdaUnderscore{}≈\AgdaUnderscore{}}}\AgdaSpace{}%
\AgdaSymbol{;}\AgdaSpace{}%
\AgdaFunction{sym}\AgdaSpace{}%
\AgdaSymbol{;}\AgdaSpace{}%
\AgdaFunction{trans}\AgdaSpace{}%
\AgdaSymbol{)}\<%
\\
%
\>[2]\AgdaKeyword{open}\AgdaSpace{}%
\AgdaModule{Environment}\AgdaSpace{}%
\AgdaBound{𝑨}\AgdaSpace{}%
\AgdaKeyword{using}\AgdaSpace{}%
\AgdaSymbol{(}\AgdaSpace{}%
\AgdaOperator{\AgdaFunction{⟦\AgdaUnderscore{}⟧}}\AgdaSpace{}%
\AgdaSymbol{)}\<%
\\
%
\>[2]\AgdaFunction{fl}\AgdaSpace{}%
\AgdaSymbol{:}\AgdaSpace{}%
\AgdaSymbol{∀}\AgdaSpace{}%
\AgdaBound{t}\AgdaSpace{}%
\AgdaSymbol{→}\AgdaSpace{}%
\AgdaOperator{\AgdaFunction{⟦}}\AgdaSpace{}%
\AgdaBound{t}\AgdaSpace{}%
\AgdaOperator{\AgdaFunction{⟧}}\AgdaSpace{}%
\AgdaOperator{\AgdaField{⟨\$⟩}}\AgdaSpace{}%
\AgdaBound{ρ}\AgdaSpace{}%
\AgdaOperator{\AgdaFunction{≈}}\AgdaSpace{}%
\AgdaFunction{free-lift}\AgdaSpace{}%
\AgdaBound{ρ}\AgdaSpace{}%
\AgdaBound{t}\<%
\\
%
\>[2]\AgdaFunction{fl}\AgdaSpace{}%
\AgdaBound{t}\AgdaSpace{}%
\AgdaSymbol{=}\AgdaSpace{}%
\AgdaFunction{free-lift-interp}\AgdaSpace{}%
\AgdaSymbol{\{}\AgdaArgument{𝑨}\AgdaSpace{}%
\AgdaSymbol{=}\AgdaSpace{}%
\AgdaBound{𝑨}\AgdaSymbol{\}}\AgdaSpace{}%
\AgdaBound{ρ}\AgdaSpace{}%
\AgdaBound{t}\<%
\\
%
\>[2]\AgdaFunction{subgoal}\AgdaSpace{}%
\AgdaSymbol{:}\AgdaSpace{}%
\AgdaOperator{\AgdaFunction{⟦}}\AgdaSpace{}%
\AgdaBound{p}\AgdaSpace{}%
\AgdaOperator{\AgdaFunction{⟧}}\AgdaSpace{}%
\AgdaOperator{\AgdaField{⟨\$⟩}}\AgdaSpace{}%
\AgdaBound{ρ}\AgdaSpace{}%
\AgdaOperator{\AgdaFunction{≈}}\AgdaSpace{}%
\AgdaOperator{\AgdaFunction{⟦}}\AgdaSpace{}%
\AgdaBound{q}\AgdaSpace{}%
\AgdaOperator{\AgdaFunction{⟧}}\AgdaSpace{}%
\AgdaOperator{\AgdaField{⟨\$⟩}}\AgdaSpace{}%
\AgdaBound{ρ}\<%
\\
%
\>[2]\AgdaFunction{subgoal}\AgdaSpace{}%
\AgdaSymbol{=}\AgdaSpace{}%
\AgdaFunction{ker𝔽⊆Equal}\AgdaSymbol{\{}\AgdaArgument{𝑨}\AgdaSpace{}%
\AgdaSymbol{=}\AgdaSpace{}%
\AgdaBound{𝑨}\AgdaSymbol{\}\{}\AgdaBound{sA}\AgdaSymbol{\}}\AgdaSpace{}%
\AgdaBound{pKq}\AgdaSpace{}%
\AgdaBound{ρ}\<%
\\
%
\>[2]\AgdaFunction{Goal}\AgdaSpace{}%
\AgdaSymbol{:}\AgdaSpace{}%
\AgdaSymbol{(}\AgdaFunction{free-lift}\AgdaSymbol{\{}\AgdaArgument{𝑨}\AgdaSpace{}%
\AgdaSymbol{=}\AgdaSpace{}%
\AgdaBound{𝑨}\AgdaSymbol{\}}\AgdaSpace{}%
\AgdaBound{ρ}\AgdaSpace{}%
\AgdaBound{p}\AgdaSymbol{)}\AgdaSpace{}%
\AgdaOperator{\AgdaFunction{≈}}\AgdaSpace{}%
\AgdaSymbol{(}\AgdaFunction{free-lift}\AgdaSymbol{\{}\AgdaArgument{𝑨}\AgdaSpace{}%
\AgdaSymbol{=}\AgdaSpace{}%
\AgdaBound{𝑨}\AgdaSymbol{\}}\AgdaSpace{}%
\AgdaBound{ρ}\AgdaSpace{}%
\AgdaBound{q}\AgdaSymbol{)}\<%
\\
%
\>[2]\AgdaFunction{Goal}\AgdaSpace{}%
\AgdaSymbol{=}\AgdaSpace{}%
\AgdaFunction{trans}\AgdaSpace{}%
\AgdaSymbol{(}\AgdaFunction{sym}\AgdaSpace{}%
\AgdaSymbol{(}\AgdaFunction{fl}\AgdaSpace{}%
\AgdaBound{p}\AgdaSymbol{))}\AgdaSpace{}%
\AgdaSymbol{(}\AgdaFunction{trans}\AgdaSpace{}%
\AgdaFunction{subgoal}\AgdaSpace{}%
\AgdaSymbol{(}\AgdaFunction{fl}\AgdaSpace{}%
\AgdaBound{q}\AgdaSymbol{))}\<%
\\
%
\\[\AgdaEmptyExtraSkip]%
%
\\[\AgdaEmptyExtraSkip]%
%
\>[1]\AgdaFunction{hom𝔽ℭ}\AgdaSpace{}%
\AgdaSymbol{:}\AgdaSpace{}%
\AgdaFunction{hom}\AgdaSpace{}%
\AgdaOperator{\AgdaFunction{𝔽[}}\AgdaSpace{}%
\AgdaBound{X}\AgdaSpace{}%
\AgdaOperator{\AgdaFunction{]}}\AgdaSpace{}%
\AgdaFunction{ℭ}\<%
\\
%
\>[1]\AgdaFunction{hom𝔽ℭ}\AgdaSpace{}%
\AgdaSymbol{=}\AgdaSpace{}%
\AgdaOperator{\AgdaFunction{∣}}\AgdaSpace{}%
\AgdaFunction{HomFactor}\AgdaSpace{}%
\AgdaFunction{ℭ}\AgdaSpace{}%
\AgdaFunction{homℭ}\AgdaSpace{}%
\AgdaOperator{\AgdaFunction{hom𝔽[}}\AgdaSpace{}%
\AgdaBound{X}\AgdaSpace{}%
\AgdaOperator{\AgdaFunction{]}}\AgdaSpace{}%
\AgdaFunction{ker𝔽⊆kerℭ}\AgdaSpace{}%
\AgdaOperator{\AgdaFunction{hom𝔽[}}\AgdaSpace{}%
\AgdaBound{X}\AgdaSpace{}%
\AgdaOperator{\AgdaFunction{]-is-epic}}\AgdaSpace{}%
\AgdaOperator{\AgdaFunction{∣}}\<%
\\
%
\\[\AgdaEmptyExtraSkip]%
%
\>[1]\AgdaFunction{kerℭ⊆ker𝔽}\AgdaSpace{}%
\AgdaSymbol{:}\AgdaSpace{}%
\AgdaSymbol{∀\{}\AgdaBound{p}\AgdaSpace{}%
\AgdaBound{q}\AgdaSymbol{\}}\AgdaSpace{}%
\AgdaSymbol{→}\AgdaSpace{}%
\AgdaSymbol{(}\AgdaBound{p}\AgdaSpace{}%
\AgdaOperator{\AgdaInductiveConstructor{,}}\AgdaSpace{}%
\AgdaBound{q}\AgdaSymbol{)}\AgdaSpace{}%
\AgdaOperator{\AgdaFunction{∈}}\AgdaSpace{}%
\AgdaFunction{ker}\AgdaSpace{}%
\AgdaOperator{\AgdaFunction{∣}}\AgdaSpace{}%
\AgdaFunction{homℭ}\AgdaSpace{}%
\AgdaOperator{\AgdaFunction{∣}}\AgdaSpace{}%
\AgdaSymbol{→}\AgdaSpace{}%
\AgdaSymbol{(}\AgdaBound{p}\AgdaSpace{}%
\AgdaOperator{\AgdaInductiveConstructor{,}}\AgdaSpace{}%
\AgdaBound{q}\AgdaSymbol{)}\AgdaSpace{}%
\AgdaOperator{\AgdaFunction{∈}}\AgdaSpace{}%
\AgdaFunction{ker}\AgdaSpace{}%
\AgdaOperator{\AgdaFunction{∣}}\AgdaSpace{}%
\AgdaOperator{\AgdaFunction{hom𝔽[}}\AgdaSpace{}%
\AgdaBound{X}\AgdaSpace{}%
\AgdaOperator{\AgdaFunction{]}}\AgdaSpace{}%
\AgdaOperator{\AgdaFunction{∣}}\<%
\\
%
\>[1]\AgdaFunction{kerℭ⊆ker𝔽}\AgdaSpace{}%
\AgdaSymbol{\{}\AgdaBound{p}\AgdaSymbol{\}\{}\AgdaBound{q}\AgdaSymbol{\}}\AgdaSpace{}%
\AgdaBound{pKq}\AgdaSpace{}%
\AgdaSymbol{=}\AgdaSpace{}%
\AgdaFunction{E⊢pq}\<%
\\
\>[1][@{}l@{\AgdaIndent{0}}]%
\>[2]\AgdaKeyword{where}\<%
\\
%
\>[2]\AgdaFunction{pqEqual}\AgdaSpace{}%
\AgdaSymbol{:}\AgdaSpace{}%
\AgdaSymbol{∀}\AgdaSpace{}%
\AgdaBound{i}\AgdaSpace{}%
\AgdaSymbol{→}\AgdaSpace{}%
\AgdaFunction{skEqual}\AgdaSpace{}%
\AgdaBound{i}\AgdaSpace{}%
\AgdaSymbol{\{}\AgdaBound{p}\AgdaSymbol{\}\{}\AgdaBound{q}\AgdaSymbol{\}}\<%
\\
%
\>[2]\AgdaFunction{pqEqual}\AgdaSpace{}%
\AgdaBound{i}\AgdaSpace{}%
\AgdaSymbol{=}\AgdaSpace{}%
\AgdaFunction{goal}\<%
\\
\>[2][@{}l@{\AgdaIndent{0}}]%
\>[3]\AgdaKeyword{where}\<%
\\
%
\>[3]\AgdaKeyword{open}\AgdaSpace{}%
\AgdaModule{Environment}\AgdaSpace{}%
\AgdaSymbol{(}\AgdaFunction{𝔄⁺}\AgdaSpace{}%
\AgdaBound{i}\AgdaSymbol{)}\AgdaSpace{}%
\AgdaKeyword{using}\AgdaSpace{}%
\AgdaSymbol{(}\AgdaSpace{}%
\AgdaOperator{\AgdaFunction{⟦\AgdaUnderscore{}⟧}}\AgdaSpace{}%
\AgdaSymbol{)}\<%
\\
%
\>[3]\AgdaKeyword{open}\AgdaSpace{}%
\AgdaModule{Setoid}\AgdaSpace{}%
\AgdaOperator{\AgdaFunction{𝔻[}}\AgdaSpace{}%
\AgdaFunction{𝔄⁺}\AgdaSpace{}%
\AgdaBound{i}\AgdaSpace{}%
\AgdaOperator{\AgdaFunction{]}}\AgdaSpace{}%
\AgdaKeyword{using}\AgdaSpace{}%
\AgdaSymbol{(}\AgdaSpace{}%
\AgdaOperator{\AgdaField{\AgdaUnderscore{}≈\AgdaUnderscore{}}}\AgdaSpace{}%
\AgdaSymbol{;}\AgdaSpace{}%
\AgdaFunction{sym}\AgdaSpace{}%
\AgdaSymbol{;}\AgdaSpace{}%
\AgdaFunction{trans}\AgdaSpace{}%
\AgdaSymbol{)}\<%
\\
%
\>[3]\AgdaFunction{goal}\AgdaSpace{}%
\AgdaSymbol{:}\AgdaSpace{}%
\AgdaOperator{\AgdaFunction{⟦}}\AgdaSpace{}%
\AgdaBound{p}\AgdaSpace{}%
\AgdaOperator{\AgdaFunction{⟧}}\AgdaSpace{}%
\AgdaOperator{\AgdaField{⟨\$⟩}}\AgdaSpace{}%
\AgdaField{snd}\AgdaSpace{}%
\AgdaOperator{\AgdaFunction{∥}}\AgdaSpace{}%
\AgdaBound{i}\AgdaSpace{}%
\AgdaOperator{\AgdaFunction{∥}}\AgdaSpace{}%
\AgdaOperator{\AgdaFunction{≈}}\AgdaSpace{}%
\AgdaOperator{\AgdaFunction{⟦}}\AgdaSpace{}%
\AgdaBound{q}\AgdaSpace{}%
\AgdaOperator{\AgdaFunction{⟧}}\AgdaSpace{}%
\AgdaOperator{\AgdaField{⟨\$⟩}}\AgdaSpace{}%
\AgdaField{snd}\AgdaSpace{}%
\AgdaOperator{\AgdaFunction{∥}}\AgdaSpace{}%
\AgdaBound{i}\AgdaSpace{}%
\AgdaOperator{\AgdaFunction{∥}}\<%
\\
%
\>[3]\AgdaFunction{goal}\AgdaSpace{}%
\AgdaSymbol{=}%
\>[7659I]\AgdaFunction{trans}\AgdaSpace{}%
\AgdaSymbol{(}\AgdaFunction{free-lift-interp}\AgdaSymbol{\{}\AgdaArgument{𝑨}\AgdaSpace{}%
\AgdaSymbol{=}\AgdaSpace{}%
\AgdaOperator{\AgdaFunction{∣}}\AgdaSpace{}%
\AgdaBound{i}\AgdaSpace{}%
\AgdaOperator{\AgdaFunction{∣}}\AgdaSymbol{\}(}\AgdaField{snd}\AgdaSpace{}%
\AgdaOperator{\AgdaFunction{∥}}\AgdaSpace{}%
\AgdaBound{i}\AgdaSpace{}%
\AgdaOperator{\AgdaFunction{∥}}\AgdaSymbol{)}\AgdaSpace{}%
\AgdaBound{p}\AgdaSymbol{)}\<%
\\
\>[7659I][@{}l@{\AgdaIndent{0}}]%
\>[11]\AgdaSymbol{(}\AgdaFunction{trans}\AgdaSpace{}%
\AgdaSymbol{(}\AgdaBound{pKq}\AgdaSpace{}%
\AgdaBound{i}\AgdaSymbol{)(}\AgdaFunction{sym}\AgdaSpace{}%
\AgdaSymbol{(}\AgdaFunction{free-lift-interp}\AgdaSymbol{\{}\AgdaArgument{𝑨}\AgdaSpace{}%
\AgdaSymbol{=}\AgdaSpace{}%
\AgdaOperator{\AgdaFunction{∣}}\AgdaSpace{}%
\AgdaBound{i}\AgdaSpace{}%
\AgdaOperator{\AgdaFunction{∣}}\AgdaSymbol{\}}\AgdaSpace{}%
\AgdaSymbol{(}\AgdaField{snd}\AgdaSpace{}%
\AgdaOperator{\AgdaFunction{∥}}\AgdaSpace{}%
\AgdaBound{i}\AgdaSpace{}%
\AgdaOperator{\AgdaFunction{∥}}\AgdaSymbol{)}\AgdaSpace{}%
\AgdaBound{q}\AgdaSymbol{)))}\<%
\\
%
\>[2]\AgdaFunction{E⊢pq}\AgdaSpace{}%
\AgdaSymbol{:}\AgdaSpace{}%
\AgdaFunction{ℰ}\AgdaSpace{}%
\AgdaOperator{\AgdaDatatype{⊢}}\AgdaSpace{}%
\AgdaBound{X}\AgdaSpace{}%
\AgdaOperator{\AgdaDatatype{▹}}\AgdaSpace{}%
\AgdaBound{p}\AgdaSpace{}%
\AgdaOperator{\AgdaDatatype{≈}}\AgdaSpace{}%
\AgdaBound{q}\<%
\\
%
\>[2]\AgdaFunction{E⊢pq}\AgdaSpace{}%
\AgdaSymbol{=}\AgdaSpace{}%
\AgdaFunction{AllEqual⊆ker𝔽}\AgdaSpace{}%
\AgdaFunction{pqEqual}\<%
\\
%
\\[\AgdaEmptyExtraSkip]%
%
\>[1]\AgdaFunction{mon𝔽ℭ}\AgdaSpace{}%
\AgdaSymbol{:}\AgdaSpace{}%
\AgdaFunction{mon}\AgdaSpace{}%
\AgdaOperator{\AgdaFunction{𝔽[}}\AgdaSpace{}%
\AgdaBound{X}\AgdaSpace{}%
\AgdaOperator{\AgdaFunction{]}}\AgdaSpace{}%
\AgdaFunction{ℭ}\<%
\\
%
\>[1]\AgdaFunction{mon𝔽ℭ}\AgdaSpace{}%
\AgdaSymbol{=}\AgdaSpace{}%
\AgdaOperator{\AgdaFunction{∣}}\AgdaSpace{}%
\AgdaFunction{hom𝔽ℭ}\AgdaSpace{}%
\AgdaOperator{\AgdaFunction{∣}}\AgdaSpace{}%
\AgdaOperator{\AgdaInductiveConstructor{,}}\AgdaSpace{}%
\AgdaFunction{isMon}\<%
\\
\>[1][@{}l@{\AgdaIndent{0}}]%
\>[2]\AgdaKeyword{where}\<%
\\
%
\>[2]\AgdaFunction{isMon}\AgdaSpace{}%
\AgdaSymbol{:}\AgdaSpace{}%
\AgdaRecord{IsMon}\AgdaSpace{}%
\AgdaOperator{\AgdaFunction{𝔽[}}\AgdaSpace{}%
\AgdaBound{X}\AgdaSpace{}%
\AgdaOperator{\AgdaFunction{]}}\AgdaSpace{}%
\AgdaFunction{ℭ}\AgdaSpace{}%
\AgdaOperator{\AgdaFunction{∣}}\AgdaSpace{}%
\AgdaFunction{hom𝔽ℭ}\AgdaSpace{}%
\AgdaOperator{\AgdaFunction{∣}}\<%
\\
%
\>[2]\AgdaField{isHom}\AgdaSpace{}%
\AgdaFunction{isMon}\AgdaSpace{}%
\AgdaSymbol{=}\AgdaSpace{}%
\AgdaOperator{\AgdaFunction{∥}}\AgdaSpace{}%
\AgdaFunction{hom𝔽ℭ}\AgdaSpace{}%
\AgdaOperator{\AgdaFunction{∥}}\<%
\\
%
\>[2]\AgdaField{isInjective}\AgdaSpace{}%
\AgdaFunction{isMon}\AgdaSpace{}%
\AgdaSymbol{\{}\AgdaBound{p}\AgdaSymbol{\}}\AgdaSpace{}%
\AgdaSymbol{\{}\AgdaBound{q}\AgdaSymbol{\}}\AgdaSpace{}%
\AgdaBound{φpq}\AgdaSpace{}%
\AgdaSymbol{=}\AgdaSpace{}%
\AgdaFunction{kerℭ⊆ker𝔽}\AgdaSpace{}%
\AgdaBound{φpq}\<%
\\
\>[0]\<%
\end{code}

Now that we have proved the existence of a monomorphism from
\texttt{𝔽{[}\ X\ {]}} to \texttt{ℭ} we are in a position to prove that
\texttt{𝔽{[}\ X\ {]}} is a subalgebra of ℭ, so belongs to
\texttt{S\ (P\ 𝒦)}. In fact, we will show that \texttt{𝔽{[}\ X\ {]}} is
a subalgebra of the \emph{lift} of \texttt{ℭ}, denoted \texttt{ℓℭ}.

\begin{code}%
\>[0]\<%
\\
\>[0][@{}l@{\AgdaIndent{1}}]%
\>[1]\AgdaFunction{𝔽≤ℭ}\AgdaSpace{}%
\AgdaSymbol{:}\AgdaSpace{}%
\AgdaOperator{\AgdaFunction{𝔽[}}\AgdaSpace{}%
\AgdaBound{X}\AgdaSpace{}%
\AgdaOperator{\AgdaFunction{]}}\AgdaSpace{}%
\AgdaOperator{\AgdaFunction{≤}}\AgdaSpace{}%
\AgdaFunction{ℭ}\<%
\\
%
\>[1]\AgdaFunction{𝔽≤ℭ}\AgdaSpace{}%
\AgdaSymbol{=}\AgdaSpace{}%
\AgdaFunction{mon→≤}\AgdaSpace{}%
\AgdaFunction{mon𝔽ℭ}\<%
\\
%
\\[\AgdaEmptyExtraSkip]%
%
\>[1]\AgdaFunction{SP𝔽}\AgdaSpace{}%
\AgdaSymbol{:}\AgdaSpace{}%
\AgdaOperator{\AgdaFunction{𝔽[}}\AgdaSpace{}%
\AgdaBound{X}\AgdaSpace{}%
\AgdaOperator{\AgdaFunction{]}}\AgdaSpace{}%
\AgdaOperator{\AgdaFunction{∈}}\AgdaSpace{}%
\AgdaFunction{S}\AgdaSpace{}%
\AgdaFunction{ι}\AgdaSpace{}%
\AgdaSymbol{(}\AgdaFunction{P}\AgdaSpace{}%
\AgdaBound{ℓ}\AgdaSpace{}%
\AgdaFunction{ι}\AgdaSpace{}%
\AgdaBound{𝒦}\AgdaSymbol{)}\<%
\\
%
\>[1]\AgdaFunction{SP𝔽}\AgdaSpace{}%
\AgdaSymbol{=}\AgdaSpace{}%
\AgdaFunction{S-idem}\AgdaSpace{}%
\AgdaFunction{SSP𝔽}\<%
\\
\>[1][@{}l@{\AgdaIndent{0}}]%
\>[2]\AgdaKeyword{where}\<%
\\
%
\>[2]\AgdaFunction{PSℭ}\AgdaSpace{}%
\AgdaSymbol{:}\AgdaSpace{}%
\AgdaFunction{ℭ}\AgdaSpace{}%
\AgdaOperator{\AgdaFunction{∈}}\AgdaSpace{}%
\AgdaFunction{P}\AgdaSpace{}%
\AgdaSymbol{(}\AgdaBound{α}\AgdaSpace{}%
\AgdaOperator{\AgdaPrimitive{⊔}}\AgdaSpace{}%
\AgdaBound{ρᵃ}\AgdaSpace{}%
\AgdaOperator{\AgdaPrimitive{⊔}}\AgdaSpace{}%
\AgdaBound{ℓ}\AgdaSymbol{)}\AgdaSpace{}%
\AgdaFunction{ι}\AgdaSpace{}%
\AgdaSymbol{(}\AgdaFunction{S}\AgdaSpace{}%
\AgdaBound{ℓ}\AgdaSpace{}%
\AgdaBound{𝒦}\AgdaSymbol{)}\<%
\\
%
\>[2]\AgdaFunction{PSℭ}\AgdaSpace{}%
\AgdaSymbol{=}\AgdaSpace{}%
\AgdaFunction{ℑ⁺}\AgdaSpace{}%
\AgdaOperator{\AgdaInductiveConstructor{,}}\AgdaSpace{}%
\AgdaSymbol{(}\AgdaFunction{𝔄⁺}\AgdaSpace{}%
\AgdaOperator{\AgdaInductiveConstructor{,}}\AgdaSpace{}%
\AgdaSymbol{((λ}\AgdaSpace{}%
\AgdaBound{i}\AgdaSpace{}%
\AgdaSymbol{→}\AgdaSpace{}%
\AgdaField{fst}\AgdaSpace{}%
\AgdaOperator{\AgdaFunction{∥}}\AgdaSpace{}%
\AgdaBound{i}\AgdaSpace{}%
\AgdaOperator{\AgdaFunction{∥}}\AgdaSymbol{)}\AgdaSpace{}%
\AgdaOperator{\AgdaInductiveConstructor{,}}\AgdaSpace{}%
\AgdaFunction{≅-refl}\AgdaSymbol{))}\<%
\\
%
\>[2]\AgdaFunction{SPℭ}\AgdaSpace{}%
\AgdaSymbol{:}\AgdaSpace{}%
\AgdaFunction{ℭ}\AgdaSpace{}%
\AgdaOperator{\AgdaFunction{∈}}\AgdaSpace{}%
\AgdaFunction{S}\AgdaSpace{}%
\AgdaFunction{ι}\AgdaSpace{}%
\AgdaSymbol{(}\AgdaFunction{P}\AgdaSpace{}%
\AgdaBound{ℓ}\AgdaSpace{}%
\AgdaFunction{ι}\AgdaSpace{}%
\AgdaBound{𝒦}\AgdaSymbol{)}\<%
\\
%
\>[2]\AgdaFunction{SPℭ}\AgdaSpace{}%
\AgdaSymbol{=}\AgdaSpace{}%
\AgdaFunction{PS⊆SP}\AgdaSpace{}%
\AgdaSymbol{\{}\AgdaArgument{ℓ}\AgdaSpace{}%
\AgdaSymbol{=}\AgdaSpace{}%
\AgdaBound{ℓ}\AgdaSymbol{\}}\AgdaSpace{}%
\AgdaFunction{PSℭ}\<%
\\
%
\>[2]\AgdaFunction{SSP𝔽}\AgdaSpace{}%
\AgdaSymbol{:}\AgdaSpace{}%
\AgdaOperator{\AgdaFunction{𝔽[}}\AgdaSpace{}%
\AgdaBound{X}\AgdaSpace{}%
\AgdaOperator{\AgdaFunction{]}}\AgdaSpace{}%
\AgdaOperator{\AgdaFunction{∈}}\AgdaSpace{}%
\AgdaFunction{S}\AgdaSpace{}%
\AgdaFunction{ι}\AgdaSpace{}%
\AgdaSymbol{(}\AgdaFunction{S}\AgdaSpace{}%
\AgdaFunction{ι}\AgdaSpace{}%
\AgdaSymbol{(}\AgdaFunction{P}\AgdaSpace{}%
\AgdaBound{ℓ}\AgdaSpace{}%
\AgdaFunction{ι}\AgdaSpace{}%
\AgdaBound{𝒦}\AgdaSymbol{))}\<%
\\
%
\>[2]\AgdaFunction{SSP𝔽}\AgdaSpace{}%
\AgdaSymbol{=}\AgdaSpace{}%
\AgdaFunction{ℭ}\AgdaSpace{}%
\AgdaOperator{\AgdaInductiveConstructor{,}}\AgdaSpace{}%
\AgdaSymbol{(}\AgdaFunction{SPℭ}\AgdaSpace{}%
\AgdaOperator{\AgdaInductiveConstructor{,}}\AgdaSpace{}%
\AgdaFunction{𝔽≤ℭ}\AgdaSymbol{)}\<%
\end{code}

\hypertarget{the-hsp-theorem}{%
\subsection{The HSP Theorem}\label{the-hsp-theorem}}

Finally, we are in a position to prove Birkhoff's celebrated variety
theorem.

\begin{code}%
\>[0]\<%
\\
\>[0]\AgdaKeyword{module}\AgdaSpace{}%
\AgdaModule{\AgdaUnderscore{}}\AgdaSpace{}%
\AgdaSymbol{\{}\AgdaBound{𝒦}\AgdaSpace{}%
\AgdaSymbol{:}\AgdaSpace{}%
\AgdaFunction{Pred}\AgdaSymbol{(}\AgdaRecord{Algebra}\AgdaSpace{}%
\AgdaGeneralizable{α}\AgdaSpace{}%
\AgdaGeneralizable{ρᵃ}\AgdaSymbol{)}\AgdaSpace{}%
\AgdaSymbol{(}\AgdaGeneralizable{α}\AgdaSpace{}%
\AgdaOperator{\AgdaPrimitive{⊔}}\AgdaSpace{}%
\AgdaGeneralizable{ρᵃ}\AgdaSpace{}%
\AgdaOperator{\AgdaPrimitive{⊔}}\AgdaSpace{}%
\AgdaFunction{ov}\AgdaSpace{}%
\AgdaGeneralizable{ℓ}\AgdaSymbol{)\}}\AgdaSpace{}%
\AgdaKeyword{where}\<%
\\
\>[0][@{}l@{\AgdaIndent{0}}]%
\>[1]\AgdaKeyword{private}\AgdaSpace{}%
\AgdaFunction{ι}\AgdaSpace{}%
\AgdaSymbol{=}\AgdaSpace{}%
\AgdaFunction{ov}\AgdaSymbol{(}\AgdaBound{α}\AgdaSpace{}%
\AgdaOperator{\AgdaPrimitive{⊔}}\AgdaSpace{}%
\AgdaBound{ρᵃ}\AgdaSpace{}%
\AgdaOperator{\AgdaPrimitive{⊔}}\AgdaSpace{}%
\AgdaBound{ℓ}\AgdaSymbol{)}\<%
\\
%
\>[1]\AgdaKeyword{open}\AgdaSpace{}%
\AgdaModule{FreeHom}\AgdaSpace{}%
\AgdaSymbol{\{}\AgdaArgument{ℓ}\AgdaSpace{}%
\AgdaSymbol{=}\AgdaSpace{}%
\AgdaBound{ℓ}\AgdaSymbol{\}(}\AgdaBound{α}\AgdaSpace{}%
\AgdaOperator{\AgdaPrimitive{⊔}}\AgdaSpace{}%
\AgdaBound{ρᵃ}\AgdaSpace{}%
\AgdaOperator{\AgdaPrimitive{⊔}}\AgdaSpace{}%
\AgdaBound{ℓ}\AgdaSymbol{)\{}\AgdaBound{𝒦}\AgdaSymbol{\}}\<%
\\
%
\>[1]\AgdaKeyword{open}\AgdaSpace{}%
\AgdaModule{FreeAlgebra}\AgdaSpace{}%
\AgdaSymbol{\{}\AgdaArgument{ι}\AgdaSpace{}%
\AgdaSymbol{=}\AgdaSpace{}%
\AgdaFunction{ι}\AgdaSymbol{\}\{}\AgdaArgument{I}\AgdaSpace{}%
\AgdaSymbol{=}\AgdaSpace{}%
\AgdaFunction{ℐ}\AgdaSymbol{\}}\AgdaSpace{}%
\AgdaFunction{ℰ}\AgdaSpace{}%
\AgdaKeyword{using}\AgdaSpace{}%
\AgdaSymbol{(}\AgdaSpace{}%
\AgdaOperator{\AgdaFunction{𝔽[\AgdaUnderscore{}]}}\AgdaSpace{}%
\AgdaSymbol{)}\<%
\\
%
\\[\AgdaEmptyExtraSkip]%
%
\>[1]\AgdaFunction{Birkhoff}\AgdaSpace{}%
\AgdaSymbol{:}\AgdaSpace{}%
\AgdaSymbol{∀}\AgdaSpace{}%
\AgdaBound{𝑨}\AgdaSpace{}%
\AgdaSymbol{→}\AgdaSpace{}%
\AgdaBound{𝑨}\AgdaSpace{}%
\AgdaOperator{\AgdaFunction{∈}}\AgdaSpace{}%
\AgdaFunction{Mod}\AgdaSpace{}%
\AgdaSymbol{(}\AgdaFunction{Th}\AgdaSpace{}%
\AgdaSymbol{(}\AgdaFunction{V}\AgdaSpace{}%
\AgdaBound{ℓ}\AgdaSpace{}%
\AgdaFunction{ι}\AgdaSpace{}%
\AgdaBound{𝒦}\AgdaSymbol{))}\AgdaSpace{}%
\AgdaSymbol{→}\AgdaSpace{}%
\AgdaBound{𝑨}\AgdaSpace{}%
\AgdaOperator{\AgdaFunction{∈}}\AgdaSpace{}%
\AgdaFunction{V}\AgdaSpace{}%
\AgdaBound{ℓ}\AgdaSpace{}%
\AgdaFunction{ι}\AgdaSpace{}%
\AgdaBound{𝒦}\<%
\\
%
\>[1]\AgdaFunction{Birkhoff}\AgdaSpace{}%
\AgdaBound{𝑨}\AgdaSpace{}%
\AgdaBound{ModThA}\AgdaSpace{}%
\AgdaSymbol{=}\AgdaSpace{}%
\AgdaFunction{V-≅-lc}\AgdaSymbol{\{}\AgdaBound{α}\AgdaSymbol{\}\{}\AgdaBound{ρᵃ}\AgdaSymbol{\}\{}\AgdaBound{ℓ}\AgdaSymbol{\}}\AgdaSpace{}%
\AgdaBound{𝒦}\AgdaSpace{}%
\AgdaBound{𝑨}\AgdaSpace{}%
\AgdaFunction{VlA}\<%
\\
\>[1][@{}l@{\AgdaIndent{0}}]%
\>[2]\AgdaKeyword{where}\<%
\\
%
\>[2]\AgdaKeyword{open}\AgdaSpace{}%
\AgdaModule{Setoid}\AgdaSpace{}%
\AgdaOperator{\AgdaFunction{𝔻[}}\AgdaSpace{}%
\AgdaBound{𝑨}\AgdaSpace{}%
\AgdaOperator{\AgdaFunction{]}}\AgdaSpace{}%
\AgdaKeyword{using}\AgdaSpace{}%
\AgdaSymbol{()}\AgdaSpace{}%
\AgdaKeyword{renaming}\AgdaSpace{}%
\AgdaSymbol{(}\AgdaSpace{}%
\AgdaField{Carrier}\AgdaSpace{}%
\AgdaSymbol{to}\AgdaSpace{}%
\AgdaField{A}\AgdaSpace{}%
\AgdaSymbol{)}\<%
\\
%
\>[2]\AgdaFunction{sp𝔽A}\AgdaSpace{}%
\AgdaSymbol{:}\AgdaSpace{}%
\AgdaOperator{\AgdaFunction{𝔽[}}\AgdaSpace{}%
\AgdaFunction{A}\AgdaSpace{}%
\AgdaOperator{\AgdaFunction{]}}\AgdaSpace{}%
\AgdaOperator{\AgdaFunction{∈}}\AgdaSpace{}%
\AgdaFunction{S}\AgdaSymbol{\{}\AgdaFunction{ι}\AgdaSymbol{\}}\AgdaSpace{}%
\AgdaFunction{ι}\AgdaSpace{}%
\AgdaSymbol{(}\AgdaFunction{P}\AgdaSpace{}%
\AgdaBound{ℓ}\AgdaSpace{}%
\AgdaFunction{ι}\AgdaSpace{}%
\AgdaBound{𝒦}\AgdaSymbol{)}\<%
\\
%
\>[2]\AgdaFunction{sp𝔽A}\AgdaSpace{}%
\AgdaSymbol{=}\AgdaSpace{}%
\AgdaFunction{SP𝔽}\AgdaSymbol{\{}\AgdaArgument{ℓ}\AgdaSpace{}%
\AgdaSymbol{=}\AgdaSpace{}%
\AgdaBound{ℓ}\AgdaSymbol{\}}\AgdaSpace{}%
\AgdaBound{𝒦}\<%
\\
%
\>[2]\AgdaFunction{epi𝔽lA}\AgdaSpace{}%
\AgdaSymbol{:}\AgdaSpace{}%
\AgdaFunction{epi}\AgdaSpace{}%
\AgdaOperator{\AgdaFunction{𝔽[}}\AgdaSpace{}%
\AgdaFunction{A}\AgdaSpace{}%
\AgdaOperator{\AgdaFunction{]}}\AgdaSpace{}%
\AgdaSymbol{(}\AgdaFunction{Lift-Alg}\AgdaSpace{}%
\AgdaBound{𝑨}\AgdaSpace{}%
\AgdaFunction{ι}\AgdaSpace{}%
\AgdaFunction{ι}\AgdaSymbol{)}\<%
\\
%
\>[2]\AgdaFunction{epi𝔽lA}\AgdaSpace{}%
\AgdaSymbol{=}\AgdaSpace{}%
\AgdaFunction{𝔽-ModTh-epi-lift}\AgdaSymbol{\{}\AgdaArgument{ℓ}\AgdaSpace{}%
\AgdaSymbol{=}\AgdaSpace{}%
\AgdaBound{ℓ}\AgdaSymbol{\}}\AgdaSpace{}%
\AgdaSymbol{(λ}\AgdaSpace{}%
\AgdaSymbol{\{}\AgdaBound{p}\AgdaSpace{}%
\AgdaBound{q}\AgdaSymbol{\}}\AgdaSpace{}%
\AgdaSymbol{→}\AgdaSpace{}%
\AgdaBound{ModThA}\AgdaSymbol{\{}\AgdaArgument{p}\AgdaSpace{}%
\AgdaSymbol{=}\AgdaSpace{}%
\AgdaBound{p}\AgdaSymbol{\}\{}\AgdaBound{q}\AgdaSymbol{\})}\<%
\\
%
\>[2]\AgdaFunction{lAimg𝔽A}\AgdaSpace{}%
\AgdaSymbol{:}\AgdaSpace{}%
\AgdaFunction{Lift-Alg}\AgdaSpace{}%
\AgdaBound{𝑨}\AgdaSpace{}%
\AgdaFunction{ι}\AgdaSpace{}%
\AgdaFunction{ι}\AgdaSpace{}%
\AgdaOperator{\AgdaFunction{IsHomImageOf}}\AgdaSpace{}%
\AgdaOperator{\AgdaFunction{𝔽[}}\AgdaSpace{}%
\AgdaFunction{A}\AgdaSpace{}%
\AgdaOperator{\AgdaFunction{]}}\<%
\\
%
\>[2]\AgdaFunction{lAimg𝔽A}\AgdaSpace{}%
\AgdaSymbol{=}\AgdaSpace{}%
\AgdaFunction{epi→ontohom}\AgdaSpace{}%
\AgdaOperator{\AgdaFunction{𝔽[}}\AgdaSpace{}%
\AgdaFunction{A}\AgdaSpace{}%
\AgdaOperator{\AgdaFunction{]}}\AgdaSpace{}%
\AgdaSymbol{(}\AgdaFunction{Lift-Alg}\AgdaSpace{}%
\AgdaBound{𝑨}\AgdaSpace{}%
\AgdaFunction{ι}\AgdaSpace{}%
\AgdaFunction{ι}\AgdaSymbol{)}\AgdaSpace{}%
\AgdaFunction{epi𝔽lA}\<%
\\
%
\>[2]\AgdaFunction{VlA}\AgdaSpace{}%
\AgdaSymbol{:}\AgdaSpace{}%
\AgdaFunction{Lift-Alg}\AgdaSpace{}%
\AgdaBound{𝑨}\AgdaSpace{}%
\AgdaFunction{ι}\AgdaSpace{}%
\AgdaFunction{ι}\AgdaSpace{}%
\AgdaOperator{\AgdaFunction{∈}}\AgdaSpace{}%
\AgdaFunction{V}\AgdaSpace{}%
\AgdaBound{ℓ}\AgdaSpace{}%
\AgdaFunction{ι}\AgdaSpace{}%
\AgdaBound{𝒦}\<%
\\
%
\>[2]\AgdaFunction{VlA}\AgdaSpace{}%
\AgdaSymbol{=}\AgdaSpace{}%
\AgdaOperator{\AgdaFunction{𝔽[}}\AgdaSpace{}%
\AgdaFunction{A}\AgdaSpace{}%
\AgdaOperator{\AgdaFunction{]}}\AgdaSpace{}%
\AgdaOperator{\AgdaInductiveConstructor{,}}\AgdaSpace{}%
\AgdaFunction{sp𝔽A}\AgdaSpace{}%
\AgdaOperator{\AgdaInductiveConstructor{,}}\AgdaSpace{}%
\AgdaFunction{lAimg𝔽A}\<%
\\
\>[0]\<%
\end{code}

The converse inclusion, \texttt{V\ 𝒦\ ⊆\ Mod\ (Th\ (V\ 𝒦))}, is a simple
consequence of the fact that \texttt{Mod\ Th} is a closure operator.
Nonetheless, completeness demands that we formalize this inclusion as
well, however trivial the proof.

\begin{code}%
\>[0]\<%
\\
\>[0][@{}l@{\AgdaIndent{1}}]%
\>[1]\AgdaKeyword{module}\AgdaSpace{}%
\AgdaModule{\AgdaUnderscore{}}\AgdaSpace{}%
\AgdaSymbol{\{}\AgdaBound{𝑨}\AgdaSpace{}%
\AgdaSymbol{:}\AgdaSpace{}%
\AgdaRecord{Algebra}\AgdaSpace{}%
\AgdaBound{α}\AgdaSpace{}%
\AgdaBound{ρᵃ}\AgdaSymbol{\}}\AgdaSpace{}%
\AgdaKeyword{where}\<%
\\
\>[1][@{}l@{\AgdaIndent{0}}]%
\>[2]\AgdaFunction{Birkhoff-converse}\AgdaSpace{}%
\AgdaSymbol{:}\AgdaSpace{}%
\AgdaBound{𝑨}\AgdaSpace{}%
\AgdaOperator{\AgdaFunction{∈}}\AgdaSpace{}%
\AgdaFunction{V}\AgdaSymbol{\{}\AgdaBound{α}\AgdaSymbol{\}\{}\AgdaBound{ρᵃ}\AgdaSymbol{\}\{}\AgdaBound{α}\AgdaSymbol{\}\{}\AgdaBound{ρᵃ}\AgdaSymbol{\}\{}\AgdaBound{α}\AgdaSymbol{\}\{}\AgdaBound{ρᵃ}\AgdaSymbol{\}}\AgdaSpace{}%
\AgdaBound{ℓ}\AgdaSpace{}%
\AgdaFunction{ι}\AgdaSpace{}%
\AgdaBound{𝒦}\AgdaSpace{}%
\AgdaSymbol{→}\AgdaSpace{}%
\AgdaBound{𝑨}\AgdaSpace{}%
\AgdaOperator{\AgdaFunction{∈}}\AgdaSpace{}%
\AgdaFunction{Mod}\AgdaSymbol{\{}\AgdaArgument{X}\AgdaSpace{}%
\AgdaSymbol{=}\AgdaSpace{}%
\AgdaOperator{\AgdaFunction{𝕌[}}\AgdaSpace{}%
\AgdaBound{𝑨}\AgdaSpace{}%
\AgdaOperator{\AgdaFunction{]}}\AgdaSymbol{\}}\AgdaSpace{}%
\AgdaSymbol{(}\AgdaFunction{Th}\AgdaSpace{}%
\AgdaSymbol{(}\AgdaFunction{V}\AgdaSpace{}%
\AgdaBound{ℓ}\AgdaSpace{}%
\AgdaFunction{ι}\AgdaSpace{}%
\AgdaBound{𝒦}\AgdaSymbol{))}\<%
\\
%
\>[2]\AgdaFunction{Birkhoff-converse}\AgdaSpace{}%
\AgdaBound{vA}\AgdaSpace{}%
\AgdaBound{pThq}\AgdaSpace{}%
\AgdaSymbol{=}\AgdaSpace{}%
\AgdaBound{pThq}\AgdaSpace{}%
\AgdaBound{𝑨}\AgdaSpace{}%
\AgdaBound{vA}\<%
\\
\>[0]\<%
\end{code}

We have thus proved that every variety is an equational class.

Readers familiar with the classical formulation of the Birkhoff HSP
theorem as an ``if and only if'' assertion might worry that the proof is
still incomplete. However, recall that in the
\href{https://ualib.github.io/agda-algebras/Varieties.Func.Preservation.html}{Varieties.Func.Preservation}
module we proved the following identity preservation lemma:

\texttt{V-id1\ :\ 𝒦\ ⊫\ p\ ≈̇\ q\ →\ V\ 𝒦\ ⊫\ p\ ≈̇\ q}

Thus, if \texttt{𝒦} is an equational class---that is, if \texttt{𝒦} is
the class of algebras satisfying all identities in some set---then
\texttt{V\ 𝒦} ⊆ 𝒦\texttt{.\ \ On\ the\ other\ hand,\ we\ proved\ that}V`
is expansive in the
\href{https://ualib.github.io/agda-algebras/Varieties.Func.Closure.html}{Varieties.Func.Closure}
module:

\texttt{V-expa\ :\ 𝒦\ ⊆\ V\ 𝒦}

so \texttt{𝒦} (= \texttt{V\ 𝒦} = \texttt{HSP\ 𝒦}) is a variety.

Taken together, \texttt{V-id1} and \texttt{V-expa} constitute formal
proof that every equational class is a variety.

This completes the formal proof of Birkhoff's variety theorem.


%\hypertarget{introduction}{%
\subsection{Introduction}\label{introduction}}

The Agda Universal Algebra Library
(\href{https://github.com/ualib/agda-algebras}{agda-algebras}) is a
collection of types and programs (theorems and proofs) formalizing the
foundations of universal algebra in dependent type theory using the
\href{https://wiki.portal.chalmers.se/agda/pmwiki.php}{Agda} programming
language and proof assistant. The agda-algebras library now includes a
substantial collection of definitions, theorems, and proofs from
universal algebra and equational logic and as such provides many
examples that exhibit the power of inductive and dependent types for
representing and reasoning about general algebraic and relational
structures.

The first major milestone of the
\href{https://github.com/ualib/agda-algebras}{agda-algebras} project is
a new formal proof of \emph{Birkhoff's variety theorem} (also known as
the \emph{HSP theorem}), the first version of which was completed in
\href{https://github.com/ualib/ualib.github.io/blob/b968e8af1117fc77700d3a588746cbefbd464835/sandbox/birkhoff-exp-new-new.lagda}{January
of 2021}. To the best of our knowledge, this was the first time
Birkhoff's theorem had been formulated and proved in dependent type
theory and verified with a proof assistant.

In this paper, we present a single Agda module called
{[}Demos.HSP{]}{[}{]}. This module extracts only those parts of the
library needed to prove Birkhoff's variety theorem. In order to meet
page limit guidelines, and to reduce strain on the reader, we omit
proofs of some routine or technical lemmas that do not provide much
insight into the overall development. However, a long version of this
paper, which includes all code in the {[}Demos.HSP{]}{[}{]} module, is
available on the arXiv. {[}reference needed{]}

In the course of our exposition of the proof of the HSP theorem, we
discuss some of the more challenging aspects of formalizing
\emph{universal algebra} in type theory and the issues that arise when
attempting to constructively prove some of the basic results in this
area. We demonstrate that dependent type theory and Agda, despite the
demands they place on the user, are accessible to working mathematicians
who have sufficient patience and a strong enough desire to
constructively codify their work and formally verify the correctness of
their results. Perhpas our presentation will be viewed as a sobering
glimpse of the painstaking process of doing mathematics in the languages
of dependent type theory using the Agda proof assistant. Nonetheless we
hope to make a compelling case for investing in these technologies.
Indeed, we are excited to share the gratifying rewards that come with
some mastery of type theory and interactive theorem proving.

\hypertarget{prior-art}{%
\subsubsection{Prior art}\label{prior-art}}

There have been a number of efforts to formalize parts of universal
algebra in type theory prior to ours, most notably

\begin{enumerate}
\def\labelenumi{\arabic{enumi}.}
\tightlist
\item
  Capretta {[}@Capretta:1999{]} (1999) formalized the basics of
  universal algebra in the Calculus of Inductive Constructions using the
  Coq proof assistant;
\item
  Spitters and van der Weegen {[}@Spitters:2011{]} (2011) formalized the
  basics of universal algebra and some classical algebraic structures,
  also in the Calculus of Inductive Constructions using the Coq proof
  assistant, promoting the use of type classes;
\item
  Gunther, et al {[}@Gunther:2018{]} (2018) developed what seems to be
  (prior to the
  \href{https://github.com/ualib/agda-algebras}{agda-algebras} library)
  the most extensive library of formal universal algebra to date; this
  work is based on dependent type theory and programmed in Agda; it
  treats multisorted algebras and goes beyond the basic Noether
  isomorphism theorems to include some basic equational logic.
\item
  Lynge and Spitters {[}@Lynge:2019{]} (2019) formalize basic,
  mutisorted universal algebra, up to the Noether isomorphism theorems,
  in homotopy type theory; in this setting, the authors can avoid using
  setoids by postulating a strong extensionality axiom called
  univalence.
\end{enumerate}

Some other projects aimed at formalizing mathematics generally, and
algebra in particular, have developed into very extensive libraries that
include definitions, theorems, and proofs about algebraic structures,
such as groups, rings, modules, etc. However, the goals of these efforts
seem to be the formalization of special classical algebraic structures,
as opposed to the general theory of (universal) algebras. Moreover, the
part of universal algebra and equational logic formalized in the
\href{https://github.com/ualib/agda-algebras}{agda-algebras} library
extends beyond the scope of prior efforts.

\hypertarget{preliminaries}{%
\subsection{Preliminaries}\label{preliminaries}}

\hypertarget{logical-foundations}{%
\subsubsection{Logical foundations}\label{logical-foundations}}

An Agda program typically begins by setting some language options and by
importing types from existing Agda libraries. The language options are
specified using the \texttt{OPTIONS} \emph{pragma} which affect control
the way Agda behaves by controlling the deduction rules that are
available to us and the logical axioms that are assumed when the program
is type-checked by Agda to verify its correctness. Every Agda program in
the \href{https://github.com/ualib/agda-algebras}{agda-algebras}
library, including the present module ({[}Demos.HSP{]}{[}{]}), begins
with the following line.

\begin{code}%
\>[0]\<%
\\
\>[0]\AgdaSymbol{\{-\#}\AgdaSpace{}%
\AgdaKeyword{OPTIONS}\AgdaSpace{}%
\AgdaPragma{--without-K}\AgdaSpace{}%
\AgdaPragma{--exact-split}\AgdaSpace{}%
\AgdaPragma{--safe}\AgdaSpace{}%
\AgdaSymbol{\#-\}}\<%
\\
\>[0]\<%
\end{code}

Here we provide a brief overview (along with links to more details) of
how these options affect our foundational assumptions.

\begin{itemize}
\item
  \texttt{without-K} disables
  \href{https://ncatlab.org/nlab/show/axiom+K+\%28type+theory\%29}{Streicher's
  K axiom} ; see also the
  \href{https://agda.readthedocs.io/en/v2.6.1/language/without-k.html}{section
  on axiom K} in the
  \href{https://agda.readthedocs.io/en/v2.6.1.3/language}{Agda Language
  Reference Manual}.
\item
  \texttt{exact-split} makes Agda accept only those definitions that
  behave like so-called \emph{judgmental} equalities.
  \href{https://www.cs.bham.ac.uk/~mhe}{Martín Escardó} explains this by
  saying it ``makes sure that pattern matching corresponds to Martin-Löf
  eliminators;'' see also the
  \href{https://agda.readthedocs.io/en/v2.6.1/tools/command-line-options.html\#pattern-matching-and-equality}{Pattern
  matching and equality section} of the
  \href{https://agda.readthedocs.io/en/v2.6.1.3/tools/}{Agda Tools}
  documentation.
\item
  \texttt{safe} ensures that nothing is postulated outright---every
  non-MLTT axiom has to be an explicit assumption (e.g., an argument to
  a function or module); see also
  \href{https://agda.readthedocs.io/en/v2.6.1/tools/command-line-options.html\#cmdoption-safe}{this
  section} of the
  \href{https://agda.readthedocs.io/en/v2.6.1.3/tools/}{Agda Tools}
  documentation and the
  \href{https://agda.readthedocs.io/en/v2.6.1/language/safe-agda.html\#safe-agda}{Safe
  Agda section} of the
  \href{https://agda.readthedocs.io/en/v2.6.1.3/language}{Agda Language
  Reference}.
\end{itemize}

The \texttt{OPTIONS} pragma is usually followed by the start of a module
and a list of \texttt{import} directives. We won't reproduce all of the
imports we use here. Rather we show only those imports that rename
objects from the standard library to our own notation which might be
less standard.

\begin{code}[hide]%
\>[0]\AgdaSymbol{\{-\#}\AgdaSpace{}%
\AgdaKeyword{OPTIONS}\AgdaSpace{}%
\AgdaPragma{--without-K}\AgdaSpace{}%
\AgdaPragma{--exact-split}\AgdaSpace{}%
\AgdaPragma{--safe}\AgdaSpace{}%
\AgdaSymbol{\#-\}}\<%
\\
\>[0]\AgdaKeyword{open}\AgdaSpace{}%
\AgdaKeyword{import}\AgdaSpace{}%
\AgdaModule{Algebras.Basic}\AgdaSpace{}%
\AgdaKeyword{using}\AgdaSpace{}%
\AgdaSymbol{(}\AgdaSpace{}%
\AgdaGeneralizable{𝓞}\AgdaSpace{}%
\AgdaSymbol{;}\AgdaSpace{}%
\AgdaGeneralizable{𝓥}\AgdaSpace{}%
\AgdaSymbol{;}\AgdaSpace{}%
\AgdaFunction{Signature}\AgdaSpace{}%
\AgdaSymbol{)}\<%
\\
\>[0]\AgdaKeyword{module}\AgdaSpace{}%
\AgdaModule{Demos.HSP}\AgdaSpace{}%
\AgdaSymbol{\{}\AgdaBound{𝑆}\AgdaSpace{}%
\AgdaSymbol{:}\AgdaSpace{}%
\AgdaFunction{Signature}\AgdaSpace{}%
\AgdaGeneralizable{𝓞}\AgdaSpace{}%
\AgdaGeneralizable{𝓥}\AgdaSymbol{\}}\AgdaSpace{}%
\AgdaKeyword{where}\<%
\end{code}
\begin{code}%
\>[0]\<%
\\
\>[0]\AgdaKeyword{open}\AgdaSpace{}%
\AgdaKeyword{import}\AgdaSpace{}%
\AgdaModule{Agda.Primitive}%
\>[28]\AgdaKeyword{using}\AgdaSpace{}%
\AgdaSymbol{(}\AgdaOperator{\AgdaPrimitive{\AgdaUnderscore{}⊔\AgdaUnderscore{}}}\AgdaSpace{}%
\AgdaSymbol{;}\AgdaSpace{}%
\AgdaPrimitive{lsuc}\AgdaSymbol{)}\AgdaSpace{}%
\AgdaKeyword{renaming}\AgdaSpace{}%
\AgdaSymbol{(}\AgdaPrimitive{Set}\AgdaSpace{}%
\AgdaSymbol{to}\AgdaSpace{}%
\AgdaPrimitive{Type}\AgdaSymbol{)}\<%
\\
\>[0]\AgdaKeyword{open}\AgdaSpace{}%
\AgdaKeyword{import}\AgdaSpace{}%
\AgdaModule{Data.Product}%
\>[28]\AgdaKeyword{using}\AgdaSpace{}%
\AgdaSymbol{(}\AgdaFunction{Σ-syntax}\AgdaSpace{}%
\AgdaSymbol{;}\AgdaSpace{}%
\AgdaOperator{\AgdaFunction{\AgdaUnderscore{}×\AgdaUnderscore{}}}\AgdaSpace{}%
\AgdaSymbol{;}\AgdaSpace{}%
\AgdaOperator{\AgdaInductiveConstructor{\AgdaUnderscore{},\AgdaUnderscore{}}}\AgdaSpace{}%
\AgdaSymbol{;}\AgdaSpace{}%
\AgdaRecord{Σ}\AgdaSymbol{)}\AgdaSpace{}%
\AgdaKeyword{renaming}\AgdaSpace{}%
\AgdaSymbol{(}\AgdaField{proj₁}\AgdaSpace{}%
\AgdaSymbol{to}\AgdaSpace{}%
\AgdaField{fst}\AgdaSpace{}%
\AgdaSymbol{;}\AgdaSpace{}%
\AgdaField{proj₂}\AgdaSpace{}%
\AgdaSymbol{to}\AgdaSpace{}%
\AgdaField{snd}\AgdaSymbol{)}\<%
\\
\>[0]\AgdaKeyword{open}\AgdaSpace{}%
\AgdaKeyword{import}\AgdaSpace{}%
\AgdaModule{Function}%
\>[28]\AgdaKeyword{using}\AgdaSpace{}%
\AgdaSymbol{(}\AgdaFunction{id}\AgdaSpace{}%
\AgdaSymbol{;}\AgdaSpace{}%
\AgdaOperator{\AgdaFunction{\AgdaUnderscore{}∘\AgdaUnderscore{}}}\AgdaSpace{}%
\AgdaSymbol{;}\AgdaSpace{}%
\AgdaFunction{flip}\AgdaSpace{}%
\AgdaSymbol{;}\AgdaSpace{}%
\AgdaRecord{Injection}\AgdaSpace{}%
\AgdaSymbol{;}\AgdaSpace{}%
\AgdaRecord{Surjection}\AgdaSymbol{)}\AgdaSpace{}%
\AgdaKeyword{renaming}\AgdaSpace{}%
\AgdaSymbol{(}\AgdaRecord{Func}\AgdaSpace{}%
\AgdaSymbol{to}\AgdaSpace{}%
\AgdaRecord{\AgdaUnderscore{}⟶\AgdaUnderscore{}}\AgdaSymbol{)}\<%
\\
\>[0]\AgdaKeyword{open}\AgdaSpace{}%
\AgdaModule{\AgdaUnderscore{}⟶\AgdaUnderscore{}}%
\>[28]\AgdaKeyword{using}\AgdaSpace{}%
\AgdaSymbol{(}\AgdaField{cong}\AgdaSymbol{)}\AgdaSpace{}%
\AgdaKeyword{renaming}\AgdaSpace{}%
\AgdaSymbol{(}\AgdaField{f}\AgdaSpace{}%
\AgdaSymbol{to}\AgdaSpace{}%
\AgdaField{\AgdaUnderscore{}⟨\$⟩\AgdaUnderscore{}}\AgdaSymbol{)}\<%
\\
%
\\[\AgdaEmptyExtraSkip]%
\>[0]\AgdaKeyword{open}\AgdaSpace{}%
\AgdaKeyword{import}%
\>[13]\AgdaModule{Relation.Binary.PropositionalEquality}%
\>[52]\AgdaSymbol{as}\AgdaSpace{}%
\AgdaModule{≡}\AgdaSpace{}%
\AgdaKeyword{using}\AgdaSpace{}%
\AgdaSymbol{(}\AgdaOperator{\AgdaDatatype{\AgdaUnderscore{}≡\AgdaUnderscore{}}}\AgdaSymbol{)}\<%
\\
\>[0]\AgdaKeyword{import}%
\>[13]\AgdaModule{Relation.Binary.Reasoning.Setoid}%
\>[52]\AgdaSymbol{as}\AgdaSpace{}%
\AgdaModule{SetoidReasoning}\<%
\\
\>[0]\AgdaKeyword{import}%
\>[13]\AgdaModule{Function.Definitions}%
\>[52]\AgdaSymbol{as}\AgdaSpace{}%
\AgdaModule{FD}\<%
\\
\>[0]\<%
\end{code}
\begin{code}[hide]%
\>[0]\AgdaKeyword{open}\AgdaSpace{}%
\AgdaKeyword{import}\AgdaSpace{}%
\AgdaModule{Level}\AgdaSpace{}%
\AgdaKeyword{using}\AgdaSpace{}%
\AgdaSymbol{(}\AgdaSpace{}%
\AgdaPostulate{Level}\AgdaSpace{}%
\AgdaSymbol{)}\<%
\\
\>[0]\AgdaKeyword{open}\AgdaSpace{}%
\AgdaKeyword{import}\AgdaSpace{}%
\AgdaModule{Relation.Binary}\AgdaSpace{}%
\AgdaKeyword{using}\AgdaSpace{}%
\AgdaSymbol{(}\AgdaSpace{}%
\AgdaRecord{Setoid}\AgdaSpace{}%
\AgdaSymbol{;}\AgdaSpace{}%
\AgdaFunction{Rel}\AgdaSpace{}%
\AgdaSymbol{;}\AgdaSpace{}%
\AgdaRecord{IsEquivalence}\AgdaSpace{}%
\AgdaSymbol{)}\<%
\\
\>[0]\AgdaKeyword{open}\AgdaSpace{}%
\AgdaKeyword{import}\AgdaSpace{}%
\AgdaModule{Relation.Binary.Definitions}\AgdaSpace{}%
\AgdaKeyword{using}\AgdaSpace{}%
\AgdaSymbol{(}\AgdaSpace{}%
\AgdaFunction{Reflexive}\AgdaSpace{}%
\AgdaSymbol{;}\AgdaSpace{}%
\AgdaFunction{Sym}\AgdaSpace{}%
\AgdaSymbol{;}\AgdaSpace{}%
\AgdaFunction{Trans}\AgdaSpace{}%
\AgdaSymbol{;}\AgdaSpace{}%
\AgdaFunction{Symmetric}\AgdaSpace{}%
\AgdaSymbol{;}\AgdaSpace{}%
\AgdaFunction{Transitive}\AgdaSpace{}%
\AgdaSymbol{)}\<%
\\
\>[0]\AgdaKeyword{open}\AgdaSpace{}%
\AgdaKeyword{import}\AgdaSpace{}%
\AgdaModule{Relation.Unary}\AgdaSpace{}%
\AgdaKeyword{using}\AgdaSpace{}%
\AgdaSymbol{(}\AgdaSpace{}%
\AgdaFunction{Pred}\AgdaSpace{}%
\AgdaSymbol{;}\AgdaSpace{}%
\AgdaOperator{\AgdaFunction{\AgdaUnderscore{}⊆\AgdaUnderscore{}}}\AgdaSpace{}%
\AgdaSymbol{;}\AgdaSpace{}%
\AgdaOperator{\AgdaFunction{\AgdaUnderscore{}∈\AgdaUnderscore{}}}\AgdaSpace{}%
\AgdaSymbol{)}\<%
\\
\>[0]\AgdaKeyword{open}\AgdaSpace{}%
\AgdaModule{Setoid}\AgdaSpace{}%
\AgdaKeyword{using}\AgdaSpace{}%
\AgdaSymbol{(}\AgdaSpace{}%
\AgdaField{Carrier}\AgdaSpace{}%
\AgdaSymbol{;}\AgdaSpace{}%
\AgdaField{isEquivalence}\AgdaSpace{}%
\AgdaSymbol{)}\<%
\\
\>[0]\AgdaKeyword{private}\AgdaSpace{}%
\AgdaKeyword{variable}\<%
\\
\>[0][@{}l@{\AgdaIndent{0}}]%
\>[1]\AgdaGeneralizable{α}\AgdaSpace{}%
\AgdaGeneralizable{ρᵃ}\AgdaSpace{}%
\AgdaGeneralizable{β}\AgdaSpace{}%
\AgdaGeneralizable{ρᵇ}\AgdaSpace{}%
\AgdaGeneralizable{γ}\AgdaSpace{}%
\AgdaGeneralizable{ρᶜ}\AgdaSpace{}%
\AgdaGeneralizable{δ}\AgdaSpace{}%
\AgdaGeneralizable{ρᵈ}\AgdaSpace{}%
\AgdaGeneralizable{ρ}\AgdaSpace{}%
\AgdaGeneralizable{χ}\AgdaSpace{}%
\AgdaGeneralizable{ℓ}\AgdaSpace{}%
\AgdaSymbol{:}\AgdaSpace{}%
\AgdaPostulate{Level}\<%
\\
%
\>[1]\AgdaGeneralizable{Γ}\AgdaSpace{}%
\AgdaGeneralizable{Δ}\AgdaSpace{}%
\AgdaSymbol{:}\AgdaSpace{}%
\AgdaPrimitive{Type}\AgdaSpace{}%
\AgdaGeneralizable{χ}\<%
\\
%
\>[1]\AgdaGeneralizable{𝑓}\AgdaSpace{}%
\AgdaSymbol{:}\AgdaSpace{}%
\AgdaField{fst}\AgdaSpace{}%
\AgdaBound{𝑆}\<%
\end{code}

Note, in particular, we rename the \texttt{Set} and \texttt{Func} types
of the standard library to \texttt{Type} and the infix long arrow symbol
\texttt{\_⟶\_}, respectively, and we use \texttt{fst} and \texttt{snd}
in place of \texttt{proj₁} and \texttt{proj₂} for the first and second
projections out of the product type \texttt{\_×\_}. In addition, when it
improves readability of the code, we use the alternative notation
\texttt{∣\_∣} and \texttt{∥\_∥} (defined elsewhere) for the first and
second projections.

\hypertarget{setoids}{%
\subsubsection{Setoids}\label{setoids}}

A \emph{setoid} is a type packaged with an equivalence relation on the
collection of inhabitants of that type. Setoids are useful for
representing classical (set-theory-based) mathematics in a constructive,
type-theoretic way because most mathematical structures are assumed to
come equipped with some (often implicit) equivalence relation
manifesting a notion of equality of elements, and therefore a
type-theoretic representation of such a structure should also model its
equality relation.

The \href{https://github.com/ualib/agda-algebras}{agda-algebras} library
was first developed without the use of setoids, opting instead for
specially constructed experimental quotient types. However, this
approach resulted in code that was hard to comprehend and it became
difficult to determine whether the resulting proofs were fully
constructive. In particular, our initial proof of the Birkhoff variety
theorem required postulating function extensionality, an axiom that is
not provable in pure Martin-Löf type theory.{[}reference needed{]}

In contrast, our current approach using setoids makes the equality
relation of a given type explicit and this transparency can make it
easier to determine the correctness and constructivity of the proofs.
Using setiods we need no additional axioms beyond Martin-Löf type
theory; in particular, no function extensionality axioms are postulated
in our current formalization of Birkhoff's variety theorem.

\begin{code}[hide]%
\>[0]\AgdaFunction{𝑖𝑑}\AgdaSpace{}%
\AgdaSymbol{:}\AgdaSpace{}%
\AgdaSymbol{\{}\AgdaBound{A}\AgdaSpace{}%
\AgdaSymbol{:}\AgdaSpace{}%
\AgdaRecord{Setoid}\AgdaSpace{}%
\AgdaGeneralizable{α}\AgdaSpace{}%
\AgdaGeneralizable{ρᵃ}\AgdaSymbol{\}}\AgdaSpace{}%
\AgdaSymbol{→}\AgdaSpace{}%
\AgdaBound{A}\AgdaSpace{}%
\AgdaOperator{\AgdaRecord{⟶}}\AgdaSpace{}%
\AgdaBound{A}\<%
\\
\>[0]\AgdaFunction{𝑖𝑑}\AgdaSpace{}%
\AgdaSymbol{\{}\AgdaBound{A}\AgdaSymbol{\}}\AgdaSpace{}%
\AgdaSymbol{=}\AgdaSpace{}%
\AgdaKeyword{record}\AgdaSpace{}%
\AgdaSymbol{\{}\AgdaSpace{}%
\AgdaField{f}\AgdaSpace{}%
\AgdaSymbol{=}\AgdaSpace{}%
\AgdaFunction{id}\AgdaSpace{}%
\AgdaSymbol{;}\AgdaSpace{}%
\AgdaField{cong}\AgdaSpace{}%
\AgdaSymbol{=}\AgdaSpace{}%
\AgdaFunction{id}\AgdaSpace{}%
\AgdaSymbol{\}}\<%
\\
%
\\[\AgdaEmptyExtraSkip]%
\>[0]\AgdaOperator{\AgdaFunction{\AgdaUnderscore{}⟨∘⟩\AgdaUnderscore{}}}\AgdaSpace{}%
\AgdaSymbol{:}%
\>[9]\AgdaSymbol{\{}\AgdaBound{A}\AgdaSpace{}%
\AgdaSymbol{:}\AgdaSpace{}%
\AgdaRecord{Setoid}\AgdaSpace{}%
\AgdaGeneralizable{α}\AgdaSpace{}%
\AgdaGeneralizable{ρᵃ}\AgdaSymbol{\}}\AgdaSpace{}%
\AgdaSymbol{\{}\AgdaBound{B}\AgdaSpace{}%
\AgdaSymbol{:}\AgdaSpace{}%
\AgdaRecord{Setoid}\AgdaSpace{}%
\AgdaGeneralizable{β}\AgdaSpace{}%
\AgdaGeneralizable{ρᵇ}\AgdaSymbol{\}}\AgdaSpace{}%
\AgdaSymbol{\{}\AgdaBound{C}\AgdaSpace{}%
\AgdaSymbol{:}\AgdaSpace{}%
\AgdaRecord{Setoid}\AgdaSpace{}%
\AgdaGeneralizable{γ}\AgdaSpace{}%
\AgdaGeneralizable{ρᶜ}\AgdaSymbol{\}}\<%
\\
\>[0][@{}l@{\AgdaIndent{0}}]%
\>[1]\AgdaSymbol{→}%
\>[9]\AgdaBound{B}\AgdaSpace{}%
\AgdaOperator{\AgdaRecord{⟶}}\AgdaSpace{}%
\AgdaBound{C}%
\>[16]\AgdaSymbol{→}%
\>[19]\AgdaBound{A}\AgdaSpace{}%
\AgdaOperator{\AgdaRecord{⟶}}\AgdaSpace{}%
\AgdaBound{B}%
\>[26]\AgdaSymbol{→}%
\>[29]\AgdaBound{A}\AgdaSpace{}%
\AgdaOperator{\AgdaRecord{⟶}}\AgdaSpace{}%
\AgdaBound{C}\<%
\\
%
\\[\AgdaEmptyExtraSkip]%
\>[0]\AgdaBound{f}\AgdaSpace{}%
\AgdaOperator{\AgdaFunction{⟨∘⟩}}\AgdaSpace{}%
\AgdaBound{g}\AgdaSpace{}%
\AgdaSymbol{=}\AgdaSpace{}%
\AgdaKeyword{record}%
\>[18]\AgdaSymbol{\{}\AgdaSpace{}%
\AgdaField{f}\AgdaSpace{}%
\AgdaSymbol{=}\AgdaSpace{}%
\AgdaSymbol{(}\AgdaOperator{\AgdaField{\AgdaUnderscore{}⟨\$⟩\AgdaUnderscore{}}}\AgdaSpace{}%
\AgdaBound{f}\AgdaSymbol{)}\AgdaSpace{}%
\AgdaOperator{\AgdaFunction{∘}}\AgdaSpace{}%
\AgdaSymbol{(}\AgdaOperator{\AgdaField{\AgdaUnderscore{}⟨\$⟩\AgdaUnderscore{}}}\AgdaSpace{}%
\AgdaBound{g}\AgdaSymbol{)}\<%
\\
%
\>[18]\AgdaSymbol{;}\AgdaSpace{}%
\AgdaField{cong}\AgdaSpace{}%
\AgdaSymbol{=}\AgdaSpace{}%
\AgdaSymbol{(}\AgdaField{cong}\AgdaSpace{}%
\AgdaBound{f}\AgdaSymbol{)}\AgdaSpace{}%
\AgdaOperator{\AgdaFunction{∘}}\AgdaSpace{}%
\AgdaSymbol{(}\AgdaField{cong}\AgdaSpace{}%
\AgdaBound{g}\AgdaSymbol{)}\AgdaSpace{}%
\AgdaSymbol{\}}\<%
\\
%
\\[\AgdaEmptyExtraSkip]%
\>[0]\AgdaKeyword{module}\AgdaSpace{}%
\AgdaModule{\AgdaUnderscore{}}\AgdaSpace{}%
\AgdaSymbol{\{}\AgdaBound{A}\AgdaSpace{}%
\AgdaSymbol{:}\AgdaSpace{}%
\AgdaPrimitive{Type}\AgdaSpace{}%
\AgdaGeneralizable{α}\AgdaSpace{}%
\AgdaSymbol{\}\{}\AgdaBound{B}\AgdaSpace{}%
\AgdaSymbol{:}\AgdaSpace{}%
\AgdaBound{A}\AgdaSpace{}%
\AgdaSymbol{→}\AgdaSpace{}%
\AgdaPrimitive{Type}\AgdaSpace{}%
\AgdaGeneralizable{β}\AgdaSymbol{\}}\AgdaSpace{}%
\AgdaKeyword{where}\<%
\\
%
\\[\AgdaEmptyExtraSkip]%
\>[0][@{}l@{\AgdaIndent{0}}]%
\>[1]\AgdaOperator{\AgdaFunction{∣\AgdaUnderscore{}∣}}\AgdaSpace{}%
\AgdaSymbol{:}\AgdaSpace{}%
\AgdaFunction{Σ[}\AgdaSpace{}%
\AgdaBound{x}\AgdaSpace{}%
\AgdaFunction{∈}\AgdaSpace{}%
\AgdaBound{A}\AgdaSpace{}%
\AgdaFunction{]}\AgdaSpace{}%
\AgdaBound{B}\AgdaSpace{}%
\AgdaBound{x}\AgdaSpace{}%
\AgdaSymbol{→}\AgdaSpace{}%
\AgdaBound{A}\<%
\\
%
\>[1]\AgdaOperator{\AgdaFunction{∣\AgdaUnderscore{}∣}}\AgdaSpace{}%
\AgdaSymbol{=}\AgdaSpace{}%
\AgdaField{fst}\<%
\\
%
\\[\AgdaEmptyExtraSkip]%
%
\>[1]\AgdaOperator{\AgdaFunction{∥\AgdaUnderscore{}∥}}\AgdaSpace{}%
\AgdaSymbol{:}\AgdaSpace{}%
\AgdaSymbol{(}\AgdaBound{z}\AgdaSpace{}%
\AgdaSymbol{:}\AgdaSpace{}%
\AgdaFunction{Σ[}\AgdaSpace{}%
\AgdaBound{a}\AgdaSpace{}%
\AgdaFunction{∈}\AgdaSpace{}%
\AgdaBound{A}\AgdaSpace{}%
\AgdaFunction{]}\AgdaSpace{}%
\AgdaBound{B}\AgdaSpace{}%
\AgdaBound{a}\AgdaSymbol{)}\AgdaSpace{}%
\AgdaSymbol{→}\AgdaSpace{}%
\AgdaBound{B}\AgdaSpace{}%
\AgdaOperator{\AgdaFunction{∣}}\AgdaSpace{}%
\AgdaBound{z}\AgdaSpace{}%
\AgdaOperator{\AgdaFunction{∣}}\<%
\\
%
\>[1]\AgdaOperator{\AgdaFunction{∥\AgdaUnderscore{}∥}}\AgdaSpace{}%
\AgdaSymbol{=}\AgdaSpace{}%
\AgdaField{snd}\<%
\end{code}

\hypertarget{inverses-of-setoid-functions}{%
\subsubsection{Inverses of setoid
functions}\label{inverses-of-setoid-functions}}

We define a data type that represent the semantic concept of the
\emph{image} of a function (cf.~the
\href{https://ualib.github.io/agda-algebras/Overture.Func.Inverses.html}{Overture.Func.Inverses}
module of the
\href{https://github.com/ualib/agda-algebras}{agda-algebras} library).

\begin{code}%
\>[0]\<%
\\
\>[0]\AgdaKeyword{module}\AgdaSpace{}%
\AgdaModule{\AgdaUnderscore{}}\AgdaSpace{}%
\AgdaSymbol{\{}\AgdaBound{𝑨}\AgdaSpace{}%
\AgdaSymbol{:}\AgdaSpace{}%
\AgdaRecord{Setoid}\AgdaSpace{}%
\AgdaGeneralizable{α}\AgdaSpace{}%
\AgdaGeneralizable{ρᵃ}\AgdaSymbol{\}\{}\AgdaBound{𝑩}\AgdaSpace{}%
\AgdaSymbol{:}\AgdaSpace{}%
\AgdaRecord{Setoid}\AgdaSpace{}%
\AgdaGeneralizable{β}\AgdaSpace{}%
\AgdaGeneralizable{ρᵇ}\AgdaSymbol{\}}\AgdaSpace{}%
\AgdaKeyword{where}\<%
\\
\>[0][@{}l@{\AgdaIndent{0}}]%
\>[1]\AgdaKeyword{open}\AgdaSpace{}%
\AgdaModule{Setoid}\AgdaSpace{}%
\AgdaBound{𝑩}\AgdaSpace{}%
\AgdaKeyword{using}\AgdaSpace{}%
\AgdaSymbol{(}\AgdaSpace{}%
\AgdaOperator{\AgdaField{\AgdaUnderscore{}≈\AgdaUnderscore{}}}\AgdaSpace{}%
\AgdaSymbol{;}\AgdaSpace{}%
\AgdaFunction{sym}\AgdaSpace{}%
\AgdaSymbol{)}\AgdaSpace{}%
\AgdaKeyword{renaming}\AgdaSpace{}%
\AgdaSymbol{(}\AgdaSpace{}%
\AgdaField{Carrier}\AgdaSpace{}%
\AgdaSymbol{to}\AgdaSpace{}%
\AgdaField{B}\AgdaSpace{}%
\AgdaSymbol{)}\<%
\\
%
\\[\AgdaEmptyExtraSkip]%
%
\>[1]\AgdaKeyword{data}\AgdaSpace{}%
\AgdaOperator{\AgdaDatatype{Image\AgdaUnderscore{}∋\AgdaUnderscore{}}}\AgdaSpace{}%
\AgdaSymbol{(}\AgdaBound{f}\AgdaSpace{}%
\AgdaSymbol{:}\AgdaSpace{}%
\AgdaBound{𝑨}\AgdaSpace{}%
\AgdaOperator{\AgdaRecord{⟶}}\AgdaSpace{}%
\AgdaBound{𝑩}\AgdaSymbol{)}\AgdaSpace{}%
\AgdaSymbol{:}\AgdaSpace{}%
\AgdaField{B}\AgdaSpace{}%
\AgdaSymbol{→}\AgdaSpace{}%
\AgdaPrimitive{Type}\AgdaSpace{}%
\AgdaSymbol{(}\AgdaBound{α}\AgdaSpace{}%
\AgdaOperator{\AgdaPrimitive{⊔}}\AgdaSpace{}%
\AgdaBound{β}\AgdaSpace{}%
\AgdaOperator{\AgdaPrimitive{⊔}}\AgdaSpace{}%
\AgdaBound{ρᵇ}\AgdaSymbol{)}\AgdaSpace{}%
\AgdaKeyword{where}\<%
\\
\>[1][@{}l@{\AgdaIndent{0}}]%
\>[2]\AgdaInductiveConstructor{eq}\AgdaSpace{}%
\AgdaSymbol{:}\AgdaSpace{}%
\AgdaSymbol{\{}\AgdaBound{b}\AgdaSpace{}%
\AgdaSymbol{:}\AgdaSpace{}%
\AgdaField{B}\AgdaSymbol{\}}\AgdaSpace{}%
\AgdaSymbol{→}\AgdaSpace{}%
\AgdaSymbol{∀}\AgdaSpace{}%
\AgdaBound{a}\AgdaSpace{}%
\AgdaSymbol{→}\AgdaSpace{}%
\AgdaBound{b}\AgdaSpace{}%
\AgdaOperator{\AgdaField{≈}}\AgdaSpace{}%
\AgdaSymbol{(}\AgdaBound{f}\AgdaSpace{}%
\AgdaOperator{\AgdaField{⟨\$⟩}}\AgdaSpace{}%
\AgdaBound{a}\AgdaSymbol{)}\AgdaSpace{}%
\AgdaSymbol{→}\AgdaSpace{}%
\AgdaOperator{\AgdaDatatype{Image}}\AgdaSpace{}%
\AgdaBound{f}\AgdaSpace{}%
\AgdaOperator{\AgdaDatatype{∋}}\AgdaSpace{}%
\AgdaBound{b}\<%
\\
\>[0]\<%
\end{code}

An inhabitant of \texttt{Image\ f\ ∋\ b} is a dependent pair
\texttt{(a\ ,\ p)}, where \texttt{a\ :\ A} and \texttt{p\ :\ b\ ≈\ f\ a}
is a proof that \texttt{f} maps \texttt{a} to \texttt{b}. Since the
proof that \texttt{b} belongs to the image of \texttt{f} is always
accompanied by a witness \texttt{a\ :\ A}, we can actually
\emph{compute} a (pseudo)inverse of \texttt{f}. For convenience, we
define this inverse function, which we call \texttt{Inv}, and which
takes an arbitrary \texttt{b\ :\ B} and a (witness, proof)-pair,
\texttt{(a\ ,\ p)\ :\ Image\ f\ ∋\ b}, and returns the witness
\texttt{a}.

\begin{code}%
\>[0]\<%
\\
\>[0][@{}l@{\AgdaIndent{1}}]%
\>[1]\AgdaFunction{Inv}\AgdaSpace{}%
\AgdaSymbol{:}\AgdaSpace{}%
\AgdaSymbol{(}\AgdaBound{f}\AgdaSpace{}%
\AgdaSymbol{:}\AgdaSpace{}%
\AgdaBound{𝑨}\AgdaSpace{}%
\AgdaOperator{\AgdaRecord{⟶}}\AgdaSpace{}%
\AgdaBound{𝑩}\AgdaSymbol{)\{}\AgdaBound{b}\AgdaSpace{}%
\AgdaSymbol{:}\AgdaSpace{}%
\AgdaField{B}\AgdaSymbol{\}}\AgdaSpace{}%
\AgdaSymbol{→}\AgdaSpace{}%
\AgdaOperator{\AgdaDatatype{Image}}\AgdaSpace{}%
\AgdaBound{f}\AgdaSpace{}%
\AgdaOperator{\AgdaDatatype{∋}}\AgdaSpace{}%
\AgdaBound{b}\AgdaSpace{}%
\AgdaSymbol{→}\AgdaSpace{}%
\AgdaField{Carrier}\AgdaSpace{}%
\AgdaBound{𝑨}\<%
\\
%
\>[1]\AgdaFunction{Inv}\AgdaSpace{}%
\AgdaSymbol{\AgdaUnderscore{}}\AgdaSpace{}%
\AgdaSymbol{(}\AgdaInductiveConstructor{eq}\AgdaSpace{}%
\AgdaBound{a}\AgdaSpace{}%
\AgdaSymbol{\AgdaUnderscore{})}\AgdaSpace{}%
\AgdaSymbol{=}\AgdaSpace{}%
\AgdaBound{a}\<%
\\
%
\\[\AgdaEmptyExtraSkip]%
%
\>[1]\AgdaFunction{InvIsInverseʳ}\AgdaSpace{}%
\AgdaSymbol{:}\AgdaSpace{}%
\AgdaSymbol{\{}\AgdaBound{f}\AgdaSpace{}%
\AgdaSymbol{:}\AgdaSpace{}%
\AgdaBound{𝑨}\AgdaSpace{}%
\AgdaOperator{\AgdaRecord{⟶}}\AgdaSpace{}%
\AgdaBound{𝑩}\AgdaSymbol{\}\{}\AgdaBound{b}\AgdaSpace{}%
\AgdaSymbol{:}\AgdaSpace{}%
\AgdaField{B}\AgdaSymbol{\}(}\AgdaBound{q}\AgdaSpace{}%
\AgdaSymbol{:}\AgdaSpace{}%
\AgdaOperator{\AgdaDatatype{Image}}\AgdaSpace{}%
\AgdaBound{f}\AgdaSpace{}%
\AgdaOperator{\AgdaDatatype{∋}}\AgdaSpace{}%
\AgdaBound{b}\AgdaSymbol{)}\AgdaSpace{}%
\AgdaSymbol{→}\AgdaSpace{}%
\AgdaSymbol{(}\AgdaBound{f}\AgdaSpace{}%
\AgdaOperator{\AgdaField{⟨\$⟩}}\AgdaSpace{}%
\AgdaSymbol{(}\AgdaFunction{Inv}\AgdaSpace{}%
\AgdaBound{f}\AgdaSpace{}%
\AgdaBound{q}\AgdaSymbol{))}\AgdaSpace{}%
\AgdaOperator{\AgdaField{≈}}\AgdaSpace{}%
\AgdaBound{b}\<%
\\
%
\>[1]\AgdaFunction{InvIsInverseʳ}\AgdaSpace{}%
\AgdaSymbol{(}\AgdaInductiveConstructor{eq}\AgdaSpace{}%
\AgdaSymbol{\AgdaUnderscore{}}\AgdaSpace{}%
\AgdaBound{p}\AgdaSymbol{)}\AgdaSpace{}%
\AgdaSymbol{=}\AgdaSpace{}%
\AgdaFunction{sym}\AgdaSpace{}%
\AgdaBound{p}\<%
\\
\>[0]\<%
\end{code}

\texttt{InvIsInverseʳ} proves that \texttt{Inv\ f} is the
range-restricted right-inverse of the setoid function \texttt{f}.

\hypertarget{injective-and-surjective-setoid-functions}{%
\subsubsection{Injective and surjective setoid
functions}\label{injective-and-surjective-setoid-functions}}

If \texttt{f\ :\ 𝑨\ ⟶\ 𝑩} is a setoid function from
\texttt{𝑨\ =\ (A\ ,\ ≈₀)} to \texttt{𝑩\ =\ (B\ ,\ ≈₁)}, then we call
\texttt{f} \emph{injective} provided \texttt{∀\ (a₀\ a₁\ :\ A)},
\texttt{f\ ⟨\$⟩\ a₀\ ≈₁\ f\ ⟨\$⟩\ a₁} implies \texttt{a₀\ ≈₀\ a₁}; we
call \texttt{f} \emph{surjective} provided \texttt{∀\ (b\ :\ B)},
\texttt{∃\ (a\ :\ A)} such that \texttt{f\ ⟨\$⟩\ a\ ≈₁\ b}. We codify
these definitions in Agda and prove some of their properties inside the
next submodule where we first set the stage by declaring two setoids
\texttt{𝑨} and \texttt{𝑩}, naming their equality relations, and making
some definitions from the standard library available.

\begin{code}%
\>[0]\<%
\\
\>[0]\AgdaKeyword{module}\AgdaSpace{}%
\AgdaModule{\AgdaUnderscore{}}\AgdaSpace{}%
\AgdaSymbol{\{}\AgdaBound{𝑨}\AgdaSpace{}%
\AgdaSymbol{:}\AgdaSpace{}%
\AgdaRecord{Setoid}\AgdaSpace{}%
\AgdaGeneralizable{α}\AgdaSpace{}%
\AgdaGeneralizable{ρᵃ}\AgdaSymbol{\}\{}\AgdaBound{𝑩}\AgdaSpace{}%
\AgdaSymbol{:}\AgdaSpace{}%
\AgdaRecord{Setoid}\AgdaSpace{}%
\AgdaGeneralizable{β}\AgdaSpace{}%
\AgdaGeneralizable{ρᵇ}\AgdaSymbol{\}}\AgdaSpace{}%
\AgdaKeyword{where}\<%
\\
\>[0][@{}l@{\AgdaIndent{0}}]%
\>[1]\AgdaKeyword{open}\AgdaSpace{}%
\AgdaModule{Setoid}\AgdaSpace{}%
\AgdaBound{𝑨}\AgdaSpace{}%
\AgdaKeyword{using}\AgdaSpace{}%
\AgdaSymbol{()}\AgdaSpace{}%
\AgdaKeyword{renaming}\AgdaSpace{}%
\AgdaSymbol{(}\AgdaSpace{}%
\AgdaOperator{\AgdaField{\AgdaUnderscore{}≈\AgdaUnderscore{}}}\AgdaSpace{}%
\AgdaSymbol{to}\AgdaSpace{}%
\AgdaOperator{\AgdaField{\AgdaUnderscore{}≈₁\AgdaUnderscore{}}}\AgdaSpace{}%
\AgdaSymbol{;}\AgdaSpace{}%
\AgdaField{Carrier}\AgdaSpace{}%
\AgdaSymbol{to}\AgdaSpace{}%
\AgdaField{A}\AgdaSpace{}%
\AgdaSymbol{)}\<%
\\
%
\>[1]\AgdaKeyword{open}\AgdaSpace{}%
\AgdaModule{Setoid}\AgdaSpace{}%
\AgdaBound{𝑩}\AgdaSpace{}%
\AgdaKeyword{using}\AgdaSpace{}%
\AgdaSymbol{()}\AgdaSpace{}%
\AgdaKeyword{renaming}\AgdaSpace{}%
\AgdaSymbol{(}\AgdaSpace{}%
\AgdaOperator{\AgdaField{\AgdaUnderscore{}≈\AgdaUnderscore{}}}\AgdaSpace{}%
\AgdaSymbol{to}\AgdaSpace{}%
\AgdaOperator{\AgdaField{\AgdaUnderscore{}≈₂\AgdaUnderscore{}}}\AgdaSpace{}%
\AgdaSymbol{;}\AgdaSpace{}%
\AgdaField{Carrier}\AgdaSpace{}%
\AgdaSymbol{to}\AgdaSpace{}%
\AgdaField{B}\AgdaSpace{}%
\AgdaSymbol{)}\<%
\\
%
\>[1]\AgdaKeyword{open}\AgdaSpace{}%
\AgdaModule{FD}\AgdaSpace{}%
\AgdaOperator{\AgdaFunction{\AgdaUnderscore{}≈₁\AgdaUnderscore{}}}\AgdaSpace{}%
\AgdaOperator{\AgdaField{\AgdaUnderscore{}≈₂\AgdaUnderscore{}}}\<%
\\
\>[0]\<%
\end{code}

The \href{https://agda.github.io/agda-stdlib/}{Agda Standard Library}
represents injective functions on bare types by the type
\texttt{Injective}, which we now use to define \texttt{IsInjective}
representing the property of being an injective setoid function. We then
define the type \texttt{IsSurjective} to represent the property of being
a surjective setoid function. Finally, we define \texttt{SurjInv} to
represent the \emph{right-inverse} of a surjective function. The
definitions are as follows
(cf.~\href{https://ualib.github.io/agda-algebras/Overture.Func.Injective.html}{Overture.Func.Injective}
and
\href{https://ualib.github.io/agda-algebras/Overture.Func.Surjective.html}{Overture.Func.Surjective}
in the \href{https://github.com/ualib/agda-algebras}{agda-algebras}
library).

\begin{code}%
\>[0]\<%
\\
\>[0][@{}l@{\AgdaIndent{1}}]%
\>[1]\AgdaFunction{IsInjective}\AgdaSpace{}%
\AgdaSymbol{:}\AgdaSpace{}%
\AgdaSymbol{(}\AgdaBound{𝑨}\AgdaSpace{}%
\AgdaOperator{\AgdaRecord{⟶}}\AgdaSpace{}%
\AgdaBound{𝑩}\AgdaSymbol{)}\AgdaSpace{}%
\AgdaSymbol{→}%
\>[26]\AgdaPrimitive{Type}\AgdaSpace{}%
\AgdaSymbol{(}\AgdaBound{α}\AgdaSpace{}%
\AgdaOperator{\AgdaPrimitive{⊔}}\AgdaSpace{}%
\AgdaBound{ρᵃ}\AgdaSpace{}%
\AgdaOperator{\AgdaPrimitive{⊔}}\AgdaSpace{}%
\AgdaBound{ρᵇ}\AgdaSymbol{)}\<%
\\
%
\>[1]\AgdaFunction{IsInjective}\AgdaSpace{}%
\AgdaBound{f}\AgdaSpace{}%
\AgdaSymbol{=}\AgdaSpace{}%
\AgdaFunction{Injective}\AgdaSpace{}%
\AgdaSymbol{(}\AgdaOperator{\AgdaField{\AgdaUnderscore{}⟨\$⟩\AgdaUnderscore{}}}\AgdaSpace{}%
\AgdaBound{f}\AgdaSymbol{)}\<%
\\
%
\\[\AgdaEmptyExtraSkip]%
%
\>[1]\AgdaFunction{IsSurjective}\AgdaSpace{}%
\AgdaSymbol{:}\AgdaSpace{}%
\AgdaSymbol{(}\AgdaBound{𝑨}\AgdaSpace{}%
\AgdaOperator{\AgdaRecord{⟶}}\AgdaSpace{}%
\AgdaBound{𝑩}\AgdaSymbol{)}\AgdaSpace{}%
\AgdaSymbol{→}%
\>[27]\AgdaPrimitive{Type}\AgdaSpace{}%
\AgdaSymbol{(}\AgdaBound{α}\AgdaSpace{}%
\AgdaOperator{\AgdaPrimitive{⊔}}\AgdaSpace{}%
\AgdaBound{β}\AgdaSpace{}%
\AgdaOperator{\AgdaPrimitive{⊔}}\AgdaSpace{}%
\AgdaBound{ρᵇ}\AgdaSymbol{)}\<%
\\
%
\>[1]\AgdaFunction{IsSurjective}\AgdaSpace{}%
\AgdaBound{F}\AgdaSpace{}%
\AgdaSymbol{=}\AgdaSpace{}%
\AgdaSymbol{∀}\AgdaSpace{}%
\AgdaSymbol{\{}\AgdaBound{y}\AgdaSymbol{\}}\AgdaSpace{}%
\AgdaSymbol{→}\AgdaSpace{}%
\AgdaOperator{\AgdaDatatype{Image}}\AgdaSpace{}%
\AgdaBound{F}\AgdaSpace{}%
\AgdaOperator{\AgdaDatatype{∋}}\AgdaSpace{}%
\AgdaBound{y}\<%
\\
%
\\[\AgdaEmptyExtraSkip]%
%
\>[1]\AgdaFunction{SurjInv}\AgdaSpace{}%
\AgdaSymbol{:}\AgdaSpace{}%
\AgdaSymbol{(}\AgdaBound{f}\AgdaSpace{}%
\AgdaSymbol{:}\AgdaSpace{}%
\AgdaBound{𝑨}\AgdaSpace{}%
\AgdaOperator{\AgdaRecord{⟶}}\AgdaSpace{}%
\AgdaBound{𝑩}\AgdaSymbol{)}\AgdaSpace{}%
\AgdaSymbol{→}\AgdaSpace{}%
\AgdaFunction{IsSurjective}\AgdaSpace{}%
\AgdaBound{f}\AgdaSpace{}%
\AgdaSymbol{→}\AgdaSpace{}%
\AgdaField{B}\AgdaSpace{}%
\AgdaSymbol{→}\AgdaSpace{}%
\AgdaFunction{A}\<%
\\
%
\>[1]\AgdaFunction{SurjInv}\AgdaSpace{}%
\AgdaBound{f}\AgdaSpace{}%
\AgdaBound{fonto}\AgdaSpace{}%
\AgdaBound{b}\AgdaSpace{}%
\AgdaSymbol{=}\AgdaSpace{}%
\AgdaFunction{Inv}\AgdaSpace{}%
\AgdaBound{f}\AgdaSpace{}%
\AgdaSymbol{(}\AgdaBound{fonto}\AgdaSpace{}%
\AgdaSymbol{\{}\AgdaBound{b}\AgdaSymbol{\})}\<%
\end{code}

\hypertarget{composition-of-injective-and-surjective-setoid-functions}{%
\paragraph{Composition of injective and surjective setoid
functions}\label{composition-of-injective-and-surjective-setoid-functions}}

Proving that the composition of injective setoid functions is again
injective is simply a matter of composing the two assumed witnesses to
injectivity. Proving that surjectivity is preserved under composition is
only slightly more involved.

\begin{code}%
\>[0]\<%
\\
\>[0]\AgdaKeyword{module}\AgdaSpace{}%
\AgdaModule{\AgdaUnderscore{}}%
\>[10]\AgdaSymbol{\{}\AgdaBound{𝑨}\AgdaSpace{}%
\AgdaSymbol{:}\AgdaSpace{}%
\AgdaRecord{Setoid}\AgdaSpace{}%
\AgdaGeneralizable{α}\AgdaSpace{}%
\AgdaGeneralizable{ρᵃ}\AgdaSymbol{\}\{}\AgdaBound{𝑩}\AgdaSpace{}%
\AgdaSymbol{:}\AgdaSpace{}%
\AgdaRecord{Setoid}\AgdaSpace{}%
\AgdaGeneralizable{β}\AgdaSpace{}%
\AgdaGeneralizable{ρᵇ}\AgdaSymbol{\}\{}\AgdaBound{𝑪}\AgdaSpace{}%
\AgdaSymbol{:}\AgdaSpace{}%
\AgdaRecord{Setoid}\AgdaSpace{}%
\AgdaGeneralizable{γ}\AgdaSpace{}%
\AgdaGeneralizable{ρᶜ}\AgdaSymbol{\}}\<%
\\
%
\>[10]\AgdaSymbol{(}\AgdaBound{f}\AgdaSpace{}%
\AgdaSymbol{:}\AgdaSpace{}%
\AgdaBound{𝑨}\AgdaSpace{}%
\AgdaOperator{\AgdaRecord{⟶}}\AgdaSpace{}%
\AgdaBound{𝑩}\AgdaSymbol{)(}\AgdaBound{g}\AgdaSpace{}%
\AgdaSymbol{:}\AgdaSpace{}%
\AgdaBound{𝑩}\AgdaSpace{}%
\AgdaOperator{\AgdaRecord{⟶}}\AgdaSpace{}%
\AgdaBound{𝑪}\AgdaSymbol{)}\AgdaSpace{}%
\AgdaKeyword{where}\<%
\\
%
\\[\AgdaEmptyExtraSkip]%
\>[0][@{}l@{\AgdaIndent{0}}]%
\>[1]\AgdaFunction{∘-IsInjective}\AgdaSpace{}%
\AgdaSymbol{:}\AgdaSpace{}%
\AgdaFunction{IsInjective}\AgdaSpace{}%
\AgdaBound{f}\AgdaSpace{}%
\AgdaSymbol{→}\AgdaSpace{}%
\AgdaFunction{IsInjective}\AgdaSpace{}%
\AgdaBound{g}\AgdaSpace{}%
\AgdaSymbol{→}\AgdaSpace{}%
\AgdaFunction{IsInjective}\AgdaSpace{}%
\AgdaSymbol{(}\AgdaBound{g}\AgdaSpace{}%
\AgdaOperator{\AgdaFunction{⟨∘⟩}}\AgdaSpace{}%
\AgdaBound{f}\AgdaSymbol{)}\<%
\\
%
\>[1]\AgdaFunction{∘-IsInjective}\AgdaSpace{}%
\AgdaBound{finj}\AgdaSpace{}%
\AgdaBound{ginj}\AgdaSpace{}%
\AgdaSymbol{=}\AgdaSpace{}%
\AgdaBound{finj}\AgdaSpace{}%
\AgdaOperator{\AgdaFunction{∘}}\AgdaSpace{}%
\AgdaBound{ginj}\<%
\\
%
\\[\AgdaEmptyExtraSkip]%
%
\>[1]\AgdaFunction{∘-IsSurjective}\AgdaSpace{}%
\AgdaSymbol{:}\AgdaSpace{}%
\AgdaFunction{IsSurjective}\AgdaSpace{}%
\AgdaBound{f}\AgdaSpace{}%
\AgdaSymbol{→}\AgdaSpace{}%
\AgdaFunction{IsSurjective}\AgdaSpace{}%
\AgdaBound{g}\AgdaSpace{}%
\AgdaSymbol{→}\AgdaSpace{}%
\AgdaFunction{IsSurjective}\AgdaSpace{}%
\AgdaSymbol{(}\AgdaBound{g}\AgdaSpace{}%
\AgdaOperator{\AgdaFunction{⟨∘⟩}}\AgdaSpace{}%
\AgdaBound{f}\AgdaSymbol{)}\<%
\\
%
\>[1]\AgdaFunction{∘-IsSurjective}\AgdaSpace{}%
\AgdaBound{fonto}\AgdaSpace{}%
\AgdaBound{gonto}\AgdaSpace{}%
\AgdaSymbol{\{}\AgdaBound{y}\AgdaSymbol{\}}\AgdaSpace{}%
\AgdaSymbol{=}\AgdaSpace{}%
\AgdaFunction{Goal}\<%
\\
\>[1][@{}l@{\AgdaIndent{0}}]%
\>[2]\AgdaKeyword{where}\<%
\\
%
\>[2]\AgdaFunction{mp}\AgdaSpace{}%
\AgdaSymbol{:}\AgdaSpace{}%
\AgdaOperator{\AgdaDatatype{Image}}\AgdaSpace{}%
\AgdaBound{g}\AgdaSpace{}%
\AgdaOperator{\AgdaDatatype{∋}}\AgdaSpace{}%
\AgdaBound{y}\AgdaSpace{}%
\AgdaSymbol{→}\AgdaSpace{}%
\AgdaOperator{\AgdaDatatype{Image}}\AgdaSpace{}%
\AgdaBound{g}\AgdaSpace{}%
\AgdaOperator{\AgdaFunction{⟨∘⟩}}\AgdaSpace{}%
\AgdaBound{f}\AgdaSpace{}%
\AgdaOperator{\AgdaDatatype{∋}}\AgdaSpace{}%
\AgdaBound{y}\<%
\\
%
\>[2]\AgdaFunction{mp}\AgdaSpace{}%
\AgdaSymbol{(}\AgdaInductiveConstructor{eq}\AgdaSpace{}%
\AgdaBound{c}\AgdaSpace{}%
\AgdaBound{p}\AgdaSymbol{)}\AgdaSpace{}%
\AgdaSymbol{=}\AgdaSpace{}%
\AgdaFunction{η}\AgdaSpace{}%
\AgdaBound{fonto}\<%
\\
\>[2][@{}l@{\AgdaIndent{0}}]%
\>[3]\AgdaKeyword{where}\<%
\\
%
\>[3]\AgdaKeyword{open}\AgdaSpace{}%
\AgdaModule{Setoid}\AgdaSpace{}%
\AgdaBound{𝑪}\AgdaSpace{}%
\AgdaKeyword{using}\AgdaSpace{}%
\AgdaSymbol{(}\AgdaSpace{}%
\AgdaFunction{trans}\AgdaSpace{}%
\AgdaSymbol{)}\<%
\\
%
\>[3]\AgdaFunction{η}\AgdaSpace{}%
\AgdaSymbol{:}\AgdaSpace{}%
\AgdaOperator{\AgdaDatatype{Image}}\AgdaSpace{}%
\AgdaBound{f}\AgdaSpace{}%
\AgdaOperator{\AgdaDatatype{∋}}\AgdaSpace{}%
\AgdaBound{c}\AgdaSpace{}%
\AgdaSymbol{→}\AgdaSpace{}%
\AgdaOperator{\AgdaDatatype{Image}}\AgdaSpace{}%
\AgdaBound{g}\AgdaSpace{}%
\AgdaOperator{\AgdaFunction{⟨∘⟩}}\AgdaSpace{}%
\AgdaBound{f}\AgdaSpace{}%
\AgdaOperator{\AgdaDatatype{∋}}\AgdaSpace{}%
\AgdaBound{y}\<%
\\
%
\>[3]\AgdaFunction{η}\AgdaSpace{}%
\AgdaSymbol{(}\AgdaInductiveConstructor{eq}\AgdaSpace{}%
\AgdaBound{a}\AgdaSpace{}%
\AgdaBound{q}\AgdaSymbol{)}\AgdaSpace{}%
\AgdaSymbol{=}\AgdaSpace{}%
\AgdaInductiveConstructor{eq}\AgdaSpace{}%
\AgdaBound{a}\AgdaSpace{}%
\AgdaSymbol{(}\AgdaFunction{trans}\AgdaSpace{}%
\AgdaBound{p}\AgdaSpace{}%
\AgdaSymbol{(}\AgdaField{cong}\AgdaSpace{}%
\AgdaBound{g}\AgdaSpace{}%
\AgdaBound{q}\AgdaSymbol{))}\<%
\\
%
\\[\AgdaEmptyExtraSkip]%
%
\>[2]\AgdaFunction{Goal}\AgdaSpace{}%
\AgdaSymbol{:}\AgdaSpace{}%
\AgdaOperator{\AgdaDatatype{Image}}\AgdaSpace{}%
\AgdaBound{g}\AgdaSpace{}%
\AgdaOperator{\AgdaFunction{⟨∘⟩}}\AgdaSpace{}%
\AgdaBound{f}\AgdaSpace{}%
\AgdaOperator{\AgdaDatatype{∋}}\AgdaSpace{}%
\AgdaBound{y}\<%
\\
%
\>[2]\AgdaFunction{Goal}\AgdaSpace{}%
\AgdaSymbol{=}\AgdaSpace{}%
\AgdaFunction{mp}\AgdaSpace{}%
\AgdaBound{gonto}\<%
\end{code}

\hypertarget{kernels}{%
\subsubsection{Kernels}\label{kernels}}

The \emph{kernel} of a function \texttt{f\ :\ A\ →\ B} (where \texttt{A}
and \texttt{B} are bare types) is defined informally by
\texttt{\{(x\ ,\ y)\ ∈\ A\ ×\ A\ :\ f\ x\ =\ f\ y\}}. This can be
represented in Agda in a number of ways, but for our purposes we find it
most convenient to define the kernel as an inhabitant of a (unary)
predicate over the square of the function's domain.

\begin{code}%
\>[0]\<%
\\
\>[0]\AgdaKeyword{module}\AgdaSpace{}%
\AgdaModule{\AgdaUnderscore{}}\AgdaSpace{}%
\AgdaSymbol{\{}\AgdaBound{A}\AgdaSpace{}%
\AgdaSymbol{:}\AgdaSpace{}%
\AgdaPrimitive{Type}\AgdaSpace{}%
\AgdaGeneralizable{α}\AgdaSymbol{\}\{}\AgdaBound{B}\AgdaSpace{}%
\AgdaSymbol{:}\AgdaSpace{}%
\AgdaPrimitive{Type}\AgdaSpace{}%
\AgdaGeneralizable{β}\AgdaSymbol{\}}\AgdaSpace{}%
\AgdaKeyword{where}\<%
\\
%
\\[\AgdaEmptyExtraSkip]%
\>[0][@{}l@{\AgdaIndent{0}}]%
\>[1]\AgdaFunction{kernel}\AgdaSpace{}%
\AgdaSymbol{:}\AgdaSpace{}%
\AgdaFunction{Rel}\AgdaSpace{}%
\AgdaBound{B}\AgdaSpace{}%
\AgdaGeneralizable{ρ}\AgdaSpace{}%
\AgdaSymbol{→}\AgdaSpace{}%
\AgdaSymbol{(}\AgdaBound{A}\AgdaSpace{}%
\AgdaSymbol{→}\AgdaSpace{}%
\AgdaBound{B}\AgdaSymbol{)}\AgdaSpace{}%
\AgdaSymbol{→}\AgdaSpace{}%
\AgdaFunction{Pred}\AgdaSpace{}%
\AgdaSymbol{(}\AgdaBound{A}\AgdaSpace{}%
\AgdaOperator{\AgdaFunction{×}}\AgdaSpace{}%
\AgdaBound{A}\AgdaSymbol{)}\AgdaSpace{}%
\AgdaGeneralizable{ρ}\<%
\\
%
\>[1]\AgdaFunction{kernel}\AgdaSpace{}%
\AgdaOperator{\AgdaBound{\AgdaUnderscore{}≈\AgdaUnderscore{}}}\AgdaSpace{}%
\AgdaBound{f}\AgdaSpace{}%
\AgdaSymbol{(}\AgdaBound{x}\AgdaSpace{}%
\AgdaOperator{\AgdaInductiveConstructor{,}}\AgdaSpace{}%
\AgdaBound{y}\AgdaSymbol{)}\AgdaSpace{}%
\AgdaSymbol{=}\AgdaSpace{}%
\AgdaBound{f}\AgdaSpace{}%
\AgdaBound{x}\AgdaSpace{}%
\AgdaOperator{\AgdaBound{≈}}\AgdaSpace{}%
\AgdaBound{f}\AgdaSpace{}%
\AgdaBound{y}\<%
\\
\>[0]\<%
\end{code}

The kernel of a setoid function \texttt{f\ :\ 𝐴\ ⟶\ 𝐵} is defined
informally by
\texttt{\{(x\ ,\ y)\ ∈\ A\ ×\ A\ :\ f\ ⟨\$⟩\ x\ ≈\ f\ ⟨\$⟩\ y\}}, where
\texttt{\_≈\_} denotes the equality of \texttt{𝐵}.

\begin{code}%
\>[0]\<%
\\
\>[0]\AgdaKeyword{module}\AgdaSpace{}%
\AgdaModule{\AgdaUnderscore{}}\AgdaSpace{}%
\AgdaSymbol{\{}\AgdaBound{𝐴}\AgdaSpace{}%
\AgdaSymbol{:}\AgdaSpace{}%
\AgdaRecord{Setoid}\AgdaSpace{}%
\AgdaGeneralizable{α}\AgdaSpace{}%
\AgdaGeneralizable{ρᵃ}\AgdaSymbol{\}\{}\AgdaBound{𝐵}\AgdaSpace{}%
\AgdaSymbol{:}\AgdaSpace{}%
\AgdaRecord{Setoid}\AgdaSpace{}%
\AgdaGeneralizable{β}\AgdaSpace{}%
\AgdaGeneralizable{ρᵇ}\AgdaSymbol{\}}\AgdaSpace{}%
\AgdaKeyword{where}\<%
\\
\>[0][@{}l@{\AgdaIndent{0}}]%
\>[1]\AgdaKeyword{open}\AgdaSpace{}%
\AgdaModule{Setoid}\AgdaSpace{}%
\AgdaBound{𝐴}\AgdaSpace{}%
\AgdaKeyword{using}\AgdaSpace{}%
\AgdaSymbol{()}\AgdaSpace{}%
\AgdaKeyword{renaming}\AgdaSpace{}%
\AgdaSymbol{(}\AgdaSpace{}%
\AgdaField{Carrier}\AgdaSpace{}%
\AgdaSymbol{to}\AgdaSpace{}%
\AgdaField{A}\AgdaSpace{}%
\AgdaSymbol{)}\<%
\\
%
\\[\AgdaEmptyExtraSkip]%
%
\>[1]\AgdaFunction{ker}\AgdaSpace{}%
\AgdaSymbol{:}\AgdaSpace{}%
\AgdaSymbol{(}\AgdaBound{𝐴}\AgdaSpace{}%
\AgdaOperator{\AgdaRecord{⟶}}\AgdaSpace{}%
\AgdaBound{𝐵}\AgdaSymbol{)}\AgdaSpace{}%
\AgdaSymbol{→}\AgdaSpace{}%
\AgdaFunction{Pred}\AgdaSpace{}%
\AgdaSymbol{(}\AgdaFunction{A}\AgdaSpace{}%
\AgdaOperator{\AgdaFunction{×}}\AgdaSpace{}%
\AgdaFunction{A}\AgdaSymbol{)}\AgdaSpace{}%
\AgdaBound{ρᵇ}\<%
\\
%
\>[1]\AgdaFunction{ker}\AgdaSpace{}%
\AgdaBound{g}\AgdaSpace{}%
\AgdaSymbol{(}\AgdaBound{x}\AgdaSpace{}%
\AgdaOperator{\AgdaInductiveConstructor{,}}\AgdaSpace{}%
\AgdaBound{y}\AgdaSymbol{)}\AgdaSpace{}%
\AgdaSymbol{=}\AgdaSpace{}%
\AgdaBound{g}\AgdaSpace{}%
\AgdaOperator{\AgdaField{⟨\$⟩}}\AgdaSpace{}%
\AgdaBound{x}\AgdaSpace{}%
\AgdaOperator{\AgdaFunction{≈}}\AgdaSpace{}%
\AgdaBound{g}\AgdaSpace{}%
\AgdaOperator{\AgdaField{⟨\$⟩}}\AgdaSpace{}%
\AgdaBound{y}\AgdaSpace{}%
\AgdaKeyword{where}\AgdaSpace{}%
\AgdaKeyword{open}\AgdaSpace{}%
\AgdaModule{Setoid}\AgdaSpace{}%
\AgdaBound{𝐵}\AgdaSpace{}%
\AgdaKeyword{using}\AgdaSpace{}%
\AgdaSymbol{(}\AgdaSpace{}%
\AgdaOperator{\AgdaField{\AgdaUnderscore{}≈\AgdaUnderscore{}}}\AgdaSpace{}%
\AgdaSymbol{)}\<%
\end{code}

\hypertarget{algebras}{%
\subsection{Algebras}\label{algebras}}

In this section we define the notion of an algebraic structure whose
\emph{domain} (or ``carrier'' or ``universe'') is a setoid. Our first
goal is to develop a working vocabulary and formal types for classical
(single-sorted, set-based) universal algebra.

\hypertarget{signatures-of-an-algebra}{%
\subsubsection{Signatures of an
algebra}\label{signatures-of-an-algebra}}

In \href{https://en.wikipedia.org/wiki/Model_theory}{model theory}, the
\emph{signature} \texttt{𝑆\ =\ (𝐶,\ 𝐹,\ 𝑅,\ ρ)} of a structure consists
of three (possibly empty) sets \texttt{𝐶}, \texttt{𝐹}, and
\texttt{𝑅}---called \emph{constant symbols}, \emph{function symbols},
and \emph{relation symbols}, respectively---along with a function
\texttt{ρ\ :\ 𝐶\ +\ 𝐹\ +\ 𝑅\ →\ 𝑁} that assigns an \emph{arity} to each
symbol. Often, but not always, \texttt{𝑁} is taken to be the set of
natural numbers.

As our focus here is universal algebra, we are more concerned with the
restricted notion of an \emph{algebraic signature} (or \emph{signature}
for algebraic structures), by which we mean a pair
\texttt{𝑆\ =\ (F,\ ρ)} consisting of a collection \texttt{𝐹} of
\emph{operation symbols} and an \emph{arity function}
\texttt{ρ\ :\ F\ →\ N} that maps each operation symbol to its arity;
here, 𝑁 denotes the \emph{arity type}. Heuristically, the arity
\texttt{ρ\ f} of an operation symbol \texttt{f\ ∈\ F} may be thought of
as the ``number of arguments'' that \texttt{f} takes as ``input''.

If the arity of \texttt{f} is \texttt{n}, then we call \texttt{f} an
\texttt{n}-\emph{ary} operation symbol. In case \texttt{n} is 0 (or 1 or
2 or 3, respectively) we call the function \emph{nullary} (or
\emph{unary} or \emph{binary} or \emph{ternary}, respectively).

If \texttt{A} is a set and \texttt{f} is a (\texttt{ρ\ f})-ary operation
on \texttt{A}, then the arguments of \texttt{f} form a
(\texttt{ρ\ f})-tuple, say,
\texttt{(a\ 0,\ a\ 1,\ \ldots{},\ a\ (ρf-1))}, which may be viewed as
the graph of a function, say, \texttt{a\ :\ ρf\ →\ A}. When the codomain
of \texttt{ρ} is \texttt{ℕ}, we may view \texttt{ρ\ f} as the finite set
\texttt{\{0,\ 1,\ \ldots{},\ ρf\ -\ 1\}}. Thus, by identifying the
\texttt{ρf}-th power of \texttt{A} with the type \texttt{ρ\ f\ →\ A} of
functions from \texttt{\{0,\ 1,\ \ldots{},\ ρf\ -\ 1\}} to \texttt{A},
we identify the collection of all tuples of arguments of \texttt{f} with
the function type \texttt{(ρf\ →\ A)\ →\ A}.

\hypertarget{signature-type}{%
\paragraph{Signature type}\label{signature-type}}

The \href{https://github.com/ualib/agda-algebras}{agda-algebras} library
represents a \emph{signature} as an inhabitant of the following
dependent pair type.

\begin{verbatim}
Signature : (𝓞 𝓥 : Level) → Type (lsuc (𝓞 ⊔ 𝓥))
Signature 𝓞 𝓥 = Σ[ F ∈ Type 𝓞 ] (F → Type 𝓥)
\end{verbatim}

Using special syntax for the first and second
projections---\texttt{∣\_∣} and \texttt{∥\_∥}, respectively---if
\texttt{𝑆\ :\ Signature\ 𝓞\ 𝓥} is a signature, then

\begin{itemize}
\tightlist
\item
  \texttt{∣\ 𝑆\ ∣} denotes the set of operation symbols;
\item
  \texttt{∥\ 𝑆\ ∥} denotes the arity function.
\end{itemize}

Thus, if \texttt{𝑓\ :\ ∣\ 𝑆\ ∣} is an operation symbol in the signature
\texttt{𝑆}, then \texttt{∥\ 𝑆\ ∥\ 𝑓} is the arity of \texttt{𝑓}.

We need to augment the ordinary \texttt{Signature} type so that it
supports algebras over setoid domains. To do so, we define an operator
\texttt{⟨\_⟩} which translates an ordinary signature into a setoid
signature, that is, a signature over a setoid domain. But first we must
resolve a techinical issue involving dependent types that we now
describe.

Suppose we are given two operations \texttt{f} and \texttt{g} and we
have a tuple of arguments for \texttt{f}, say,
\texttt{u\ :\ ∥\ 𝑆\ ∥\ f\ →\ A}, and a tuple of arguments for
\texttt{g}, say, \texttt{v\ :\ ∥\ 𝑆\ ∥\ g\ →\ A}. If we know that
\texttt{f} is identically equal to \texttt{g}---that is,
\texttt{f\ ≡\ g} (intensionally)---then we should be able to check
whether \texttt{u} and \texttt{v} are pointwise equal. The problem here
is that \texttt{u} and \texttt{v} ostensibly inhabit different types. To
compare \texttt{u} and \texttt{v} we must convince Agda that, from
\texttt{f\ ≡\ g} we can deduce that \texttt{u} and \texttt{v} are
actually of the same type. The type \texttt{EqArgs} (defined below, and
adapted from Andreas Abel's development {[}ref needed{]}) resolves this
technical issue nicely.

\begin{code}%
\>[0]\<%
\\
\>[0]\AgdaFunction{EqArgs}\AgdaSpace{}%
\AgdaSymbol{:}%
\>[10]\AgdaSymbol{\{}\AgdaBound{𝑆}\AgdaSpace{}%
\AgdaSymbol{:}\AgdaSpace{}%
\AgdaFunction{Signature}\AgdaSpace{}%
\AgdaBound{𝓞}\AgdaSpace{}%
\AgdaBound{𝓥}\AgdaSymbol{\}\{}\AgdaBound{ξ}\AgdaSpace{}%
\AgdaSymbol{:}\AgdaSpace{}%
\AgdaRecord{Setoid}\AgdaSpace{}%
\AgdaGeneralizable{α}\AgdaSpace{}%
\AgdaGeneralizable{ρᵃ}\AgdaSymbol{\}}\<%
\\
\>[0][@{}l@{\AgdaIndent{0}}]%
\>[1]\AgdaSymbol{→}%
\>[10]\AgdaSymbol{∀}\AgdaSpace{}%
\AgdaSymbol{\{}\AgdaBound{f}\AgdaSpace{}%
\AgdaBound{g}\AgdaSymbol{\}}\AgdaSpace{}%
\AgdaSymbol{→}\AgdaSpace{}%
\AgdaBound{f}\AgdaSpace{}%
\AgdaOperator{\AgdaDatatype{≡}}\AgdaSpace{}%
\AgdaBound{g}\AgdaSpace{}%
\AgdaSymbol{→}\AgdaSpace{}%
\AgdaSymbol{(}\AgdaOperator{\AgdaFunction{∥}}\AgdaSpace{}%
\AgdaBound{𝑆}\AgdaSpace{}%
\AgdaOperator{\AgdaFunction{∥}}\AgdaSpace{}%
\AgdaBound{f}\AgdaSpace{}%
\AgdaSymbol{→}\AgdaSpace{}%
\AgdaField{Carrier}\AgdaSpace{}%
\AgdaBound{ξ}\AgdaSymbol{)}\AgdaSpace{}%
\AgdaSymbol{→}\AgdaSpace{}%
\AgdaSymbol{(}\AgdaOperator{\AgdaFunction{∥}}\AgdaSpace{}%
\AgdaBound{𝑆}\AgdaSpace{}%
\AgdaOperator{\AgdaFunction{∥}}\AgdaSpace{}%
\AgdaBound{g}\AgdaSpace{}%
\AgdaSymbol{→}\AgdaSpace{}%
\AgdaField{Carrier}\AgdaSpace{}%
\AgdaBound{ξ}\AgdaSymbol{)}\AgdaSpace{}%
\AgdaSymbol{→}\AgdaSpace{}%
\AgdaPrimitive{Type}\AgdaSpace{}%
\AgdaSymbol{(}\AgdaBound{𝓥}\AgdaSpace{}%
\AgdaOperator{\AgdaPrimitive{⊔}}\AgdaSpace{}%
\AgdaGeneralizable{ρᵃ}\AgdaSymbol{)}\<%
\\
%
\\[\AgdaEmptyExtraSkip]%
\>[0]\AgdaFunction{EqArgs}\AgdaSpace{}%
\AgdaSymbol{\{}\AgdaArgument{ξ}\AgdaSpace{}%
\AgdaSymbol{=}\AgdaSpace{}%
\AgdaBound{ξ}\AgdaSymbol{\}}\AgdaSpace{}%
\AgdaInductiveConstructor{≡.refl}\AgdaSpace{}%
\AgdaBound{u}\AgdaSpace{}%
\AgdaBound{v}\AgdaSpace{}%
\AgdaSymbol{=}\AgdaSpace{}%
\AgdaSymbol{∀}\AgdaSpace{}%
\AgdaBound{i}\AgdaSpace{}%
\AgdaSymbol{→}\AgdaSpace{}%
\AgdaBound{u}\AgdaSpace{}%
\AgdaBound{i}\AgdaSpace{}%
\AgdaOperator{\AgdaFunction{≈}}\AgdaSpace{}%
\AgdaBound{v}\AgdaSpace{}%
\AgdaBound{i}\AgdaSpace{}%
\AgdaKeyword{where}\AgdaSpace{}%
\AgdaKeyword{open}\AgdaSpace{}%
\AgdaModule{Setoid}\AgdaSpace{}%
\AgdaBound{ξ}\AgdaSpace{}%
\AgdaKeyword{using}\AgdaSpace{}%
\AgdaSymbol{(}\AgdaSpace{}%
\AgdaOperator{\AgdaField{\AgdaUnderscore{}≈\AgdaUnderscore{}}}\AgdaSpace{}%
\AgdaSymbol{)}\<%
\\
\>[0]\<%
\end{code}

Now we are in a position to define the \texttt{⟨\_⟩} operator, which
translates an ordinary signature into a setoid signature.

\begin{code}%
\>[0]\<%
\\
\>[0]\AgdaKeyword{module}\AgdaSpace{}%
\AgdaModule{\AgdaUnderscore{}}\AgdaSpace{}%
\AgdaKeyword{where}\<%
\\
\>[0][@{}l@{\AgdaIndent{0}}]%
\>[1]\AgdaKeyword{open}\AgdaSpace{}%
\AgdaModule{Setoid}%
\>[20]\AgdaKeyword{using}\AgdaSpace{}%
\AgdaSymbol{(}\AgdaSpace{}%
\AgdaOperator{\AgdaField{\AgdaUnderscore{}≈\AgdaUnderscore{}}}\AgdaSpace{}%
\AgdaSymbol{)}\<%
\\
%
\>[1]\AgdaKeyword{open}\AgdaSpace{}%
\AgdaModule{IsEquivalence}\AgdaSpace{}%
\AgdaKeyword{using}\AgdaSpace{}%
\AgdaSymbol{(}\AgdaSpace{}%
\AgdaField{refl}\AgdaSpace{}%
\AgdaSymbol{;}\AgdaSpace{}%
\AgdaField{sym}\AgdaSpace{}%
\AgdaSymbol{;}\AgdaSpace{}%
\AgdaField{trans}\AgdaSpace{}%
\AgdaSymbol{)}\<%
\\
%
\\[\AgdaEmptyExtraSkip]%
%
\>[1]\AgdaOperator{\AgdaFunction{⟨\AgdaUnderscore{}⟩}}\AgdaSpace{}%
\AgdaSymbol{:}\AgdaSpace{}%
\AgdaFunction{Signature}\AgdaSpace{}%
\AgdaBound{𝓞}\AgdaSpace{}%
\AgdaBound{𝓥}\AgdaSpace{}%
\AgdaSymbol{→}\AgdaSpace{}%
\AgdaRecord{Setoid}\AgdaSpace{}%
\AgdaGeneralizable{α}\AgdaSpace{}%
\AgdaGeneralizable{ρᵃ}\AgdaSpace{}%
\AgdaSymbol{→}\AgdaSpace{}%
\AgdaRecord{Setoid}\AgdaSpace{}%
\AgdaSymbol{\AgdaUnderscore{}}\AgdaSpace{}%
\AgdaSymbol{\AgdaUnderscore{}}\<%
\\
%
\\[\AgdaEmptyExtraSkip]%
%
\>[1]\AgdaField{Carrier}\AgdaSpace{}%
\AgdaSymbol{(}\AgdaOperator{\AgdaFunction{⟨}}\AgdaSpace{}%
\AgdaBound{𝑆}\AgdaSpace{}%
\AgdaOperator{\AgdaFunction{⟩}}\AgdaSpace{}%
\AgdaBound{ξ}\AgdaSymbol{)}%
\>[32]\AgdaSymbol{=}\AgdaSpace{}%
\AgdaFunction{Σ[}\AgdaSpace{}%
\AgdaBound{f}\AgdaSpace{}%
\AgdaFunction{∈}\AgdaSpace{}%
\AgdaOperator{\AgdaFunction{∣}}\AgdaSpace{}%
\AgdaBound{𝑆}\AgdaSpace{}%
\AgdaOperator{\AgdaFunction{∣}}\AgdaSpace{}%
\AgdaFunction{]}\AgdaSpace{}%
\AgdaSymbol{((}\AgdaOperator{\AgdaFunction{∥}}\AgdaSpace{}%
\AgdaBound{𝑆}\AgdaSpace{}%
\AgdaOperator{\AgdaFunction{∥}}\AgdaSpace{}%
\AgdaBound{f}\AgdaSymbol{)}\AgdaSpace{}%
\AgdaSymbol{→}\AgdaSpace{}%
\AgdaBound{ξ}\AgdaSpace{}%
\AgdaSymbol{.}\AgdaField{Carrier}\AgdaSymbol{)}\<%
\\
%
\\[\AgdaEmptyExtraSkip]%
%
\>[1]\AgdaOperator{\AgdaField{\AgdaUnderscore{}≈\AgdaUnderscore{}}}\AgdaSpace{}%
\AgdaSymbol{(}\AgdaOperator{\AgdaFunction{⟨}}\AgdaSpace{}%
\AgdaBound{𝑆}\AgdaSpace{}%
\AgdaOperator{\AgdaFunction{⟩}}\AgdaSpace{}%
\AgdaBound{ξ}\AgdaSymbol{)}\AgdaSpace{}%
\AgdaSymbol{(}\AgdaBound{f}\AgdaSpace{}%
\AgdaOperator{\AgdaInductiveConstructor{,}}\AgdaSpace{}%
\AgdaBound{u}\AgdaSymbol{)}\AgdaSpace{}%
\AgdaSymbol{(}\AgdaBound{g}\AgdaSpace{}%
\AgdaOperator{\AgdaInductiveConstructor{,}}\AgdaSpace{}%
\AgdaBound{v}\AgdaSymbol{)}%
\>[32]\AgdaSymbol{=}\AgdaSpace{}%
\AgdaFunction{Σ[}\AgdaSpace{}%
\AgdaBound{eqv}\AgdaSpace{}%
\AgdaFunction{∈}\AgdaSpace{}%
\AgdaBound{f}\AgdaSpace{}%
\AgdaOperator{\AgdaDatatype{≡}}\AgdaSpace{}%
\AgdaBound{g}\AgdaSpace{}%
\AgdaFunction{]}\AgdaSpace{}%
\AgdaFunction{EqArgs}\AgdaSymbol{\{}\AgdaArgument{ξ}\AgdaSpace{}%
\AgdaSymbol{=}\AgdaSpace{}%
\AgdaBound{ξ}\AgdaSymbol{\}}\AgdaSpace{}%
\AgdaBound{eqv}\AgdaSpace{}%
\AgdaBound{u}\AgdaSpace{}%
\AgdaBound{v}\<%
\\
%
\\[\AgdaEmptyExtraSkip]%
%
\>[1]\AgdaField{refl}%
\>[8]\AgdaSymbol{(}\AgdaField{isEquivalence}\AgdaSpace{}%
\AgdaSymbol{(}\AgdaOperator{\AgdaFunction{⟨}}\AgdaSpace{}%
\AgdaBound{𝑆}\AgdaSpace{}%
\AgdaOperator{\AgdaFunction{⟩}}\AgdaSpace{}%
\AgdaBound{ξ}\AgdaSymbol{))}%
\>[60]\AgdaSymbol{=}\AgdaSpace{}%
\AgdaInductiveConstructor{≡.refl}\AgdaSpace{}%
\AgdaOperator{\AgdaInductiveConstructor{,}}\AgdaSpace{}%
\AgdaSymbol{λ}\AgdaSpace{}%
\AgdaBound{i}\AgdaSpace{}%
\AgdaSymbol{→}\AgdaSpace{}%
\AgdaFunction{Setoid.refl}%
\>[91]\AgdaBound{ξ}\<%
\\
%
\>[1]\AgdaField{sym}%
\>[8]\AgdaSymbol{(}\AgdaField{isEquivalence}\AgdaSpace{}%
\AgdaSymbol{(}\AgdaOperator{\AgdaFunction{⟨}}\AgdaSpace{}%
\AgdaBound{𝑆}\AgdaSpace{}%
\AgdaOperator{\AgdaFunction{⟩}}\AgdaSpace{}%
\AgdaBound{ξ}\AgdaSymbol{))}\AgdaSpace{}%
\AgdaSymbol{(}\AgdaInductiveConstructor{≡.refl}\AgdaSpace{}%
\AgdaOperator{\AgdaInductiveConstructor{,}}\AgdaSpace{}%
\AgdaBound{g}\AgdaSymbol{)}%
\>[60]\AgdaSymbol{=}\AgdaSpace{}%
\AgdaInductiveConstructor{≡.refl}\AgdaSpace{}%
\AgdaOperator{\AgdaInductiveConstructor{,}}\AgdaSpace{}%
\AgdaSymbol{λ}\AgdaSpace{}%
\AgdaBound{i}\AgdaSpace{}%
\AgdaSymbol{→}\AgdaSpace{}%
\AgdaFunction{Setoid.sym}%
\>[91]\AgdaBound{ξ}\AgdaSpace{}%
\AgdaSymbol{(}\AgdaBound{g}\AgdaSpace{}%
\AgdaBound{i}\AgdaSymbol{)}\<%
\\
%
\>[1]\AgdaField{trans}%
\>[8]\AgdaSymbol{(}\AgdaField{isEquivalence}\AgdaSpace{}%
\AgdaSymbol{(}\AgdaOperator{\AgdaFunction{⟨}}\AgdaSpace{}%
\AgdaBound{𝑆}\AgdaSpace{}%
\AgdaOperator{\AgdaFunction{⟩}}\AgdaSpace{}%
\AgdaBound{ξ}\AgdaSymbol{))}\AgdaSpace{}%
\AgdaSymbol{(}\AgdaInductiveConstructor{≡.refl}\AgdaSpace{}%
\AgdaOperator{\AgdaInductiveConstructor{,}}\AgdaSpace{}%
\AgdaBound{g}\AgdaSymbol{)(}\AgdaInductiveConstructor{≡.refl}\AgdaSpace{}%
\AgdaOperator{\AgdaInductiveConstructor{,}}\AgdaSpace{}%
\AgdaBound{h}\AgdaSymbol{)}%
\>[60]\AgdaSymbol{=}\AgdaSpace{}%
\AgdaInductiveConstructor{≡.refl}\AgdaSpace{}%
\AgdaOperator{\AgdaInductiveConstructor{,}}\AgdaSpace{}%
\AgdaSymbol{λ}\AgdaSpace{}%
\AgdaBound{i}\AgdaSpace{}%
\AgdaSymbol{→}\AgdaSpace{}%
\AgdaFunction{Setoid.trans}%
\>[91]\AgdaBound{ξ}\AgdaSpace{}%
\AgdaSymbol{(}\AgdaBound{g}\AgdaSpace{}%
\AgdaBound{i}\AgdaSymbol{)}\AgdaSpace{}%
\AgdaSymbol{(}\AgdaBound{h}\AgdaSpace{}%
\AgdaBound{i}\AgdaSymbol{)}\<%
\end{code}

\hypertarget{algebra-type}{%
\subsubsection{Algebra type}\label{algebra-type}}

Informally, an \emph{algebraic structure in the signature}
\texttt{𝑆\ =\ (F,\ ρ)} (or \texttt{𝑆}-\emph{algebra}) is typically
denoted by \texttt{𝑨\ =\ (A,\ Fᴬ)} and consists of

\begin{itemize}
\tightlist
\item
  \texttt{A} := a \emph{nonempty} set (or type), called the
  \emph{domain} (or \emph{carrier} or \emph{universe}) of the algebra;
\item
  \texttt{Fᴬ} :=
  \texttt{\{\ fᴬ\ ∣\ f\ ∈\ F,\ fᴬ\ :\ (ρf\ →\ A)\ →\ A\ \}}, a
  collection of \emph{operations} on \texttt{𝐴};
\item
  a (potentially empty) collection of \emph{identities} satisfied by
  elements and operations of \texttt{𝐴}.
\end{itemize}

We represent an algebra in Agda using a record type with two fields:

\begin{itemize}
\tightlist
\item
  \texttt{Domain} is a setoid denoting the underlying \emph{universe} of
  the algebra (informally, the set of elements of the algebra);
\item
  \texttt{Interp} represents the \emph{interpretation} in the algebra of
  each operation symbol of the given signature. The record type
  \texttt{Func} from the Agda Standard Library provides what we need for
  an operation on the domain setoid.
\end{itemize}

Let us present the definition of the \texttt{Algebra} type and then
discuss the definition of the \texttt{Func} type that provides the
interpretation of each operation symbol.

\begin{code}%
\>[0]\<%
\\
\>[0]\AgdaKeyword{record}\AgdaSpace{}%
\AgdaRecord{Algebra}\AgdaSpace{}%
\AgdaBound{α}\AgdaSpace{}%
\AgdaBound{ρ}\AgdaSpace{}%
\AgdaSymbol{:}\AgdaSpace{}%
\AgdaPrimitive{Type}\AgdaSpace{}%
\AgdaSymbol{(}\AgdaBound{𝓞}\AgdaSpace{}%
\AgdaOperator{\AgdaPrimitive{⊔}}\AgdaSpace{}%
\AgdaBound{𝓥}\AgdaSpace{}%
\AgdaOperator{\AgdaPrimitive{⊔}}\AgdaSpace{}%
\AgdaPrimitive{lsuc}\AgdaSpace{}%
\AgdaSymbol{(}\AgdaBound{α}\AgdaSpace{}%
\AgdaOperator{\AgdaPrimitive{⊔}}\AgdaSpace{}%
\AgdaBound{ρ}\AgdaSymbol{))}\AgdaSpace{}%
\AgdaKeyword{where}\<%
\\
\>[0][@{}l@{\AgdaIndent{0}}]%
\>[1]\AgdaKeyword{field}%
\>[8]\AgdaField{Domain}\AgdaSpace{}%
\AgdaSymbol{:}\AgdaSpace{}%
\AgdaRecord{Setoid}\AgdaSpace{}%
\AgdaBound{α}\AgdaSpace{}%
\AgdaBound{ρ}\<%
\\
%
\>[8]\AgdaField{Interp}\AgdaSpace{}%
\AgdaSymbol{:}\AgdaSpace{}%
\AgdaSymbol{(}\AgdaOperator{\AgdaFunction{⟨}}\AgdaSpace{}%
\AgdaBound{𝑆}\AgdaSpace{}%
\AgdaOperator{\AgdaFunction{⟩}}\AgdaSpace{}%
\AgdaField{Domain}\AgdaSymbol{)}\AgdaSpace{}%
\AgdaOperator{\AgdaRecord{⟶}}\AgdaSpace{}%
\AgdaField{Domain}\<%
\\
\>[0]\<%
\end{code}

The \texttt{Interp} field actually has type
\texttt{Func\ (⟨\ 𝑆\ ⟩\ Domain)\ Domain} (recall we renamed
\texttt{Func} as the infix long-arrow symbol). The \texttt{Func} type is
from the standard library and is defined as a record type with two
fields. In the present instance, the fields are

\begin{enumerate}
\def\labelenumi{\arabic{enumi}.}
\tightlist
\item
  a function
  \texttt{f\ :\ Carrier\ (⟨\ 𝑆\ ⟩\ Domain)\ →\ Carrier\ Domain}
\item
  a proof \texttt{cong\ :\ f\ Preserves\ \_≈₁\_\ ⟶\ \_≈₂\_} that
  \texttt{f} preserves the underlying setoid equalities.
\end{enumerate}

Thus \texttt{Interp} gives, for each operation symbol in the signature
\texttt{𝑆}, a setoid function \texttt{f}---namely, a function where the
domain is a power of \texttt{Domain} and the codomain is
\texttt{Domain}---along with a proof that all operations so interpreted
respect the underlying setoid equality on \texttt{Domain}.

Next we define some syntactic sugar that will make our Agda code easier
to read and comprehend. If \texttt{𝑨} is an algebra, then

\begin{itemize}
\tightlist
\item
  \texttt{f\ ̂\ 𝑨} will denote the interpretation in the algebra
  \texttt{𝑨} of the operation symbol \texttt{f},
\item
  \texttt{𝔻{[}\ 𝑨\ {]}} will denote the setoid \texttt{Domain\ 𝑨}, and
\item
  \texttt{𝕌{[}\ 𝑨\ {]}} will be the underlying carrier or ``universe''
  of the algebra \texttt{𝑨}.
\end{itemize}

\begin{code}%
\>[0]\<%
\\
\>[0]\AgdaKeyword{open}\AgdaSpace{}%
\AgdaModule{Algebra}\<%
\\
%
\\[\AgdaEmptyExtraSkip]%
\>[0]\AgdaOperator{\AgdaFunction{𝕌[\AgdaUnderscore{}]}}\AgdaSpace{}%
\AgdaSymbol{:}\AgdaSpace{}%
\AgdaRecord{Algebra}\AgdaSpace{}%
\AgdaGeneralizable{α}\AgdaSpace{}%
\AgdaGeneralizable{ρᵃ}\AgdaSpace{}%
\AgdaSymbol{→}%
\>[23]\AgdaPrimitive{Type}\AgdaSpace{}%
\AgdaGeneralizable{α}\<%
\\
\>[0]\AgdaOperator{\AgdaFunction{𝕌[}}\AgdaSpace{}%
\AgdaBound{𝑨}\AgdaSpace{}%
\AgdaOperator{\AgdaFunction{]}}\AgdaSpace{}%
\AgdaSymbol{=}\AgdaSpace{}%
\AgdaField{Carrier}\AgdaSpace{}%
\AgdaSymbol{(}\AgdaField{Domain}\AgdaSpace{}%
\AgdaBound{𝑨}\AgdaSymbol{)}\<%
\\
%
\\[\AgdaEmptyExtraSkip]%
\>[0]\AgdaOperator{\AgdaFunction{𝔻[\AgdaUnderscore{}]}}\AgdaSpace{}%
\AgdaSymbol{:}\AgdaSpace{}%
\AgdaRecord{Algebra}\AgdaSpace{}%
\AgdaGeneralizable{α}\AgdaSpace{}%
\AgdaGeneralizable{ρᵃ}\AgdaSpace{}%
\AgdaSymbol{→}%
\>[23]\AgdaRecord{Setoid}\AgdaSpace{}%
\AgdaGeneralizable{α}\AgdaSpace{}%
\AgdaGeneralizable{ρᵃ}\<%
\\
\>[0]\AgdaOperator{\AgdaFunction{𝔻[}}\AgdaSpace{}%
\AgdaBound{𝑨}\AgdaSpace{}%
\AgdaOperator{\AgdaFunction{]}}\AgdaSpace{}%
\AgdaSymbol{=}\AgdaSpace{}%
\AgdaField{Domain}\AgdaSpace{}%
\AgdaBound{𝑨}\<%
\\
%
\\[\AgdaEmptyExtraSkip]%
\>[0]\AgdaOperator{\AgdaFunction{\AgdaUnderscore{}̂\AgdaUnderscore{}}}\AgdaSpace{}%
\AgdaSymbol{:}\AgdaSpace{}%
\AgdaSymbol{(}\AgdaBound{f}\AgdaSpace{}%
\AgdaSymbol{:}\AgdaSpace{}%
\AgdaOperator{\AgdaFunction{∣}}\AgdaSpace{}%
\AgdaBound{𝑆}\AgdaSpace{}%
\AgdaOperator{\AgdaFunction{∣}}\AgdaSymbol{)(}\AgdaBound{𝑨}\AgdaSpace{}%
\AgdaSymbol{:}\AgdaSpace{}%
\AgdaRecord{Algebra}\AgdaSpace{}%
\AgdaGeneralizable{α}\AgdaSpace{}%
\AgdaGeneralizable{ρᵃ}\AgdaSymbol{)}\AgdaSpace{}%
\AgdaSymbol{→}\AgdaSpace{}%
\AgdaSymbol{(}\AgdaOperator{\AgdaFunction{∥}}\AgdaSpace{}%
\AgdaBound{𝑆}\AgdaSpace{}%
\AgdaOperator{\AgdaFunction{∥}}\AgdaSpace{}%
\AgdaBound{f}%
\>[48]\AgdaSymbol{→}%
\>[51]\AgdaOperator{\AgdaFunction{𝕌[}}\AgdaSpace{}%
\AgdaBound{𝑨}\AgdaSpace{}%
\AgdaOperator{\AgdaFunction{]}}\AgdaSymbol{)}\AgdaSpace{}%
\AgdaSymbol{→}\AgdaSpace{}%
\AgdaOperator{\AgdaFunction{𝕌[}}\AgdaSpace{}%
\AgdaBound{𝑨}\AgdaSpace{}%
\AgdaOperator{\AgdaFunction{]}}\<%
\\
%
\\[\AgdaEmptyExtraSkip]%
\>[0]\AgdaBound{f}\AgdaSpace{}%
\AgdaOperator{\AgdaFunction{̂}}\AgdaSpace{}%
\AgdaBound{𝑨}\AgdaSpace{}%
\AgdaSymbol{=}\AgdaSpace{}%
\AgdaSymbol{λ}\AgdaSpace{}%
\AgdaBound{a}\AgdaSpace{}%
\AgdaSymbol{→}\AgdaSpace{}%
\AgdaSymbol{(}\AgdaField{Interp}\AgdaSpace{}%
\AgdaBound{𝑨}\AgdaSymbol{)}\AgdaSpace{}%
\AgdaOperator{\AgdaField{⟨\$⟩}}\AgdaSpace{}%
\AgdaSymbol{(}\AgdaBound{f}\AgdaSpace{}%
\AgdaOperator{\AgdaInductiveConstructor{,}}\AgdaSpace{}%
\AgdaBound{a}\AgdaSymbol{)}\<%
\end{code}

\hypertarget{universe-lifting-of-algebra-types}{%
\subsubsection{Universe lifting of algebra
types}\label{universe-lifting-of-algebra-types}}

A technical aspect of dealing with the noncumulativity of the hierarchy
of type levels in Agda\ldots{}

\begin{code}%
\>[0]\<%
\\
\>[0]\AgdaKeyword{module}\AgdaSpace{}%
\AgdaModule{\AgdaUnderscore{}}\AgdaSpace{}%
\AgdaSymbol{(}\AgdaBound{𝑨}\AgdaSpace{}%
\AgdaSymbol{:}\AgdaSpace{}%
\AgdaRecord{Algebra}\AgdaSpace{}%
\AgdaGeneralizable{α}\AgdaSpace{}%
\AgdaGeneralizable{ρᵃ}\AgdaSymbol{)}\AgdaSpace{}%
\AgdaKeyword{where}\<%
\\
\>[0][@{}l@{\AgdaIndent{0}}]%
\>[1]\AgdaKeyword{open}\AgdaSpace{}%
\AgdaModule{Setoid}\AgdaSpace{}%
\AgdaOperator{\AgdaFunction{𝔻[}}\AgdaSpace{}%
\AgdaBound{𝑨}\AgdaSpace{}%
\AgdaOperator{\AgdaFunction{]}}\AgdaSpace{}%
\AgdaKeyword{using}\AgdaSpace{}%
\AgdaSymbol{(}\AgdaSpace{}%
\AgdaOperator{\AgdaField{\AgdaUnderscore{}≈\AgdaUnderscore{}}}\AgdaSpace{}%
\AgdaSymbol{;}\AgdaSpace{}%
\AgdaFunction{refl}\AgdaSpace{}%
\AgdaSymbol{;}\AgdaSpace{}%
\AgdaFunction{sym}\AgdaSpace{}%
\AgdaSymbol{;}\AgdaSpace{}%
\AgdaFunction{trans}\AgdaSpace{}%
\AgdaSymbol{)}\<%
\\
%
\>[1]\AgdaKeyword{open}\AgdaSpace{}%
\AgdaModule{Level}\<%
\\
%
\\[\AgdaEmptyExtraSkip]%
%
\>[1]\AgdaFunction{Lift-Algˡ}\AgdaSpace{}%
\AgdaSymbol{:}\AgdaSpace{}%
\AgdaSymbol{(}\AgdaBound{ℓ}\AgdaSpace{}%
\AgdaSymbol{:}\AgdaSpace{}%
\AgdaPostulate{Level}\AgdaSymbol{)}\AgdaSpace{}%
\AgdaSymbol{→}\AgdaSpace{}%
\AgdaRecord{Algebra}\AgdaSpace{}%
\AgdaSymbol{(}\AgdaBound{α}\AgdaSpace{}%
\AgdaOperator{\AgdaPrimitive{⊔}}\AgdaSpace{}%
\AgdaBound{ℓ}\AgdaSymbol{)}\AgdaSpace{}%
\AgdaBound{ρᵃ}\<%
\\
%
\>[1]\AgdaField{Domain}\AgdaSpace{}%
\AgdaSymbol{(}\AgdaFunction{Lift-Algˡ}\AgdaSpace{}%
\AgdaBound{ℓ}\AgdaSymbol{)}\AgdaSpace{}%
\AgdaSymbol{=}\<%
\\
\>[1][@{}l@{\AgdaIndent{0}}]%
\>[2]\AgdaKeyword{record}%
\>[11]\AgdaSymbol{\{}\AgdaSpace{}%
\AgdaField{Carrier}\AgdaSpace{}%
\AgdaSymbol{=}\AgdaSpace{}%
\AgdaRecord{Lift}\AgdaSpace{}%
\AgdaBound{ℓ}\AgdaSpace{}%
\AgdaOperator{\AgdaFunction{𝕌[}}\AgdaSpace{}%
\AgdaBound{𝑨}\AgdaSpace{}%
\AgdaOperator{\AgdaFunction{]}}\<%
\\
%
\>[11]\AgdaSymbol{;}\AgdaSpace{}%
\AgdaOperator{\AgdaField{\AgdaUnderscore{}≈\AgdaUnderscore{}}}\AgdaSpace{}%
\AgdaSymbol{=}\AgdaSpace{}%
\AgdaSymbol{λ}\AgdaSpace{}%
\AgdaBound{x}\AgdaSpace{}%
\AgdaBound{y}\AgdaSpace{}%
\AgdaSymbol{→}\AgdaSpace{}%
\AgdaField{lower}\AgdaSpace{}%
\AgdaBound{x}\AgdaSpace{}%
\AgdaOperator{\AgdaFunction{≈}}\AgdaSpace{}%
\AgdaField{lower}\AgdaSpace{}%
\AgdaBound{y}\<%
\\
%
\>[11]\AgdaSymbol{;}\AgdaSpace{}%
\AgdaField{isEquivalence}\AgdaSpace{}%
\AgdaSymbol{=}\AgdaSpace{}%
\AgdaKeyword{record}\AgdaSpace{}%
\AgdaSymbol{\{}\AgdaSpace{}%
\AgdaField{refl}\AgdaSpace{}%
\AgdaSymbol{=}\AgdaSpace{}%
\AgdaFunction{refl}\AgdaSpace{}%
\AgdaSymbol{;}\AgdaSpace{}%
\AgdaField{sym}\AgdaSpace{}%
\AgdaSymbol{=}\AgdaSpace{}%
\AgdaFunction{sym}\AgdaSpace{}%
\AgdaSymbol{;}\AgdaSpace{}%
\AgdaField{trans}\AgdaSpace{}%
\AgdaSymbol{=}\AgdaSpace{}%
\AgdaFunction{trans}\AgdaSpace{}%
\AgdaSymbol{\}\}}\<%
\\
%
\>[1]\AgdaField{Interp}\AgdaSpace{}%
\AgdaSymbol{(}\AgdaFunction{Lift-Algˡ}\AgdaSpace{}%
\AgdaBound{ℓ}\AgdaSymbol{)}\AgdaSpace{}%
\AgdaOperator{\AgdaField{⟨\$⟩}}\AgdaSpace{}%
\AgdaSymbol{(}\AgdaBound{f}\AgdaSpace{}%
\AgdaOperator{\AgdaInductiveConstructor{,}}\AgdaSpace{}%
\AgdaBound{la}\AgdaSymbol{)}\AgdaSpace{}%
\AgdaSymbol{=}\AgdaSpace{}%
\AgdaInductiveConstructor{lift}\AgdaSpace{}%
\AgdaSymbol{((}\AgdaBound{f}\AgdaSpace{}%
\AgdaOperator{\AgdaFunction{̂}}\AgdaSpace{}%
\AgdaBound{𝑨}\AgdaSymbol{)}\AgdaSpace{}%
\AgdaSymbol{(}\AgdaField{lower}\AgdaSpace{}%
\AgdaOperator{\AgdaFunction{∘}}\AgdaSpace{}%
\AgdaBound{la}\AgdaSymbol{))}\<%
\\
%
\>[1]\AgdaField{cong}\AgdaSpace{}%
\AgdaSymbol{(}\AgdaField{Interp}\AgdaSpace{}%
\AgdaSymbol{(}\AgdaFunction{Lift-Algˡ}\AgdaSpace{}%
\AgdaBound{ℓ}\AgdaSymbol{))}\AgdaSpace{}%
\AgdaSymbol{(}\AgdaInductiveConstructor{≡.refl}\AgdaSpace{}%
\AgdaOperator{\AgdaInductiveConstructor{,}}\AgdaSpace{}%
\AgdaBound{la=lb}\AgdaSymbol{)}\AgdaSpace{}%
\AgdaSymbol{=}\AgdaSpace{}%
\AgdaField{cong}\AgdaSpace{}%
\AgdaSymbol{(}\AgdaField{Interp}\AgdaSpace{}%
\AgdaBound{𝑨}\AgdaSymbol{)}\AgdaSpace{}%
\AgdaSymbol{((}\AgdaInductiveConstructor{≡.refl}\AgdaSpace{}%
\AgdaOperator{\AgdaInductiveConstructor{,}}\AgdaSpace{}%
\AgdaBound{la=lb}\AgdaSymbol{))}\<%
\\
%
\\[\AgdaEmptyExtraSkip]%
%
\>[1]\AgdaFunction{Lift-Algʳ}\AgdaSpace{}%
\AgdaSymbol{:}\AgdaSpace{}%
\AgdaSymbol{(}\AgdaBound{ℓ}\AgdaSpace{}%
\AgdaSymbol{:}\AgdaSpace{}%
\AgdaPostulate{Level}\AgdaSymbol{)}\AgdaSpace{}%
\AgdaSymbol{→}\AgdaSpace{}%
\AgdaRecord{Algebra}\AgdaSpace{}%
\AgdaBound{α}\AgdaSpace{}%
\AgdaSymbol{(}\AgdaBound{ρᵃ}\AgdaSpace{}%
\AgdaOperator{\AgdaPrimitive{⊔}}\AgdaSpace{}%
\AgdaBound{ℓ}\AgdaSymbol{)}\<%
\\
%
\>[1]\AgdaField{Domain}\AgdaSpace{}%
\AgdaSymbol{(}\AgdaFunction{Lift-Algʳ}\AgdaSpace{}%
\AgdaBound{ℓ}\AgdaSymbol{)}\AgdaSpace{}%
\AgdaSymbol{=}\<%
\\
\>[1][@{}l@{\AgdaIndent{0}}]%
\>[2]\AgdaKeyword{record}%
\>[10]\AgdaSymbol{\{}\AgdaSpace{}%
\AgdaField{Carrier}\AgdaSpace{}%
\AgdaSymbol{=}\AgdaSpace{}%
\AgdaOperator{\AgdaFunction{𝕌[}}\AgdaSpace{}%
\AgdaBound{𝑨}\AgdaSpace{}%
\AgdaOperator{\AgdaFunction{]}}\<%
\\
%
\>[10]\AgdaSymbol{;}\AgdaSpace{}%
\AgdaOperator{\AgdaField{\AgdaUnderscore{}≈\AgdaUnderscore{}}}\AgdaSpace{}%
\AgdaSymbol{=}\AgdaSpace{}%
\AgdaSymbol{λ}\AgdaSpace{}%
\AgdaBound{x}\AgdaSpace{}%
\AgdaBound{y}\AgdaSpace{}%
\AgdaSymbol{→}\AgdaSpace{}%
\AgdaRecord{Lift}\AgdaSpace{}%
\AgdaBound{ℓ}\AgdaSpace{}%
\AgdaSymbol{(}\AgdaBound{x}\AgdaSpace{}%
\AgdaOperator{\AgdaFunction{≈}}\AgdaSpace{}%
\AgdaBound{y}\AgdaSymbol{)}\<%
\\
%
\>[10]\AgdaSymbol{;}\AgdaSpace{}%
\AgdaField{isEquivalence}\AgdaSpace{}%
\AgdaSymbol{=}\AgdaSpace{}%
\AgdaKeyword{record}%
\>[36]\AgdaSymbol{\{}\AgdaSpace{}%
\AgdaField{refl}%
\>[44]\AgdaSymbol{=}\AgdaSpace{}%
\AgdaInductiveConstructor{lift}\AgdaSpace{}%
\AgdaFunction{refl}\<%
\\
%
\>[36]\AgdaSymbol{;}\AgdaSpace{}%
\AgdaField{sym}%
\>[44]\AgdaSymbol{=}\AgdaSpace{}%
\AgdaSymbol{λ}\AgdaSpace{}%
\AgdaBound{x}\AgdaSpace{}%
\AgdaSymbol{→}\AgdaSpace{}%
\AgdaInductiveConstructor{lift}\AgdaSpace{}%
\AgdaSymbol{(}\AgdaFunction{sym}\AgdaSpace{}%
\AgdaSymbol{(}\AgdaField{lower}\AgdaSpace{}%
\AgdaBound{x}\AgdaSymbol{))}\<%
\\
%
\>[36]\AgdaSymbol{;}\AgdaSpace{}%
\AgdaField{trans}\AgdaSpace{}%
\AgdaSymbol{=}\AgdaSpace{}%
\AgdaSymbol{λ}\AgdaSpace{}%
\AgdaBound{x}\AgdaSpace{}%
\AgdaBound{y}\AgdaSpace{}%
\AgdaSymbol{→}\AgdaSpace{}%
\AgdaInductiveConstructor{lift}\AgdaSpace{}%
\AgdaSymbol{(}\AgdaFunction{trans}\AgdaSpace{}%
\AgdaSymbol{(}\AgdaField{lower}\AgdaSpace{}%
\AgdaBound{x}\AgdaSymbol{)}\AgdaSpace{}%
\AgdaSymbol{(}\AgdaField{lower}\AgdaSpace{}%
\AgdaBound{y}\AgdaSymbol{))}\AgdaSpace{}%
\AgdaSymbol{\}\}}\<%
\\
%
\>[1]\AgdaField{Interp}\AgdaSpace{}%
\AgdaSymbol{(}\AgdaFunction{Lift-Algʳ}\AgdaSpace{}%
\AgdaBound{ℓ}\AgdaSpace{}%
\AgdaSymbol{)}\AgdaSpace{}%
\AgdaOperator{\AgdaField{⟨\$⟩}}\AgdaSpace{}%
\AgdaSymbol{(}\AgdaBound{f}\AgdaSpace{}%
\AgdaOperator{\AgdaInductiveConstructor{,}}\AgdaSpace{}%
\AgdaBound{la}\AgdaSymbol{)}\AgdaSpace{}%
\AgdaSymbol{=}\AgdaSpace{}%
\AgdaSymbol{(}\AgdaBound{f}\AgdaSpace{}%
\AgdaOperator{\AgdaFunction{̂}}\AgdaSpace{}%
\AgdaBound{𝑨}\AgdaSymbol{)}\AgdaSpace{}%
\AgdaBound{la}\<%
\\
%
\>[1]\AgdaField{cong}\AgdaSpace{}%
\AgdaSymbol{(}\AgdaField{Interp}\AgdaSpace{}%
\AgdaSymbol{(}\AgdaFunction{Lift-Algʳ}\AgdaSpace{}%
\AgdaBound{ℓ}\AgdaSymbol{))(}\AgdaInductiveConstructor{≡.refl}\AgdaSpace{}%
\AgdaOperator{\AgdaInductiveConstructor{,}}\AgdaSpace{}%
\AgdaBound{la≡lb}\AgdaSymbol{)}\AgdaSpace{}%
\AgdaSymbol{=}\AgdaSpace{}%
\AgdaInductiveConstructor{lift}\AgdaSymbol{(}\AgdaField{cong}\AgdaSymbol{(}\AgdaField{Interp}\AgdaSpace{}%
\AgdaBound{𝑨}\AgdaSymbol{)(}\AgdaInductiveConstructor{≡.refl}\AgdaSpace{}%
\AgdaOperator{\AgdaInductiveConstructor{,}}\AgdaSpace{}%
\AgdaSymbol{λ}\AgdaSpace{}%
\AgdaBound{i}\AgdaSpace{}%
\AgdaSymbol{→}\AgdaSpace{}%
\AgdaField{lower}\AgdaSpace{}%
\AgdaSymbol{(}\AgdaBound{la≡lb}\AgdaSpace{}%
\AgdaBound{i}\AgdaSymbol{)))}\<%
\\
%
\\[\AgdaEmptyExtraSkip]%
\>[0]\AgdaFunction{Lift-Alg}\AgdaSpace{}%
\AgdaSymbol{:}\AgdaSpace{}%
\AgdaSymbol{(}\AgdaBound{𝑨}\AgdaSpace{}%
\AgdaSymbol{:}\AgdaSpace{}%
\AgdaRecord{Algebra}\AgdaSpace{}%
\AgdaGeneralizable{α}\AgdaSpace{}%
\AgdaGeneralizable{ρᵃ}\AgdaSymbol{)(}\AgdaBound{ℓ₀}\AgdaSpace{}%
\AgdaBound{ℓ₁}\AgdaSpace{}%
\AgdaSymbol{:}\AgdaSpace{}%
\AgdaPostulate{Level}\AgdaSymbol{)}\AgdaSpace{}%
\AgdaSymbol{→}\AgdaSpace{}%
\AgdaRecord{Algebra}\AgdaSpace{}%
\AgdaSymbol{(}\AgdaGeneralizable{α}\AgdaSpace{}%
\AgdaOperator{\AgdaPrimitive{⊔}}\AgdaSpace{}%
\AgdaBound{ℓ₀}\AgdaSymbol{)}\AgdaSpace{}%
\AgdaSymbol{(}\AgdaGeneralizable{ρᵃ}\AgdaSpace{}%
\AgdaOperator{\AgdaPrimitive{⊔}}\AgdaSpace{}%
\AgdaBound{ℓ₁}\AgdaSymbol{)}\<%
\\
\>[0]\AgdaFunction{Lift-Alg}\AgdaSpace{}%
\AgdaBound{𝑨}\AgdaSpace{}%
\AgdaBound{ℓ₀}\AgdaSpace{}%
\AgdaBound{ℓ₁}\AgdaSpace{}%
\AgdaSymbol{=}\AgdaSpace{}%
\AgdaFunction{Lift-Algʳ}\AgdaSpace{}%
\AgdaSymbol{(}\AgdaFunction{Lift-Algˡ}\AgdaSpace{}%
\AgdaBound{𝑨}\AgdaSpace{}%
\AgdaBound{ℓ₀}\AgdaSymbol{)}\AgdaSpace{}%
\AgdaBound{ℓ₁}\<%
\end{code}

\hypertarget{product-algebras}{%
\subsubsection{Product Algebras}\label{product-algebras}}

(cf.~the
\href{https://ualib.github.io/agda-algebras/Algebras.Func.Products.html}{Algebras.Func.Products}
module of the \href{https://ualib.github.io/agda-algebras}{Agda
Universal Algebra Library}.)

\begin{code}%
\>[0]\<%
\\
\>[0]\AgdaKeyword{module}\AgdaSpace{}%
\AgdaModule{\AgdaUnderscore{}}\AgdaSpace{}%
\AgdaSymbol{\{}\AgdaBound{ι}\AgdaSpace{}%
\AgdaSymbol{:}\AgdaSpace{}%
\AgdaPostulate{Level}\AgdaSymbol{\}\{}\AgdaBound{I}\AgdaSpace{}%
\AgdaSymbol{:}\AgdaSpace{}%
\AgdaPrimitive{Type}\AgdaSpace{}%
\AgdaBound{ι}\AgdaSpace{}%
\AgdaSymbol{\}}\AgdaSpace{}%
\AgdaKeyword{where}\<%
\\
%
\\[\AgdaEmptyExtraSkip]%
\>[0][@{}l@{\AgdaIndent{0}}]%
\>[1]\AgdaFunction{⨅}\AgdaSpace{}%
\AgdaSymbol{:}\AgdaSpace{}%
\AgdaSymbol{(}\AgdaBound{𝒜}\AgdaSpace{}%
\AgdaSymbol{:}\AgdaSpace{}%
\AgdaBound{I}\AgdaSpace{}%
\AgdaSymbol{→}\AgdaSpace{}%
\AgdaRecord{Algebra}\AgdaSpace{}%
\AgdaGeneralizable{α}\AgdaSpace{}%
\AgdaGeneralizable{ρᵃ}\AgdaSymbol{)}\AgdaSpace{}%
\AgdaSymbol{→}\AgdaSpace{}%
\AgdaRecord{Algebra}\AgdaSpace{}%
\AgdaSymbol{(}\AgdaGeneralizable{α}\AgdaSpace{}%
\AgdaOperator{\AgdaPrimitive{⊔}}\AgdaSpace{}%
\AgdaBound{ι}\AgdaSymbol{)}\AgdaSpace{}%
\AgdaSymbol{(}\AgdaGeneralizable{ρᵃ}\AgdaSpace{}%
\AgdaOperator{\AgdaPrimitive{⊔}}\AgdaSpace{}%
\AgdaBound{ι}\AgdaSymbol{)}\<%
\\
%
\>[1]\AgdaField{Domain}\AgdaSpace{}%
\AgdaSymbol{(}\AgdaFunction{⨅}\AgdaSpace{}%
\AgdaBound{𝒜}\AgdaSymbol{)}\AgdaSpace{}%
\AgdaSymbol{=}\<%
\\
\>[1][@{}l@{\AgdaIndent{0}}]%
\>[2]\AgdaKeyword{record}%
\>[10]\AgdaSymbol{\{}\AgdaSpace{}%
\AgdaField{Carrier}\AgdaSpace{}%
\AgdaSymbol{=}\AgdaSpace{}%
\AgdaSymbol{∀}\AgdaSpace{}%
\AgdaBound{i}\AgdaSpace{}%
\AgdaSymbol{→}\AgdaSpace{}%
\AgdaOperator{\AgdaFunction{𝕌[}}\AgdaSpace{}%
\AgdaBound{𝒜}\AgdaSpace{}%
\AgdaBound{i}\AgdaSpace{}%
\AgdaOperator{\AgdaFunction{]}}\<%
\\
%
\>[10]\AgdaSymbol{;}\AgdaSpace{}%
\AgdaOperator{\AgdaField{\AgdaUnderscore{}≈\AgdaUnderscore{}}}\AgdaSpace{}%
\AgdaSymbol{=}\AgdaSpace{}%
\AgdaSymbol{λ}\AgdaSpace{}%
\AgdaBound{a}\AgdaSpace{}%
\AgdaBound{b}\AgdaSpace{}%
\AgdaSymbol{→}\AgdaSpace{}%
\AgdaSymbol{∀}\AgdaSpace{}%
\AgdaBound{i}\AgdaSpace{}%
\AgdaSymbol{→}\AgdaSpace{}%
\AgdaSymbol{(}\AgdaOperator{\AgdaField{Setoid.\AgdaUnderscore{}≈\AgdaUnderscore{}}}\AgdaSpace{}%
\AgdaOperator{\AgdaFunction{𝔻[}}\AgdaSpace{}%
\AgdaBound{𝒜}\AgdaSpace{}%
\AgdaBound{i}\AgdaSpace{}%
\AgdaOperator{\AgdaFunction{]}}\AgdaSymbol{)}\AgdaSpace{}%
\AgdaSymbol{(}\AgdaBound{a}\AgdaSpace{}%
\AgdaBound{i}\AgdaSymbol{)(}\AgdaBound{b}\AgdaSpace{}%
\AgdaBound{i}\AgdaSymbol{)}\<%
\\
%
\>[10]\AgdaSymbol{;}%
\>[1165I]\AgdaField{isEquivalence}\AgdaSpace{}%
\AgdaSymbol{=}\<%
\\
\>[.][@{}l@{}]\<[1165I]%
\>[12]\AgdaKeyword{record}%
\>[20]\AgdaSymbol{\{}\AgdaSpace{}%
\AgdaField{refl}%
\>[29]\AgdaSymbol{=}\AgdaSpace{}%
\AgdaSymbol{λ}\AgdaSpace{}%
\AgdaBound{i}\AgdaSpace{}%
\AgdaSymbol{→}%
\>[42]\AgdaField{IsEquivalence.refl}%
\>[63]\AgdaSymbol{(}\AgdaField{isEquivalence}\AgdaSpace{}%
\AgdaOperator{\AgdaFunction{𝔻[}}\AgdaSpace{}%
\AgdaBound{𝒜}\AgdaSpace{}%
\AgdaBound{i}\AgdaSpace{}%
\AgdaOperator{\AgdaFunction{]}}\AgdaSymbol{)}\<%
\\
%
\>[20]\AgdaSymbol{;}\AgdaSpace{}%
\AgdaField{sym}%
\>[29]\AgdaSymbol{=}\AgdaSpace{}%
\AgdaSymbol{λ}\AgdaSpace{}%
\AgdaBound{x}\AgdaSpace{}%
\AgdaBound{i}\AgdaSpace{}%
\AgdaSymbol{→}%
\>[42]\AgdaField{IsEquivalence.sym}%
\>[63]\AgdaSymbol{(}\AgdaField{isEquivalence}\AgdaSpace{}%
\AgdaOperator{\AgdaFunction{𝔻[}}\AgdaSpace{}%
\AgdaBound{𝒜}\AgdaSpace{}%
\AgdaBound{i}\AgdaSpace{}%
\AgdaOperator{\AgdaFunction{]}}\AgdaSymbol{)(}\AgdaBound{x}\AgdaSpace{}%
\AgdaBound{i}\AgdaSymbol{)}\<%
\\
%
\>[20]\AgdaSymbol{;}\AgdaSpace{}%
\AgdaField{trans}%
\>[29]\AgdaSymbol{=}\AgdaSpace{}%
\AgdaSymbol{λ}\AgdaSpace{}%
\AgdaBound{x}\AgdaSpace{}%
\AgdaBound{y}\AgdaSpace{}%
\AgdaBound{i}\AgdaSpace{}%
\AgdaSymbol{→}%
\>[42]\AgdaField{IsEquivalence.trans}%
\>[63]\AgdaSymbol{(}\AgdaField{isEquivalence}\AgdaSpace{}%
\AgdaOperator{\AgdaFunction{𝔻[}}\AgdaSpace{}%
\AgdaBound{𝒜}\AgdaSpace{}%
\AgdaBound{i}\AgdaSpace{}%
\AgdaOperator{\AgdaFunction{]}}\AgdaSymbol{)(}\AgdaBound{x}\AgdaSpace{}%
\AgdaBound{i}\AgdaSymbol{)(}\AgdaBound{y}\AgdaSpace{}%
\AgdaBound{i}\AgdaSymbol{)}\AgdaSpace{}%
\AgdaSymbol{\}\}}\<%
\\
%
\>[1]\AgdaField{Interp}\AgdaSpace{}%
\AgdaSymbol{(}\AgdaFunction{⨅}\AgdaSpace{}%
\AgdaBound{𝒜}\AgdaSymbol{)}\AgdaSpace{}%
\AgdaOperator{\AgdaField{⟨\$⟩}}\AgdaSpace{}%
\AgdaSymbol{(}\AgdaBound{f}\AgdaSpace{}%
\AgdaOperator{\AgdaInductiveConstructor{,}}\AgdaSpace{}%
\AgdaBound{a}\AgdaSymbol{)}\AgdaSpace{}%
\AgdaSymbol{=}\AgdaSpace{}%
\AgdaSymbol{λ}\AgdaSpace{}%
\AgdaBound{i}\AgdaSpace{}%
\AgdaSymbol{→}\AgdaSpace{}%
\AgdaSymbol{(}\AgdaBound{f}\AgdaSpace{}%
\AgdaOperator{\AgdaFunction{̂}}\AgdaSpace{}%
\AgdaSymbol{(}\AgdaBound{𝒜}\AgdaSpace{}%
\AgdaBound{i}\AgdaSymbol{))}\AgdaSpace{}%
\AgdaSymbol{(}\AgdaFunction{flip}\AgdaSpace{}%
\AgdaBound{a}\AgdaSpace{}%
\AgdaBound{i}\AgdaSymbol{)}\<%
\\
%
\>[1]\AgdaField{cong}\AgdaSpace{}%
\AgdaSymbol{(}\AgdaField{Interp}\AgdaSpace{}%
\AgdaSymbol{(}\AgdaFunction{⨅}\AgdaSpace{}%
\AgdaBound{𝒜}\AgdaSymbol{))}\AgdaSpace{}%
\AgdaSymbol{(}\AgdaInductiveConstructor{≡.refl}\AgdaSpace{}%
\AgdaOperator{\AgdaInductiveConstructor{,}}\AgdaSpace{}%
\AgdaBound{f=g}\AgdaSpace{}%
\AgdaSymbol{)}\AgdaSpace{}%
\AgdaSymbol{=}\AgdaSpace{}%
\AgdaSymbol{λ}\AgdaSpace{}%
\AgdaBound{i}\AgdaSpace{}%
\AgdaSymbol{→}\AgdaSpace{}%
\AgdaField{cong}\AgdaSpace{}%
\AgdaSymbol{(}\AgdaField{Interp}\AgdaSpace{}%
\AgdaSymbol{(}\AgdaBound{𝒜}\AgdaSpace{}%
\AgdaBound{i}\AgdaSymbol{))}\AgdaSpace{}%
\AgdaSymbol{(}\AgdaInductiveConstructor{≡.refl}\AgdaSpace{}%
\AgdaOperator{\AgdaInductiveConstructor{,}}\AgdaSpace{}%
\AgdaFunction{flip}\AgdaSpace{}%
\AgdaBound{f=g}\AgdaSpace{}%
\AgdaBound{i}\AgdaSpace{}%
\AgdaSymbol{)}\<%
\end{code}

\hypertarget{homomorphisms}{%
\subsection{Homomorphisms}\label{homomorphisms}}

\hypertarget{basic-definitions}{%
\subsubsection{Basic definitions}\label{basic-definitions}}

Here are some useful definitions and theorems extracted from the
\href{https://ualib.github.io/agda-algebras/Homomorphisms.Func.Basic.html}{Homomorphisms.Func.Basic}
module of the \href{https://ualib.github.io/agda-algebras}{Agda
Universal Algebra Library}.

\begin{code}%
\>[0]\<%
\\
\>[0]\AgdaKeyword{module}\AgdaSpace{}%
\AgdaModule{\AgdaUnderscore{}}\AgdaSpace{}%
\AgdaSymbol{(}\AgdaBound{𝑨}\AgdaSpace{}%
\AgdaSymbol{:}\AgdaSpace{}%
\AgdaRecord{Algebra}\AgdaSpace{}%
\AgdaGeneralizable{α}\AgdaSpace{}%
\AgdaGeneralizable{ρᵃ}\AgdaSymbol{)(}\AgdaBound{𝑩}\AgdaSpace{}%
\AgdaSymbol{:}\AgdaSpace{}%
\AgdaRecord{Algebra}\AgdaSpace{}%
\AgdaGeneralizable{β}\AgdaSpace{}%
\AgdaGeneralizable{ρᵇ}\AgdaSymbol{)}\AgdaSpace{}%
\AgdaKeyword{where}\<%
\\
\>[0][@{}l@{\AgdaIndent{0}}]%
\>[1]\AgdaKeyword{private}\AgdaSpace{}%
\AgdaFunction{ov}\AgdaSpace{}%
\AgdaSymbol{=}\AgdaSpace{}%
\AgdaBound{𝓞}\AgdaSpace{}%
\AgdaOperator{\AgdaPrimitive{⊔}}\AgdaSpace{}%
\AgdaBound{𝓥}\AgdaSpace{}%
\AgdaSymbol{;}\AgdaSpace{}%
\AgdaFunction{a}\AgdaSpace{}%
\AgdaSymbol{=}\AgdaSpace{}%
\AgdaBound{α}\AgdaSpace{}%
\AgdaOperator{\AgdaPrimitive{⊔}}\AgdaSpace{}%
\AgdaBound{ρᵃ}\AgdaSpace{}%
\AgdaSymbol{;}\AgdaSpace{}%
\AgdaFunction{b}\AgdaSpace{}%
\AgdaSymbol{=}\AgdaSpace{}%
\AgdaBound{β}\AgdaSpace{}%
\AgdaOperator{\AgdaPrimitive{⊔}}\AgdaSpace{}%
\AgdaBound{ρᵇ}\AgdaSpace{}%
\AgdaSymbol{;}\AgdaSpace{}%
\AgdaFunction{c}\AgdaSpace{}%
\AgdaSymbol{=}\AgdaSpace{}%
\AgdaBound{𝓞}\AgdaSpace{}%
\AgdaOperator{\AgdaPrimitive{⊔}}\AgdaSpace{}%
\AgdaBound{𝓥}\AgdaSpace{}%
\AgdaOperator{\AgdaPrimitive{⊔}}\AgdaSpace{}%
\AgdaBound{α}\AgdaSpace{}%
\AgdaOperator{\AgdaPrimitive{⊔}}\AgdaSpace{}%
\AgdaBound{ρᵃ}\AgdaSpace{}%
\AgdaOperator{\AgdaPrimitive{⊔}}\AgdaSpace{}%
\AgdaBound{β}\AgdaSpace{}%
\AgdaOperator{\AgdaPrimitive{⊔}}\AgdaSpace{}%
\AgdaBound{ρᵇ}\<%
\\
%
\\[\AgdaEmptyExtraSkip]%
%
\>[1]\AgdaFunction{compatible-map-op}\AgdaSpace{}%
\AgdaSymbol{:}\AgdaSpace{}%
\AgdaSymbol{(}\AgdaOperator{\AgdaFunction{𝔻[}}\AgdaSpace{}%
\AgdaBound{𝑨}\AgdaSpace{}%
\AgdaOperator{\AgdaFunction{]}}\AgdaSpace{}%
\AgdaOperator{\AgdaRecord{⟶}}\AgdaSpace{}%
\AgdaOperator{\AgdaFunction{𝔻[}}\AgdaSpace{}%
\AgdaBound{𝑩}\AgdaSpace{}%
\AgdaOperator{\AgdaFunction{]}}\AgdaSymbol{)}\AgdaSpace{}%
\AgdaSymbol{→}\AgdaSpace{}%
\AgdaOperator{\AgdaFunction{∣}}\AgdaSpace{}%
\AgdaBound{𝑆}\AgdaSpace{}%
\AgdaOperator{\AgdaFunction{∣}}\AgdaSpace{}%
\AgdaSymbol{→}\AgdaSpace{}%
\AgdaPrimitive{Type}\AgdaSpace{}%
\AgdaSymbol{(}\AgdaBound{𝓥}\AgdaSpace{}%
\AgdaOperator{\AgdaPrimitive{⊔}}\AgdaSpace{}%
\AgdaBound{α}\AgdaSpace{}%
\AgdaOperator{\AgdaPrimitive{⊔}}\AgdaSpace{}%
\AgdaBound{ρᵇ}\AgdaSymbol{)}\<%
\\
%
\>[1]\AgdaFunction{compatible-map-op}\AgdaSpace{}%
\AgdaBound{h}\AgdaSpace{}%
\AgdaBound{f}\AgdaSpace{}%
\AgdaSymbol{=}\AgdaSpace{}%
\AgdaSymbol{∀}\AgdaSpace{}%
\AgdaSymbol{\{}\AgdaBound{a}\AgdaSymbol{\}}\AgdaSpace{}%
\AgdaSymbol{→}\AgdaSpace{}%
\AgdaSymbol{(}\AgdaBound{h}\AgdaSpace{}%
\AgdaOperator{\AgdaField{⟨\$⟩}}\AgdaSpace{}%
\AgdaSymbol{((}\AgdaBound{f}\AgdaSpace{}%
\AgdaOperator{\AgdaFunction{̂}}\AgdaSpace{}%
\AgdaBound{𝑨}\AgdaSymbol{)}\AgdaSpace{}%
\AgdaBound{a}\AgdaSymbol{))}\AgdaSpace{}%
\AgdaOperator{\AgdaFunction{≈₂}}\AgdaSpace{}%
\AgdaSymbol{((}\AgdaBound{f}\AgdaSpace{}%
\AgdaOperator{\AgdaFunction{̂}}\AgdaSpace{}%
\AgdaBound{𝑩}\AgdaSymbol{)}\AgdaSpace{}%
\AgdaSymbol{(λ}\AgdaSpace{}%
\AgdaBound{x}\AgdaSpace{}%
\AgdaSymbol{→}\AgdaSpace{}%
\AgdaSymbol{(}\AgdaBound{h}\AgdaSpace{}%
\AgdaOperator{\AgdaField{⟨\$⟩}}\AgdaSpace{}%
\AgdaSymbol{(}\AgdaBound{a}\AgdaSpace{}%
\AgdaBound{x}\AgdaSymbol{))))}\<%
\\
\>[1][@{}l@{\AgdaIndent{0}}]%
\>[2]\AgdaKeyword{where}%
\>[9]\AgdaKeyword{open}\AgdaSpace{}%
\AgdaModule{Setoid}\AgdaSpace{}%
\AgdaOperator{\AgdaFunction{𝔻[}}\AgdaSpace{}%
\AgdaBound{𝑩}\AgdaSpace{}%
\AgdaOperator{\AgdaFunction{]}}\AgdaSpace{}%
\AgdaKeyword{using}\AgdaSpace{}%
\AgdaSymbol{()}\AgdaSpace{}%
\AgdaKeyword{renaming}\AgdaSpace{}%
\AgdaSymbol{(}\AgdaSpace{}%
\AgdaOperator{\AgdaField{\AgdaUnderscore{}≈\AgdaUnderscore{}}}\AgdaSpace{}%
\AgdaSymbol{to}\AgdaSpace{}%
\AgdaOperator{\AgdaField{\AgdaUnderscore{}≈₂\AgdaUnderscore{}}}\AgdaSpace{}%
\AgdaSymbol{)}\<%
\\
%
\\[\AgdaEmptyExtraSkip]%
%
\>[1]\AgdaFunction{compatible-map}\AgdaSpace{}%
\AgdaSymbol{:}\AgdaSpace{}%
\AgdaSymbol{(}\AgdaOperator{\AgdaFunction{𝔻[}}\AgdaSpace{}%
\AgdaBound{𝑨}\AgdaSpace{}%
\AgdaOperator{\AgdaFunction{]}}\AgdaSpace{}%
\AgdaOperator{\AgdaRecord{⟶}}\AgdaSpace{}%
\AgdaOperator{\AgdaFunction{𝔻[}}\AgdaSpace{}%
\AgdaBound{𝑩}\AgdaSpace{}%
\AgdaOperator{\AgdaFunction{]}}\AgdaSymbol{)}\AgdaSpace{}%
\AgdaSymbol{→}\AgdaSpace{}%
\AgdaPrimitive{Type}\AgdaSpace{}%
\AgdaSymbol{(}\AgdaFunction{ov}\AgdaSpace{}%
\AgdaOperator{\AgdaPrimitive{⊔}}\AgdaSpace{}%
\AgdaBound{α}\AgdaSpace{}%
\AgdaOperator{\AgdaPrimitive{⊔}}\AgdaSpace{}%
\AgdaBound{ρᵇ}\AgdaSymbol{)}\<%
\\
%
\>[1]\AgdaFunction{compatible-map}\AgdaSpace{}%
\AgdaBound{h}\AgdaSpace{}%
\AgdaSymbol{=}\AgdaSpace{}%
\AgdaSymbol{∀}\AgdaSpace{}%
\AgdaSymbol{\{}\AgdaBound{f}\AgdaSymbol{\}}\AgdaSpace{}%
\AgdaSymbol{→}\AgdaSpace{}%
\AgdaFunction{compatible-map-op}\AgdaSpace{}%
\AgdaBound{h}\AgdaSpace{}%
\AgdaBound{f}\<%
\\
%
\\[\AgdaEmptyExtraSkip]%
%
\>[1]\AgdaComment{--\ The\ property\ of\ being\ a\ homomorphism.}\<%
\\
%
\>[1]\AgdaKeyword{record}\AgdaSpace{}%
\AgdaRecord{IsHom}\AgdaSpace{}%
\AgdaSymbol{(}\AgdaBound{h}\AgdaSpace{}%
\AgdaSymbol{:}\AgdaSpace{}%
\AgdaOperator{\AgdaFunction{𝔻[}}\AgdaSpace{}%
\AgdaBound{𝑨}\AgdaSpace{}%
\AgdaOperator{\AgdaFunction{]}}\AgdaSpace{}%
\AgdaOperator{\AgdaRecord{⟶}}\AgdaSpace{}%
\AgdaOperator{\AgdaFunction{𝔻[}}\AgdaSpace{}%
\AgdaBound{𝑩}\AgdaSpace{}%
\AgdaOperator{\AgdaFunction{]}}\AgdaSymbol{)}\AgdaSpace{}%
\AgdaSymbol{:}\AgdaSpace{}%
\AgdaPrimitive{Type}\AgdaSpace{}%
\AgdaSymbol{(}\AgdaFunction{ov}\AgdaSpace{}%
\AgdaOperator{\AgdaPrimitive{⊔}}\AgdaSpace{}%
\AgdaBound{α}\AgdaSpace{}%
\AgdaOperator{\AgdaPrimitive{⊔}}\AgdaSpace{}%
\AgdaBound{ρᵇ}\AgdaSymbol{)}\AgdaSpace{}%
\AgdaKeyword{where}\<%
\\
\>[1][@{}l@{\AgdaIndent{0}}]%
\>[2]\AgdaKeyword{constructor}\AgdaSpace{}%
\AgdaInductiveConstructor{mkhom}\<%
\\
%
\>[2]\AgdaKeyword{field}\AgdaSpace{}%
\AgdaField{compatible}\AgdaSpace{}%
\AgdaSymbol{:}\AgdaSpace{}%
\AgdaFunction{compatible-map}\AgdaSpace{}%
\AgdaBound{h}\<%
\\
%
\\[\AgdaEmptyExtraSkip]%
%
\>[1]\AgdaComment{--\ The\ type\ of\ homomorphisms.}\<%
\\
%
\>[1]\AgdaFunction{hom}\AgdaSpace{}%
\AgdaSymbol{:}\AgdaSpace{}%
\AgdaPrimitive{Type}\AgdaSpace{}%
\AgdaFunction{c}\<%
\\
%
\>[1]\AgdaFunction{hom}\AgdaSpace{}%
\AgdaSymbol{=}\AgdaSpace{}%
\AgdaRecord{Σ}\AgdaSpace{}%
\AgdaSymbol{(}\AgdaOperator{\AgdaFunction{𝔻[}}\AgdaSpace{}%
\AgdaBound{𝑨}\AgdaSpace{}%
\AgdaOperator{\AgdaFunction{]}}\AgdaSpace{}%
\AgdaOperator{\AgdaRecord{⟶}}\AgdaSpace{}%
\AgdaOperator{\AgdaFunction{𝔻[}}\AgdaSpace{}%
\AgdaBound{𝑩}\AgdaSpace{}%
\AgdaOperator{\AgdaFunction{]}}\AgdaSymbol{)}\AgdaSpace{}%
\AgdaRecord{IsHom}\<%
\end{code}

\hypertarget{monomorphisms-and-epimorphisms}{%
\subsubsection{Monomorphisms and
epimorphisms}\label{monomorphisms-and-epimorphisms}}

\begin{code}%
\>[0]\<%
\\
%
\>[1]\AgdaKeyword{record}\AgdaSpace{}%
\AgdaRecord{IsMon}\AgdaSpace{}%
\AgdaSymbol{(}\AgdaBound{h}\AgdaSpace{}%
\AgdaSymbol{:}\AgdaSpace{}%
\AgdaOperator{\AgdaFunction{𝔻[}}\AgdaSpace{}%
\AgdaBound{𝑨}\AgdaSpace{}%
\AgdaOperator{\AgdaFunction{]}}\AgdaSpace{}%
\AgdaOperator{\AgdaRecord{⟶}}\AgdaSpace{}%
\AgdaOperator{\AgdaFunction{𝔻[}}\AgdaSpace{}%
\AgdaBound{𝑩}\AgdaSpace{}%
\AgdaOperator{\AgdaFunction{]}}\AgdaSymbol{)}\AgdaSpace{}%
\AgdaSymbol{:}\AgdaSpace{}%
\AgdaPrimitive{Type}\AgdaSpace{}%
\AgdaSymbol{(}\AgdaFunction{ov}\AgdaSpace{}%
\AgdaOperator{\AgdaPrimitive{⊔}}\AgdaSpace{}%
\AgdaFunction{a}\AgdaSpace{}%
\AgdaOperator{\AgdaPrimitive{⊔}}\AgdaSpace{}%
\AgdaBound{ρᵇ}\AgdaSymbol{)}\AgdaSpace{}%
\AgdaKeyword{where}\<%
\\
\>[1][@{}l@{\AgdaIndent{0}}]%
\>[2]\AgdaKeyword{field}%
\>[9]\AgdaField{isHom}\AgdaSpace{}%
\AgdaSymbol{:}\AgdaSpace{}%
\AgdaRecord{IsHom}\AgdaSpace{}%
\AgdaBound{h}\<%
\\
%
\>[9]\AgdaField{isInjective}\AgdaSpace{}%
\AgdaSymbol{:}\AgdaSpace{}%
\AgdaFunction{IsInjective}\AgdaSpace{}%
\AgdaBound{h}\<%
\\
%
\\[\AgdaEmptyExtraSkip]%
%
\>[2]\AgdaFunction{HomReduct}\AgdaSpace{}%
\AgdaSymbol{:}\AgdaSpace{}%
\AgdaFunction{hom}\<%
\\
%
\>[2]\AgdaFunction{HomReduct}\AgdaSpace{}%
\AgdaSymbol{=}\AgdaSpace{}%
\AgdaBound{h}\AgdaSpace{}%
\AgdaOperator{\AgdaInductiveConstructor{,}}\AgdaSpace{}%
\AgdaField{isHom}\<%
\\
%
\\[\AgdaEmptyExtraSkip]%
%
\>[1]\AgdaFunction{mon}\AgdaSpace{}%
\AgdaSymbol{:}\AgdaSpace{}%
\AgdaPrimitive{Type}\AgdaSpace{}%
\AgdaFunction{c}\<%
\\
%
\>[1]\AgdaFunction{mon}\AgdaSpace{}%
\AgdaSymbol{=}\AgdaSpace{}%
\AgdaRecord{Σ}\AgdaSpace{}%
\AgdaSymbol{(}\AgdaOperator{\AgdaFunction{𝔻[}}\AgdaSpace{}%
\AgdaBound{𝑨}\AgdaSpace{}%
\AgdaOperator{\AgdaFunction{]}}\AgdaSpace{}%
\AgdaOperator{\AgdaRecord{⟶}}\AgdaSpace{}%
\AgdaOperator{\AgdaFunction{𝔻[}}\AgdaSpace{}%
\AgdaBound{𝑩}\AgdaSpace{}%
\AgdaOperator{\AgdaFunction{]}}\AgdaSymbol{)}\AgdaSpace{}%
\AgdaRecord{IsMon}\<%
\\
%
\\[\AgdaEmptyExtraSkip]%
%
\>[1]\AgdaKeyword{record}\AgdaSpace{}%
\AgdaRecord{IsEpi}\AgdaSpace{}%
\AgdaSymbol{(}\AgdaBound{h}\AgdaSpace{}%
\AgdaSymbol{:}\AgdaSpace{}%
\AgdaOperator{\AgdaFunction{𝔻[}}\AgdaSpace{}%
\AgdaBound{𝑨}\AgdaSpace{}%
\AgdaOperator{\AgdaFunction{]}}\AgdaSpace{}%
\AgdaOperator{\AgdaRecord{⟶}}\AgdaSpace{}%
\AgdaOperator{\AgdaFunction{𝔻[}}\AgdaSpace{}%
\AgdaBound{𝑩}\AgdaSpace{}%
\AgdaOperator{\AgdaFunction{]}}\AgdaSymbol{)}\AgdaSpace{}%
\AgdaSymbol{:}\AgdaSpace{}%
\AgdaPrimitive{Type}\AgdaSpace{}%
\AgdaSymbol{(}\AgdaFunction{ov}\AgdaSpace{}%
\AgdaOperator{\AgdaPrimitive{⊔}}\AgdaSpace{}%
\AgdaBound{α}\AgdaSpace{}%
\AgdaOperator{\AgdaPrimitive{⊔}}\AgdaSpace{}%
\AgdaFunction{b}\AgdaSymbol{)}\AgdaSpace{}%
\AgdaKeyword{where}\<%
\\
\>[1][@{}l@{\AgdaIndent{0}}]%
\>[2]\AgdaKeyword{field}%
\>[9]\AgdaField{isHom}\AgdaSpace{}%
\AgdaSymbol{:}\AgdaSpace{}%
\AgdaRecord{IsHom}\AgdaSpace{}%
\AgdaBound{h}\<%
\\
%
\>[9]\AgdaField{isSurjective}\AgdaSpace{}%
\AgdaSymbol{:}\AgdaSpace{}%
\AgdaFunction{IsSurjective}\AgdaSpace{}%
\AgdaBound{h}\<%
\\
%
\\[\AgdaEmptyExtraSkip]%
%
\>[2]\AgdaFunction{HomReduct}\AgdaSpace{}%
\AgdaSymbol{:}\AgdaSpace{}%
\AgdaFunction{hom}\<%
\\
%
\>[2]\AgdaFunction{HomReduct}\AgdaSpace{}%
\AgdaSymbol{=}\AgdaSpace{}%
\AgdaBound{h}\AgdaSpace{}%
\AgdaOperator{\AgdaInductiveConstructor{,}}\AgdaSpace{}%
\AgdaField{isHom}\<%
\\
%
\\[\AgdaEmptyExtraSkip]%
%
\>[1]\AgdaFunction{epi}\AgdaSpace{}%
\AgdaSymbol{:}\AgdaSpace{}%
\AgdaPrimitive{Type}\AgdaSpace{}%
\AgdaFunction{c}\<%
\\
%
\>[1]\AgdaFunction{epi}\AgdaSpace{}%
\AgdaSymbol{=}\AgdaSpace{}%
\AgdaRecord{Σ}\AgdaSpace{}%
\AgdaSymbol{(}\AgdaOperator{\AgdaFunction{𝔻[}}\AgdaSpace{}%
\AgdaBound{𝑨}\AgdaSpace{}%
\AgdaOperator{\AgdaFunction{]}}\AgdaSpace{}%
\AgdaOperator{\AgdaRecord{⟶}}\AgdaSpace{}%
\AgdaOperator{\AgdaFunction{𝔻[}}\AgdaSpace{}%
\AgdaBound{𝑩}\AgdaSpace{}%
\AgdaOperator{\AgdaFunction{]}}\AgdaSymbol{)}\AgdaSpace{}%
\AgdaRecord{IsEpi}\<%
\\
%
\\[\AgdaEmptyExtraSkip]%
\>[0]\AgdaKeyword{open}\AgdaSpace{}%
\AgdaModule{IsHom}\AgdaSpace{}%
\AgdaSymbol{;}\AgdaSpace{}%
\AgdaKeyword{open}\AgdaSpace{}%
\AgdaModule{IsMon}\AgdaSpace{}%
\AgdaSymbol{;}\AgdaSpace{}%
\AgdaKeyword{open}\AgdaSpace{}%
\AgdaModule{IsEpi}\<%
\\
%
\\[\AgdaEmptyExtraSkip]%
\>[0]\AgdaKeyword{module}\AgdaSpace{}%
\AgdaModule{\AgdaUnderscore{}}\AgdaSpace{}%
\AgdaSymbol{(}\AgdaBound{𝑨}\AgdaSpace{}%
\AgdaSymbol{:}\AgdaSpace{}%
\AgdaRecord{Algebra}\AgdaSpace{}%
\AgdaGeneralizable{α}\AgdaSpace{}%
\AgdaGeneralizable{ρᵃ}\AgdaSymbol{)(}\AgdaBound{𝑩}\AgdaSpace{}%
\AgdaSymbol{:}\AgdaSpace{}%
\AgdaRecord{Algebra}\AgdaSpace{}%
\AgdaGeneralizable{β}\AgdaSpace{}%
\AgdaGeneralizable{ρᵇ}\AgdaSymbol{)}\AgdaSpace{}%
\AgdaKeyword{where}\<%
\\
%
\\[\AgdaEmptyExtraSkip]%
\>[0][@{}l@{\AgdaIndent{0}}]%
\>[1]\AgdaFunction{mon→intohom}\AgdaSpace{}%
\AgdaSymbol{:}\AgdaSpace{}%
\AgdaFunction{mon}\AgdaSpace{}%
\AgdaBound{𝑨}\AgdaSpace{}%
\AgdaBound{𝑩}\AgdaSpace{}%
\AgdaSymbol{→}\AgdaSpace{}%
\AgdaFunction{Σ[}\AgdaSpace{}%
\AgdaBound{h}\AgdaSpace{}%
\AgdaFunction{∈}\AgdaSpace{}%
\AgdaFunction{hom}\AgdaSpace{}%
\AgdaBound{𝑨}\AgdaSpace{}%
\AgdaBound{𝑩}\AgdaSpace{}%
\AgdaFunction{]}\AgdaSpace{}%
\AgdaFunction{IsInjective}\AgdaSpace{}%
\AgdaOperator{\AgdaFunction{∣}}\AgdaSpace{}%
\AgdaBound{h}\AgdaSpace{}%
\AgdaOperator{\AgdaFunction{∣}}\<%
\\
%
\>[1]\AgdaFunction{mon→intohom}\AgdaSpace{}%
\AgdaSymbol{(}\AgdaBound{hh}\AgdaSpace{}%
\AgdaOperator{\AgdaInductiveConstructor{,}}\AgdaSpace{}%
\AgdaBound{hhM}\AgdaSymbol{)}\AgdaSpace{}%
\AgdaSymbol{=}\AgdaSpace{}%
\AgdaSymbol{(}\AgdaBound{hh}\AgdaSpace{}%
\AgdaOperator{\AgdaInductiveConstructor{,}}\AgdaSpace{}%
\AgdaField{isHom}\AgdaSpace{}%
\AgdaBound{hhM}\AgdaSymbol{)}\AgdaSpace{}%
\AgdaOperator{\AgdaInductiveConstructor{,}}\AgdaSpace{}%
\AgdaField{isInjective}\AgdaSpace{}%
\AgdaBound{hhM}\<%
\\
%
\\[\AgdaEmptyExtraSkip]%
%
\>[1]\AgdaFunction{epi→ontohom}\AgdaSpace{}%
\AgdaSymbol{:}\AgdaSpace{}%
\AgdaFunction{epi}\AgdaSpace{}%
\AgdaBound{𝑨}\AgdaSpace{}%
\AgdaBound{𝑩}\AgdaSpace{}%
\AgdaSymbol{→}\AgdaSpace{}%
\AgdaFunction{Σ[}\AgdaSpace{}%
\AgdaBound{h}\AgdaSpace{}%
\AgdaFunction{∈}\AgdaSpace{}%
\AgdaFunction{hom}\AgdaSpace{}%
\AgdaBound{𝑨}\AgdaSpace{}%
\AgdaBound{𝑩}\AgdaSpace{}%
\AgdaFunction{]}\AgdaSpace{}%
\AgdaFunction{IsSurjective}\AgdaSpace{}%
\AgdaOperator{\AgdaFunction{∣}}\AgdaSpace{}%
\AgdaBound{h}\AgdaSpace{}%
\AgdaOperator{\AgdaFunction{∣}}\<%
\\
%
\>[1]\AgdaFunction{epi→ontohom}\AgdaSpace{}%
\AgdaSymbol{(}\AgdaBound{hh}\AgdaSpace{}%
\AgdaOperator{\AgdaInductiveConstructor{,}}\AgdaSpace{}%
\AgdaBound{hhE}\AgdaSymbol{)}\AgdaSpace{}%
\AgdaSymbol{=}\AgdaSpace{}%
\AgdaSymbol{(}\AgdaBound{hh}\AgdaSpace{}%
\AgdaOperator{\AgdaInductiveConstructor{,}}\AgdaSpace{}%
\AgdaField{isHom}\AgdaSpace{}%
\AgdaBound{hhE}\AgdaSymbol{)}\AgdaSpace{}%
\AgdaOperator{\AgdaInductiveConstructor{,}}\AgdaSpace{}%
\AgdaField{isSurjective}\AgdaSpace{}%
\AgdaBound{hhE}\<%
\end{code}

\hypertarget{basic-properties-of-homomorphisms}{%
\subsubsection{Basic properties of
homomorphisms}\label{basic-properties-of-homomorphisms}}

Here are some definitions and theorems extracted from the
\href{https://ualib.github.io/agda-algebras/Homomorphisms.Func.Properties.html}{Homomorphisms.Func.Properties}
module of the \href{https://ualib.github.io/agda-algebras}{Agda
Universal Algebra Library}.

\hypertarget{composition-of-homomorphisms}{%
\paragraph{Composition of
homomorphisms}\label{composition-of-homomorphisms}}

The composition of homomorphisms is again a homomorphism. Similarly, the
composition of epimorphisms is again an epimorphism.

\begin{code}%
\>[0]\<%
\\
\>[0]\AgdaKeyword{module}\AgdaSpace{}%
\AgdaModule{\AgdaUnderscore{}}%
\>[10]\AgdaSymbol{\{}\AgdaBound{𝑨}\AgdaSpace{}%
\AgdaSymbol{:}\AgdaSpace{}%
\AgdaRecord{Algebra}\AgdaSpace{}%
\AgdaGeneralizable{α}\AgdaSpace{}%
\AgdaGeneralizable{ρᵃ}\AgdaSymbol{\}}\AgdaSpace{}%
\AgdaSymbol{\{}\AgdaBound{𝑩}\AgdaSpace{}%
\AgdaSymbol{:}\AgdaSpace{}%
\AgdaRecord{Algebra}\AgdaSpace{}%
\AgdaGeneralizable{β}\AgdaSpace{}%
\AgdaGeneralizable{ρᵇ}\AgdaSymbol{\}}\AgdaSpace{}%
\AgdaSymbol{\{}\AgdaBound{𝑪}\AgdaSpace{}%
\AgdaSymbol{:}\AgdaSpace{}%
\AgdaRecord{Algebra}\AgdaSpace{}%
\AgdaGeneralizable{γ}\AgdaSpace{}%
\AgdaGeneralizable{ρᶜ}\AgdaSymbol{\}}\<%
\\
%
\>[10]\AgdaSymbol{\{}\AgdaBound{g}\AgdaSpace{}%
\AgdaSymbol{:}\AgdaSpace{}%
\AgdaOperator{\AgdaFunction{𝔻[}}\AgdaSpace{}%
\AgdaBound{𝑨}\AgdaSpace{}%
\AgdaOperator{\AgdaFunction{]}}\AgdaSpace{}%
\AgdaOperator{\AgdaRecord{⟶}}\AgdaSpace{}%
\AgdaOperator{\AgdaFunction{𝔻[}}\AgdaSpace{}%
\AgdaBound{𝑩}\AgdaSpace{}%
\AgdaOperator{\AgdaFunction{]}}\AgdaSymbol{\}\{}\AgdaBound{h}\AgdaSpace{}%
\AgdaSymbol{:}\AgdaSpace{}%
\AgdaOperator{\AgdaFunction{𝔻[}}\AgdaSpace{}%
\AgdaBound{𝑩}\AgdaSpace{}%
\AgdaOperator{\AgdaFunction{]}}\AgdaSpace{}%
\AgdaOperator{\AgdaRecord{⟶}}\AgdaSpace{}%
\AgdaOperator{\AgdaFunction{𝔻[}}\AgdaSpace{}%
\AgdaBound{𝑪}\AgdaSpace{}%
\AgdaOperator{\AgdaFunction{]}}\AgdaSymbol{\}}\AgdaSpace{}%
\AgdaKeyword{where}\<%
\\
%
\\[\AgdaEmptyExtraSkip]%
\>[0][@{}l@{\AgdaIndent{0}}]%
\>[2]\AgdaKeyword{open}\AgdaSpace{}%
\AgdaModule{Setoid}\AgdaSpace{}%
\AgdaOperator{\AgdaFunction{𝔻[}}\AgdaSpace{}%
\AgdaBound{𝑪}\AgdaSpace{}%
\AgdaOperator{\AgdaFunction{]}}\AgdaSpace{}%
\AgdaKeyword{using}\AgdaSpace{}%
\AgdaSymbol{(}\AgdaSpace{}%
\AgdaFunction{trans}\AgdaSpace{}%
\AgdaSymbol{)}\<%
\\
%
\\[\AgdaEmptyExtraSkip]%
%
\>[2]\AgdaFunction{∘-is-hom}\AgdaSpace{}%
\AgdaSymbol{:}\AgdaSpace{}%
\AgdaRecord{IsHom}\AgdaSpace{}%
\AgdaBound{𝑨}\AgdaSpace{}%
\AgdaBound{𝑩}\AgdaSpace{}%
\AgdaBound{g}\AgdaSpace{}%
\AgdaSymbol{→}\AgdaSpace{}%
\AgdaRecord{IsHom}\AgdaSpace{}%
\AgdaBound{𝑩}\AgdaSpace{}%
\AgdaBound{𝑪}\AgdaSpace{}%
\AgdaBound{h}\AgdaSpace{}%
\AgdaSymbol{→}\AgdaSpace{}%
\AgdaRecord{IsHom}\AgdaSpace{}%
\AgdaBound{𝑨}\AgdaSpace{}%
\AgdaBound{𝑪}\AgdaSpace{}%
\AgdaSymbol{(}\AgdaBound{h}\AgdaSpace{}%
\AgdaOperator{\AgdaFunction{⟨∘⟩}}\AgdaSpace{}%
\AgdaBound{g}\AgdaSymbol{)}\<%
\\
%
\>[2]\AgdaFunction{∘-is-hom}\AgdaSpace{}%
\AgdaBound{ghom}\AgdaSpace{}%
\AgdaBound{hhom}\AgdaSpace{}%
\AgdaSymbol{=}\AgdaSpace{}%
\AgdaInductiveConstructor{mkhom}\AgdaSpace{}%
\AgdaFunction{c}\<%
\\
\>[2][@{}l@{\AgdaIndent{0}}]%
\>[3]\AgdaKeyword{where}\<%
\\
%
\>[3]\AgdaFunction{c}\AgdaSpace{}%
\AgdaSymbol{:}\AgdaSpace{}%
\AgdaFunction{compatible-map}\AgdaSpace{}%
\AgdaBound{𝑨}\AgdaSpace{}%
\AgdaBound{𝑪}\AgdaSpace{}%
\AgdaSymbol{(}\AgdaBound{h}\AgdaSpace{}%
\AgdaOperator{\AgdaFunction{⟨∘⟩}}\AgdaSpace{}%
\AgdaBound{g}\AgdaSymbol{)}\<%
\\
%
\>[3]\AgdaFunction{c}\AgdaSpace{}%
\AgdaSymbol{=}\AgdaSpace{}%
\AgdaFunction{trans}\AgdaSpace{}%
\AgdaSymbol{(}\AgdaField{cong}\AgdaSpace{}%
\AgdaBound{h}\AgdaSpace{}%
\AgdaSymbol{(}\AgdaField{compatible}\AgdaSpace{}%
\AgdaBound{ghom}\AgdaSymbol{))}\AgdaSpace{}%
\AgdaSymbol{(}\AgdaField{compatible}\AgdaSpace{}%
\AgdaBound{hhom}\AgdaSymbol{)}\<%
\\
%
\\[\AgdaEmptyExtraSkip]%
%
\>[2]\AgdaFunction{∘-is-epi}\AgdaSpace{}%
\AgdaSymbol{:}\AgdaSpace{}%
\AgdaRecord{IsEpi}\AgdaSpace{}%
\AgdaBound{𝑨}\AgdaSpace{}%
\AgdaBound{𝑩}\AgdaSpace{}%
\AgdaBound{g}\AgdaSpace{}%
\AgdaSymbol{→}\AgdaSpace{}%
\AgdaRecord{IsEpi}\AgdaSpace{}%
\AgdaBound{𝑩}\AgdaSpace{}%
\AgdaBound{𝑪}\AgdaSpace{}%
\AgdaBound{h}\AgdaSpace{}%
\AgdaSymbol{→}\AgdaSpace{}%
\AgdaRecord{IsEpi}\AgdaSpace{}%
\AgdaBound{𝑨}\AgdaSpace{}%
\AgdaBound{𝑪}\AgdaSpace{}%
\AgdaSymbol{(}\AgdaBound{h}\AgdaSpace{}%
\AgdaOperator{\AgdaFunction{⟨∘⟩}}\AgdaSpace{}%
\AgdaBound{g}\AgdaSymbol{)}\<%
\\
%
\>[2]\AgdaFunction{∘-is-epi}\AgdaSpace{}%
\AgdaBound{gE}\AgdaSpace{}%
\AgdaBound{hE}\AgdaSpace{}%
\AgdaSymbol{=}\AgdaSpace{}%
\AgdaKeyword{record}%
\>[27]\AgdaSymbol{\{}\AgdaSpace{}%
\AgdaField{isHom}\AgdaSpace{}%
\AgdaSymbol{=}\AgdaSpace{}%
\AgdaFunction{∘-is-hom}\AgdaSpace{}%
\AgdaSymbol{(}\AgdaField{isHom}\AgdaSpace{}%
\AgdaBound{gE}\AgdaSymbol{)}\AgdaSpace{}%
\AgdaSymbol{(}\AgdaField{isHom}\AgdaSpace{}%
\AgdaBound{hE}\AgdaSymbol{)}\<%
\\
%
\>[27]\AgdaSymbol{;}\AgdaSpace{}%
\AgdaField{isSurjective}\AgdaSpace{}%
\AgdaSymbol{=}\AgdaSpace{}%
\AgdaFunction{∘-IsSurjective}\AgdaSpace{}%
\AgdaBound{g}\AgdaSpace{}%
\AgdaBound{h}\AgdaSpace{}%
\AgdaSymbol{(}\AgdaField{isSurjective}\AgdaSpace{}%
\AgdaBound{gE}\AgdaSymbol{)}\AgdaSpace{}%
\AgdaSymbol{(}\AgdaField{isSurjective}\AgdaSpace{}%
\AgdaBound{hE}\AgdaSymbol{)}\AgdaSpace{}%
\AgdaSymbol{\}}\<%
\\
%
\\[\AgdaEmptyExtraSkip]%
%
\\[\AgdaEmptyExtraSkip]%
\>[0]\AgdaKeyword{module}\AgdaSpace{}%
\AgdaModule{\AgdaUnderscore{}}\AgdaSpace{}%
\AgdaSymbol{\{}\AgdaBound{𝑨}\AgdaSpace{}%
\AgdaSymbol{:}\AgdaSpace{}%
\AgdaRecord{Algebra}\AgdaSpace{}%
\AgdaGeneralizable{α}\AgdaSpace{}%
\AgdaGeneralizable{ρᵃ}\AgdaSymbol{\}}\AgdaSpace{}%
\AgdaSymbol{\{}\AgdaBound{𝑩}\AgdaSpace{}%
\AgdaSymbol{:}\AgdaSpace{}%
\AgdaRecord{Algebra}\AgdaSpace{}%
\AgdaGeneralizable{β}\AgdaSpace{}%
\AgdaGeneralizable{ρᵇ}\AgdaSymbol{\}}\AgdaSpace{}%
\AgdaSymbol{\{}\AgdaBound{𝑪}\AgdaSpace{}%
\AgdaSymbol{:}\AgdaSpace{}%
\AgdaRecord{Algebra}\AgdaSpace{}%
\AgdaGeneralizable{γ}\AgdaSpace{}%
\AgdaGeneralizable{ρᶜ}\AgdaSymbol{\}}\AgdaSpace{}%
\AgdaKeyword{where}\<%
\\
%
\\[\AgdaEmptyExtraSkip]%
\>[0][@{}l@{\AgdaIndent{0}}]%
\>[2]\AgdaFunction{∘-hom}\AgdaSpace{}%
\AgdaSymbol{:}\AgdaSpace{}%
\AgdaFunction{hom}\AgdaSpace{}%
\AgdaBound{𝑨}\AgdaSpace{}%
\AgdaBound{𝑩}\AgdaSpace{}%
\AgdaSymbol{→}\AgdaSpace{}%
\AgdaFunction{hom}\AgdaSpace{}%
\AgdaBound{𝑩}\AgdaSpace{}%
\AgdaBound{𝑪}%
\>[29]\AgdaSymbol{→}\AgdaSpace{}%
\AgdaFunction{hom}\AgdaSpace{}%
\AgdaBound{𝑨}\AgdaSpace{}%
\AgdaBound{𝑪}\<%
\\
%
\>[2]\AgdaFunction{∘-hom}\AgdaSpace{}%
\AgdaSymbol{(}\AgdaBound{h}\AgdaSpace{}%
\AgdaOperator{\AgdaInductiveConstructor{,}}\AgdaSpace{}%
\AgdaBound{hhom}\AgdaSymbol{)}\AgdaSpace{}%
\AgdaSymbol{(}\AgdaBound{g}\AgdaSpace{}%
\AgdaOperator{\AgdaInductiveConstructor{,}}\AgdaSpace{}%
\AgdaBound{ghom}\AgdaSymbol{)}\AgdaSpace{}%
\AgdaSymbol{=}\AgdaSpace{}%
\AgdaSymbol{(}\AgdaBound{g}\AgdaSpace{}%
\AgdaOperator{\AgdaFunction{⟨∘⟩}}\AgdaSpace{}%
\AgdaBound{h}\AgdaSymbol{)}\AgdaSpace{}%
\AgdaOperator{\AgdaInductiveConstructor{,}}\AgdaSpace{}%
\AgdaFunction{∘-is-hom}\AgdaSpace{}%
\AgdaBound{hhom}\AgdaSpace{}%
\AgdaBound{ghom}\<%
\\
%
\\[\AgdaEmptyExtraSkip]%
%
\>[2]\AgdaFunction{∘-epi}\AgdaSpace{}%
\AgdaSymbol{:}\AgdaSpace{}%
\AgdaFunction{epi}\AgdaSpace{}%
\AgdaBound{𝑨}\AgdaSpace{}%
\AgdaBound{𝑩}\AgdaSpace{}%
\AgdaSymbol{→}\AgdaSpace{}%
\AgdaFunction{epi}\AgdaSpace{}%
\AgdaBound{𝑩}\AgdaSpace{}%
\AgdaBound{𝑪}%
\>[29]\AgdaSymbol{→}\AgdaSpace{}%
\AgdaFunction{epi}\AgdaSpace{}%
\AgdaBound{𝑨}\AgdaSpace{}%
\AgdaBound{𝑪}\<%
\\
%
\>[2]\AgdaFunction{∘-epi}\AgdaSpace{}%
\AgdaSymbol{(}\AgdaBound{h}\AgdaSpace{}%
\AgdaOperator{\AgdaInductiveConstructor{,}}\AgdaSpace{}%
\AgdaBound{hepi}\AgdaSymbol{)}\AgdaSpace{}%
\AgdaSymbol{(}\AgdaBound{g}\AgdaSpace{}%
\AgdaOperator{\AgdaInductiveConstructor{,}}\AgdaSpace{}%
\AgdaBound{gepi}\AgdaSymbol{)}\AgdaSpace{}%
\AgdaSymbol{=}\AgdaSpace{}%
\AgdaSymbol{(}\AgdaBound{g}\AgdaSpace{}%
\AgdaOperator{\AgdaFunction{⟨∘⟩}}\AgdaSpace{}%
\AgdaBound{h}\AgdaSymbol{)}\AgdaSpace{}%
\AgdaOperator{\AgdaInductiveConstructor{,}}\AgdaSpace{}%
\AgdaFunction{∘-is-epi}\AgdaSpace{}%
\AgdaBound{hepi}\AgdaSpace{}%
\AgdaBound{gepi}\<%
\end{code}

\hypertarget{universe-lifting-of-homomorphisms}{%
\paragraph{Universe lifting of
homomorphisms}\label{universe-lifting-of-homomorphisms}}

First we define the identity homomorphism for setoid algebras and then
we prove that the operations of lifting and lowering of a setoid algebra
are homomorphisms.

\begin{code}[hide]%
\>[0]\<%
\\
\>[0]\AgdaFunction{𝒾𝒹}\AgdaSpace{}%
\AgdaSymbol{:}\AgdaSpace{}%
\AgdaSymbol{\{}\AgdaBound{𝑨}\AgdaSpace{}%
\AgdaSymbol{:}\AgdaSpace{}%
\AgdaRecord{Algebra}\AgdaSpace{}%
\AgdaGeneralizable{α}\AgdaSpace{}%
\AgdaGeneralizable{ρᵃ}\AgdaSymbol{\}}\AgdaSpace{}%
\AgdaSymbol{→}\AgdaSpace{}%
\AgdaFunction{hom}\AgdaSpace{}%
\AgdaBound{𝑨}\AgdaSpace{}%
\AgdaBound{𝑨}\<%
\\
\>[0]\AgdaFunction{𝒾𝒹}\AgdaSpace{}%
\AgdaSymbol{\{}\AgdaArgument{𝑨}\AgdaSpace{}%
\AgdaSymbol{=}\AgdaSpace{}%
\AgdaBound{𝑨}\AgdaSymbol{\}}\AgdaSpace{}%
\AgdaSymbol{=}\AgdaSpace{}%
\AgdaFunction{𝑖𝑑}\AgdaSpace{}%
\AgdaOperator{\AgdaInductiveConstructor{,}}\AgdaSpace{}%
\AgdaInductiveConstructor{mkhom}\AgdaSpace{}%
\AgdaSymbol{(}\AgdaFunction{reflexive}\AgdaSpace{}%
\AgdaInductiveConstructor{≡.refl}\AgdaSymbol{)}\AgdaSpace{}%
\AgdaKeyword{where}\AgdaSpace{}%
\AgdaKeyword{open}\AgdaSpace{}%
\AgdaModule{Setoid}\AgdaSpace{}%
\AgdaSymbol{(}\AgdaSpace{}%
\AgdaField{Domain}\AgdaSpace{}%
\AgdaBound{𝑨}\AgdaSpace{}%
\AgdaSymbol{)}\AgdaSpace{}%
\AgdaKeyword{using}\AgdaSpace{}%
\AgdaSymbol{(}\AgdaSpace{}%
\AgdaFunction{reflexive}\AgdaSpace{}%
\AgdaSymbol{)}\<%
\\
%
\\[\AgdaEmptyExtraSkip]%
\>[0]\AgdaKeyword{module}\AgdaSpace{}%
\AgdaModule{\AgdaUnderscore{}}\AgdaSpace{}%
\AgdaSymbol{\{}\AgdaBound{𝑨}\AgdaSpace{}%
\AgdaSymbol{:}\AgdaSpace{}%
\AgdaRecord{Algebra}\AgdaSpace{}%
\AgdaGeneralizable{α}\AgdaSpace{}%
\AgdaGeneralizable{ρᵃ}\AgdaSymbol{\}\{}\AgdaBound{ℓ}\AgdaSpace{}%
\AgdaSymbol{:}\AgdaSpace{}%
\AgdaPostulate{Level}\AgdaSymbol{\}}\AgdaSpace{}%
\AgdaKeyword{where}\<%
\\
\>[0][@{}l@{\AgdaIndent{0}}]%
\>[1]\AgdaKeyword{open}\AgdaSpace{}%
\AgdaModule{Setoid}\AgdaSpace{}%
\AgdaOperator{\AgdaFunction{𝔻[}}\AgdaSpace{}%
\AgdaBound{𝑨}\AgdaSpace{}%
\AgdaOperator{\AgdaFunction{]}}%
\>[33]\AgdaKeyword{using}\AgdaSpace{}%
\AgdaSymbol{(}\AgdaSpace{}%
\AgdaFunction{reflexive}\AgdaSpace{}%
\AgdaSymbol{)}%
\>[54]\AgdaKeyword{renaming}\AgdaSpace{}%
\AgdaSymbol{(}\AgdaSpace{}%
\AgdaOperator{\AgdaField{\AgdaUnderscore{}≈\AgdaUnderscore{}}}\AgdaSpace{}%
\AgdaSymbol{to}\AgdaSpace{}%
\AgdaOperator{\AgdaField{\AgdaUnderscore{}≈₁\AgdaUnderscore{}}}\AgdaSpace{}%
\AgdaSymbol{;}\AgdaSpace{}%
\AgdaFunction{refl}\AgdaSpace{}%
\AgdaSymbol{to}\AgdaSpace{}%
\AgdaFunction{refl₁}\AgdaSpace{}%
\AgdaSymbol{)}\<%
\\
%
\>[1]\AgdaKeyword{open}\AgdaSpace{}%
\AgdaModule{Setoid}\AgdaSpace{}%
\AgdaOperator{\AgdaFunction{𝔻[}}\AgdaSpace{}%
\AgdaFunction{Lift-Algˡ}\AgdaSpace{}%
\AgdaBound{𝑨}\AgdaSpace{}%
\AgdaBound{ℓ}\AgdaSpace{}%
\AgdaOperator{\AgdaFunction{]}}%
\>[33]\AgdaKeyword{using}\AgdaSpace{}%
\AgdaSymbol{()}%
\>[54]\AgdaKeyword{renaming}\AgdaSpace{}%
\AgdaSymbol{(}\AgdaSpace{}%
\AgdaOperator{\AgdaField{\AgdaUnderscore{}≈\AgdaUnderscore{}}}\AgdaSpace{}%
\AgdaSymbol{to}\AgdaSpace{}%
\AgdaOperator{\AgdaField{\AgdaUnderscore{}≈ˡ\AgdaUnderscore{}}}\AgdaSpace{}%
\AgdaSymbol{;}\AgdaSpace{}%
\AgdaFunction{refl}\AgdaSpace{}%
\AgdaSymbol{to}\AgdaSpace{}%
\AgdaFunction{reflˡ}\AgdaSymbol{)}\<%
\\
%
\>[1]\AgdaKeyword{open}\AgdaSpace{}%
\AgdaModule{Setoid}\AgdaSpace{}%
\AgdaOperator{\AgdaFunction{𝔻[}}\AgdaSpace{}%
\AgdaFunction{Lift-Algʳ}\AgdaSpace{}%
\AgdaBound{𝑨}\AgdaSpace{}%
\AgdaBound{ℓ}\AgdaSpace{}%
\AgdaOperator{\AgdaFunction{]}}%
\>[33]\AgdaKeyword{using}\AgdaSpace{}%
\AgdaSymbol{()}%
\>[54]\AgdaKeyword{renaming}\AgdaSpace{}%
\AgdaSymbol{(}\AgdaSpace{}%
\AgdaOperator{\AgdaField{\AgdaUnderscore{}≈\AgdaUnderscore{}}}\AgdaSpace{}%
\AgdaSymbol{to}\AgdaSpace{}%
\AgdaOperator{\AgdaField{\AgdaUnderscore{}≈ʳ\AgdaUnderscore{}}}\AgdaSpace{}%
\AgdaSymbol{;}\AgdaSpace{}%
\AgdaFunction{refl}\AgdaSpace{}%
\AgdaSymbol{to}\AgdaSpace{}%
\AgdaFunction{reflʳ}\AgdaSymbol{)}\<%
\\
%
\>[1]\AgdaKeyword{open}\AgdaSpace{}%
\AgdaModule{Level}\<%
\\
%
\\[\AgdaEmptyExtraSkip]%
%
\>[1]\AgdaFunction{ToLiftˡ}\AgdaSpace{}%
\AgdaSymbol{:}\AgdaSpace{}%
\AgdaFunction{hom}\AgdaSpace{}%
\AgdaBound{𝑨}\AgdaSpace{}%
\AgdaSymbol{(}\AgdaFunction{Lift-Algˡ}\AgdaSpace{}%
\AgdaBound{𝑨}\AgdaSpace{}%
\AgdaBound{ℓ}\AgdaSymbol{)}\<%
\\
%
\>[1]\AgdaFunction{ToLiftˡ}\AgdaSpace{}%
\AgdaSymbol{=}\AgdaSpace{}%
\AgdaKeyword{record}\AgdaSpace{}%
\AgdaSymbol{\{}\AgdaSpace{}%
\AgdaField{f}\AgdaSpace{}%
\AgdaSymbol{=}\AgdaSpace{}%
\AgdaInductiveConstructor{lift}\AgdaSpace{}%
\AgdaSymbol{;}\AgdaSpace{}%
\AgdaField{cong}\AgdaSpace{}%
\AgdaSymbol{=}\AgdaSpace{}%
\AgdaFunction{id}\AgdaSpace{}%
\AgdaSymbol{\}}\AgdaSpace{}%
\AgdaOperator{\AgdaInductiveConstructor{,}}\AgdaSpace{}%
\AgdaInductiveConstructor{mkhom}\AgdaSpace{}%
\AgdaSymbol{(}\AgdaFunction{reflexive}\AgdaSpace{}%
\AgdaInductiveConstructor{≡.refl}\AgdaSymbol{)}\<%
\\
%
\\[\AgdaEmptyExtraSkip]%
%
\>[1]\AgdaFunction{FromLiftˡ}\AgdaSpace{}%
\AgdaSymbol{:}\AgdaSpace{}%
\AgdaFunction{hom}\AgdaSpace{}%
\AgdaSymbol{(}\AgdaFunction{Lift-Algˡ}\AgdaSpace{}%
\AgdaBound{𝑨}\AgdaSpace{}%
\AgdaBound{ℓ}\AgdaSymbol{)}\AgdaSpace{}%
\AgdaBound{𝑨}\<%
\\
%
\>[1]\AgdaFunction{FromLiftˡ}\AgdaSpace{}%
\AgdaSymbol{=}\AgdaSpace{}%
\AgdaKeyword{record}\AgdaSpace{}%
\AgdaSymbol{\{}\AgdaSpace{}%
\AgdaField{f}\AgdaSpace{}%
\AgdaSymbol{=}\AgdaSpace{}%
\AgdaField{lower}\AgdaSpace{}%
\AgdaSymbol{;}\AgdaSpace{}%
\AgdaField{cong}\AgdaSpace{}%
\AgdaSymbol{=}\AgdaSpace{}%
\AgdaFunction{id}\AgdaSpace{}%
\AgdaSymbol{\}}\AgdaSpace{}%
\AgdaOperator{\AgdaInductiveConstructor{,}}\AgdaSpace{}%
\AgdaInductiveConstructor{mkhom}\AgdaSpace{}%
\AgdaFunction{reflˡ}\<%
\\
%
\\[\AgdaEmptyExtraSkip]%
%
\>[1]\AgdaFunction{ToFromLiftˡ}\AgdaSpace{}%
\AgdaSymbol{:}\AgdaSpace{}%
\AgdaSymbol{∀}\AgdaSpace{}%
\AgdaBound{b}\AgdaSpace{}%
\AgdaSymbol{→}%
\>[22]\AgdaSymbol{(}\AgdaOperator{\AgdaFunction{∣}}\AgdaSpace{}%
\AgdaFunction{ToLiftˡ}\AgdaSpace{}%
\AgdaOperator{\AgdaFunction{∣}}\AgdaSpace{}%
\AgdaOperator{\AgdaField{⟨\$⟩}}\AgdaSpace{}%
\AgdaSymbol{(}\AgdaOperator{\AgdaFunction{∣}}\AgdaSpace{}%
\AgdaFunction{FromLiftˡ}\AgdaSpace{}%
\AgdaOperator{\AgdaFunction{∣}}\AgdaSpace{}%
\AgdaOperator{\AgdaField{⟨\$⟩}}\AgdaSpace{}%
\AgdaBound{b}\AgdaSymbol{))}\AgdaSpace{}%
\AgdaOperator{\AgdaFunction{≈ˡ}}\AgdaSpace{}%
\AgdaBound{b}\<%
\\
%
\>[1]\AgdaFunction{ToFromLiftˡ}\AgdaSpace{}%
\AgdaBound{b}\AgdaSpace{}%
\AgdaSymbol{=}\AgdaSpace{}%
\AgdaFunction{refl₁}\<%
\\
%
\\[\AgdaEmptyExtraSkip]%
%
\>[1]\AgdaFunction{FromToLiftˡ}\AgdaSpace{}%
\AgdaSymbol{:}\AgdaSpace{}%
\AgdaSymbol{∀}\AgdaSpace{}%
\AgdaBound{a}\AgdaSpace{}%
\AgdaSymbol{→}\AgdaSpace{}%
\AgdaSymbol{(}\AgdaOperator{\AgdaFunction{∣}}\AgdaSpace{}%
\AgdaFunction{FromLiftˡ}\AgdaSpace{}%
\AgdaOperator{\AgdaFunction{∣}}\AgdaSpace{}%
\AgdaOperator{\AgdaField{⟨\$⟩}}\AgdaSpace{}%
\AgdaSymbol{(}\AgdaOperator{\AgdaFunction{∣}}\AgdaSpace{}%
\AgdaFunction{ToLiftˡ}\AgdaSpace{}%
\AgdaOperator{\AgdaFunction{∣}}\AgdaSpace{}%
\AgdaOperator{\AgdaField{⟨\$⟩}}\AgdaSpace{}%
\AgdaBound{a}\AgdaSymbol{))}\AgdaSpace{}%
\AgdaOperator{\AgdaFunction{≈₁}}\AgdaSpace{}%
\AgdaBound{a}\<%
\\
%
\>[1]\AgdaFunction{FromToLiftˡ}\AgdaSpace{}%
\AgdaBound{a}\AgdaSpace{}%
\AgdaSymbol{=}\AgdaSpace{}%
\AgdaFunction{refl₁}\<%
\\
%
\\[\AgdaEmptyExtraSkip]%
%
\>[1]\AgdaFunction{ToLiftʳ}\AgdaSpace{}%
\AgdaSymbol{:}\AgdaSpace{}%
\AgdaFunction{hom}\AgdaSpace{}%
\AgdaBound{𝑨}\AgdaSpace{}%
\AgdaSymbol{(}\AgdaFunction{Lift-Algʳ}\AgdaSpace{}%
\AgdaBound{𝑨}\AgdaSpace{}%
\AgdaBound{ℓ}\AgdaSymbol{)}\<%
\\
%
\>[1]\AgdaFunction{ToLiftʳ}\AgdaSpace{}%
\AgdaSymbol{=}\AgdaSpace{}%
\AgdaKeyword{record}\AgdaSpace{}%
\AgdaSymbol{\{}\AgdaSpace{}%
\AgdaField{f}\AgdaSpace{}%
\AgdaSymbol{=}\AgdaSpace{}%
\AgdaFunction{id}\AgdaSpace{}%
\AgdaSymbol{;}\AgdaSpace{}%
\AgdaField{cong}\AgdaSpace{}%
\AgdaSymbol{=}\AgdaSpace{}%
\AgdaInductiveConstructor{lift}\AgdaSpace{}%
\AgdaSymbol{\}}\AgdaSpace{}%
\AgdaOperator{\AgdaInductiveConstructor{,}}\AgdaSpace{}%
\AgdaInductiveConstructor{mkhom}\AgdaSpace{}%
\AgdaSymbol{(}\AgdaInductiveConstructor{lift}\AgdaSpace{}%
\AgdaSymbol{(}\AgdaFunction{reflexive}\AgdaSpace{}%
\AgdaInductiveConstructor{≡.refl}\AgdaSymbol{))}\<%
\\
%
\\[\AgdaEmptyExtraSkip]%
%
\>[1]\AgdaFunction{FromLiftʳ}\AgdaSpace{}%
\AgdaSymbol{:}\AgdaSpace{}%
\AgdaFunction{hom}\AgdaSpace{}%
\AgdaSymbol{(}\AgdaFunction{Lift-Algʳ}\AgdaSpace{}%
\AgdaBound{𝑨}\AgdaSpace{}%
\AgdaBound{ℓ}\AgdaSymbol{)}\AgdaSpace{}%
\AgdaBound{𝑨}\<%
\\
%
\>[1]\AgdaFunction{FromLiftʳ}\AgdaSpace{}%
\AgdaSymbol{=}\AgdaSpace{}%
\AgdaKeyword{record}\AgdaSpace{}%
\AgdaSymbol{\{}\AgdaSpace{}%
\AgdaField{f}\AgdaSpace{}%
\AgdaSymbol{=}\AgdaSpace{}%
\AgdaFunction{id}\AgdaSpace{}%
\AgdaSymbol{;}\AgdaSpace{}%
\AgdaField{cong}\AgdaSpace{}%
\AgdaSymbol{=}\AgdaSpace{}%
\AgdaField{lower}\AgdaSpace{}%
\AgdaSymbol{\}}\AgdaSpace{}%
\AgdaOperator{\AgdaInductiveConstructor{,}}\AgdaSpace{}%
\AgdaInductiveConstructor{mkhom}\AgdaSpace{}%
\AgdaFunction{reflˡ}\<%
\\
%
\\[\AgdaEmptyExtraSkip]%
%
\>[1]\AgdaFunction{ToFromLiftʳ}\AgdaSpace{}%
\AgdaSymbol{:}\AgdaSpace{}%
\AgdaSymbol{∀}\AgdaSpace{}%
\AgdaBound{b}\AgdaSpace{}%
\AgdaSymbol{→}\AgdaSpace{}%
\AgdaSymbol{(}\AgdaOperator{\AgdaFunction{∣}}\AgdaSpace{}%
\AgdaFunction{ToLiftʳ}\AgdaSpace{}%
\AgdaOperator{\AgdaFunction{∣}}\AgdaSpace{}%
\AgdaOperator{\AgdaField{⟨\$⟩}}\AgdaSpace{}%
\AgdaSymbol{(}\AgdaOperator{\AgdaFunction{∣}}\AgdaSpace{}%
\AgdaFunction{FromLiftʳ}\AgdaSpace{}%
\AgdaOperator{\AgdaFunction{∣}}\AgdaSpace{}%
\AgdaOperator{\AgdaField{⟨\$⟩}}\AgdaSpace{}%
\AgdaBound{b}\AgdaSymbol{))}\AgdaSpace{}%
\AgdaOperator{\AgdaFunction{≈ʳ}}\AgdaSpace{}%
\AgdaBound{b}\<%
\\
%
\>[1]\AgdaFunction{ToFromLiftʳ}\AgdaSpace{}%
\AgdaBound{b}\AgdaSpace{}%
\AgdaSymbol{=}\AgdaSpace{}%
\AgdaInductiveConstructor{lift}\AgdaSpace{}%
\AgdaFunction{refl₁}\<%
\\
%
\\[\AgdaEmptyExtraSkip]%
%
\>[1]\AgdaFunction{FromToLiftʳ}\AgdaSpace{}%
\AgdaSymbol{:}\AgdaSpace{}%
\AgdaSymbol{∀}\AgdaSpace{}%
\AgdaBound{a}\AgdaSpace{}%
\AgdaSymbol{→}\AgdaSpace{}%
\AgdaSymbol{(}\AgdaOperator{\AgdaFunction{∣}}\AgdaSpace{}%
\AgdaFunction{FromLiftʳ}\AgdaSpace{}%
\AgdaOperator{\AgdaFunction{∣}}\AgdaSpace{}%
\AgdaOperator{\AgdaField{⟨\$⟩}}\AgdaSpace{}%
\AgdaSymbol{(}\AgdaOperator{\AgdaFunction{∣}}\AgdaSpace{}%
\AgdaFunction{ToLiftʳ}\AgdaSpace{}%
\AgdaOperator{\AgdaFunction{∣}}\AgdaSpace{}%
\AgdaOperator{\AgdaField{⟨\$⟩}}\AgdaSpace{}%
\AgdaBound{a}\AgdaSymbol{))}\AgdaSpace{}%
\AgdaOperator{\AgdaFunction{≈₁}}\AgdaSpace{}%
\AgdaBound{a}\<%
\\
%
\>[1]\AgdaFunction{FromToLiftʳ}\AgdaSpace{}%
\AgdaBound{a}\AgdaSpace{}%
\AgdaSymbol{=}\AgdaSpace{}%
\AgdaFunction{refl₁}\<%
\\
%
\\[\AgdaEmptyExtraSkip]%
%
\\[\AgdaEmptyExtraSkip]%
\>[0]\AgdaKeyword{module}\AgdaSpace{}%
\AgdaModule{\AgdaUnderscore{}}\AgdaSpace{}%
\AgdaSymbol{\{}\AgdaBound{𝑨}\AgdaSpace{}%
\AgdaSymbol{:}\AgdaSpace{}%
\AgdaRecord{Algebra}\AgdaSpace{}%
\AgdaGeneralizable{α}\AgdaSpace{}%
\AgdaGeneralizable{ρᵃ}\AgdaSymbol{\}\{}\AgdaBound{ℓ}\AgdaSpace{}%
\AgdaBound{r}\AgdaSpace{}%
\AgdaSymbol{:}\AgdaSpace{}%
\AgdaPostulate{Level}\AgdaSymbol{\}}\AgdaSpace{}%
\AgdaKeyword{where}\<%
\\
\>[0][@{}l@{\AgdaIndent{0}}]%
\>[1]\AgdaKeyword{open}%
\>[7]\AgdaModule{Setoid}\AgdaSpace{}%
\AgdaOperator{\AgdaFunction{𝔻[}}\AgdaSpace{}%
\AgdaBound{𝑨}\AgdaSpace{}%
\AgdaOperator{\AgdaFunction{]}}%
\>[35]\AgdaKeyword{using}\AgdaSpace{}%
\AgdaSymbol{(}\AgdaSpace{}%
\AgdaFunction{refl}\AgdaSpace{}%
\AgdaSymbol{)}\<%
\\
%
\>[1]\AgdaKeyword{open}%
\>[7]\AgdaModule{Setoid}\AgdaSpace{}%
\AgdaOperator{\AgdaFunction{𝔻[}}\AgdaSpace{}%
\AgdaFunction{Lift-Alg}\AgdaSpace{}%
\AgdaBound{𝑨}\AgdaSpace{}%
\AgdaBound{ℓ}\AgdaSpace{}%
\AgdaBound{r}\AgdaSpace{}%
\AgdaOperator{\AgdaFunction{]}}%
\>[35]\AgdaKeyword{using}\AgdaSpace{}%
\AgdaSymbol{(}\AgdaSpace{}%
\AgdaOperator{\AgdaField{\AgdaUnderscore{}≈\AgdaUnderscore{}}}\AgdaSpace{}%
\AgdaSymbol{)}\<%
\\
%
\>[1]\AgdaKeyword{open}%
\>[7]\AgdaModule{Level}\<%
\\
%
\\[\AgdaEmptyExtraSkip]%
%
\>[1]\AgdaFunction{ToLift}\AgdaSpace{}%
\AgdaSymbol{:}\AgdaSpace{}%
\AgdaFunction{hom}\AgdaSpace{}%
\AgdaBound{𝑨}\AgdaSpace{}%
\AgdaSymbol{(}\AgdaFunction{Lift-Alg}\AgdaSpace{}%
\AgdaBound{𝑨}\AgdaSpace{}%
\AgdaBound{ℓ}\AgdaSpace{}%
\AgdaBound{r}\AgdaSymbol{)}\<%
\\
%
\>[1]\AgdaFunction{ToLift}\AgdaSpace{}%
\AgdaSymbol{=}\AgdaSpace{}%
\AgdaFunction{∘-hom}\AgdaSpace{}%
\AgdaFunction{ToLiftˡ}\AgdaSpace{}%
\AgdaFunction{ToLiftʳ}\<%
\\
%
\\[\AgdaEmptyExtraSkip]%
%
\>[1]\AgdaFunction{FromLift}\AgdaSpace{}%
\AgdaSymbol{:}\AgdaSpace{}%
\AgdaFunction{hom}\AgdaSpace{}%
\AgdaSymbol{(}\AgdaFunction{Lift-Alg}\AgdaSpace{}%
\AgdaBound{𝑨}\AgdaSpace{}%
\AgdaBound{ℓ}\AgdaSpace{}%
\AgdaBound{r}\AgdaSymbol{)}\AgdaSpace{}%
\AgdaBound{𝑨}\<%
\\
%
\>[1]\AgdaFunction{FromLift}\AgdaSpace{}%
\AgdaSymbol{=}\AgdaSpace{}%
\AgdaFunction{∘-hom}\AgdaSpace{}%
\AgdaFunction{FromLiftʳ}\AgdaSpace{}%
\AgdaFunction{FromLiftˡ}\<%
\\
%
\\[\AgdaEmptyExtraSkip]%
%
\>[1]\AgdaFunction{ToFromLift}\AgdaSpace{}%
\AgdaSymbol{:}\AgdaSpace{}%
\AgdaSymbol{∀}\AgdaSpace{}%
\AgdaBound{b}\AgdaSpace{}%
\AgdaSymbol{→}\AgdaSpace{}%
\AgdaSymbol{(}\AgdaOperator{\AgdaFunction{∣}}\AgdaSpace{}%
\AgdaFunction{ToLift}\AgdaSpace{}%
\AgdaOperator{\AgdaFunction{∣}}\AgdaSpace{}%
\AgdaOperator{\AgdaField{⟨\$⟩}}\AgdaSpace{}%
\AgdaSymbol{(}\AgdaOperator{\AgdaFunction{∣}}\AgdaSpace{}%
\AgdaFunction{FromLift}\AgdaSpace{}%
\AgdaOperator{\AgdaFunction{∣}}\AgdaSpace{}%
\AgdaOperator{\AgdaField{⟨\$⟩}}\AgdaSpace{}%
\AgdaBound{b}\AgdaSymbol{))}\AgdaSpace{}%
\AgdaOperator{\AgdaFunction{≈}}\AgdaSpace{}%
\AgdaBound{b}\<%
\\
%
\>[1]\AgdaFunction{ToFromLift}\AgdaSpace{}%
\AgdaBound{b}\AgdaSpace{}%
\AgdaSymbol{=}\AgdaSpace{}%
\AgdaInductiveConstructor{lift}\AgdaSpace{}%
\AgdaFunction{refl}\<%
\\
%
\\[\AgdaEmptyExtraSkip]%
%
\>[1]\AgdaFunction{ToLift-epi}\AgdaSpace{}%
\AgdaSymbol{:}\AgdaSpace{}%
\AgdaFunction{epi}\AgdaSpace{}%
\AgdaBound{𝑨}\AgdaSpace{}%
\AgdaSymbol{(}\AgdaFunction{Lift-Alg}\AgdaSpace{}%
\AgdaBound{𝑨}\AgdaSpace{}%
\AgdaBound{ℓ}\AgdaSpace{}%
\AgdaBound{r}\AgdaSymbol{)}\<%
\\
%
\>[1]\AgdaFunction{ToLift-epi}\AgdaSpace{}%
\AgdaSymbol{=}\AgdaSpace{}%
\AgdaOperator{\AgdaFunction{∣}}\AgdaSpace{}%
\AgdaFunction{ToLift}\AgdaSpace{}%
\AgdaOperator{\AgdaFunction{∣}}\AgdaSpace{}%
\AgdaOperator{\AgdaInductiveConstructor{,}}%
\>[28]\AgdaKeyword{record}\AgdaSpace{}%
\AgdaSymbol{\{}\AgdaSpace{}%
\AgdaField{isHom}\AgdaSpace{}%
\AgdaSymbol{=}\AgdaSpace{}%
\AgdaOperator{\AgdaFunction{∥}}\AgdaSpace{}%
\AgdaFunction{ToLift}\AgdaSpace{}%
\AgdaOperator{\AgdaFunction{∥}}\<%
\\
%
\>[28]\AgdaSymbol{;}\AgdaSpace{}%
\AgdaField{isSurjective}\AgdaSpace{}%
\AgdaSymbol{=}\AgdaSpace{}%
\AgdaSymbol{λ}\AgdaSpace{}%
\AgdaSymbol{\{}\AgdaBound{y}\AgdaSymbol{\}}\AgdaSpace{}%
\AgdaSymbol{→}\AgdaSpace{}%
\AgdaInductiveConstructor{eq}\AgdaSpace{}%
\AgdaSymbol{(}\AgdaOperator{\AgdaFunction{∣}}\AgdaSpace{}%
\AgdaFunction{FromLift}\AgdaSpace{}%
\AgdaOperator{\AgdaFunction{∣}}\AgdaSpace{}%
\AgdaOperator{\AgdaField{⟨\$⟩}}\AgdaSpace{}%
\AgdaBound{y}\AgdaSymbol{)}\AgdaSpace{}%
\AgdaSymbol{(}\AgdaFunction{ToFromLift}\AgdaSpace{}%
\AgdaBound{y}\AgdaSymbol{)}\AgdaSpace{}%
\AgdaSymbol{\}}\<%
\end{code}

\begin{code}[hide]%
\>[0]\<%
\\
\>[0]\AgdaKeyword{module}\AgdaSpace{}%
\AgdaModule{\AgdaUnderscore{}}\AgdaSpace{}%
\AgdaSymbol{\{}\AgdaBound{ι}\AgdaSpace{}%
\AgdaSymbol{:}\AgdaSpace{}%
\AgdaPostulate{Level}\AgdaSymbol{\}\{}\AgdaBound{I}\AgdaSpace{}%
\AgdaSymbol{:}\AgdaSpace{}%
\AgdaPrimitive{Type}\AgdaSpace{}%
\AgdaBound{ι}\AgdaSymbol{\}\{}\AgdaBound{𝑨}\AgdaSpace{}%
\AgdaSymbol{:}\AgdaSpace{}%
\AgdaRecord{Algebra}\AgdaSpace{}%
\AgdaGeneralizable{α}\AgdaSpace{}%
\AgdaGeneralizable{ρᵃ}\AgdaSymbol{\}(}\AgdaBound{ℬ}\AgdaSpace{}%
\AgdaSymbol{:}\AgdaSpace{}%
\AgdaBound{I}\AgdaSpace{}%
\AgdaSymbol{→}\AgdaSpace{}%
\AgdaRecord{Algebra}\AgdaSpace{}%
\AgdaGeneralizable{β}\AgdaSpace{}%
\AgdaGeneralizable{ρᵇ}\AgdaSymbol{)}%
\>[74]\AgdaKeyword{where}\<%
\\
\>[0][@{}l@{\AgdaIndent{0}}]%
\>[1]\AgdaFunction{⨅-hom-co}\AgdaSpace{}%
\AgdaSymbol{:}\AgdaSpace{}%
\AgdaSymbol{(∀(}\AgdaBound{i}\AgdaSpace{}%
\AgdaSymbol{:}\AgdaSpace{}%
\AgdaBound{I}\AgdaSymbol{)}\AgdaSpace{}%
\AgdaSymbol{→}\AgdaSpace{}%
\AgdaFunction{hom}\AgdaSpace{}%
\AgdaBound{𝑨}\AgdaSpace{}%
\AgdaSymbol{(}\AgdaBound{ℬ}\AgdaSpace{}%
\AgdaBound{i}\AgdaSymbol{))}\AgdaSpace{}%
\AgdaSymbol{→}\AgdaSpace{}%
\AgdaFunction{hom}\AgdaSpace{}%
\AgdaBound{𝑨}\AgdaSpace{}%
\AgdaSymbol{(}\AgdaFunction{⨅}\AgdaSpace{}%
\AgdaBound{ℬ}\AgdaSymbol{)}\<%
\\
%
\>[1]\AgdaFunction{⨅-hom-co}\AgdaSpace{}%
\AgdaBound{𝒽}\AgdaSpace{}%
\AgdaSymbol{=}\AgdaSpace{}%
\AgdaFunction{h}\AgdaSpace{}%
\AgdaOperator{\AgdaInductiveConstructor{,}}\AgdaSpace{}%
\AgdaFunction{hhom}\<%
\\
\>[1][@{}l@{\AgdaIndent{0}}]%
\>[2]\AgdaKeyword{where}\<%
\\
%
\>[2]\AgdaFunction{h}\AgdaSpace{}%
\AgdaSymbol{:}\AgdaSpace{}%
\AgdaOperator{\AgdaFunction{𝔻[}}\AgdaSpace{}%
\AgdaBound{𝑨}\AgdaSpace{}%
\AgdaOperator{\AgdaFunction{]}}\AgdaSpace{}%
\AgdaOperator{\AgdaRecord{⟶}}\AgdaSpace{}%
\AgdaOperator{\AgdaFunction{𝔻[}}\AgdaSpace{}%
\AgdaFunction{⨅}\AgdaSpace{}%
\AgdaBound{ℬ}\AgdaSpace{}%
\AgdaOperator{\AgdaFunction{]}}\<%
\\
%
\>[2]\AgdaOperator{\AgdaField{\AgdaUnderscore{}⟨\$⟩\AgdaUnderscore{}}}\AgdaSpace{}%
\AgdaFunction{h}\AgdaSpace{}%
\AgdaSymbol{=}\AgdaSpace{}%
\AgdaSymbol{λ}\AgdaSpace{}%
\AgdaBound{a}\AgdaSpace{}%
\AgdaBound{i}\AgdaSpace{}%
\AgdaSymbol{→}\AgdaSpace{}%
\AgdaOperator{\AgdaFunction{∣}}\AgdaSpace{}%
\AgdaBound{𝒽}\AgdaSpace{}%
\AgdaBound{i}\AgdaSpace{}%
\AgdaOperator{\AgdaFunction{∣}}\AgdaSpace{}%
\AgdaOperator{\AgdaField{⟨\$⟩}}\AgdaSpace{}%
\AgdaBound{a}\<%
\\
%
\>[2]\AgdaField{cong}\AgdaSpace{}%
\AgdaFunction{h}\AgdaSpace{}%
\AgdaBound{xy}\AgdaSpace{}%
\AgdaBound{i}\AgdaSpace{}%
\AgdaSymbol{=}\AgdaSpace{}%
\AgdaField{cong}\AgdaSpace{}%
\AgdaOperator{\AgdaFunction{∣}}\AgdaSpace{}%
\AgdaBound{𝒽}\AgdaSpace{}%
\AgdaBound{i}\AgdaSpace{}%
\AgdaOperator{\AgdaFunction{∣}}\AgdaSpace{}%
\AgdaBound{xy}\<%
\\
%
\>[2]\AgdaFunction{hhom}\AgdaSpace{}%
\AgdaSymbol{:}\AgdaSpace{}%
\AgdaRecord{IsHom}\AgdaSpace{}%
\AgdaBound{𝑨}\AgdaSpace{}%
\AgdaSymbol{(}\AgdaFunction{⨅}\AgdaSpace{}%
\AgdaBound{ℬ}\AgdaSymbol{)}\AgdaSpace{}%
\AgdaFunction{h}\<%
\\
%
\>[2]\AgdaField{compatible}\AgdaSpace{}%
\AgdaFunction{hhom}\AgdaSpace{}%
\AgdaSymbol{=}\AgdaSpace{}%
\AgdaSymbol{λ}\AgdaSpace{}%
\AgdaBound{i}\AgdaSpace{}%
\AgdaSymbol{→}\AgdaSpace{}%
\AgdaField{compatible}\AgdaSpace{}%
\AgdaOperator{\AgdaFunction{∥}}\AgdaSpace{}%
\AgdaBound{𝒽}\AgdaSpace{}%
\AgdaBound{i}\AgdaSpace{}%
\AgdaOperator{\AgdaFunction{∥}}\<%
\end{code}

\hypertarget{factorization-of-homomorphisms}{%
\subsubsection{Factorization of
homomorphisms}\label{factorization-of-homomorphisms}}

If \texttt{g\ :\ hom\ 𝑨\ 𝑩}, \texttt{h\ :\ hom\ 𝑨\ 𝑪}, \texttt{h} is
surjective, and \texttt{ker\ h\ ⊆\ ker\ g}, then there exists
\texttt{φ\ :\ hom\ 𝑪\ 𝑩} such that \texttt{g\ =\ φ\ ∘\ h} (cf.~the
\href{https://ualib.github.io/agda-algebras/Homomorphisms.Func.Factor.html}{Homomorphisms.Func.Factor}
module of the \href{https://ualib.github.io/agda-algebras}{Agda
Universal Algebra Library}).

\begin{code}%
\>[0]\<%
\\
\>[0]\AgdaKeyword{module}\AgdaSpace{}%
\AgdaModule{\AgdaUnderscore{}}%
\>[2154I]\AgdaSymbol{\{}\AgdaBound{𝑨}\AgdaSpace{}%
\AgdaSymbol{:}\AgdaSpace{}%
\AgdaRecord{Algebra}\AgdaSpace{}%
\AgdaGeneralizable{α}\AgdaSpace{}%
\AgdaGeneralizable{ρᵃ}\AgdaSymbol{\}(}\AgdaBound{𝑩}\AgdaSpace{}%
\AgdaSymbol{:}\AgdaSpace{}%
\AgdaRecord{Algebra}\AgdaSpace{}%
\AgdaGeneralizable{β}\AgdaSpace{}%
\AgdaGeneralizable{ρᵇ}\AgdaSymbol{)\{}\AgdaBound{𝑪}\AgdaSpace{}%
\AgdaSymbol{:}\AgdaSpace{}%
\AgdaRecord{Algebra}\AgdaSpace{}%
\AgdaGeneralizable{γ}\AgdaSpace{}%
\AgdaGeneralizable{ρᶜ}\AgdaSymbol{\}}\<%
\\
\>[.][@{}l@{}]\<[2154I]%
\>[9]\AgdaSymbol{(}\AgdaBound{gh}\AgdaSpace{}%
\AgdaSymbol{:}\AgdaSpace{}%
\AgdaFunction{hom}\AgdaSpace{}%
\AgdaBound{𝑨}\AgdaSpace{}%
\AgdaBound{𝑩}\AgdaSymbol{)(}\AgdaBound{hh}\AgdaSpace{}%
\AgdaSymbol{:}\AgdaSpace{}%
\AgdaFunction{hom}\AgdaSpace{}%
\AgdaBound{𝑨}\AgdaSpace{}%
\AgdaBound{𝑪}\AgdaSymbol{)}\AgdaSpace{}%
\AgdaKeyword{where}\<%
\\
\>[0][@{}l@{\AgdaIndent{0}}]%
\>[1]\AgdaKeyword{open}\AgdaSpace{}%
\AgdaModule{Setoid}\AgdaSpace{}%
\AgdaOperator{\AgdaFunction{𝔻[}}\AgdaSpace{}%
\AgdaBound{𝑩}\AgdaSpace{}%
\AgdaOperator{\AgdaFunction{]}}\AgdaSpace{}%
\AgdaKeyword{using}\AgdaSpace{}%
\AgdaSymbol{()}\AgdaSpace{}%
\AgdaKeyword{renaming}\AgdaSpace{}%
\AgdaSymbol{(}\AgdaSpace{}%
\AgdaOperator{\AgdaField{\AgdaUnderscore{}≈\AgdaUnderscore{}}}\AgdaSpace{}%
\AgdaSymbol{to}\AgdaSpace{}%
\AgdaOperator{\AgdaField{\AgdaUnderscore{}≈₂\AgdaUnderscore{}}}\AgdaSpace{}%
\AgdaSymbol{;}\AgdaSpace{}%
\AgdaFunction{sym}\AgdaSpace{}%
\AgdaSymbol{to}\AgdaSpace{}%
\AgdaFunction{sym₂}\AgdaSpace{}%
\AgdaSymbol{)}\<%
\\
%
\>[1]\AgdaKeyword{open}\AgdaSpace{}%
\AgdaModule{Setoid}\AgdaSpace{}%
\AgdaOperator{\AgdaFunction{𝔻[}}\AgdaSpace{}%
\AgdaBound{𝑪}\AgdaSpace{}%
\AgdaOperator{\AgdaFunction{]}}\AgdaSpace{}%
\AgdaKeyword{using}\AgdaSpace{}%
\AgdaSymbol{(}\AgdaSpace{}%
\AgdaFunction{trans}\AgdaSpace{}%
\AgdaSymbol{)}\AgdaSpace{}%
\AgdaKeyword{renaming}\AgdaSpace{}%
\AgdaSymbol{(}\AgdaSpace{}%
\AgdaOperator{\AgdaField{\AgdaUnderscore{}≈\AgdaUnderscore{}}}\AgdaSpace{}%
\AgdaSymbol{to}\AgdaSpace{}%
\AgdaOperator{\AgdaField{\AgdaUnderscore{}≈₃\AgdaUnderscore{}}}\AgdaSpace{}%
\AgdaSymbol{;}\AgdaSpace{}%
\AgdaFunction{sym}\AgdaSpace{}%
\AgdaSymbol{to}\AgdaSpace{}%
\AgdaFunction{sym₃}\AgdaSpace{}%
\AgdaSymbol{)}\<%
\\
%
\>[1]\AgdaKeyword{open}\AgdaSpace{}%
\AgdaModule{SetoidReasoning}\AgdaSpace{}%
\AgdaOperator{\AgdaFunction{𝔻[}}\AgdaSpace{}%
\AgdaBound{𝑩}\AgdaSpace{}%
\AgdaOperator{\AgdaFunction{]}}\<%
\\
%
\>[1]\AgdaKeyword{private}\AgdaSpace{}%
\AgdaFunction{gfunc}\AgdaSpace{}%
\AgdaSymbol{=}\AgdaSpace{}%
\AgdaOperator{\AgdaFunction{∣}}\AgdaSpace{}%
\AgdaBound{gh}\AgdaSpace{}%
\AgdaOperator{\AgdaFunction{∣}}\AgdaSpace{}%
\AgdaSymbol{;}\AgdaSpace{}%
\AgdaFunction{g}\AgdaSpace{}%
\AgdaSymbol{=}\AgdaSpace{}%
\AgdaOperator{\AgdaField{\AgdaUnderscore{}⟨\$⟩\AgdaUnderscore{}}}\AgdaSpace{}%
\AgdaFunction{gfunc}\AgdaSpace{}%
\AgdaSymbol{;}\AgdaSpace{}%
\AgdaFunction{hfunc}\AgdaSpace{}%
\AgdaSymbol{=}\AgdaSpace{}%
\AgdaOperator{\AgdaFunction{∣}}\AgdaSpace{}%
\AgdaBound{hh}\AgdaSpace{}%
\AgdaOperator{\AgdaFunction{∣}}\AgdaSpace{}%
\AgdaSymbol{;}\AgdaSpace{}%
\AgdaFunction{h}\AgdaSpace{}%
\AgdaSymbol{=}\AgdaSpace{}%
\AgdaOperator{\AgdaField{\AgdaUnderscore{}⟨\$⟩\AgdaUnderscore{}}}\AgdaSpace{}%
\AgdaFunction{hfunc}\<%
\\
%
\\[\AgdaEmptyExtraSkip]%
%
\>[1]\AgdaFunction{HomFactor}\AgdaSpace{}%
\AgdaSymbol{:}%
\>[14]\AgdaFunction{kernel}\AgdaSpace{}%
\AgdaOperator{\AgdaFunction{\AgdaUnderscore{}≈₃\AgdaUnderscore{}}}\AgdaSpace{}%
\AgdaFunction{h}\AgdaSpace{}%
\AgdaOperator{\AgdaFunction{⊆}}\AgdaSpace{}%
\AgdaFunction{kernel}\AgdaSpace{}%
\AgdaOperator{\AgdaFunction{\AgdaUnderscore{}≈₂\AgdaUnderscore{}}}\AgdaSpace{}%
\AgdaFunction{g}\AgdaSpace{}%
\AgdaSymbol{→}\AgdaSpace{}%
\AgdaFunction{IsSurjective}\AgdaSpace{}%
\AgdaFunction{hfunc}\<%
\\
\>[1][@{}l@{\AgdaIndent{0}}]%
\>[2]\AgdaSymbol{→}%
\>[14]\AgdaFunction{Σ[}\AgdaSpace{}%
\AgdaBound{φ}\AgdaSpace{}%
\AgdaFunction{∈}\AgdaSpace{}%
\AgdaFunction{hom}\AgdaSpace{}%
\AgdaBound{𝑪}\AgdaSpace{}%
\AgdaBound{𝑩}\AgdaSpace{}%
\AgdaFunction{]}\AgdaSpace{}%
\AgdaSymbol{∀}\AgdaSpace{}%
\AgdaBound{a}\AgdaSpace{}%
\AgdaSymbol{→}\AgdaSpace{}%
\AgdaSymbol{(}\AgdaFunction{g}\AgdaSpace{}%
\AgdaBound{a}\AgdaSymbol{)}\AgdaSpace{}%
\AgdaOperator{\AgdaFunction{≈₂}}\AgdaSpace{}%
\AgdaOperator{\AgdaFunction{∣}}\AgdaSpace{}%
\AgdaBound{φ}\AgdaSpace{}%
\AgdaOperator{\AgdaFunction{∣}}\AgdaSpace{}%
\AgdaOperator{\AgdaField{⟨\$⟩}}\AgdaSpace{}%
\AgdaSymbol{(}\AgdaFunction{h}\AgdaSpace{}%
\AgdaBound{a}\AgdaSymbol{)}\<%
\\
%
\\[\AgdaEmptyExtraSkip]%
%
\>[1]\AgdaFunction{HomFactor}\AgdaSpace{}%
\AgdaBound{Khg}\AgdaSpace{}%
\AgdaBound{hE}\AgdaSpace{}%
\AgdaSymbol{=}\AgdaSpace{}%
\AgdaSymbol{(}\AgdaFunction{φmap}\AgdaSpace{}%
\AgdaOperator{\AgdaInductiveConstructor{,}}\AgdaSpace{}%
\AgdaFunction{φhom}\AgdaSymbol{)}\AgdaSpace{}%
\AgdaOperator{\AgdaInductiveConstructor{,}}\AgdaSpace{}%
\AgdaFunction{gφh}\<%
\\
\>[1][@{}l@{\AgdaIndent{0}}]%
\>[2]\AgdaKeyword{where}\<%
\\
%
\>[2]\AgdaFunction{kerpres}\AgdaSpace{}%
\AgdaSymbol{:}\AgdaSpace{}%
\AgdaSymbol{∀}\AgdaSpace{}%
\AgdaBound{a₀}\AgdaSpace{}%
\AgdaBound{a₁}\AgdaSpace{}%
\AgdaSymbol{→}\AgdaSpace{}%
\AgdaFunction{h}\AgdaSpace{}%
\AgdaBound{a₀}\AgdaSpace{}%
\AgdaOperator{\AgdaFunction{≈₃}}\AgdaSpace{}%
\AgdaFunction{h}\AgdaSpace{}%
\AgdaBound{a₁}\AgdaSpace{}%
\AgdaSymbol{→}\AgdaSpace{}%
\AgdaFunction{g}\AgdaSpace{}%
\AgdaBound{a₀}\AgdaSpace{}%
\AgdaOperator{\AgdaFunction{≈₂}}\AgdaSpace{}%
\AgdaFunction{g}\AgdaSpace{}%
\AgdaBound{a₁}\<%
\\
%
\>[2]\AgdaFunction{kerpres}\AgdaSpace{}%
\AgdaBound{a₀}\AgdaSpace{}%
\AgdaBound{a₁}\AgdaSpace{}%
\AgdaBound{hyp}\AgdaSpace{}%
\AgdaSymbol{=}\AgdaSpace{}%
\AgdaBound{Khg}\AgdaSpace{}%
\AgdaBound{hyp}\<%
\\
%
\\[\AgdaEmptyExtraSkip]%
%
\>[2]\AgdaFunction{h⁻¹}\AgdaSpace{}%
\AgdaSymbol{:}\AgdaSpace{}%
\AgdaOperator{\AgdaFunction{𝕌[}}\AgdaSpace{}%
\AgdaBound{𝑪}\AgdaSpace{}%
\AgdaOperator{\AgdaFunction{]}}\AgdaSpace{}%
\AgdaSymbol{→}\AgdaSpace{}%
\AgdaOperator{\AgdaFunction{𝕌[}}\AgdaSpace{}%
\AgdaBound{𝑨}\AgdaSpace{}%
\AgdaOperator{\AgdaFunction{]}}\<%
\\
%
\>[2]\AgdaFunction{h⁻¹}\AgdaSpace{}%
\AgdaSymbol{=}\AgdaSpace{}%
\AgdaFunction{SurjInv}\AgdaSpace{}%
\AgdaFunction{hfunc}\AgdaSpace{}%
\AgdaBound{hE}\<%
\\
%
\\[\AgdaEmptyExtraSkip]%
%
\>[2]\AgdaFunction{η}\AgdaSpace{}%
\AgdaSymbol{:}\AgdaSpace{}%
\AgdaSymbol{∀}\AgdaSpace{}%
\AgdaSymbol{\{}\AgdaBound{c}\AgdaSymbol{\}}\AgdaSpace{}%
\AgdaSymbol{→}\AgdaSpace{}%
\AgdaFunction{h}\AgdaSpace{}%
\AgdaSymbol{(}\AgdaFunction{h⁻¹}\AgdaSpace{}%
\AgdaBound{c}\AgdaSymbol{)}\AgdaSpace{}%
\AgdaOperator{\AgdaFunction{≈₃}}\AgdaSpace{}%
\AgdaBound{c}\<%
\\
%
\>[2]\AgdaFunction{η}\AgdaSpace{}%
\AgdaSymbol{=}\AgdaSpace{}%
\AgdaFunction{InvIsInverseʳ}\AgdaSpace{}%
\AgdaBound{hE}\<%
\\
%
\\[\AgdaEmptyExtraSkip]%
%
\>[2]\AgdaFunction{ζ}\AgdaSpace{}%
\AgdaSymbol{:}\AgdaSpace{}%
\AgdaSymbol{∀\{}\AgdaBound{x}\AgdaSpace{}%
\AgdaBound{y}\AgdaSymbol{\}}\AgdaSpace{}%
\AgdaSymbol{→}\AgdaSpace{}%
\AgdaBound{x}\AgdaSpace{}%
\AgdaOperator{\AgdaFunction{≈₃}}\AgdaSpace{}%
\AgdaBound{y}\AgdaSpace{}%
\AgdaSymbol{→}\AgdaSpace{}%
\AgdaFunction{h}\AgdaSpace{}%
\AgdaSymbol{(}\AgdaFunction{h⁻¹}\AgdaSpace{}%
\AgdaBound{x}\AgdaSymbol{)}\AgdaSpace{}%
\AgdaOperator{\AgdaFunction{≈₃}}\AgdaSpace{}%
\AgdaFunction{h}\AgdaSpace{}%
\AgdaSymbol{(}\AgdaFunction{h⁻¹}\AgdaSpace{}%
\AgdaBound{y}\AgdaSymbol{)}\<%
\\
%
\>[2]\AgdaFunction{ζ}\AgdaSpace{}%
\AgdaBound{xy}\AgdaSpace{}%
\AgdaSymbol{=}\AgdaSpace{}%
\AgdaFunction{trans}\AgdaSpace{}%
\AgdaFunction{η}\AgdaSpace{}%
\AgdaSymbol{(}\AgdaFunction{trans}\AgdaSpace{}%
\AgdaBound{xy}\AgdaSpace{}%
\AgdaSymbol{(}\AgdaFunction{sym₃}\AgdaSpace{}%
\AgdaFunction{η}\AgdaSymbol{))}\<%
\\
%
\\[\AgdaEmptyExtraSkip]%
%
\>[2]\AgdaFunction{φmap}\AgdaSpace{}%
\AgdaSymbol{:}\AgdaSpace{}%
\AgdaOperator{\AgdaFunction{𝔻[}}\AgdaSpace{}%
\AgdaBound{𝑪}\AgdaSpace{}%
\AgdaOperator{\AgdaFunction{]}}\AgdaSpace{}%
\AgdaOperator{\AgdaRecord{⟶}}\AgdaSpace{}%
\AgdaOperator{\AgdaFunction{𝔻[}}\AgdaSpace{}%
\AgdaBound{𝑩}\AgdaSpace{}%
\AgdaOperator{\AgdaFunction{]}}\<%
\\
%
\>[2]\AgdaOperator{\AgdaField{\AgdaUnderscore{}⟨\$⟩\AgdaUnderscore{}}}\AgdaSpace{}%
\AgdaFunction{φmap}\AgdaSpace{}%
\AgdaSymbol{=}\AgdaSpace{}%
\AgdaFunction{g}\AgdaSpace{}%
\AgdaOperator{\AgdaFunction{∘}}\AgdaSpace{}%
\AgdaFunction{h⁻¹}\<%
\\
%
\>[2]\AgdaField{cong}\AgdaSpace{}%
\AgdaFunction{φmap}\AgdaSpace{}%
\AgdaSymbol{=}\AgdaSpace{}%
\AgdaBound{Khg}\AgdaSpace{}%
\AgdaOperator{\AgdaFunction{∘}}\AgdaSpace{}%
\AgdaFunction{ζ}\<%
\\
%
\>[2]\AgdaKeyword{open}\AgdaSpace{}%
\AgdaModule{\AgdaUnderscore{}⟶\AgdaUnderscore{}}\AgdaSpace{}%
\AgdaFunction{φmap}\AgdaSpace{}%
\AgdaKeyword{using}\AgdaSpace{}%
\AgdaSymbol{()}\AgdaSpace{}%
\AgdaKeyword{renaming}\AgdaSpace{}%
\AgdaSymbol{(}\AgdaField{cong}\AgdaSpace{}%
\AgdaSymbol{to}\AgdaSpace{}%
\AgdaField{φcong}\AgdaSymbol{)}\<%
\\
%
\\[\AgdaEmptyExtraSkip]%
%
\>[2]\AgdaFunction{gφh}\AgdaSpace{}%
\AgdaSymbol{:}\AgdaSpace{}%
\AgdaSymbol{(}\AgdaBound{a}\AgdaSpace{}%
\AgdaSymbol{:}\AgdaSpace{}%
\AgdaOperator{\AgdaFunction{𝕌[}}\AgdaSpace{}%
\AgdaBound{𝑨}\AgdaSpace{}%
\AgdaOperator{\AgdaFunction{]}}\AgdaSymbol{)}\AgdaSpace{}%
\AgdaSymbol{→}\AgdaSpace{}%
\AgdaFunction{g}\AgdaSpace{}%
\AgdaBound{a}\AgdaSpace{}%
\AgdaOperator{\AgdaFunction{≈₂}}\AgdaSpace{}%
\AgdaFunction{φmap}\AgdaSpace{}%
\AgdaOperator{\AgdaField{⟨\$⟩}}\AgdaSpace{}%
\AgdaSymbol{(}\AgdaFunction{h}\AgdaSpace{}%
\AgdaBound{a}\AgdaSymbol{)}\<%
\\
%
\>[2]\AgdaFunction{gφh}\AgdaSpace{}%
\AgdaBound{a}\AgdaSpace{}%
\AgdaSymbol{=}\AgdaSpace{}%
\AgdaBound{Khg}\AgdaSpace{}%
\AgdaSymbol{(}\AgdaFunction{sym₃}\AgdaSpace{}%
\AgdaFunction{η}\AgdaSymbol{)}\<%
\\
%
\\[\AgdaEmptyExtraSkip]%
%
\>[2]\AgdaFunction{φcomp}\AgdaSpace{}%
\AgdaSymbol{:}\AgdaSpace{}%
\AgdaFunction{compatible-map}\AgdaSpace{}%
\AgdaBound{𝑪}\AgdaSpace{}%
\AgdaBound{𝑩}\AgdaSpace{}%
\AgdaFunction{φmap}\<%
\\
%
\>[2]\AgdaFunction{φcomp}\AgdaSpace{}%
\AgdaSymbol{\{}\AgdaBound{f}\AgdaSymbol{\}\{}\AgdaBound{c}\AgdaSymbol{\}}\AgdaSpace{}%
\AgdaSymbol{=}\<%
\\
\>[2][@{}l@{\AgdaIndent{0}}]%
\>[3]\AgdaOperator{\AgdaFunction{begin}}\<%
\\
\>[3][@{}l@{\AgdaIndent{0}}]%
\>[4]\AgdaFunction{φmap}\AgdaSpace{}%
\AgdaOperator{\AgdaField{⟨\$⟩}}\AgdaSpace{}%
\AgdaSymbol{((}\AgdaBound{f}\AgdaSpace{}%
\AgdaOperator{\AgdaFunction{̂}}\AgdaSpace{}%
\AgdaBound{𝑪}\AgdaSymbol{)}\AgdaSpace{}%
\AgdaBound{c}\AgdaSymbol{)}%
\>[36]\AgdaFunction{≈˘⟨}%
\>[41]\AgdaFunction{φcong}\AgdaSpace{}%
\AgdaSymbol{(}\AgdaField{cong}\AgdaSpace{}%
\AgdaSymbol{(}\AgdaField{Interp}\AgdaSpace{}%
\AgdaBound{𝑪}\AgdaSymbol{)}\AgdaSpace{}%
\AgdaSymbol{(}\AgdaInductiveConstructor{≡.refl}\AgdaSpace{}%
\AgdaOperator{\AgdaInductiveConstructor{,}}\AgdaSpace{}%
\AgdaSymbol{λ}\AgdaSpace{}%
\AgdaBound{\AgdaUnderscore{}}\AgdaSpace{}%
\AgdaSymbol{→}\AgdaSpace{}%
\AgdaFunction{η}\AgdaSymbol{))}%
\>[85]\AgdaFunction{⟩}\<%
\\
%
\>[4]\AgdaFunction{g}\AgdaSpace{}%
\AgdaSymbol{(}\AgdaFunction{h⁻¹}\AgdaSpace{}%
\AgdaSymbol{((}\AgdaBound{f}\AgdaSpace{}%
\AgdaOperator{\AgdaFunction{̂}}\AgdaSpace{}%
\AgdaBound{𝑪}\AgdaSymbol{)(}\AgdaFunction{h}\AgdaSpace{}%
\AgdaOperator{\AgdaFunction{∘}}\AgdaSpace{}%
\AgdaFunction{h⁻¹}\AgdaSpace{}%
\AgdaOperator{\AgdaFunction{∘}}\AgdaSpace{}%
\AgdaBound{c}\AgdaSymbol{)))}%
\>[36]\AgdaFunction{≈˘⟨}%
\>[41]\AgdaFunction{φcong}\AgdaSpace{}%
\AgdaSymbol{(}\AgdaField{compatible}\AgdaSpace{}%
\AgdaOperator{\AgdaFunction{∥}}\AgdaSpace{}%
\AgdaBound{hh}\AgdaSpace{}%
\AgdaOperator{\AgdaFunction{∥}}\AgdaSymbol{)}%
\>[85]\AgdaFunction{⟩}\<%
\\
%
\>[4]\AgdaFunction{g}\AgdaSpace{}%
\AgdaSymbol{(}\AgdaFunction{h⁻¹}\AgdaSpace{}%
\AgdaSymbol{(}\AgdaFunction{h}\AgdaSpace{}%
\AgdaSymbol{((}\AgdaBound{f}\AgdaSpace{}%
\AgdaOperator{\AgdaFunction{̂}}\AgdaSpace{}%
\AgdaBound{𝑨}\AgdaSymbol{)(}\AgdaFunction{h⁻¹}\AgdaSpace{}%
\AgdaOperator{\AgdaFunction{∘}}\AgdaSpace{}%
\AgdaBound{c}\AgdaSymbol{))))}%
\>[36]\AgdaFunction{≈˘⟨}%
\>[41]\AgdaFunction{gφh}\AgdaSpace{}%
\AgdaSymbol{((}\AgdaBound{f}\AgdaSpace{}%
\AgdaOperator{\AgdaFunction{̂}}\AgdaSpace{}%
\AgdaBound{𝑨}\AgdaSymbol{)(}\AgdaFunction{h⁻¹}\AgdaSpace{}%
\AgdaOperator{\AgdaFunction{∘}}\AgdaSpace{}%
\AgdaBound{c}\AgdaSymbol{))}%
\>[85]\AgdaFunction{⟩}\<%
\\
%
\>[4]\AgdaFunction{g}\AgdaSpace{}%
\AgdaSymbol{((}\AgdaBound{f}\AgdaSpace{}%
\AgdaOperator{\AgdaFunction{̂}}\AgdaSpace{}%
\AgdaBound{𝑨}\AgdaSymbol{)(}\AgdaFunction{h⁻¹}\AgdaSpace{}%
\AgdaOperator{\AgdaFunction{∘}}\AgdaSpace{}%
\AgdaBound{c}\AgdaSymbol{))}%
\>[36]\AgdaFunction{≈⟨}%
\>[41]\AgdaField{compatible}\AgdaSpace{}%
\AgdaOperator{\AgdaFunction{∥}}\AgdaSpace{}%
\AgdaBound{gh}\AgdaSpace{}%
\AgdaOperator{\AgdaFunction{∥}}%
\>[85]\AgdaFunction{⟩}\<%
\\
%
\>[4]\AgdaSymbol{(}\AgdaBound{f}\AgdaSpace{}%
\AgdaOperator{\AgdaFunction{̂}}\AgdaSpace{}%
\AgdaBound{𝑩}\AgdaSymbol{)(}\AgdaFunction{g}\AgdaSpace{}%
\AgdaOperator{\AgdaFunction{∘}}\AgdaSpace{}%
\AgdaSymbol{(}\AgdaFunction{h⁻¹}\AgdaSpace{}%
\AgdaOperator{\AgdaFunction{∘}}\AgdaSpace{}%
\AgdaBound{c}\AgdaSymbol{))}%
\>[36]\AgdaOperator{\AgdaFunction{∎}}\<%
\\
%
\\[\AgdaEmptyExtraSkip]%
%
\>[2]\AgdaFunction{φhom}\AgdaSpace{}%
\AgdaSymbol{:}\AgdaSpace{}%
\AgdaRecord{IsHom}\AgdaSpace{}%
\AgdaBound{𝑪}\AgdaSpace{}%
\AgdaBound{𝑩}\AgdaSpace{}%
\AgdaFunction{φmap}\<%
\\
%
\>[2]\AgdaField{compatible}\AgdaSpace{}%
\AgdaFunction{φhom}\AgdaSpace{}%
\AgdaSymbol{=}\AgdaSpace{}%
\AgdaFunction{φcomp}\<%
\end{code}

\hypertarget{isomorphisms}{%
\subsubsection{Isomorphisms}\label{isomorphisms}}

(cf.~the
\href{https://ualib.github.io/agda-algebras/Homomorphisms.Func.Isomorphisms.html}{Homomorphisms.Func.Isomorphisms}
of the \href{https://ualib.github.io/agda-algebras}{Agda Universal
Algebra Library}.)

Two structures are \emph{isomorphic} provided there are homomorphisms
going back and forth between them which compose to the identity map.

\begin{code}%
\>[0]\<%
\\
\>[0]\AgdaKeyword{module}\AgdaSpace{}%
\AgdaModule{\AgdaUnderscore{}}\AgdaSpace{}%
\AgdaSymbol{(}\AgdaBound{𝑨}\AgdaSpace{}%
\AgdaSymbol{:}\AgdaSpace{}%
\AgdaRecord{Algebra}\AgdaSpace{}%
\AgdaGeneralizable{α}\AgdaSpace{}%
\AgdaGeneralizable{ρᵃ}\AgdaSymbol{)}\AgdaSpace{}%
\AgdaSymbol{(}\AgdaBound{𝑩}\AgdaSpace{}%
\AgdaSymbol{:}\AgdaSpace{}%
\AgdaRecord{Algebra}\AgdaSpace{}%
\AgdaGeneralizable{β}\AgdaSpace{}%
\AgdaGeneralizable{ρᵇ}\AgdaSymbol{)}\AgdaSpace{}%
\AgdaKeyword{where}\<%
\\
\>[0][@{}l@{\AgdaIndent{0}}]%
\>[1]\AgdaKeyword{open}\AgdaSpace{}%
\AgdaModule{Setoid}\AgdaSpace{}%
\AgdaOperator{\AgdaFunction{𝔻[}}\AgdaSpace{}%
\AgdaBound{𝑨}\AgdaSpace{}%
\AgdaOperator{\AgdaFunction{]}}\AgdaSpace{}%
\AgdaKeyword{using}\AgdaSpace{}%
\AgdaSymbol{(}\AgdaSpace{}%
\AgdaOperator{\AgdaField{\AgdaUnderscore{}≈\AgdaUnderscore{}}}\AgdaSpace{}%
\AgdaSymbol{;}\AgdaSpace{}%
\AgdaFunction{sym}\AgdaSpace{}%
\AgdaSymbol{;}\AgdaSpace{}%
\AgdaFunction{trans}\AgdaSpace{}%
\AgdaSymbol{)}\<%
\\
%
\>[1]\AgdaKeyword{open}\AgdaSpace{}%
\AgdaModule{Setoid}\AgdaSpace{}%
\AgdaOperator{\AgdaFunction{𝔻[}}\AgdaSpace{}%
\AgdaBound{𝑩}\AgdaSpace{}%
\AgdaOperator{\AgdaFunction{]}}\AgdaSpace{}%
\AgdaKeyword{using}\AgdaSpace{}%
\AgdaSymbol{()}\AgdaSpace{}%
\AgdaKeyword{renaming}\AgdaSpace{}%
\AgdaSymbol{(}\AgdaSpace{}%
\AgdaOperator{\AgdaField{\AgdaUnderscore{}≈\AgdaUnderscore{}}}\AgdaSpace{}%
\AgdaSymbol{to}\AgdaSpace{}%
\AgdaOperator{\AgdaField{\AgdaUnderscore{}≈ᵇ\AgdaUnderscore{}}}\AgdaSpace{}%
\AgdaSymbol{;}\AgdaSpace{}%
\AgdaFunction{sym}\AgdaSpace{}%
\AgdaSymbol{to}\AgdaSpace{}%
\AgdaFunction{symᵇ}\AgdaSpace{}%
\AgdaSymbol{;}\AgdaSpace{}%
\AgdaFunction{trans}\AgdaSpace{}%
\AgdaSymbol{to}\AgdaSpace{}%
\AgdaFunction{transᵇ}\AgdaSymbol{)}\<%
\\
%
\\[\AgdaEmptyExtraSkip]%
%
\>[1]\AgdaKeyword{record}\AgdaSpace{}%
\AgdaOperator{\AgdaRecord{\AgdaUnderscore{}≅\AgdaUnderscore{}}}\AgdaSpace{}%
\AgdaSymbol{:}\AgdaSpace{}%
\AgdaPrimitive{Type}\AgdaSpace{}%
\AgdaSymbol{(}\AgdaBound{𝓞}\AgdaSpace{}%
\AgdaOperator{\AgdaPrimitive{⊔}}\AgdaSpace{}%
\AgdaBound{𝓥}\AgdaSpace{}%
\AgdaOperator{\AgdaPrimitive{⊔}}\AgdaSpace{}%
\AgdaBound{α}\AgdaSpace{}%
\AgdaOperator{\AgdaPrimitive{⊔}}\AgdaSpace{}%
\AgdaBound{β}\AgdaSpace{}%
\AgdaOperator{\AgdaPrimitive{⊔}}\AgdaSpace{}%
\AgdaBound{ρᵃ}\AgdaSpace{}%
\AgdaOperator{\AgdaPrimitive{⊔}}\AgdaSpace{}%
\AgdaBound{ρᵇ}\AgdaSpace{}%
\AgdaSymbol{)}\AgdaSpace{}%
\AgdaKeyword{where}\<%
\\
\>[1][@{}l@{\AgdaIndent{0}}]%
\>[2]\AgdaKeyword{constructor}\AgdaSpace{}%
\AgdaInductiveConstructor{mkiso}\<%
\\
%
\>[2]\AgdaKeyword{field}\<%
\\
\>[2][@{}l@{\AgdaIndent{0}}]%
\>[3]\AgdaField{to}\AgdaSpace{}%
\AgdaSymbol{:}\AgdaSpace{}%
\AgdaFunction{hom}\AgdaSpace{}%
\AgdaBound{𝑨}\AgdaSpace{}%
\AgdaBound{𝑩}\<%
\\
%
\>[3]\AgdaField{from}\AgdaSpace{}%
\AgdaSymbol{:}\AgdaSpace{}%
\AgdaFunction{hom}\AgdaSpace{}%
\AgdaBound{𝑩}\AgdaSpace{}%
\AgdaBound{𝑨}\<%
\\
%
\>[3]\AgdaField{to∼from}\AgdaSpace{}%
\AgdaSymbol{:}\AgdaSpace{}%
\AgdaSymbol{∀}\AgdaSpace{}%
\AgdaBound{b}\AgdaSpace{}%
\AgdaSymbol{→}\AgdaSpace{}%
\AgdaSymbol{(}\AgdaOperator{\AgdaFunction{∣}}\AgdaSpace{}%
\AgdaField{to}\AgdaSpace{}%
\AgdaOperator{\AgdaFunction{∣}}\AgdaSpace{}%
\AgdaOperator{\AgdaField{⟨\$⟩}}\AgdaSpace{}%
\AgdaSymbol{(}\AgdaOperator{\AgdaFunction{∣}}\AgdaSpace{}%
\AgdaField{from}\AgdaSpace{}%
\AgdaOperator{\AgdaFunction{∣}}\AgdaSpace{}%
\AgdaOperator{\AgdaField{⟨\$⟩}}\AgdaSpace{}%
\AgdaBound{b}\AgdaSymbol{))}\AgdaSpace{}%
\AgdaOperator{\AgdaFunction{≈ᵇ}}\AgdaSpace{}%
\AgdaBound{b}\<%
\\
%
\>[3]\AgdaField{from∼to}\AgdaSpace{}%
\AgdaSymbol{:}\AgdaSpace{}%
\AgdaSymbol{∀}\AgdaSpace{}%
\AgdaBound{a}\AgdaSpace{}%
\AgdaSymbol{→}\AgdaSpace{}%
\AgdaSymbol{(}\AgdaOperator{\AgdaFunction{∣}}\AgdaSpace{}%
\AgdaField{from}\AgdaSpace{}%
\AgdaOperator{\AgdaFunction{∣}}\AgdaSpace{}%
\AgdaOperator{\AgdaField{⟨\$⟩}}\AgdaSpace{}%
\AgdaSymbol{(}\AgdaOperator{\AgdaFunction{∣}}\AgdaSpace{}%
\AgdaField{to}\AgdaSpace{}%
\AgdaOperator{\AgdaFunction{∣}}\AgdaSpace{}%
\AgdaOperator{\AgdaField{⟨\$⟩}}\AgdaSpace{}%
\AgdaBound{a}\AgdaSymbol{))}\AgdaSpace{}%
\AgdaOperator{\AgdaFunction{≈}}\AgdaSpace{}%
\AgdaBound{a}\<%
\\
%
\\[\AgdaEmptyExtraSkip]%
%
\>[2]\AgdaFunction{toIsSurjective}\AgdaSpace{}%
\AgdaSymbol{:}\AgdaSpace{}%
\AgdaFunction{IsSurjective}\AgdaSpace{}%
\AgdaOperator{\AgdaFunction{∣}}\AgdaSpace{}%
\AgdaField{to}\AgdaSpace{}%
\AgdaOperator{\AgdaFunction{∣}}\<%
\\
%
\>[2]\AgdaFunction{toIsSurjective}\AgdaSpace{}%
\AgdaSymbol{\{}\AgdaBound{y}\AgdaSymbol{\}}\AgdaSpace{}%
\AgdaSymbol{=}\AgdaSpace{}%
\AgdaInductiveConstructor{eq}\AgdaSpace{}%
\AgdaSymbol{(}\AgdaOperator{\AgdaFunction{∣}}\AgdaSpace{}%
\AgdaField{from}\AgdaSpace{}%
\AgdaOperator{\AgdaFunction{∣}}\AgdaSpace{}%
\AgdaOperator{\AgdaField{⟨\$⟩}}\AgdaSpace{}%
\AgdaBound{y}\AgdaSymbol{)}\AgdaSpace{}%
\AgdaSymbol{(}\AgdaFunction{symᵇ}\AgdaSpace{}%
\AgdaSymbol{(}\AgdaField{to∼from}\AgdaSpace{}%
\AgdaBound{y}\AgdaSymbol{))}\<%
\\
%
\\[\AgdaEmptyExtraSkip]%
%
\>[2]\AgdaFunction{toIsInjective}\AgdaSpace{}%
\AgdaSymbol{:}\AgdaSpace{}%
\AgdaFunction{IsInjective}\AgdaSpace{}%
\AgdaOperator{\AgdaFunction{∣}}\AgdaSpace{}%
\AgdaField{to}\AgdaSpace{}%
\AgdaOperator{\AgdaFunction{∣}}\<%
\\
%
\>[2]\AgdaFunction{toIsInjective}\AgdaSpace{}%
\AgdaSymbol{\{}\AgdaBound{x}\AgdaSymbol{\}}\AgdaSpace{}%
\AgdaSymbol{\{}\AgdaBound{y}\AgdaSymbol{\}}\AgdaSpace{}%
\AgdaBound{xy}\AgdaSpace{}%
\AgdaSymbol{=}\AgdaSpace{}%
\AgdaFunction{Goal}\<%
\\
\>[2][@{}l@{\AgdaIndent{0}}]%
\>[3]\AgdaKeyword{where}\<%
\\
%
\>[3]\AgdaFunction{ξ}\AgdaSpace{}%
\AgdaSymbol{:}\AgdaSpace{}%
\AgdaOperator{\AgdaFunction{∣}}\AgdaSpace{}%
\AgdaField{from}\AgdaSpace{}%
\AgdaOperator{\AgdaFunction{∣}}\AgdaSpace{}%
\AgdaOperator{\AgdaField{⟨\$⟩}}\AgdaSpace{}%
\AgdaSymbol{(}\AgdaOperator{\AgdaFunction{∣}}\AgdaSpace{}%
\AgdaField{to}\AgdaSpace{}%
\AgdaOperator{\AgdaFunction{∣}}\AgdaSpace{}%
\AgdaOperator{\AgdaField{⟨\$⟩}}\AgdaSpace{}%
\AgdaBound{x}\AgdaSymbol{)}\AgdaSpace{}%
\AgdaOperator{\AgdaFunction{≈}}\AgdaSpace{}%
\AgdaOperator{\AgdaFunction{∣}}\AgdaSpace{}%
\AgdaField{from}\AgdaSpace{}%
\AgdaOperator{\AgdaFunction{∣}}\AgdaSpace{}%
\AgdaOperator{\AgdaField{⟨\$⟩}}\AgdaSpace{}%
\AgdaSymbol{(}\AgdaOperator{\AgdaFunction{∣}}\AgdaSpace{}%
\AgdaField{to}\AgdaSpace{}%
\AgdaOperator{\AgdaFunction{∣}}\AgdaSpace{}%
\AgdaOperator{\AgdaField{⟨\$⟩}}\AgdaSpace{}%
\AgdaBound{y}\AgdaSymbol{)}\<%
\\
%
\>[3]\AgdaFunction{ξ}\AgdaSpace{}%
\AgdaSymbol{=}\AgdaSpace{}%
\AgdaField{cong}\AgdaSpace{}%
\AgdaOperator{\AgdaFunction{∣}}\AgdaSpace{}%
\AgdaField{from}\AgdaSpace{}%
\AgdaOperator{\AgdaFunction{∣}}\AgdaSpace{}%
\AgdaBound{xy}\<%
\\
%
\>[3]\AgdaFunction{Goal}\AgdaSpace{}%
\AgdaSymbol{:}\AgdaSpace{}%
\AgdaBound{x}\AgdaSpace{}%
\AgdaOperator{\AgdaFunction{≈}}\AgdaSpace{}%
\AgdaBound{y}\<%
\\
%
\>[3]\AgdaFunction{Goal}\AgdaSpace{}%
\AgdaSymbol{=}\AgdaSpace{}%
\AgdaFunction{trans}\AgdaSpace{}%
\AgdaSymbol{(}\AgdaFunction{sym}\AgdaSpace{}%
\AgdaSymbol{(}\AgdaField{from∼to}\AgdaSpace{}%
\AgdaBound{x}\AgdaSymbol{))}\AgdaSpace{}%
\AgdaSymbol{(}\AgdaFunction{trans}\AgdaSpace{}%
\AgdaFunction{ξ}\AgdaSpace{}%
\AgdaSymbol{(}\AgdaField{from∼to}\AgdaSpace{}%
\AgdaBound{y}\AgdaSymbol{))}\<%
\\
%
\\[\AgdaEmptyExtraSkip]%
%
\>[2]\AgdaFunction{fromIsSurjective}\AgdaSpace{}%
\AgdaSymbol{:}\AgdaSpace{}%
\AgdaFunction{IsSurjective}\AgdaSpace{}%
\AgdaOperator{\AgdaFunction{∣}}\AgdaSpace{}%
\AgdaField{from}\AgdaSpace{}%
\AgdaOperator{\AgdaFunction{∣}}\<%
\\
%
\>[2]\AgdaFunction{fromIsSurjective}\AgdaSpace{}%
\AgdaSymbol{\{}\AgdaBound{y}\AgdaSymbol{\}}\AgdaSpace{}%
\AgdaSymbol{=}\AgdaSpace{}%
\AgdaInductiveConstructor{eq}\AgdaSpace{}%
\AgdaSymbol{(}\AgdaOperator{\AgdaFunction{∣}}\AgdaSpace{}%
\AgdaField{to}\AgdaSpace{}%
\AgdaOperator{\AgdaFunction{∣}}\AgdaSpace{}%
\AgdaOperator{\AgdaField{⟨\$⟩}}\AgdaSpace{}%
\AgdaBound{y}\AgdaSymbol{)}\AgdaSpace{}%
\AgdaSymbol{(}\AgdaFunction{sym}\AgdaSpace{}%
\AgdaSymbol{(}\AgdaField{from∼to}\AgdaSpace{}%
\AgdaBound{y}\AgdaSymbol{))}\<%
\\
%
\\[\AgdaEmptyExtraSkip]%
%
\>[2]\AgdaFunction{fromIsInjective}\AgdaSpace{}%
\AgdaSymbol{:}\AgdaSpace{}%
\AgdaFunction{IsInjective}\AgdaSpace{}%
\AgdaOperator{\AgdaFunction{∣}}\AgdaSpace{}%
\AgdaField{from}\AgdaSpace{}%
\AgdaOperator{\AgdaFunction{∣}}\<%
\\
%
\>[2]\AgdaFunction{fromIsInjective}\AgdaSpace{}%
\AgdaSymbol{\{}\AgdaBound{x}\AgdaSymbol{\}}\AgdaSpace{}%
\AgdaSymbol{\{}\AgdaBound{y}\AgdaSymbol{\}}\AgdaSpace{}%
\AgdaBound{xy}\AgdaSpace{}%
\AgdaSymbol{=}\AgdaSpace{}%
\AgdaFunction{Goal}\<%
\\
\>[2][@{}l@{\AgdaIndent{0}}]%
\>[3]\AgdaKeyword{where}\<%
\\
%
\>[3]\AgdaFunction{ξ}\AgdaSpace{}%
\AgdaSymbol{:}\AgdaSpace{}%
\AgdaOperator{\AgdaFunction{∣}}\AgdaSpace{}%
\AgdaField{to}\AgdaSpace{}%
\AgdaOperator{\AgdaFunction{∣}}\AgdaSpace{}%
\AgdaOperator{\AgdaField{⟨\$⟩}}\AgdaSpace{}%
\AgdaSymbol{(}\AgdaOperator{\AgdaFunction{∣}}\AgdaSpace{}%
\AgdaField{from}\AgdaSpace{}%
\AgdaOperator{\AgdaFunction{∣}}\AgdaSpace{}%
\AgdaOperator{\AgdaField{⟨\$⟩}}\AgdaSpace{}%
\AgdaBound{x}\AgdaSymbol{)}\AgdaSpace{}%
\AgdaOperator{\AgdaFunction{≈ᵇ}}\AgdaSpace{}%
\AgdaOperator{\AgdaFunction{∣}}\AgdaSpace{}%
\AgdaField{to}\AgdaSpace{}%
\AgdaOperator{\AgdaFunction{∣}}\AgdaSpace{}%
\AgdaOperator{\AgdaField{⟨\$⟩}}\AgdaSpace{}%
\AgdaSymbol{(}\AgdaOperator{\AgdaFunction{∣}}\AgdaSpace{}%
\AgdaField{from}\AgdaSpace{}%
\AgdaOperator{\AgdaFunction{∣}}\AgdaSpace{}%
\AgdaOperator{\AgdaField{⟨\$⟩}}\AgdaSpace{}%
\AgdaBound{y}\AgdaSymbol{)}\<%
\\
%
\>[3]\AgdaFunction{ξ}\AgdaSpace{}%
\AgdaSymbol{=}\AgdaSpace{}%
\AgdaField{cong}\AgdaSpace{}%
\AgdaOperator{\AgdaFunction{∣}}\AgdaSpace{}%
\AgdaField{to}\AgdaSpace{}%
\AgdaOperator{\AgdaFunction{∣}}\AgdaSpace{}%
\AgdaBound{xy}\<%
\\
%
\>[3]\AgdaFunction{Goal}\AgdaSpace{}%
\AgdaSymbol{:}\AgdaSpace{}%
\AgdaBound{x}\AgdaSpace{}%
\AgdaOperator{\AgdaFunction{≈ᵇ}}\AgdaSpace{}%
\AgdaBound{y}\<%
\\
%
\>[3]\AgdaFunction{Goal}\AgdaSpace{}%
\AgdaSymbol{=}\AgdaSpace{}%
\AgdaFunction{transᵇ}\AgdaSpace{}%
\AgdaSymbol{(}\AgdaFunction{symᵇ}\AgdaSpace{}%
\AgdaSymbol{(}\AgdaField{to∼from}\AgdaSpace{}%
\AgdaBound{x}\AgdaSymbol{))}\AgdaSpace{}%
\AgdaSymbol{(}\AgdaFunction{transᵇ}\AgdaSpace{}%
\AgdaFunction{ξ}\AgdaSpace{}%
\AgdaSymbol{(}\AgdaField{to∼from}\AgdaSpace{}%
\AgdaBound{y}\AgdaSymbol{))}\<%
\\
%
\\[\AgdaEmptyExtraSkip]%
\>[0]\AgdaKeyword{open}\AgdaSpace{}%
\AgdaOperator{\AgdaModule{\AgdaUnderscore{}≅\AgdaUnderscore{}}}\<%
\end{code}

\hypertarget{properties-of-isomorphisms}{%
\paragraph{Properties of
isomorphisms}\label{properties-of-isomorphisms}}

\begin{code}%
\>[0]\<%
\\
\>[0]\AgdaFunction{≅-refl}\AgdaSpace{}%
\AgdaSymbol{:}\AgdaSpace{}%
\AgdaFunction{Reflexive}\AgdaSpace{}%
\AgdaSymbol{(}\AgdaOperator{\AgdaRecord{\AgdaUnderscore{}≅\AgdaUnderscore{}}}\AgdaSpace{}%
\AgdaSymbol{\{}\AgdaGeneralizable{α}\AgdaSymbol{\}\{}\AgdaGeneralizable{ρᵃ}\AgdaSymbol{\})}\<%
\\
\>[0]\AgdaFunction{≅-refl}\AgdaSpace{}%
\AgdaSymbol{\{}\AgdaBound{α}\AgdaSymbol{\}\{}\AgdaBound{ρᵃ}\AgdaSymbol{\}\{}\AgdaBound{𝑨}\AgdaSymbol{\}}\AgdaSpace{}%
\AgdaSymbol{=}\AgdaSpace{}%
\AgdaInductiveConstructor{mkiso}\AgdaSpace{}%
\AgdaFunction{𝒾𝒹}\AgdaSpace{}%
\AgdaFunction{𝒾𝒹}\AgdaSpace{}%
\AgdaSymbol{(λ}\AgdaSpace{}%
\AgdaBound{b}\AgdaSpace{}%
\AgdaSymbol{→}\AgdaSpace{}%
\AgdaFunction{refl}\AgdaSymbol{)}\AgdaSpace{}%
\AgdaSymbol{λ}\AgdaSpace{}%
\AgdaBound{a}\AgdaSpace{}%
\AgdaSymbol{→}\AgdaSpace{}%
\AgdaFunction{refl}\<%
\\
\>[0][@{}l@{\AgdaIndent{0}}]%
\>[1]\AgdaKeyword{where}\AgdaSpace{}%
\AgdaKeyword{open}\AgdaSpace{}%
\AgdaModule{Setoid}\AgdaSpace{}%
\AgdaOperator{\AgdaFunction{𝔻[}}\AgdaSpace{}%
\AgdaBound{𝑨}\AgdaSpace{}%
\AgdaOperator{\AgdaFunction{]}}\AgdaSpace{}%
\AgdaKeyword{using}\AgdaSpace{}%
\AgdaSymbol{(}\AgdaSpace{}%
\AgdaFunction{refl}\AgdaSpace{}%
\AgdaSymbol{)}\<%
\\
%
\\[\AgdaEmptyExtraSkip]%
\>[0]\AgdaFunction{≅-sym}\AgdaSpace{}%
\AgdaSymbol{:}\AgdaSpace{}%
\AgdaFunction{Sym}\AgdaSpace{}%
\AgdaSymbol{(}\AgdaOperator{\AgdaRecord{\AgdaUnderscore{}≅\AgdaUnderscore{}}}\AgdaSymbol{\{}\AgdaGeneralizable{β}\AgdaSymbol{\}\{}\AgdaGeneralizable{ρᵇ}\AgdaSymbol{\})}\AgdaSpace{}%
\AgdaSymbol{(}\AgdaOperator{\AgdaRecord{\AgdaUnderscore{}≅\AgdaUnderscore{}}}\AgdaSymbol{\{}\AgdaGeneralizable{α}\AgdaSymbol{\}\{}\AgdaGeneralizable{ρᵃ}\AgdaSymbol{\})}\<%
\\
\>[0]\AgdaFunction{≅-sym}\AgdaSpace{}%
\AgdaBound{φ}\AgdaSpace{}%
\AgdaSymbol{=}\AgdaSpace{}%
\AgdaInductiveConstructor{mkiso}\AgdaSpace{}%
\AgdaSymbol{(}\AgdaField{from}\AgdaSpace{}%
\AgdaBound{φ}\AgdaSymbol{)}\AgdaSpace{}%
\AgdaSymbol{(}\AgdaField{to}\AgdaSpace{}%
\AgdaBound{φ}\AgdaSymbol{)}\AgdaSpace{}%
\AgdaSymbol{(}\AgdaField{from∼to}\AgdaSpace{}%
\AgdaBound{φ}\AgdaSymbol{)}\AgdaSpace{}%
\AgdaSymbol{(}\AgdaField{to∼from}\AgdaSpace{}%
\AgdaBound{φ}\AgdaSymbol{)}\<%
\\
%
\\[\AgdaEmptyExtraSkip]%
\>[0]\AgdaFunction{≅-trans}\AgdaSpace{}%
\AgdaSymbol{:}\AgdaSpace{}%
\AgdaFunction{Trans}\AgdaSpace{}%
\AgdaSymbol{(}\AgdaOperator{\AgdaRecord{\AgdaUnderscore{}≅\AgdaUnderscore{}}}\AgdaSpace{}%
\AgdaSymbol{\{}\AgdaGeneralizable{α}\AgdaSymbol{\}\{}\AgdaGeneralizable{ρᵃ}\AgdaSymbol{\})(}\AgdaOperator{\AgdaRecord{\AgdaUnderscore{}≅\AgdaUnderscore{}}}\AgdaSymbol{\{}\AgdaGeneralizable{β}\AgdaSymbol{\}\{}\AgdaGeneralizable{ρᵇ}\AgdaSymbol{\})(}\AgdaOperator{\AgdaRecord{\AgdaUnderscore{}≅\AgdaUnderscore{}}}\AgdaSymbol{\{}\AgdaGeneralizable{α}\AgdaSymbol{\}\{}\AgdaGeneralizable{ρᵃ}\AgdaSymbol{\}\{}\AgdaGeneralizable{γ}\AgdaSymbol{\}\{}\AgdaGeneralizable{ρᶜ}\AgdaSymbol{\})}\<%
\\
\>[0]\AgdaFunction{≅-trans}\AgdaSpace{}%
\AgdaSymbol{\{}\AgdaArgument{ρᶜ}\AgdaSpace{}%
\AgdaSymbol{=}\AgdaSpace{}%
\AgdaBound{ρᶜ}\AgdaSymbol{\}\{}\AgdaBound{𝑨}\AgdaSymbol{\}\{}\AgdaBound{𝑩}\AgdaSymbol{\}\{}\AgdaBound{𝑪}\AgdaSymbol{\}}\AgdaSpace{}%
\AgdaBound{ab}\AgdaSpace{}%
\AgdaBound{bc}\AgdaSpace{}%
\AgdaSymbol{=}\AgdaSpace{}%
\AgdaInductiveConstructor{mkiso}\AgdaSpace{}%
\AgdaFunction{f}\AgdaSpace{}%
\AgdaFunction{g}\AgdaSpace{}%
\AgdaFunction{τ}\AgdaSpace{}%
\AgdaFunction{ν}\<%
\\
\>[0][@{}l@{\AgdaIndent{0}}]%
\>[1]\AgdaKeyword{where}\<%
\\
\>[1][@{}l@{\AgdaIndent{0}}]%
\>[2]\AgdaKeyword{open}\AgdaSpace{}%
\AgdaModule{Setoid}\AgdaSpace{}%
\AgdaOperator{\AgdaFunction{𝔻[}}\AgdaSpace{}%
\AgdaBound{𝑨}\AgdaSpace{}%
\AgdaOperator{\AgdaFunction{]}}\AgdaSpace{}%
\AgdaKeyword{using}\AgdaSpace{}%
\AgdaSymbol{(}\AgdaSpace{}%
\AgdaOperator{\AgdaField{\AgdaUnderscore{}≈\AgdaUnderscore{}}}\AgdaSpace{}%
\AgdaSymbol{;}\AgdaSpace{}%
\AgdaFunction{trans}\AgdaSpace{}%
\AgdaSymbol{)}\<%
\\
%
\>[2]\AgdaKeyword{open}\AgdaSpace{}%
\AgdaModule{Setoid}\AgdaSpace{}%
\AgdaOperator{\AgdaFunction{𝔻[}}\AgdaSpace{}%
\AgdaBound{𝑪}\AgdaSpace{}%
\AgdaOperator{\AgdaFunction{]}}\AgdaSpace{}%
\AgdaKeyword{using}\AgdaSpace{}%
\AgdaSymbol{()}\AgdaSpace{}%
\AgdaKeyword{renaming}\AgdaSpace{}%
\AgdaSymbol{(}\AgdaSpace{}%
\AgdaOperator{\AgdaField{\AgdaUnderscore{}≈\AgdaUnderscore{}}}\AgdaSpace{}%
\AgdaSymbol{to}\AgdaSpace{}%
\AgdaOperator{\AgdaField{\AgdaUnderscore{}≈ᶜ\AgdaUnderscore{}}}\AgdaSpace{}%
\AgdaSymbol{;}\AgdaSpace{}%
\AgdaFunction{trans}\AgdaSpace{}%
\AgdaSymbol{to}\AgdaSpace{}%
\AgdaFunction{transᶜ}\AgdaSpace{}%
\AgdaSymbol{)}\<%
\\
%
\>[2]\AgdaFunction{f}\AgdaSpace{}%
\AgdaSymbol{:}\AgdaSpace{}%
\AgdaFunction{hom}\AgdaSpace{}%
\AgdaBound{𝑨}\AgdaSpace{}%
\AgdaBound{𝑪}\<%
\\
%
\>[2]\AgdaFunction{f}\AgdaSpace{}%
\AgdaSymbol{=}\AgdaSpace{}%
\AgdaFunction{∘-hom}\AgdaSpace{}%
\AgdaSymbol{(}\AgdaField{to}\AgdaSpace{}%
\AgdaBound{ab}\AgdaSymbol{)}\AgdaSpace{}%
\AgdaSymbol{(}\AgdaField{to}\AgdaSpace{}%
\AgdaBound{bc}\AgdaSymbol{)}\<%
\\
%
\>[2]\AgdaFunction{g}\AgdaSpace{}%
\AgdaSymbol{:}\AgdaSpace{}%
\AgdaFunction{hom}\AgdaSpace{}%
\AgdaBound{𝑪}\AgdaSpace{}%
\AgdaBound{𝑨}\<%
\\
%
\>[2]\AgdaFunction{g}\AgdaSpace{}%
\AgdaSymbol{=}\AgdaSpace{}%
\AgdaFunction{∘-hom}\AgdaSpace{}%
\AgdaSymbol{(}\AgdaField{from}\AgdaSpace{}%
\AgdaBound{bc}\AgdaSymbol{)}\AgdaSpace{}%
\AgdaSymbol{(}\AgdaField{from}\AgdaSpace{}%
\AgdaBound{ab}\AgdaSymbol{)}\<%
\\
%
\>[2]\AgdaFunction{τ}\AgdaSpace{}%
\AgdaSymbol{:}\AgdaSpace{}%
\AgdaSymbol{∀}\AgdaSpace{}%
\AgdaBound{b}\AgdaSpace{}%
\AgdaSymbol{→}\AgdaSpace{}%
\AgdaSymbol{(}\AgdaOperator{\AgdaFunction{∣}}\AgdaSpace{}%
\AgdaFunction{f}\AgdaSpace{}%
\AgdaOperator{\AgdaFunction{∣}}%
\>[20]\AgdaOperator{\AgdaField{⟨\$⟩}}\AgdaSpace{}%
\AgdaSymbol{(}\AgdaOperator{\AgdaFunction{∣}}\AgdaSpace{}%
\AgdaFunction{g}\AgdaSpace{}%
\AgdaOperator{\AgdaFunction{∣}}\AgdaSpace{}%
\AgdaOperator{\AgdaField{⟨\$⟩}}\AgdaSpace{}%
\AgdaBound{b}\AgdaSymbol{))}\AgdaSpace{}%
\AgdaOperator{\AgdaFunction{≈ᶜ}}\AgdaSpace{}%
\AgdaBound{b}\<%
\\
%
\>[2]\AgdaFunction{τ}\AgdaSpace{}%
\AgdaBound{b}\AgdaSpace{}%
\AgdaSymbol{=}\AgdaSpace{}%
\AgdaFunction{transᶜ}\AgdaSpace{}%
\AgdaSymbol{(}\AgdaField{cong}\AgdaSpace{}%
\AgdaOperator{\AgdaFunction{∣}}\AgdaSpace{}%
\AgdaField{to}\AgdaSpace{}%
\AgdaBound{bc}\AgdaSpace{}%
\AgdaOperator{\AgdaFunction{∣}}\AgdaSpace{}%
\AgdaSymbol{(}\AgdaField{to∼from}\AgdaSpace{}%
\AgdaBound{ab}\AgdaSpace{}%
\AgdaSymbol{(}\AgdaOperator{\AgdaFunction{∣}}\AgdaSpace{}%
\AgdaField{from}\AgdaSpace{}%
\AgdaBound{bc}\AgdaSpace{}%
\AgdaOperator{\AgdaFunction{∣}}\AgdaSpace{}%
\AgdaOperator{\AgdaField{⟨\$⟩}}\AgdaSpace{}%
\AgdaBound{b}\AgdaSymbol{)))}\AgdaSpace{}%
\AgdaSymbol{(}\AgdaField{to∼from}\AgdaSpace{}%
\AgdaBound{bc}\AgdaSpace{}%
\AgdaBound{b}\AgdaSymbol{)}\<%
\\
%
\>[2]\AgdaFunction{ν}\AgdaSpace{}%
\AgdaSymbol{:}\AgdaSpace{}%
\AgdaSymbol{∀}\AgdaSpace{}%
\AgdaBound{a}\AgdaSpace{}%
\AgdaSymbol{→}\AgdaSpace{}%
\AgdaSymbol{(}\AgdaOperator{\AgdaFunction{∣}}\AgdaSpace{}%
\AgdaFunction{g}\AgdaSpace{}%
\AgdaOperator{\AgdaFunction{∣}}\AgdaSpace{}%
\AgdaOperator{\AgdaField{⟨\$⟩}}\AgdaSpace{}%
\AgdaSymbol{(}\AgdaOperator{\AgdaFunction{∣}}\AgdaSpace{}%
\AgdaFunction{f}\AgdaSpace{}%
\AgdaOperator{\AgdaFunction{∣}}\AgdaSpace{}%
\AgdaOperator{\AgdaField{⟨\$⟩}}\AgdaSpace{}%
\AgdaBound{a}\AgdaSymbol{))}\AgdaSpace{}%
\AgdaOperator{\AgdaFunction{≈}}\AgdaSpace{}%
\AgdaBound{a}\<%
\\
%
\>[2]\AgdaFunction{ν}\AgdaSpace{}%
\AgdaBound{a}\AgdaSpace{}%
\AgdaSymbol{=}\AgdaSpace{}%
\AgdaFunction{trans}\AgdaSpace{}%
\AgdaSymbol{(}\AgdaField{cong}\AgdaSpace{}%
\AgdaOperator{\AgdaFunction{∣}}\AgdaSpace{}%
\AgdaField{from}\AgdaSpace{}%
\AgdaBound{ab}\AgdaSpace{}%
\AgdaOperator{\AgdaFunction{∣}}\AgdaSpace{}%
\AgdaSymbol{(}\AgdaField{from∼to}\AgdaSpace{}%
\AgdaBound{bc}\AgdaSpace{}%
\AgdaSymbol{(}\AgdaOperator{\AgdaFunction{∣}}\AgdaSpace{}%
\AgdaField{to}\AgdaSpace{}%
\AgdaBound{ab}\AgdaSpace{}%
\AgdaOperator{\AgdaFunction{∣}}\AgdaSpace{}%
\AgdaOperator{\AgdaField{⟨\$⟩}}\AgdaSpace{}%
\AgdaBound{a}\AgdaSymbol{)))}\AgdaSpace{}%
\AgdaSymbol{(}\AgdaField{from∼to}\AgdaSpace{}%
\AgdaBound{ab}\AgdaSpace{}%
\AgdaBound{a}\AgdaSymbol{)}\<%
\\
\>[0]\<%
\end{code}

Fortunately, the lift operation preserves isomorphism (i.e., it's an
\emph{algebraic invariant}). As our focus is universal algebra, this is
important and is what makes the lift operation a workable solution to
the technical problems that arise from the noncumulativity of Agda's
universe hierarchy.

\begin{code}[hide]%
\>[0]\AgdaKeyword{module}\AgdaSpace{}%
\AgdaModule{\AgdaUnderscore{}}\AgdaSpace{}%
\AgdaSymbol{\{}\AgdaBound{𝑨}\AgdaSpace{}%
\AgdaSymbol{:}\AgdaSpace{}%
\AgdaRecord{Algebra}\AgdaSpace{}%
\AgdaGeneralizable{α}\AgdaSpace{}%
\AgdaGeneralizable{ρᵃ}\AgdaSymbol{\}\{}\AgdaBound{ℓ}\AgdaSpace{}%
\AgdaSymbol{:}\AgdaSpace{}%
\AgdaPostulate{Level}\AgdaSymbol{\}}\AgdaSpace{}%
\AgdaKeyword{where}\<%
\\
\>[0][@{}l@{\AgdaIndent{0}}]%
\>[1]\AgdaFunction{Lift-≅ˡ}\AgdaSpace{}%
\AgdaSymbol{:}\AgdaSpace{}%
\AgdaBound{𝑨}\AgdaSpace{}%
\AgdaOperator{\AgdaRecord{≅}}\AgdaSpace{}%
\AgdaSymbol{(}\AgdaFunction{Lift-Algˡ}\AgdaSpace{}%
\AgdaBound{𝑨}\AgdaSpace{}%
\AgdaBound{ℓ}\AgdaSymbol{)}\<%
\\
%
\>[1]\AgdaFunction{Lift-≅ˡ}\AgdaSpace{}%
\AgdaSymbol{=}\AgdaSpace{}%
\AgdaInductiveConstructor{mkiso}\AgdaSpace{}%
\AgdaFunction{ToLiftˡ}\AgdaSpace{}%
\AgdaFunction{FromLiftˡ}\AgdaSpace{}%
\AgdaSymbol{(}\AgdaFunction{ToFromLiftˡ}\AgdaSymbol{\{}\AgdaArgument{𝑨}\AgdaSpace{}%
\AgdaSymbol{=}\AgdaSpace{}%
\AgdaBound{𝑨}\AgdaSymbol{\})}\AgdaSpace{}%
\AgdaSymbol{(}\AgdaFunction{FromToLiftˡ}\AgdaSymbol{\{}\AgdaArgument{𝑨}\AgdaSpace{}%
\AgdaSymbol{=}\AgdaSpace{}%
\AgdaBound{𝑨}\AgdaSymbol{\}\{}\AgdaBound{ℓ}\AgdaSymbol{\})}\<%
\\
%
\\[\AgdaEmptyExtraSkip]%
%
\>[1]\AgdaFunction{Lift-≅ʳ}\AgdaSpace{}%
\AgdaSymbol{:}\AgdaSpace{}%
\AgdaBound{𝑨}\AgdaSpace{}%
\AgdaOperator{\AgdaRecord{≅}}\AgdaSpace{}%
\AgdaSymbol{(}\AgdaFunction{Lift-Algʳ}\AgdaSpace{}%
\AgdaBound{𝑨}\AgdaSpace{}%
\AgdaBound{ℓ}\AgdaSymbol{)}\<%
\\
%
\>[1]\AgdaFunction{Lift-≅ʳ}\AgdaSpace{}%
\AgdaSymbol{=}\AgdaSpace{}%
\AgdaInductiveConstructor{mkiso}\AgdaSpace{}%
\AgdaFunction{ToLiftʳ}\AgdaSpace{}%
\AgdaFunction{FromLiftʳ}\AgdaSpace{}%
\AgdaSymbol{(}\AgdaFunction{ToFromLiftʳ}\AgdaSymbol{\{}\AgdaArgument{𝑨}\AgdaSpace{}%
\AgdaSymbol{=}\AgdaSpace{}%
\AgdaBound{𝑨}\AgdaSymbol{\})}\AgdaSpace{}%
\AgdaSymbol{(}\AgdaFunction{FromToLiftʳ}\AgdaSymbol{\{}\AgdaArgument{𝑨}\AgdaSpace{}%
\AgdaSymbol{=}\AgdaSpace{}%
\AgdaBound{𝑨}\AgdaSymbol{\}\{}\AgdaBound{ℓ}\AgdaSymbol{\})}\<%
\\
%
\\[\AgdaEmptyExtraSkip]%
\>[0]\AgdaFunction{Lift-≅}\AgdaSpace{}%
\AgdaSymbol{:}\AgdaSpace{}%
\AgdaSymbol{\{}\AgdaBound{𝑨}\AgdaSpace{}%
\AgdaSymbol{:}\AgdaSpace{}%
\AgdaRecord{Algebra}\AgdaSpace{}%
\AgdaGeneralizable{α}\AgdaSpace{}%
\AgdaGeneralizable{ρᵃ}\AgdaSymbol{\}\{}\AgdaBound{ℓ}\AgdaSpace{}%
\AgdaBound{ρ}\AgdaSpace{}%
\AgdaSymbol{:}\AgdaSpace{}%
\AgdaPostulate{Level}\AgdaSymbol{\}}\AgdaSpace{}%
\AgdaSymbol{→}\AgdaSpace{}%
\AgdaBound{𝑨}\AgdaSpace{}%
\AgdaOperator{\AgdaRecord{≅}}\AgdaSpace{}%
\AgdaSymbol{(}\AgdaFunction{Lift-Alg}\AgdaSpace{}%
\AgdaBound{𝑨}\AgdaSpace{}%
\AgdaBound{ℓ}\AgdaSpace{}%
\AgdaBound{ρ}\AgdaSymbol{)}\<%
\\
\>[0]\AgdaFunction{Lift-≅}\AgdaSpace{}%
\AgdaSymbol{=}\AgdaSpace{}%
\AgdaFunction{≅-trans}\AgdaSpace{}%
\AgdaFunction{Lift-≅ˡ}\AgdaSpace{}%
\AgdaFunction{Lift-≅ʳ}\<%
\end{code}

\hypertarget{homomorphic-images}{%
\subsubsection{Homomorphic Images}\label{homomorphic-images}}

We begin with what for our purposes is the most useful way to represent
the class of \emph{homomorphic images} of an algebra in dependent type
theory (cf.~the
\href{https://ualib.github.io/agda-algebras/Homomorphisms.Func.HomomorphicImages.html}{Homomorphisms.Func.HomomorphicImages}
module of the \href{https://ualib.github.io/agda-algebras}{Agda
Universal Algebra Library}). (The first definition is merely a
short-hand.)

\begin{code}%
\>[0]\AgdaFunction{ov}\AgdaSpace{}%
\AgdaSymbol{:}\AgdaSpace{}%
\AgdaPostulate{Level}\AgdaSpace{}%
\AgdaSymbol{→}\AgdaSpace{}%
\AgdaPostulate{Level}\<%
\\
\>[0]\AgdaFunction{ov}\AgdaSpace{}%
\AgdaBound{α}\AgdaSpace{}%
\AgdaSymbol{=}\AgdaSpace{}%
\AgdaBound{𝓞}\AgdaSpace{}%
\AgdaOperator{\AgdaPrimitive{⊔}}\AgdaSpace{}%
\AgdaBound{𝓥}\AgdaSpace{}%
\AgdaOperator{\AgdaPrimitive{⊔}}\AgdaSpace{}%
\AgdaPrimitive{lsuc}\AgdaSpace{}%
\AgdaBound{α}\<%
\\
%
\\[\AgdaEmptyExtraSkip]%
\>[0]\AgdaOperator{\AgdaFunction{\AgdaUnderscore{}IsHomImageOf\AgdaUnderscore{}}}\AgdaSpace{}%
\AgdaSymbol{:}\AgdaSpace{}%
\AgdaSymbol{(}\AgdaBound{𝑩}\AgdaSpace{}%
\AgdaSymbol{:}\AgdaSpace{}%
\AgdaRecord{Algebra}\AgdaSpace{}%
\AgdaGeneralizable{β}\AgdaSpace{}%
\AgdaGeneralizable{ρᵇ}\AgdaSymbol{)(}\AgdaBound{𝑨}\AgdaSpace{}%
\AgdaSymbol{:}\AgdaSpace{}%
\AgdaRecord{Algebra}\AgdaSpace{}%
\AgdaGeneralizable{α}\AgdaSpace{}%
\AgdaGeneralizable{ρᵃ}\AgdaSymbol{)}\AgdaSpace{}%
\AgdaSymbol{→}\AgdaSpace{}%
\AgdaPrimitive{Type}\AgdaSpace{}%
\AgdaSymbol{(}\AgdaBound{𝓞}\AgdaSpace{}%
\AgdaOperator{\AgdaPrimitive{⊔}}\AgdaSpace{}%
\AgdaBound{𝓥}\AgdaSpace{}%
\AgdaOperator{\AgdaPrimitive{⊔}}\AgdaSpace{}%
\AgdaGeneralizable{α}\AgdaSpace{}%
\AgdaOperator{\AgdaPrimitive{⊔}}\AgdaSpace{}%
\AgdaGeneralizable{β}\AgdaSpace{}%
\AgdaOperator{\AgdaPrimitive{⊔}}\AgdaSpace{}%
\AgdaGeneralizable{ρᵃ}\AgdaSpace{}%
\AgdaOperator{\AgdaPrimitive{⊔}}\AgdaSpace{}%
\AgdaGeneralizable{ρᵇ}\AgdaSymbol{)}\<%
\\
\>[0]\AgdaBound{𝑩}\AgdaSpace{}%
\AgdaOperator{\AgdaFunction{IsHomImageOf}}\AgdaSpace{}%
\AgdaBound{𝑨}\AgdaSpace{}%
\AgdaSymbol{=}\AgdaSpace{}%
\AgdaFunction{Σ[}\AgdaSpace{}%
\AgdaBound{φ}\AgdaSpace{}%
\AgdaFunction{∈}\AgdaSpace{}%
\AgdaFunction{hom}\AgdaSpace{}%
\AgdaBound{𝑨}\AgdaSpace{}%
\AgdaBound{𝑩}\AgdaSpace{}%
\AgdaFunction{]}\AgdaSpace{}%
\AgdaFunction{IsSurjective}\AgdaSpace{}%
\AgdaOperator{\AgdaFunction{∣}}\AgdaSpace{}%
\AgdaBound{φ}\AgdaSpace{}%
\AgdaOperator{\AgdaFunction{∣}}\<%
\\
%
\\[\AgdaEmptyExtraSkip]%
\>[0]\AgdaFunction{HomImages}\AgdaSpace{}%
\AgdaSymbol{:}\AgdaSpace{}%
\AgdaRecord{Algebra}\AgdaSpace{}%
\AgdaGeneralizable{α}\AgdaSpace{}%
\AgdaGeneralizable{ρᵃ}\AgdaSpace{}%
\AgdaSymbol{→}\AgdaSpace{}%
\AgdaPrimitive{Type}\AgdaSpace{}%
\AgdaSymbol{(}\AgdaGeneralizable{α}\AgdaSpace{}%
\AgdaOperator{\AgdaPrimitive{⊔}}\AgdaSpace{}%
\AgdaGeneralizable{ρᵃ}\AgdaSpace{}%
\AgdaOperator{\AgdaPrimitive{⊔}}\AgdaSpace{}%
\AgdaFunction{ov}\AgdaSpace{}%
\AgdaSymbol{(}\AgdaGeneralizable{β}\AgdaSpace{}%
\AgdaOperator{\AgdaPrimitive{⊔}}\AgdaSpace{}%
\AgdaGeneralizable{ρᵇ}\AgdaSymbol{))}\<%
\\
\>[0]\AgdaFunction{HomImages}\AgdaSpace{}%
\AgdaSymbol{\{}\AgdaArgument{β}\AgdaSpace{}%
\AgdaSymbol{=}\AgdaSpace{}%
\AgdaBound{β}\AgdaSymbol{\}\{}\AgdaArgument{ρᵇ}\AgdaSpace{}%
\AgdaSymbol{=}\AgdaSpace{}%
\AgdaBound{ρᵇ}\AgdaSymbol{\}}\AgdaSpace{}%
\AgdaBound{𝑨}\AgdaSpace{}%
\AgdaSymbol{=}\AgdaSpace{}%
\AgdaFunction{Σ[}\AgdaSpace{}%
\AgdaBound{𝑩}\AgdaSpace{}%
\AgdaFunction{∈}\AgdaSpace{}%
\AgdaRecord{Algebra}\AgdaSpace{}%
\AgdaBound{β}\AgdaSpace{}%
\AgdaBound{ρᵇ}\AgdaSpace{}%
\AgdaFunction{]}\AgdaSpace{}%
\AgdaBound{𝑩}\AgdaSpace{}%
\AgdaOperator{\AgdaFunction{IsHomImageOf}}\AgdaSpace{}%
\AgdaBound{𝑨}\<%
\\
\>[0]\<%
\end{code}

These types should be self-explanatory, but just to be sure, let's
describe the Sigma type appearing in the second definition. Given an
\texttt{𝑆}-algebra \texttt{𝑨\ :\ Algebra\ α\ ρ}, the type
\texttt{HomImages\ 𝑨} denotes the class of algebras
\texttt{𝑩\ :\ Algebra\ β\ ρ} with a map
\texttt{φ\ :\ ∣\ 𝑨\ ∣\ →\ ∣\ 𝑩\ ∣} such that \texttt{φ} is a surjective
homomorphism.

\begin{code}[hide]%
\>[0]\AgdaKeyword{module}\AgdaSpace{}%
\AgdaModule{\AgdaUnderscore{}}\AgdaSpace{}%
\AgdaSymbol{\{}\AgdaBound{𝑨}\AgdaSpace{}%
\AgdaSymbol{:}\AgdaSpace{}%
\AgdaRecord{Algebra}\AgdaSpace{}%
\AgdaGeneralizable{α}\AgdaSpace{}%
\AgdaGeneralizable{ρᵃ}\AgdaSymbol{\}\{}\AgdaBound{𝑩}\AgdaSpace{}%
\AgdaSymbol{:}\AgdaSpace{}%
\AgdaRecord{Algebra}\AgdaSpace{}%
\AgdaGeneralizable{β}\AgdaSpace{}%
\AgdaGeneralizable{ρᵇ}\AgdaSymbol{\}}\AgdaSpace{}%
\AgdaKeyword{where}\<%
\\
%
\\[\AgdaEmptyExtraSkip]%
\>[0][@{}l@{\AgdaIndent{0}}]%
\>[1]\AgdaFunction{Lift-HomImage-lemma}\AgdaSpace{}%
\AgdaSymbol{:}\AgdaSpace{}%
\AgdaSymbol{∀\{}\AgdaBound{γ}\AgdaSymbol{\}}\AgdaSpace{}%
\AgdaSymbol{→}\AgdaSpace{}%
\AgdaSymbol{(}\AgdaFunction{Lift-Alg}\AgdaSpace{}%
\AgdaBound{𝑨}\AgdaSpace{}%
\AgdaBound{γ}\AgdaSpace{}%
\AgdaBound{γ}\AgdaSymbol{)}\AgdaSpace{}%
\AgdaOperator{\AgdaFunction{IsHomImageOf}}\AgdaSpace{}%
\AgdaBound{𝑩}\AgdaSpace{}%
\AgdaSymbol{→}\AgdaSpace{}%
\AgdaBound{𝑨}\AgdaSpace{}%
\AgdaOperator{\AgdaFunction{IsHomImageOf}}\AgdaSpace{}%
\AgdaBound{𝑩}\<%
\\
%
\>[1]\AgdaFunction{Lift-HomImage-lemma}\AgdaSpace{}%
\AgdaSymbol{\{}\AgdaBound{γ}\AgdaSymbol{\}}\AgdaSpace{}%
\AgdaBound{φ}\AgdaSpace{}%
\AgdaSymbol{=}%
\>[3018I]\AgdaFunction{∘-hom}\AgdaSpace{}%
\AgdaOperator{\AgdaFunction{∣}}\AgdaSpace{}%
\AgdaBound{φ}\AgdaSpace{}%
\AgdaOperator{\AgdaFunction{∣}}\AgdaSpace{}%
\AgdaSymbol{(}\AgdaField{from}\AgdaSpace{}%
\AgdaFunction{Lift-≅}\AgdaSymbol{)}\AgdaSpace{}%
\AgdaOperator{\AgdaInductiveConstructor{,}}\<%
\\
\>[.][@{}l@{}]\<[3018I]%
\>[29]\AgdaFunction{∘-IsSurjective}\AgdaSpace{}%
\AgdaSymbol{\AgdaUnderscore{}}\AgdaSpace{}%
\AgdaSymbol{\AgdaUnderscore{}}\AgdaSpace{}%
\AgdaOperator{\AgdaFunction{∥}}\AgdaSpace{}%
\AgdaBound{φ}\AgdaSpace{}%
\AgdaOperator{\AgdaFunction{∥}}\AgdaSpace{}%
\AgdaSymbol{(}\AgdaFunction{fromIsSurjective}\AgdaSpace{}%
\AgdaSymbol{(}\AgdaFunction{Lift-≅}\AgdaSymbol{\{}\AgdaArgument{𝑨}\AgdaSpace{}%
\AgdaSymbol{=}\AgdaSpace{}%
\AgdaBound{𝑨}\AgdaSymbol{\}))}\<%
\\
%
\\[\AgdaEmptyExtraSkip]%
\>[0]\AgdaKeyword{module}\AgdaSpace{}%
\AgdaModule{\AgdaUnderscore{}}\AgdaSpace{}%
\AgdaSymbol{\{}\AgdaBound{𝑨}\AgdaSpace{}%
\AgdaBound{𝑨'}\AgdaSpace{}%
\AgdaSymbol{:}\AgdaSpace{}%
\AgdaRecord{Algebra}\AgdaSpace{}%
\AgdaGeneralizable{α}\AgdaSpace{}%
\AgdaGeneralizable{ρᵃ}\AgdaSymbol{\}\{}\AgdaBound{𝑩}\AgdaSpace{}%
\AgdaSymbol{:}\AgdaSpace{}%
\AgdaRecord{Algebra}\AgdaSpace{}%
\AgdaGeneralizable{β}\AgdaSpace{}%
\AgdaGeneralizable{ρᵇ}\AgdaSymbol{\}}\AgdaSpace{}%
\AgdaKeyword{where}\<%
\\
%
\\[\AgdaEmptyExtraSkip]%
\>[0][@{}l@{\AgdaIndent{0}}]%
\>[1]\AgdaFunction{HomImage-≅}\AgdaSpace{}%
\AgdaSymbol{:}\AgdaSpace{}%
\AgdaBound{𝑨}\AgdaSpace{}%
\AgdaOperator{\AgdaFunction{IsHomImageOf}}\AgdaSpace{}%
\AgdaBound{𝑨'}\AgdaSpace{}%
\AgdaSymbol{→}\AgdaSpace{}%
\AgdaBound{𝑨}\AgdaSpace{}%
\AgdaOperator{\AgdaRecord{≅}}\AgdaSpace{}%
\AgdaBound{𝑩}\AgdaSpace{}%
\AgdaSymbol{→}\AgdaSpace{}%
\AgdaBound{𝑩}\AgdaSpace{}%
\AgdaOperator{\AgdaFunction{IsHomImageOf}}\AgdaSpace{}%
\AgdaBound{𝑨'}\<%
\\
%
\>[1]\AgdaFunction{HomImage-≅}\AgdaSpace{}%
\AgdaBound{φ}\AgdaSpace{}%
\AgdaBound{A≅B}\AgdaSpace{}%
\AgdaSymbol{=}\AgdaSpace{}%
\AgdaFunction{∘-hom}\AgdaSpace{}%
\AgdaOperator{\AgdaFunction{∣}}\AgdaSpace{}%
\AgdaBound{φ}\AgdaSpace{}%
\AgdaOperator{\AgdaFunction{∣}}\AgdaSpace{}%
\AgdaSymbol{(}\AgdaField{to}\AgdaSpace{}%
\AgdaBound{A≅B}\AgdaSymbol{)}\AgdaSpace{}%
\AgdaOperator{\AgdaInductiveConstructor{,}}\AgdaSpace{}%
\AgdaFunction{∘-IsSurjective}\AgdaSpace{}%
\AgdaSymbol{\AgdaUnderscore{}}\AgdaSpace{}%
\AgdaSymbol{\AgdaUnderscore{}}\AgdaSpace{}%
\AgdaOperator{\AgdaFunction{∥}}\AgdaSpace{}%
\AgdaBound{φ}\AgdaSpace{}%
\AgdaOperator{\AgdaFunction{∥}}\AgdaSpace{}%
\AgdaSymbol{(}\AgdaFunction{toIsSurjective}\AgdaSpace{}%
\AgdaBound{A≅B}\AgdaSymbol{)}\<%
\end{code}

\hypertarget{subalgebras}{%
\subsection{Subalgebras}\label{subalgebras}}

\hypertarget{basic-definitions-1}{%
\subsubsection{Basic definitions}\label{basic-definitions-1}}

\begin{code}%
\>[0]\<%
\\
\>[0]\AgdaOperator{\AgdaFunction{\AgdaUnderscore{}≤\AgdaUnderscore{}}}\AgdaSpace{}%
\AgdaSymbol{:}\AgdaSpace{}%
\AgdaRecord{Algebra}\AgdaSpace{}%
\AgdaGeneralizable{α}\AgdaSpace{}%
\AgdaGeneralizable{ρᵃ}\AgdaSpace{}%
\AgdaSymbol{→}\AgdaSpace{}%
\AgdaRecord{Algebra}\AgdaSpace{}%
\AgdaGeneralizable{β}\AgdaSpace{}%
\AgdaGeneralizable{ρᵇ}\AgdaSpace{}%
\AgdaSymbol{→}\AgdaSpace{}%
\AgdaPrimitive{Type}\AgdaSpace{}%
\AgdaSymbol{(}\AgdaBound{𝓞}\AgdaSpace{}%
\AgdaOperator{\AgdaPrimitive{⊔}}\AgdaSpace{}%
\AgdaBound{𝓥}\AgdaSpace{}%
\AgdaOperator{\AgdaPrimitive{⊔}}\AgdaSpace{}%
\AgdaGeneralizable{α}\AgdaSpace{}%
\AgdaOperator{\AgdaPrimitive{⊔}}\AgdaSpace{}%
\AgdaGeneralizable{ρᵃ}\AgdaSpace{}%
\AgdaOperator{\AgdaPrimitive{⊔}}\AgdaSpace{}%
\AgdaGeneralizable{β}\AgdaSpace{}%
\AgdaOperator{\AgdaPrimitive{⊔}}\AgdaSpace{}%
\AgdaGeneralizable{ρᵇ}\AgdaSymbol{)}\<%
\\
\>[0]\AgdaBound{𝑨}\AgdaSpace{}%
\AgdaOperator{\AgdaFunction{≤}}\AgdaSpace{}%
\AgdaBound{𝑩}\AgdaSpace{}%
\AgdaSymbol{=}\AgdaSpace{}%
\AgdaFunction{Σ[}\AgdaSpace{}%
\AgdaBound{h}\AgdaSpace{}%
\AgdaFunction{∈}\AgdaSpace{}%
\AgdaFunction{hom}\AgdaSpace{}%
\AgdaBound{𝑨}\AgdaSpace{}%
\AgdaBound{𝑩}\AgdaSpace{}%
\AgdaFunction{]}\AgdaSpace{}%
\AgdaFunction{IsInjective}\AgdaSpace{}%
\AgdaOperator{\AgdaFunction{∣}}\AgdaSpace{}%
\AgdaBound{h}\AgdaSpace{}%
\AgdaOperator{\AgdaFunction{∣}}\<%
\end{code}

\hypertarget{basic-properties}{%
\subsubsection{Basic properties}\label{basic-properties}}

\begin{code}%
\>[0]\<%
\\
\>[0]\AgdaFunction{≤-reflexive}\AgdaSpace{}%
\AgdaSymbol{:}\AgdaSpace{}%
\AgdaSymbol{\{}\AgdaBound{𝑨}\AgdaSpace{}%
\AgdaSymbol{:}\AgdaSpace{}%
\AgdaRecord{Algebra}\AgdaSpace{}%
\AgdaGeneralizable{α}\AgdaSpace{}%
\AgdaGeneralizable{ρᵃ}\AgdaSymbol{\}}\AgdaSpace{}%
\AgdaSymbol{→}\AgdaSpace{}%
\AgdaBound{𝑨}\AgdaSpace{}%
\AgdaOperator{\AgdaFunction{≤}}\AgdaSpace{}%
\AgdaBound{𝑨}\<%
\\
\>[0]\AgdaFunction{≤-reflexive}\AgdaSpace{}%
\AgdaSymbol{\{}\AgdaArgument{𝑨}\AgdaSpace{}%
\AgdaSymbol{=}\AgdaSpace{}%
\AgdaBound{𝑨}\AgdaSymbol{\}}\AgdaSpace{}%
\AgdaSymbol{=}\AgdaSpace{}%
\AgdaFunction{𝒾𝒹}\AgdaSpace{}%
\AgdaOperator{\AgdaInductiveConstructor{,}}\AgdaSpace{}%
\AgdaFunction{id}\<%
\\
%
\\[\AgdaEmptyExtraSkip]%
\>[0]\AgdaFunction{mon→≤}\AgdaSpace{}%
\AgdaSymbol{:}\AgdaSpace{}%
\AgdaSymbol{\{}\AgdaBound{𝑨}\AgdaSpace{}%
\AgdaSymbol{:}\AgdaSpace{}%
\AgdaRecord{Algebra}\AgdaSpace{}%
\AgdaGeneralizable{α}\AgdaSpace{}%
\AgdaGeneralizable{ρᵃ}\AgdaSymbol{\}\{}\AgdaBound{𝑩}\AgdaSpace{}%
\AgdaSymbol{:}\AgdaSpace{}%
\AgdaRecord{Algebra}\AgdaSpace{}%
\AgdaGeneralizable{β}\AgdaSpace{}%
\AgdaGeneralizable{ρᵇ}\AgdaSymbol{\}}\AgdaSpace{}%
\AgdaSymbol{→}\AgdaSpace{}%
\AgdaFunction{mon}\AgdaSpace{}%
\AgdaBound{𝑨}\AgdaSpace{}%
\AgdaBound{𝑩}\AgdaSpace{}%
\AgdaSymbol{→}\AgdaSpace{}%
\AgdaBound{𝑨}\AgdaSpace{}%
\AgdaOperator{\AgdaFunction{≤}}\AgdaSpace{}%
\AgdaBound{𝑩}\<%
\\
\>[0]\AgdaFunction{mon→≤}\AgdaSpace{}%
\AgdaSymbol{\{}\AgdaArgument{𝑨}\AgdaSpace{}%
\AgdaSymbol{=}\AgdaSpace{}%
\AgdaBound{𝑨}\AgdaSymbol{\}\{}\AgdaBound{𝑩}\AgdaSymbol{\}}\AgdaSpace{}%
\AgdaBound{x}\AgdaSpace{}%
\AgdaSymbol{=}\AgdaSpace{}%
\AgdaFunction{mon→intohom}\AgdaSpace{}%
\AgdaBound{𝑨}\AgdaSpace{}%
\AgdaBound{𝑩}\AgdaSpace{}%
\AgdaBound{x}\<%
\\
%
\\[\AgdaEmptyExtraSkip]%
\>[0]\AgdaKeyword{module}\AgdaSpace{}%
\AgdaModule{\AgdaUnderscore{}}\AgdaSpace{}%
\AgdaSymbol{\{}\AgdaBound{𝑨}\AgdaSpace{}%
\AgdaSymbol{:}\AgdaSpace{}%
\AgdaRecord{Algebra}\AgdaSpace{}%
\AgdaGeneralizable{α}\AgdaSpace{}%
\AgdaGeneralizable{ρᵃ}\AgdaSymbol{\}\{}\AgdaBound{𝑩}\AgdaSpace{}%
\AgdaSymbol{:}\AgdaSpace{}%
\AgdaRecord{Algebra}\AgdaSpace{}%
\AgdaGeneralizable{β}\AgdaSpace{}%
\AgdaGeneralizable{ρᵇ}\AgdaSymbol{\}\{}\AgdaBound{𝑪}\AgdaSpace{}%
\AgdaSymbol{:}\AgdaSpace{}%
\AgdaRecord{Algebra}\AgdaSpace{}%
\AgdaGeneralizable{γ}\AgdaSpace{}%
\AgdaGeneralizable{ρᶜ}\AgdaSymbol{\}}\AgdaSpace{}%
\AgdaKeyword{where}\<%
\\
\>[0][@{}l@{\AgdaIndent{0}}]%
\>[1]\AgdaFunction{≤-trans}\AgdaSpace{}%
\AgdaSymbol{:}\AgdaSpace{}%
\AgdaBound{𝑨}\AgdaSpace{}%
\AgdaOperator{\AgdaFunction{≤}}\AgdaSpace{}%
\AgdaBound{𝑩}\AgdaSpace{}%
\AgdaSymbol{→}\AgdaSpace{}%
\AgdaBound{𝑩}\AgdaSpace{}%
\AgdaOperator{\AgdaFunction{≤}}\AgdaSpace{}%
\AgdaBound{𝑪}\AgdaSpace{}%
\AgdaSymbol{→}\AgdaSpace{}%
\AgdaBound{𝑨}\AgdaSpace{}%
\AgdaOperator{\AgdaFunction{≤}}\AgdaSpace{}%
\AgdaBound{𝑪}\<%
\\
%
\>[1]\AgdaFunction{≤-trans}\AgdaSpace{}%
\AgdaSymbol{(}\AgdaSpace{}%
\AgdaBound{f}\AgdaSpace{}%
\AgdaOperator{\AgdaInductiveConstructor{,}}\AgdaSpace{}%
\AgdaBound{finj}\AgdaSpace{}%
\AgdaSymbol{)}\AgdaSpace{}%
\AgdaSymbol{(}\AgdaSpace{}%
\AgdaBound{g}\AgdaSpace{}%
\AgdaOperator{\AgdaInductiveConstructor{,}}\AgdaSpace{}%
\AgdaBound{ginj}\AgdaSpace{}%
\AgdaSymbol{)}\AgdaSpace{}%
\AgdaSymbol{=}\AgdaSpace{}%
\AgdaSymbol{(}\AgdaFunction{∘-hom}\AgdaSpace{}%
\AgdaBound{f}\AgdaSpace{}%
\AgdaBound{g}\AgdaSpace{}%
\AgdaSymbol{)}\AgdaSpace{}%
\AgdaOperator{\AgdaInductiveConstructor{,}}\AgdaSpace{}%
\AgdaFunction{∘-IsInjective}\AgdaSpace{}%
\AgdaOperator{\AgdaFunction{∣}}\AgdaSpace{}%
\AgdaBound{f}\AgdaSpace{}%
\AgdaOperator{\AgdaFunction{∣}}\AgdaSpace{}%
\AgdaOperator{\AgdaFunction{∣}}\AgdaSpace{}%
\AgdaBound{g}\AgdaSpace{}%
\AgdaOperator{\AgdaFunction{∣}}\AgdaSpace{}%
\AgdaBound{finj}\AgdaSpace{}%
\AgdaBound{ginj}\<%
\\
%
\\[\AgdaEmptyExtraSkip]%
%
\>[1]\AgdaFunction{≅-trans-≤}\AgdaSpace{}%
\AgdaSymbol{:}\AgdaSpace{}%
\AgdaBound{𝑨}\AgdaSpace{}%
\AgdaOperator{\AgdaRecord{≅}}\AgdaSpace{}%
\AgdaBound{𝑩}\AgdaSpace{}%
\AgdaSymbol{→}\AgdaSpace{}%
\AgdaBound{𝑩}\AgdaSpace{}%
\AgdaOperator{\AgdaFunction{≤}}\AgdaSpace{}%
\AgdaBound{𝑪}\AgdaSpace{}%
\AgdaSymbol{→}\AgdaSpace{}%
\AgdaBound{𝑨}\AgdaSpace{}%
\AgdaOperator{\AgdaFunction{≤}}\AgdaSpace{}%
\AgdaBound{𝑪}\<%
\\
%
\>[1]\AgdaFunction{≅-trans-≤}\AgdaSpace{}%
\AgdaBound{A≅B}\AgdaSpace{}%
\AgdaSymbol{(}\AgdaBound{h}\AgdaSpace{}%
\AgdaOperator{\AgdaInductiveConstructor{,}}\AgdaSpace{}%
\AgdaBound{hinj}\AgdaSymbol{)}\AgdaSpace{}%
\AgdaSymbol{=}\AgdaSpace{}%
\AgdaSymbol{(}\AgdaFunction{∘-hom}\AgdaSpace{}%
\AgdaSymbol{(}\AgdaField{to}\AgdaSpace{}%
\AgdaBound{A≅B}\AgdaSymbol{)}\AgdaSpace{}%
\AgdaBound{h}\AgdaSymbol{)}\AgdaSpace{}%
\AgdaOperator{\AgdaInductiveConstructor{,}}\AgdaSpace{}%
\AgdaSymbol{(}\AgdaFunction{∘-IsInjective}\AgdaSpace{}%
\AgdaOperator{\AgdaFunction{∣}}\AgdaSpace{}%
\AgdaField{to}\AgdaSpace{}%
\AgdaBound{A≅B}\AgdaSpace{}%
\AgdaOperator{\AgdaFunction{∣}}\AgdaSpace{}%
\AgdaOperator{\AgdaFunction{∣}}\AgdaSpace{}%
\AgdaBound{h}\AgdaSpace{}%
\AgdaOperator{\AgdaFunction{∣}}\AgdaSpace{}%
\AgdaSymbol{(}\AgdaFunction{toIsInjective}\AgdaSpace{}%
\AgdaBound{A≅B}\AgdaSymbol{)}\AgdaSpace{}%
\AgdaBound{hinj}\AgdaSymbol{)}\<%
\end{code}

\hypertarget{products-of-subalgebras}{%
\subsubsection{Products of subalgebras}\label{products-of-subalgebras}}

\begin{code}%
\>[0]\<%
\\
\>[0]\AgdaKeyword{module}\AgdaSpace{}%
\AgdaModule{\AgdaUnderscore{}}\AgdaSpace{}%
\AgdaSymbol{\{}\AgdaBound{ι}\AgdaSpace{}%
\AgdaSymbol{:}\AgdaSpace{}%
\AgdaPostulate{Level}\AgdaSymbol{\}}\AgdaSpace{}%
\AgdaSymbol{\{}\AgdaBound{I}\AgdaSpace{}%
\AgdaSymbol{:}\AgdaSpace{}%
\AgdaPrimitive{Type}\AgdaSpace{}%
\AgdaBound{ι}\AgdaSymbol{\}\{}\AgdaBound{𝒜}\AgdaSpace{}%
\AgdaSymbol{:}\AgdaSpace{}%
\AgdaBound{I}\AgdaSpace{}%
\AgdaSymbol{→}\AgdaSpace{}%
\AgdaRecord{Algebra}\AgdaSpace{}%
\AgdaGeneralizable{α}\AgdaSpace{}%
\AgdaGeneralizable{ρᵃ}\AgdaSymbol{\}\{}\AgdaBound{ℬ}\AgdaSpace{}%
\AgdaSymbol{:}\AgdaSpace{}%
\AgdaBound{I}\AgdaSpace{}%
\AgdaSymbol{→}\AgdaSpace{}%
\AgdaRecord{Algebra}\AgdaSpace{}%
\AgdaGeneralizable{β}\AgdaSpace{}%
\AgdaGeneralizable{ρᵇ}\AgdaSymbol{\}}\AgdaSpace{}%
\AgdaKeyword{where}\<%
\\
%
\\[\AgdaEmptyExtraSkip]%
\>[0][@{}l@{\AgdaIndent{0}}]%
\>[1]\AgdaFunction{⨅-≤}\AgdaSpace{}%
\AgdaSymbol{:}\AgdaSpace{}%
\AgdaSymbol{(∀}\AgdaSpace{}%
\AgdaBound{i}\AgdaSpace{}%
\AgdaSymbol{→}\AgdaSpace{}%
\AgdaBound{ℬ}\AgdaSpace{}%
\AgdaBound{i}\AgdaSpace{}%
\AgdaOperator{\AgdaFunction{≤}}\AgdaSpace{}%
\AgdaBound{𝒜}\AgdaSpace{}%
\AgdaBound{i}\AgdaSymbol{)}\AgdaSpace{}%
\AgdaSymbol{→}\AgdaSpace{}%
\AgdaFunction{⨅}\AgdaSpace{}%
\AgdaBound{ℬ}\AgdaSpace{}%
\AgdaOperator{\AgdaFunction{≤}}\AgdaSpace{}%
\AgdaFunction{⨅}\AgdaSpace{}%
\AgdaBound{𝒜}\<%
\\
%
\>[1]\AgdaFunction{⨅-≤}\AgdaSpace{}%
\AgdaBound{B≤A}\AgdaSpace{}%
\AgdaSymbol{=}\AgdaSpace{}%
\AgdaSymbol{(}\AgdaFunction{hfunc}\AgdaSpace{}%
\AgdaOperator{\AgdaInductiveConstructor{,}}\AgdaSpace{}%
\AgdaFunction{hhom}\AgdaSymbol{)}\AgdaSpace{}%
\AgdaOperator{\AgdaInductiveConstructor{,}}\AgdaSpace{}%
\AgdaFunction{hM}\<%
\\
\>[1][@{}l@{\AgdaIndent{0}}]%
\>[2]\AgdaKeyword{where}\<%
\\
%
\>[2]\AgdaFunction{hi}\AgdaSpace{}%
\AgdaSymbol{:}\AgdaSpace{}%
\AgdaSymbol{∀}\AgdaSpace{}%
\AgdaBound{i}\AgdaSpace{}%
\AgdaSymbol{→}\AgdaSpace{}%
\AgdaFunction{hom}\AgdaSpace{}%
\AgdaSymbol{(}\AgdaBound{ℬ}\AgdaSpace{}%
\AgdaBound{i}\AgdaSymbol{)}\AgdaSpace{}%
\AgdaSymbol{(}\AgdaBound{𝒜}\AgdaSpace{}%
\AgdaBound{i}\AgdaSymbol{)}\<%
\\
%
\>[2]\AgdaFunction{hi}\AgdaSpace{}%
\AgdaBound{i}\AgdaSpace{}%
\AgdaSymbol{=}\AgdaSpace{}%
\AgdaOperator{\AgdaFunction{∣}}\AgdaSpace{}%
\AgdaBound{B≤A}\AgdaSpace{}%
\AgdaBound{i}\AgdaSpace{}%
\AgdaOperator{\AgdaFunction{∣}}\<%
\\
%
\\[\AgdaEmptyExtraSkip]%
%
\>[2]\AgdaFunction{hfunc}\AgdaSpace{}%
\AgdaSymbol{:}\AgdaSpace{}%
\AgdaOperator{\AgdaFunction{𝔻[}}\AgdaSpace{}%
\AgdaFunction{⨅}\AgdaSpace{}%
\AgdaBound{ℬ}\AgdaSpace{}%
\AgdaOperator{\AgdaFunction{]}}\AgdaSpace{}%
\AgdaOperator{\AgdaRecord{⟶}}\AgdaSpace{}%
\AgdaOperator{\AgdaFunction{𝔻[}}\AgdaSpace{}%
\AgdaFunction{⨅}\AgdaSpace{}%
\AgdaBound{𝒜}\AgdaSpace{}%
\AgdaOperator{\AgdaFunction{]}}\<%
\\
%
\>[2]\AgdaSymbol{(}\AgdaFunction{hfunc}\AgdaSpace{}%
\AgdaOperator{\AgdaField{⟨\$⟩}}\AgdaSpace{}%
\AgdaBound{x}\AgdaSymbol{)}\AgdaSpace{}%
\AgdaBound{i}\AgdaSpace{}%
\AgdaSymbol{=}\AgdaSpace{}%
\AgdaOperator{\AgdaFunction{∣}}\AgdaSpace{}%
\AgdaFunction{hi}\AgdaSpace{}%
\AgdaBound{i}\AgdaSpace{}%
\AgdaOperator{\AgdaFunction{∣}}\AgdaSpace{}%
\AgdaOperator{\AgdaField{⟨\$⟩}}\AgdaSpace{}%
\AgdaSymbol{(}\AgdaBound{x}\AgdaSpace{}%
\AgdaBound{i}\AgdaSymbol{)}\<%
\\
%
\>[2]\AgdaField{cong}\AgdaSpace{}%
\AgdaFunction{hfunc}\AgdaSpace{}%
\AgdaSymbol{=}\AgdaSpace{}%
\AgdaSymbol{λ}\AgdaSpace{}%
\AgdaBound{xy}\AgdaSpace{}%
\AgdaBound{i}\AgdaSpace{}%
\AgdaSymbol{→}\AgdaSpace{}%
\AgdaField{cong}\AgdaSpace{}%
\AgdaOperator{\AgdaFunction{∣}}\AgdaSpace{}%
\AgdaFunction{hi}\AgdaSpace{}%
\AgdaBound{i}\AgdaSpace{}%
\AgdaOperator{\AgdaFunction{∣}}\AgdaSpace{}%
\AgdaSymbol{(}\AgdaBound{xy}\AgdaSpace{}%
\AgdaBound{i}\AgdaSymbol{)}\<%
\\
%
\\[\AgdaEmptyExtraSkip]%
%
\>[2]\AgdaFunction{hhom}\AgdaSpace{}%
\AgdaSymbol{:}\AgdaSpace{}%
\AgdaRecord{IsHom}\AgdaSpace{}%
\AgdaSymbol{(}\AgdaFunction{⨅}\AgdaSpace{}%
\AgdaBound{ℬ}\AgdaSymbol{)}\AgdaSpace{}%
\AgdaSymbol{(}\AgdaFunction{⨅}\AgdaSpace{}%
\AgdaBound{𝒜}\AgdaSymbol{)}\AgdaSpace{}%
\AgdaFunction{hfunc}\<%
\\
%
\>[2]\AgdaField{compatible}\AgdaSpace{}%
\AgdaFunction{hhom}\AgdaSpace{}%
\AgdaSymbol{=}\AgdaSpace{}%
\AgdaSymbol{λ}\AgdaSpace{}%
\AgdaBound{i}\AgdaSpace{}%
\AgdaSymbol{→}\AgdaSpace{}%
\AgdaField{compatible}\AgdaSpace{}%
\AgdaOperator{\AgdaFunction{∥}}\AgdaSpace{}%
\AgdaFunction{hi}\AgdaSpace{}%
\AgdaBound{i}\AgdaSpace{}%
\AgdaOperator{\AgdaFunction{∥}}\<%
\\
%
\\[\AgdaEmptyExtraSkip]%
%
\>[2]\AgdaFunction{hM}\AgdaSpace{}%
\AgdaSymbol{:}\AgdaSpace{}%
\AgdaFunction{IsInjective}\AgdaSpace{}%
\AgdaFunction{hfunc}\<%
\\
%
\>[2]\AgdaFunction{hM}\AgdaSpace{}%
\AgdaSymbol{=}\AgdaSpace{}%
\AgdaSymbol{λ}\AgdaSpace{}%
\AgdaBound{xy}\AgdaSpace{}%
\AgdaBound{i}\AgdaSpace{}%
\AgdaSymbol{→}\AgdaSpace{}%
\AgdaOperator{\AgdaFunction{∥}}\AgdaSpace{}%
\AgdaBound{B≤A}\AgdaSpace{}%
\AgdaBound{i}\AgdaSpace{}%
\AgdaOperator{\AgdaFunction{∥}}\AgdaSpace{}%
\AgdaSymbol{(}\AgdaBound{xy}\AgdaSpace{}%
\AgdaBound{i}\AgdaSymbol{)}\<%
\end{code}

\hypertarget{terms}{%
\subsection{Terms}\label{terms}}

\hypertarget{basic-definitions-2}{%
\subsubsection{Basic definitions}\label{basic-definitions-2}}

Fix a signature \texttt{𝑆} and let \texttt{X} denote an arbitrary
nonempty collection of variable symbols. Assume the symbols in
\texttt{X} are distinct from the operation symbols of \texttt{𝑆}, that
is \texttt{X\ ∩\ ∣\ 𝑆\ ∣\ =\ ∅}.

By a \emph{word} in the language of \texttt{𝑆}, we mean a nonempty,
finite sequence of members of \texttt{X\ ∪\ ∣\ 𝑆\ ∣}. We denote the
concatenation of such sequences by simple juxtaposition.

Let \texttt{S₀} denote the set of nullary operation symbols of
\texttt{𝑆}. We define by induction on \texttt{n} the sets \texttt{𝑇ₙ} of
\emph{words} over \texttt{X\ ∪\ ∣\ 𝑆\ ∣} as follows
(cf.~\href{https://www.amazon.com/gp/product/1439851298/ref=as_li_tl?ie=UTF8\&camp=1789\&creative=9325\&creativeASIN=1439851298\&linkCode=as2\&tag=typefunc-20\&linkId=440725c9b1e60817d071c1167dff95fa}{Bergman
(2012)} Def. 4.19):

\texttt{𝑇₀\ :=\ X\ ∪\ S₀} and \texttt{𝑇ₙ₊₁\ :=\ 𝑇ₙ\ ∪\ 𝒯ₙ}

where \texttt{𝒯ₙ} is the collection of all \texttt{f\ t} such that
\texttt{f\ :\ ∣\ 𝑆\ ∣} and \texttt{t\ :\ ∥\ 𝑆\ ∥\ f\ →\ 𝑇ₙ}. (Recall,
\texttt{∥\ 𝑆\ ∥\ f} is the arity of the operation symbol f.)

We define the collection of \emph{terms} in the signature \texttt{𝑆}
over \texttt{X} by \texttt{Term\ X\ :=\ ⋃ₙ\ 𝑇ₙ}. By an 𝑆-\emph{term} we
mean a term in the language of \texttt{𝑆}.

The definition of \texttt{Term\ X} is recursive, indicating that an
inductive type could be used to represent the semantic notion of terms
in type theory. Indeed, such a representation is given by the following
inductive type.

\begin{code}%
\>[0]\<%
\\
\>[0]\AgdaKeyword{data}\AgdaSpace{}%
\AgdaDatatype{Term}\AgdaSpace{}%
\AgdaSymbol{(}\AgdaBound{X}\AgdaSpace{}%
\AgdaSymbol{:}\AgdaSpace{}%
\AgdaPrimitive{Type}\AgdaSpace{}%
\AgdaGeneralizable{χ}\AgdaSpace{}%
\AgdaSymbol{)}\AgdaSpace{}%
\AgdaSymbol{:}\AgdaSpace{}%
\AgdaPrimitive{Type}\AgdaSpace{}%
\AgdaSymbol{(}\AgdaFunction{ov}\AgdaSpace{}%
\AgdaBound{χ}\AgdaSymbol{)}%
\>[39]\AgdaKeyword{where}\<%
\\
\>[0][@{}l@{\AgdaIndent{0}}]%
\>[1]\AgdaInductiveConstructor{ℊ}\AgdaSpace{}%
\AgdaSymbol{:}\AgdaSpace{}%
\AgdaBound{X}\AgdaSpace{}%
\AgdaSymbol{→}\AgdaSpace{}%
\AgdaDatatype{Term}\AgdaSpace{}%
\AgdaBound{X}\<%
\\
%
\>[1]\AgdaInductiveConstructor{node}\AgdaSpace{}%
\AgdaSymbol{:}\AgdaSpace{}%
\AgdaSymbol{(}\AgdaBound{f}\AgdaSpace{}%
\AgdaSymbol{:}\AgdaSpace{}%
\AgdaOperator{\AgdaFunction{∣}}\AgdaSpace{}%
\AgdaBound{𝑆}\AgdaSpace{}%
\AgdaOperator{\AgdaFunction{∣}}\AgdaSymbol{)(}\AgdaBound{t}\AgdaSpace{}%
\AgdaSymbol{:}\AgdaSpace{}%
\AgdaOperator{\AgdaFunction{∥}}\AgdaSpace{}%
\AgdaBound{𝑆}\AgdaSpace{}%
\AgdaOperator{\AgdaFunction{∥}}\AgdaSpace{}%
\AgdaBound{f}\AgdaSpace{}%
\AgdaSymbol{→}\AgdaSpace{}%
\AgdaDatatype{Term}\AgdaSpace{}%
\AgdaBound{X}\AgdaSymbol{)}\AgdaSpace{}%
\AgdaSymbol{→}\AgdaSpace{}%
\AgdaDatatype{Term}\AgdaSpace{}%
\AgdaBound{X}\<%
\\
\>[0]\AgdaKeyword{open}\AgdaSpace{}%
\AgdaModule{Term}\<%
\\
\>[0]\<%
\end{code}

This is a very basic inductive type that represents each term as a tree
with an operation symbol at each \texttt{node} and a variable symbol at
each leaf (\texttt{generator}); hence the constructor names (\texttt{ℊ}
for ``generator'' and \texttt{node} for node).

\textbf{Notation}. As usual, the type \texttt{X} represents an arbitrary
collection of variable symbols. Recall, \texttt{ov\ χ} is our shorthand
notation for the universe level \texttt{𝓞\ ⊔\ 𝓥\ ⊔\ lsuc\ χ}.

\hypertarget{equality-of-terms}{%
\subsubsection{Equality of terms}\label{equality-of-terms}}

We take a different approach here, using Setoids instead of quotient
types. That is, we will define the collection of terms in a signature as
a setoid with a particular equality-of-terms relation, which we must
define. Ultimately we will use this to define the (absolutely free) term
algebra as a Algebra whose carrier is the setoid of terms.

\begin{code}%
\>[0]\<%
\\
\>[0]\AgdaKeyword{module}\AgdaSpace{}%
\AgdaModule{\AgdaUnderscore{}}\AgdaSpace{}%
\AgdaSymbol{\{}\AgdaBound{X}\AgdaSpace{}%
\AgdaSymbol{:}\AgdaSpace{}%
\AgdaPrimitive{Type}\AgdaSpace{}%
\AgdaGeneralizable{χ}\AgdaSpace{}%
\AgdaSymbol{\}}\AgdaSpace{}%
\AgdaKeyword{where}\<%
\\
%
\\[\AgdaEmptyExtraSkip]%
\>[0][@{}l@{\AgdaIndent{0}}]%
\>[1]\AgdaKeyword{data}\AgdaSpace{}%
\AgdaOperator{\AgdaDatatype{\AgdaUnderscore{}≐\AgdaUnderscore{}}}\AgdaSpace{}%
\AgdaSymbol{:}\AgdaSpace{}%
\AgdaDatatype{Term}\AgdaSpace{}%
\AgdaBound{X}\AgdaSpace{}%
\AgdaSymbol{→}\AgdaSpace{}%
\AgdaDatatype{Term}\AgdaSpace{}%
\AgdaBound{X}\AgdaSpace{}%
\AgdaSymbol{→}\AgdaSpace{}%
\AgdaPrimitive{Type}\AgdaSpace{}%
\AgdaSymbol{(}\AgdaFunction{ov}\AgdaSpace{}%
\AgdaBound{χ}\AgdaSymbol{)}\AgdaSpace{}%
\AgdaKeyword{where}\<%
\\
\>[1][@{}l@{\AgdaIndent{0}}]%
\>[2]\AgdaInductiveConstructor{rfl}\AgdaSpace{}%
\AgdaSymbol{:}\AgdaSpace{}%
\AgdaSymbol{\{}\AgdaBound{x}\AgdaSpace{}%
\AgdaBound{y}\AgdaSpace{}%
\AgdaSymbol{:}\AgdaSpace{}%
\AgdaBound{X}\AgdaSymbol{\}}\AgdaSpace{}%
\AgdaSymbol{→}\AgdaSpace{}%
\AgdaBound{x}\AgdaSpace{}%
\AgdaOperator{\AgdaDatatype{≡}}\AgdaSpace{}%
\AgdaBound{y}\AgdaSpace{}%
\AgdaSymbol{→}\AgdaSpace{}%
\AgdaSymbol{(}\AgdaInductiveConstructor{ℊ}\AgdaSpace{}%
\AgdaBound{x}\AgdaSymbol{)}\AgdaSpace{}%
\AgdaOperator{\AgdaDatatype{≐}}\AgdaSpace{}%
\AgdaSymbol{(}\AgdaInductiveConstructor{ℊ}\AgdaSpace{}%
\AgdaBound{y}\AgdaSymbol{)}\<%
\\
%
\>[2]\AgdaInductiveConstructor{gnl}\AgdaSpace{}%
\AgdaSymbol{:}\AgdaSpace{}%
\AgdaSymbol{∀}\AgdaSpace{}%
\AgdaSymbol{\{}\AgdaBound{f}\AgdaSymbol{\}\{}\AgdaBound{s}\AgdaSpace{}%
\AgdaBound{t}\AgdaSpace{}%
\AgdaSymbol{:}\AgdaSpace{}%
\AgdaOperator{\AgdaFunction{∥}}\AgdaSpace{}%
\AgdaBound{𝑆}\AgdaSpace{}%
\AgdaOperator{\AgdaFunction{∥}}\AgdaSpace{}%
\AgdaBound{f}\AgdaSpace{}%
\AgdaSymbol{→}\AgdaSpace{}%
\AgdaDatatype{Term}\AgdaSpace{}%
\AgdaBound{X}\AgdaSymbol{\}}\AgdaSpace{}%
\AgdaSymbol{→}\AgdaSpace{}%
\AgdaSymbol{(∀}\AgdaSpace{}%
\AgdaBound{i}\AgdaSpace{}%
\AgdaSymbol{→}\AgdaSpace{}%
\AgdaSymbol{(}\AgdaBound{s}\AgdaSpace{}%
\AgdaBound{i}\AgdaSymbol{)}\AgdaSpace{}%
\AgdaOperator{\AgdaDatatype{≐}}\AgdaSpace{}%
\AgdaSymbol{(}\AgdaBound{t}\AgdaSpace{}%
\AgdaBound{i}\AgdaSymbol{))}\AgdaSpace{}%
\AgdaSymbol{→}\AgdaSpace{}%
\AgdaSymbol{(}\AgdaInductiveConstructor{node}\AgdaSpace{}%
\AgdaBound{f}\AgdaSpace{}%
\AgdaBound{s}\AgdaSymbol{)}\AgdaSpace{}%
\AgdaOperator{\AgdaDatatype{≐}}\AgdaSpace{}%
\AgdaSymbol{(}\AgdaInductiveConstructor{node}\AgdaSpace{}%
\AgdaBound{f}\AgdaSpace{}%
\AgdaBound{t}\AgdaSymbol{)}\<%
\\
\>[0]\<%
\end{code}

It is easy to show that the equality-of-terms relation \texttt{\_≐\_} is
an equivalence relation, so we omit the formal proof. (See the
\href{https://ualib.github.io/agda-algebras/Terms.Func.Basic.html}{Terms.Func.Basic}
module of the
\href{https://github.com/ualib/agda-algebras}{agda-algebras} library for
details.)

\begin{code}[hide]%
\>[0][@{}l@{\AgdaIndent{1}}]%
\>[1]\AgdaFunction{≐-isRefl}\AgdaSpace{}%
\AgdaSymbol{:}\AgdaSpace{}%
\AgdaFunction{Reflexive}\AgdaSpace{}%
\AgdaOperator{\AgdaDatatype{\AgdaUnderscore{}≐\AgdaUnderscore{}}}\<%
\\
%
\>[1]\AgdaFunction{≐-isRefl}\AgdaSpace{}%
\AgdaSymbol{\{}\AgdaInductiveConstructor{ℊ}\AgdaSpace{}%
\AgdaSymbol{\AgdaUnderscore{}\}}\AgdaSpace{}%
\AgdaSymbol{=}\AgdaSpace{}%
\AgdaInductiveConstructor{rfl}\AgdaSpace{}%
\AgdaInductiveConstructor{≡.refl}\<%
\\
%
\>[1]\AgdaFunction{≐-isRefl}\AgdaSpace{}%
\AgdaSymbol{\{}\AgdaInductiveConstructor{node}\AgdaSpace{}%
\AgdaSymbol{\AgdaUnderscore{}}\AgdaSpace{}%
\AgdaSymbol{\AgdaUnderscore{}\}}\AgdaSpace{}%
\AgdaSymbol{=}\AgdaSpace{}%
\AgdaInductiveConstructor{gnl}\AgdaSpace{}%
\AgdaSymbol{(λ}\AgdaSpace{}%
\AgdaBound{\AgdaUnderscore{}}\AgdaSpace{}%
\AgdaSymbol{→}\AgdaSpace{}%
\AgdaFunction{≐-isRefl}\AgdaSymbol{)}\<%
\\
%
\\[\AgdaEmptyExtraSkip]%
%
\>[1]\AgdaFunction{≐-isSym}\AgdaSpace{}%
\AgdaSymbol{:}\AgdaSpace{}%
\AgdaFunction{Symmetric}\AgdaSpace{}%
\AgdaOperator{\AgdaDatatype{\AgdaUnderscore{}≐\AgdaUnderscore{}}}\<%
\\
%
\>[1]\AgdaFunction{≐-isSym}\AgdaSpace{}%
\AgdaSymbol{(}\AgdaInductiveConstructor{rfl}\AgdaSpace{}%
\AgdaBound{x}\AgdaSymbol{)}\AgdaSpace{}%
\AgdaSymbol{=}\AgdaSpace{}%
\AgdaInductiveConstructor{rfl}\AgdaSpace{}%
\AgdaSymbol{(}\AgdaFunction{≡.sym}\AgdaSpace{}%
\AgdaBound{x}\AgdaSymbol{)}\<%
\\
%
\>[1]\AgdaFunction{≐-isSym}\AgdaSpace{}%
\AgdaSymbol{(}\AgdaInductiveConstructor{gnl}\AgdaSpace{}%
\AgdaBound{x}\AgdaSymbol{)}\AgdaSpace{}%
\AgdaSymbol{=}\AgdaSpace{}%
\AgdaInductiveConstructor{gnl}\AgdaSpace{}%
\AgdaSymbol{(λ}\AgdaSpace{}%
\AgdaBound{i}\AgdaSpace{}%
\AgdaSymbol{→}\AgdaSpace{}%
\AgdaFunction{≐-isSym}\AgdaSpace{}%
\AgdaSymbol{(}\AgdaBound{x}\AgdaSpace{}%
\AgdaBound{i}\AgdaSymbol{))}\<%
\\
%
\\[\AgdaEmptyExtraSkip]%
%
\>[1]\AgdaFunction{≐-isTrans}\AgdaSpace{}%
\AgdaSymbol{:}\AgdaSpace{}%
\AgdaFunction{Transitive}\AgdaSpace{}%
\AgdaOperator{\AgdaDatatype{\AgdaUnderscore{}≐\AgdaUnderscore{}}}\<%
\\
%
\>[1]\AgdaFunction{≐-isTrans}\AgdaSpace{}%
\AgdaSymbol{(}\AgdaInductiveConstructor{rfl}\AgdaSpace{}%
\AgdaBound{x}\AgdaSymbol{)}\AgdaSpace{}%
\AgdaSymbol{(}\AgdaInductiveConstructor{rfl}\AgdaSpace{}%
\AgdaBound{y}\AgdaSymbol{)}\AgdaSpace{}%
\AgdaSymbol{=}\AgdaSpace{}%
\AgdaInductiveConstructor{rfl}\AgdaSpace{}%
\AgdaSymbol{(}\AgdaFunction{≡.trans}\AgdaSpace{}%
\AgdaBound{x}\AgdaSpace{}%
\AgdaBound{y}\AgdaSymbol{)}\<%
\\
%
\>[1]\AgdaFunction{≐-isTrans}\AgdaSpace{}%
\AgdaSymbol{(}\AgdaInductiveConstructor{gnl}\AgdaSpace{}%
\AgdaBound{x}\AgdaSymbol{)}\AgdaSpace{}%
\AgdaSymbol{(}\AgdaInductiveConstructor{gnl}\AgdaSpace{}%
\AgdaBound{y}\AgdaSymbol{)}\AgdaSpace{}%
\AgdaSymbol{=}\AgdaSpace{}%
\AgdaInductiveConstructor{gnl}\AgdaSpace{}%
\AgdaSymbol{(λ}\AgdaSpace{}%
\AgdaBound{i}\AgdaSpace{}%
\AgdaSymbol{→}\AgdaSpace{}%
\AgdaFunction{≐-isTrans}\AgdaSpace{}%
\AgdaSymbol{(}\AgdaBound{x}\AgdaSpace{}%
\AgdaBound{i}\AgdaSymbol{)}\AgdaSpace{}%
\AgdaSymbol{(}\AgdaBound{y}\AgdaSpace{}%
\AgdaBound{i}\AgdaSymbol{))}\<%
\\
%
\\[\AgdaEmptyExtraSkip]%
%
\>[1]\AgdaFunction{≐-isEquiv}\AgdaSpace{}%
\AgdaSymbol{:}\AgdaSpace{}%
\AgdaRecord{IsEquivalence}\AgdaSpace{}%
\AgdaOperator{\AgdaDatatype{\AgdaUnderscore{}≐\AgdaUnderscore{}}}\<%
\\
%
\>[1]\AgdaFunction{≐-isEquiv}\AgdaSpace{}%
\AgdaSymbol{=}\AgdaSpace{}%
\AgdaKeyword{record}\AgdaSpace{}%
\AgdaSymbol{\{}\AgdaSpace{}%
\AgdaField{refl}\AgdaSpace{}%
\AgdaSymbol{=}\AgdaSpace{}%
\AgdaFunction{≐-isRefl}\AgdaSpace{}%
\AgdaSymbol{;}\AgdaSpace{}%
\AgdaField{sym}\AgdaSpace{}%
\AgdaSymbol{=}\AgdaSpace{}%
\AgdaFunction{≐-isSym}\AgdaSpace{}%
\AgdaSymbol{;}\AgdaSpace{}%
\AgdaField{trans}\AgdaSpace{}%
\AgdaSymbol{=}\AgdaSpace{}%
\AgdaFunction{≐-isTrans}\AgdaSpace{}%
\AgdaSymbol{\}}\<%
\end{code}

\hypertarget{the-term-algebra}{%
\subsubsection{The term algebra}\label{the-term-algebra}}

For a given signature \texttt{𝑆}, if the type \texttt{Term\ X} is
nonempty (equivalently, if \texttt{X} or \texttt{∣\ 𝑆\ ∣} is nonempty),
then we can define an algebraic structure, denoted by \texttt{𝑻\ X} and
called the \emph{term algebra in the signature} \texttt{𝑆} \emph{over}
\texttt{X}. Terms are viewed as acting on other terms, so both the
domain and basic operations of the algebra are the terms themselves.

\begin{itemize}
\tightlist
\item
  For each operation symbol \texttt{f\ :\ ∣\ 𝑆\ ∣}, denote by
  \texttt{f\ ̂\ (𝑻\ X)} the operation on \texttt{Term\ X} that maps a
  tuple \texttt{t\ :\ ∥\ 𝑆\ ∥\ f\ →\ ∣\ 𝑻\ X\ ∣} to the formal term
  \texttt{f\ t}.
\item
  Define \texttt{𝑻\ X} to be the algebra with universe
  \texttt{∣\ 𝑻\ X\ ∣\ :=\ Term\ X} and operations \texttt{f\ ̂\ (𝑻\ X)},
  one for each symbol \texttt{f} in \texttt{∣\ 𝑆\ ∣}.
\end{itemize}

In \href{https://wiki.portal.chalmers.se/agda/pmwiki.php}{Agda} the term
algebra can be defined as simply as one might hope.

\begin{code}%
\>[0]\<%
\\
\>[0]\AgdaFunction{TermSetoid}\AgdaSpace{}%
\AgdaSymbol{:}\AgdaSpace{}%
\AgdaSymbol{(}\AgdaBound{X}\AgdaSpace{}%
\AgdaSymbol{:}\AgdaSpace{}%
\AgdaPrimitive{Type}\AgdaSpace{}%
\AgdaGeneralizable{χ}\AgdaSymbol{)}\AgdaSpace{}%
\AgdaSymbol{→}\AgdaSpace{}%
\AgdaRecord{Setoid}\AgdaSpace{}%
\AgdaSymbol{(}\AgdaFunction{ov}\AgdaSpace{}%
\AgdaGeneralizable{χ}\AgdaSymbol{)}\AgdaSpace{}%
\AgdaSymbol{(}\AgdaFunction{ov}\AgdaSpace{}%
\AgdaGeneralizable{χ}\AgdaSymbol{)}\<%
\\
\>[0]\AgdaFunction{TermSetoid}\AgdaSpace{}%
\AgdaBound{X}\AgdaSpace{}%
\AgdaSymbol{=}\AgdaSpace{}%
\AgdaKeyword{record}\AgdaSpace{}%
\AgdaSymbol{\{}\AgdaSpace{}%
\AgdaField{Carrier}\AgdaSpace{}%
\AgdaSymbol{=}\AgdaSpace{}%
\AgdaDatatype{Term}\AgdaSpace{}%
\AgdaBound{X}\AgdaSpace{}%
\AgdaSymbol{;}\AgdaSpace{}%
\AgdaOperator{\AgdaField{\AgdaUnderscore{}≈\AgdaUnderscore{}}}\AgdaSpace{}%
\AgdaSymbol{=}\AgdaSpace{}%
\AgdaOperator{\AgdaDatatype{\AgdaUnderscore{}≐\AgdaUnderscore{}}}\AgdaSpace{}%
\AgdaSymbol{;}\AgdaSpace{}%
\AgdaField{isEquivalence}\AgdaSpace{}%
\AgdaSymbol{=}\AgdaSpace{}%
\AgdaFunction{≐-isEquiv}\AgdaSpace{}%
\AgdaSymbol{\}}\<%
\\
%
\\[\AgdaEmptyExtraSkip]%
\>[0]\AgdaFunction{𝑻}\AgdaSpace{}%
\AgdaSymbol{:}\AgdaSpace{}%
\AgdaSymbol{(}\AgdaBound{X}\AgdaSpace{}%
\AgdaSymbol{:}\AgdaSpace{}%
\AgdaPrimitive{Type}\AgdaSpace{}%
\AgdaGeneralizable{χ}\AgdaSymbol{)}\AgdaSpace{}%
\AgdaSymbol{→}\AgdaSpace{}%
\AgdaRecord{Algebra}\AgdaSpace{}%
\AgdaSymbol{(}\AgdaFunction{ov}\AgdaSpace{}%
\AgdaGeneralizable{χ}\AgdaSymbol{)}\AgdaSpace{}%
\AgdaSymbol{(}\AgdaFunction{ov}\AgdaSpace{}%
\AgdaGeneralizable{χ}\AgdaSymbol{)}\<%
\\
\>[0]\AgdaField{Algebra.Domain}\AgdaSpace{}%
\AgdaSymbol{(}\AgdaFunction{𝑻}\AgdaSpace{}%
\AgdaBound{X}\AgdaSymbol{)}\AgdaSpace{}%
\AgdaSymbol{=}\AgdaSpace{}%
\AgdaFunction{TermSetoid}\AgdaSpace{}%
\AgdaBound{X}\<%
\\
\>[0]\AgdaField{Algebra.Interp}\AgdaSpace{}%
\AgdaSymbol{(}\AgdaFunction{𝑻}\AgdaSpace{}%
\AgdaBound{X}\AgdaSymbol{)}\AgdaSpace{}%
\AgdaOperator{\AgdaField{⟨\$⟩}}\AgdaSpace{}%
\AgdaSymbol{(}\AgdaBound{f}\AgdaSpace{}%
\AgdaOperator{\AgdaInductiveConstructor{,}}\AgdaSpace{}%
\AgdaBound{ts}\AgdaSymbol{)}\AgdaSpace{}%
\AgdaSymbol{=}\AgdaSpace{}%
\AgdaInductiveConstructor{node}\AgdaSpace{}%
\AgdaBound{f}\AgdaSpace{}%
\AgdaBound{ts}\<%
\\
\>[0]\AgdaField{cong}\AgdaSpace{}%
\AgdaSymbol{(}\AgdaField{Algebra.Interp}\AgdaSpace{}%
\AgdaSymbol{(}\AgdaFunction{𝑻}\AgdaSpace{}%
\AgdaBound{X}\AgdaSymbol{))}\AgdaSpace{}%
\AgdaSymbol{(}\AgdaInductiveConstructor{≡.refl}\AgdaSpace{}%
\AgdaOperator{\AgdaInductiveConstructor{,}}\AgdaSpace{}%
\AgdaBound{ss≐ts}\AgdaSymbol{)}\AgdaSpace{}%
\AgdaSymbol{=}\AgdaSpace{}%
\AgdaInductiveConstructor{gnl}\AgdaSpace{}%
\AgdaBound{ss≐ts}\<%
\end{code}

\hypertarget{interpretation-of-terms}{%
\subsubsection{Interpretation of terms}\label{interpretation-of-terms}}

The approach to terms and their interpretation in this module was
inspired by
\href{http://www.cse.chalmers.se/~abela/agda/MultiSortedAlgebra.pdf}{Andreas
Abel's formal proof of Birkhoff's completeness theorem}.

A substitution from \texttt{X} to \texttt{Y} associates a term in
\texttt{X} with each variable in \texttt{Y}. The definition of
\texttt{Sub} given here is essentially the same as the one given by
Andreas Abel, as is the recursive definition of the syntax
\texttt{t\ {[}\ σ\ {]}}, which denotes a term \texttt{t} applied to a
substitution \texttt{σ}.

\begin{code}%
\>[0]\<%
\\
\>[0]\AgdaFunction{Sub}\AgdaSpace{}%
\AgdaSymbol{:}\AgdaSpace{}%
\AgdaPrimitive{Type}\AgdaSpace{}%
\AgdaGeneralizable{χ}\AgdaSpace{}%
\AgdaSymbol{→}\AgdaSpace{}%
\AgdaPrimitive{Type}\AgdaSpace{}%
\AgdaGeneralizable{χ}\AgdaSpace{}%
\AgdaSymbol{→}\AgdaSpace{}%
\AgdaPrimitive{Type}\AgdaSpace{}%
\AgdaSymbol{(}\AgdaFunction{ov}\AgdaSpace{}%
\AgdaGeneralizable{χ}\AgdaSymbol{)}\<%
\\
\>[0]\AgdaFunction{Sub}\AgdaSpace{}%
\AgdaBound{X}\AgdaSpace{}%
\AgdaBound{Y}\AgdaSpace{}%
\AgdaSymbol{=}\AgdaSpace{}%
\AgdaSymbol{(}\AgdaBound{y}\AgdaSpace{}%
\AgdaSymbol{:}\AgdaSpace{}%
\AgdaBound{Y}\AgdaSymbol{)}\AgdaSpace{}%
\AgdaSymbol{→}\AgdaSpace{}%
\AgdaDatatype{Term}\AgdaSpace{}%
\AgdaBound{X}\<%
\\
%
\\[\AgdaEmptyExtraSkip]%
\>[0]\AgdaOperator{\AgdaFunction{\AgdaUnderscore{}[\AgdaUnderscore{}]}}\AgdaSpace{}%
\AgdaSymbol{:}\AgdaSpace{}%
\AgdaSymbol{\{}\AgdaBound{X}\AgdaSpace{}%
\AgdaBound{Y}\AgdaSpace{}%
\AgdaSymbol{:}\AgdaSpace{}%
\AgdaPrimitive{Type}\AgdaSpace{}%
\AgdaGeneralizable{χ}\AgdaSymbol{\}(}\AgdaBound{t}\AgdaSpace{}%
\AgdaSymbol{:}\AgdaSpace{}%
\AgdaDatatype{Term}\AgdaSpace{}%
\AgdaBound{Y}\AgdaSymbol{)}\AgdaSpace{}%
\AgdaSymbol{(}\AgdaBound{σ}\AgdaSpace{}%
\AgdaSymbol{:}\AgdaSpace{}%
\AgdaFunction{Sub}\AgdaSpace{}%
\AgdaBound{X}\AgdaSpace{}%
\AgdaBound{Y}\AgdaSymbol{)}\AgdaSpace{}%
\AgdaSymbol{→}\AgdaSpace{}%
\AgdaDatatype{Term}\AgdaSpace{}%
\AgdaBound{X}\<%
\\
\>[0]\AgdaSymbol{(}\AgdaInductiveConstructor{ℊ}\AgdaSpace{}%
\AgdaBound{x}\AgdaSymbol{)}\AgdaSpace{}%
\AgdaOperator{\AgdaFunction{[}}\AgdaSpace{}%
\AgdaBound{σ}\AgdaSpace{}%
\AgdaOperator{\AgdaFunction{]}}\AgdaSpace{}%
\AgdaSymbol{=}\AgdaSpace{}%
\AgdaBound{σ}\AgdaSpace{}%
\AgdaBound{x}\<%
\\
\>[0]\AgdaSymbol{(}\AgdaInductiveConstructor{node}\AgdaSpace{}%
\AgdaBound{f}\AgdaSpace{}%
\AgdaBound{ts}\AgdaSymbol{)}\AgdaSpace{}%
\AgdaOperator{\AgdaFunction{[}}\AgdaSpace{}%
\AgdaBound{σ}\AgdaSpace{}%
\AgdaOperator{\AgdaFunction{]}}\AgdaSpace{}%
\AgdaSymbol{=}\AgdaSpace{}%
\AgdaInductiveConstructor{node}\AgdaSpace{}%
\AgdaBound{f}\AgdaSpace{}%
\AgdaSymbol{(λ}\AgdaSpace{}%
\AgdaBound{i}\AgdaSpace{}%
\AgdaSymbol{→}\AgdaSpace{}%
\AgdaBound{ts}\AgdaSpace{}%
\AgdaBound{i}\AgdaSpace{}%
\AgdaOperator{\AgdaFunction{[}}\AgdaSpace{}%
\AgdaBound{σ}\AgdaSpace{}%
\AgdaOperator{\AgdaFunction{]}}\AgdaSymbol{)}\<%
\\
\>[0]\<%
\end{code}

An environment for an algebra \texttt{𝑨} in a context \texttt{X} is a
map that assigns to each variable \texttt{x\ :\ X} an element in the
domain of \texttt{𝑨}, packaged together with an equality of
environments, which is simply pointwise equality (relatively to the
setoid equality of the underlying domain of \texttt{𝑨}).

\begin{code}%
\>[0]\<%
\\
\>[0]\AgdaKeyword{module}\AgdaSpace{}%
\AgdaModule{Environment}\AgdaSpace{}%
\AgdaSymbol{(}\AgdaBound{𝑨}\AgdaSpace{}%
\AgdaSymbol{:}\AgdaSpace{}%
\AgdaRecord{Algebra}\AgdaSpace{}%
\AgdaGeneralizable{α}\AgdaSpace{}%
\AgdaGeneralizable{ℓ}\AgdaSymbol{)}\AgdaSpace{}%
\AgdaKeyword{where}\<%
\\
\>[0][@{}l@{\AgdaIndent{0}}]%
\>[1]\AgdaKeyword{open}\AgdaSpace{}%
\AgdaModule{Setoid}\AgdaSpace{}%
\AgdaOperator{\AgdaFunction{𝔻[}}\AgdaSpace{}%
\AgdaBound{𝑨}\AgdaSpace{}%
\AgdaOperator{\AgdaFunction{]}}\AgdaSpace{}%
\AgdaKeyword{using}\AgdaSpace{}%
\AgdaSymbol{(}\AgdaSpace{}%
\AgdaOperator{\AgdaField{\AgdaUnderscore{}≈\AgdaUnderscore{}}}\AgdaSpace{}%
\AgdaSymbol{;}\AgdaSpace{}%
\AgdaFunction{refl}\AgdaSpace{}%
\AgdaSymbol{;}\AgdaSpace{}%
\AgdaFunction{sym}\AgdaSpace{}%
\AgdaSymbol{;}\AgdaSpace{}%
\AgdaFunction{trans}\AgdaSpace{}%
\AgdaSymbol{)}\<%
\\
%
\>[1]\AgdaFunction{Env}\AgdaSpace{}%
\AgdaSymbol{:}\AgdaSpace{}%
\AgdaPrimitive{Type}\AgdaSpace{}%
\AgdaGeneralizable{χ}\AgdaSpace{}%
\AgdaSymbol{→}\AgdaSpace{}%
\AgdaRecord{Setoid}\AgdaSpace{}%
\AgdaSymbol{\AgdaUnderscore{}}\AgdaSpace{}%
\AgdaSymbol{\AgdaUnderscore{}}\<%
\\
%
\>[1]\AgdaFunction{Env}\AgdaSpace{}%
\AgdaBound{X}\AgdaSpace{}%
\AgdaSymbol{=}\AgdaSpace{}%
\AgdaKeyword{record}%
\>[17]\AgdaSymbol{\{}\AgdaSpace{}%
\AgdaField{Carrier}\AgdaSpace{}%
\AgdaSymbol{=}\AgdaSpace{}%
\AgdaBound{X}\AgdaSpace{}%
\AgdaSymbol{→}\AgdaSpace{}%
\AgdaOperator{\AgdaFunction{𝕌[}}\AgdaSpace{}%
\AgdaBound{𝑨}\AgdaSpace{}%
\AgdaOperator{\AgdaFunction{]}}\<%
\\
%
\>[17]\AgdaSymbol{;}\AgdaSpace{}%
\AgdaOperator{\AgdaField{\AgdaUnderscore{}≈\AgdaUnderscore{}}}\AgdaSpace{}%
\AgdaSymbol{=}\AgdaSpace{}%
\AgdaSymbol{λ}\AgdaSpace{}%
\AgdaBound{ρ}\AgdaSpace{}%
\AgdaBound{ρ'}\AgdaSpace{}%
\AgdaSymbol{→}\AgdaSpace{}%
\AgdaSymbol{(}\AgdaBound{x}\AgdaSpace{}%
\AgdaSymbol{:}\AgdaSpace{}%
\AgdaBound{X}\AgdaSymbol{)}\AgdaSpace{}%
\AgdaSymbol{→}\AgdaSpace{}%
\AgdaBound{ρ}\AgdaSpace{}%
\AgdaBound{x}\AgdaSpace{}%
\AgdaOperator{\AgdaFunction{≈}}\AgdaSpace{}%
\AgdaBound{ρ'}\AgdaSpace{}%
\AgdaBound{x}\<%
\\
%
\>[17]\AgdaSymbol{;}\AgdaSpace{}%
\AgdaField{isEquivalence}\AgdaSpace{}%
\AgdaSymbol{=}\AgdaSpace{}%
\AgdaKeyword{record}%
\>[43]\AgdaSymbol{\{}\AgdaSpace{}%
\AgdaField{refl}%
\>[52]\AgdaSymbol{=}\AgdaSpace{}%
\AgdaSymbol{λ}\AgdaSpace{}%
\AgdaBound{\AgdaUnderscore{}}\AgdaSpace{}%
\AgdaSymbol{→}\AgdaSpace{}%
\AgdaFunction{refl}\<%
\\
%
\>[43]\AgdaSymbol{;}\AgdaSpace{}%
\AgdaField{sym}%
\>[52]\AgdaSymbol{=}\AgdaSpace{}%
\AgdaSymbol{λ}\AgdaSpace{}%
\AgdaBound{h}\AgdaSpace{}%
\AgdaBound{x}\AgdaSpace{}%
\AgdaSymbol{→}\AgdaSpace{}%
\AgdaFunction{sym}\AgdaSpace{}%
\AgdaSymbol{(}\AgdaBound{h}\AgdaSpace{}%
\AgdaBound{x}\AgdaSymbol{)}\<%
\\
%
\>[43]\AgdaSymbol{;}\AgdaSpace{}%
\AgdaField{trans}%
\>[52]\AgdaSymbol{=}\AgdaSpace{}%
\AgdaSymbol{λ}\AgdaSpace{}%
\AgdaBound{g}\AgdaSpace{}%
\AgdaBound{h}\AgdaSpace{}%
\AgdaBound{x}\AgdaSpace{}%
\AgdaSymbol{→}\AgdaSpace{}%
\AgdaFunction{trans}\AgdaSpace{}%
\AgdaSymbol{(}\AgdaBound{g}\AgdaSpace{}%
\AgdaBound{x}\AgdaSymbol{)(}\AgdaBound{h}\AgdaSpace{}%
\AgdaBound{x}\AgdaSymbol{)}\AgdaSpace{}%
\AgdaSymbol{\}\}}\<%
\\
\>[0]\<%
\end{code}

Next we define \emph{evaluation of a term} in an environment \texttt{ρ},
interpreted in the algebra \texttt{𝑨}.

\begin{code}%
\>[0]\<%
\\
\>[0][@{}l@{\AgdaIndent{1}}]%
\>[1]\AgdaOperator{\AgdaFunction{⟦\AgdaUnderscore{}⟧}}\AgdaSpace{}%
\AgdaSymbol{:}\AgdaSpace{}%
\AgdaSymbol{\{}\AgdaBound{X}\AgdaSpace{}%
\AgdaSymbol{:}\AgdaSpace{}%
\AgdaPrimitive{Type}\AgdaSpace{}%
\AgdaGeneralizable{χ}\AgdaSymbol{\}(}\AgdaBound{t}\AgdaSpace{}%
\AgdaSymbol{:}\AgdaSpace{}%
\AgdaDatatype{Term}\AgdaSpace{}%
\AgdaBound{X}\AgdaSymbol{)}\AgdaSpace{}%
\AgdaSymbol{→}\AgdaSpace{}%
\AgdaSymbol{(}\AgdaFunction{Env}\AgdaSpace{}%
\AgdaBound{X}\AgdaSymbol{)}\AgdaSpace{}%
\AgdaOperator{\AgdaRecord{⟶}}\AgdaSpace{}%
\AgdaOperator{\AgdaFunction{𝔻[}}\AgdaSpace{}%
\AgdaBound{𝑨}\AgdaSpace{}%
\AgdaOperator{\AgdaFunction{]}}\<%
\\
%
\>[1]\AgdaOperator{\AgdaFunction{⟦}}\AgdaSpace{}%
\AgdaInductiveConstructor{ℊ}\AgdaSpace{}%
\AgdaBound{x}\AgdaSpace{}%
\AgdaOperator{\AgdaFunction{⟧}}\AgdaSpace{}%
\AgdaOperator{\AgdaField{⟨\$⟩}}\AgdaSpace{}%
\AgdaBound{ρ}\AgdaSpace{}%
\AgdaSymbol{=}\AgdaSpace{}%
\AgdaBound{ρ}\AgdaSpace{}%
\AgdaBound{x}\<%
\\
%
\>[1]\AgdaOperator{\AgdaFunction{⟦}}\AgdaSpace{}%
\AgdaInductiveConstructor{node}\AgdaSpace{}%
\AgdaBound{f}\AgdaSpace{}%
\AgdaBound{args}\AgdaSpace{}%
\AgdaOperator{\AgdaFunction{⟧}}\AgdaSpace{}%
\AgdaOperator{\AgdaField{⟨\$⟩}}\AgdaSpace{}%
\AgdaBound{ρ}\AgdaSpace{}%
\AgdaSymbol{=}\AgdaSpace{}%
\AgdaSymbol{(}\AgdaField{Interp}\AgdaSpace{}%
\AgdaBound{𝑨}\AgdaSymbol{)}\AgdaSpace{}%
\AgdaOperator{\AgdaField{⟨\$⟩}}\AgdaSpace{}%
\AgdaSymbol{(}\AgdaBound{f}\AgdaSpace{}%
\AgdaOperator{\AgdaInductiveConstructor{,}}\AgdaSpace{}%
\AgdaSymbol{λ}\AgdaSpace{}%
\AgdaBound{i}\AgdaSpace{}%
\AgdaSymbol{→}\AgdaSpace{}%
\AgdaOperator{\AgdaFunction{⟦}}\AgdaSpace{}%
\AgdaBound{args}\AgdaSpace{}%
\AgdaBound{i}\AgdaSpace{}%
\AgdaOperator{\AgdaFunction{⟧}}\AgdaSpace{}%
\AgdaOperator{\AgdaField{⟨\$⟩}}\AgdaSpace{}%
\AgdaBound{ρ}\AgdaSymbol{)}\<%
\\
%
\>[1]\AgdaField{cong}\AgdaSpace{}%
\AgdaOperator{\AgdaFunction{⟦}}\AgdaSpace{}%
\AgdaInductiveConstructor{ℊ}\AgdaSpace{}%
\AgdaBound{x}\AgdaSpace{}%
\AgdaOperator{\AgdaFunction{⟧}}\AgdaSpace{}%
\AgdaBound{u≈v}\AgdaSpace{}%
\AgdaSymbol{=}\AgdaSpace{}%
\AgdaBound{u≈v}\AgdaSpace{}%
\AgdaBound{x}\<%
\\
%
\>[1]\AgdaField{cong}\AgdaSpace{}%
\AgdaOperator{\AgdaFunction{⟦}}\AgdaSpace{}%
\AgdaInductiveConstructor{node}\AgdaSpace{}%
\AgdaBound{f}\AgdaSpace{}%
\AgdaBound{args}\AgdaSpace{}%
\AgdaOperator{\AgdaFunction{⟧}}\AgdaSpace{}%
\AgdaBound{x≈y}\AgdaSpace{}%
\AgdaSymbol{=}\AgdaSpace{}%
\AgdaField{cong}\AgdaSpace{}%
\AgdaSymbol{(}\AgdaField{Interp}\AgdaSpace{}%
\AgdaBound{𝑨}\AgdaSymbol{)(}\AgdaInductiveConstructor{≡.refl}\AgdaSpace{}%
\AgdaOperator{\AgdaInductiveConstructor{,}}\AgdaSpace{}%
\AgdaSymbol{λ}\AgdaSpace{}%
\AgdaBound{i}\AgdaSpace{}%
\AgdaSymbol{→}\AgdaSpace{}%
\AgdaField{cong}\AgdaSpace{}%
\AgdaOperator{\AgdaFunction{⟦}}\AgdaSpace{}%
\AgdaBound{args}\AgdaSpace{}%
\AgdaBound{i}\AgdaSpace{}%
\AgdaOperator{\AgdaFunction{⟧}}\AgdaSpace{}%
\AgdaBound{x≈y}\AgdaSpace{}%
\AgdaSymbol{)}\<%
\\
\>[0]\<%
\end{code}

An equality between two terms holds in a model if the two terms are
equal under all valuations of their free variables
(cf.~\href{http://www.cse.chalmers.se/~abela/agda/MultiSortedAlgebra.pdf}{Andreas
Abel's formal proof of Birkhoff's completeness theorem}).

\begin{code}%
\>[0]\<%
\\
\>[0][@{}l@{\AgdaIndent{1}}]%
\>[1]\AgdaFunction{Equal}\AgdaSpace{}%
\AgdaSymbol{:}\AgdaSpace{}%
\AgdaSymbol{∀}\AgdaSpace{}%
\AgdaSymbol{\{}\AgdaBound{X}\AgdaSpace{}%
\AgdaSymbol{:}\AgdaSpace{}%
\AgdaPrimitive{Type}\AgdaSpace{}%
\AgdaGeneralizable{χ}\AgdaSymbol{\}}\AgdaSpace{}%
\AgdaSymbol{(}\AgdaBound{s}\AgdaSpace{}%
\AgdaBound{t}\AgdaSpace{}%
\AgdaSymbol{:}\AgdaSpace{}%
\AgdaDatatype{Term}\AgdaSpace{}%
\AgdaBound{X}\AgdaSymbol{)}\AgdaSpace{}%
\AgdaSymbol{→}\AgdaSpace{}%
\AgdaPrimitive{Type}\AgdaSpace{}%
\AgdaSymbol{\AgdaUnderscore{}}\<%
\\
%
\>[1]\AgdaFunction{Equal}\AgdaSpace{}%
\AgdaSymbol{\{}\AgdaArgument{X}\AgdaSpace{}%
\AgdaSymbol{=}\AgdaSpace{}%
\AgdaBound{X}\AgdaSymbol{\}}\AgdaSpace{}%
\AgdaBound{s}\AgdaSpace{}%
\AgdaBound{t}\AgdaSpace{}%
\AgdaSymbol{=}\AgdaSpace{}%
\AgdaSymbol{∀}\AgdaSpace{}%
\AgdaSymbol{(}\AgdaBound{ρ}\AgdaSpace{}%
\AgdaSymbol{:}\AgdaSpace{}%
\AgdaField{Carrier}\AgdaSpace{}%
\AgdaSymbol{(}\AgdaFunction{Env}\AgdaSpace{}%
\AgdaBound{X}\AgdaSymbol{))}\AgdaSpace{}%
\AgdaSymbol{→}%
\>[48]\AgdaOperator{\AgdaFunction{⟦}}\AgdaSpace{}%
\AgdaBound{s}\AgdaSpace{}%
\AgdaOperator{\AgdaFunction{⟧}}\AgdaSpace{}%
\AgdaOperator{\AgdaField{⟨\$⟩}}\AgdaSpace{}%
\AgdaBound{ρ}\AgdaSpace{}%
\AgdaOperator{\AgdaFunction{≈}}\AgdaSpace{}%
\AgdaOperator{\AgdaFunction{⟦}}\AgdaSpace{}%
\AgdaBound{t}\AgdaSpace{}%
\AgdaOperator{\AgdaFunction{⟧}}\AgdaSpace{}%
\AgdaOperator{\AgdaField{⟨\$⟩}}\AgdaSpace{}%
\AgdaBound{ρ}\<%
\\
%
\\[\AgdaEmptyExtraSkip]%
%
\>[1]\AgdaFunction{≐→Equal}\AgdaSpace{}%
\AgdaSymbol{:}\AgdaSpace{}%
\AgdaSymbol{\{}\AgdaBound{X}\AgdaSpace{}%
\AgdaSymbol{:}\AgdaSpace{}%
\AgdaPrimitive{Type}\AgdaSpace{}%
\AgdaGeneralizable{χ}\AgdaSymbol{\}(}\AgdaBound{s}\AgdaSpace{}%
\AgdaBound{t}\AgdaSpace{}%
\AgdaSymbol{:}\AgdaSpace{}%
\AgdaDatatype{Term}\AgdaSpace{}%
\AgdaBound{X}\AgdaSymbol{)}\AgdaSpace{}%
\AgdaSymbol{→}\AgdaSpace{}%
\AgdaBound{s}\AgdaSpace{}%
\AgdaOperator{\AgdaDatatype{≐}}\AgdaSpace{}%
\AgdaBound{t}\AgdaSpace{}%
\AgdaSymbol{→}\AgdaSpace{}%
\AgdaFunction{Equal}\AgdaSpace{}%
\AgdaBound{s}\AgdaSpace{}%
\AgdaBound{t}\<%
\\
%
\>[1]\AgdaFunction{≐→Equal}\AgdaSpace{}%
\AgdaDottedPattern{\AgdaSymbol{.(}}\AgdaDottedPattern{\AgdaInductiveConstructor{ℊ}}\AgdaSpace{}%
\AgdaDottedPattern{\AgdaSymbol{\AgdaUnderscore{})}}\AgdaSpace{}%
\AgdaDottedPattern{\AgdaSymbol{.(}}\AgdaDottedPattern{\AgdaInductiveConstructor{ℊ}}\AgdaSpace{}%
\AgdaDottedPattern{\AgdaSymbol{\AgdaUnderscore{})}}\AgdaSpace{}%
\AgdaSymbol{(}\AgdaInductiveConstructor{rfl}\AgdaSpace{}%
\AgdaInductiveConstructor{≡.refl}\AgdaSymbol{)}\AgdaSpace{}%
\AgdaSymbol{=}\AgdaSpace{}%
\AgdaSymbol{λ}\AgdaSpace{}%
\AgdaBound{\AgdaUnderscore{}}\AgdaSpace{}%
\AgdaSymbol{→}\AgdaSpace{}%
\AgdaFunction{refl}\<%
\\
%
\>[1]\AgdaFunction{≐→Equal}\AgdaSpace{}%
\AgdaSymbol{(}\AgdaInductiveConstructor{node}\AgdaSpace{}%
\AgdaSymbol{\AgdaUnderscore{}}\AgdaSpace{}%
\AgdaBound{s}\AgdaSymbol{)(}\AgdaInductiveConstructor{node}\AgdaSpace{}%
\AgdaSymbol{\AgdaUnderscore{}}\AgdaSpace{}%
\AgdaBound{t}\AgdaSymbol{)(}\AgdaInductiveConstructor{gnl}\AgdaSpace{}%
\AgdaBound{x}\AgdaSymbol{)}\AgdaSpace{}%
\AgdaSymbol{=}\<%
\\
\>[1][@{}l@{\AgdaIndent{0}}]%
\>[2]\AgdaSymbol{λ}\AgdaSpace{}%
\AgdaBound{ρ}\AgdaSpace{}%
\AgdaSymbol{→}\AgdaSpace{}%
\AgdaField{cong}\AgdaSpace{}%
\AgdaSymbol{(}\AgdaField{Interp}\AgdaSpace{}%
\AgdaBound{𝑨}\AgdaSymbol{)(}\AgdaInductiveConstructor{≡.refl}\AgdaSpace{}%
\AgdaOperator{\AgdaInductiveConstructor{,}}\AgdaSpace{}%
\AgdaSymbol{λ}\AgdaSpace{}%
\AgdaBound{i}\AgdaSpace{}%
\AgdaSymbol{→}\AgdaSpace{}%
\AgdaFunction{≐→Equal}\AgdaSymbol{(}\AgdaBound{s}\AgdaSpace{}%
\AgdaBound{i}\AgdaSymbol{)(}\AgdaBound{t}\AgdaSpace{}%
\AgdaBound{i}\AgdaSymbol{)(}\AgdaBound{x}\AgdaSpace{}%
\AgdaBound{i}\AgdaSymbol{)}\AgdaBound{ρ}\AgdaSpace{}%
\AgdaSymbol{)}\<%
\\
\>[0]\<%
\end{code}

The proof that \texttt{Equal} is an equivalence relation is trivial, so
we omit it. (See the
\href{https://ualib.github.io/agda-algebras/Varieties.Func.SoundAndComplete.html}{Varieties.Func.SoundAndComplete}
module of the
\href{https://github.com/ualib/agda-algebras}{agda-algebras} library for
details.)

\begin{code}[hide]%
\>[0][@{}l@{\AgdaIndent{1}}]%
\>[1]\AgdaFunction{EqualIsEquiv}\AgdaSpace{}%
\AgdaSymbol{:}\AgdaSpace{}%
\AgdaSymbol{\{}\AgdaBound{Γ}\AgdaSpace{}%
\AgdaSymbol{:}\AgdaSpace{}%
\AgdaPrimitive{Type}\AgdaSpace{}%
\AgdaGeneralizable{χ}\AgdaSymbol{\}}\AgdaSpace{}%
\AgdaSymbol{→}\AgdaSpace{}%
\AgdaRecord{IsEquivalence}\AgdaSpace{}%
\AgdaSymbol{(}\AgdaFunction{Equal}\AgdaSpace{}%
\AgdaSymbol{\{}\AgdaArgument{X}\AgdaSpace{}%
\AgdaSymbol{=}\AgdaSpace{}%
\AgdaBound{Γ}\AgdaSymbol{\})}\<%
\\
%
\>[1]\AgdaField{IsEquivalence.refl}%
\>[21]\AgdaFunction{EqualIsEquiv}\AgdaSpace{}%
\AgdaSymbol{=}\AgdaSpace{}%
\AgdaSymbol{λ}\AgdaSpace{}%
\AgdaBound{\AgdaUnderscore{}}\AgdaSpace{}%
\AgdaSymbol{→}\AgdaSpace{}%
\AgdaFunction{refl}\<%
\\
%
\>[1]\AgdaField{IsEquivalence.sym}%
\>[21]\AgdaFunction{EqualIsEquiv}\AgdaSpace{}%
\AgdaSymbol{=}\AgdaSpace{}%
\AgdaSymbol{λ}\AgdaSpace{}%
\AgdaBound{x=y}\AgdaSpace{}%
\AgdaBound{ρ}\AgdaSpace{}%
\AgdaSymbol{→}\AgdaSpace{}%
\AgdaFunction{sym}\AgdaSpace{}%
\AgdaSymbol{(}\AgdaBound{x=y}\AgdaSpace{}%
\AgdaBound{ρ}\AgdaSymbol{)}\<%
\\
%
\>[1]\AgdaField{IsEquivalence.trans}\AgdaSpace{}%
\AgdaFunction{EqualIsEquiv}\AgdaSpace{}%
\AgdaSymbol{=}\AgdaSpace{}%
\AgdaSymbol{λ}\AgdaSpace{}%
\AgdaBound{ij}\AgdaSpace{}%
\AgdaBound{jk}\AgdaSpace{}%
\AgdaBound{ρ}\AgdaSpace{}%
\AgdaSymbol{→}\AgdaSpace{}%
\AgdaFunction{trans}\AgdaSpace{}%
\AgdaSymbol{(}\AgdaBound{ij}\AgdaSpace{}%
\AgdaBound{ρ}\AgdaSymbol{)}\AgdaSpace{}%
\AgdaSymbol{(}\AgdaBound{jk}\AgdaSpace{}%
\AgdaBound{ρ}\AgdaSymbol{)}\<%
\end{code}

Evaluation of a substitution gives an environment
(cf.~\href{http://www.cse.chalmers.se/~abela/agda/MultiSortedAlgebra.pdf}{Andreas
Abel's formal proof of Birkhoff's completeness theorem})

\begin{code}%
\>[0]\<%
\\
%
\>[1]\AgdaOperator{\AgdaFunction{⟦\AgdaUnderscore{}⟧s}}\AgdaSpace{}%
\AgdaSymbol{:}\AgdaSpace{}%
\AgdaSymbol{\{}\AgdaBound{X}\AgdaSpace{}%
\AgdaBound{Y}\AgdaSpace{}%
\AgdaSymbol{:}\AgdaSpace{}%
\AgdaPrimitive{Type}\AgdaSpace{}%
\AgdaGeneralizable{χ}\AgdaSymbol{\}}\AgdaSpace{}%
\AgdaSymbol{→}\AgdaSpace{}%
\AgdaFunction{Sub}\AgdaSpace{}%
\AgdaBound{X}\AgdaSpace{}%
\AgdaBound{Y}\AgdaSpace{}%
\AgdaSymbol{→}\AgdaSpace{}%
\AgdaField{Carrier}\AgdaSymbol{(}\AgdaFunction{Env}\AgdaSpace{}%
\AgdaBound{X}\AgdaSymbol{)}\AgdaSpace{}%
\AgdaSymbol{→}\AgdaSpace{}%
\AgdaField{Carrier}\AgdaSpace{}%
\AgdaSymbol{(}\AgdaFunction{Env}\AgdaSpace{}%
\AgdaBound{Y}\AgdaSymbol{)}\<%
\\
%
\>[1]\AgdaOperator{\AgdaFunction{⟦}}\AgdaSpace{}%
\AgdaBound{σ}\AgdaSpace{}%
\AgdaOperator{\AgdaFunction{⟧s}}\AgdaSpace{}%
\AgdaBound{ρ}\AgdaSpace{}%
\AgdaBound{x}\AgdaSpace{}%
\AgdaSymbol{=}\AgdaSpace{}%
\AgdaOperator{\AgdaFunction{⟦}}\AgdaSpace{}%
\AgdaBound{σ}\AgdaSpace{}%
\AgdaBound{x}\AgdaSpace{}%
\AgdaOperator{\AgdaFunction{⟧}}\AgdaSpace{}%
\AgdaOperator{\AgdaField{⟨\$⟩}}\AgdaSpace{}%
\AgdaBound{ρ}\<%
\\
\>[0]\<%
\end{code}

Next we prove that ⟦t{[}σ{]}⟧ρ ≃ ⟦t⟧⟦σ⟧ρ
(cf.~\href{http://www.cse.chalmers.se/~abela/agda/MultiSortedAlgebra.pdf}{Andreas
Abel's formal proof of Birkhoff's completeness theorem}).

\begin{code}%
\>[0]\<%
\\
\>[0][@{}l@{\AgdaIndent{1}}]%
\>[1]\AgdaFunction{substitution}\AgdaSpace{}%
\AgdaSymbol{:}\AgdaSpace{}%
\AgdaSymbol{\{}\AgdaBound{X}\AgdaSpace{}%
\AgdaBound{Y}\AgdaSpace{}%
\AgdaSymbol{:}\AgdaSpace{}%
\AgdaPrimitive{Type}\AgdaSpace{}%
\AgdaGeneralizable{χ}\AgdaSymbol{\}}\AgdaSpace{}%
\AgdaSymbol{→}\AgdaSpace{}%
\AgdaSymbol{(}\AgdaBound{t}\AgdaSpace{}%
\AgdaSymbol{:}\AgdaSpace{}%
\AgdaDatatype{Term}\AgdaSpace{}%
\AgdaBound{Y}\AgdaSymbol{)}\AgdaSpace{}%
\AgdaSymbol{(}\AgdaBound{σ}\AgdaSpace{}%
\AgdaSymbol{:}\AgdaSpace{}%
\AgdaFunction{Sub}\AgdaSpace{}%
\AgdaBound{X}\AgdaSpace{}%
\AgdaBound{Y}\AgdaSymbol{)}\AgdaSpace{}%
\AgdaSymbol{(}\AgdaBound{ρ}\AgdaSpace{}%
\AgdaSymbol{:}\AgdaSpace{}%
\AgdaField{Carrier}\AgdaSymbol{(}\AgdaSpace{}%
\AgdaFunction{Env}\AgdaSpace{}%
\AgdaBound{X}\AgdaSpace{}%
\AgdaSymbol{)}\AgdaSpace{}%
\AgdaSymbol{)}\<%
\\
\>[1][@{}l@{\AgdaIndent{0}}]%
\>[2]\AgdaSymbol{→}%
\>[16]\AgdaOperator{\AgdaFunction{⟦}}\AgdaSpace{}%
\AgdaBound{t}\AgdaSpace{}%
\AgdaOperator{\AgdaFunction{[}}\AgdaSpace{}%
\AgdaBound{σ}\AgdaSpace{}%
\AgdaOperator{\AgdaFunction{]}}\AgdaSpace{}%
\AgdaOperator{\AgdaFunction{⟧}}\AgdaSpace{}%
\AgdaOperator{\AgdaField{⟨\$⟩}}\AgdaSpace{}%
\AgdaBound{ρ}%
\>[35]\AgdaOperator{\AgdaFunction{≈}}%
\>[38]\AgdaOperator{\AgdaFunction{⟦}}\AgdaSpace{}%
\AgdaBound{t}\AgdaSpace{}%
\AgdaOperator{\AgdaFunction{⟧}}\AgdaSpace{}%
\AgdaOperator{\AgdaField{⟨\$⟩}}\AgdaSpace{}%
\AgdaSymbol{(}\AgdaOperator{\AgdaFunction{⟦}}\AgdaSpace{}%
\AgdaBound{σ}\AgdaSpace{}%
\AgdaOperator{\AgdaFunction{⟧s}}\AgdaSpace{}%
\AgdaBound{ρ}\AgdaSymbol{)}\<%
\\
%
\\[\AgdaEmptyExtraSkip]%
%
\>[1]\AgdaFunction{substitution}\AgdaSpace{}%
\AgdaSymbol{(}\AgdaInductiveConstructor{ℊ}\AgdaSpace{}%
\AgdaBound{x}\AgdaSymbol{)}\AgdaSpace{}%
\AgdaBound{σ}\AgdaSpace{}%
\AgdaBound{ρ}\AgdaSpace{}%
\AgdaSymbol{=}\AgdaSpace{}%
\AgdaFunction{refl}\<%
\\
%
\>[1]\AgdaFunction{substitution}\AgdaSpace{}%
\AgdaSymbol{(}\AgdaInductiveConstructor{node}\AgdaSpace{}%
\AgdaBound{f}\AgdaSpace{}%
\AgdaBound{ts}\AgdaSymbol{)}\AgdaSpace{}%
\AgdaBound{σ}\AgdaSpace{}%
\AgdaBound{ρ}\AgdaSpace{}%
\AgdaSymbol{=}\AgdaSpace{}%
\AgdaField{cong}\AgdaSpace{}%
\AgdaSymbol{(}\AgdaField{Interp}\AgdaSpace{}%
\AgdaBound{𝑨}\AgdaSymbol{)(}\AgdaInductiveConstructor{≡.refl}\AgdaSpace{}%
\AgdaOperator{\AgdaInductiveConstructor{,}}\AgdaSpace{}%
\AgdaSymbol{λ}\AgdaSpace{}%
\AgdaBound{i}\AgdaSpace{}%
\AgdaSymbol{→}\AgdaSpace{}%
\AgdaFunction{substitution}\AgdaSpace{}%
\AgdaSymbol{(}\AgdaBound{ts}\AgdaSpace{}%
\AgdaBound{i}\AgdaSymbol{)}\AgdaSpace{}%
\AgdaBound{σ}\AgdaSpace{}%
\AgdaBound{ρ}\AgdaSymbol{)}\<%
\end{code}

\hypertarget{compatibility-of-terms}{%
\subsubsection{Compatibility of terms}\label{compatibility-of-terms}}

We now prove two important facts about term operations. The first of
these, which is used very often in the sequel, asserts that every term
commutes with every homomorphism.

\begin{code}%
\>[0]\<%
\\
\>[0]\AgdaKeyword{module}\AgdaSpace{}%
\AgdaModule{\AgdaUnderscore{}}\AgdaSpace{}%
\AgdaSymbol{\{}\AgdaBound{X}\AgdaSpace{}%
\AgdaSymbol{:}\AgdaSpace{}%
\AgdaPrimitive{Type}\AgdaSpace{}%
\AgdaGeneralizable{χ}\AgdaSymbol{\}\{}\AgdaBound{𝑨}\AgdaSpace{}%
\AgdaSymbol{:}\AgdaSpace{}%
\AgdaRecord{Algebra}\AgdaSpace{}%
\AgdaGeneralizable{α}\AgdaSpace{}%
\AgdaGeneralizable{ρᵃ}\AgdaSymbol{\}\{}\AgdaBound{𝑩}\AgdaSpace{}%
\AgdaSymbol{:}\AgdaSpace{}%
\AgdaRecord{Algebra}\AgdaSpace{}%
\AgdaGeneralizable{β}\AgdaSpace{}%
\AgdaGeneralizable{ρᵇ}\AgdaSymbol{\}(}\AgdaBound{hh}\AgdaSpace{}%
\AgdaSymbol{:}\AgdaSpace{}%
\AgdaFunction{hom}\AgdaSpace{}%
\AgdaBound{𝑨}\AgdaSpace{}%
\AgdaBound{𝑩}\AgdaSymbol{)}\AgdaSpace{}%
\AgdaKeyword{where}\<%
\\
\>[0][@{}l@{\AgdaIndent{0}}]%
\>[1]\AgdaKeyword{open}\AgdaSpace{}%
\AgdaModule{Setoid}\AgdaSpace{}%
\AgdaOperator{\AgdaFunction{𝔻[}}\AgdaSpace{}%
\AgdaBound{𝑩}\AgdaSpace{}%
\AgdaOperator{\AgdaFunction{]}}\AgdaSpace{}%
\AgdaKeyword{using}\AgdaSpace{}%
\AgdaSymbol{(}\AgdaSpace{}%
\AgdaOperator{\AgdaField{\AgdaUnderscore{}≈\AgdaUnderscore{}}}\AgdaSpace{}%
\AgdaSymbol{;}\AgdaSpace{}%
\AgdaFunction{refl}\AgdaSpace{}%
\AgdaSymbol{)}\<%
\\
%
\>[1]\AgdaKeyword{open}\AgdaSpace{}%
\AgdaModule{SetoidReasoning}\AgdaSpace{}%
\AgdaOperator{\AgdaFunction{𝔻[}}\AgdaSpace{}%
\AgdaBound{𝑩}\AgdaSpace{}%
\AgdaOperator{\AgdaFunction{]}}\<%
\\
%
\>[1]\AgdaKeyword{private}\AgdaSpace{}%
\AgdaFunction{hfunc}\AgdaSpace{}%
\AgdaSymbol{=}\AgdaSpace{}%
\AgdaOperator{\AgdaFunction{∣}}\AgdaSpace{}%
\AgdaBound{hh}\AgdaSpace{}%
\AgdaOperator{\AgdaFunction{∣}}\AgdaSpace{}%
\AgdaSymbol{;}\AgdaSpace{}%
\AgdaFunction{h}\AgdaSpace{}%
\AgdaSymbol{=}\AgdaSpace{}%
\AgdaOperator{\AgdaField{\AgdaUnderscore{}⟨\$⟩\AgdaUnderscore{}}}\AgdaSpace{}%
\AgdaFunction{hfunc}\<%
\\
%
\>[1]\AgdaKeyword{open}\AgdaSpace{}%
\AgdaModule{Environment}\AgdaSpace{}%
\AgdaBound{𝑨}\AgdaSpace{}%
\AgdaKeyword{using}\AgdaSpace{}%
\AgdaSymbol{(}\AgdaSpace{}%
\AgdaOperator{\AgdaFunction{⟦\AgdaUnderscore{}⟧}}\AgdaSpace{}%
\AgdaSymbol{)}\<%
\\
%
\>[1]\AgdaKeyword{open}\AgdaSpace{}%
\AgdaModule{Environment}\AgdaSpace{}%
\AgdaBound{𝑩}\AgdaSpace{}%
\AgdaKeyword{using}\AgdaSpace{}%
\AgdaSymbol{()}\AgdaSpace{}%
\AgdaKeyword{renaming}\AgdaSpace{}%
\AgdaSymbol{(}\AgdaSpace{}%
\AgdaOperator{\AgdaFunction{⟦\AgdaUnderscore{}⟧}}\AgdaSpace{}%
\AgdaSymbol{to}\AgdaSpace{}%
\AgdaOperator{\AgdaFunction{⟦\AgdaUnderscore{}⟧ᴮ}}\AgdaSpace{}%
\AgdaSymbol{)}\<%
\\
%
\\[\AgdaEmptyExtraSkip]%
%
\>[1]\AgdaFunction{comm-hom-term}\AgdaSpace{}%
\AgdaSymbol{:}\AgdaSpace{}%
\AgdaSymbol{(}\AgdaBound{t}\AgdaSpace{}%
\AgdaSymbol{:}\AgdaSpace{}%
\AgdaDatatype{Term}\AgdaSpace{}%
\AgdaBound{X}\AgdaSymbol{)}\AgdaSpace{}%
\AgdaSymbol{(}\AgdaBound{a}\AgdaSpace{}%
\AgdaSymbol{:}\AgdaSpace{}%
\AgdaBound{X}\AgdaSpace{}%
\AgdaSymbol{→}\AgdaSpace{}%
\AgdaOperator{\AgdaFunction{𝕌[}}\AgdaSpace{}%
\AgdaBound{𝑨}\AgdaSpace{}%
\AgdaOperator{\AgdaFunction{]}}\AgdaSymbol{)}\AgdaSpace{}%
\AgdaSymbol{→}\AgdaSpace{}%
\AgdaFunction{h}\AgdaSpace{}%
\AgdaSymbol{(}\AgdaOperator{\AgdaFunction{⟦}}\AgdaSpace{}%
\AgdaBound{t}\AgdaSpace{}%
\AgdaOperator{\AgdaFunction{⟧}}\AgdaSpace{}%
\AgdaOperator{\AgdaField{⟨\$⟩}}\AgdaSpace{}%
\AgdaBound{a}\AgdaSymbol{)}\AgdaSpace{}%
\AgdaOperator{\AgdaFunction{≈}}\AgdaSpace{}%
\AgdaOperator{\AgdaFunction{⟦}}\AgdaSpace{}%
\AgdaBound{t}\AgdaSpace{}%
\AgdaOperator{\AgdaFunction{⟧ᴮ}}\AgdaSpace{}%
\AgdaOperator{\AgdaField{⟨\$⟩}}\AgdaSpace{}%
\AgdaSymbol{(}\AgdaFunction{h}\AgdaSpace{}%
\AgdaOperator{\AgdaFunction{∘}}\AgdaSpace{}%
\AgdaBound{a}\AgdaSymbol{)}\<%
\\
%
\>[1]\AgdaFunction{comm-hom-term}\AgdaSpace{}%
\AgdaSymbol{(}\AgdaInductiveConstructor{ℊ}\AgdaSpace{}%
\AgdaBound{x}\AgdaSymbol{)}\AgdaSpace{}%
\AgdaBound{a}\AgdaSpace{}%
\AgdaSymbol{=}\AgdaSpace{}%
\AgdaFunction{refl}\<%
\\
%
\>[1]\AgdaFunction{comm-hom-term}\AgdaSpace{}%
\AgdaSymbol{(}\AgdaInductiveConstructor{node}\AgdaSpace{}%
\AgdaBound{f}\AgdaSpace{}%
\AgdaBound{t}\AgdaSymbol{)}\AgdaSpace{}%
\AgdaBound{a}\AgdaSpace{}%
\AgdaSymbol{=}\AgdaSpace{}%
\AgdaFunction{goal}\<%
\\
\>[1][@{}l@{\AgdaIndent{0}}]%
\>[2]\AgdaKeyword{where}\<%
\\
%
\>[2]\AgdaFunction{goal}\AgdaSpace{}%
\AgdaSymbol{:}\AgdaSpace{}%
\AgdaFunction{h}\AgdaSpace{}%
\AgdaSymbol{(}\AgdaOperator{\AgdaFunction{⟦}}\AgdaSpace{}%
\AgdaInductiveConstructor{node}\AgdaSpace{}%
\AgdaBound{f}\AgdaSpace{}%
\AgdaBound{t}\AgdaSpace{}%
\AgdaOperator{\AgdaFunction{⟧}}\AgdaSpace{}%
\AgdaOperator{\AgdaField{⟨\$⟩}}\AgdaSpace{}%
\AgdaBound{a}\AgdaSymbol{)}\AgdaSpace{}%
\AgdaOperator{\AgdaFunction{≈}}\AgdaSpace{}%
\AgdaSymbol{(}\AgdaOperator{\AgdaFunction{⟦}}\AgdaSpace{}%
\AgdaInductiveConstructor{node}\AgdaSpace{}%
\AgdaBound{f}\AgdaSpace{}%
\AgdaBound{t}\AgdaSpace{}%
\AgdaOperator{\AgdaFunction{⟧ᴮ}}\AgdaSpace{}%
\AgdaOperator{\AgdaField{⟨\$⟩}}\AgdaSpace{}%
\AgdaSymbol{(}\AgdaFunction{h}\AgdaSpace{}%
\AgdaOperator{\AgdaFunction{∘}}\AgdaSpace{}%
\AgdaBound{a}\AgdaSymbol{))}\<%
\\
%
\>[2]\AgdaFunction{goal}\AgdaSpace{}%
\AgdaSymbol{=}%
\>[4141I]\AgdaOperator{\AgdaFunction{begin}}\<%
\\
\>[4141I][@{}l@{\AgdaIndent{0}}]%
\>[10]\AgdaFunction{h}\AgdaSpace{}%
\AgdaSymbol{(}\AgdaOperator{\AgdaFunction{⟦}}\AgdaSpace{}%
\AgdaInductiveConstructor{node}\AgdaSpace{}%
\AgdaBound{f}\AgdaSpace{}%
\AgdaBound{t}\AgdaSpace{}%
\AgdaOperator{\AgdaFunction{⟧}}\AgdaSpace{}%
\AgdaOperator{\AgdaField{⟨\$⟩}}\AgdaSpace{}%
\AgdaBound{a}\AgdaSymbol{)}%
\>[46]\AgdaFunction{≈⟨}%
\>[50]\AgdaField{compatible}\AgdaSpace{}%
\AgdaOperator{\AgdaFunction{∥}}\AgdaSpace{}%
\AgdaBound{hh}\AgdaSpace{}%
\AgdaOperator{\AgdaFunction{∥}}%
\>[104]\AgdaFunction{⟩}\<%
\\
%
\>[10]\AgdaSymbol{(}\AgdaBound{f}\AgdaSpace{}%
\AgdaOperator{\AgdaFunction{̂}}\AgdaSpace{}%
\AgdaBound{𝑩}\AgdaSymbol{)(λ}\AgdaSpace{}%
\AgdaBound{i}\AgdaSpace{}%
\AgdaSymbol{→}\AgdaSpace{}%
\AgdaFunction{h}\AgdaSpace{}%
\AgdaSymbol{(}\AgdaOperator{\AgdaFunction{⟦}}\AgdaSpace{}%
\AgdaBound{t}\AgdaSpace{}%
\AgdaBound{i}\AgdaSpace{}%
\AgdaOperator{\AgdaFunction{⟧}}\AgdaSpace{}%
\AgdaOperator{\AgdaField{⟨\$⟩}}\AgdaSpace{}%
\AgdaBound{a}\AgdaSymbol{))}%
\>[47]\AgdaFunction{≈⟨}%
\>[51]\AgdaField{cong}\AgdaSymbol{(}\AgdaField{Interp}\AgdaSpace{}%
\AgdaBound{𝑩}\AgdaSymbol{)(}\AgdaInductiveConstructor{≡.refl}\AgdaSpace{}%
\AgdaOperator{\AgdaInductiveConstructor{,}}\AgdaSpace{}%
\AgdaSymbol{λ}\AgdaSpace{}%
\AgdaBound{i}\AgdaSpace{}%
\AgdaSymbol{→}\AgdaSpace{}%
\AgdaFunction{comm-hom-term}\AgdaSpace{}%
\AgdaSymbol{(}\AgdaBound{t}\AgdaSpace{}%
\AgdaBound{i}\AgdaSymbol{)}\AgdaSpace{}%
\AgdaBound{a}\AgdaSymbol{)}%
\>[105]\AgdaFunction{⟩}\<%
\\
%
\>[10]\AgdaSymbol{(}\AgdaBound{f}\AgdaSpace{}%
\AgdaOperator{\AgdaFunction{̂}}\AgdaSpace{}%
\AgdaBound{𝑩}\AgdaSymbol{)(λ}\AgdaSpace{}%
\AgdaBound{i}\AgdaSpace{}%
\AgdaSymbol{→}\AgdaSpace{}%
\AgdaOperator{\AgdaFunction{⟦}}\AgdaSpace{}%
\AgdaBound{t}\AgdaSpace{}%
\AgdaBound{i}\AgdaSpace{}%
\AgdaOperator{\AgdaFunction{⟧ᴮ}}\AgdaSpace{}%
\AgdaOperator{\AgdaField{⟨\$⟩}}\AgdaSpace{}%
\AgdaSymbol{(}\AgdaFunction{h}\AgdaSpace{}%
\AgdaOperator{\AgdaFunction{∘}}\AgdaSpace{}%
\AgdaBound{a}\AgdaSymbol{))}%
\>[47]\AgdaFunction{≈⟨}%
\>[51]\AgdaFunction{refl}%
\>[105]\AgdaFunction{⟩}\<%
\\
%
\>[10]\AgdaSymbol{(}\AgdaOperator{\AgdaFunction{⟦}}\AgdaSpace{}%
\AgdaInductiveConstructor{node}\AgdaSpace{}%
\AgdaBound{f}\AgdaSpace{}%
\AgdaBound{t}\AgdaSpace{}%
\AgdaOperator{\AgdaFunction{⟧ᴮ}}\AgdaSpace{}%
\AgdaOperator{\AgdaField{⟨\$⟩}}\AgdaSpace{}%
\AgdaSymbol{(}\AgdaFunction{h}\AgdaSpace{}%
\AgdaOperator{\AgdaFunction{∘}}\AgdaSpace{}%
\AgdaBound{a}\AgdaSymbol{))}%
\>[46]\AgdaOperator{\AgdaFunction{∎}}\<%
\end{code}

\begin{code}[hide]%
\>[0]\AgdaKeyword{module}\AgdaSpace{}%
\AgdaModule{\AgdaUnderscore{}}\AgdaSpace{}%
\AgdaSymbol{\{}\AgdaBound{X}\AgdaSpace{}%
\AgdaSymbol{:}\AgdaSpace{}%
\AgdaPrimitive{Type}\AgdaSpace{}%
\AgdaGeneralizable{χ}\AgdaSymbol{\}\{}\AgdaBound{ι}\AgdaSpace{}%
\AgdaSymbol{:}\AgdaSpace{}%
\AgdaPostulate{Level}\AgdaSymbol{\}}\AgdaSpace{}%
\AgdaSymbol{\{}\AgdaBound{I}\AgdaSpace{}%
\AgdaSymbol{:}\AgdaSpace{}%
\AgdaPrimitive{Type}\AgdaSpace{}%
\AgdaBound{ι}\AgdaSymbol{\}}\AgdaSpace{}%
\AgdaSymbol{(}\AgdaBound{𝒜}\AgdaSpace{}%
\AgdaSymbol{:}\AgdaSpace{}%
\AgdaBound{I}\AgdaSpace{}%
\AgdaSymbol{→}\AgdaSpace{}%
\AgdaRecord{Algebra}\AgdaSpace{}%
\AgdaGeneralizable{α}\AgdaSpace{}%
\AgdaGeneralizable{ρᵃ}\AgdaSymbol{)}\AgdaSpace{}%
\AgdaKeyword{where}\<%
\\
\>[0][@{}l@{\AgdaIndent{0}}]%
\>[1]\AgdaKeyword{open}\AgdaSpace{}%
\AgdaModule{Setoid}\AgdaSpace{}%
\AgdaOperator{\AgdaFunction{𝔻[}}\AgdaSpace{}%
\AgdaFunction{⨅}\AgdaSpace{}%
\AgdaBound{𝒜}\AgdaSpace{}%
\AgdaOperator{\AgdaFunction{]}}\AgdaSpace{}%
\AgdaKeyword{using}\AgdaSpace{}%
\AgdaSymbol{(}\AgdaSpace{}%
\AgdaOperator{\AgdaField{\AgdaUnderscore{}≈\AgdaUnderscore{}}}\AgdaSpace{}%
\AgdaSymbol{)}\<%
\\
%
\>[1]\AgdaKeyword{open}\AgdaSpace{}%
\AgdaModule{Environment}\AgdaSpace{}%
\AgdaSymbol{(}\AgdaFunction{⨅}\AgdaSpace{}%
\AgdaBound{𝒜}\AgdaSymbol{)}\AgdaSpace{}%
\AgdaKeyword{using}\AgdaSpace{}%
\AgdaSymbol{()}\AgdaSpace{}%
\AgdaKeyword{renaming}\AgdaSpace{}%
\AgdaSymbol{(}\AgdaSpace{}%
\AgdaOperator{\AgdaFunction{⟦\AgdaUnderscore{}⟧}}\AgdaSpace{}%
\AgdaSymbol{to}\AgdaSpace{}%
\AgdaOperator{\AgdaFunction{⟦\AgdaUnderscore{}⟧₁}}\AgdaSpace{}%
\AgdaSymbol{)}\<%
\\
%
\>[1]\AgdaKeyword{open}\AgdaSpace{}%
\AgdaModule{Environment}\AgdaSpace{}%
\AgdaKeyword{using}\AgdaSpace{}%
\AgdaSymbol{(}\AgdaSpace{}%
\AgdaOperator{\AgdaFunction{⟦\AgdaUnderscore{}⟧}}\AgdaSpace{}%
\AgdaSymbol{;}\AgdaSpace{}%
\AgdaFunction{≐→Equal}\AgdaSpace{}%
\AgdaSymbol{)}\<%
\\
%
\\[\AgdaEmptyExtraSkip]%
%
\>[1]\AgdaFunction{interp-prod}\AgdaSpace{}%
\AgdaSymbol{:}\AgdaSpace{}%
\AgdaSymbol{(}\AgdaBound{p}\AgdaSpace{}%
\AgdaSymbol{:}\AgdaSpace{}%
\AgdaDatatype{Term}\AgdaSpace{}%
\AgdaBound{X}\AgdaSymbol{)}\AgdaSpace{}%
\AgdaSymbol{→}\AgdaSpace{}%
\AgdaSymbol{∀}\AgdaSpace{}%
\AgdaBound{ρ}\AgdaSpace{}%
\AgdaSymbol{→}\AgdaSpace{}%
\AgdaOperator{\AgdaFunction{⟦}}\AgdaSpace{}%
\AgdaBound{p}\AgdaSpace{}%
\AgdaOperator{\AgdaFunction{⟧₁}}\AgdaSpace{}%
\AgdaOperator{\AgdaField{⟨\$⟩}}\AgdaSpace{}%
\AgdaBound{ρ}\AgdaSpace{}%
\AgdaOperator{\AgdaFunction{≈}}\AgdaSpace{}%
\AgdaSymbol{(λ}\AgdaSpace{}%
\AgdaBound{i}\AgdaSpace{}%
\AgdaSymbol{→}\AgdaSpace{}%
\AgdaSymbol{(}\AgdaOperator{\AgdaFunction{⟦}}\AgdaSpace{}%
\AgdaBound{𝒜}\AgdaSpace{}%
\AgdaBound{i}\AgdaSpace{}%
\AgdaOperator{\AgdaFunction{⟧}}\AgdaSpace{}%
\AgdaBound{p}\AgdaSymbol{)}\AgdaSpace{}%
\AgdaOperator{\AgdaField{⟨\$⟩}}\AgdaSpace{}%
\AgdaSymbol{(λ}\AgdaSpace{}%
\AgdaBound{x}\AgdaSpace{}%
\AgdaSymbol{→}\AgdaSpace{}%
\AgdaSymbol{(}\AgdaBound{ρ}\AgdaSpace{}%
\AgdaBound{x}\AgdaSymbol{)}\AgdaSpace{}%
\AgdaBound{i}\AgdaSymbol{))}\<%
\\
%
\>[1]\AgdaFunction{interp-prod}\AgdaSpace{}%
\AgdaSymbol{(}\AgdaInductiveConstructor{ℊ}\AgdaSpace{}%
\AgdaBound{x}\AgdaSymbol{)}\AgdaSpace{}%
\AgdaSymbol{=}\AgdaSpace{}%
\AgdaSymbol{λ}\AgdaSpace{}%
\AgdaBound{ρ}\AgdaSpace{}%
\AgdaBound{i}\AgdaSpace{}%
\AgdaSymbol{→}\AgdaSpace{}%
\AgdaFunction{≐→Equal}\AgdaSpace{}%
\AgdaSymbol{(}\AgdaBound{𝒜}\AgdaSpace{}%
\AgdaBound{i}\AgdaSymbol{)}\AgdaSpace{}%
\AgdaSymbol{(}\AgdaInductiveConstructor{ℊ}\AgdaSpace{}%
\AgdaBound{x}\AgdaSymbol{)}\AgdaSpace{}%
\AgdaSymbol{(}\AgdaInductiveConstructor{ℊ}\AgdaSpace{}%
\AgdaBound{x}\AgdaSymbol{)}\AgdaSpace{}%
\AgdaFunction{≐-isRefl}\AgdaSpace{}%
\AgdaSymbol{λ}\AgdaSpace{}%
\AgdaBound{x'}\AgdaSpace{}%
\AgdaSymbol{→}\AgdaSpace{}%
\AgdaSymbol{(}\AgdaBound{ρ}\AgdaSpace{}%
\AgdaBound{x}\AgdaSymbol{)}\AgdaSpace{}%
\AgdaBound{i}\<%
\\
%
\>[1]\AgdaFunction{interp-prod}\AgdaSpace{}%
\AgdaSymbol{(}\AgdaInductiveConstructor{node}\AgdaSpace{}%
\AgdaBound{f}\AgdaSpace{}%
\AgdaBound{t}\AgdaSymbol{)}\AgdaSpace{}%
\AgdaSymbol{=}\AgdaSpace{}%
\AgdaSymbol{λ}\AgdaSpace{}%
\AgdaBound{ρ}\AgdaSpace{}%
\AgdaBound{i}\AgdaSpace{}%
\AgdaSymbol{→}\AgdaSpace{}%
\AgdaField{cong}\AgdaSpace{}%
\AgdaSymbol{(}\AgdaField{Interp}\AgdaSpace{}%
\AgdaSymbol{(}\AgdaFunction{⨅}\AgdaSpace{}%
\AgdaBound{𝒜}\AgdaSymbol{))(}\AgdaInductiveConstructor{≡.refl}\AgdaSpace{}%
\AgdaOperator{\AgdaInductiveConstructor{,}}\AgdaSpace{}%
\AgdaSymbol{(λ}\AgdaSpace{}%
\AgdaBound{j}\AgdaSpace{}%
\AgdaBound{k}\AgdaSpace{}%
\AgdaSymbol{→}\AgdaSpace{}%
\AgdaFunction{interp-prod}\AgdaSpace{}%
\AgdaSymbol{(}\AgdaBound{t}\AgdaSpace{}%
\AgdaBound{j}\AgdaSymbol{)}\AgdaSpace{}%
\AgdaBound{ρ}\AgdaSpace{}%
\AgdaBound{k}\AgdaSymbol{))}\AgdaSpace{}%
\AgdaBound{i}\<%
\end{code}

\hypertarget{model-theory-and-equational-logic}{%
\subsection{Model Theory and Equational
Logic}\label{model-theory-and-equational-logic}}

(cf.~the
\href{https://ualib.github.io/agda-algebras/Varieties.Func.SoundAndComplete.html}{Varieties.Func.SoundAndComplete}
module of the \href{https://ualib.github.io/agda-algebras}{Agda
Universal Algebra Library})

\hypertarget{basic-definitions-3}{%
\subsubsection{Basic definitions}\label{basic-definitions-3}}

Let \texttt{𝑆} be a signature. By an \emph{identity} or \emph{equation}
in \texttt{𝑆} we mean an ordered pair of terms in a given context. For
instance, if the context happens to be the type \texttt{X\ :\ Type\ χ},
then an equation will be a pair of inhabitants of the domain of term
algebra \texttt{𝑻\ X}.

We define an equation in Agda using the following record type with
fields denoting the left-hand and right-hand sides of the equation,
along with an equation ``context'' representing the underlying
collection of variable symbols
(cf.~\href{http://www.cse.chalmers.se/~abela/agda/MultiSortedAlgebra.pdf}{Andreas
Abel's formal proof of Birkhoff's completeness theorem}).

\begin{code}%
\>[0]\<%
\\
\>[0]\AgdaKeyword{record}\AgdaSpace{}%
\AgdaRecord{Eq}\AgdaSpace{}%
\AgdaSymbol{:}\AgdaSpace{}%
\AgdaPrimitive{Type}\AgdaSpace{}%
\AgdaSymbol{(}\AgdaFunction{ov}\AgdaSpace{}%
\AgdaBound{χ}\AgdaSymbol{)}\AgdaSpace{}%
\AgdaKeyword{where}\<%
\\
\>[0][@{}l@{\AgdaIndent{0}}]%
\>[1]\AgdaKeyword{constructor}\AgdaSpace{}%
\AgdaOperator{\AgdaInductiveConstructor{\AgdaUnderscore{}≈̇\AgdaUnderscore{}}}\<%
\\
%
\>[1]\AgdaKeyword{field}%
\>[8]\AgdaSymbol{\{}\AgdaField{cxt}\AgdaSymbol{\}}%
\>[15]\AgdaSymbol{:}\AgdaSpace{}%
\AgdaPrimitive{Type}\AgdaSpace{}%
\AgdaBound{χ}\<%
\\
%
\>[8]\AgdaField{lhs}%
\>[15]\AgdaSymbol{:}\AgdaSpace{}%
\AgdaDatatype{Term}\AgdaSpace{}%
\AgdaField{cxt}\<%
\\
%
\>[8]\AgdaField{rhs}%
\>[15]\AgdaSymbol{:}\AgdaSpace{}%
\AgdaDatatype{Term}\AgdaSpace{}%
\AgdaField{cxt}\<%
\\
%
\\[\AgdaEmptyExtraSkip]%
\>[0]\AgdaKeyword{open}\AgdaSpace{}%
\AgdaModule{Eq}\AgdaSpace{}%
\AgdaKeyword{public}\<%
\\
\>[0]\<%
\end{code}

We now define a type representing the notion of an equation
\texttt{p\ ≈̇\ q} holding (when \texttt{p} and \texttt{q} are
interpreted) in algebra \texttt{𝑨}.

If \texttt{𝑨} is an \texttt{𝑆}-algebra we say that \texttt{𝑨}
\emph{satisfies} \texttt{p\ ≈\ q} provided for all environments
\texttt{ρ\ :\ X\ →\ ∣\ 𝑨\ ∣} (assigning values in the domain of
\texttt{𝑨} to variable symbols in \texttt{X}) we have
\texttt{⟦\ p\ ⟧⟨\$⟩\ ρ\ ≈\ ⟦\ q\ ⟧\ ⟨\$⟩\ ρ}. In this situation, we
write \texttt{𝑨\ ⊧\ (p\ ≈̇\ q)} and say that \texttt{𝑨} \emph{models} the
identity \texttt{p\ ≈\ q}.

If \texttt{𝒦} is a class of algebras, all of the same signature, we
write \texttt{𝒦\ ⊫\ (p\ ≈̇\ q)\ if,\ for\ every}𝑨 ∈
𝒦\texttt{,\ we\ have}𝑨 ⊧ (p ≈̇ q)`.

Because a class of structures has a different type than a single
structure, we must use a slightly different syntax to avoid overloading
the relations \texttt{⊧} and \texttt{≈}. As a reasonable alternative to
what we would normally express informally as \texttt{𝒦\ ⊧\ p\ ≈\ q}, we
have settled on \texttt{𝒦\ ⊫\ (p\ ≈̇\ q)} to denote this relation. To
reiterate, if \texttt{𝒦} is a class of \texttt{𝑆}-algebras, we write
\texttt{𝒦\ ⊫\ (p\ ≈̇\ q)} provided every \texttt{𝑨\ ∈\ 𝒦} satisfies
\texttt{𝑨\ ⊧\ (p\ ≈̇\ q)}.

\begin{code}%
\>[0]\<%
\\
\>[0]\AgdaOperator{\AgdaFunction{\AgdaUnderscore{}⊧\AgdaUnderscore{}}}\AgdaSpace{}%
\AgdaSymbol{:}\AgdaSpace{}%
\AgdaSymbol{(}\AgdaBound{𝑨}\AgdaSpace{}%
\AgdaSymbol{:}\AgdaSpace{}%
\AgdaRecord{Algebra}\AgdaSpace{}%
\AgdaGeneralizable{α}\AgdaSpace{}%
\AgdaGeneralizable{ρᵃ}\AgdaSymbol{)(}\AgdaBound{term-identity}\AgdaSpace{}%
\AgdaSymbol{:}\AgdaSpace{}%
\AgdaRecord{Eq}\AgdaSymbol{\{}\AgdaGeneralizable{χ}\AgdaSymbol{\})}\AgdaSpace{}%
\AgdaSymbol{→}\AgdaSpace{}%
\AgdaPrimitive{Type}\AgdaSpace{}%
\AgdaSymbol{\AgdaUnderscore{}}\<%
\\
\>[0]\AgdaBound{𝑨}\AgdaSpace{}%
\AgdaOperator{\AgdaFunction{⊧}}\AgdaSpace{}%
\AgdaSymbol{(}\AgdaBound{p}\AgdaSpace{}%
\AgdaOperator{\AgdaInductiveConstructor{≈̇}}\AgdaSpace{}%
\AgdaBound{q}\AgdaSymbol{)}\AgdaSpace{}%
\AgdaSymbol{=}\AgdaSpace{}%
\AgdaFunction{Equal}\AgdaSpace{}%
\AgdaBound{p}\AgdaSpace{}%
\AgdaBound{q}\AgdaSpace{}%
\AgdaKeyword{where}\AgdaSpace{}%
\AgdaKeyword{open}\AgdaSpace{}%
\AgdaModule{Environment}\AgdaSpace{}%
\AgdaBound{𝑨}\<%
\\
%
\\[\AgdaEmptyExtraSkip]%
\>[0]\AgdaOperator{\AgdaFunction{\AgdaUnderscore{}⊫\AgdaUnderscore{}}}\AgdaSpace{}%
\AgdaSymbol{:}\AgdaSpace{}%
\AgdaFunction{Pred}\AgdaSpace{}%
\AgdaSymbol{(}\AgdaRecord{Algebra}\AgdaSpace{}%
\AgdaGeneralizable{α}\AgdaSpace{}%
\AgdaGeneralizable{ρᵃ}\AgdaSymbol{)}\AgdaSpace{}%
\AgdaGeneralizable{ℓ}\AgdaSpace{}%
\AgdaSymbol{→}\AgdaSpace{}%
\AgdaRecord{Eq}\AgdaSymbol{\{}\AgdaGeneralizable{χ}\AgdaSymbol{\}}\AgdaSpace{}%
\AgdaSymbol{→}\AgdaSpace{}%
\AgdaPrimitive{Type}\AgdaSpace{}%
\AgdaSymbol{(}\AgdaGeneralizable{ℓ}\AgdaSpace{}%
\AgdaOperator{\AgdaPrimitive{⊔}}\AgdaSpace{}%
\AgdaGeneralizable{χ}\AgdaSpace{}%
\AgdaOperator{\AgdaPrimitive{⊔}}\AgdaSpace{}%
\AgdaFunction{ov}\AgdaSymbol{(}\AgdaGeneralizable{α}\AgdaSpace{}%
\AgdaOperator{\AgdaPrimitive{⊔}}\AgdaSpace{}%
\AgdaGeneralizable{ρᵃ}\AgdaSymbol{))}\<%
\\
\>[0]\AgdaBound{𝒦}\AgdaSpace{}%
\AgdaOperator{\AgdaFunction{⊫}}\AgdaSpace{}%
\AgdaBound{equ}\AgdaSpace{}%
\AgdaSymbol{=}\AgdaSpace{}%
\AgdaSymbol{∀}\AgdaSpace{}%
\AgdaBound{𝑨}\AgdaSpace{}%
\AgdaSymbol{→}\AgdaSpace{}%
\AgdaBound{𝒦}\AgdaSpace{}%
\AgdaBound{𝑨}\AgdaSpace{}%
\AgdaSymbol{→}\AgdaSpace{}%
\AgdaBound{𝑨}\AgdaSpace{}%
\AgdaOperator{\AgdaFunction{⊧}}\AgdaSpace{}%
\AgdaBound{equ}\<%
\\
\>[0]\<%
\end{code}

We denote by \texttt{𝑨\ ⊨\ ℰ} the assertion that the algebra 𝑨 models
every equation in a collection \texttt{ℰ} of equations.

\begin{code}%
\>[0]\<%
\\
\>[0]\AgdaOperator{\AgdaFunction{\AgdaUnderscore{}⊨\AgdaUnderscore{}}}\AgdaSpace{}%
\AgdaSymbol{:}\AgdaSpace{}%
\AgdaSymbol{(}\AgdaBound{𝑨}\AgdaSpace{}%
\AgdaSymbol{:}\AgdaSpace{}%
\AgdaRecord{Algebra}\AgdaSpace{}%
\AgdaGeneralizable{α}\AgdaSpace{}%
\AgdaGeneralizable{ρᵃ}\AgdaSymbol{)}\AgdaSpace{}%
\AgdaSymbol{→}\AgdaSpace{}%
\AgdaSymbol{\{}\AgdaBound{ι}\AgdaSpace{}%
\AgdaSymbol{:}\AgdaSpace{}%
\AgdaPostulate{Level}\AgdaSymbol{\}\{}\AgdaBound{I}\AgdaSpace{}%
\AgdaSymbol{:}\AgdaSpace{}%
\AgdaPrimitive{Type}\AgdaSpace{}%
\AgdaBound{ι}\AgdaSymbol{\}}\AgdaSpace{}%
\AgdaSymbol{→}\AgdaSpace{}%
\AgdaSymbol{(}\AgdaBound{I}\AgdaSpace{}%
\AgdaSymbol{→}\AgdaSpace{}%
\AgdaRecord{Eq}\AgdaSymbol{\{}\AgdaGeneralizable{χ}\AgdaSymbol{\})}\AgdaSpace{}%
\AgdaSymbol{→}\AgdaSpace{}%
\AgdaPrimitive{Type}\AgdaSpace{}%
\AgdaSymbol{\AgdaUnderscore{}}\<%
\\
\>[0]\AgdaBound{𝑨}\AgdaSpace{}%
\AgdaOperator{\AgdaFunction{⊨}}\AgdaSpace{}%
\AgdaBound{ℰ}\AgdaSpace{}%
\AgdaSymbol{=}\AgdaSpace{}%
\AgdaSymbol{∀}\AgdaSpace{}%
\AgdaBound{i}\AgdaSpace{}%
\AgdaSymbol{→}\AgdaSpace{}%
\AgdaFunction{Equal}\AgdaSpace{}%
\AgdaSymbol{(}\AgdaField{lhs}\AgdaSpace{}%
\AgdaSymbol{(}\AgdaBound{ℰ}\AgdaSpace{}%
\AgdaBound{i}\AgdaSymbol{))(}\AgdaField{rhs}\AgdaSpace{}%
\AgdaSymbol{(}\AgdaBound{ℰ}\AgdaSpace{}%
\AgdaBound{i}\AgdaSymbol{))}\AgdaSpace{}%
\AgdaKeyword{where}\AgdaSpace{}%
\AgdaKeyword{open}\AgdaSpace{}%
\AgdaModule{Environment}\AgdaSpace{}%
\AgdaBound{𝑨}\<%
\end{code}

\hypertarget{equational-theories-and-models}{%
\subsubsection{Equational theories and
models}\label{equational-theories-and-models}}

If \texttt{𝒦} denotes a class of structures, then \texttt{Th\ 𝒦}
represents the set of identities modeled by the members of \texttt{𝒦}.

\begin{code}%
\>[0]\<%
\\
\>[0]\AgdaFunction{Th}\AgdaSpace{}%
\AgdaSymbol{:}\AgdaSpace{}%
\AgdaSymbol{\{}\AgdaBound{X}\AgdaSpace{}%
\AgdaSymbol{:}\AgdaSpace{}%
\AgdaPrimitive{Type}\AgdaSpace{}%
\AgdaGeneralizable{χ}\AgdaSymbol{\}}\AgdaSpace{}%
\AgdaSymbol{→}\AgdaSpace{}%
\AgdaFunction{Pred}\AgdaSpace{}%
\AgdaSymbol{(}\AgdaRecord{Algebra}\AgdaSpace{}%
\AgdaGeneralizable{α}\AgdaSpace{}%
\AgdaGeneralizable{ρᵃ}\AgdaSymbol{)}\AgdaSpace{}%
\AgdaGeneralizable{ℓ}\AgdaSpace{}%
\AgdaSymbol{→}\AgdaSpace{}%
\AgdaFunction{Pred}\AgdaSymbol{(}\AgdaDatatype{Term}\AgdaSpace{}%
\AgdaBound{X}\AgdaSpace{}%
\AgdaOperator{\AgdaFunction{×}}\AgdaSpace{}%
\AgdaDatatype{Term}\AgdaSpace{}%
\AgdaBound{X}\AgdaSymbol{)}\AgdaSpace{}%
\AgdaSymbol{\AgdaUnderscore{}}\<%
\\
\>[0]\AgdaFunction{Th}\AgdaSpace{}%
\AgdaBound{𝒦}\AgdaSpace{}%
\AgdaSymbol{=}\AgdaSpace{}%
\AgdaSymbol{λ}\AgdaSpace{}%
\AgdaSymbol{(}\AgdaBound{p}\AgdaSpace{}%
\AgdaOperator{\AgdaInductiveConstructor{,}}\AgdaSpace{}%
\AgdaBound{q}\AgdaSymbol{)}\AgdaSpace{}%
\AgdaSymbol{→}\AgdaSpace{}%
\AgdaBound{𝒦}\AgdaSpace{}%
\AgdaOperator{\AgdaFunction{⊫}}\AgdaSpace{}%
\AgdaSymbol{(}\AgdaBound{p}\AgdaSpace{}%
\AgdaOperator{\AgdaInductiveConstructor{≈̇}}\AgdaSpace{}%
\AgdaBound{q}\AgdaSymbol{)}\<%
\\
%
\\[\AgdaEmptyExtraSkip]%
\>[0]\AgdaFunction{Mod}\AgdaSpace{}%
\AgdaSymbol{:}\AgdaSpace{}%
\AgdaSymbol{\{}\AgdaBound{X}\AgdaSpace{}%
\AgdaSymbol{:}\AgdaSpace{}%
\AgdaPrimitive{Type}\AgdaSpace{}%
\AgdaGeneralizable{χ}\AgdaSymbol{\}}\AgdaSpace{}%
\AgdaSymbol{→}\AgdaSpace{}%
\AgdaFunction{Pred}\AgdaSymbol{(}\AgdaDatatype{Term}\AgdaSpace{}%
\AgdaBound{X}\AgdaSpace{}%
\AgdaOperator{\AgdaFunction{×}}\AgdaSpace{}%
\AgdaDatatype{Term}\AgdaSpace{}%
\AgdaBound{X}\AgdaSymbol{)}\AgdaSpace{}%
\AgdaGeneralizable{ℓ}\AgdaSpace{}%
\AgdaSymbol{→}\AgdaSpace{}%
\AgdaFunction{Pred}\AgdaSpace{}%
\AgdaSymbol{(}\AgdaRecord{Algebra}\AgdaSpace{}%
\AgdaGeneralizable{α}\AgdaSpace{}%
\AgdaGeneralizable{ρᵃ}\AgdaSymbol{)}\AgdaSpace{}%
\AgdaSymbol{\AgdaUnderscore{}}\<%
\\
\>[0]\AgdaFunction{Mod}\AgdaSpace{}%
\AgdaBound{ℰ}\AgdaSpace{}%
\AgdaBound{𝑨}\AgdaSpace{}%
\AgdaSymbol{=}\AgdaSpace{}%
\AgdaSymbol{∀}\AgdaSpace{}%
\AgdaSymbol{\{}\AgdaBound{p}\AgdaSpace{}%
\AgdaBound{q}\AgdaSymbol{\}}\AgdaSpace{}%
\AgdaSymbol{→}\AgdaSpace{}%
\AgdaSymbol{(}\AgdaBound{p}\AgdaSpace{}%
\AgdaOperator{\AgdaInductiveConstructor{,}}\AgdaSpace{}%
\AgdaBound{q}\AgdaSymbol{)}\AgdaSpace{}%
\AgdaOperator{\AgdaFunction{∈}}\AgdaSpace{}%
\AgdaBound{ℰ}\AgdaSpace{}%
\AgdaSymbol{→}\AgdaSpace{}%
\AgdaFunction{Equal}\AgdaSpace{}%
\AgdaBound{p}\AgdaSpace{}%
\AgdaBound{q}\AgdaSpace{}%
\AgdaKeyword{where}\AgdaSpace{}%
\AgdaKeyword{open}\AgdaSpace{}%
\AgdaModule{Environment}\AgdaSpace{}%
\AgdaBound{𝑨}\<%
\end{code}

\hypertarget{the-entailment-relation}{%
\subsubsection{The entailment relation}\label{the-entailment-relation}}

Based on
\href{http://www.cse.chalmers.se/~abela/agda/MultiSortedAlgebra.pdf}{Andreas
Abel's Agda formalization of Birkhoff's completeness theorem}.)

\begin{code}%
\>[0]\<%
\\
\>[0]\AgdaKeyword{module}\AgdaSpace{}%
\AgdaModule{\AgdaUnderscore{}}\AgdaSpace{}%
\AgdaSymbol{\{}\AgdaBound{χ}\AgdaSpace{}%
\AgdaBound{ι}\AgdaSpace{}%
\AgdaSymbol{:}\AgdaSpace{}%
\AgdaPostulate{Level}\AgdaSymbol{\}}\AgdaSpace{}%
\AgdaKeyword{where}\<%
\\
%
\\[\AgdaEmptyExtraSkip]%
\>[0][@{}l@{\AgdaIndent{0}}]%
\>[1]\AgdaKeyword{data}\AgdaSpace{}%
\AgdaOperator{\AgdaDatatype{\AgdaUnderscore{}⊢\AgdaUnderscore{}▹\AgdaUnderscore{}≈\AgdaUnderscore{}}}\AgdaSpace{}%
\AgdaSymbol{\{}\AgdaBound{I}\AgdaSpace{}%
\AgdaSymbol{:}\AgdaSpace{}%
\AgdaPrimitive{Type}\AgdaSpace{}%
\AgdaBound{ι}\AgdaSymbol{\}(}\AgdaBound{ℰ}\AgdaSpace{}%
\AgdaSymbol{:}\AgdaSpace{}%
\AgdaBound{I}\AgdaSpace{}%
\AgdaSymbol{→}\AgdaSpace{}%
\AgdaRecord{Eq}\AgdaSymbol{)}\AgdaSpace{}%
\AgdaSymbol{:}\AgdaSpace{}%
\AgdaSymbol{(}\AgdaBound{X}\AgdaSpace{}%
\AgdaSymbol{:}\AgdaSpace{}%
\AgdaPrimitive{Type}\AgdaSpace{}%
\AgdaBound{χ}\AgdaSymbol{)(}\AgdaBound{p}\AgdaSpace{}%
\AgdaBound{q}\AgdaSpace{}%
\AgdaSymbol{:}\AgdaSpace{}%
\AgdaDatatype{Term}\AgdaSpace{}%
\AgdaBound{X}\AgdaSymbol{)}\AgdaSpace{}%
\AgdaSymbol{→}\AgdaSpace{}%
\AgdaPrimitive{Type}\AgdaSpace{}%
\AgdaSymbol{(}\AgdaBound{ι}\AgdaSpace{}%
\AgdaOperator{\AgdaPrimitive{⊔}}\AgdaSpace{}%
\AgdaFunction{ov}\AgdaSpace{}%
\AgdaBound{χ}\AgdaSymbol{)}\AgdaSpace{}%
\AgdaKeyword{where}\<%
\\
\>[1][@{}l@{\AgdaIndent{0}}]%
\>[2]\AgdaInductiveConstructor{hyp}\AgdaSpace{}%
\AgdaSymbol{:}\AgdaSpace{}%
\AgdaSymbol{∀}\AgdaSpace{}%
\AgdaBound{i}\AgdaSpace{}%
\AgdaSymbol{→}\AgdaSpace{}%
\AgdaKeyword{let}\AgdaSpace{}%
\AgdaBound{p}\AgdaSpace{}%
\AgdaOperator{\AgdaInductiveConstructor{≈̇}}\AgdaSpace{}%
\AgdaBound{q}\AgdaSpace{}%
\AgdaSymbol{=}\AgdaSpace{}%
\AgdaBound{ℰ}\AgdaSpace{}%
\AgdaBound{i}\AgdaSpace{}%
\AgdaKeyword{in}\AgdaSpace{}%
\AgdaBound{ℰ}\AgdaSpace{}%
\AgdaOperator{\AgdaDatatype{⊢}}\AgdaSpace{}%
\AgdaSymbol{\AgdaUnderscore{}}\AgdaSpace{}%
\AgdaOperator{\AgdaDatatype{▹}}\AgdaSpace{}%
\AgdaBound{p}\AgdaSpace{}%
\AgdaOperator{\AgdaDatatype{≈}}\AgdaSpace{}%
\AgdaBound{q}\<%
\\
%
\>[2]\AgdaInductiveConstructor{app}\AgdaSpace{}%
\AgdaSymbol{:}\AgdaSpace{}%
\AgdaSymbol{∀}\AgdaSpace{}%
\AgdaSymbol{\{}\AgdaBound{ps}\AgdaSpace{}%
\AgdaBound{qs}\AgdaSymbol{\}}\AgdaSpace{}%
\AgdaSymbol{→}\AgdaSpace{}%
\AgdaSymbol{(∀}\AgdaSpace{}%
\AgdaBound{i}\AgdaSpace{}%
\AgdaSymbol{→}\AgdaSpace{}%
\AgdaBound{ℰ}\AgdaSpace{}%
\AgdaOperator{\AgdaDatatype{⊢}}\AgdaSpace{}%
\AgdaGeneralizable{Γ}\AgdaSpace{}%
\AgdaOperator{\AgdaDatatype{▹}}\AgdaSpace{}%
\AgdaBound{ps}\AgdaSpace{}%
\AgdaBound{i}\AgdaSpace{}%
\AgdaOperator{\AgdaDatatype{≈}}\AgdaSpace{}%
\AgdaBound{qs}\AgdaSpace{}%
\AgdaBound{i}\AgdaSymbol{)}\AgdaSpace{}%
\AgdaSymbol{→}\AgdaSpace{}%
\AgdaBound{ℰ}\AgdaSpace{}%
\AgdaOperator{\AgdaDatatype{⊢}}\AgdaSpace{}%
\AgdaGeneralizable{Γ}\AgdaSpace{}%
\AgdaOperator{\AgdaDatatype{▹}}\AgdaSpace{}%
\AgdaSymbol{(}\AgdaInductiveConstructor{node}\AgdaSpace{}%
\AgdaGeneralizable{𝑓}\AgdaSpace{}%
\AgdaBound{ps}\AgdaSymbol{)}\AgdaSpace{}%
\AgdaOperator{\AgdaDatatype{≈}}\AgdaSpace{}%
\AgdaSymbol{(}\AgdaInductiveConstructor{node}\AgdaSpace{}%
\AgdaGeneralizable{𝑓}\AgdaSpace{}%
\AgdaBound{qs}\AgdaSymbol{)}\<%
\\
%
\>[2]\AgdaInductiveConstructor{sub}\AgdaSpace{}%
\AgdaSymbol{:}\AgdaSpace{}%
\AgdaSymbol{∀}\AgdaSpace{}%
\AgdaSymbol{\{}\AgdaBound{p}\AgdaSpace{}%
\AgdaBound{q}\AgdaSymbol{\}}\AgdaSpace{}%
\AgdaSymbol{→}\AgdaSpace{}%
\AgdaBound{ℰ}\AgdaSpace{}%
\AgdaOperator{\AgdaDatatype{⊢}}\AgdaSpace{}%
\AgdaGeneralizable{Δ}\AgdaSpace{}%
\AgdaOperator{\AgdaDatatype{▹}}\AgdaSpace{}%
\AgdaBound{p}\AgdaSpace{}%
\AgdaOperator{\AgdaDatatype{≈}}\AgdaSpace{}%
\AgdaBound{q}\AgdaSpace{}%
\AgdaSymbol{→}\AgdaSpace{}%
\AgdaSymbol{∀}\AgdaSpace{}%
\AgdaSymbol{(}\AgdaBound{σ}\AgdaSpace{}%
\AgdaSymbol{:}\AgdaSpace{}%
\AgdaFunction{Sub}\AgdaSpace{}%
\AgdaGeneralizable{Γ}\AgdaSpace{}%
\AgdaGeneralizable{Δ}\AgdaSymbol{)}\AgdaSpace{}%
\AgdaSymbol{→}\AgdaSpace{}%
\AgdaBound{ℰ}\AgdaSpace{}%
\AgdaOperator{\AgdaDatatype{⊢}}\AgdaSpace{}%
\AgdaGeneralizable{Γ}\AgdaSpace{}%
\AgdaOperator{\AgdaDatatype{▹}}\AgdaSpace{}%
\AgdaSymbol{(}\AgdaBound{p}\AgdaSpace{}%
\AgdaOperator{\AgdaFunction{[}}\AgdaSpace{}%
\AgdaBound{σ}\AgdaSpace{}%
\AgdaOperator{\AgdaFunction{]}}\AgdaSymbol{)}\AgdaSpace{}%
\AgdaOperator{\AgdaDatatype{≈}}\AgdaSpace{}%
\AgdaSymbol{(}\AgdaBound{q}\AgdaSpace{}%
\AgdaOperator{\AgdaFunction{[}}\AgdaSpace{}%
\AgdaBound{σ}\AgdaSpace{}%
\AgdaOperator{\AgdaFunction{]}}\AgdaSymbol{)}\<%
\\
%
\\[\AgdaEmptyExtraSkip]%
%
\>[2]\AgdaInductiveConstructor{⊢refl}%
\>[10]\AgdaSymbol{:}\AgdaSpace{}%
\AgdaSymbol{∀}\AgdaSpace{}%
\AgdaSymbol{\{}\AgdaBound{p}\AgdaSymbol{\}}%
\>[32]\AgdaSymbol{→}\AgdaSpace{}%
\AgdaBound{ℰ}\AgdaSpace{}%
\AgdaOperator{\AgdaDatatype{⊢}}\AgdaSpace{}%
\AgdaGeneralizable{Γ}\AgdaSpace{}%
\AgdaOperator{\AgdaDatatype{▹}}\AgdaSpace{}%
\AgdaBound{p}\AgdaSpace{}%
\AgdaOperator{\AgdaDatatype{≈}}\AgdaSpace{}%
\AgdaBound{p}\<%
\\
%
\>[2]\AgdaInductiveConstructor{⊢sym}%
\>[10]\AgdaSymbol{:}\AgdaSpace{}%
\AgdaSymbol{∀}\AgdaSpace{}%
\AgdaSymbol{\{}\AgdaBound{p}\AgdaSpace{}%
\AgdaBound{q}\AgdaSpace{}%
\AgdaSymbol{:}\AgdaSpace{}%
\AgdaDatatype{Term}\AgdaSpace{}%
\AgdaGeneralizable{Γ}\AgdaSymbol{\}}%
\>[32]\AgdaSymbol{→}\AgdaSpace{}%
\AgdaBound{ℰ}\AgdaSpace{}%
\AgdaOperator{\AgdaDatatype{⊢}}\AgdaSpace{}%
\AgdaGeneralizable{Γ}\AgdaSpace{}%
\AgdaOperator{\AgdaDatatype{▹}}\AgdaSpace{}%
\AgdaBound{p}\AgdaSpace{}%
\AgdaOperator{\AgdaDatatype{≈}}\AgdaSpace{}%
\AgdaBound{q}\AgdaSpace{}%
\AgdaSymbol{→}\AgdaSpace{}%
\AgdaBound{ℰ}\AgdaSpace{}%
\AgdaOperator{\AgdaDatatype{⊢}}\AgdaSpace{}%
\AgdaGeneralizable{Γ}\AgdaSpace{}%
\AgdaOperator{\AgdaDatatype{▹}}\AgdaSpace{}%
\AgdaBound{q}\AgdaSpace{}%
\AgdaOperator{\AgdaDatatype{≈}}\AgdaSpace{}%
\AgdaBound{p}\<%
\\
%
\>[2]\AgdaInductiveConstructor{⊢trans}%
\>[10]\AgdaSymbol{:}\AgdaSpace{}%
\AgdaSymbol{∀}\AgdaSpace{}%
\AgdaSymbol{\{}\AgdaBound{p}\AgdaSpace{}%
\AgdaBound{q}\AgdaSpace{}%
\AgdaBound{r}\AgdaSpace{}%
\AgdaSymbol{:}\AgdaSpace{}%
\AgdaDatatype{Term}\AgdaSpace{}%
\AgdaGeneralizable{Γ}\AgdaSymbol{\}}%
\>[32]\AgdaSymbol{→}\AgdaSpace{}%
\AgdaBound{ℰ}\AgdaSpace{}%
\AgdaOperator{\AgdaDatatype{⊢}}\AgdaSpace{}%
\AgdaGeneralizable{Γ}\AgdaSpace{}%
\AgdaOperator{\AgdaDatatype{▹}}\AgdaSpace{}%
\AgdaBound{p}\AgdaSpace{}%
\AgdaOperator{\AgdaDatatype{≈}}\AgdaSpace{}%
\AgdaBound{q}\AgdaSpace{}%
\AgdaSymbol{→}\AgdaSpace{}%
\AgdaBound{ℰ}\AgdaSpace{}%
\AgdaOperator{\AgdaDatatype{⊢}}\AgdaSpace{}%
\AgdaGeneralizable{Γ}\AgdaSpace{}%
\AgdaOperator{\AgdaDatatype{▹}}\AgdaSpace{}%
\AgdaBound{q}\AgdaSpace{}%
\AgdaOperator{\AgdaDatatype{≈}}\AgdaSpace{}%
\AgdaBound{r}\AgdaSpace{}%
\AgdaSymbol{→}\AgdaSpace{}%
\AgdaBound{ℰ}\AgdaSpace{}%
\AgdaOperator{\AgdaDatatype{⊢}}\AgdaSpace{}%
\AgdaGeneralizable{Γ}\AgdaSpace{}%
\AgdaOperator{\AgdaDatatype{▹}}\AgdaSpace{}%
\AgdaBound{p}\AgdaSpace{}%
\AgdaOperator{\AgdaDatatype{≈}}\AgdaSpace{}%
\AgdaBound{r}\<%
\\
%
\\[\AgdaEmptyExtraSkip]%
%
\>[1]\AgdaFunction{⊢▹≈IsEquiv}\AgdaSpace{}%
\AgdaSymbol{:}\AgdaSpace{}%
\AgdaSymbol{\{}\AgdaBound{X}\AgdaSpace{}%
\AgdaSymbol{:}\AgdaSpace{}%
\AgdaPrimitive{Type}\AgdaSpace{}%
\AgdaBound{χ}\AgdaSymbol{\}\{}\AgdaBound{I}\AgdaSpace{}%
\AgdaSymbol{:}\AgdaSpace{}%
\AgdaPrimitive{Type}\AgdaSpace{}%
\AgdaBound{ι}\AgdaSymbol{\}\{}\AgdaBound{ℰ}\AgdaSpace{}%
\AgdaSymbol{:}\AgdaSpace{}%
\AgdaBound{I}\AgdaSpace{}%
\AgdaSymbol{→}\AgdaSpace{}%
\AgdaRecord{Eq}\AgdaSymbol{\}}\AgdaSpace{}%
\AgdaSymbol{→}\AgdaSpace{}%
\AgdaRecord{IsEquivalence}\AgdaSpace{}%
\AgdaSymbol{(}\AgdaBound{ℰ}\AgdaSpace{}%
\AgdaOperator{\AgdaDatatype{⊢}}\AgdaSpace{}%
\AgdaBound{X}\AgdaSpace{}%
\AgdaOperator{\AgdaDatatype{▹\AgdaUnderscore{}≈\AgdaUnderscore{}}}\AgdaSymbol{)}\<%
\\
%
\>[1]\AgdaFunction{⊢▹≈IsEquiv}\AgdaSpace{}%
\AgdaSymbol{=}\AgdaSpace{}%
\AgdaKeyword{record}\AgdaSpace{}%
\AgdaSymbol{\{}\AgdaSpace{}%
\AgdaField{refl}\AgdaSpace{}%
\AgdaSymbol{=}\AgdaSpace{}%
\AgdaInductiveConstructor{⊢refl}\AgdaSpace{}%
\AgdaSymbol{;}\AgdaSpace{}%
\AgdaField{sym}\AgdaSpace{}%
\AgdaSymbol{=}\AgdaSpace{}%
\AgdaInductiveConstructor{⊢sym}\AgdaSpace{}%
\AgdaSymbol{;}\AgdaSpace{}%
\AgdaField{trans}\AgdaSpace{}%
\AgdaSymbol{=}\AgdaSpace{}%
\AgdaInductiveConstructor{⊢trans}\AgdaSpace{}%
\AgdaSymbol{\}}\<%
\end{code}

\hypertarget{soundness}{%
\subsubsection{Soundness}\label{soundness}}

In any model 𝑨 that satisfies the equations ℰ, derived equality is
actual equality
(cf.~\href{http://www.cse.chalmers.se/~abela/agda/MultiSortedAlgebra.pdf}{Andreas
Abel's Agda formalization of Birkhoff's completeness theorem}.)

\begin{code}%
\>[0]\<%
\\
\>[0]\AgdaKeyword{module}\AgdaSpace{}%
\AgdaModule{Soundness}%
\>[18]\AgdaSymbol{\{}\AgdaBound{χ}\AgdaSpace{}%
\AgdaBound{α}\AgdaSpace{}%
\AgdaBound{ι}\AgdaSpace{}%
\AgdaSymbol{:}\AgdaSpace{}%
\AgdaPostulate{Level}\AgdaSymbol{\}\{}\AgdaBound{I}\AgdaSpace{}%
\AgdaSymbol{:}\AgdaSpace{}%
\AgdaPrimitive{Type}\AgdaSpace{}%
\AgdaBound{ι}\AgdaSymbol{\}}\AgdaSpace{}%
\AgdaSymbol{(}\AgdaBound{ℰ}\AgdaSpace{}%
\AgdaSymbol{:}\AgdaSpace{}%
\AgdaBound{I}\AgdaSpace{}%
\AgdaSymbol{→}\AgdaSpace{}%
\AgdaRecord{Eq}\AgdaSymbol{\{}\AgdaBound{χ}\AgdaSymbol{\})}\<%
\\
%
\>[18]\AgdaSymbol{(}\AgdaBound{𝑨}\AgdaSpace{}%
\AgdaSymbol{:}\AgdaSpace{}%
\AgdaRecord{Algebra}\AgdaSpace{}%
\AgdaBound{α}\AgdaSpace{}%
\AgdaGeneralizable{ρᵃ}\AgdaSymbol{)}%
\>[41]\AgdaComment{--\ We\ assume\ an\ algebra\ 𝑨}\<%
\\
%
\>[18]\AgdaSymbol{(}\AgdaBound{V}\AgdaSpace{}%
\AgdaSymbol{:}\AgdaSpace{}%
\AgdaBound{𝑨}\AgdaSpace{}%
\AgdaOperator{\AgdaFunction{⊨}}\AgdaSpace{}%
\AgdaBound{ℰ}\AgdaSymbol{)}%
\>[41]\AgdaComment{--\ that\ models\ all\ equations\ in\ ℰ.}\<%
\\
%
\>[18]\AgdaKeyword{where}\<%
\\
%
\\[\AgdaEmptyExtraSkip]%
\>[0][@{}l@{\AgdaIndent{0}}]%
\>[1]\AgdaKeyword{open}\AgdaSpace{}%
\AgdaModule{SetoidReasoning}\AgdaSpace{}%
\AgdaOperator{\AgdaFunction{𝔻[}}\AgdaSpace{}%
\AgdaBound{𝑨}\AgdaSpace{}%
\AgdaOperator{\AgdaFunction{]}}\<%
\\
%
\>[1]\AgdaKeyword{open}\AgdaSpace{}%
\AgdaModule{Environment}\AgdaSpace{}%
\AgdaBound{𝑨}\<%
\\
%
\>[1]\AgdaKeyword{open}\AgdaSpace{}%
\AgdaModule{IsEquivalence}\AgdaSpace{}%
\AgdaKeyword{using}\AgdaSpace{}%
\AgdaSymbol{(}\AgdaSpace{}%
\AgdaField{refl}\AgdaSpace{}%
\AgdaSymbol{;}\AgdaSpace{}%
\AgdaField{sym}\AgdaSpace{}%
\AgdaSymbol{;}\AgdaSpace{}%
\AgdaField{trans}\AgdaSpace{}%
\AgdaSymbol{)}\<%
\\
%
\\[\AgdaEmptyExtraSkip]%
%
\>[1]\AgdaFunction{sound}\AgdaSpace{}%
\AgdaSymbol{:}\AgdaSpace{}%
\AgdaSymbol{∀}\AgdaSpace{}%
\AgdaSymbol{\{}\AgdaBound{p}\AgdaSpace{}%
\AgdaBound{q}\AgdaSymbol{\}}\AgdaSpace{}%
\AgdaSymbol{→}\AgdaSpace{}%
\AgdaBound{ℰ}\AgdaSpace{}%
\AgdaOperator{\AgdaDatatype{⊢}}\AgdaSpace{}%
\AgdaGeneralizable{Γ}\AgdaSpace{}%
\AgdaOperator{\AgdaDatatype{▹}}\AgdaSpace{}%
\AgdaBound{p}\AgdaSpace{}%
\AgdaOperator{\AgdaDatatype{≈}}\AgdaSpace{}%
\AgdaBound{q}\AgdaSpace{}%
\AgdaSymbol{→}\AgdaSpace{}%
\AgdaBound{𝑨}\AgdaSpace{}%
\AgdaOperator{\AgdaFunction{⊧}}\AgdaSpace{}%
\AgdaSymbol{(}\AgdaBound{p}\AgdaSpace{}%
\AgdaOperator{\AgdaInductiveConstructor{≈̇}}\AgdaSpace{}%
\AgdaBound{q}\AgdaSymbol{)}\<%
\\
%
\>[1]\AgdaFunction{sound}\AgdaSpace{}%
\AgdaSymbol{(}\AgdaInductiveConstructor{hyp}\AgdaSpace{}%
\AgdaBound{i}\AgdaSymbol{)}\AgdaSpace{}%
\AgdaSymbol{=}\AgdaSpace{}%
\AgdaBound{V}\AgdaSpace{}%
\AgdaBound{i}\<%
\\
%
\>[1]\AgdaFunction{sound}\AgdaSpace{}%
\AgdaSymbol{(}\AgdaInductiveConstructor{app}\AgdaSpace{}%
\AgdaBound{es}\AgdaSymbol{)}\AgdaSpace{}%
\AgdaBound{ρ}\AgdaSpace{}%
\AgdaSymbol{=}\AgdaSpace{}%
\AgdaField{cong}\AgdaSpace{}%
\AgdaSymbol{(}\AgdaField{Interp}\AgdaSpace{}%
\AgdaBound{𝑨}\AgdaSymbol{)}\AgdaSpace{}%
\AgdaSymbol{(}\AgdaInductiveConstructor{≡.refl}\AgdaSpace{}%
\AgdaOperator{\AgdaInductiveConstructor{,}}\AgdaSpace{}%
\AgdaSymbol{λ}\AgdaSpace{}%
\AgdaBound{i}\AgdaSpace{}%
\AgdaSymbol{→}\AgdaSpace{}%
\AgdaFunction{sound}\AgdaSpace{}%
\AgdaSymbol{(}\AgdaBound{es}\AgdaSpace{}%
\AgdaBound{i}\AgdaSymbol{)}\AgdaSpace{}%
\AgdaBound{ρ}\AgdaSymbol{)}\<%
\\
%
\>[1]\AgdaFunction{sound}\AgdaSpace{}%
\AgdaSymbol{(}\AgdaInductiveConstructor{sub}\AgdaSpace{}%
\AgdaSymbol{\{}\AgdaArgument{p}\AgdaSpace{}%
\AgdaSymbol{=}\AgdaSpace{}%
\AgdaBound{p}\AgdaSymbol{\}\{}\AgdaBound{q}\AgdaSymbol{\}}\AgdaSpace{}%
\AgdaBound{Epq}\AgdaSpace{}%
\AgdaBound{σ}\AgdaSymbol{)}\AgdaSpace{}%
\AgdaBound{ρ}\AgdaSpace{}%
\AgdaSymbol{=}\<%
\\
\>[1][@{}l@{\AgdaIndent{0}}]%
\>[2]\AgdaOperator{\AgdaFunction{begin}}\<%
\\
\>[2][@{}l@{\AgdaIndent{0}}]%
\>[3]\AgdaOperator{\AgdaFunction{⟦}}\AgdaSpace{}%
\AgdaBound{p}\AgdaSpace{}%
\AgdaOperator{\AgdaFunction{[}}\AgdaSpace{}%
\AgdaBound{σ}\AgdaSpace{}%
\AgdaOperator{\AgdaFunction{]}}\AgdaSpace{}%
\AgdaOperator{\AgdaFunction{⟧}}\AgdaSpace{}%
\AgdaOperator{\AgdaField{⟨\$⟩}}%
\>[27]\AgdaBound{ρ}\AgdaSpace{}%
\AgdaFunction{≈⟨}%
\>[34]\AgdaFunction{substitution}\AgdaSpace{}%
\AgdaBound{p}\AgdaSpace{}%
\AgdaBound{σ}\AgdaSpace{}%
\AgdaBound{ρ}%
\>[56]\AgdaFunction{⟩}\<%
\\
%
\>[3]\AgdaOperator{\AgdaFunction{⟦}}\AgdaSpace{}%
\AgdaBound{p}%
\>[13]\AgdaOperator{\AgdaFunction{⟧}}\AgdaSpace{}%
\AgdaOperator{\AgdaField{⟨\$⟩}}\AgdaSpace{}%
\AgdaOperator{\AgdaFunction{⟦}}\AgdaSpace{}%
\AgdaBound{σ}\AgdaSpace{}%
\AgdaOperator{\AgdaFunction{⟧s}}%
\>[27]\AgdaBound{ρ}\AgdaSpace{}%
\AgdaFunction{≈⟨}%
\>[34]\AgdaFunction{sound}\AgdaSpace{}%
\AgdaBound{Epq}\AgdaSpace{}%
\AgdaSymbol{(}\AgdaOperator{\AgdaFunction{⟦}}\AgdaSpace{}%
\AgdaBound{σ}\AgdaSpace{}%
\AgdaOperator{\AgdaFunction{⟧s}}\AgdaSpace{}%
\AgdaBound{ρ}\AgdaSymbol{)}%
\>[56]\AgdaFunction{⟩}\<%
\\
%
\>[3]\AgdaOperator{\AgdaFunction{⟦}}\AgdaSpace{}%
\AgdaBound{q}%
\>[13]\AgdaOperator{\AgdaFunction{⟧}}\AgdaSpace{}%
\AgdaOperator{\AgdaField{⟨\$⟩}}\AgdaSpace{}%
\AgdaOperator{\AgdaFunction{⟦}}\AgdaSpace{}%
\AgdaBound{σ}\AgdaSpace{}%
\AgdaOperator{\AgdaFunction{⟧s}}%
\>[27]\AgdaBound{ρ}\AgdaSpace{}%
\AgdaFunction{≈˘⟨}%
\>[34]\AgdaFunction{substitution}%
\>[48]\AgdaBound{q}\AgdaSpace{}%
\AgdaBound{σ}\AgdaSpace{}%
\AgdaBound{ρ}%
\>[56]\AgdaFunction{⟩}\<%
\\
%
\>[3]\AgdaOperator{\AgdaFunction{⟦}}\AgdaSpace{}%
\AgdaBound{q}\AgdaSpace{}%
\AgdaOperator{\AgdaFunction{[}}\AgdaSpace{}%
\AgdaBound{σ}\AgdaSpace{}%
\AgdaOperator{\AgdaFunction{]}}\AgdaSpace{}%
\AgdaOperator{\AgdaFunction{⟧}}\AgdaSpace{}%
\AgdaOperator{\AgdaField{⟨\$⟩}}%
\>[27]\AgdaBound{ρ}\AgdaSpace{}%
\AgdaOperator{\AgdaFunction{∎}}\<%
\\
%
\>[1]\AgdaFunction{sound}\AgdaSpace{}%
\AgdaSymbol{(}\AgdaSpace{}%
\AgdaInductiveConstructor{⊢refl}%
\>[17]\AgdaSymbol{\{}\AgdaArgument{p}\AgdaSpace{}%
\AgdaSymbol{=}\AgdaSpace{}%
\AgdaBound{p}\AgdaSymbol{\}}%
\>[41]\AgdaSymbol{)}\AgdaSpace{}%
\AgdaSymbol{=}\AgdaSpace{}%
\AgdaField{refl}%
\>[52]\AgdaFunction{EqualIsEquiv}\AgdaSpace{}%
\AgdaSymbol{\{}\AgdaArgument{x}\AgdaSpace{}%
\AgdaSymbol{=}\AgdaSpace{}%
\AgdaBound{p}\AgdaSymbol{\}}\<%
\\
%
\>[1]\AgdaFunction{sound}\AgdaSpace{}%
\AgdaSymbol{(}\AgdaSpace{}%
\AgdaInductiveConstructor{⊢sym}%
\>[17]\AgdaSymbol{\{}\AgdaArgument{p}\AgdaSpace{}%
\AgdaSymbol{=}\AgdaSpace{}%
\AgdaBound{p}\AgdaSymbol{\}\{}\AgdaBound{q}\AgdaSymbol{\}}%
\>[32]\AgdaBound{Epq}%
\>[41]\AgdaSymbol{)}\AgdaSpace{}%
\AgdaSymbol{=}\AgdaSpace{}%
\AgdaField{sym}%
\>[52]\AgdaFunction{EqualIsEquiv}\AgdaSpace{}%
\AgdaSymbol{\{}\AgdaArgument{x}\AgdaSpace{}%
\AgdaSymbol{=}\AgdaSpace{}%
\AgdaBound{p}\AgdaSymbol{\}\{}\AgdaBound{q}\AgdaSymbol{\}}%
\>[80]\AgdaSymbol{(}\AgdaFunction{sound}\AgdaSpace{}%
\AgdaBound{Epq}\AgdaSymbol{)}\<%
\\
%
\>[1]\AgdaFunction{sound}\AgdaSpace{}%
\AgdaSymbol{(}\AgdaSpace{}%
\AgdaInductiveConstructor{⊢trans}%
\>[17]\AgdaSymbol{\{}\AgdaArgument{p}\AgdaSpace{}%
\AgdaSymbol{=}\AgdaSpace{}%
\AgdaBound{p}\AgdaSymbol{\}\{}\AgdaBound{q}\AgdaSymbol{\}\{}\AgdaBound{r}\AgdaSymbol{\}}%
\>[32]\AgdaBound{Epq}\AgdaSpace{}%
\AgdaBound{Eqr}%
\>[41]\AgdaSymbol{)}\AgdaSpace{}%
\AgdaSymbol{=}\AgdaSpace{}%
\AgdaField{trans}%
\>[52]\AgdaFunction{EqualIsEquiv}\AgdaSpace{}%
\AgdaSymbol{\{}\AgdaArgument{i}\AgdaSpace{}%
\AgdaSymbol{=}\AgdaSpace{}%
\AgdaBound{p}\AgdaSymbol{\}\{}\AgdaBound{q}\AgdaSymbol{\}\{}\AgdaBound{r}\AgdaSymbol{\}}%
\>[80]\AgdaSymbol{(}\AgdaFunction{sound}\AgdaSpace{}%
\AgdaBound{Epq}\AgdaSymbol{)(}\AgdaFunction{sound}\AgdaSpace{}%
\AgdaBound{Eqr}\AgdaSymbol{)}\<%
\end{code}

\hypertarget{the-closure-operators-h-s-p-and-v}{%
\subsubsection{The Closure Operators H, S, P and
V}\label{the-closure-operators-h-s-p-and-v}}

Fix a signature \texttt{𝑆}, let \texttt{𝒦} be a class of
\texttt{𝑆}-algebras, and define

\begin{itemize}
\tightlist
\item
  \texttt{H\ 𝒦} = algebras isomorphic to a homomorphic image of a member
  of \texttt{𝒦};
\item
  \texttt{S\ 𝒦} = algebras isomorphic to a subalgebra of a member of
  \texttt{𝒦};
\item
  \texttt{P\ 𝒦} = algebras isomorphic to a product of members of
  \texttt{𝒦}.
\end{itemize}

A straight-forward verification confirms that \texttt{H}, \texttt{S},
and \texttt{P} are \emph{closure operators} (expansive, monotone, and
idempotent). A class \texttt{𝒦} of \texttt{𝑆}-algebras is said to be
\emph{closed under the taking of homomorphic images} provided
\texttt{H\ 𝒦\ ⊆\ 𝒦}. Similarly, \texttt{𝒦} is \emph{closed under the
taking of subalgebras} (resp., \emph{arbitrary products}) provided
\texttt{S\ 𝒦\ ⊆\ 𝒦} (resp., \texttt{P\ 𝒦\ ⊆\ 𝒦}). The operators
\texttt{H}, \texttt{S}, and \texttt{P} can be composed with one another
repeatedly, forming yet more closure operators.

An algebra is a homomorphic image (resp., subalgebra; resp., product) of
every algebra to which it is isomorphic. Thus, the class \texttt{H\ 𝒦}
(resp., \texttt{S\ 𝒦}; resp., \texttt{P\ 𝒦}) is closed under
isomorphism.

A \emph{variety} is a class of \texttt{𝑆}-algebras that is closed under
the taking of homomorphic images, subalgebras, and arbitrary products.
To represent varieties we define types for the closure operators
\texttt{H}, \texttt{S}, and \texttt{P} that are composable. Separately,
we define a type \texttt{V} which represents closure under all three
operators, \texttt{H}, \texttt{S}, and \texttt{P}.

We now define the type \texttt{H} to represent classes of algebras that
include all homomorphic images of algebras in the class---i.e., classes
that are closed under the taking of homomorphic images---the type
\texttt{S} to represent classes of algebras that closed under the taking
of subalgebras, and the type \texttt{P} to represent classes of algebras
closed under the taking of arbitrary products.

\begin{code}%
\>[0]\<%
\\
\>[0]\AgdaKeyword{module}\AgdaSpace{}%
\AgdaModule{\AgdaUnderscore{}}%
\>[10]\AgdaSymbol{\{}\AgdaBound{α}\AgdaSpace{}%
\AgdaBound{ρᵃ}\AgdaSpace{}%
\AgdaBound{β}\AgdaSpace{}%
\AgdaBound{ρᵇ}\AgdaSpace{}%
\AgdaSymbol{:}\AgdaSpace{}%
\AgdaPostulate{Level}\AgdaSymbol{\}}\AgdaSpace{}%
\AgdaKeyword{where}\<%
\\
\>[0][@{}l@{\AgdaIndent{0}}]%
\>[1]\AgdaKeyword{private}\AgdaSpace{}%
\AgdaFunction{a}\AgdaSpace{}%
\AgdaSymbol{=}\AgdaSpace{}%
\AgdaBound{α}\AgdaSpace{}%
\AgdaOperator{\AgdaPrimitive{⊔}}\AgdaSpace{}%
\AgdaBound{ρᵃ}\AgdaSpace{}%
\AgdaSymbol{;}\AgdaSpace{}%
\AgdaFunction{b}\AgdaSpace{}%
\AgdaSymbol{=}\AgdaSpace{}%
\AgdaBound{β}\AgdaSpace{}%
\AgdaOperator{\AgdaPrimitive{⊔}}\AgdaSpace{}%
\AgdaBound{ρᵇ}\<%
\\
%
\\[\AgdaEmptyExtraSkip]%
%
\>[1]\AgdaFunction{H}\AgdaSpace{}%
\AgdaSymbol{:}\AgdaSpace{}%
\AgdaSymbol{∀}\AgdaSpace{}%
\AgdaBound{ℓ}\AgdaSpace{}%
\AgdaSymbol{→}\AgdaSpace{}%
\AgdaFunction{Pred}\AgdaSymbol{(}\AgdaRecord{Algebra}\AgdaSpace{}%
\AgdaBound{α}\AgdaSpace{}%
\AgdaBound{ρᵃ}\AgdaSymbol{)}\AgdaSpace{}%
\AgdaSymbol{(}\AgdaFunction{a}\AgdaSpace{}%
\AgdaOperator{\AgdaPrimitive{⊔}}\AgdaSpace{}%
\AgdaFunction{ov}\AgdaSpace{}%
\AgdaBound{ℓ}\AgdaSymbol{)}\AgdaSpace{}%
\AgdaSymbol{→}\AgdaSpace{}%
\AgdaFunction{Pred}\AgdaSymbol{(}\AgdaRecord{Algebra}\AgdaSpace{}%
\AgdaBound{β}\AgdaSpace{}%
\AgdaBound{ρᵇ}\AgdaSymbol{)}\AgdaSpace{}%
\AgdaSymbol{(}\AgdaFunction{b}\AgdaSpace{}%
\AgdaOperator{\AgdaPrimitive{⊔}}\AgdaSpace{}%
\AgdaFunction{ov}\AgdaSymbol{(}\AgdaFunction{a}\AgdaSpace{}%
\AgdaOperator{\AgdaPrimitive{⊔}}\AgdaSpace{}%
\AgdaBound{ℓ}\AgdaSymbol{))}\<%
\\
%
\>[1]\AgdaFunction{H}\AgdaSpace{}%
\AgdaSymbol{\AgdaUnderscore{}}\AgdaSpace{}%
\AgdaBound{𝒦}\AgdaSpace{}%
\AgdaBound{𝑩}\AgdaSpace{}%
\AgdaSymbol{=}\AgdaSpace{}%
\AgdaFunction{Σ[}\AgdaSpace{}%
\AgdaBound{𝑨}\AgdaSpace{}%
\AgdaFunction{∈}\AgdaSpace{}%
\AgdaRecord{Algebra}\AgdaSpace{}%
\AgdaBound{α}\AgdaSpace{}%
\AgdaBound{ρᵃ}\AgdaSpace{}%
\AgdaFunction{]}\AgdaSpace{}%
\AgdaBound{𝑨}\AgdaSpace{}%
\AgdaOperator{\AgdaFunction{∈}}\AgdaSpace{}%
\AgdaBound{𝒦}\AgdaSpace{}%
\AgdaOperator{\AgdaFunction{×}}\AgdaSpace{}%
\AgdaBound{𝑩}\AgdaSpace{}%
\AgdaOperator{\AgdaFunction{IsHomImageOf}}\AgdaSpace{}%
\AgdaBound{𝑨}\<%
\\
%
\\[\AgdaEmptyExtraSkip]%
%
\>[1]\AgdaFunction{S}\AgdaSpace{}%
\AgdaSymbol{:}\AgdaSpace{}%
\AgdaSymbol{∀}\AgdaSpace{}%
\AgdaBound{ℓ}\AgdaSpace{}%
\AgdaSymbol{→}\AgdaSpace{}%
\AgdaFunction{Pred}\AgdaSymbol{(}\AgdaRecord{Algebra}\AgdaSpace{}%
\AgdaBound{α}\AgdaSpace{}%
\AgdaBound{ρᵃ}\AgdaSymbol{)}\AgdaSpace{}%
\AgdaSymbol{(}\AgdaFunction{a}\AgdaSpace{}%
\AgdaOperator{\AgdaPrimitive{⊔}}\AgdaSpace{}%
\AgdaFunction{ov}\AgdaSpace{}%
\AgdaBound{ℓ}\AgdaSymbol{)}\AgdaSpace{}%
\AgdaSymbol{→}\AgdaSpace{}%
\AgdaFunction{Pred}\AgdaSymbol{(}\AgdaRecord{Algebra}\AgdaSpace{}%
\AgdaBound{β}\AgdaSpace{}%
\AgdaBound{ρᵇ}\AgdaSymbol{)}\AgdaSpace{}%
\AgdaSymbol{(}\AgdaFunction{b}\AgdaSpace{}%
\AgdaOperator{\AgdaPrimitive{⊔}}\AgdaSpace{}%
\AgdaFunction{ov}\AgdaSymbol{(}\AgdaFunction{a}\AgdaSpace{}%
\AgdaOperator{\AgdaPrimitive{⊔}}\AgdaSpace{}%
\AgdaBound{ℓ}\AgdaSymbol{))}\<%
\\
%
\>[1]\AgdaFunction{S}\AgdaSpace{}%
\AgdaSymbol{\AgdaUnderscore{}}\AgdaSpace{}%
\AgdaBound{𝒦}\AgdaSpace{}%
\AgdaBound{𝑩}\AgdaSpace{}%
\AgdaSymbol{=}\AgdaSpace{}%
\AgdaFunction{Σ[}\AgdaSpace{}%
\AgdaBound{𝑨}\AgdaSpace{}%
\AgdaFunction{∈}\AgdaSpace{}%
\AgdaRecord{Algebra}\AgdaSpace{}%
\AgdaBound{α}\AgdaSpace{}%
\AgdaBound{ρᵃ}\AgdaSpace{}%
\AgdaFunction{]}\AgdaSpace{}%
\AgdaBound{𝑨}\AgdaSpace{}%
\AgdaOperator{\AgdaFunction{∈}}\AgdaSpace{}%
\AgdaBound{𝒦}\AgdaSpace{}%
\AgdaOperator{\AgdaFunction{×}}\AgdaSpace{}%
\AgdaBound{𝑩}\AgdaSpace{}%
\AgdaOperator{\AgdaFunction{≤}}\AgdaSpace{}%
\AgdaBound{𝑨}\<%
\\
%
\\[\AgdaEmptyExtraSkip]%
%
\>[1]\AgdaFunction{P}\AgdaSpace{}%
\AgdaSymbol{:}\AgdaSpace{}%
\AgdaSymbol{∀}\AgdaSpace{}%
\AgdaBound{ℓ}\AgdaSpace{}%
\AgdaBound{ι}\AgdaSpace{}%
\AgdaSymbol{→}\AgdaSpace{}%
\AgdaFunction{Pred}\AgdaSymbol{(}\AgdaRecord{Algebra}\AgdaSpace{}%
\AgdaBound{α}\AgdaSpace{}%
\AgdaBound{ρᵃ}\AgdaSymbol{)}\AgdaSpace{}%
\AgdaSymbol{(}\AgdaFunction{a}\AgdaSpace{}%
\AgdaOperator{\AgdaPrimitive{⊔}}\AgdaSpace{}%
\AgdaFunction{ov}\AgdaSpace{}%
\AgdaBound{ℓ}\AgdaSymbol{)}\AgdaSpace{}%
\AgdaSymbol{→}\AgdaSpace{}%
\AgdaFunction{Pred}\AgdaSymbol{(}\AgdaRecord{Algebra}\AgdaSpace{}%
\AgdaBound{β}\AgdaSpace{}%
\AgdaBound{ρᵇ}\AgdaSymbol{)}\AgdaSpace{}%
\AgdaSymbol{(}\AgdaFunction{b}\AgdaSpace{}%
\AgdaOperator{\AgdaPrimitive{⊔}}\AgdaSpace{}%
\AgdaFunction{ov}\AgdaSymbol{(}\AgdaFunction{a}\AgdaSpace{}%
\AgdaOperator{\AgdaPrimitive{⊔}}\AgdaSpace{}%
\AgdaBound{ℓ}\AgdaSpace{}%
\AgdaOperator{\AgdaPrimitive{⊔}}\AgdaSpace{}%
\AgdaBound{ι}\AgdaSymbol{))}\<%
\\
%
\>[1]\AgdaFunction{P}\AgdaSpace{}%
\AgdaSymbol{\AgdaUnderscore{}}\AgdaSpace{}%
\AgdaBound{ι}\AgdaSpace{}%
\AgdaBound{𝒦}\AgdaSpace{}%
\AgdaBound{𝑩}\AgdaSpace{}%
\AgdaSymbol{=}\AgdaSpace{}%
\AgdaFunction{Σ[}\AgdaSpace{}%
\AgdaBound{I}\AgdaSpace{}%
\AgdaFunction{∈}\AgdaSpace{}%
\AgdaPrimitive{Type}\AgdaSpace{}%
\AgdaBound{ι}\AgdaSpace{}%
\AgdaFunction{]}\AgdaSpace{}%
\AgdaSymbol{(}\AgdaFunction{Σ[}\AgdaSpace{}%
\AgdaBound{𝒜}\AgdaSpace{}%
\AgdaFunction{∈}\AgdaSpace{}%
\AgdaSymbol{(}\AgdaBound{I}\AgdaSpace{}%
\AgdaSymbol{→}\AgdaSpace{}%
\AgdaRecord{Algebra}\AgdaSpace{}%
\AgdaBound{α}\AgdaSpace{}%
\AgdaBound{ρᵃ}\AgdaSymbol{)}\AgdaSpace{}%
\AgdaFunction{]}\AgdaSpace{}%
\AgdaSymbol{(∀}\AgdaSpace{}%
\AgdaBound{i}\AgdaSpace{}%
\AgdaSymbol{→}\AgdaSpace{}%
\AgdaBound{𝒜}\AgdaSpace{}%
\AgdaBound{i}\AgdaSpace{}%
\AgdaOperator{\AgdaFunction{∈}}\AgdaSpace{}%
\AgdaBound{𝒦}\AgdaSymbol{)}\AgdaSpace{}%
\AgdaOperator{\AgdaFunction{×}}\AgdaSpace{}%
\AgdaSymbol{(}\AgdaBound{𝑩}\AgdaSpace{}%
\AgdaOperator{\AgdaRecord{≅}}\AgdaSpace{}%
\AgdaFunction{⨅}\AgdaSpace{}%
\AgdaBound{𝒜}\AgdaSymbol{))}\<%
\\
\>[0]\<%
\end{code}

A class \texttt{𝒦} of \texttt{𝑆}-algebras is called a \emph{variety} if
it is closed under each of the closure operators \texttt{H}, \texttt{S},
and \texttt{P} defined above. The corresponding closure operator is
often denoted \texttt{𝕍} or \texttt{𝒱}, but we will denote it by
\texttt{V}.

\begin{code}%
\>[0]\<%
\\
\>[0]\AgdaKeyword{module}\AgdaSpace{}%
\AgdaModule{\AgdaUnderscore{}}%
\>[10]\AgdaSymbol{\{}\AgdaBound{α}\AgdaSpace{}%
\AgdaBound{ρᵃ}\AgdaSpace{}%
\AgdaBound{β}\AgdaSpace{}%
\AgdaBound{ρᵇ}\AgdaSpace{}%
\AgdaBound{γ}\AgdaSpace{}%
\AgdaBound{ρᶜ}\AgdaSpace{}%
\AgdaBound{δ}\AgdaSpace{}%
\AgdaBound{ρᵈ}\AgdaSpace{}%
\AgdaSymbol{:}\AgdaSpace{}%
\AgdaPostulate{Level}\AgdaSymbol{\}}\AgdaSpace{}%
\AgdaKeyword{where}\<%
\\
\>[0][@{}l@{\AgdaIndent{0}}]%
\>[1]\AgdaKeyword{private}\AgdaSpace{}%
\AgdaFunction{a}\AgdaSpace{}%
\AgdaSymbol{=}\AgdaSpace{}%
\AgdaBound{α}\AgdaSpace{}%
\AgdaOperator{\AgdaPrimitive{⊔}}\AgdaSpace{}%
\AgdaBound{ρᵃ}\AgdaSpace{}%
\AgdaSymbol{;}\AgdaSpace{}%
\AgdaFunction{b}\AgdaSpace{}%
\AgdaSymbol{=}\AgdaSpace{}%
\AgdaBound{β}\AgdaSpace{}%
\AgdaOperator{\AgdaPrimitive{⊔}}\AgdaSpace{}%
\AgdaBound{ρᵇ}\AgdaSpace{}%
\AgdaSymbol{;}\AgdaSpace{}%
\AgdaFunction{c}\AgdaSpace{}%
\AgdaSymbol{=}\AgdaSpace{}%
\AgdaBound{γ}\AgdaSpace{}%
\AgdaOperator{\AgdaPrimitive{⊔}}\AgdaSpace{}%
\AgdaBound{ρᶜ}\AgdaSpace{}%
\AgdaSymbol{;}\AgdaSpace{}%
\AgdaFunction{d}\AgdaSpace{}%
\AgdaSymbol{=}\AgdaSpace{}%
\AgdaBound{δ}\AgdaSpace{}%
\AgdaOperator{\AgdaPrimitive{⊔}}\AgdaSpace{}%
\AgdaBound{ρᵈ}\<%
\\
%
\\[\AgdaEmptyExtraSkip]%
%
\>[1]\AgdaFunction{V}\AgdaSpace{}%
\AgdaSymbol{:}\AgdaSpace{}%
\AgdaSymbol{∀}\AgdaSpace{}%
\AgdaBound{ℓ}\AgdaSpace{}%
\AgdaBound{ι}\AgdaSpace{}%
\AgdaSymbol{→}\AgdaSpace{}%
\AgdaFunction{Pred}\AgdaSymbol{(}\AgdaRecord{Algebra}\AgdaSpace{}%
\AgdaBound{α}\AgdaSpace{}%
\AgdaBound{ρᵃ}\AgdaSymbol{)}\AgdaSpace{}%
\AgdaSymbol{(}\AgdaFunction{a}\AgdaSpace{}%
\AgdaOperator{\AgdaPrimitive{⊔}}\AgdaSpace{}%
\AgdaFunction{ov}\AgdaSpace{}%
\AgdaBound{ℓ}\AgdaSymbol{)}\AgdaSpace{}%
\AgdaSymbol{→}%
\>[46]\AgdaFunction{Pred}\AgdaSymbol{(}\AgdaRecord{Algebra}\AgdaSpace{}%
\AgdaBound{δ}\AgdaSpace{}%
\AgdaBound{ρᵈ}\AgdaSymbol{)}\AgdaSpace{}%
\AgdaSymbol{(}\AgdaFunction{d}\AgdaSpace{}%
\AgdaOperator{\AgdaPrimitive{⊔}}\AgdaSpace{}%
\AgdaFunction{ov}\AgdaSymbol{(}\AgdaFunction{a}\AgdaSpace{}%
\AgdaOperator{\AgdaPrimitive{⊔}}\AgdaSpace{}%
\AgdaFunction{b}\AgdaSpace{}%
\AgdaOperator{\AgdaPrimitive{⊔}}\AgdaSpace{}%
\AgdaFunction{c}\AgdaSpace{}%
\AgdaOperator{\AgdaPrimitive{⊔}}\AgdaSpace{}%
\AgdaBound{ℓ}\AgdaSpace{}%
\AgdaOperator{\AgdaPrimitive{⊔}}\AgdaSpace{}%
\AgdaBound{ι}\AgdaSymbol{))}\<%
\\
%
\>[1]\AgdaFunction{V}\AgdaSpace{}%
\AgdaBound{ℓ}\AgdaSpace{}%
\AgdaBound{ι}\AgdaSpace{}%
\AgdaBound{𝒦}\AgdaSpace{}%
\AgdaSymbol{=}\AgdaSpace{}%
\AgdaFunction{H}\AgdaSymbol{\{}\AgdaBound{γ}\AgdaSymbol{\}\{}\AgdaBound{ρᶜ}\AgdaSymbol{\}\{}\AgdaBound{δ}\AgdaSymbol{\}\{}\AgdaBound{ρᵈ}\AgdaSymbol{\}}\AgdaSpace{}%
\AgdaSymbol{(}\AgdaFunction{a}\AgdaSpace{}%
\AgdaOperator{\AgdaPrimitive{⊔}}\AgdaSpace{}%
\AgdaFunction{b}\AgdaSpace{}%
\AgdaOperator{\AgdaPrimitive{⊔}}\AgdaSpace{}%
\AgdaBound{ℓ}\AgdaSpace{}%
\AgdaOperator{\AgdaPrimitive{⊔}}\AgdaSpace{}%
\AgdaBound{ι}\AgdaSymbol{)}\AgdaSpace{}%
\AgdaSymbol{(}\AgdaFunction{S}\AgdaSymbol{\{}\AgdaBound{β}\AgdaSymbol{\}\{}\AgdaBound{ρᵇ}\AgdaSymbol{\}}\AgdaSpace{}%
\AgdaSymbol{(}\AgdaFunction{a}\AgdaSpace{}%
\AgdaOperator{\AgdaPrimitive{⊔}}\AgdaSpace{}%
\AgdaBound{ℓ}\AgdaSpace{}%
\AgdaOperator{\AgdaPrimitive{⊔}}\AgdaSpace{}%
\AgdaBound{ι}\AgdaSymbol{)}\AgdaSpace{}%
\AgdaSymbol{(}\AgdaFunction{P}\AgdaSpace{}%
\AgdaBound{ℓ}\AgdaSpace{}%
\AgdaBound{ι}\AgdaSpace{}%
\AgdaBound{𝒦}\AgdaSymbol{))}\<%
\\
%
\\[\AgdaEmptyExtraSkip]%
\>[0]\AgdaKeyword{module}\AgdaSpace{}%
\AgdaModule{\AgdaUnderscore{}}%
\>[5070I]\AgdaSymbol{\{}\AgdaBound{α}\AgdaSpace{}%
\AgdaBound{ρᵃ}\AgdaSpace{}%
\AgdaBound{ℓ}\AgdaSpace{}%
\AgdaSymbol{:}\AgdaSpace{}%
\AgdaPostulate{Level}\AgdaSymbol{\}(}\AgdaBound{𝒦}\AgdaSpace{}%
\AgdaSymbol{:}\AgdaSpace{}%
\AgdaFunction{Pred}\AgdaSymbol{(}\AgdaRecord{Algebra}\AgdaSpace{}%
\AgdaBound{α}\AgdaSpace{}%
\AgdaBound{ρᵃ}\AgdaSymbol{)}\AgdaSpace{}%
\AgdaSymbol{(}\AgdaBound{α}\AgdaSpace{}%
\AgdaOperator{\AgdaPrimitive{⊔}}\AgdaSpace{}%
\AgdaBound{ρᵃ}\AgdaSpace{}%
\AgdaOperator{\AgdaPrimitive{⊔}}\AgdaSpace{}%
\AgdaFunction{ov}\AgdaSpace{}%
\AgdaBound{ℓ}\AgdaSymbol{))}\<%
\\
\>[.][@{}l@{}]\<[5070I]%
\>[9]\AgdaSymbol{(}\AgdaBound{𝑨}\AgdaSpace{}%
\AgdaSymbol{:}\AgdaSpace{}%
\AgdaRecord{Algebra}\AgdaSpace{}%
\AgdaSymbol{(}\AgdaBound{α}\AgdaSpace{}%
\AgdaOperator{\AgdaPrimitive{⊔}}\AgdaSpace{}%
\AgdaBound{ρᵃ}\AgdaSpace{}%
\AgdaOperator{\AgdaPrimitive{⊔}}\AgdaSpace{}%
\AgdaBound{ℓ}\AgdaSymbol{)}\AgdaSpace{}%
\AgdaSymbol{(}\AgdaBound{α}\AgdaSpace{}%
\AgdaOperator{\AgdaPrimitive{⊔}}\AgdaSpace{}%
\AgdaBound{ρᵃ}\AgdaSpace{}%
\AgdaOperator{\AgdaPrimitive{⊔}}\AgdaSpace{}%
\AgdaBound{ℓ}\AgdaSymbol{))}\AgdaSpace{}%
\AgdaKeyword{where}\<%
\\
\>[0][@{}l@{\AgdaIndent{0}}]%
\>[1]\AgdaKeyword{private}\AgdaSpace{}%
\AgdaFunction{ι}\AgdaSpace{}%
\AgdaSymbol{=}\AgdaSpace{}%
\AgdaFunction{ov}\AgdaSymbol{(}\AgdaBound{α}\AgdaSpace{}%
\AgdaOperator{\AgdaPrimitive{⊔}}\AgdaSpace{}%
\AgdaBound{ρᵃ}\AgdaSpace{}%
\AgdaOperator{\AgdaPrimitive{⊔}}\AgdaSpace{}%
\AgdaBound{ℓ}\AgdaSymbol{)}\<%
\\
%
\\[\AgdaEmptyExtraSkip]%
%
\>[1]\AgdaFunction{V-≅-lc}\AgdaSpace{}%
\AgdaSymbol{:}\AgdaSpace{}%
\AgdaFunction{Lift-Alg}\AgdaSpace{}%
\AgdaBound{𝑨}\AgdaSpace{}%
\AgdaFunction{ι}\AgdaSpace{}%
\AgdaFunction{ι}\AgdaSpace{}%
\AgdaOperator{\AgdaFunction{∈}}\AgdaSpace{}%
\AgdaFunction{V}\AgdaSymbol{\{}\AgdaArgument{β}\AgdaSpace{}%
\AgdaSymbol{=}\AgdaSpace{}%
\AgdaFunction{ι}\AgdaSymbol{\}\{}\AgdaFunction{ι}\AgdaSymbol{\}}\AgdaSpace{}%
\AgdaBound{ℓ}\AgdaSpace{}%
\AgdaFunction{ι}\AgdaSpace{}%
\AgdaBound{𝒦}\AgdaSpace{}%
\AgdaSymbol{→}\AgdaSpace{}%
\AgdaBound{𝑨}\AgdaSpace{}%
\AgdaOperator{\AgdaFunction{∈}}\AgdaSpace{}%
\AgdaFunction{V}\AgdaSymbol{\{}\AgdaArgument{γ}\AgdaSpace{}%
\AgdaSymbol{=}\AgdaSpace{}%
\AgdaFunction{ι}\AgdaSymbol{\}\{}\AgdaFunction{ι}\AgdaSymbol{\}}\AgdaSpace{}%
\AgdaBound{ℓ}\AgdaSpace{}%
\AgdaFunction{ι}\AgdaSpace{}%
\AgdaBound{𝒦}\<%
\\
%
\>[1]\AgdaFunction{V-≅-lc}\AgdaSpace{}%
\AgdaSymbol{(}\AgdaBound{𝑨'}\AgdaSpace{}%
\AgdaOperator{\AgdaInductiveConstructor{,}}\AgdaSpace{}%
\AgdaBound{spA'}\AgdaSpace{}%
\AgdaOperator{\AgdaInductiveConstructor{,}}\AgdaSpace{}%
\AgdaBound{lAimgA'}\AgdaSymbol{)}\AgdaSpace{}%
\AgdaSymbol{=}\AgdaSpace{}%
\AgdaBound{𝑨'}\AgdaSpace{}%
\AgdaOperator{\AgdaInductiveConstructor{,}}\AgdaSpace{}%
\AgdaSymbol{(}\AgdaBound{spA'}\AgdaSpace{}%
\AgdaOperator{\AgdaInductiveConstructor{,}}\AgdaSpace{}%
\AgdaFunction{AimgA'}\AgdaSymbol{)}\<%
\\
\>[1][@{}l@{\AgdaIndent{0}}]%
\>[2]\AgdaKeyword{where}\<%
\\
%
\>[2]\AgdaFunction{AimgA'}\AgdaSpace{}%
\AgdaSymbol{:}\AgdaSpace{}%
\AgdaBound{𝑨}\AgdaSpace{}%
\AgdaOperator{\AgdaFunction{IsHomImageOf}}\AgdaSpace{}%
\AgdaBound{𝑨'}\<%
\\
%
\>[2]\AgdaFunction{AimgA'}\AgdaSpace{}%
\AgdaSymbol{=}\AgdaSpace{}%
\AgdaFunction{Lift-HomImage-lemma}\AgdaSpace{}%
\AgdaBound{lAimgA'}\<%
\end{code}

\hypertarget{idempotence-of-s}{%
\paragraph{Idempotence of S}\label{idempotence-of-s}}

\texttt{S} is a closure operator. The facts that S is monotone and
expansive won't be needed, so we omit the proof of these facts. However,
we will make use of idempotence of \texttt{S}, so we prove that property
as follows.

\begin{code}%
\>[0]\<%
\\
\>[0]\AgdaFunction{S-idem}\AgdaSpace{}%
\AgdaSymbol{:}\AgdaSpace{}%
\AgdaSymbol{\{}\AgdaBound{𝒦}\AgdaSpace{}%
\AgdaSymbol{:}\AgdaSpace{}%
\AgdaFunction{Pred}\AgdaSpace{}%
\AgdaSymbol{(}\AgdaRecord{Algebra}\AgdaSpace{}%
\AgdaGeneralizable{α}\AgdaSpace{}%
\AgdaGeneralizable{ρᵃ}\AgdaSymbol{)(}\AgdaGeneralizable{α}\AgdaSpace{}%
\AgdaOperator{\AgdaPrimitive{⊔}}\AgdaSpace{}%
\AgdaGeneralizable{ρᵃ}\AgdaSpace{}%
\AgdaOperator{\AgdaPrimitive{⊔}}\AgdaSpace{}%
\AgdaFunction{ov}\AgdaSpace{}%
\AgdaGeneralizable{ℓ}\AgdaSymbol{)\}}\<%
\\
\>[0][@{}l@{\AgdaIndent{0}}]%
\>[1]\AgdaSymbol{→}%
\>[9]\AgdaFunction{S}\AgdaSymbol{\{}\AgdaArgument{β}\AgdaSpace{}%
\AgdaSymbol{=}\AgdaSpace{}%
\AgdaGeneralizable{γ}\AgdaSymbol{\}\{}\AgdaGeneralizable{ρᶜ}\AgdaSymbol{\}}\AgdaSpace{}%
\AgdaSymbol{(}\AgdaGeneralizable{α}\AgdaSpace{}%
\AgdaOperator{\AgdaPrimitive{⊔}}\AgdaSpace{}%
\AgdaGeneralizable{ρᵃ}%
\>[31]\AgdaOperator{\AgdaPrimitive{⊔}}\AgdaSpace{}%
\AgdaGeneralizable{ℓ}\AgdaSymbol{)}\AgdaSpace{}%
\AgdaSymbol{(}\AgdaFunction{S}\AgdaSymbol{\{}\AgdaArgument{β}\AgdaSpace{}%
\AgdaSymbol{=}\AgdaSpace{}%
\AgdaGeneralizable{β}\AgdaSymbol{\}\{}\AgdaGeneralizable{ρᵇ}\AgdaSymbol{\}}\AgdaSpace{}%
\AgdaGeneralizable{ℓ}\AgdaSpace{}%
\AgdaBound{𝒦}\AgdaSymbol{)}\AgdaSpace{}%
\AgdaOperator{\AgdaFunction{⊆}}\AgdaSpace{}%
\AgdaFunction{S}\AgdaSymbol{\{}\AgdaArgument{β}\AgdaSpace{}%
\AgdaSymbol{=}\AgdaSpace{}%
\AgdaGeneralizable{γ}\AgdaSymbol{\}\{}\AgdaGeneralizable{ρᶜ}\AgdaSymbol{\}}\AgdaSpace{}%
\AgdaGeneralizable{ℓ}\AgdaSpace{}%
\AgdaBound{𝒦}\<%
\\
%
\\[\AgdaEmptyExtraSkip]%
\>[0]\AgdaFunction{S-idem}\AgdaSpace{}%
\AgdaSymbol{(}\AgdaBound{𝑨}\AgdaSpace{}%
\AgdaOperator{\AgdaInductiveConstructor{,}}\AgdaSpace{}%
\AgdaSymbol{(}\AgdaBound{𝑩}\AgdaSpace{}%
\AgdaOperator{\AgdaInductiveConstructor{,}}\AgdaSpace{}%
\AgdaBound{sB}\AgdaSpace{}%
\AgdaOperator{\AgdaInductiveConstructor{,}}\AgdaSpace{}%
\AgdaBound{A≤B}\AgdaSymbol{)}\AgdaSpace{}%
\AgdaOperator{\AgdaInductiveConstructor{,}}\AgdaSpace{}%
\AgdaBound{x≤A}\AgdaSymbol{)}\AgdaSpace{}%
\AgdaSymbol{=}\AgdaSpace{}%
\AgdaBound{𝑩}\AgdaSpace{}%
\AgdaOperator{\AgdaInductiveConstructor{,}}\AgdaSpace{}%
\AgdaSymbol{(}\AgdaBound{sB}\AgdaSpace{}%
\AgdaOperator{\AgdaInductiveConstructor{,}}\AgdaSpace{}%
\AgdaFunction{≤-trans}\AgdaSpace{}%
\AgdaBound{x≤A}\AgdaSpace{}%
\AgdaBound{A≤B}\AgdaSymbol{)}\<%
\end{code}

\hypertarget{algebraic-invariance-of}{%
\paragraph{Algebraic invariance of ⊧}\label{algebraic-invariance-of}}

The binary relation \texttt{⊧} would be practically useless if it were
not an \emph{algebraic invariant} (i.e., invariant under isomorphism).
Let us now verify that the models relation we defined above has this
essential property.

\begin{code}%
\>[0]\<%
\\
\>[0]\AgdaKeyword{module}\AgdaSpace{}%
\AgdaModule{\AgdaUnderscore{}}\AgdaSpace{}%
\AgdaSymbol{\{}\AgdaBound{X}\AgdaSpace{}%
\AgdaSymbol{:}\AgdaSpace{}%
\AgdaPrimitive{Type}\AgdaSpace{}%
\AgdaGeneralizable{χ}\AgdaSymbol{\}\{}\AgdaBound{𝑨}\AgdaSpace{}%
\AgdaSymbol{:}\AgdaSpace{}%
\AgdaRecord{Algebra}\AgdaSpace{}%
\AgdaGeneralizable{α}\AgdaSpace{}%
\AgdaGeneralizable{ρᵃ}\AgdaSymbol{\}(}\AgdaBound{𝑩}\AgdaSpace{}%
\AgdaSymbol{:}\AgdaSpace{}%
\AgdaRecord{Algebra}\AgdaSpace{}%
\AgdaGeneralizable{β}\AgdaSpace{}%
\AgdaGeneralizable{ρᵇ}\AgdaSymbol{)(}\AgdaBound{p}\AgdaSpace{}%
\AgdaBound{q}\AgdaSpace{}%
\AgdaSymbol{:}\AgdaSpace{}%
\AgdaDatatype{Term}\AgdaSpace{}%
\AgdaBound{X}\AgdaSymbol{)}\AgdaSpace{}%
\AgdaKeyword{where}\<%
\\
%
\\[\AgdaEmptyExtraSkip]%
\>[0][@{}l@{\AgdaIndent{0}}]%
\>[1]\AgdaFunction{⊧-I-invar}\AgdaSpace{}%
\AgdaSymbol{:}\AgdaSpace{}%
\AgdaBound{𝑨}\AgdaSpace{}%
\AgdaOperator{\AgdaFunction{⊧}}\AgdaSpace{}%
\AgdaSymbol{(}\AgdaBound{p}\AgdaSpace{}%
\AgdaOperator{\AgdaInductiveConstructor{≈̇}}\AgdaSpace{}%
\AgdaBound{q}\AgdaSymbol{)}%
\>[27]\AgdaSymbol{→}%
\>[30]\AgdaBound{𝑨}\AgdaSpace{}%
\AgdaOperator{\AgdaRecord{≅}}\AgdaSpace{}%
\AgdaBound{𝑩}%
\>[37]\AgdaSymbol{→}%
\>[40]\AgdaBound{𝑩}\AgdaSpace{}%
\AgdaOperator{\AgdaFunction{⊧}}\AgdaSpace{}%
\AgdaSymbol{(}\AgdaBound{p}\AgdaSpace{}%
\AgdaOperator{\AgdaInductiveConstructor{≈̇}}\AgdaSpace{}%
\AgdaBound{q}\AgdaSymbol{)}\<%
\\
%
\>[1]\AgdaFunction{⊧-I-invar}\AgdaSpace{}%
\AgdaBound{Apq}\AgdaSpace{}%
\AgdaSymbol{(}\AgdaInductiveConstructor{mkiso}\AgdaSpace{}%
\AgdaBound{fh}\AgdaSpace{}%
\AgdaBound{gh}\AgdaSpace{}%
\AgdaBound{f∼g}\AgdaSpace{}%
\AgdaBound{g∼f}\AgdaSymbol{)}\AgdaSpace{}%
\AgdaBound{ρ}\AgdaSpace{}%
\AgdaSymbol{=}\<%
\\
\>[1][@{}l@{\AgdaIndent{0}}]%
\>[2]\AgdaOperator{\AgdaFunction{begin}}\<%
\\
\>[2][@{}l@{\AgdaIndent{0}}]%
\>[3]\AgdaOperator{\AgdaFunction{⟦}}\AgdaSpace{}%
\AgdaBound{p}\AgdaSpace{}%
\AgdaOperator{\AgdaFunction{⟧₂}}\AgdaSpace{}%
\AgdaOperator{\AgdaField{⟨\$⟩}}\AgdaSpace{}%
\AgdaBound{ρ}%
\>[29]\AgdaFunction{≈˘⟨}%
\>[34]\AgdaField{cong}\AgdaSpace{}%
\AgdaOperator{\AgdaFunction{⟦}}\AgdaSpace{}%
\AgdaBound{p}\AgdaSpace{}%
\AgdaOperator{\AgdaFunction{⟧₂}}\AgdaSpace{}%
\AgdaSymbol{(λ}\AgdaSpace{}%
\AgdaBound{x}\AgdaSpace{}%
\AgdaSymbol{→}\AgdaSpace{}%
\AgdaBound{f∼g}\AgdaSpace{}%
\AgdaSymbol{(}\AgdaBound{ρ}\AgdaSpace{}%
\AgdaBound{x}\AgdaSymbol{))}%
\>[65]\AgdaFunction{⟩}\<%
\\
%
\>[3]\AgdaOperator{\AgdaFunction{⟦}}\AgdaSpace{}%
\AgdaBound{p}\AgdaSpace{}%
\AgdaOperator{\AgdaFunction{⟧₂}}\AgdaSpace{}%
\AgdaOperator{\AgdaField{⟨\$⟩}}\AgdaSpace{}%
\AgdaSymbol{(}\AgdaFunction{f}\AgdaSpace{}%
\AgdaOperator{\AgdaFunction{∘}}\AgdaSpace{}%
\AgdaSymbol{(}\AgdaFunction{g}\AgdaSpace{}%
\AgdaOperator{\AgdaFunction{∘}}\AgdaSpace{}%
\AgdaBound{ρ}\AgdaSymbol{))}%
\>[29]\AgdaFunction{≈˘⟨}%
\>[34]\AgdaFunction{comm-hom-term}\AgdaSpace{}%
\AgdaBound{fh}\AgdaSpace{}%
\AgdaBound{p}\AgdaSpace{}%
\AgdaSymbol{(}\AgdaFunction{g}\AgdaSpace{}%
\AgdaOperator{\AgdaFunction{∘}}\AgdaSpace{}%
\AgdaBound{ρ}\AgdaSymbol{)}%
\>[65]\AgdaFunction{⟩}\<%
\\
%
\>[3]\AgdaFunction{f}\AgdaSpace{}%
\AgdaSymbol{(}\AgdaOperator{\AgdaFunction{⟦}}\AgdaSpace{}%
\AgdaBound{p}\AgdaSpace{}%
\AgdaOperator{\AgdaFunction{⟧₁}}\AgdaSpace{}%
\AgdaOperator{\AgdaField{⟨\$⟩}}\AgdaSpace{}%
\AgdaSymbol{(}\AgdaFunction{g}\AgdaSpace{}%
\AgdaOperator{\AgdaFunction{∘}}\AgdaSpace{}%
\AgdaBound{ρ}\AgdaSymbol{))}%
\>[29]\AgdaFunction{≈⟨}%
\>[34]\AgdaField{cong}\AgdaSpace{}%
\AgdaOperator{\AgdaFunction{∣}}\AgdaSpace{}%
\AgdaBound{fh}\AgdaSpace{}%
\AgdaOperator{\AgdaFunction{∣}}\AgdaSpace{}%
\AgdaSymbol{(}\AgdaBound{Apq}\AgdaSpace{}%
\AgdaSymbol{(}\AgdaFunction{g}\AgdaSpace{}%
\AgdaOperator{\AgdaFunction{∘}}\AgdaSpace{}%
\AgdaBound{ρ}\AgdaSymbol{))}%
\>[65]\AgdaFunction{⟩}\<%
\\
%
\>[3]\AgdaFunction{f}\AgdaSpace{}%
\AgdaSymbol{(}\AgdaOperator{\AgdaFunction{⟦}}\AgdaSpace{}%
\AgdaBound{q}\AgdaSpace{}%
\AgdaOperator{\AgdaFunction{⟧₁}}\AgdaSpace{}%
\AgdaOperator{\AgdaField{⟨\$⟩}}\AgdaSpace{}%
\AgdaSymbol{(}\AgdaFunction{g}\AgdaSpace{}%
\AgdaOperator{\AgdaFunction{∘}}\AgdaSpace{}%
\AgdaBound{ρ}\AgdaSymbol{))}%
\>[29]\AgdaFunction{≈⟨}%
\>[34]\AgdaFunction{comm-hom-term}\AgdaSpace{}%
\AgdaBound{fh}\AgdaSpace{}%
\AgdaBound{q}\AgdaSpace{}%
\AgdaSymbol{(}\AgdaFunction{g}\AgdaSpace{}%
\AgdaOperator{\AgdaFunction{∘}}\AgdaSpace{}%
\AgdaBound{ρ}\AgdaSymbol{)}%
\>[65]\AgdaFunction{⟩}\<%
\\
%
\>[3]\AgdaOperator{\AgdaFunction{⟦}}\AgdaSpace{}%
\AgdaBound{q}\AgdaSpace{}%
\AgdaOperator{\AgdaFunction{⟧₂}}\AgdaSpace{}%
\AgdaOperator{\AgdaField{⟨\$⟩}}\AgdaSpace{}%
\AgdaSymbol{(}\AgdaFunction{f}\AgdaSpace{}%
\AgdaOperator{\AgdaFunction{∘}}\AgdaSpace{}%
\AgdaSymbol{(}\AgdaFunction{g}\AgdaSpace{}%
\AgdaOperator{\AgdaFunction{∘}}\AgdaSpace{}%
\AgdaBound{ρ}\AgdaSymbol{))}%
\>[29]\AgdaFunction{≈⟨}%
\>[34]\AgdaField{cong}\AgdaSpace{}%
\AgdaOperator{\AgdaFunction{⟦}}\AgdaSpace{}%
\AgdaBound{q}\AgdaSpace{}%
\AgdaOperator{\AgdaFunction{⟧₂}}\AgdaSpace{}%
\AgdaSymbol{(λ}\AgdaSpace{}%
\AgdaBound{x}\AgdaSpace{}%
\AgdaSymbol{→}\AgdaSpace{}%
\AgdaBound{f∼g}\AgdaSpace{}%
\AgdaSymbol{(}\AgdaBound{ρ}\AgdaSpace{}%
\AgdaBound{x}\AgdaSymbol{))}%
\>[65]\AgdaFunction{⟩}\<%
\\
%
\>[3]\AgdaOperator{\AgdaFunction{⟦}}\AgdaSpace{}%
\AgdaBound{q}\AgdaSpace{}%
\AgdaOperator{\AgdaFunction{⟧₂}}\AgdaSpace{}%
\AgdaOperator{\AgdaField{⟨\$⟩}}\AgdaSpace{}%
\AgdaBound{ρ}%
\>[29]\AgdaOperator{\AgdaFunction{∎}}\<%
\\
%
\>[2]\AgdaKeyword{where}\<%
\\
%
\>[2]\AgdaKeyword{open}\AgdaSpace{}%
\AgdaModule{Environment}\AgdaSpace{}%
\AgdaBound{𝑨}%
\>[25]\AgdaKeyword{using}\AgdaSpace{}%
\AgdaSymbol{()}\AgdaSpace{}%
\AgdaKeyword{renaming}\AgdaSpace{}%
\AgdaSymbol{(}\AgdaSpace{}%
\AgdaOperator{\AgdaFunction{⟦\AgdaUnderscore{}⟧}}\AgdaSpace{}%
\AgdaSymbol{to}\AgdaSpace{}%
\AgdaOperator{\AgdaFunction{⟦\AgdaUnderscore{}⟧₁}}\AgdaSpace{}%
\AgdaSymbol{)}\<%
\\
%
\>[2]\AgdaKeyword{open}\AgdaSpace{}%
\AgdaModule{Environment}\AgdaSpace{}%
\AgdaBound{𝑩}%
\>[25]\AgdaKeyword{using}\AgdaSpace{}%
\AgdaSymbol{()}\AgdaSpace{}%
\AgdaKeyword{renaming}\AgdaSpace{}%
\AgdaSymbol{(}\AgdaSpace{}%
\AgdaOperator{\AgdaFunction{⟦\AgdaUnderscore{}⟧}}\AgdaSpace{}%
\AgdaSymbol{to}\AgdaSpace{}%
\AgdaOperator{\AgdaFunction{⟦\AgdaUnderscore{}⟧₂}}\AgdaSpace{}%
\AgdaSymbol{)}\<%
\\
%
\>[2]\AgdaKeyword{open}\AgdaSpace{}%
\AgdaModule{SetoidReasoning}\AgdaSpace{}%
\AgdaOperator{\AgdaFunction{𝔻[}}\AgdaSpace{}%
\AgdaBound{𝑩}\AgdaSpace{}%
\AgdaOperator{\AgdaFunction{]}}\<%
\\
%
\>[2]\AgdaKeyword{private}\AgdaSpace{}%
\AgdaFunction{f}\AgdaSpace{}%
\AgdaSymbol{=}\AgdaSpace{}%
\AgdaOperator{\AgdaField{\AgdaUnderscore{}⟨\$⟩\AgdaUnderscore{}}}\AgdaSpace{}%
\AgdaOperator{\AgdaFunction{∣}}\AgdaSpace{}%
\AgdaBound{fh}\AgdaSpace{}%
\AgdaOperator{\AgdaFunction{∣}}\AgdaSpace{}%
\AgdaSymbol{;}\AgdaSpace{}%
\AgdaFunction{g}\AgdaSpace{}%
\AgdaSymbol{=}\AgdaSpace{}%
\AgdaOperator{\AgdaField{\AgdaUnderscore{}⟨\$⟩\AgdaUnderscore{}}}\AgdaSpace{}%
\AgdaOperator{\AgdaFunction{∣}}\AgdaSpace{}%
\AgdaBound{gh}\AgdaSpace{}%
\AgdaOperator{\AgdaFunction{∣}}\<%
\end{code}

\hypertarget{subalgebraic-invariance-of}{%
\paragraph{Subalgebraic invariance of
⊧}\label{subalgebraic-invariance-of}}

Identities modeled by an algebra \texttt{𝑨} are also modeled by every
subalgebra of \texttt{𝑨}, which fact can be formalized as follows.

\begin{code}%
\>[0]\<%
\\
\>[0]\AgdaKeyword{module}\AgdaSpace{}%
\AgdaModule{\AgdaUnderscore{}}\AgdaSpace{}%
\AgdaSymbol{\{}\AgdaBound{X}\AgdaSpace{}%
\AgdaSymbol{:}\AgdaSpace{}%
\AgdaPrimitive{Type}\AgdaSpace{}%
\AgdaGeneralizable{χ}\AgdaSymbol{\}\{}\AgdaBound{𝑨}\AgdaSpace{}%
\AgdaSymbol{:}\AgdaSpace{}%
\AgdaRecord{Algebra}\AgdaSpace{}%
\AgdaGeneralizable{α}\AgdaSpace{}%
\AgdaGeneralizable{ρᵃ}\AgdaSymbol{\}\{}\AgdaBound{𝑩}\AgdaSpace{}%
\AgdaSymbol{:}\AgdaSpace{}%
\AgdaRecord{Algebra}\AgdaSpace{}%
\AgdaGeneralizable{β}\AgdaSpace{}%
\AgdaGeneralizable{ρᵇ}\AgdaSymbol{\}\{}\AgdaBound{p}\AgdaSpace{}%
\AgdaBound{q}\AgdaSpace{}%
\AgdaSymbol{:}\AgdaSpace{}%
\AgdaDatatype{Term}\AgdaSpace{}%
\AgdaBound{X}\AgdaSymbol{\}}\AgdaSpace{}%
\AgdaKeyword{where}\<%
\\
%
\\[\AgdaEmptyExtraSkip]%
\>[0][@{}l@{\AgdaIndent{0}}]%
\>[1]\AgdaFunction{⊧-S-invar}\AgdaSpace{}%
\AgdaSymbol{:}\AgdaSpace{}%
\AgdaBound{𝑨}\AgdaSpace{}%
\AgdaOperator{\AgdaFunction{⊧}}\AgdaSpace{}%
\AgdaSymbol{(}\AgdaBound{p}\AgdaSpace{}%
\AgdaOperator{\AgdaInductiveConstructor{≈̇}}\AgdaSpace{}%
\AgdaBound{q}\AgdaSymbol{)}\AgdaSpace{}%
\AgdaSymbol{→}%
\>[29]\AgdaBound{𝑩}\AgdaSpace{}%
\AgdaOperator{\AgdaFunction{≤}}\AgdaSpace{}%
\AgdaBound{𝑨}%
\>[36]\AgdaSymbol{→}%
\>[39]\AgdaBound{𝑩}\AgdaSpace{}%
\AgdaOperator{\AgdaFunction{⊧}}\AgdaSpace{}%
\AgdaSymbol{(}\AgdaBound{p}\AgdaSpace{}%
\AgdaOperator{\AgdaInductiveConstructor{≈̇}}\AgdaSpace{}%
\AgdaBound{q}\AgdaSymbol{)}\<%
\\
%
\>[1]\AgdaFunction{⊧-S-invar}\AgdaSpace{}%
\AgdaBound{Apq}\AgdaSpace{}%
\AgdaBound{B≤A}\AgdaSpace{}%
\AgdaBound{b}\AgdaSpace{}%
\AgdaSymbol{=}\AgdaSpace{}%
\AgdaFunction{goal}\<%
\\
\>[1][@{}l@{\AgdaIndent{0}}]%
\>[2]\AgdaKeyword{where}\<%
\\
%
\>[2]\AgdaKeyword{open}\AgdaSpace{}%
\AgdaModule{Setoid}\AgdaSpace{}%
\AgdaOperator{\AgdaFunction{𝔻[}}\AgdaSpace{}%
\AgdaBound{𝑨}\AgdaSpace{}%
\AgdaOperator{\AgdaFunction{]}}\AgdaSpace{}%
\AgdaKeyword{using}\AgdaSpace{}%
\AgdaSymbol{(}\AgdaSpace{}%
\AgdaOperator{\AgdaField{\AgdaUnderscore{}≈\AgdaUnderscore{}}}\AgdaSpace{}%
\AgdaSymbol{)}\<%
\\
%
\>[2]\AgdaKeyword{open}\AgdaSpace{}%
\AgdaModule{Environment}\AgdaSpace{}%
\AgdaBound{𝑨}\AgdaSpace{}%
\AgdaKeyword{using}\AgdaSpace{}%
\AgdaSymbol{()}\AgdaSpace{}%
\AgdaKeyword{renaming}\AgdaSpace{}%
\AgdaSymbol{(}\AgdaSpace{}%
\AgdaOperator{\AgdaFunction{⟦\AgdaUnderscore{}⟧}}\AgdaSpace{}%
\AgdaSymbol{to}\AgdaSpace{}%
\AgdaOperator{\AgdaFunction{⟦\AgdaUnderscore{}⟧ᵃ}}\AgdaSpace{}%
\AgdaSymbol{)}\<%
\\
%
\>[2]\AgdaKeyword{open}\AgdaSpace{}%
\AgdaModule{Setoid}\AgdaSpace{}%
\AgdaOperator{\AgdaFunction{𝔻[}}\AgdaSpace{}%
\AgdaBound{𝑩}\AgdaSpace{}%
\AgdaOperator{\AgdaFunction{]}}\AgdaSpace{}%
\AgdaKeyword{using}\AgdaSpace{}%
\AgdaSymbol{()}\AgdaSpace{}%
\AgdaKeyword{renaming}\AgdaSpace{}%
\AgdaSymbol{(}\AgdaSpace{}%
\AgdaOperator{\AgdaField{\AgdaUnderscore{}≈\AgdaUnderscore{}}}\AgdaSpace{}%
\AgdaSymbol{to}\AgdaSpace{}%
\AgdaOperator{\AgdaField{\AgdaUnderscore{}≈ᵇ\AgdaUnderscore{}}}\AgdaSpace{}%
\AgdaSymbol{)}\<%
\\
%
\>[2]\AgdaKeyword{open}\AgdaSpace{}%
\AgdaModule{Environment}\AgdaSpace{}%
\AgdaBound{𝑩}\AgdaSpace{}%
\AgdaKeyword{using}\AgdaSpace{}%
\AgdaSymbol{(}\AgdaSpace{}%
\AgdaOperator{\AgdaFunction{⟦\AgdaUnderscore{}⟧}}\AgdaSpace{}%
\AgdaSymbol{)}\<%
\\
%
\>[2]\AgdaKeyword{open}\AgdaSpace{}%
\AgdaModule{SetoidReasoning}\AgdaSpace{}%
\AgdaOperator{\AgdaFunction{𝔻[}}\AgdaSpace{}%
\AgdaBound{𝑨}\AgdaSpace{}%
\AgdaOperator{\AgdaFunction{]}}\<%
\\
%
\>[2]\AgdaFunction{hh}\AgdaSpace{}%
\AgdaSymbol{:}\AgdaSpace{}%
\AgdaFunction{hom}\AgdaSpace{}%
\AgdaBound{𝑩}\AgdaSpace{}%
\AgdaBound{𝑨}\<%
\\
%
\>[2]\AgdaFunction{hh}\AgdaSpace{}%
\AgdaSymbol{=}\AgdaSpace{}%
\AgdaOperator{\AgdaFunction{∣}}\AgdaSpace{}%
\AgdaBound{B≤A}\AgdaSpace{}%
\AgdaOperator{\AgdaFunction{∣}}\<%
\\
%
\>[2]\AgdaFunction{h}\AgdaSpace{}%
\AgdaSymbol{=}\AgdaSpace{}%
\AgdaOperator{\AgdaField{\AgdaUnderscore{}⟨\$⟩\AgdaUnderscore{}}}\AgdaSpace{}%
\AgdaOperator{\AgdaFunction{∣}}\AgdaSpace{}%
\AgdaFunction{hh}\AgdaSpace{}%
\AgdaOperator{\AgdaFunction{∣}}\<%
\\
%
\>[2]\AgdaFunction{ξ}\AgdaSpace{}%
\AgdaSymbol{:}\AgdaSpace{}%
\AgdaSymbol{∀}\AgdaSpace{}%
\AgdaBound{b}\AgdaSpace{}%
\AgdaSymbol{→}\AgdaSpace{}%
\AgdaFunction{h}\AgdaSpace{}%
\AgdaSymbol{(}\AgdaOperator{\AgdaFunction{⟦}}\AgdaSpace{}%
\AgdaBound{p}\AgdaSpace{}%
\AgdaOperator{\AgdaFunction{⟧}}\AgdaSpace{}%
\AgdaOperator{\AgdaField{⟨\$⟩}}\AgdaSpace{}%
\AgdaBound{b}\AgdaSymbol{)}\AgdaSpace{}%
\AgdaOperator{\AgdaFunction{≈}}\AgdaSpace{}%
\AgdaFunction{h}\AgdaSpace{}%
\AgdaSymbol{(}\AgdaOperator{\AgdaFunction{⟦}}\AgdaSpace{}%
\AgdaBound{q}\AgdaSpace{}%
\AgdaOperator{\AgdaFunction{⟧}}\AgdaSpace{}%
\AgdaOperator{\AgdaField{⟨\$⟩}}\AgdaSpace{}%
\AgdaBound{b}\AgdaSymbol{)}\<%
\\
%
\>[2]\AgdaFunction{ξ}\AgdaSpace{}%
\AgdaBound{b}\AgdaSpace{}%
\AgdaSymbol{=}%
\>[5444I]\AgdaOperator{\AgdaFunction{begin}}\<%
\\
\>[5444I][@{}l@{\AgdaIndent{0}}]%
\>[9]\AgdaFunction{h}\AgdaSpace{}%
\AgdaSymbol{(}\AgdaOperator{\AgdaFunction{⟦}}\AgdaSpace{}%
\AgdaBound{p}\AgdaSpace{}%
\AgdaOperator{\AgdaFunction{⟧}}\AgdaSpace{}%
\AgdaOperator{\AgdaField{⟨\$⟩}}\AgdaSpace{}%
\AgdaBound{b}\AgdaSymbol{)}%
\>[29]\AgdaFunction{≈⟨}\AgdaSpace{}%
\AgdaFunction{comm-hom-term}\AgdaSpace{}%
\AgdaFunction{hh}\AgdaSpace{}%
\AgdaBound{p}\AgdaSpace{}%
\AgdaBound{b}%
\>[55]\AgdaFunction{⟩}\<%
\\
%
\>[9]\AgdaOperator{\AgdaFunction{⟦}}\AgdaSpace{}%
\AgdaBound{p}\AgdaSpace{}%
\AgdaOperator{\AgdaFunction{⟧ᵃ}}\AgdaSpace{}%
\AgdaOperator{\AgdaField{⟨\$⟩}}\AgdaSpace{}%
\AgdaSymbol{(}\AgdaFunction{h}\AgdaSpace{}%
\AgdaOperator{\AgdaFunction{∘}}\AgdaSpace{}%
\AgdaBound{b}\AgdaSymbol{)}%
\>[29]\AgdaFunction{≈⟨}\AgdaSpace{}%
\AgdaBound{Apq}\AgdaSpace{}%
\AgdaSymbol{(}\AgdaFunction{h}\AgdaSpace{}%
\AgdaOperator{\AgdaFunction{∘}}\AgdaSpace{}%
\AgdaBound{b}\AgdaSymbol{)}%
\>[55]\AgdaFunction{⟩}\<%
\\
%
\>[9]\AgdaOperator{\AgdaFunction{⟦}}\AgdaSpace{}%
\AgdaBound{q}\AgdaSpace{}%
\AgdaOperator{\AgdaFunction{⟧ᵃ}}\AgdaSpace{}%
\AgdaOperator{\AgdaField{⟨\$⟩}}\AgdaSpace{}%
\AgdaSymbol{(}\AgdaFunction{h}\AgdaSpace{}%
\AgdaOperator{\AgdaFunction{∘}}\AgdaSpace{}%
\AgdaBound{b}\AgdaSymbol{)}%
\>[29]\AgdaFunction{≈˘⟨}\AgdaSpace{}%
\AgdaFunction{comm-hom-term}\AgdaSpace{}%
\AgdaFunction{hh}\AgdaSpace{}%
\AgdaBound{q}\AgdaSpace{}%
\AgdaBound{b}%
\>[55]\AgdaFunction{⟩}\<%
\\
%
\>[9]\AgdaFunction{h}\AgdaSpace{}%
\AgdaSymbol{(}\AgdaOperator{\AgdaFunction{⟦}}\AgdaSpace{}%
\AgdaBound{q}\AgdaSpace{}%
\AgdaOperator{\AgdaFunction{⟧}}\AgdaSpace{}%
\AgdaOperator{\AgdaField{⟨\$⟩}}\AgdaSpace{}%
\AgdaBound{b}\AgdaSymbol{)}%
\>[29]\AgdaOperator{\AgdaFunction{∎}}\<%
\\
%
\>[2]\AgdaFunction{goal}\AgdaSpace{}%
\AgdaSymbol{:}\AgdaSpace{}%
\AgdaOperator{\AgdaFunction{⟦}}\AgdaSpace{}%
\AgdaBound{p}\AgdaSpace{}%
\AgdaOperator{\AgdaFunction{⟧}}\AgdaSpace{}%
\AgdaOperator{\AgdaField{⟨\$⟩}}\AgdaSpace{}%
\AgdaBound{b}\AgdaSpace{}%
\AgdaOperator{\AgdaFunction{≈ᵇ}}\AgdaSpace{}%
\AgdaOperator{\AgdaFunction{⟦}}\AgdaSpace{}%
\AgdaBound{q}\AgdaSpace{}%
\AgdaOperator{\AgdaFunction{⟧}}\AgdaSpace{}%
\AgdaOperator{\AgdaField{⟨\$⟩}}\AgdaSpace{}%
\AgdaBound{b}\<%
\\
%
\>[2]\AgdaFunction{goal}\AgdaSpace{}%
\AgdaSymbol{=}\AgdaSpace{}%
\AgdaOperator{\AgdaFunction{∥}}\AgdaSpace{}%
\AgdaBound{B≤A}\AgdaSpace{}%
\AgdaOperator{\AgdaFunction{∥}}\AgdaSpace{}%
\AgdaSymbol{(}\AgdaFunction{ξ}\AgdaSpace{}%
\AgdaBound{b}\AgdaSymbol{)}\<%
\end{code}

\hypertarget{product-invariance-of}{%
\paragraph{Product invariance of ⊧}\label{product-invariance-of}}

An identity satisfied by all algebras in an indexed collection is also
satisfied by the product of algebras in that collection.

\begin{code}%
\>[0]\<%
\\
%
\\[\AgdaEmptyExtraSkip]%
\>[0]\AgdaKeyword{module}\AgdaSpace{}%
\AgdaModule{\AgdaUnderscore{}}\AgdaSpace{}%
\AgdaSymbol{\{}\AgdaBound{X}\AgdaSpace{}%
\AgdaSymbol{:}\AgdaSpace{}%
\AgdaPrimitive{Type}\AgdaSpace{}%
\AgdaGeneralizable{χ}\AgdaSymbol{\}\{}\AgdaBound{I}\AgdaSpace{}%
\AgdaSymbol{:}\AgdaSpace{}%
\AgdaPrimitive{Type}\AgdaSpace{}%
\AgdaGeneralizable{ℓ}\AgdaSymbol{\}(}\AgdaBound{𝒜}\AgdaSpace{}%
\AgdaSymbol{:}\AgdaSpace{}%
\AgdaBound{I}\AgdaSpace{}%
\AgdaSymbol{→}\AgdaSpace{}%
\AgdaRecord{Algebra}\AgdaSpace{}%
\AgdaGeneralizable{α}\AgdaSpace{}%
\AgdaGeneralizable{ρᵃ}\AgdaSymbol{)\{}\AgdaBound{p}\AgdaSpace{}%
\AgdaBound{q}\AgdaSpace{}%
\AgdaSymbol{:}\AgdaSpace{}%
\AgdaDatatype{Term}\AgdaSpace{}%
\AgdaBound{X}\AgdaSymbol{\}}\AgdaSpace{}%
\AgdaKeyword{where}\<%
\\
%
\\[\AgdaEmptyExtraSkip]%
\>[0][@{}l@{\AgdaIndent{0}}]%
\>[1]\AgdaFunction{⊧-P-invar}\AgdaSpace{}%
\AgdaSymbol{:}\AgdaSpace{}%
\AgdaSymbol{(∀}\AgdaSpace{}%
\AgdaBound{i}\AgdaSpace{}%
\AgdaSymbol{→}\AgdaSpace{}%
\AgdaBound{𝒜}\AgdaSpace{}%
\AgdaBound{i}\AgdaSpace{}%
\AgdaOperator{\AgdaFunction{⊧}}\AgdaSpace{}%
\AgdaSymbol{(}\AgdaBound{p}\AgdaSpace{}%
\AgdaOperator{\AgdaInductiveConstructor{≈̇}}\AgdaSpace{}%
\AgdaBound{q}\AgdaSymbol{))}\AgdaSpace{}%
\AgdaSymbol{→}\AgdaSpace{}%
\AgdaFunction{⨅}\AgdaSpace{}%
\AgdaBound{𝒜}\AgdaSpace{}%
\AgdaOperator{\AgdaFunction{⊧}}\AgdaSpace{}%
\AgdaSymbol{(}\AgdaBound{p}\AgdaSpace{}%
\AgdaOperator{\AgdaInductiveConstructor{≈̇}}\AgdaSpace{}%
\AgdaBound{q}\AgdaSymbol{)}\<%
\\
%
\>[1]\AgdaFunction{⊧-P-invar}\AgdaSpace{}%
\AgdaBound{𝒜pq}\AgdaSpace{}%
\AgdaBound{a}\AgdaSpace{}%
\AgdaSymbol{=}\AgdaSpace{}%
\AgdaFunction{goal}\<%
\\
\>[1][@{}l@{\AgdaIndent{0}}]%
\>[2]\AgdaKeyword{where}\<%
\\
%
\>[2]\AgdaKeyword{open}\AgdaSpace{}%
\AgdaModule{Environment}\AgdaSpace{}%
\AgdaSymbol{(}\AgdaFunction{⨅}\AgdaSpace{}%
\AgdaBound{𝒜}\AgdaSymbol{)}\AgdaSpace{}%
\AgdaKeyword{using}\AgdaSpace{}%
\AgdaSymbol{()}\AgdaSpace{}%
\AgdaKeyword{renaming}\AgdaSpace{}%
\AgdaSymbol{(}\AgdaSpace{}%
\AgdaOperator{\AgdaFunction{⟦\AgdaUnderscore{}⟧}}\AgdaSpace{}%
\AgdaSymbol{to}\AgdaSpace{}%
\AgdaOperator{\AgdaFunction{⟦\AgdaUnderscore{}⟧₁}}\AgdaSpace{}%
\AgdaSymbol{)}\<%
\\
%
\>[2]\AgdaKeyword{open}\AgdaSpace{}%
\AgdaModule{Environment}\AgdaSpace{}%
\AgdaKeyword{using}\AgdaSpace{}%
\AgdaSymbol{(}\AgdaSpace{}%
\AgdaOperator{\AgdaFunction{⟦\AgdaUnderscore{}⟧}}\AgdaSpace{}%
\AgdaSymbol{)}\<%
\\
%
\>[2]\AgdaKeyword{open}\AgdaSpace{}%
\AgdaModule{Setoid}\AgdaSpace{}%
\AgdaOperator{\AgdaFunction{𝔻[}}\AgdaSpace{}%
\AgdaFunction{⨅}\AgdaSpace{}%
\AgdaBound{𝒜}\AgdaSpace{}%
\AgdaOperator{\AgdaFunction{]}}\AgdaSpace{}%
\AgdaKeyword{using}\AgdaSpace{}%
\AgdaSymbol{(}\AgdaSpace{}%
\AgdaOperator{\AgdaField{\AgdaUnderscore{}≈\AgdaUnderscore{}}}\AgdaSpace{}%
\AgdaSymbol{)}\<%
\\
%
\>[2]\AgdaKeyword{open}\AgdaSpace{}%
\AgdaModule{SetoidReasoning}\AgdaSpace{}%
\AgdaOperator{\AgdaFunction{𝔻[}}\AgdaSpace{}%
\AgdaFunction{⨅}\AgdaSpace{}%
\AgdaBound{𝒜}\AgdaSpace{}%
\AgdaOperator{\AgdaFunction{]}}\<%
\\
%
\>[2]\AgdaFunction{ξ}\AgdaSpace{}%
\AgdaSymbol{:}\AgdaSpace{}%
\AgdaSymbol{(λ}\AgdaSpace{}%
\AgdaBound{i}\AgdaSpace{}%
\AgdaSymbol{→}\AgdaSpace{}%
\AgdaSymbol{(}\AgdaOperator{\AgdaFunction{⟦}}\AgdaSpace{}%
\AgdaBound{𝒜}\AgdaSpace{}%
\AgdaBound{i}\AgdaSpace{}%
\AgdaOperator{\AgdaFunction{⟧}}\AgdaSpace{}%
\AgdaBound{p}\AgdaSymbol{)}\AgdaSpace{}%
\AgdaOperator{\AgdaField{⟨\$⟩}}\AgdaSpace{}%
\AgdaSymbol{(λ}\AgdaSpace{}%
\AgdaBound{x}\AgdaSpace{}%
\AgdaSymbol{→}\AgdaSpace{}%
\AgdaSymbol{(}\AgdaBound{a}\AgdaSpace{}%
\AgdaBound{x}\AgdaSymbol{)}\AgdaSpace{}%
\AgdaBound{i}\AgdaSymbol{))}\AgdaSpace{}%
\AgdaOperator{\AgdaFunction{≈}}\AgdaSpace{}%
\AgdaSymbol{(λ}\AgdaSpace{}%
\AgdaBound{i}\AgdaSpace{}%
\AgdaSymbol{→}\AgdaSpace{}%
\AgdaSymbol{(}\AgdaOperator{\AgdaFunction{⟦}}\AgdaSpace{}%
\AgdaBound{𝒜}\AgdaSpace{}%
\AgdaBound{i}\AgdaSpace{}%
\AgdaOperator{\AgdaFunction{⟧}}\AgdaSpace{}%
\AgdaBound{q}\AgdaSymbol{)}\AgdaSpace{}%
\AgdaOperator{\AgdaField{⟨\$⟩}}\AgdaSpace{}%
\AgdaSymbol{(λ}\AgdaSpace{}%
\AgdaBound{x}\AgdaSpace{}%
\AgdaSymbol{→}\AgdaSpace{}%
\AgdaSymbol{(}\AgdaBound{a}\AgdaSpace{}%
\AgdaBound{x}\AgdaSymbol{)}\AgdaSpace{}%
\AgdaBound{i}\AgdaSymbol{))}\<%
\\
%
\>[2]\AgdaFunction{ξ}\AgdaSpace{}%
\AgdaSymbol{=}\AgdaSpace{}%
\AgdaSymbol{λ}\AgdaSpace{}%
\AgdaBound{i}\AgdaSpace{}%
\AgdaSymbol{→}\AgdaSpace{}%
\AgdaBound{𝒜pq}\AgdaSpace{}%
\AgdaBound{i}\AgdaSpace{}%
\AgdaSymbol{(λ}\AgdaSpace{}%
\AgdaBound{x}\AgdaSpace{}%
\AgdaSymbol{→}\AgdaSpace{}%
\AgdaSymbol{(}\AgdaBound{a}\AgdaSpace{}%
\AgdaBound{x}\AgdaSymbol{)}\AgdaSpace{}%
\AgdaBound{i}\AgdaSymbol{)}\<%
\\
%
\>[2]\AgdaFunction{goal}\AgdaSpace{}%
\AgdaSymbol{:}\AgdaSpace{}%
\AgdaOperator{\AgdaFunction{⟦}}\AgdaSpace{}%
\AgdaBound{p}\AgdaSpace{}%
\AgdaOperator{\AgdaFunction{⟧₁}}\AgdaSpace{}%
\AgdaOperator{\AgdaField{⟨\$⟩}}\AgdaSpace{}%
\AgdaBound{a}\AgdaSpace{}%
\AgdaOperator{\AgdaFunction{≈}}\AgdaSpace{}%
\AgdaOperator{\AgdaFunction{⟦}}\AgdaSpace{}%
\AgdaBound{q}\AgdaSpace{}%
\AgdaOperator{\AgdaFunction{⟧₁}}\AgdaSpace{}%
\AgdaOperator{\AgdaField{⟨\$⟩}}\AgdaSpace{}%
\AgdaBound{a}\<%
\\
%
\>[2]\AgdaFunction{goal}\AgdaSpace{}%
\AgdaSymbol{=}%
\>[5624I]\AgdaOperator{\AgdaFunction{begin}}\<%
\\
\>[5624I][@{}l@{\AgdaIndent{0}}]%
\>[10]\AgdaOperator{\AgdaFunction{⟦}}\AgdaSpace{}%
\AgdaBound{p}\AgdaSpace{}%
\AgdaOperator{\AgdaFunction{⟧₁}}\AgdaSpace{}%
\AgdaOperator{\AgdaField{⟨\$⟩}}\AgdaSpace{}%
\AgdaBound{a}%
\>[51]\AgdaFunction{≈⟨}\AgdaSpace{}%
\AgdaFunction{interp-prod}\AgdaSpace{}%
\AgdaBound{𝒜}\AgdaSpace{}%
\AgdaBound{p}\AgdaSpace{}%
\AgdaBound{a}%
\>[74]\AgdaFunction{⟩}\<%
\\
%
\>[10]\AgdaSymbol{(λ}\AgdaSpace{}%
\AgdaBound{i}\AgdaSpace{}%
\AgdaSymbol{→}\AgdaSpace{}%
\AgdaSymbol{(}\AgdaOperator{\AgdaFunction{⟦}}\AgdaSpace{}%
\AgdaBound{𝒜}\AgdaSpace{}%
\AgdaBound{i}\AgdaSpace{}%
\AgdaOperator{\AgdaFunction{⟧}}\AgdaSpace{}%
\AgdaBound{p}\AgdaSymbol{)}\AgdaSpace{}%
\AgdaOperator{\AgdaField{⟨\$⟩}}\AgdaSpace{}%
\AgdaSymbol{(λ}\AgdaSpace{}%
\AgdaBound{x}\AgdaSpace{}%
\AgdaSymbol{→}\AgdaSpace{}%
\AgdaSymbol{(}\AgdaBound{a}\AgdaSpace{}%
\AgdaBound{x}\AgdaSymbol{)}\AgdaSpace{}%
\AgdaBound{i}\AgdaSymbol{))}%
\>[51]\AgdaFunction{≈⟨}\AgdaSpace{}%
\AgdaFunction{ξ}%
\>[74]\AgdaFunction{⟩}\<%
\\
%
\>[10]\AgdaSymbol{(λ}\AgdaSpace{}%
\AgdaBound{i}\AgdaSpace{}%
\AgdaSymbol{→}\AgdaSpace{}%
\AgdaSymbol{(}\AgdaOperator{\AgdaFunction{⟦}}\AgdaSpace{}%
\AgdaBound{𝒜}\AgdaSpace{}%
\AgdaBound{i}\AgdaSpace{}%
\AgdaOperator{\AgdaFunction{⟧}}\AgdaSpace{}%
\AgdaBound{q}\AgdaSymbol{)}\AgdaSpace{}%
\AgdaOperator{\AgdaField{⟨\$⟩}}\AgdaSpace{}%
\AgdaSymbol{(λ}\AgdaSpace{}%
\AgdaBound{x}\AgdaSpace{}%
\AgdaSymbol{→}\AgdaSpace{}%
\AgdaSymbol{(}\AgdaBound{a}\AgdaSpace{}%
\AgdaBound{x}\AgdaSymbol{)}\AgdaSpace{}%
\AgdaBound{i}\AgdaSymbol{))}%
\>[51]\AgdaFunction{≈˘⟨}\AgdaSpace{}%
\AgdaFunction{interp-prod}\AgdaSpace{}%
\AgdaBound{𝒜}\AgdaSpace{}%
\AgdaBound{q}\AgdaSpace{}%
\AgdaBound{a}%
\>[74]\AgdaFunction{⟩}\<%
\\
%
\>[10]\AgdaOperator{\AgdaFunction{⟦}}\AgdaSpace{}%
\AgdaBound{q}\AgdaSpace{}%
\AgdaOperator{\AgdaFunction{⟧₁}}\AgdaSpace{}%
\AgdaOperator{\AgdaField{⟨\$⟩}}\AgdaSpace{}%
\AgdaBound{a}%
\>[51]\AgdaOperator{\AgdaFunction{∎}}\<%
\end{code}

\hypertarget{ps-sp}{%
\paragraph{PS ⊆ SP}\label{ps-sp}}

Another important fact we will need about the operators \texttt{S} and
\texttt{P} is that a product of subalgebras of algebras in a class
\texttt{𝒦} is a subalgebra of a product of algebras in \texttt{𝒦}. We
denote this inclusion by \texttt{PS⊆SP}, which we state and prove as
follows.

\begin{code}%
\>[0]\<%
\\
\>[0]\AgdaKeyword{module}\AgdaSpace{}%
\AgdaModule{\AgdaUnderscore{}}%
\>[10]\AgdaSymbol{\{}\AgdaBound{𝒦}\AgdaSpace{}%
\AgdaSymbol{:}\AgdaSpace{}%
\AgdaFunction{Pred}\AgdaSymbol{(}\AgdaRecord{Algebra}\AgdaSpace{}%
\AgdaGeneralizable{α}\AgdaSpace{}%
\AgdaGeneralizable{ρᵃ}\AgdaSymbol{)}\AgdaSpace{}%
\AgdaSymbol{(}\AgdaGeneralizable{α}\AgdaSpace{}%
\AgdaOperator{\AgdaPrimitive{⊔}}\AgdaSpace{}%
\AgdaGeneralizable{ρᵃ}\AgdaSpace{}%
\AgdaOperator{\AgdaPrimitive{⊔}}\AgdaSpace{}%
\AgdaFunction{ov}\AgdaSpace{}%
\AgdaGeneralizable{ℓ}\AgdaSymbol{)\}}\AgdaSpace{}%
\AgdaKeyword{where}\<%
\\
\>[0][@{}l@{\AgdaIndent{0}}]%
\>[1]\AgdaKeyword{private}\<%
\\
\>[1][@{}l@{\AgdaIndent{0}}]%
\>[2]\AgdaFunction{a}\AgdaSpace{}%
\AgdaSymbol{=}\AgdaSpace{}%
\AgdaBound{α}\AgdaSpace{}%
\AgdaOperator{\AgdaPrimitive{⊔}}\AgdaSpace{}%
\AgdaBound{ρᵃ}\<%
\\
%
\>[2]\AgdaFunction{oaℓ}\AgdaSpace{}%
\AgdaSymbol{=}\AgdaSpace{}%
\AgdaFunction{ov}\AgdaSpace{}%
\AgdaSymbol{(}\AgdaFunction{a}\AgdaSpace{}%
\AgdaOperator{\AgdaPrimitive{⊔}}\AgdaSpace{}%
\AgdaBound{ℓ}\AgdaSymbol{)}\<%
\\
%
\\[\AgdaEmptyExtraSkip]%
%
\>[1]\AgdaFunction{PS⊆SP}\AgdaSpace{}%
\AgdaSymbol{:}\AgdaSpace{}%
\AgdaFunction{P}\AgdaSpace{}%
\AgdaSymbol{(}\AgdaFunction{a}\AgdaSpace{}%
\AgdaOperator{\AgdaPrimitive{⊔}}\AgdaSpace{}%
\AgdaBound{ℓ}\AgdaSymbol{)}\AgdaSpace{}%
\AgdaFunction{oaℓ}\AgdaSpace{}%
\AgdaSymbol{(}\AgdaFunction{S}\AgdaSymbol{\{}\AgdaArgument{β}\AgdaSpace{}%
\AgdaSymbol{=}\AgdaSpace{}%
\AgdaBound{α}\AgdaSymbol{\}\{}\AgdaBound{ρᵃ}\AgdaSymbol{\}}\AgdaSpace{}%
\AgdaBound{ℓ}\AgdaSpace{}%
\AgdaBound{𝒦}\AgdaSymbol{)}\AgdaSpace{}%
\AgdaOperator{\AgdaFunction{⊆}}\AgdaSpace{}%
\AgdaFunction{S}\AgdaSpace{}%
\AgdaFunction{oaℓ}\AgdaSpace{}%
\AgdaSymbol{(}\AgdaFunction{P}\AgdaSpace{}%
\AgdaBound{ℓ}\AgdaSpace{}%
\AgdaFunction{oaℓ}\AgdaSpace{}%
\AgdaBound{𝒦}\AgdaSymbol{)}\<%
\\
%
\>[1]\AgdaFunction{PS⊆SP}\AgdaSpace{}%
\AgdaSymbol{\{}\AgdaBound{𝑩}\AgdaSymbol{\}}\AgdaSpace{}%
\AgdaSymbol{(}\AgdaBound{I}\AgdaSpace{}%
\AgdaOperator{\AgdaInductiveConstructor{,}}\AgdaSpace{}%
\AgdaSymbol{(}\AgdaSpace{}%
\AgdaBound{𝒜}\AgdaSpace{}%
\AgdaOperator{\AgdaInductiveConstructor{,}}\AgdaSpace{}%
\AgdaBound{sA}\AgdaSpace{}%
\AgdaOperator{\AgdaInductiveConstructor{,}}\AgdaSpace{}%
\AgdaBound{B≅⨅A}\AgdaSpace{}%
\AgdaSymbol{))}\AgdaSpace{}%
\AgdaSymbol{=}\AgdaSpace{}%
\AgdaFunction{Goal}\<%
\\
\>[1][@{}l@{\AgdaIndent{0}}]%
\>[2]\AgdaKeyword{where}\<%
\\
%
\>[2]\AgdaFunction{ℬ}\AgdaSpace{}%
\AgdaSymbol{:}\AgdaSpace{}%
\AgdaBound{I}\AgdaSpace{}%
\AgdaSymbol{→}\AgdaSpace{}%
\AgdaRecord{Algebra}\AgdaSpace{}%
\AgdaBound{α}\AgdaSpace{}%
\AgdaBound{ρᵃ}\<%
\\
%
\>[2]\AgdaFunction{ℬ}\AgdaSpace{}%
\AgdaBound{i}\AgdaSpace{}%
\AgdaSymbol{=}\AgdaSpace{}%
\AgdaOperator{\AgdaFunction{∣}}\AgdaSpace{}%
\AgdaBound{sA}\AgdaSpace{}%
\AgdaBound{i}\AgdaSpace{}%
\AgdaOperator{\AgdaFunction{∣}}\<%
\\
%
\>[2]\AgdaFunction{kB}\AgdaSpace{}%
\AgdaSymbol{:}\AgdaSpace{}%
\AgdaSymbol{(}\AgdaBound{i}\AgdaSpace{}%
\AgdaSymbol{:}\AgdaSpace{}%
\AgdaBound{I}\AgdaSymbol{)}\AgdaSpace{}%
\AgdaSymbol{→}\AgdaSpace{}%
\AgdaFunction{ℬ}\AgdaSpace{}%
\AgdaBound{i}\AgdaSpace{}%
\AgdaOperator{\AgdaFunction{∈}}\AgdaSpace{}%
\AgdaBound{𝒦}\<%
\\
%
\>[2]\AgdaFunction{kB}\AgdaSpace{}%
\AgdaBound{i}\AgdaSpace{}%
\AgdaSymbol{=}%
\>[10]\AgdaField{fst}\AgdaSpace{}%
\AgdaOperator{\AgdaFunction{∥}}\AgdaSpace{}%
\AgdaBound{sA}\AgdaSpace{}%
\AgdaBound{i}\AgdaSpace{}%
\AgdaOperator{\AgdaFunction{∥}}\<%
\\
%
\>[2]\AgdaFunction{⨅A≤⨅B}\AgdaSpace{}%
\AgdaSymbol{:}\AgdaSpace{}%
\AgdaFunction{⨅}\AgdaSpace{}%
\AgdaBound{𝒜}\AgdaSpace{}%
\AgdaOperator{\AgdaFunction{≤}}\AgdaSpace{}%
\AgdaFunction{⨅}\AgdaSpace{}%
\AgdaFunction{ℬ}\<%
\\
%
\>[2]\AgdaFunction{⨅A≤⨅B}\AgdaSpace{}%
\AgdaSymbol{=}\AgdaSpace{}%
\AgdaFunction{⨅-≤}\AgdaSpace{}%
\AgdaSymbol{λ}\AgdaSpace{}%
\AgdaBound{i}\AgdaSpace{}%
\AgdaSymbol{→}\AgdaSpace{}%
\AgdaField{snd}\AgdaSpace{}%
\AgdaOperator{\AgdaFunction{∥}}\AgdaSpace{}%
\AgdaBound{sA}\AgdaSpace{}%
\AgdaBound{i}\AgdaSpace{}%
\AgdaOperator{\AgdaFunction{∥}}\<%
\\
%
\>[2]\AgdaFunction{Goal}\AgdaSpace{}%
\AgdaSymbol{:}\AgdaSpace{}%
\AgdaBound{𝑩}\AgdaSpace{}%
\AgdaOperator{\AgdaFunction{∈}}\AgdaSpace{}%
\AgdaFunction{S}\AgdaSymbol{\{}\AgdaArgument{β}\AgdaSpace{}%
\AgdaSymbol{=}\AgdaSpace{}%
\AgdaFunction{oaℓ}\AgdaSymbol{\}\{}\AgdaFunction{oaℓ}\AgdaSymbol{\}}\AgdaFunction{oaℓ}\AgdaSpace{}%
\AgdaSymbol{(}\AgdaFunction{P}\AgdaSpace{}%
\AgdaSymbol{\{}\AgdaArgument{β}\AgdaSpace{}%
\AgdaSymbol{=}\AgdaSpace{}%
\AgdaFunction{oaℓ}\AgdaSymbol{\}\{}\AgdaFunction{oaℓ}\AgdaSymbol{\}}\AgdaSpace{}%
\AgdaBound{ℓ}\AgdaSpace{}%
\AgdaFunction{oaℓ}\AgdaSpace{}%
\AgdaBound{𝒦}\AgdaSymbol{)}\<%
\\
%
\>[2]\AgdaFunction{Goal}\AgdaSpace{}%
\AgdaSymbol{=}\AgdaSpace{}%
\AgdaFunction{⨅}\AgdaSpace{}%
\AgdaFunction{ℬ}\AgdaSpace{}%
\AgdaOperator{\AgdaInductiveConstructor{,}}\AgdaSpace{}%
\AgdaSymbol{(}\AgdaBound{I}\AgdaSpace{}%
\AgdaOperator{\AgdaInductiveConstructor{,}}\AgdaSpace{}%
\AgdaSymbol{(}\AgdaFunction{ℬ}\AgdaSpace{}%
\AgdaOperator{\AgdaInductiveConstructor{,}}\AgdaSpace{}%
\AgdaSymbol{(}\AgdaFunction{kB}\AgdaSpace{}%
\AgdaOperator{\AgdaInductiveConstructor{,}}\AgdaSpace{}%
\AgdaFunction{≅-refl}\AgdaSymbol{)))}\AgdaSpace{}%
\AgdaOperator{\AgdaInductiveConstructor{,}}\AgdaSpace{}%
\AgdaSymbol{(}\AgdaFunction{≅-trans-≤}\AgdaSpace{}%
\AgdaBound{B≅⨅A}\AgdaSpace{}%
\AgdaFunction{⨅A≤⨅B}\AgdaSymbol{)}\<%
\end{code}

\hypertarget{identity-preservation}{%
\paragraph{Identity preservation}\label{identity-preservation}}

The classes \texttt{H\ 𝒦}, \texttt{S\ 𝒦}, \texttt{P\ 𝒦}, and
\texttt{V\ 𝒦} all satisfy the same set of equations. We will only use a
subset of the inclusions used to prove this fact. (For a complete proof,
see the
\href{https://ualib.github.io/agda-algebras/Varieties.Func.Preservation.html}{Varieties.Func.Preservation}
module of the \href{https://ualib.github.io/agda-algebras}{Agda
Universal Algebra Library}.)

\hypertarget{h-preserves-identities}{%
\subparagraph{H preserves identities}\label{h-preserves-identities}}

First we prove that the closure operator H is compatible with identities
that hold in the given class.

\begin{code}%
\>[0]\<%
\\
\>[0]\AgdaKeyword{module}\AgdaSpace{}%
\AgdaModule{\AgdaUnderscore{}}%
\>[10]\AgdaSymbol{\{}\AgdaBound{X}\AgdaSpace{}%
\AgdaSymbol{:}\AgdaSpace{}%
\AgdaPrimitive{Type}\AgdaSpace{}%
\AgdaGeneralizable{χ}\AgdaSymbol{\}\{}\AgdaBound{𝒦}\AgdaSpace{}%
\AgdaSymbol{:}\AgdaSpace{}%
\AgdaFunction{Pred}\AgdaSymbol{(}\AgdaRecord{Algebra}\AgdaSpace{}%
\AgdaGeneralizable{α}\AgdaSpace{}%
\AgdaGeneralizable{ρᵃ}\AgdaSymbol{)}\AgdaSpace{}%
\AgdaSymbol{(}\AgdaGeneralizable{α}\AgdaSpace{}%
\AgdaOperator{\AgdaPrimitive{⊔}}\AgdaSpace{}%
\AgdaGeneralizable{ρᵃ}\AgdaSpace{}%
\AgdaOperator{\AgdaPrimitive{⊔}}\AgdaSpace{}%
\AgdaFunction{ov}\AgdaSpace{}%
\AgdaGeneralizable{ℓ}\AgdaSymbol{)\}\{}\AgdaBound{p}\AgdaSpace{}%
\AgdaBound{q}\AgdaSpace{}%
\AgdaSymbol{:}\AgdaSpace{}%
\AgdaDatatype{Term}\AgdaSpace{}%
\AgdaBound{X}\AgdaSymbol{\}}\AgdaSpace{}%
\AgdaKeyword{where}\<%
\\
%
\\[\AgdaEmptyExtraSkip]%
\>[0][@{}l@{\AgdaIndent{0}}]%
\>[1]\AgdaFunction{H-id1}\AgdaSpace{}%
\AgdaSymbol{:}\AgdaSpace{}%
\AgdaBound{𝒦}\AgdaSpace{}%
\AgdaOperator{\AgdaFunction{⊫}}\AgdaSpace{}%
\AgdaSymbol{(}\AgdaBound{p}\AgdaSpace{}%
\AgdaOperator{\AgdaInductiveConstructor{≈̇}}\AgdaSpace{}%
\AgdaBound{q}\AgdaSymbol{)}\AgdaSpace{}%
\AgdaSymbol{→}\AgdaSpace{}%
\AgdaSymbol{(}\AgdaFunction{H}\AgdaSpace{}%
\AgdaSymbol{\{}\AgdaArgument{β}\AgdaSpace{}%
\AgdaSymbol{=}\AgdaSpace{}%
\AgdaBound{α}\AgdaSymbol{\}\{}\AgdaBound{ρᵃ}\AgdaSymbol{\}}\AgdaBound{ℓ}\AgdaSpace{}%
\AgdaBound{𝒦}\AgdaSymbol{)}\AgdaSpace{}%
\AgdaOperator{\AgdaFunction{⊫}}\AgdaSpace{}%
\AgdaSymbol{(}\AgdaBound{p}\AgdaSpace{}%
\AgdaOperator{\AgdaInductiveConstructor{≈̇}}\AgdaSpace{}%
\AgdaBound{q}\AgdaSymbol{)}\<%
\\
%
\>[1]\AgdaFunction{H-id1}\AgdaSpace{}%
\AgdaBound{σ}\AgdaSpace{}%
\AgdaBound{𝑩}\AgdaSpace{}%
\AgdaSymbol{(}\AgdaBound{𝑨}\AgdaSpace{}%
\AgdaOperator{\AgdaInductiveConstructor{,}}\AgdaSpace{}%
\AgdaBound{kA}\AgdaSpace{}%
\AgdaOperator{\AgdaInductiveConstructor{,}}\AgdaSpace{}%
\AgdaBound{BimgOfA}\AgdaSymbol{)}\AgdaSpace{}%
\AgdaBound{ρ}\AgdaSpace{}%
\AgdaSymbol{=}\AgdaSpace{}%
\AgdaFunction{B⊧pq}\<%
\\
\>[1][@{}l@{\AgdaIndent{0}}]%
\>[2]\AgdaKeyword{where}\<%
\\
%
\>[2]\AgdaFunction{IH}\AgdaSpace{}%
\AgdaSymbol{:}\AgdaSpace{}%
\AgdaBound{𝑨}\AgdaSpace{}%
\AgdaOperator{\AgdaFunction{⊧}}\AgdaSpace{}%
\AgdaSymbol{(}\AgdaBound{p}\AgdaSpace{}%
\AgdaOperator{\AgdaInductiveConstructor{≈̇}}\AgdaSpace{}%
\AgdaBound{q}\AgdaSymbol{)}\<%
\\
%
\>[2]\AgdaFunction{IH}\AgdaSpace{}%
\AgdaSymbol{=}\AgdaSpace{}%
\AgdaBound{σ}\AgdaSpace{}%
\AgdaBound{𝑨}\AgdaSpace{}%
\AgdaBound{kA}\<%
\\
%
\>[2]\AgdaKeyword{open}\AgdaSpace{}%
\AgdaModule{Environment}\AgdaSpace{}%
\AgdaBound{𝑨}\AgdaSpace{}%
\AgdaKeyword{using}\AgdaSpace{}%
\AgdaSymbol{()}\AgdaSpace{}%
\AgdaKeyword{renaming}\AgdaSpace{}%
\AgdaSymbol{(}\AgdaSpace{}%
\AgdaOperator{\AgdaFunction{⟦\AgdaUnderscore{}⟧}}\AgdaSpace{}%
\AgdaSymbol{to}\AgdaSpace{}%
\AgdaOperator{\AgdaFunction{⟦\AgdaUnderscore{}⟧₁}}\AgdaSymbol{)}\<%
\\
%
\>[2]\AgdaKeyword{open}\AgdaSpace{}%
\AgdaModule{Environment}\AgdaSpace{}%
\AgdaBound{𝑩}\AgdaSpace{}%
\AgdaKeyword{using}\AgdaSpace{}%
\AgdaSymbol{(}\AgdaSpace{}%
\AgdaOperator{\AgdaFunction{⟦\AgdaUnderscore{}⟧}}\AgdaSpace{}%
\AgdaSymbol{)}\<%
\\
%
\>[2]\AgdaKeyword{open}\AgdaSpace{}%
\AgdaModule{Setoid}\AgdaSpace{}%
\AgdaOperator{\AgdaFunction{𝔻[}}\AgdaSpace{}%
\AgdaBound{𝑩}\AgdaSpace{}%
\AgdaOperator{\AgdaFunction{]}}\AgdaSpace{}%
\AgdaKeyword{using}\AgdaSpace{}%
\AgdaSymbol{(}\AgdaSpace{}%
\AgdaOperator{\AgdaField{\AgdaUnderscore{}≈\AgdaUnderscore{}}}\AgdaSpace{}%
\AgdaSymbol{)}\<%
\\
%
\>[2]\AgdaKeyword{open}\AgdaSpace{}%
\AgdaModule{SetoidReasoning}\AgdaSpace{}%
\AgdaOperator{\AgdaFunction{𝔻[}}\AgdaSpace{}%
\AgdaBound{𝑩}\AgdaSpace{}%
\AgdaOperator{\AgdaFunction{]}}\<%
\\
%
\\[\AgdaEmptyExtraSkip]%
%
\>[2]\AgdaFunction{φ}\AgdaSpace{}%
\AgdaSymbol{:}\AgdaSpace{}%
\AgdaFunction{hom}\AgdaSpace{}%
\AgdaBound{𝑨}\AgdaSpace{}%
\AgdaBound{𝑩}\<%
\\
%
\>[2]\AgdaFunction{φ}\AgdaSpace{}%
\AgdaSymbol{=}\AgdaSpace{}%
\AgdaOperator{\AgdaFunction{∣}}\AgdaSpace{}%
\AgdaBound{BimgOfA}\AgdaSpace{}%
\AgdaOperator{\AgdaFunction{∣}}\<%
\\
%
\>[2]\AgdaFunction{φE}\AgdaSpace{}%
\AgdaSymbol{:}\AgdaSpace{}%
\AgdaFunction{IsSurjective}\AgdaSpace{}%
\AgdaOperator{\AgdaFunction{∣}}\AgdaSpace{}%
\AgdaFunction{φ}\AgdaSpace{}%
\AgdaOperator{\AgdaFunction{∣}}\<%
\\
%
\>[2]\AgdaFunction{φE}\AgdaSpace{}%
\AgdaSymbol{=}\AgdaSpace{}%
\AgdaOperator{\AgdaFunction{∥}}\AgdaSpace{}%
\AgdaBound{BimgOfA}\AgdaSpace{}%
\AgdaOperator{\AgdaFunction{∥}}\<%
\\
%
\>[2]\AgdaFunction{φ⁻¹}\AgdaSpace{}%
\AgdaSymbol{:}\AgdaSpace{}%
\AgdaOperator{\AgdaFunction{𝕌[}}\AgdaSpace{}%
\AgdaBound{𝑩}\AgdaSpace{}%
\AgdaOperator{\AgdaFunction{]}}\AgdaSpace{}%
\AgdaSymbol{→}\AgdaSpace{}%
\AgdaOperator{\AgdaFunction{𝕌[}}\AgdaSpace{}%
\AgdaBound{𝑨}\AgdaSpace{}%
\AgdaOperator{\AgdaFunction{]}}\<%
\\
%
\>[2]\AgdaFunction{φ⁻¹}\AgdaSpace{}%
\AgdaSymbol{=}\AgdaSpace{}%
\AgdaFunction{SurjInv}\AgdaSpace{}%
\AgdaOperator{\AgdaFunction{∣}}\AgdaSpace{}%
\AgdaFunction{φ}\AgdaSpace{}%
\AgdaOperator{\AgdaFunction{∣}}\AgdaSpace{}%
\AgdaFunction{φE}\<%
\\
%
\\[\AgdaEmptyExtraSkip]%
%
\>[2]\AgdaFunction{ζ}\AgdaSpace{}%
\AgdaSymbol{:}\AgdaSpace{}%
\AgdaSymbol{∀}\AgdaSpace{}%
\AgdaBound{x}\AgdaSpace{}%
\AgdaSymbol{→}\AgdaSpace{}%
\AgdaSymbol{(}\AgdaOperator{\AgdaFunction{∣}}\AgdaSpace{}%
\AgdaFunction{φ}\AgdaSpace{}%
\AgdaOperator{\AgdaFunction{∣}}\AgdaSpace{}%
\AgdaOperator{\AgdaField{⟨\$⟩}}\AgdaSpace{}%
\AgdaSymbol{(}\AgdaFunction{φ⁻¹}\AgdaSpace{}%
\AgdaOperator{\AgdaFunction{∘}}\AgdaSpace{}%
\AgdaBound{ρ}\AgdaSymbol{)}\AgdaSpace{}%
\AgdaBound{x}\AgdaSpace{}%
\AgdaSymbol{)}\AgdaSpace{}%
\AgdaOperator{\AgdaFunction{≈}}\AgdaSpace{}%
\AgdaBound{ρ}\AgdaSpace{}%
\AgdaBound{x}\<%
\\
%
\>[2]\AgdaFunction{ζ}\AgdaSpace{}%
\AgdaSymbol{=}\AgdaSpace{}%
\AgdaSymbol{λ}\AgdaSpace{}%
\AgdaBound{\AgdaUnderscore{}}\AgdaSpace{}%
\AgdaSymbol{→}\AgdaSpace{}%
\AgdaFunction{InvIsInverseʳ}\AgdaSpace{}%
\AgdaFunction{φE}\<%
\\
%
\\[\AgdaEmptyExtraSkip]%
%
\>[2]\AgdaFunction{B⊧pq}\AgdaSpace{}%
\AgdaSymbol{:}\AgdaSpace{}%
\AgdaSymbol{(}\AgdaOperator{\AgdaFunction{⟦}}\AgdaSpace{}%
\AgdaBound{p}\AgdaSpace{}%
\AgdaOperator{\AgdaFunction{⟧}}\AgdaSpace{}%
\AgdaOperator{\AgdaField{⟨\$⟩}}\AgdaSpace{}%
\AgdaBound{ρ}\AgdaSymbol{)}\AgdaSpace{}%
\AgdaOperator{\AgdaFunction{≈}}\AgdaSpace{}%
\AgdaSymbol{(}\AgdaOperator{\AgdaFunction{⟦}}\AgdaSpace{}%
\AgdaBound{q}\AgdaSpace{}%
\AgdaOperator{\AgdaFunction{⟧}}\AgdaSpace{}%
\AgdaOperator{\AgdaField{⟨\$⟩}}\AgdaSpace{}%
\AgdaBound{ρ}\AgdaSymbol{)}\<%
\\
%
\>[2]\AgdaFunction{B⊧pq}\AgdaSpace{}%
\AgdaSymbol{=}%
\>[5940I]\AgdaOperator{\AgdaFunction{begin}}\<%
\\
\>[.][@{}l@{}]\<[5940I]%
\>[9]\AgdaOperator{\AgdaFunction{⟦}}\AgdaSpace{}%
\AgdaBound{p}\AgdaSpace{}%
\AgdaOperator{\AgdaFunction{⟧}}\AgdaSpace{}%
\AgdaOperator{\AgdaField{⟨\$⟩}}\AgdaSpace{}%
\AgdaBound{ρ}%
\>[52]\AgdaFunction{≈˘⟨}\AgdaSpace{}%
\AgdaField{cong}\AgdaSpace{}%
\AgdaOperator{\AgdaFunction{⟦}}\AgdaSpace{}%
\AgdaBound{p}\AgdaSpace{}%
\AgdaOperator{\AgdaFunction{⟧}}\AgdaSpace{}%
\AgdaFunction{ζ}%
\>[85]\AgdaFunction{⟩}\<%
\\
%
\>[9]\AgdaOperator{\AgdaFunction{⟦}}\AgdaSpace{}%
\AgdaBound{p}\AgdaSpace{}%
\AgdaOperator{\AgdaFunction{⟧}}\AgdaSpace{}%
\AgdaOperator{\AgdaField{⟨\$⟩}}\AgdaSpace{}%
\AgdaSymbol{(λ}\AgdaSpace{}%
\AgdaBound{x}\AgdaSpace{}%
\AgdaSymbol{→}\AgdaSpace{}%
\AgdaSymbol{(}\AgdaOperator{\AgdaFunction{∣}}\AgdaSpace{}%
\AgdaFunction{φ}\AgdaSpace{}%
\AgdaOperator{\AgdaFunction{∣}}\AgdaSpace{}%
\AgdaOperator{\AgdaField{⟨\$⟩}}\AgdaSpace{}%
\AgdaSymbol{(}\AgdaFunction{φ⁻¹}\AgdaSpace{}%
\AgdaOperator{\AgdaFunction{∘}}\AgdaSpace{}%
\AgdaBound{ρ}\AgdaSymbol{)}\AgdaSpace{}%
\AgdaBound{x}\AgdaSymbol{))}%
\>[52]\AgdaFunction{≈˘⟨}\AgdaSpace{}%
\AgdaFunction{comm-hom-term}\AgdaSpace{}%
\AgdaFunction{φ}\AgdaSpace{}%
\AgdaBound{p}\AgdaSpace{}%
\AgdaSymbol{(}\AgdaFunction{φ⁻¹}\AgdaSpace{}%
\AgdaOperator{\AgdaFunction{∘}}\AgdaSpace{}%
\AgdaBound{ρ}\AgdaSymbol{)}%
\>[85]\AgdaFunction{⟩}\<%
\\
%
\>[9]\AgdaOperator{\AgdaFunction{∣}}\AgdaSpace{}%
\AgdaFunction{φ}\AgdaSpace{}%
\AgdaOperator{\AgdaFunction{∣}}\AgdaSpace{}%
\AgdaOperator{\AgdaField{⟨\$⟩}}\AgdaSpace{}%
\AgdaSymbol{(}\AgdaOperator{\AgdaFunction{⟦}}\AgdaSpace{}%
\AgdaBound{p}\AgdaSpace{}%
\AgdaOperator{\AgdaFunction{⟧₁}}\AgdaSpace{}%
\AgdaOperator{\AgdaField{⟨\$⟩}}\AgdaSpace{}%
\AgdaSymbol{(}\AgdaFunction{φ⁻¹}\AgdaSpace{}%
\AgdaOperator{\AgdaFunction{∘}}\AgdaSpace{}%
\AgdaBound{ρ}\AgdaSymbol{))}%
\>[52]\AgdaFunction{≈⟨}\AgdaSpace{}%
\AgdaField{cong}\AgdaSpace{}%
\AgdaOperator{\AgdaFunction{∣}}\AgdaSpace{}%
\AgdaFunction{φ}\AgdaSpace{}%
\AgdaOperator{\AgdaFunction{∣}}\AgdaSpace{}%
\AgdaSymbol{(}\AgdaFunction{IH}\AgdaSpace{}%
\AgdaSymbol{(}\AgdaFunction{φ⁻¹}\AgdaSpace{}%
\AgdaOperator{\AgdaFunction{∘}}\AgdaSpace{}%
\AgdaBound{ρ}\AgdaSymbol{))}%
\>[85]\AgdaFunction{⟩}\<%
\\
%
\>[9]\AgdaOperator{\AgdaFunction{∣}}\AgdaSpace{}%
\AgdaFunction{φ}\AgdaSpace{}%
\AgdaOperator{\AgdaFunction{∣}}\AgdaSpace{}%
\AgdaOperator{\AgdaField{⟨\$⟩}}\AgdaSpace{}%
\AgdaSymbol{(}\AgdaOperator{\AgdaFunction{⟦}}\AgdaSpace{}%
\AgdaBound{q}\AgdaSpace{}%
\AgdaOperator{\AgdaFunction{⟧₁}}\AgdaSpace{}%
\AgdaOperator{\AgdaField{⟨\$⟩}}\AgdaSpace{}%
\AgdaSymbol{(}\AgdaFunction{φ⁻¹}\AgdaSpace{}%
\AgdaOperator{\AgdaFunction{∘}}\AgdaSpace{}%
\AgdaBound{ρ}\AgdaSymbol{))}%
\>[52]\AgdaFunction{≈⟨}\AgdaSpace{}%
\AgdaFunction{comm-hom-term}\AgdaSpace{}%
\AgdaFunction{φ}\AgdaSpace{}%
\AgdaBound{q}\AgdaSpace{}%
\AgdaSymbol{(}\AgdaFunction{φ⁻¹}\AgdaSpace{}%
\AgdaOperator{\AgdaFunction{∘}}\AgdaSpace{}%
\AgdaBound{ρ}\AgdaSymbol{)}%
\>[85]\AgdaFunction{⟩}\<%
\\
%
\>[9]\AgdaOperator{\AgdaFunction{⟦}}\AgdaSpace{}%
\AgdaBound{q}\AgdaSpace{}%
\AgdaOperator{\AgdaFunction{⟧}}\AgdaSpace{}%
\AgdaOperator{\AgdaField{⟨\$⟩}}\AgdaSpace{}%
\AgdaSymbol{(λ}\AgdaSpace{}%
\AgdaBound{x}\AgdaSpace{}%
\AgdaSymbol{→}\AgdaSpace{}%
\AgdaSymbol{(}\AgdaOperator{\AgdaFunction{∣}}\AgdaSpace{}%
\AgdaFunction{φ}\AgdaSpace{}%
\AgdaOperator{\AgdaFunction{∣}}\AgdaSpace{}%
\AgdaOperator{\AgdaField{⟨\$⟩}}\AgdaSpace{}%
\AgdaSymbol{(}\AgdaFunction{φ⁻¹}\AgdaSpace{}%
\AgdaOperator{\AgdaFunction{∘}}\AgdaSpace{}%
\AgdaBound{ρ}\AgdaSymbol{)}\AgdaSpace{}%
\AgdaBound{x}\AgdaSymbol{))}%
\>[52]\AgdaFunction{≈⟨}\AgdaSpace{}%
\AgdaField{cong}\AgdaSpace{}%
\AgdaOperator{\AgdaFunction{⟦}}\AgdaSpace{}%
\AgdaBound{q}\AgdaSpace{}%
\AgdaOperator{\AgdaFunction{⟧}}\AgdaSpace{}%
\AgdaFunction{ζ}%
\>[85]\AgdaFunction{⟩}\<%
\\
%
\>[9]\AgdaOperator{\AgdaFunction{⟦}}\AgdaSpace{}%
\AgdaBound{q}\AgdaSpace{}%
\AgdaOperator{\AgdaFunction{⟧}}\AgdaSpace{}%
\AgdaOperator{\AgdaField{⟨\$⟩}}\AgdaSpace{}%
\AgdaBound{ρ}%
\>[52]\AgdaOperator{\AgdaFunction{∎}}\<%
\end{code}

\hypertarget{s-preserves-identities}{%
\subparagraph{S preserves identities}\label{s-preserves-identities}}

\begin{code}%
\>[0]\<%
\\
%
\>[1]\AgdaFunction{S-id1}\AgdaSpace{}%
\AgdaSymbol{:}\AgdaSpace{}%
\AgdaBound{𝒦}\AgdaSpace{}%
\AgdaOperator{\AgdaFunction{⊫}}\AgdaSpace{}%
\AgdaSymbol{(}\AgdaBound{p}\AgdaSpace{}%
\AgdaOperator{\AgdaInductiveConstructor{≈̇}}\AgdaSpace{}%
\AgdaBound{q}\AgdaSymbol{)}\AgdaSpace{}%
\AgdaSymbol{→}\AgdaSpace{}%
\AgdaSymbol{(}\AgdaFunction{S}\AgdaSpace{}%
\AgdaSymbol{\{}\AgdaArgument{β}\AgdaSpace{}%
\AgdaSymbol{=}\AgdaSpace{}%
\AgdaBound{α}\AgdaSymbol{\}\{}\AgdaBound{ρᵃ}\AgdaSymbol{\}}\AgdaSpace{}%
\AgdaBound{ℓ}\AgdaSpace{}%
\AgdaBound{𝒦}\AgdaSymbol{)}\AgdaSpace{}%
\AgdaOperator{\AgdaFunction{⊫}}\AgdaSpace{}%
\AgdaSymbol{(}\AgdaBound{p}\AgdaSpace{}%
\AgdaOperator{\AgdaInductiveConstructor{≈̇}}\AgdaSpace{}%
\AgdaBound{q}\AgdaSymbol{)}\<%
\\
%
\>[1]\AgdaFunction{S-id1}\AgdaSpace{}%
\AgdaBound{σ}\AgdaSpace{}%
\AgdaBound{𝑩}\AgdaSpace{}%
\AgdaSymbol{(}\AgdaBound{𝑨}\AgdaSpace{}%
\AgdaOperator{\AgdaInductiveConstructor{,}}\AgdaSpace{}%
\AgdaBound{kA}\AgdaSpace{}%
\AgdaOperator{\AgdaInductiveConstructor{,}}\AgdaSpace{}%
\AgdaBound{B≤A}\AgdaSymbol{)}\AgdaSpace{}%
\AgdaSymbol{=}\AgdaSpace{}%
\AgdaFunction{⊧-S-invar}\AgdaSymbol{\{}\AgdaArgument{p}\AgdaSpace{}%
\AgdaSymbol{=}\AgdaSpace{}%
\AgdaBound{p}\AgdaSymbol{\}\{}\AgdaBound{q}\AgdaSymbol{\}}\AgdaSpace{}%
\AgdaSymbol{(}\AgdaBound{σ}\AgdaSpace{}%
\AgdaBound{𝑨}\AgdaSpace{}%
\AgdaBound{kA}\AgdaSymbol{)}\AgdaSpace{}%
\AgdaBound{B≤A}\<%
\\
\>[0]\<%
\end{code}

The obvious converse is barely worth the bits needed to formalize it,
but we will use it below, so let's prove it now.

\begin{code}%
\>[0]\<%
\\
\>[0][@{}l@{\AgdaIndent{1}}]%
\>[1]\AgdaFunction{S-id2}\AgdaSpace{}%
\AgdaSymbol{:}\AgdaSpace{}%
\AgdaFunction{S}\AgdaSpace{}%
\AgdaBound{ℓ}\AgdaSpace{}%
\AgdaBound{𝒦}\AgdaSpace{}%
\AgdaOperator{\AgdaFunction{⊫}}\AgdaSpace{}%
\AgdaSymbol{(}\AgdaBound{p}\AgdaSpace{}%
\AgdaOperator{\AgdaInductiveConstructor{≈̇}}\AgdaSpace{}%
\AgdaBound{q}\AgdaSymbol{)}\AgdaSpace{}%
\AgdaSymbol{→}\AgdaSpace{}%
\AgdaBound{𝒦}\AgdaSpace{}%
\AgdaOperator{\AgdaFunction{⊫}}\AgdaSpace{}%
\AgdaSymbol{(}\AgdaBound{p}\AgdaSpace{}%
\AgdaOperator{\AgdaInductiveConstructor{≈̇}}\AgdaSpace{}%
\AgdaBound{q}\AgdaSymbol{)}\<%
\\
%
\>[1]\AgdaFunction{S-id2}\AgdaSpace{}%
\AgdaBound{Spq}\AgdaSpace{}%
\AgdaBound{𝑨}\AgdaSpace{}%
\AgdaBound{kA}\AgdaSpace{}%
\AgdaSymbol{=}\AgdaSpace{}%
\AgdaBound{Spq}\AgdaSpace{}%
\AgdaBound{𝑨}\AgdaSpace{}%
\AgdaSymbol{(}\AgdaBound{𝑨}\AgdaSpace{}%
\AgdaOperator{\AgdaInductiveConstructor{,}}\AgdaSpace{}%
\AgdaSymbol{(}\AgdaBound{kA}\AgdaSpace{}%
\AgdaOperator{\AgdaInductiveConstructor{,}}\AgdaSpace{}%
\AgdaFunction{≤-reflexive}\AgdaSymbol{))}\<%
\end{code}

\hypertarget{p-preserves-identities}{%
\subparagraph{P preserves identities}\label{p-preserves-identities}}

\begin{code}%
\>[0]\<%
\\
%
\>[1]\AgdaFunction{P-id1}\AgdaSpace{}%
\AgdaSymbol{:}\AgdaSpace{}%
\AgdaSymbol{∀\{}\AgdaBound{ι}\AgdaSymbol{\}}\AgdaSpace{}%
\AgdaSymbol{→}\AgdaSpace{}%
\AgdaBound{𝒦}\AgdaSpace{}%
\AgdaOperator{\AgdaFunction{⊫}}\AgdaSpace{}%
\AgdaSymbol{(}\AgdaBound{p}\AgdaSpace{}%
\AgdaOperator{\AgdaInductiveConstructor{≈̇}}\AgdaSpace{}%
\AgdaBound{q}\AgdaSymbol{)}\AgdaSpace{}%
\AgdaSymbol{→}\AgdaSpace{}%
\AgdaFunction{P}\AgdaSpace{}%
\AgdaSymbol{\{}\AgdaArgument{β}\AgdaSpace{}%
\AgdaSymbol{=}\AgdaSpace{}%
\AgdaBound{α}\AgdaSymbol{\}\{}\AgdaBound{ρᵃ}\AgdaSymbol{\}}\AgdaBound{ℓ}\AgdaSpace{}%
\AgdaBound{ι}\AgdaSpace{}%
\AgdaBound{𝒦}\AgdaSpace{}%
\AgdaOperator{\AgdaFunction{⊫}}\AgdaSpace{}%
\AgdaSymbol{(}\AgdaBound{p}\AgdaSpace{}%
\AgdaOperator{\AgdaInductiveConstructor{≈̇}}\AgdaSpace{}%
\AgdaBound{q}\AgdaSymbol{)}\<%
\\
%
\>[1]\AgdaFunction{P-id1}\AgdaSpace{}%
\AgdaBound{σ}\AgdaSpace{}%
\AgdaBound{𝑨}\AgdaSpace{}%
\AgdaSymbol{(}\AgdaBound{I}\AgdaSpace{}%
\AgdaOperator{\AgdaInductiveConstructor{,}}\AgdaSpace{}%
\AgdaBound{𝒜}\AgdaSpace{}%
\AgdaOperator{\AgdaInductiveConstructor{,}}\AgdaSpace{}%
\AgdaBound{kA}\AgdaSpace{}%
\AgdaOperator{\AgdaInductiveConstructor{,}}\AgdaSpace{}%
\AgdaBound{A≅⨅A}\AgdaSymbol{)}\AgdaSpace{}%
\AgdaSymbol{=}\AgdaSpace{}%
\AgdaFunction{⊧-I-invar}\AgdaSpace{}%
\AgdaBound{𝑨}\AgdaSpace{}%
\AgdaBound{p}\AgdaSpace{}%
\AgdaBound{q}\AgdaSpace{}%
\AgdaFunction{IH}\AgdaSpace{}%
\AgdaSymbol{(}\AgdaFunction{≅-sym}\AgdaSpace{}%
\AgdaBound{A≅⨅A}\AgdaSymbol{)}\<%
\\
\>[1][@{}l@{\AgdaIndent{0}}]%
\>[2]\AgdaKeyword{where}\<%
\\
%
\>[2]\AgdaFunction{ih}\AgdaSpace{}%
\AgdaSymbol{:}\AgdaSpace{}%
\AgdaSymbol{∀}\AgdaSpace{}%
\AgdaBound{i}\AgdaSpace{}%
\AgdaSymbol{→}\AgdaSpace{}%
\AgdaBound{𝒜}\AgdaSpace{}%
\AgdaBound{i}\AgdaSpace{}%
\AgdaOperator{\AgdaFunction{⊧}}\AgdaSpace{}%
\AgdaSymbol{(}\AgdaBound{p}\AgdaSpace{}%
\AgdaOperator{\AgdaInductiveConstructor{≈̇}}\AgdaSpace{}%
\AgdaBound{q}\AgdaSymbol{)}\<%
\\
%
\>[2]\AgdaFunction{ih}\AgdaSpace{}%
\AgdaBound{i}\AgdaSpace{}%
\AgdaSymbol{=}\AgdaSpace{}%
\AgdaBound{σ}\AgdaSpace{}%
\AgdaSymbol{(}\AgdaBound{𝒜}\AgdaSpace{}%
\AgdaBound{i}\AgdaSymbol{)}\AgdaSpace{}%
\AgdaSymbol{(}\AgdaBound{kA}\AgdaSpace{}%
\AgdaBound{i}\AgdaSymbol{)}\<%
\\
%
\>[2]\AgdaFunction{IH}\AgdaSpace{}%
\AgdaSymbol{:}\AgdaSpace{}%
\AgdaFunction{⨅}\AgdaSpace{}%
\AgdaBound{𝒜}\AgdaSpace{}%
\AgdaOperator{\AgdaFunction{⊧}}\AgdaSpace{}%
\AgdaSymbol{(}\AgdaBound{p}\AgdaSpace{}%
\AgdaOperator{\AgdaInductiveConstructor{≈̇}}\AgdaSpace{}%
\AgdaBound{q}\AgdaSymbol{)}\<%
\\
%
\>[2]\AgdaFunction{IH}\AgdaSpace{}%
\AgdaSymbol{=}\AgdaSpace{}%
\AgdaFunction{⊧-P-invar}\AgdaSpace{}%
\AgdaBound{𝒜}\AgdaSpace{}%
\AgdaSymbol{\{}\AgdaBound{p}\AgdaSymbol{\}\{}\AgdaBound{q}\AgdaSymbol{\}}\AgdaSpace{}%
\AgdaFunction{ih}\<%
\end{code}

\hypertarget{v-preserves-identities}{%
\subparagraph{V preserves identities}\label{v-preserves-identities}}

Finally, we prove the analogous preservation lemmas for the closure
operator \texttt{V}.

\begin{code}%
\>[0]\<%
\\
\>[0]\AgdaKeyword{module}\AgdaSpace{}%
\AgdaModule{\AgdaUnderscore{}}\AgdaSpace{}%
\AgdaSymbol{\{}\AgdaBound{X}\AgdaSpace{}%
\AgdaSymbol{:}\AgdaSpace{}%
\AgdaPrimitive{Type}\AgdaSpace{}%
\AgdaGeneralizable{χ}\AgdaSymbol{\}\{}\AgdaBound{ι}\AgdaSpace{}%
\AgdaSymbol{:}\AgdaSpace{}%
\AgdaPostulate{Level}\AgdaSymbol{\}\{}\AgdaBound{𝒦}\AgdaSpace{}%
\AgdaSymbol{:}\AgdaSpace{}%
\AgdaFunction{Pred}\AgdaSymbol{(}\AgdaRecord{Algebra}\AgdaSpace{}%
\AgdaGeneralizable{α}\AgdaSpace{}%
\AgdaGeneralizable{ρᵃ}\AgdaSymbol{)(}\AgdaGeneralizable{α}\AgdaSpace{}%
\AgdaOperator{\AgdaPrimitive{⊔}}\AgdaSpace{}%
\AgdaGeneralizable{ρᵃ}\AgdaSpace{}%
\AgdaOperator{\AgdaPrimitive{⊔}}\AgdaSpace{}%
\AgdaFunction{ov}\AgdaSpace{}%
\AgdaGeneralizable{ℓ}\AgdaSymbol{)\}\{}\AgdaBound{p}\AgdaSpace{}%
\AgdaBound{q}\AgdaSpace{}%
\AgdaSymbol{:}\AgdaSpace{}%
\AgdaDatatype{Term}\AgdaSpace{}%
\AgdaBound{X}\AgdaSymbol{\}}\AgdaSpace{}%
\AgdaKeyword{where}\<%
\\
\>[0][@{}l@{\AgdaIndent{0}}]%
\>[1]\AgdaKeyword{private}\<%
\\
\>[1][@{}l@{\AgdaIndent{0}}]%
\>[2]\AgdaFunction{aℓι}\AgdaSpace{}%
\AgdaSymbol{=}\AgdaSpace{}%
\AgdaBound{α}\AgdaSpace{}%
\AgdaOperator{\AgdaPrimitive{⊔}}\AgdaSpace{}%
\AgdaBound{ρᵃ}\AgdaSpace{}%
\AgdaOperator{\AgdaPrimitive{⊔}}\AgdaSpace{}%
\AgdaBound{ℓ}\AgdaSpace{}%
\AgdaOperator{\AgdaPrimitive{⊔}}\AgdaSpace{}%
\AgdaBound{ι}\<%
\\
%
\\[\AgdaEmptyExtraSkip]%
%
\>[1]\AgdaFunction{V-id1}\AgdaSpace{}%
\AgdaSymbol{:}\AgdaSpace{}%
\AgdaBound{𝒦}\AgdaSpace{}%
\AgdaOperator{\AgdaFunction{⊫}}\AgdaSpace{}%
\AgdaSymbol{(}\AgdaBound{p}\AgdaSpace{}%
\AgdaOperator{\AgdaInductiveConstructor{≈̇}}\AgdaSpace{}%
\AgdaBound{q}\AgdaSymbol{)}\AgdaSpace{}%
\AgdaSymbol{→}\AgdaSpace{}%
\AgdaFunction{V}\AgdaSpace{}%
\AgdaBound{ℓ}\AgdaSpace{}%
\AgdaBound{ι}\AgdaSpace{}%
\AgdaBound{𝒦}\AgdaSpace{}%
\AgdaOperator{\AgdaFunction{⊫}}\AgdaSpace{}%
\AgdaSymbol{(}\AgdaBound{p}\AgdaSpace{}%
\AgdaOperator{\AgdaInductiveConstructor{≈̇}}\AgdaSpace{}%
\AgdaBound{q}\AgdaSymbol{)}\<%
\\
%
\>[1]\AgdaFunction{V-id1}\AgdaSpace{}%
\AgdaBound{σ}\AgdaSpace{}%
\AgdaBound{𝑩}\AgdaSpace{}%
\AgdaSymbol{(}\AgdaBound{𝑨}\AgdaSpace{}%
\AgdaOperator{\AgdaInductiveConstructor{,}}\AgdaSpace{}%
\AgdaSymbol{(}\AgdaBound{⨅A}\AgdaSpace{}%
\AgdaOperator{\AgdaInductiveConstructor{,}}\AgdaSpace{}%
\AgdaBound{p⨅A}\AgdaSpace{}%
\AgdaOperator{\AgdaInductiveConstructor{,}}\AgdaSpace{}%
\AgdaBound{A≤⨅A}\AgdaSymbol{)}\AgdaSpace{}%
\AgdaOperator{\AgdaInductiveConstructor{,}}\AgdaSpace{}%
\AgdaBound{BimgA}\AgdaSymbol{)}\AgdaSpace{}%
\AgdaSymbol{=}\<%
\\
\>[1][@{}l@{\AgdaIndent{0}}]%
\>[2]\AgdaFunction{H-id1}\AgdaSymbol{\{}\AgdaArgument{ℓ}\AgdaSpace{}%
\AgdaSymbol{=}\AgdaSpace{}%
\AgdaFunction{aℓι}\AgdaSymbol{\}\{}\AgdaArgument{𝒦}\AgdaSpace{}%
\AgdaSymbol{=}\AgdaSpace{}%
\AgdaFunction{S}\AgdaSpace{}%
\AgdaFunction{aℓι}\AgdaSpace{}%
\AgdaSymbol{(}\AgdaFunction{P}\AgdaSpace{}%
\AgdaSymbol{\{}\AgdaArgument{β}\AgdaSpace{}%
\AgdaSymbol{=}\AgdaSpace{}%
\AgdaBound{α}\AgdaSymbol{\}\{}\AgdaBound{ρᵃ}\AgdaSymbol{\}}\AgdaBound{ℓ}\AgdaSpace{}%
\AgdaBound{ι}\AgdaSpace{}%
\AgdaBound{𝒦}\AgdaSymbol{)\}\{}\AgdaArgument{p}\AgdaSpace{}%
\AgdaSymbol{=}\AgdaSpace{}%
\AgdaBound{p}\AgdaSymbol{\}\{}\AgdaBound{q}\AgdaSymbol{\}}\AgdaSpace{}%
\AgdaFunction{spK⊧pq}\AgdaSpace{}%
\AgdaBound{𝑩}\AgdaSpace{}%
\AgdaSymbol{(}\AgdaBound{𝑨}\AgdaSpace{}%
\AgdaOperator{\AgdaInductiveConstructor{,}}\AgdaSpace{}%
\AgdaSymbol{(}\AgdaFunction{spA}\AgdaSpace{}%
\AgdaOperator{\AgdaInductiveConstructor{,}}\AgdaSpace{}%
\AgdaBound{BimgA}\AgdaSymbol{))}\<%
\\
\>[2][@{}l@{\AgdaIndent{0}}]%
\>[3]\AgdaKeyword{where}\<%
\\
%
\>[3]\AgdaFunction{spA}\AgdaSpace{}%
\AgdaSymbol{:}\AgdaSpace{}%
\AgdaBound{𝑨}\AgdaSpace{}%
\AgdaOperator{\AgdaFunction{∈}}\AgdaSpace{}%
\AgdaFunction{S}\AgdaSpace{}%
\AgdaFunction{aℓι}\AgdaSpace{}%
\AgdaSymbol{(}\AgdaFunction{P}\AgdaSpace{}%
\AgdaSymbol{\{}\AgdaArgument{β}\AgdaSpace{}%
\AgdaSymbol{=}\AgdaSpace{}%
\AgdaBound{α}\AgdaSymbol{\}\{}\AgdaBound{ρᵃ}\AgdaSymbol{\}}\AgdaBound{ℓ}\AgdaSpace{}%
\AgdaBound{ι}\AgdaSpace{}%
\AgdaBound{𝒦}\AgdaSymbol{)}\<%
\\
%
\>[3]\AgdaFunction{spA}\AgdaSpace{}%
\AgdaSymbol{=}\AgdaSpace{}%
\AgdaBound{⨅A}\AgdaSpace{}%
\AgdaOperator{\AgdaInductiveConstructor{,}}\AgdaSpace{}%
\AgdaSymbol{(}\AgdaBound{p⨅A}\AgdaSpace{}%
\AgdaOperator{\AgdaInductiveConstructor{,}}\AgdaSpace{}%
\AgdaBound{A≤⨅A}\AgdaSymbol{)}\<%
\\
%
\>[3]\AgdaFunction{spK⊧pq}\AgdaSpace{}%
\AgdaSymbol{:}\AgdaSpace{}%
\AgdaFunction{S}\AgdaSpace{}%
\AgdaFunction{aℓι}\AgdaSpace{}%
\AgdaSymbol{(}\AgdaFunction{P}\AgdaSpace{}%
\AgdaBound{ℓ}\AgdaSpace{}%
\AgdaBound{ι}\AgdaSpace{}%
\AgdaBound{𝒦}\AgdaSymbol{)}\AgdaSpace{}%
\AgdaOperator{\AgdaFunction{⊫}}\AgdaSpace{}%
\AgdaSymbol{(}\AgdaBound{p}\AgdaSpace{}%
\AgdaOperator{\AgdaInductiveConstructor{≈̇}}\AgdaSpace{}%
\AgdaBound{q}\AgdaSymbol{)}\<%
\\
%
\>[3]\AgdaFunction{spK⊧pq}\AgdaSpace{}%
\AgdaSymbol{=}\AgdaSpace{}%
\AgdaFunction{S-id1}\AgdaSymbol{\{}\AgdaArgument{ℓ}\AgdaSpace{}%
\AgdaSymbol{=}\AgdaSpace{}%
\AgdaFunction{aℓι}\AgdaSymbol{\}\{}\AgdaArgument{p}\AgdaSpace{}%
\AgdaSymbol{=}\AgdaSpace{}%
\AgdaBound{p}\AgdaSymbol{\}\{}\AgdaBound{q}\AgdaSymbol{\}}\AgdaSpace{}%
\AgdaSymbol{(}\AgdaFunction{P-id1}\AgdaSymbol{\{}\AgdaArgument{ℓ}\AgdaSpace{}%
\AgdaSymbol{=}\AgdaSpace{}%
\AgdaBound{ℓ}\AgdaSymbol{\}}\AgdaSpace{}%
\AgdaSymbol{\{}\AgdaArgument{𝒦}\AgdaSpace{}%
\AgdaSymbol{=}\AgdaSpace{}%
\AgdaBound{𝒦}\AgdaSymbol{\}\{}\AgdaArgument{p}\AgdaSpace{}%
\AgdaSymbol{=}\AgdaSpace{}%
\AgdaBound{p}\AgdaSymbol{\}\{}\AgdaBound{q}\AgdaSymbol{\}}\AgdaSpace{}%
\AgdaBound{σ}\AgdaSymbol{)}\<%
\end{code}

\hypertarget{th-ux1d4a6-th-v-ux1d4a6}{%
\paragraph{Th 𝒦 ⊆ Th (V 𝒦)}\label{th-ux1d4a6-th-v-ux1d4a6}}

From \texttt{V-id1} it follows that if 𝒦 is a class of algebras, then
the set of identities modeled by the algebras in \texttt{𝒦} is contained
in the set of identities modeled by the algebras in \texttt{V\ 𝒦}. In
other terms, \texttt{Th\ 𝒦\ ⊆\ Th\ (V\ 𝒦)}. We formalize this
observation as follows.

\begin{code}%
\>[0]\<%
\\
%
\>[1]\AgdaFunction{classIds-⊆-VIds}\AgdaSpace{}%
\AgdaSymbol{:}\AgdaSpace{}%
\AgdaBound{𝒦}\AgdaSpace{}%
\AgdaOperator{\AgdaFunction{⊫}}\AgdaSpace{}%
\AgdaSymbol{(}\AgdaBound{p}\AgdaSpace{}%
\AgdaOperator{\AgdaInductiveConstructor{≈̇}}\AgdaSpace{}%
\AgdaBound{q}\AgdaSymbol{)}%
\>[33]\AgdaSymbol{→}\AgdaSpace{}%
\AgdaSymbol{(}\AgdaBound{p}\AgdaSpace{}%
\AgdaOperator{\AgdaInductiveConstructor{,}}\AgdaSpace{}%
\AgdaBound{q}\AgdaSymbol{)}\AgdaSpace{}%
\AgdaOperator{\AgdaFunction{∈}}\AgdaSpace{}%
\AgdaFunction{Th}\AgdaSpace{}%
\AgdaSymbol{(}\AgdaFunction{V}\AgdaSpace{}%
\AgdaBound{ℓ}\AgdaSpace{}%
\AgdaBound{ι}\AgdaSpace{}%
\AgdaBound{𝒦}\AgdaSymbol{)}\<%
\\
%
\>[1]\AgdaFunction{classIds-⊆-VIds}\AgdaSpace{}%
\AgdaBound{pKq}\AgdaSpace{}%
\AgdaBound{𝑨}\AgdaSpace{}%
\AgdaSymbol{=}\AgdaSpace{}%
\AgdaFunction{V-id1}\AgdaSpace{}%
\AgdaBound{pKq}\AgdaSpace{}%
\AgdaBound{𝑨}\<%
\end{code}

\hypertarget{free-algebras}{%
\subsection{Free Algebras}\label{free-algebras}}

\hypertarget{the-absolutely-free-algebra-ux1d47b-x}{%
\subsubsection{The absolutely free algebra 𝑻
X}\label{the-absolutely-free-algebra-ux1d47b-x}}

The term algebra \texttt{𝑻\ X} is \emph{absolutely free} (or
\emph{universal}, or \emph{initial}) for algebras in the signature
\texttt{𝑆}. That is, for every 𝑆-algebra \texttt{𝑨}, the following hold.

\begin{enumerate}
\def\labelenumi{\arabic{enumi}.}
\tightlist
\item
  Every function from \texttt{𝑋} to \texttt{∣\ 𝑨\ ∣} lifts to a
  homomorphism from \texttt{𝑻\ X} to \texttt{𝑨}.
\item
  The homomorphism that exists by item 1 is unique.
\end{enumerate}

We now prove this in
\href{https://wiki.portal.chalmers.se/agda/pmwiki.php}{Agda}, starting
with the fact that every map from \texttt{X} to \texttt{∣\ 𝑨\ ∣} lifts
to a map from \texttt{∣\ 𝑻\ X\ ∣} to \texttt{∣\ 𝑨\ ∣} in a natural way,
by induction on the structure of the given term.

\begin{code}%
\>[0]\<%
\\
\>[0]\AgdaKeyword{module}\AgdaSpace{}%
\AgdaModule{\AgdaUnderscore{}}\AgdaSpace{}%
\AgdaSymbol{\{}\AgdaBound{X}\AgdaSpace{}%
\AgdaSymbol{:}\AgdaSpace{}%
\AgdaPrimitive{Type}\AgdaSpace{}%
\AgdaGeneralizable{χ}\AgdaSymbol{\}\{}\AgdaBound{𝑨}\AgdaSpace{}%
\AgdaSymbol{:}\AgdaSpace{}%
\AgdaRecord{Algebra}\AgdaSpace{}%
\AgdaGeneralizable{α}\AgdaSpace{}%
\AgdaGeneralizable{ρᵃ}\AgdaSymbol{\}(}\AgdaBound{h}\AgdaSpace{}%
\AgdaSymbol{:}\AgdaSpace{}%
\AgdaBound{X}\AgdaSpace{}%
\AgdaSymbol{→}\AgdaSpace{}%
\AgdaOperator{\AgdaFunction{𝕌[}}\AgdaSpace{}%
\AgdaBound{𝑨}\AgdaSpace{}%
\AgdaOperator{\AgdaFunction{]}}\AgdaSymbol{)}\AgdaSpace{}%
\AgdaKeyword{where}\<%
\\
\>[0][@{}l@{\AgdaIndent{0}}]%
\>[1]\AgdaKeyword{open}\AgdaSpace{}%
\AgdaModule{Setoid}\AgdaSpace{}%
\AgdaOperator{\AgdaFunction{𝔻[}}\AgdaSpace{}%
\AgdaBound{𝑨}\AgdaSpace{}%
\AgdaOperator{\AgdaFunction{]}}\AgdaSpace{}%
\AgdaKeyword{using}\AgdaSpace{}%
\AgdaSymbol{(}\AgdaSpace{}%
\AgdaOperator{\AgdaField{\AgdaUnderscore{}≈\AgdaUnderscore{}}}\AgdaSpace{}%
\AgdaSymbol{;}\AgdaSpace{}%
\AgdaFunction{reflexive}\AgdaSpace{}%
\AgdaSymbol{;}\AgdaSpace{}%
\AgdaFunction{refl}\AgdaSpace{}%
\AgdaSymbol{;}\AgdaSpace{}%
\AgdaFunction{trans}\AgdaSpace{}%
\AgdaSymbol{)}\<%
\\
%
\\[\AgdaEmptyExtraSkip]%
%
\>[1]\AgdaFunction{free-lift}\AgdaSpace{}%
\AgdaSymbol{:}\AgdaSpace{}%
\AgdaOperator{\AgdaFunction{𝕌[}}\AgdaSpace{}%
\AgdaFunction{𝑻}\AgdaSpace{}%
\AgdaBound{X}\AgdaSpace{}%
\AgdaOperator{\AgdaFunction{]}}\AgdaSpace{}%
\AgdaSymbol{→}\AgdaSpace{}%
\AgdaOperator{\AgdaFunction{𝕌[}}\AgdaSpace{}%
\AgdaBound{𝑨}\AgdaSpace{}%
\AgdaOperator{\AgdaFunction{]}}\<%
\\
%
\>[1]\AgdaFunction{free-lift}\AgdaSpace{}%
\AgdaSymbol{(}\AgdaInductiveConstructor{ℊ}\AgdaSpace{}%
\AgdaBound{x}\AgdaSymbol{)}\AgdaSpace{}%
\AgdaSymbol{=}\AgdaSpace{}%
\AgdaBound{h}\AgdaSpace{}%
\AgdaBound{x}\<%
\\
%
\>[1]\AgdaFunction{free-lift}\AgdaSpace{}%
\AgdaSymbol{(}\AgdaInductiveConstructor{node}\AgdaSpace{}%
\AgdaBound{f}\AgdaSpace{}%
\AgdaBound{t}\AgdaSymbol{)}\AgdaSpace{}%
\AgdaSymbol{=}\AgdaSpace{}%
\AgdaSymbol{(}\AgdaBound{f}\AgdaSpace{}%
\AgdaOperator{\AgdaFunction{̂}}\AgdaSpace{}%
\AgdaBound{𝑨}\AgdaSymbol{)}\AgdaSpace{}%
\AgdaSymbol{(λ}\AgdaSpace{}%
\AgdaBound{i}\AgdaSpace{}%
\AgdaSymbol{→}\AgdaSpace{}%
\AgdaFunction{free-lift}\AgdaSpace{}%
\AgdaSymbol{(}\AgdaBound{t}\AgdaSpace{}%
\AgdaBound{i}\AgdaSymbol{))}\<%
\\
%
\\[\AgdaEmptyExtraSkip]%
%
\>[1]\AgdaFunction{free-lift-func}\AgdaSpace{}%
\AgdaSymbol{:}\AgdaSpace{}%
\AgdaOperator{\AgdaFunction{𝔻[}}\AgdaSpace{}%
\AgdaFunction{𝑻}\AgdaSpace{}%
\AgdaBound{X}\AgdaSpace{}%
\AgdaOperator{\AgdaFunction{]}}\AgdaSpace{}%
\AgdaOperator{\AgdaRecord{⟶}}\AgdaSpace{}%
\AgdaOperator{\AgdaFunction{𝔻[}}\AgdaSpace{}%
\AgdaBound{𝑨}\AgdaSpace{}%
\AgdaOperator{\AgdaFunction{]}}\<%
\\
%
\>[1]\AgdaFunction{free-lift-func}\AgdaSpace{}%
\AgdaOperator{\AgdaField{⟨\$⟩}}\AgdaSpace{}%
\AgdaBound{x}\AgdaSpace{}%
\AgdaSymbol{=}\AgdaSpace{}%
\AgdaFunction{free-lift}\AgdaSpace{}%
\AgdaBound{x}\<%
\\
%
\>[1]\AgdaField{cong}\AgdaSpace{}%
\AgdaFunction{free-lift-func}\AgdaSpace{}%
\AgdaSymbol{=}\AgdaSpace{}%
\AgdaFunction{flcong}\<%
\\
\>[1][@{}l@{\AgdaIndent{0}}]%
\>[2]\AgdaKeyword{where}\<%
\\
%
\>[2]\AgdaFunction{flcong}\AgdaSpace{}%
\AgdaSymbol{:}\AgdaSpace{}%
\AgdaSymbol{∀}\AgdaSpace{}%
\AgdaSymbol{\{}\AgdaBound{s}\AgdaSpace{}%
\AgdaBound{t}\AgdaSymbol{\}}\AgdaSpace{}%
\AgdaSymbol{→}\AgdaSpace{}%
\AgdaBound{s}\AgdaSpace{}%
\AgdaOperator{\AgdaDatatype{≐}}\AgdaSpace{}%
\AgdaBound{t}\AgdaSpace{}%
\AgdaSymbol{→}%
\>[30]\AgdaFunction{free-lift}\AgdaSpace{}%
\AgdaBound{s}\AgdaSpace{}%
\AgdaOperator{\AgdaFunction{≈}}\AgdaSpace{}%
\AgdaFunction{free-lift}\AgdaSpace{}%
\AgdaBound{t}\<%
\\
%
\>[2]\AgdaFunction{flcong}\AgdaSpace{}%
\AgdaSymbol{(}\AgdaInductiveConstructor{\AgdaUnderscore{}≐\AgdaUnderscore{}.rfl}\AgdaSpace{}%
\AgdaBound{x}\AgdaSymbol{)}\AgdaSpace{}%
\AgdaSymbol{=}\AgdaSpace{}%
\AgdaFunction{reflexive}\AgdaSpace{}%
\AgdaSymbol{(}\AgdaFunction{≡.cong}\AgdaSpace{}%
\AgdaBound{h}\AgdaSpace{}%
\AgdaBound{x}\AgdaSymbol{)}\<%
\\
%
\>[2]\AgdaFunction{flcong}\AgdaSpace{}%
\AgdaSymbol{(}\AgdaInductiveConstructor{\AgdaUnderscore{}≐\AgdaUnderscore{}.gnl}\AgdaSpace{}%
\AgdaBound{x}\AgdaSymbol{)}\AgdaSpace{}%
\AgdaSymbol{=}\AgdaSpace{}%
\AgdaField{cong}\AgdaSpace{}%
\AgdaSymbol{(}\AgdaField{Interp}\AgdaSpace{}%
\AgdaBound{𝑨}\AgdaSymbol{)}\AgdaSpace{}%
\AgdaSymbol{(}\AgdaInductiveConstructor{≡.refl}\AgdaSpace{}%
\AgdaOperator{\AgdaInductiveConstructor{,}}\AgdaSpace{}%
\AgdaSymbol{(λ}\AgdaSpace{}%
\AgdaBound{i}\AgdaSpace{}%
\AgdaSymbol{→}\AgdaSpace{}%
\AgdaFunction{flcong}\AgdaSpace{}%
\AgdaSymbol{(}\AgdaBound{x}\AgdaSpace{}%
\AgdaBound{i}\AgdaSymbol{)))}\<%
\\
\>[0]\<%
\end{code}

Naturally, at the base step of the induction, when the term has the form
\texttt{generator} x, the free lift of \texttt{h} agrees with
\texttt{h}. For the inductive step, when the given term has the form
\texttt{node\ f\ t}, the free lift is defined as follows: Assuming (the
induction hypothesis) that we know the image of each subterm
\texttt{t\ i} under the free lift of \texttt{h}, define the free lift at
the full term by applying \texttt{f\ ̂\ 𝑨} to the images of the subterms.

The free lift so defined is a homomorphism by construction. Indeed, here
is the trivial proof.

\begin{code}%
\>[0]\<%
\\
\>[0][@{}l@{\AgdaIndent{1}}]%
\>[1]\AgdaFunction{lift-hom}\AgdaSpace{}%
\AgdaSymbol{:}\AgdaSpace{}%
\AgdaFunction{hom}\AgdaSpace{}%
\AgdaSymbol{(}\AgdaFunction{𝑻}\AgdaSpace{}%
\AgdaBound{X}\AgdaSymbol{)}\AgdaSpace{}%
\AgdaBound{𝑨}\<%
\\
%
\>[1]\AgdaFunction{lift-hom}\AgdaSpace{}%
\AgdaSymbol{=}\AgdaSpace{}%
\AgdaFunction{free-lift-func}\AgdaSpace{}%
\AgdaOperator{\AgdaInductiveConstructor{,}}\AgdaSpace{}%
\AgdaFunction{hhom}\<%
\\
\>[1][@{}l@{\AgdaIndent{0}}]%
\>[2]\AgdaKeyword{where}\<%
\\
%
\>[2]\AgdaFunction{hfunc}\AgdaSpace{}%
\AgdaSymbol{:}\AgdaSpace{}%
\AgdaOperator{\AgdaFunction{𝔻[}}\AgdaSpace{}%
\AgdaFunction{𝑻}\AgdaSpace{}%
\AgdaBound{X}\AgdaSpace{}%
\AgdaOperator{\AgdaFunction{]}}\AgdaSpace{}%
\AgdaOperator{\AgdaRecord{⟶}}\AgdaSpace{}%
\AgdaOperator{\AgdaFunction{𝔻[}}\AgdaSpace{}%
\AgdaBound{𝑨}\AgdaSpace{}%
\AgdaOperator{\AgdaFunction{]}}\<%
\\
%
\>[2]\AgdaFunction{hfunc}\AgdaSpace{}%
\AgdaSymbol{=}\AgdaSpace{}%
\AgdaFunction{free-lift-func}\<%
\\
%
\\[\AgdaEmptyExtraSkip]%
%
\>[2]\AgdaFunction{hcomp}\AgdaSpace{}%
\AgdaSymbol{:}\AgdaSpace{}%
\AgdaFunction{compatible-map}\AgdaSpace{}%
\AgdaSymbol{(}\AgdaFunction{𝑻}\AgdaSpace{}%
\AgdaBound{X}\AgdaSymbol{)}\AgdaSpace{}%
\AgdaBound{𝑨}\AgdaSpace{}%
\AgdaFunction{free-lift-func}\<%
\\
%
\>[2]\AgdaFunction{hcomp}\AgdaSpace{}%
\AgdaSymbol{\{}\AgdaBound{f}\AgdaSymbol{\}\{}\AgdaBound{a}\AgdaSymbol{\}}\AgdaSpace{}%
\AgdaSymbol{=}\AgdaSpace{}%
\AgdaField{cong}\AgdaSpace{}%
\AgdaSymbol{(}\AgdaField{Interp}\AgdaSpace{}%
\AgdaBound{𝑨}\AgdaSymbol{)}\AgdaSpace{}%
\AgdaSymbol{(}\AgdaInductiveConstructor{≡.refl}\AgdaSpace{}%
\AgdaOperator{\AgdaInductiveConstructor{,}}\AgdaSpace{}%
\AgdaSymbol{(λ}\AgdaSpace{}%
\AgdaBound{i}\AgdaSpace{}%
\AgdaSymbol{→}\AgdaSpace{}%
\AgdaSymbol{(}\AgdaField{cong}\AgdaSpace{}%
\AgdaFunction{free-lift-func}\AgdaSymbol{)\{}\AgdaBound{a}\AgdaSpace{}%
\AgdaBound{i}\AgdaSymbol{\}}\AgdaSpace{}%
\AgdaFunction{≐-isRefl}\AgdaSymbol{))}\<%
\\
%
\\[\AgdaEmptyExtraSkip]%
%
\>[2]\AgdaFunction{hhom}\AgdaSpace{}%
\AgdaSymbol{:}\AgdaSpace{}%
\AgdaRecord{IsHom}\AgdaSpace{}%
\AgdaSymbol{(}\AgdaFunction{𝑻}\AgdaSpace{}%
\AgdaBound{X}\AgdaSymbol{)}\AgdaSpace{}%
\AgdaBound{𝑨}\AgdaSpace{}%
\AgdaFunction{hfunc}\<%
\\
%
\>[2]\AgdaFunction{hhom}\AgdaSpace{}%
\AgdaSymbol{=}\AgdaSpace{}%
\AgdaInductiveConstructor{mkhom}\AgdaSpace{}%
\AgdaSymbol{(λ\{}\AgdaBound{f}\AgdaSymbol{\}\{}\AgdaBound{a}\AgdaSymbol{\}}\AgdaSpace{}%
\AgdaSymbol{→}\AgdaSpace{}%
\AgdaFunction{hcomp}\AgdaSymbol{\{}\AgdaBound{f}\AgdaSymbol{\}\{}\AgdaBound{a}\AgdaSymbol{\})}\<%
\\
%
\\[\AgdaEmptyExtraSkip]%
%
\\[\AgdaEmptyExtraSkip]%
\>[0]\AgdaKeyword{module}\AgdaSpace{}%
\AgdaModule{\AgdaUnderscore{}}\AgdaSpace{}%
\AgdaSymbol{\{}\AgdaBound{X}\AgdaSpace{}%
\AgdaSymbol{:}\AgdaSpace{}%
\AgdaPrimitive{Type}\AgdaSpace{}%
\AgdaGeneralizable{χ}\AgdaSymbol{\}\{}\AgdaBound{𝑨}\AgdaSpace{}%
\AgdaSymbol{:}\AgdaSpace{}%
\AgdaRecord{Algebra}\AgdaSpace{}%
\AgdaGeneralizable{α}\AgdaSpace{}%
\AgdaGeneralizable{ρᵃ}\AgdaSymbol{\}}\AgdaSpace{}%
\AgdaKeyword{where}\<%
\\
\>[0][@{}l@{\AgdaIndent{0}}]%
\>[1]\AgdaKeyword{open}\AgdaSpace{}%
\AgdaModule{Setoid}\AgdaSpace{}%
\AgdaOperator{\AgdaFunction{𝔻[}}\AgdaSpace{}%
\AgdaBound{𝑨}\AgdaSpace{}%
\AgdaOperator{\AgdaFunction{]}}\AgdaSpace{}%
\AgdaKeyword{using}\AgdaSpace{}%
\AgdaSymbol{(}\AgdaSpace{}%
\AgdaOperator{\AgdaField{\AgdaUnderscore{}≈\AgdaUnderscore{}}}\AgdaSpace{}%
\AgdaSymbol{;}\AgdaSpace{}%
\AgdaFunction{refl}\AgdaSpace{}%
\AgdaSymbol{)}\<%
\\
%
\>[1]\AgdaKeyword{open}\AgdaSpace{}%
\AgdaModule{Environment}\AgdaSpace{}%
\AgdaBound{𝑨}\AgdaSpace{}%
\AgdaKeyword{using}\AgdaSpace{}%
\AgdaSymbol{(}\AgdaSpace{}%
\AgdaOperator{\AgdaFunction{⟦\AgdaUnderscore{}⟧}}\AgdaSpace{}%
\AgdaSymbol{)}\<%
\\
%
\\[\AgdaEmptyExtraSkip]%
%
\>[1]\AgdaFunction{free-lift-interp}\AgdaSpace{}%
\AgdaSymbol{:}\AgdaSpace{}%
\AgdaSymbol{(}\AgdaBound{η}\AgdaSpace{}%
\AgdaSymbol{:}\AgdaSpace{}%
\AgdaBound{X}\AgdaSpace{}%
\AgdaSymbol{→}\AgdaSpace{}%
\AgdaOperator{\AgdaFunction{𝕌[}}\AgdaSpace{}%
\AgdaBound{𝑨}\AgdaSpace{}%
\AgdaOperator{\AgdaFunction{]}}\AgdaSymbol{)(}\AgdaBound{p}\AgdaSpace{}%
\AgdaSymbol{:}\AgdaSpace{}%
\AgdaDatatype{Term}\AgdaSpace{}%
\AgdaBound{X}\AgdaSymbol{)}\AgdaSpace{}%
\AgdaSymbol{→}\AgdaSpace{}%
\AgdaOperator{\AgdaFunction{⟦}}\AgdaSpace{}%
\AgdaBound{p}\AgdaSpace{}%
\AgdaOperator{\AgdaFunction{⟧}}\AgdaSpace{}%
\AgdaOperator{\AgdaField{⟨\$⟩}}\AgdaSpace{}%
\AgdaBound{η}\AgdaSpace{}%
\AgdaOperator{\AgdaFunction{≈}}\AgdaSpace{}%
\AgdaSymbol{(}\AgdaFunction{free-lift}\AgdaSpace{}%
\AgdaSymbol{\{}\AgdaArgument{𝑨}\AgdaSpace{}%
\AgdaSymbol{=}\AgdaSpace{}%
\AgdaBound{𝑨}\AgdaSymbol{\}}\AgdaSpace{}%
\AgdaBound{η}\AgdaSymbol{)}\AgdaSpace{}%
\AgdaBound{p}\<%
\\
%
\>[1]\AgdaFunction{free-lift-interp}\AgdaSpace{}%
\AgdaBound{η}\AgdaSpace{}%
\AgdaSymbol{(}\AgdaInductiveConstructor{ℊ}\AgdaSpace{}%
\AgdaBound{x}\AgdaSymbol{)}\AgdaSpace{}%
\AgdaSymbol{=}\AgdaSpace{}%
\AgdaFunction{refl}\<%
\\
%
\>[1]\AgdaFunction{free-lift-interp}\AgdaSpace{}%
\AgdaBound{η}\AgdaSpace{}%
\AgdaSymbol{(}\AgdaInductiveConstructor{node}\AgdaSpace{}%
\AgdaBound{f}\AgdaSpace{}%
\AgdaBound{t}\AgdaSymbol{)}\AgdaSpace{}%
\AgdaSymbol{=}\AgdaSpace{}%
\AgdaField{cong}\AgdaSpace{}%
\AgdaSymbol{(}\AgdaField{Interp}\AgdaSpace{}%
\AgdaBound{𝑨}\AgdaSymbol{)}\AgdaSpace{}%
\AgdaSymbol{(}\AgdaInductiveConstructor{≡.refl}\AgdaSpace{}%
\AgdaOperator{\AgdaInductiveConstructor{,}}\AgdaSpace{}%
\AgdaSymbol{(}\AgdaFunction{free-lift-interp}\AgdaSpace{}%
\AgdaBound{η}\AgdaSymbol{)}\AgdaSpace{}%
\AgdaOperator{\AgdaFunction{∘}}\AgdaSpace{}%
\AgdaBound{t}\AgdaSymbol{)}\<%
\end{code}

\hypertarget{the-relatively-free-algebra-ux1d53d}{%
\subsubsection{The relatively free algebra
𝔽}\label{the-relatively-free-algebra-ux1d53d}}

We now define the algebra \texttt{𝔽{[}\ X\ {]}}, which plays the role of
the relatively free algebra, along with the natural epimorphism
\texttt{epi𝔽\ :\ epi\ (𝑻\ X)\ 𝔽{[}\ X\ {]}} from \texttt{𝑻\ X} to
\texttt{𝔽{[}\ X\ {]}}.

\begin{code}%
\>[0]\<%
\\
\>[0]\AgdaKeyword{module}\AgdaSpace{}%
\AgdaModule{FreeAlgebra}\AgdaSpace{}%
\AgdaSymbol{\{}\AgdaBound{χ}\AgdaSpace{}%
\AgdaSymbol{:}\AgdaSpace{}%
\AgdaPostulate{Level}\AgdaSymbol{\}\{}\AgdaBound{ι}\AgdaSpace{}%
\AgdaSymbol{:}\AgdaSpace{}%
\AgdaPostulate{Level}\AgdaSymbol{\}\{}\AgdaBound{I}\AgdaSpace{}%
\AgdaSymbol{:}\AgdaSpace{}%
\AgdaPrimitive{Type}\AgdaSpace{}%
\AgdaBound{ι}\AgdaSymbol{\}(}\AgdaBound{ℰ}\AgdaSpace{}%
\AgdaSymbol{:}\AgdaSpace{}%
\AgdaBound{I}\AgdaSpace{}%
\AgdaSymbol{→}\AgdaSpace{}%
\AgdaRecord{Eq}\AgdaSymbol{)}\AgdaSpace{}%
\AgdaKeyword{where}\<%
\\
\>[0][@{}l@{\AgdaIndent{0}}]%
\>[1]\AgdaKeyword{open}\AgdaSpace{}%
\AgdaModule{Algebra}\<%
\\
%
\\[\AgdaEmptyExtraSkip]%
%
\>[1]\AgdaFunction{FreeDomain}\AgdaSpace{}%
\AgdaSymbol{:}\AgdaSpace{}%
\AgdaPrimitive{Type}\AgdaSpace{}%
\AgdaBound{χ}\AgdaSpace{}%
\AgdaSymbol{→}\AgdaSpace{}%
\AgdaRecord{Setoid}\AgdaSpace{}%
\AgdaSymbol{\AgdaUnderscore{}}\AgdaSpace{}%
\AgdaSymbol{\AgdaUnderscore{}}\<%
\\
%
\>[1]\AgdaFunction{FreeDomain}\AgdaSpace{}%
\AgdaBound{X}\AgdaSpace{}%
\AgdaSymbol{=}\AgdaSpace{}%
\AgdaKeyword{record}%
\>[24]\AgdaSymbol{\{}\AgdaSpace{}%
\AgdaField{Carrier}%
\>[41]\AgdaSymbol{=}\AgdaSpace{}%
\AgdaDatatype{Term}\AgdaSpace{}%
\AgdaBound{X}\<%
\\
%
\>[24]\AgdaSymbol{;}\AgdaSpace{}%
\AgdaOperator{\AgdaField{\AgdaUnderscore{}≈\AgdaUnderscore{}}}%
\>[41]\AgdaSymbol{=}\AgdaSpace{}%
\AgdaBound{ℰ}\AgdaSpace{}%
\AgdaOperator{\AgdaDatatype{⊢}}\AgdaSpace{}%
\AgdaBound{X}\AgdaSpace{}%
\AgdaOperator{\AgdaDatatype{▹\AgdaUnderscore{}≈\AgdaUnderscore{}}}\<%
\\
%
\>[24]\AgdaSymbol{;}\AgdaSpace{}%
\AgdaField{isEquivalence}%
\>[41]\AgdaSymbol{=}\AgdaSpace{}%
\AgdaFunction{⊢▹≈IsEquiv}\AgdaSpace{}%
\AgdaSymbol{\}}\<%
\end{code}

The interpretation of an operation is simply the operation itself. This
works since \texttt{ℰ\ ⊢\ X\ ▹\_≈\_} is a congruence.

\begin{code}%
\>[0]\<%
\\
%
\>[1]\AgdaFunction{FreeInterp}\AgdaSpace{}%
\AgdaSymbol{:}\AgdaSpace{}%
\AgdaSymbol{∀}\AgdaSpace{}%
\AgdaSymbol{\{}\AgdaBound{X}\AgdaSymbol{\}}\AgdaSpace{}%
\AgdaSymbol{→}\AgdaSpace{}%
\AgdaOperator{\AgdaFunction{⟨}}\AgdaSpace{}%
\AgdaBound{𝑆}\AgdaSpace{}%
\AgdaOperator{\AgdaFunction{⟩}}\AgdaSpace{}%
\AgdaSymbol{(}\AgdaFunction{FreeDomain}\AgdaSpace{}%
\AgdaBound{X}\AgdaSymbol{)}\AgdaSpace{}%
\AgdaOperator{\AgdaRecord{⟶}}\AgdaSpace{}%
\AgdaFunction{FreeDomain}\AgdaSpace{}%
\AgdaBound{X}\<%
\\
%
\>[1]\AgdaFunction{FreeInterp}\AgdaSpace{}%
\AgdaOperator{\AgdaField{⟨\$⟩}}\AgdaSpace{}%
\AgdaSymbol{(}\AgdaBound{f}\AgdaSpace{}%
\AgdaOperator{\AgdaInductiveConstructor{,}}\AgdaSpace{}%
\AgdaBound{ts}\AgdaSymbol{)}\AgdaSpace{}%
\AgdaSymbol{=}\AgdaSpace{}%
\AgdaInductiveConstructor{node}\AgdaSpace{}%
\AgdaBound{f}\AgdaSpace{}%
\AgdaBound{ts}\<%
\\
%
\>[1]\AgdaField{cong}\AgdaSpace{}%
\AgdaFunction{FreeInterp}\AgdaSpace{}%
\AgdaSymbol{(}\AgdaInductiveConstructor{≡.refl}\AgdaSpace{}%
\AgdaOperator{\AgdaInductiveConstructor{,}}\AgdaSpace{}%
\AgdaBound{h}\AgdaSymbol{)}\AgdaSpace{}%
\AgdaSymbol{=}\AgdaSpace{}%
\AgdaInductiveConstructor{app}\AgdaSpace{}%
\AgdaBound{h}\<%
\\
%
\\[\AgdaEmptyExtraSkip]%
%
\>[1]\AgdaOperator{\AgdaFunction{𝔽[\AgdaUnderscore{}]}}\AgdaSpace{}%
\AgdaSymbol{:}\AgdaSpace{}%
\AgdaPrimitive{Type}\AgdaSpace{}%
\AgdaBound{χ}\AgdaSpace{}%
\AgdaSymbol{→}\AgdaSpace{}%
\AgdaRecord{Algebra}\AgdaSpace{}%
\AgdaSymbol{(}\AgdaFunction{ov}\AgdaSpace{}%
\AgdaBound{χ}\AgdaSymbol{)}\AgdaSpace{}%
\AgdaSymbol{(}\AgdaBound{ι}\AgdaSpace{}%
\AgdaOperator{\AgdaPrimitive{⊔}}\AgdaSpace{}%
\AgdaFunction{ov}\AgdaSpace{}%
\AgdaBound{χ}\AgdaSymbol{)}\<%
\\
%
\>[1]\AgdaField{Domain}\AgdaSpace{}%
\AgdaOperator{\AgdaFunction{𝔽[}}\AgdaSpace{}%
\AgdaBound{X}\AgdaSpace{}%
\AgdaOperator{\AgdaFunction{]}}\AgdaSpace{}%
\AgdaSymbol{=}\AgdaSpace{}%
\AgdaFunction{FreeDomain}\AgdaSpace{}%
\AgdaBound{X}\<%
\\
%
\>[1]\AgdaField{Interp}\AgdaSpace{}%
\AgdaOperator{\AgdaFunction{𝔽[}}\AgdaSpace{}%
\AgdaBound{X}\AgdaSpace{}%
\AgdaOperator{\AgdaFunction{]}}\AgdaSpace{}%
\AgdaSymbol{=}\AgdaSpace{}%
\AgdaFunction{FreeInterp}\<%
\end{code}

\hypertarget{basic-properties-of-free-algebras}{%
\subsubsection{Basic properties of free
algebras}\label{basic-properties-of-free-algebras}}

In the code below, \texttt{X} will play the role of an arbitrary
collection of variables; it would suffice to take \texttt{X} to be the
cardinality of the largest algebra in 𝒦, but since we don't know that
cardinality, we leave \texttt{X} aribtrary for now.

\begin{code}%
\>[0]\<%
\\
\>[0]\AgdaKeyword{module}\AgdaSpace{}%
\AgdaModule{FreeHom}\AgdaSpace{}%
\AgdaSymbol{(}\AgdaBound{χ}\AgdaSpace{}%
\AgdaSymbol{:}\AgdaSpace{}%
\AgdaPostulate{Level}\AgdaSymbol{)}\AgdaSpace{}%
\AgdaSymbol{\{}\AgdaBound{𝒦}\AgdaSpace{}%
\AgdaSymbol{:}\AgdaSpace{}%
\AgdaFunction{Pred}\AgdaSymbol{(}\AgdaRecord{Algebra}\AgdaSpace{}%
\AgdaGeneralizable{α}\AgdaSpace{}%
\AgdaGeneralizable{ρᵃ}\AgdaSymbol{)}\AgdaSpace{}%
\AgdaSymbol{(}\AgdaGeneralizable{α}\AgdaSpace{}%
\AgdaOperator{\AgdaPrimitive{⊔}}\AgdaSpace{}%
\AgdaGeneralizable{ρᵃ}\AgdaSpace{}%
\AgdaOperator{\AgdaPrimitive{⊔}}\AgdaSpace{}%
\AgdaFunction{ov}\AgdaSpace{}%
\AgdaGeneralizable{ℓ}\AgdaSymbol{)\}}\AgdaSpace{}%
\AgdaKeyword{where}\<%
\\
\>[0][@{}l@{\AgdaIndent{0}}]%
\>[1]\AgdaKeyword{private}\AgdaSpace{}%
\AgdaFunction{ι}\AgdaSpace{}%
\AgdaSymbol{=}\AgdaSpace{}%
\AgdaFunction{ov}\AgdaSymbol{(}\AgdaBound{χ}\AgdaSpace{}%
\AgdaOperator{\AgdaPrimitive{⊔}}\AgdaSpace{}%
\AgdaBound{α}\AgdaSpace{}%
\AgdaOperator{\AgdaPrimitive{⊔}}\AgdaSpace{}%
\AgdaBound{ρᵃ}\AgdaSpace{}%
\AgdaOperator{\AgdaPrimitive{⊔}}\AgdaSpace{}%
\AgdaBound{ℓ}\AgdaSymbol{)}\<%
\\
%
\>[1]\AgdaKeyword{open}\AgdaSpace{}%
\AgdaModule{Eq}\<%
\\
%
\\[\AgdaEmptyExtraSkip]%
%
\>[1]\AgdaFunction{ℐ}\AgdaSpace{}%
\AgdaSymbol{:}\AgdaSpace{}%
\AgdaPrimitive{Type}\AgdaSpace{}%
\AgdaFunction{ι}\AgdaSpace{}%
\AgdaComment{--\ indexes\ the\ collection\ of\ equations\ modeled\ by\ 𝒦}\<%
\\
%
\>[1]\AgdaFunction{ℐ}\AgdaSpace{}%
\AgdaSymbol{=}\AgdaSpace{}%
\AgdaFunction{Σ[}\AgdaSpace{}%
\AgdaBound{eq}\AgdaSpace{}%
\AgdaFunction{∈}\AgdaSpace{}%
\AgdaRecord{Eq}\AgdaSymbol{\{}\AgdaBound{χ}\AgdaSymbol{\}}\AgdaSpace{}%
\AgdaFunction{]}\AgdaSpace{}%
\AgdaBound{𝒦}\AgdaSpace{}%
\AgdaOperator{\AgdaFunction{⊫}}\AgdaSpace{}%
\AgdaSymbol{((}\AgdaField{lhs}\AgdaSpace{}%
\AgdaBound{eq}\AgdaSymbol{)}\AgdaSpace{}%
\AgdaOperator{\AgdaInductiveConstructor{≈̇}}\AgdaSpace{}%
\AgdaSymbol{(}\AgdaField{rhs}\AgdaSpace{}%
\AgdaBound{eq}\AgdaSymbol{))}\<%
\\
%
\\[\AgdaEmptyExtraSkip]%
%
\>[1]\AgdaFunction{ℰ}\AgdaSpace{}%
\AgdaSymbol{:}\AgdaSpace{}%
\AgdaFunction{ℐ}\AgdaSpace{}%
\AgdaSymbol{→}\AgdaSpace{}%
\AgdaRecord{Eq}\<%
\\
%
\>[1]\AgdaFunction{ℰ}\AgdaSpace{}%
\AgdaSymbol{(}\AgdaBound{eqv}\AgdaSpace{}%
\AgdaOperator{\AgdaInductiveConstructor{,}}\AgdaSpace{}%
\AgdaBound{p}\AgdaSymbol{)}\AgdaSpace{}%
\AgdaSymbol{=}\AgdaSpace{}%
\AgdaBound{eqv}\<%
\\
%
\\[\AgdaEmptyExtraSkip]%
%
\>[1]\AgdaOperator{\AgdaFunction{ℰ⊢[\AgdaUnderscore{}]▹Th𝒦}}\AgdaSpace{}%
\AgdaSymbol{:}\AgdaSpace{}%
\AgdaSymbol{(}\AgdaBound{X}\AgdaSpace{}%
\AgdaSymbol{:}\AgdaSpace{}%
\AgdaPrimitive{Type}\AgdaSpace{}%
\AgdaBound{χ}\AgdaSymbol{)}\AgdaSpace{}%
\AgdaSymbol{→}\AgdaSpace{}%
\AgdaSymbol{∀\{}\AgdaBound{p}\AgdaSpace{}%
\AgdaBound{q}\AgdaSymbol{\}}\AgdaSpace{}%
\AgdaSymbol{→}\AgdaSpace{}%
\AgdaFunction{ℰ}\AgdaSpace{}%
\AgdaOperator{\AgdaDatatype{⊢}}\AgdaSpace{}%
\AgdaBound{X}\AgdaSpace{}%
\AgdaOperator{\AgdaDatatype{▹}}\AgdaSpace{}%
\AgdaBound{p}\AgdaSpace{}%
\AgdaOperator{\AgdaDatatype{≈}}\AgdaSpace{}%
\AgdaBound{q}\AgdaSpace{}%
\AgdaSymbol{→}\AgdaSpace{}%
\AgdaBound{𝒦}\AgdaSpace{}%
\AgdaOperator{\AgdaFunction{⊫}}\AgdaSpace{}%
\AgdaSymbol{(}\AgdaBound{p}\AgdaSpace{}%
\AgdaOperator{\AgdaInductiveConstructor{≈̇}}\AgdaSpace{}%
\AgdaBound{q}\AgdaSymbol{)}\<%
\\
%
\>[1]\AgdaOperator{\AgdaFunction{ℰ⊢[}}\AgdaSpace{}%
\AgdaBound{X}\AgdaSpace{}%
\AgdaOperator{\AgdaFunction{]▹Th𝒦}}\AgdaSpace{}%
\AgdaBound{x}\AgdaSpace{}%
\AgdaBound{𝑨}\AgdaSpace{}%
\AgdaBound{kA}\AgdaSpace{}%
\AgdaSymbol{=}\AgdaSpace{}%
\AgdaFunction{sound}\AgdaSpace{}%
\AgdaSymbol{(λ}\AgdaSpace{}%
\AgdaBound{i}\AgdaSpace{}%
\AgdaBound{ρ}\AgdaSpace{}%
\AgdaSymbol{→}\AgdaSpace{}%
\AgdaOperator{\AgdaFunction{∥}}\AgdaSpace{}%
\AgdaBound{i}\AgdaSpace{}%
\AgdaOperator{\AgdaFunction{∥}}\AgdaSpace{}%
\AgdaBound{𝑨}\AgdaSpace{}%
\AgdaBound{kA}\AgdaSpace{}%
\AgdaBound{ρ}\AgdaSymbol{)}\AgdaSpace{}%
\AgdaBound{x}\<%
\\
\>[1][@{}l@{\AgdaIndent{0}}]%
\>[2]\AgdaKeyword{where}\AgdaSpace{}%
\AgdaKeyword{open}\AgdaSpace{}%
\AgdaModule{Soundness}\AgdaSpace{}%
\AgdaFunction{ℰ}\AgdaSpace{}%
\AgdaBound{𝑨}\<%
\\
%
\>[1]\AgdaKeyword{open}\AgdaSpace{}%
\AgdaModule{FreeAlgebra}\AgdaSpace{}%
\AgdaSymbol{\{}\AgdaArgument{ι}\AgdaSpace{}%
\AgdaSymbol{=}\AgdaSpace{}%
\AgdaFunction{ι}\AgdaSymbol{\}\{}\AgdaArgument{I}\AgdaSpace{}%
\AgdaSymbol{=}\AgdaSpace{}%
\AgdaFunction{ℐ}\AgdaSymbol{\}}\AgdaSpace{}%
\AgdaFunction{ℰ}\AgdaSpace{}%
\AgdaKeyword{using}\AgdaSpace{}%
\AgdaSymbol{(}\AgdaSpace{}%
\AgdaOperator{\AgdaFunction{𝔽[\AgdaUnderscore{}]}}\AgdaSpace{}%
\AgdaSymbol{)}\<%
\end{code}

\hypertarget{the-natural-epimorphism-from-ux1d47b-x-to-ux1d53d-x}{%
\paragraph{The natural epimorphism from 𝑻 X to 𝔽{[} X
{]}}\label{the-natural-epimorphism-from-ux1d47b-x-to-ux1d53d-x}}

Next we define an epimorphism from \texttt{𝑻\ X} onto the relatively
free algebra \texttt{𝔽{[}\ X\ {]}}. Of course, the kernel of this
epimorphism will be the congruence of \texttt{𝑻\ X} defined by
identities modeled by (\texttt{S\ 𝒦}, hence) \texttt{𝒦}.

\begin{code}%
\>[0]\<%
\\
%
\>[1]\AgdaOperator{\AgdaFunction{epi𝔽[\AgdaUnderscore{}]}}\AgdaSpace{}%
\AgdaSymbol{:}\AgdaSpace{}%
\AgdaSymbol{(}\AgdaBound{X}\AgdaSpace{}%
\AgdaSymbol{:}\AgdaSpace{}%
\AgdaPrimitive{Type}\AgdaSpace{}%
\AgdaBound{χ}\AgdaSymbol{)}\AgdaSpace{}%
\AgdaSymbol{→}\AgdaSpace{}%
\AgdaFunction{epi}\AgdaSpace{}%
\AgdaSymbol{(}\AgdaFunction{𝑻}\AgdaSpace{}%
\AgdaBound{X}\AgdaSymbol{)}\AgdaSpace{}%
\AgdaOperator{\AgdaFunction{𝔽[}}\AgdaSpace{}%
\AgdaBound{X}\AgdaSpace{}%
\AgdaOperator{\AgdaFunction{]}}\<%
\\
%
\>[1]\AgdaOperator{\AgdaFunction{epi𝔽[}}\AgdaSpace{}%
\AgdaBound{X}\AgdaSpace{}%
\AgdaOperator{\AgdaFunction{]}}\AgdaSpace{}%
\AgdaSymbol{=}\AgdaSpace{}%
\AgdaFunction{h}\AgdaSpace{}%
\AgdaOperator{\AgdaInductiveConstructor{,}}\AgdaSpace{}%
\AgdaFunction{hepi}\<%
\\
\>[1][@{}l@{\AgdaIndent{0}}]%
\>[2]\AgdaKeyword{where}\<%
\\
%
\>[2]\AgdaKeyword{open}\AgdaSpace{}%
\AgdaModule{Algebra}\AgdaSpace{}%
\AgdaSymbol{(}\AgdaFunction{𝑻}\AgdaSpace{}%
\AgdaBound{X}\AgdaSymbol{)}\AgdaSpace{}%
\AgdaKeyword{using}\AgdaSpace{}%
\AgdaSymbol{()}\AgdaSpace{}%
\AgdaKeyword{renaming}\AgdaSpace{}%
\AgdaSymbol{(}\AgdaField{Domain}\AgdaSpace{}%
\AgdaSymbol{to}\AgdaSpace{}%
\AgdaField{TX}\AgdaSymbol{)}\<%
\\
%
\>[2]\AgdaKeyword{open}\AgdaSpace{}%
\AgdaModule{Setoid}\AgdaSpace{}%
\AgdaFunction{TX}\AgdaSpace{}%
\AgdaKeyword{using}\AgdaSpace{}%
\AgdaSymbol{()}\AgdaSpace{}%
\AgdaKeyword{renaming}\AgdaSpace{}%
\AgdaSymbol{(}\AgdaSpace{}%
\AgdaOperator{\AgdaField{\AgdaUnderscore{}≈\AgdaUnderscore{}}}\AgdaSpace{}%
\AgdaSymbol{to}\AgdaSpace{}%
\AgdaOperator{\AgdaField{\AgdaUnderscore{}≈₀\AgdaUnderscore{}}}\AgdaSpace{}%
\AgdaSymbol{;}\AgdaSpace{}%
\AgdaFunction{refl}\AgdaSpace{}%
\AgdaSymbol{to}\AgdaSpace{}%
\AgdaFunction{refl₀}\AgdaSpace{}%
\AgdaSymbol{)}\<%
\\
%
\>[2]\AgdaKeyword{open}\AgdaSpace{}%
\AgdaModule{Algebra}\AgdaSpace{}%
\AgdaOperator{\AgdaFunction{𝔽[}}\AgdaSpace{}%
\AgdaBound{X}\AgdaSpace{}%
\AgdaOperator{\AgdaFunction{]}}\AgdaSpace{}%
\AgdaKeyword{using}\AgdaSpace{}%
\AgdaSymbol{()}\AgdaSpace{}%
\AgdaKeyword{renaming}\AgdaSpace{}%
\AgdaSymbol{(}\AgdaSpace{}%
\AgdaField{Domain}\AgdaSpace{}%
\AgdaSymbol{to}\AgdaSpace{}%
\AgdaField{F}\AgdaSpace{}%
\AgdaSymbol{)}\<%
\\
%
\>[2]\AgdaKeyword{open}\AgdaSpace{}%
\AgdaModule{Setoid}\AgdaSpace{}%
\AgdaFunction{F}\AgdaSpace{}%
\AgdaKeyword{using}\AgdaSpace{}%
\AgdaSymbol{()}\AgdaSpace{}%
\AgdaKeyword{renaming}\AgdaSpace{}%
\AgdaSymbol{(}\AgdaSpace{}%
\AgdaOperator{\AgdaField{\AgdaUnderscore{}≈\AgdaUnderscore{}}}%
\>[41]\AgdaSymbol{to}\AgdaSpace{}%
\AgdaOperator{\AgdaField{\AgdaUnderscore{}≈₁\AgdaUnderscore{}}}\AgdaSpace{}%
\AgdaSymbol{;}\AgdaSpace{}%
\AgdaFunction{refl}\AgdaSpace{}%
\AgdaSymbol{to}\AgdaSpace{}%
\AgdaFunction{refl₁}\AgdaSpace{}%
\AgdaSymbol{)}\<%
\\
%
\>[2]\AgdaKeyword{open}\AgdaSpace{}%
\AgdaOperator{\AgdaModule{\AgdaUnderscore{}≐\AgdaUnderscore{}}}\<%
\\
%
\\[\AgdaEmptyExtraSkip]%
%
\>[2]\AgdaFunction{c}\AgdaSpace{}%
\AgdaSymbol{:}\AgdaSpace{}%
\AgdaSymbol{∀}\AgdaSpace{}%
\AgdaSymbol{\{}\AgdaBound{x}\AgdaSpace{}%
\AgdaBound{y}\AgdaSymbol{\}}\AgdaSpace{}%
\AgdaSymbol{→}\AgdaSpace{}%
\AgdaBound{x}\AgdaSpace{}%
\AgdaOperator{\AgdaFunction{≈₀}}\AgdaSpace{}%
\AgdaBound{y}\AgdaSpace{}%
\AgdaSymbol{→}\AgdaSpace{}%
\AgdaBound{x}\AgdaSpace{}%
\AgdaOperator{\AgdaFunction{≈₁}}\AgdaSpace{}%
\AgdaBound{y}\<%
\\
%
\>[2]\AgdaFunction{c}\AgdaSpace{}%
\AgdaSymbol{(}\AgdaInductiveConstructor{rfl}\AgdaSpace{}%
\AgdaSymbol{\{}\AgdaBound{x}\AgdaSymbol{\}\{}\AgdaBound{y}\AgdaSymbol{\}}\AgdaSpace{}%
\AgdaInductiveConstructor{≡.refl}\AgdaSymbol{)}\AgdaSpace{}%
\AgdaSymbol{=}\AgdaSpace{}%
\AgdaFunction{refl₁}\<%
\\
%
\>[2]\AgdaFunction{c}\AgdaSpace{}%
\AgdaSymbol{(}\AgdaInductiveConstructor{gnl}\AgdaSpace{}%
\AgdaSymbol{\{}\AgdaBound{f}\AgdaSymbol{\}\{}\AgdaBound{s}\AgdaSymbol{\}\{}\AgdaBound{t}\AgdaSymbol{\}}\AgdaSpace{}%
\AgdaBound{x}\AgdaSymbol{)}\AgdaSpace{}%
\AgdaSymbol{=}\AgdaSpace{}%
\AgdaField{cong}\AgdaSpace{}%
\AgdaSymbol{(}\AgdaField{Interp}\AgdaSpace{}%
\AgdaOperator{\AgdaFunction{𝔽[}}\AgdaSpace{}%
\AgdaBound{X}\AgdaSpace{}%
\AgdaOperator{\AgdaFunction{]}}\AgdaSymbol{)}\AgdaSpace{}%
\AgdaSymbol{(}\AgdaInductiveConstructor{≡.refl}\AgdaSpace{}%
\AgdaOperator{\AgdaInductiveConstructor{,}}\AgdaSpace{}%
\AgdaFunction{c}\AgdaSpace{}%
\AgdaOperator{\AgdaFunction{∘}}\AgdaSpace{}%
\AgdaBound{x}\AgdaSymbol{)}\<%
\\
%
\\[\AgdaEmptyExtraSkip]%
%
\>[2]\AgdaFunction{h}\AgdaSpace{}%
\AgdaSymbol{:}\AgdaSpace{}%
\AgdaFunction{TX}\AgdaSpace{}%
\AgdaOperator{\AgdaRecord{⟶}}\AgdaSpace{}%
\AgdaFunction{F}\<%
\\
%
\>[2]\AgdaFunction{h}\AgdaSpace{}%
\AgdaSymbol{=}\AgdaSpace{}%
\AgdaKeyword{record}\AgdaSpace{}%
\AgdaSymbol{\{}\AgdaSpace{}%
\AgdaField{f}\AgdaSpace{}%
\AgdaSymbol{=}\AgdaSpace{}%
\AgdaFunction{id}\AgdaSpace{}%
\AgdaSymbol{;}\AgdaSpace{}%
\AgdaField{cong}\AgdaSpace{}%
\AgdaSymbol{=}\AgdaSpace{}%
\AgdaFunction{c}\AgdaSpace{}%
\AgdaSymbol{\}}\<%
\\
%
\\[\AgdaEmptyExtraSkip]%
%
\>[2]\AgdaFunction{hepi}\AgdaSpace{}%
\AgdaSymbol{:}\AgdaSpace{}%
\AgdaRecord{IsEpi}\AgdaSpace{}%
\AgdaSymbol{(}\AgdaFunction{𝑻}\AgdaSpace{}%
\AgdaBound{X}\AgdaSymbol{)}\AgdaSpace{}%
\AgdaOperator{\AgdaFunction{𝔽[}}\AgdaSpace{}%
\AgdaBound{X}\AgdaSpace{}%
\AgdaOperator{\AgdaFunction{]}}\AgdaSpace{}%
\AgdaFunction{h}\<%
\\
%
\>[2]\AgdaField{compatible}\AgdaSpace{}%
\AgdaSymbol{(}\AgdaField{isHom}\AgdaSpace{}%
\AgdaFunction{hepi}\AgdaSymbol{)}\AgdaSpace{}%
\AgdaSymbol{=}\AgdaSpace{}%
\AgdaField{cong}\AgdaSpace{}%
\AgdaFunction{h}\AgdaSpace{}%
\AgdaFunction{refl₀}\<%
\\
%
\>[2]\AgdaField{isSurjective}\AgdaSpace{}%
\AgdaFunction{hepi}\AgdaSpace{}%
\AgdaSymbol{\{}\AgdaBound{y}\AgdaSymbol{\}}\AgdaSpace{}%
\AgdaSymbol{=}\AgdaSpace{}%
\AgdaInductiveConstructor{eq}\AgdaSpace{}%
\AgdaBound{y}\AgdaSpace{}%
\AgdaFunction{refl₁}\<%
\\
%
\\[\AgdaEmptyExtraSkip]%
%
\>[1]\AgdaOperator{\AgdaFunction{hom𝔽[\AgdaUnderscore{}]}}\AgdaSpace{}%
\AgdaSymbol{:}\AgdaSpace{}%
\AgdaSymbol{(}\AgdaBound{X}\AgdaSpace{}%
\AgdaSymbol{:}\AgdaSpace{}%
\AgdaPrimitive{Type}\AgdaSpace{}%
\AgdaBound{χ}\AgdaSymbol{)}\AgdaSpace{}%
\AgdaSymbol{→}\AgdaSpace{}%
\AgdaFunction{hom}\AgdaSpace{}%
\AgdaSymbol{(}\AgdaFunction{𝑻}\AgdaSpace{}%
\AgdaBound{X}\AgdaSymbol{)}\AgdaSpace{}%
\AgdaOperator{\AgdaFunction{𝔽[}}\AgdaSpace{}%
\AgdaBound{X}\AgdaSpace{}%
\AgdaOperator{\AgdaFunction{]}}\<%
\\
%
\>[1]\AgdaOperator{\AgdaFunction{hom𝔽[}}\AgdaSpace{}%
\AgdaBound{X}\AgdaSpace{}%
\AgdaOperator{\AgdaFunction{]}}\AgdaSpace{}%
\AgdaSymbol{=}\AgdaSpace{}%
\AgdaFunction{IsEpi.HomReduct}\AgdaSpace{}%
\AgdaOperator{\AgdaFunction{∥}}\AgdaSpace{}%
\AgdaOperator{\AgdaFunction{epi𝔽[}}\AgdaSpace{}%
\AgdaBound{X}\AgdaSpace{}%
\AgdaOperator{\AgdaFunction{]}}\AgdaSpace{}%
\AgdaOperator{\AgdaFunction{∥}}\<%
\\
%
\\[\AgdaEmptyExtraSkip]%
%
\>[1]\AgdaOperator{\AgdaFunction{hom𝔽[\AgdaUnderscore{}]-is-epic}}\AgdaSpace{}%
\AgdaSymbol{:}\AgdaSpace{}%
\AgdaSymbol{(}\AgdaBound{X}\AgdaSpace{}%
\AgdaSymbol{:}\AgdaSpace{}%
\AgdaPrimitive{Type}\AgdaSpace{}%
\AgdaBound{χ}\AgdaSymbol{)}\AgdaSpace{}%
\AgdaSymbol{→}\AgdaSpace{}%
\AgdaFunction{IsSurjective}\AgdaSpace{}%
\AgdaOperator{\AgdaFunction{∣}}\AgdaSpace{}%
\AgdaOperator{\AgdaFunction{hom𝔽[}}\AgdaSpace{}%
\AgdaBound{X}\AgdaSpace{}%
\AgdaOperator{\AgdaFunction{]}}\AgdaSpace{}%
\AgdaOperator{\AgdaFunction{∣}}\<%
\\
%
\>[1]\AgdaOperator{\AgdaFunction{hom𝔽[}}\AgdaSpace{}%
\AgdaBound{X}\AgdaSpace{}%
\AgdaOperator{\AgdaFunction{]-is-epic}}\AgdaSpace{}%
\AgdaSymbol{=}\AgdaSpace{}%
\AgdaField{IsEpi.isSurjective}\AgdaSpace{}%
\AgdaSymbol{(}\AgdaField{snd}\AgdaSpace{}%
\AgdaSymbol{(}\AgdaOperator{\AgdaFunction{epi𝔽[}}\AgdaSpace{}%
\AgdaBound{X}\AgdaSpace{}%
\AgdaOperator{\AgdaFunction{]}}\AgdaSymbol{))}\<%
\end{code}

\hypertarget{the-kernel-of-the-natural-epimorphism}{%
\paragraph{The kernel of the natural
epimorphism}\label{the-kernel-of-the-natural-epimorphism}}

\begin{code}%
\>[0]\<%
\\
%
\>[1]\AgdaFunction{class-models-kernel}\AgdaSpace{}%
\AgdaSymbol{:}\AgdaSpace{}%
\AgdaSymbol{∀\{}\AgdaBound{X}\AgdaSpace{}%
\AgdaBound{p}\AgdaSpace{}%
\AgdaBound{q}\AgdaSymbol{\}}\AgdaSpace{}%
\AgdaSymbol{→}\AgdaSpace{}%
\AgdaSymbol{(}\AgdaBound{p}\AgdaSpace{}%
\AgdaOperator{\AgdaInductiveConstructor{,}}\AgdaSpace{}%
\AgdaBound{q}\AgdaSymbol{)}\AgdaSpace{}%
\AgdaOperator{\AgdaFunction{∈}}\AgdaSpace{}%
\AgdaFunction{ker}\AgdaSpace{}%
\AgdaOperator{\AgdaFunction{∣}}\AgdaSpace{}%
\AgdaOperator{\AgdaFunction{hom𝔽[}}\AgdaSpace{}%
\AgdaBound{X}\AgdaSpace{}%
\AgdaOperator{\AgdaFunction{]}}\AgdaSpace{}%
\AgdaOperator{\AgdaFunction{∣}}\AgdaSpace{}%
\AgdaSymbol{→}\AgdaSpace{}%
\AgdaBound{𝒦}\AgdaSpace{}%
\AgdaOperator{\AgdaFunction{⊫}}\AgdaSpace{}%
\AgdaSymbol{(}\AgdaBound{p}\AgdaSpace{}%
\AgdaOperator{\AgdaInductiveConstructor{≈̇}}\AgdaSpace{}%
\AgdaBound{q}\AgdaSymbol{)}\<%
\\
%
\>[1]\AgdaFunction{class-models-kernel}\AgdaSpace{}%
\AgdaSymbol{\{}\AgdaArgument{X}\AgdaSpace{}%
\AgdaSymbol{=}\AgdaSpace{}%
\AgdaBound{X}\AgdaSymbol{\}\{}\AgdaBound{p}\AgdaSymbol{\}\{}\AgdaBound{q}\AgdaSymbol{\}}\AgdaSpace{}%
\AgdaBound{pKq}\AgdaSpace{}%
\AgdaSymbol{=}\AgdaSpace{}%
\AgdaOperator{\AgdaFunction{ℰ⊢[}}\AgdaSpace{}%
\AgdaBound{X}\AgdaSpace{}%
\AgdaOperator{\AgdaFunction{]▹Th𝒦}}\AgdaSpace{}%
\AgdaBound{pKq}\<%
\\
%
\\[\AgdaEmptyExtraSkip]%
%
\>[1]\AgdaFunction{kernel-in-theory}\AgdaSpace{}%
\AgdaSymbol{:}\AgdaSpace{}%
\AgdaSymbol{\{}\AgdaBound{X}\AgdaSpace{}%
\AgdaSymbol{:}\AgdaSpace{}%
\AgdaPrimitive{Type}\AgdaSpace{}%
\AgdaBound{χ}\AgdaSymbol{\}}\AgdaSpace{}%
\AgdaSymbol{→}\AgdaSpace{}%
\AgdaFunction{ker}\AgdaSpace{}%
\AgdaOperator{\AgdaFunction{∣}}\AgdaSpace{}%
\AgdaOperator{\AgdaFunction{hom𝔽[}}\AgdaSpace{}%
\AgdaBound{X}\AgdaSpace{}%
\AgdaOperator{\AgdaFunction{]}}\AgdaSpace{}%
\AgdaOperator{\AgdaFunction{∣}}\AgdaSpace{}%
\AgdaOperator{\AgdaFunction{⊆}}\AgdaSpace{}%
\AgdaFunction{Th}\AgdaSpace{}%
\AgdaSymbol{(}\AgdaFunction{V}\AgdaSpace{}%
\AgdaBound{ℓ}\AgdaSpace{}%
\AgdaFunction{ι}\AgdaSpace{}%
\AgdaBound{𝒦}\AgdaSymbol{)}\<%
\\
%
\>[1]\AgdaFunction{kernel-in-theory}\AgdaSpace{}%
\AgdaSymbol{\{}\AgdaArgument{X}\AgdaSpace{}%
\AgdaSymbol{=}\AgdaSpace{}%
\AgdaBound{X}\AgdaSymbol{\}}\AgdaSpace{}%
\AgdaSymbol{\{}\AgdaBound{p}\AgdaSpace{}%
\AgdaOperator{\AgdaInductiveConstructor{,}}\AgdaSpace{}%
\AgdaBound{q}\AgdaSymbol{\}}\AgdaSpace{}%
\AgdaBound{pKq}\AgdaSpace{}%
\AgdaBound{vkA}\AgdaSpace{}%
\AgdaBound{x}\AgdaSpace{}%
\AgdaSymbol{=}%
\>[6941I]\AgdaFunction{classIds-⊆-VIds}\AgdaSpace{}%
\AgdaSymbol{\{}\AgdaArgument{ℓ}\AgdaSpace{}%
\AgdaSymbol{=}\AgdaSpace{}%
\AgdaBound{ℓ}\AgdaSymbol{\}\{}\AgdaArgument{p}\AgdaSpace{}%
\AgdaSymbol{=}\AgdaSpace{}%
\AgdaBound{p}\AgdaSymbol{\}\{}\AgdaBound{q}\AgdaSymbol{\}}\<%
\\
\>[6941I][@{}l@{\AgdaIndent{0}}]%
\>[47]\AgdaSymbol{(}\AgdaFunction{class-models-kernel}\AgdaSpace{}%
\AgdaBound{pKq}\AgdaSymbol{)}\AgdaSpace{}%
\AgdaBound{vkA}\AgdaSpace{}%
\AgdaBound{x}\<%
\\
%
\\[\AgdaEmptyExtraSkip]%
%
\>[1]\AgdaKeyword{module}\AgdaSpace{}%
\AgdaModule{\AgdaUnderscore{}}\AgdaSpace{}%
\AgdaSymbol{\{}\AgdaBound{X}\AgdaSpace{}%
\AgdaSymbol{:}\AgdaSpace{}%
\AgdaPrimitive{Type}\AgdaSpace{}%
\AgdaBound{χ}\AgdaSymbol{\}}\AgdaSpace{}%
\AgdaSymbol{\{}\AgdaBound{𝑨}\AgdaSpace{}%
\AgdaSymbol{:}\AgdaSpace{}%
\AgdaRecord{Algebra}\AgdaSpace{}%
\AgdaBound{α}\AgdaSpace{}%
\AgdaBound{ρᵃ}\AgdaSymbol{\}\{}\AgdaBound{sA}\AgdaSpace{}%
\AgdaSymbol{:}\AgdaSpace{}%
\AgdaBound{𝑨}\AgdaSpace{}%
\AgdaOperator{\AgdaFunction{∈}}\AgdaSpace{}%
\AgdaFunction{S}\AgdaSpace{}%
\AgdaSymbol{\{}\AgdaArgument{β}\AgdaSpace{}%
\AgdaSymbol{=}\AgdaSpace{}%
\AgdaBound{α}\AgdaSymbol{\}\{}\AgdaBound{ρᵃ}\AgdaSymbol{\}}\AgdaSpace{}%
\AgdaBound{ℓ}\AgdaSpace{}%
\AgdaBound{𝒦}\AgdaSymbol{\}}\AgdaSpace{}%
\AgdaKeyword{where}\<%
\\
\>[1][@{}l@{\AgdaIndent{0}}]%
\>[2]\AgdaKeyword{open}\AgdaSpace{}%
\AgdaModule{Environment}\AgdaSpace{}%
\AgdaBound{𝑨}\AgdaSpace{}%
\AgdaKeyword{using}\AgdaSpace{}%
\AgdaSymbol{(}\AgdaSpace{}%
\AgdaFunction{Equal}\AgdaSpace{}%
\AgdaSymbol{)}\<%
\\
%
\>[2]\AgdaFunction{ker𝔽⊆Equal}\AgdaSpace{}%
\AgdaSymbol{:}\AgdaSpace{}%
\AgdaSymbol{∀\{}\AgdaBound{p}\AgdaSpace{}%
\AgdaBound{q}\AgdaSymbol{\}}\AgdaSpace{}%
\AgdaSymbol{→}\AgdaSpace{}%
\AgdaSymbol{(}\AgdaBound{p}\AgdaSpace{}%
\AgdaOperator{\AgdaInductiveConstructor{,}}\AgdaSpace{}%
\AgdaBound{q}\AgdaSymbol{)}\AgdaSpace{}%
\AgdaOperator{\AgdaFunction{∈}}\AgdaSpace{}%
\AgdaFunction{ker}\AgdaSpace{}%
\AgdaOperator{\AgdaFunction{∣}}\AgdaSpace{}%
\AgdaOperator{\AgdaFunction{hom𝔽[}}\AgdaSpace{}%
\AgdaBound{X}\AgdaSpace{}%
\AgdaOperator{\AgdaFunction{]}}\AgdaSpace{}%
\AgdaOperator{\AgdaFunction{∣}}\AgdaSpace{}%
\AgdaSymbol{→}\AgdaSpace{}%
\AgdaFunction{Equal}\AgdaSpace{}%
\AgdaBound{p}\AgdaSpace{}%
\AgdaBound{q}\<%
\\
%
\>[2]\AgdaFunction{ker𝔽⊆Equal}\AgdaSymbol{\{}\AgdaArgument{p}\AgdaSpace{}%
\AgdaSymbol{=}\AgdaSpace{}%
\AgdaBound{p}\AgdaSymbol{\}\{}\AgdaBound{q}\AgdaSymbol{\}}\AgdaSpace{}%
\AgdaBound{x}\AgdaSpace{}%
\AgdaSymbol{=}\AgdaSpace{}%
\AgdaFunction{S-id1}\AgdaSymbol{\{}\AgdaArgument{ℓ}\AgdaSpace{}%
\AgdaSymbol{=}\AgdaSpace{}%
\AgdaBound{ℓ}\AgdaSymbol{\}\{}\AgdaArgument{p}\AgdaSpace{}%
\AgdaSymbol{=}\AgdaSpace{}%
\AgdaBound{p}\AgdaSymbol{\}\{}\AgdaBound{q}\AgdaSymbol{\}}\AgdaSpace{}%
\AgdaSymbol{(}\AgdaOperator{\AgdaFunction{ℰ⊢[}}\AgdaSpace{}%
\AgdaBound{X}\AgdaSpace{}%
\AgdaOperator{\AgdaFunction{]▹Th𝒦}}\AgdaSpace{}%
\AgdaBound{x}\AgdaSymbol{)}\AgdaSpace{}%
\AgdaBound{𝑨}\AgdaSpace{}%
\AgdaBound{sA}\<%
\\
%
\\[\AgdaEmptyExtraSkip]%
%
\>[1]\AgdaFunction{𝒦⊫→ℰ⊢}\AgdaSpace{}%
\AgdaSymbol{:}\AgdaSpace{}%
\AgdaSymbol{\{}\AgdaBound{X}\AgdaSpace{}%
\AgdaSymbol{:}\AgdaSpace{}%
\AgdaPrimitive{Type}\AgdaSpace{}%
\AgdaBound{χ}\AgdaSymbol{\}}\AgdaSpace{}%
\AgdaSymbol{→}\AgdaSpace{}%
\AgdaSymbol{∀\{}\AgdaBound{p}\AgdaSpace{}%
\AgdaBound{q}\AgdaSymbol{\}}\AgdaSpace{}%
\AgdaSymbol{→}\AgdaSpace{}%
\AgdaBound{𝒦}\AgdaSpace{}%
\AgdaOperator{\AgdaFunction{⊫}}\AgdaSpace{}%
\AgdaSymbol{(}\AgdaBound{p}\AgdaSpace{}%
\AgdaOperator{\AgdaInductiveConstructor{≈̇}}\AgdaSpace{}%
\AgdaBound{q}\AgdaSymbol{)}\AgdaSpace{}%
\AgdaSymbol{→}\AgdaSpace{}%
\AgdaFunction{ℰ}\AgdaSpace{}%
\AgdaOperator{\AgdaDatatype{⊢}}\AgdaSpace{}%
\AgdaBound{X}\AgdaSpace{}%
\AgdaOperator{\AgdaDatatype{▹}}\AgdaSpace{}%
\AgdaBound{p}\AgdaSpace{}%
\AgdaOperator{\AgdaDatatype{≈}}\AgdaSpace{}%
\AgdaBound{q}\<%
\\
%
\>[1]\AgdaFunction{𝒦⊫→ℰ⊢}\AgdaSpace{}%
\AgdaSymbol{\{}\AgdaArgument{p}\AgdaSpace{}%
\AgdaSymbol{=}\AgdaSpace{}%
\AgdaBound{p}\AgdaSymbol{\}}\AgdaSpace{}%
\AgdaSymbol{\{}\AgdaBound{q}\AgdaSymbol{\}}\AgdaSpace{}%
\AgdaBound{pKq}\AgdaSpace{}%
\AgdaSymbol{=}\AgdaSpace{}%
\AgdaInductiveConstructor{hyp}\AgdaSpace{}%
\AgdaSymbol{((}\AgdaBound{p}\AgdaSpace{}%
\AgdaOperator{\AgdaInductiveConstructor{≈̇}}\AgdaSpace{}%
\AgdaBound{q}\AgdaSymbol{)}\AgdaSpace{}%
\AgdaOperator{\AgdaInductiveConstructor{,}}\AgdaSpace{}%
\AgdaBound{pKq}\AgdaSymbol{)}\AgdaSpace{}%
\AgdaKeyword{where}\AgdaSpace{}%
\AgdaKeyword{open}\AgdaSpace{}%
\AgdaOperator{\AgdaModule{\AgdaUnderscore{}⊢\AgdaUnderscore{}▹\AgdaUnderscore{}≈\AgdaUnderscore{}}}\AgdaSpace{}%
\AgdaKeyword{using}\AgdaSpace{}%
\AgdaSymbol{(}\AgdaInductiveConstructor{hyp}\AgdaSymbol{)}\<%
\end{code}

\hypertarget{the-universal-property}{%
\paragraph{The universal property}\label{the-universal-property}}

\begin{code}%
\>[0]\<%
\\
\>[0]\AgdaKeyword{module}\AgdaSpace{}%
\AgdaModule{\AgdaUnderscore{}}%
\>[10]\AgdaSymbol{\{}\AgdaBound{𝑨}\AgdaSpace{}%
\AgdaSymbol{:}\AgdaSpace{}%
\AgdaRecord{Algebra}\AgdaSpace{}%
\AgdaSymbol{(}\AgdaGeneralizable{α}\AgdaSpace{}%
\AgdaOperator{\AgdaPrimitive{⊔}}\AgdaSpace{}%
\AgdaGeneralizable{ρᵃ}\AgdaSpace{}%
\AgdaOperator{\AgdaPrimitive{⊔}}\AgdaSpace{}%
\AgdaGeneralizable{ℓ}\AgdaSymbol{)}\AgdaSpace{}%
\AgdaSymbol{(}\AgdaGeneralizable{α}\AgdaSpace{}%
\AgdaOperator{\AgdaPrimitive{⊔}}\AgdaSpace{}%
\AgdaGeneralizable{ρᵃ}\AgdaSpace{}%
\AgdaOperator{\AgdaPrimitive{⊔}}\AgdaSpace{}%
\AgdaGeneralizable{ℓ}\AgdaSymbol{)\}}\AgdaSpace{}%
\AgdaSymbol{\{}\AgdaBound{𝒦}\AgdaSpace{}%
\AgdaSymbol{:}\AgdaSpace{}%
\AgdaFunction{Pred}\AgdaSymbol{(}\AgdaRecord{Algebra}\AgdaSpace{}%
\AgdaGeneralizable{α}\AgdaSpace{}%
\AgdaGeneralizable{ρᵃ}\AgdaSymbol{)}\AgdaSpace{}%
\AgdaSymbol{(}\AgdaGeneralizable{α}\AgdaSpace{}%
\AgdaOperator{\AgdaPrimitive{⊔}}\AgdaSpace{}%
\AgdaGeneralizable{ρᵃ}\AgdaSpace{}%
\AgdaOperator{\AgdaPrimitive{⊔}}\AgdaSpace{}%
\AgdaFunction{ov}\AgdaSpace{}%
\AgdaGeneralizable{ℓ}\AgdaSymbol{)\}}\AgdaSpace{}%
\AgdaKeyword{where}\<%
\\
\>[0][@{}l@{\AgdaIndent{0}}]%
\>[1]\AgdaKeyword{private}\AgdaSpace{}%
\AgdaFunction{ι}\AgdaSpace{}%
\AgdaSymbol{=}\AgdaSpace{}%
\AgdaFunction{ov}\AgdaSymbol{(}\AgdaBound{α}\AgdaSpace{}%
\AgdaOperator{\AgdaPrimitive{⊔}}\AgdaSpace{}%
\AgdaBound{ρᵃ}\AgdaSpace{}%
\AgdaOperator{\AgdaPrimitive{⊔}}\AgdaSpace{}%
\AgdaBound{ℓ}\AgdaSymbol{)}\<%
\\
%
\\[\AgdaEmptyExtraSkip]%
%
\>[1]\AgdaKeyword{open}\AgdaSpace{}%
\AgdaModule{FreeHom}\AgdaSpace{}%
\AgdaSymbol{\{}\AgdaArgument{ℓ}\AgdaSpace{}%
\AgdaSymbol{=}\AgdaSpace{}%
\AgdaBound{ℓ}\AgdaSymbol{\}(}\AgdaBound{α}\AgdaSpace{}%
\AgdaOperator{\AgdaPrimitive{⊔}}\AgdaSpace{}%
\AgdaBound{ρᵃ}\AgdaSpace{}%
\AgdaOperator{\AgdaPrimitive{⊔}}\AgdaSpace{}%
\AgdaBound{ℓ}\AgdaSymbol{)}\AgdaSpace{}%
\AgdaSymbol{\{}\AgdaBound{𝒦}\AgdaSymbol{\}}\<%
\\
%
\>[1]\AgdaKeyword{open}\AgdaSpace{}%
\AgdaModule{FreeAlgebra}\AgdaSpace{}%
\AgdaSymbol{\{}\AgdaArgument{ι}\AgdaSpace{}%
\AgdaSymbol{=}\AgdaSpace{}%
\AgdaFunction{ι}\AgdaSymbol{\}\{}\AgdaArgument{I}\AgdaSpace{}%
\AgdaSymbol{=}\AgdaSpace{}%
\AgdaFunction{ℐ}\AgdaSymbol{\}}\AgdaSpace{}%
\AgdaFunction{ℰ}%
\>[36]\AgdaKeyword{using}\AgdaSpace{}%
\AgdaSymbol{(}\AgdaSpace{}%
\AgdaOperator{\AgdaFunction{𝔽[\AgdaUnderscore{}]}}\AgdaSpace{}%
\AgdaSymbol{)}\<%
\\
%
\>[1]\AgdaKeyword{open}\AgdaSpace{}%
\AgdaModule{Setoid}\AgdaSpace{}%
\AgdaOperator{\AgdaFunction{𝔻[}}\AgdaSpace{}%
\AgdaBound{𝑨}\AgdaSpace{}%
\AgdaOperator{\AgdaFunction{]}}%
\>[36]\AgdaKeyword{using}\AgdaSpace{}%
\AgdaSymbol{(}\AgdaSpace{}%
\AgdaFunction{trans}\AgdaSpace{}%
\AgdaSymbol{;}\AgdaSpace{}%
\AgdaFunction{sym}\AgdaSpace{}%
\AgdaSymbol{;}\AgdaSpace{}%
\AgdaFunction{refl}\AgdaSpace{}%
\AgdaSymbol{)}\AgdaSpace{}%
\AgdaKeyword{renaming}\AgdaSpace{}%
\AgdaSymbol{(}\AgdaSpace{}%
\AgdaField{Carrier}\AgdaSpace{}%
\AgdaSymbol{to}\AgdaSpace{}%
\AgdaField{A}\AgdaSpace{}%
\AgdaSymbol{)}\<%
\\
%
\\[\AgdaEmptyExtraSkip]%
%
\\[\AgdaEmptyExtraSkip]%
%
\>[1]\AgdaFunction{𝔽-ModTh-epi}\AgdaSpace{}%
\AgdaSymbol{:}\AgdaSpace{}%
\AgdaBound{𝑨}\AgdaSpace{}%
\AgdaOperator{\AgdaFunction{∈}}\AgdaSpace{}%
\AgdaFunction{Mod}\AgdaSpace{}%
\AgdaSymbol{(}\AgdaFunction{Th}\AgdaSpace{}%
\AgdaSymbol{(}\AgdaFunction{V}\AgdaSpace{}%
\AgdaBound{ℓ}\AgdaSpace{}%
\AgdaFunction{ι}\AgdaSpace{}%
\AgdaBound{𝒦}\AgdaSymbol{))}\<%
\\
\>[1][@{}l@{\AgdaIndent{0}}]%
\>[2]\AgdaSymbol{→}%
\>[15]\AgdaFunction{epi}\AgdaSpace{}%
\AgdaOperator{\AgdaFunction{𝔽[}}\AgdaSpace{}%
\AgdaFunction{A}\AgdaSpace{}%
\AgdaOperator{\AgdaFunction{]}}\AgdaSpace{}%
\AgdaBound{𝑨}\<%
\\
%
\>[1]\AgdaFunction{𝔽-ModTh-epi}\AgdaSpace{}%
\AgdaBound{A∈ModThK}\AgdaSpace{}%
\AgdaSymbol{=}\AgdaSpace{}%
\AgdaFunction{φ}\AgdaSpace{}%
\AgdaOperator{\AgdaInductiveConstructor{,}}\AgdaSpace{}%
\AgdaFunction{isEpi}\<%
\\
\>[1][@{}l@{\AgdaIndent{0}}]%
\>[2]\AgdaKeyword{where}\<%
\\
\>[2][@{}l@{\AgdaIndent{0}}]%
\>[3]\AgdaFunction{φ}\AgdaSpace{}%
\AgdaSymbol{:}\AgdaSpace{}%
\AgdaOperator{\AgdaFunction{𝔻[}}\AgdaSpace{}%
\AgdaOperator{\AgdaFunction{𝔽[}}\AgdaSpace{}%
\AgdaFunction{A}\AgdaSpace{}%
\AgdaOperator{\AgdaFunction{]}}\AgdaSpace{}%
\AgdaOperator{\AgdaFunction{]}}\AgdaSpace{}%
\AgdaOperator{\AgdaRecord{⟶}}\AgdaSpace{}%
\AgdaOperator{\AgdaFunction{𝔻[}}\AgdaSpace{}%
\AgdaBound{𝑨}\AgdaSpace{}%
\AgdaOperator{\AgdaFunction{]}}\<%
\\
%
\>[3]\AgdaOperator{\AgdaField{\AgdaUnderscore{}⟨\$⟩\AgdaUnderscore{}}}\AgdaSpace{}%
\AgdaFunction{φ}\AgdaSpace{}%
\AgdaSymbol{=}\AgdaSpace{}%
\AgdaFunction{free-lift}\AgdaSymbol{\{}\AgdaArgument{𝑨}\AgdaSpace{}%
\AgdaSymbol{=}\AgdaSpace{}%
\AgdaBound{𝑨}\AgdaSymbol{\}}\AgdaSpace{}%
\AgdaFunction{id}\<%
\\
%
\>[3]\AgdaField{cong}\AgdaSpace{}%
\AgdaFunction{φ}\AgdaSpace{}%
\AgdaSymbol{\{}\AgdaBound{p}\AgdaSymbol{\}}\AgdaSpace{}%
\AgdaSymbol{\{}\AgdaBound{q}\AgdaSymbol{\}}\AgdaSpace{}%
\AgdaBound{pq}%
\>[22]\AgdaSymbol{=}\AgdaSpace{}%
\AgdaFunction{trans}%
\>[31]\AgdaSymbol{(}\AgdaSpace{}%
\AgdaFunction{sym}\AgdaSpace{}%
\AgdaSymbol{(}\AgdaFunction{free-lift-interp}\AgdaSymbol{\{}\AgdaArgument{𝑨}\AgdaSpace{}%
\AgdaSymbol{=}\AgdaSpace{}%
\AgdaBound{𝑨}\AgdaSymbol{\}}\AgdaSpace{}%
\AgdaFunction{id}\AgdaSpace{}%
\AgdaBound{p}\AgdaSymbol{)}\AgdaSpace{}%
\AgdaSymbol{)}\<%
\\
%
\>[22]\AgdaSymbol{(}\AgdaSpace{}%
\AgdaFunction{trans}%
\>[31]\AgdaSymbol{(}\AgdaSpace{}%
\AgdaBound{A∈ModThK}\AgdaSymbol{\{}\AgdaArgument{p}\AgdaSpace{}%
\AgdaSymbol{=}\AgdaSpace{}%
\AgdaBound{p}\AgdaSymbol{\}\{}\AgdaBound{q}\AgdaSymbol{\}}\AgdaSpace{}%
\AgdaSymbol{(}\AgdaFunction{kernel-in-theory}\AgdaSpace{}%
\AgdaBound{pq}\AgdaSymbol{)}\AgdaSpace{}%
\AgdaFunction{id}\AgdaSpace{}%
\AgdaSymbol{)}\<%
\\
%
\>[31]\AgdaSymbol{(}\AgdaSpace{}%
\AgdaFunction{free-lift-interp}\AgdaSymbol{\{}\AgdaArgument{𝑨}\AgdaSpace{}%
\AgdaSymbol{=}\AgdaSpace{}%
\AgdaBound{𝑨}\AgdaSymbol{\}}\AgdaSpace{}%
\AgdaFunction{id}\AgdaSpace{}%
\AgdaBound{q}\AgdaSpace{}%
\AgdaSymbol{)}\AgdaSpace{}%
\AgdaSymbol{)}\<%
\\
%
\>[3]\AgdaFunction{isEpi}\AgdaSpace{}%
\AgdaSymbol{:}\AgdaSpace{}%
\AgdaRecord{IsEpi}\AgdaSpace{}%
\AgdaOperator{\AgdaFunction{𝔽[}}\AgdaSpace{}%
\AgdaFunction{A}\AgdaSpace{}%
\AgdaOperator{\AgdaFunction{]}}\AgdaSpace{}%
\AgdaBound{𝑨}\AgdaSpace{}%
\AgdaFunction{φ}\<%
\\
%
\>[3]\AgdaField{compatible}\AgdaSpace{}%
\AgdaSymbol{(}\AgdaField{isHom}\AgdaSpace{}%
\AgdaFunction{isEpi}\AgdaSymbol{)}\AgdaSpace{}%
\AgdaSymbol{=}\AgdaSpace{}%
\AgdaField{cong}\AgdaSpace{}%
\AgdaSymbol{(}\AgdaField{Interp}\AgdaSpace{}%
\AgdaBound{𝑨}\AgdaSymbol{)}\AgdaSpace{}%
\AgdaSymbol{(}\AgdaInductiveConstructor{≡.refl}\AgdaSpace{}%
\AgdaOperator{\AgdaInductiveConstructor{,}}\AgdaSpace{}%
\AgdaSymbol{(λ}\AgdaSpace{}%
\AgdaBound{\AgdaUnderscore{}}\AgdaSpace{}%
\AgdaSymbol{→}\AgdaSpace{}%
\AgdaFunction{refl}\AgdaSymbol{))}\<%
\\
%
\>[3]\AgdaField{isSurjective}\AgdaSpace{}%
\AgdaFunction{isEpi}\AgdaSpace{}%
\AgdaSymbol{\{}\AgdaBound{y}\AgdaSymbol{\}}\AgdaSpace{}%
\AgdaSymbol{=}\AgdaSpace{}%
\AgdaInductiveConstructor{eq}\AgdaSpace{}%
\AgdaSymbol{(}\AgdaInductiveConstructor{ℊ}\AgdaSpace{}%
\AgdaBound{y}\AgdaSymbol{)}\AgdaSpace{}%
\AgdaFunction{refl}\<%
\\
%
\\[\AgdaEmptyExtraSkip]%
%
\>[1]\AgdaFunction{𝔽-ModTh-epi-lift}\AgdaSpace{}%
\AgdaSymbol{:}\AgdaSpace{}%
\AgdaBound{𝑨}\AgdaSpace{}%
\AgdaOperator{\AgdaFunction{∈}}\AgdaSpace{}%
\AgdaFunction{Mod}\AgdaSpace{}%
\AgdaSymbol{(}\AgdaFunction{Th}\AgdaSpace{}%
\AgdaSymbol{(}\AgdaFunction{V}\AgdaSpace{}%
\AgdaBound{ℓ}\AgdaSpace{}%
\AgdaFunction{ι}\AgdaSpace{}%
\AgdaBound{𝒦}\AgdaSymbol{))}\AgdaSpace{}%
\AgdaSymbol{→}\AgdaSpace{}%
\AgdaFunction{epi}\AgdaSpace{}%
\AgdaOperator{\AgdaFunction{𝔽[}}\AgdaSpace{}%
\AgdaFunction{A}\AgdaSpace{}%
\AgdaOperator{\AgdaFunction{]}}\AgdaSpace{}%
\AgdaSymbol{(}\AgdaFunction{Lift-Alg}\AgdaSpace{}%
\AgdaBound{𝑨}\AgdaSpace{}%
\AgdaFunction{ι}\AgdaSpace{}%
\AgdaFunction{ι}\AgdaSymbol{)}\<%
\\
%
\>[1]\AgdaFunction{𝔽-ModTh-epi-lift}\AgdaSpace{}%
\AgdaBound{A∈ModThK}\AgdaSpace{}%
\AgdaSymbol{=}\AgdaSpace{}%
\AgdaFunction{∘-epi}\AgdaSpace{}%
\AgdaSymbol{(}\AgdaFunction{𝔽-ModTh-epi}\AgdaSpace{}%
\AgdaSymbol{(λ}\AgdaSpace{}%
\AgdaSymbol{\{}\AgdaBound{p}\AgdaSpace{}%
\AgdaBound{q}\AgdaSymbol{\}}\AgdaSpace{}%
\AgdaSymbol{→}\AgdaSpace{}%
\AgdaBound{A∈ModThK}\AgdaSymbol{\{}\AgdaArgument{p}\AgdaSpace{}%
\AgdaSymbol{=}\AgdaSpace{}%
\AgdaBound{p}\AgdaSymbol{\}\{}\AgdaBound{q}\AgdaSymbol{\}))}\AgdaSpace{}%
\AgdaFunction{ToLift-epi}\<%
\end{code}

\hypertarget{products-of-classes-of-algebras}{%
\subsection{Products of classes of
algebras}\label{products-of-classes-of-algebras}}

We want to pair each \texttt{(𝑨\ ,\ p)} (where p : 𝑨 ∈ S 𝒦) with an
environment \texttt{ρ\ :\ X\ →\ ∣\ 𝑨\ ∣} so that we can quantify over
all algebras \emph{and} all assignments of values in the domain
\texttt{∣\ 𝑨\ ∣} to variables in \texttt{X}.

\begin{code}%
\>[0]\<%
\\
\>[0]\AgdaKeyword{module}\AgdaSpace{}%
\AgdaModule{\AgdaUnderscore{}}\AgdaSpace{}%
\AgdaSymbol{(}\AgdaBound{𝒦}\AgdaSpace{}%
\AgdaSymbol{:}\AgdaSpace{}%
\AgdaFunction{Pred}\AgdaSymbol{(}\AgdaRecord{Algebra}\AgdaSpace{}%
\AgdaGeneralizable{α}\AgdaSpace{}%
\AgdaGeneralizable{ρᵃ}\AgdaSymbol{)}\AgdaSpace{}%
\AgdaSymbol{(}\AgdaGeneralizable{α}\AgdaSpace{}%
\AgdaOperator{\AgdaPrimitive{⊔}}\AgdaSpace{}%
\AgdaGeneralizable{ρᵃ}\AgdaSpace{}%
\AgdaOperator{\AgdaPrimitive{⊔}}\AgdaSpace{}%
\AgdaFunction{ov}\AgdaSpace{}%
\AgdaGeneralizable{ℓ}\AgdaSymbol{))\{}\AgdaBound{X}\AgdaSpace{}%
\AgdaSymbol{:}\AgdaSpace{}%
\AgdaPrimitive{Type}\AgdaSpace{}%
\AgdaSymbol{(}\AgdaGeneralizable{α}\AgdaSpace{}%
\AgdaOperator{\AgdaPrimitive{⊔}}\AgdaSpace{}%
\AgdaGeneralizable{ρᵃ}\AgdaSpace{}%
\AgdaOperator{\AgdaPrimitive{⊔}}\AgdaSpace{}%
\AgdaGeneralizable{ℓ}\AgdaSymbol{)\}}\AgdaSpace{}%
\AgdaKeyword{where}\<%
\\
\>[0][@{}l@{\AgdaIndent{0}}]%
\>[1]\AgdaKeyword{private}\AgdaSpace{}%
\AgdaFunction{ι}\AgdaSpace{}%
\AgdaSymbol{=}\AgdaSpace{}%
\AgdaFunction{ov}\AgdaSymbol{(}\AgdaBound{α}\AgdaSpace{}%
\AgdaOperator{\AgdaPrimitive{⊔}}\AgdaSpace{}%
\AgdaBound{ρᵃ}\AgdaSpace{}%
\AgdaOperator{\AgdaPrimitive{⊔}}\AgdaSpace{}%
\AgdaBound{ℓ}\AgdaSymbol{)}\<%
\\
%
\>[1]\AgdaKeyword{open}\AgdaSpace{}%
\AgdaModule{FreeHom}\AgdaSpace{}%
\AgdaSymbol{\{}\AgdaArgument{ℓ}\AgdaSpace{}%
\AgdaSymbol{=}\AgdaSpace{}%
\AgdaBound{ℓ}\AgdaSymbol{\}}\AgdaSpace{}%
\AgdaSymbol{(}\AgdaBound{α}\AgdaSpace{}%
\AgdaOperator{\AgdaPrimitive{⊔}}\AgdaSpace{}%
\AgdaBound{ρᵃ}\AgdaSpace{}%
\AgdaOperator{\AgdaPrimitive{⊔}}\AgdaSpace{}%
\AgdaBound{ℓ}\AgdaSymbol{)\{}\AgdaBound{𝒦}\AgdaSymbol{\}}\<%
\\
%
\>[1]\AgdaKeyword{open}\AgdaSpace{}%
\AgdaModule{FreeAlgebra}\AgdaSpace{}%
\AgdaSymbol{\{}\AgdaArgument{ι}\AgdaSpace{}%
\AgdaSymbol{=}\AgdaSpace{}%
\AgdaFunction{ι}\AgdaSymbol{\}\{}\AgdaArgument{I}\AgdaSpace{}%
\AgdaSymbol{=}\AgdaSpace{}%
\AgdaFunction{ℐ}\AgdaSymbol{\}}\AgdaSpace{}%
\AgdaFunction{ℰ}\AgdaSpace{}%
\AgdaKeyword{using}\AgdaSpace{}%
\AgdaSymbol{(}\AgdaSpace{}%
\AgdaOperator{\AgdaFunction{𝔽[\AgdaUnderscore{}]}}\AgdaSpace{}%
\AgdaSymbol{)}\<%
\\
%
\>[1]\AgdaKeyword{open}\AgdaSpace{}%
\AgdaModule{Environment}%
\>[20]\AgdaKeyword{using}\AgdaSpace{}%
\AgdaSymbol{(}\AgdaSpace{}%
\AgdaFunction{Env}\AgdaSpace{}%
\AgdaSymbol{)}\<%
\\
%
\\[\AgdaEmptyExtraSkip]%
%
\>[1]\AgdaFunction{ℑ⁺}\AgdaSpace{}%
\AgdaSymbol{:}\AgdaSpace{}%
\AgdaPrimitive{Type}\AgdaSpace{}%
\AgdaFunction{ι}\<%
\\
%
\>[1]\AgdaFunction{ℑ⁺}\AgdaSpace{}%
\AgdaSymbol{=}\AgdaSpace{}%
\AgdaFunction{Σ[}\AgdaSpace{}%
\AgdaBound{𝑨}\AgdaSpace{}%
\AgdaFunction{∈}\AgdaSpace{}%
\AgdaSymbol{(}\AgdaRecord{Algebra}\AgdaSpace{}%
\AgdaBound{α}\AgdaSpace{}%
\AgdaBound{ρᵃ}\AgdaSymbol{)}\AgdaSpace{}%
\AgdaFunction{]}\AgdaSpace{}%
\AgdaSymbol{(}\AgdaBound{𝑨}\AgdaSpace{}%
\AgdaOperator{\AgdaFunction{∈}}\AgdaSpace{}%
\AgdaFunction{S}\AgdaSpace{}%
\AgdaBound{ℓ}\AgdaSpace{}%
\AgdaBound{𝒦}\AgdaSymbol{)}\AgdaSpace{}%
\AgdaOperator{\AgdaFunction{×}}\AgdaSpace{}%
\AgdaSymbol{(}\AgdaField{Carrier}\AgdaSpace{}%
\AgdaSymbol{(}\AgdaFunction{Env}\AgdaSpace{}%
\AgdaBound{𝑨}\AgdaSpace{}%
\AgdaBound{X}\AgdaSymbol{))}\<%
\\
%
\\[\AgdaEmptyExtraSkip]%
%
\>[1]\AgdaFunction{𝔄⁺}\AgdaSpace{}%
\AgdaSymbol{:}\AgdaSpace{}%
\AgdaFunction{ℑ⁺}\AgdaSpace{}%
\AgdaSymbol{→}\AgdaSpace{}%
\AgdaRecord{Algebra}\AgdaSpace{}%
\AgdaBound{α}\AgdaSpace{}%
\AgdaBound{ρᵃ}\<%
\\
%
\>[1]\AgdaFunction{𝔄⁺}\AgdaSpace{}%
\AgdaBound{i}\AgdaSpace{}%
\AgdaSymbol{=}\AgdaSpace{}%
\AgdaOperator{\AgdaFunction{∣}}\AgdaSpace{}%
\AgdaBound{i}\AgdaSpace{}%
\AgdaOperator{\AgdaFunction{∣}}\<%
\\
%
\\[\AgdaEmptyExtraSkip]%
%
\>[1]\AgdaFunction{ℭ}\AgdaSpace{}%
\AgdaSymbol{:}\AgdaSpace{}%
\AgdaRecord{Algebra}\AgdaSpace{}%
\AgdaFunction{ι}\AgdaSpace{}%
\AgdaFunction{ι}\<%
\\
%
\>[1]\AgdaFunction{ℭ}\AgdaSpace{}%
\AgdaSymbol{=}\AgdaSpace{}%
\AgdaFunction{⨅}\AgdaSpace{}%
\AgdaFunction{𝔄⁺}\<%
\\
\>[0]\<%
\end{code}

Next we define a useful type, \texttt{skEqual}, which we use to
represent a term identity \texttt{p\ ≈\ q} for any given
\texttt{i\ =\ (𝑨\ ,\ sA\ ,\ ρ)} (where \texttt{𝑨} is an algebra,
\texttt{sA\ :\ 𝑨\ ∈\ S\ 𝒦} is a proof that \texttt{𝑨} belongs to
\texttt{S\ 𝒦}, and \texttt{ρ} is a mapping from \texttt{X} to the domain
of \texttt{𝑨}). Then we prove \texttt{AllEqual⊆ker𝔽} which asserts that
if the identity \texttt{p\ ≈\ q} holds in all \texttt{𝑨\ ∈\ S\ 𝒦} (for
all environments), then \texttt{p\ ≈\ q} holds in the relatively free
algebra \texttt{𝔽{[}\ X\ {]}}; equivalently, the pair \texttt{(p\ ,\ q)}
belongs to the kernel of the natural homomorphism from \texttt{𝑻\ X}
onto \texttt{𝔽{[}\ X\ {]}}. We will use this fact below to prove that
there is a monomorphism from \texttt{𝔽{[}\ X\ {]}} into \texttt{ℭ}, and
thus \texttt{𝔽{[}\ X\ {]}} is a subalgebra of ℭ, so belongs to
\texttt{S\ (P\ 𝒦)}.

\begin{code}%
\>[0]\<%
\\
\>[0][@{}l@{\AgdaIndent{1}}]%
\>[1]\AgdaFunction{skEqual}\AgdaSpace{}%
\AgdaSymbol{:}\AgdaSpace{}%
\AgdaSymbol{(}\AgdaBound{i}\AgdaSpace{}%
\AgdaSymbol{:}\AgdaSpace{}%
\AgdaFunction{ℑ⁺}\AgdaSymbol{)}\AgdaSpace{}%
\AgdaSymbol{→}\AgdaSpace{}%
\AgdaSymbol{∀\{}\AgdaBound{p}\AgdaSpace{}%
\AgdaBound{q}\AgdaSymbol{\}}\AgdaSpace{}%
\AgdaSymbol{→}\AgdaSpace{}%
\AgdaPrimitive{Type}\AgdaSpace{}%
\AgdaBound{ρᵃ}\<%
\\
%
\>[1]\AgdaFunction{skEqual}\AgdaSpace{}%
\AgdaBound{i}\AgdaSpace{}%
\AgdaSymbol{\{}\AgdaBound{p}\AgdaSymbol{\}\{}\AgdaBound{q}\AgdaSymbol{\}}\AgdaSpace{}%
\AgdaSymbol{=}\AgdaSpace{}%
\AgdaOperator{\AgdaFunction{⟦}}\AgdaSpace{}%
\AgdaBound{p}\AgdaSpace{}%
\AgdaOperator{\AgdaFunction{⟧}}\AgdaSpace{}%
\AgdaOperator{\AgdaField{⟨\$⟩}}\AgdaSpace{}%
\AgdaField{snd}\AgdaSpace{}%
\AgdaOperator{\AgdaFunction{∥}}\AgdaSpace{}%
\AgdaBound{i}\AgdaSpace{}%
\AgdaOperator{\AgdaFunction{∥}}\AgdaSpace{}%
\AgdaOperator{\AgdaFunction{≈}}\AgdaSpace{}%
\AgdaOperator{\AgdaFunction{⟦}}\AgdaSpace{}%
\AgdaBound{q}\AgdaSpace{}%
\AgdaOperator{\AgdaFunction{⟧}}\AgdaSpace{}%
\AgdaOperator{\AgdaField{⟨\$⟩}}\AgdaSpace{}%
\AgdaField{snd}\AgdaSpace{}%
\AgdaOperator{\AgdaFunction{∥}}\AgdaSpace{}%
\AgdaBound{i}\AgdaSpace{}%
\AgdaOperator{\AgdaFunction{∥}}\<%
\\
\>[1][@{}l@{\AgdaIndent{0}}]%
\>[2]\AgdaKeyword{where}\<%
\\
%
\>[2]\AgdaKeyword{open}\AgdaSpace{}%
\AgdaModule{Setoid}\AgdaSpace{}%
\AgdaOperator{\AgdaFunction{𝔻[}}\AgdaSpace{}%
\AgdaFunction{𝔄⁺}\AgdaSpace{}%
\AgdaBound{i}\AgdaSpace{}%
\AgdaOperator{\AgdaFunction{]}}\AgdaSpace{}%
\AgdaKeyword{using}\AgdaSpace{}%
\AgdaSymbol{(}\AgdaSpace{}%
\AgdaOperator{\AgdaField{\AgdaUnderscore{}≈\AgdaUnderscore{}}}\AgdaSpace{}%
\AgdaSymbol{)}\<%
\\
%
\>[2]\AgdaKeyword{open}\AgdaSpace{}%
\AgdaModule{Environment}\AgdaSpace{}%
\AgdaSymbol{(}\AgdaFunction{𝔄⁺}\AgdaSpace{}%
\AgdaBound{i}\AgdaSymbol{)}\AgdaSpace{}%
\AgdaKeyword{using}\AgdaSpace{}%
\AgdaSymbol{(}\AgdaSpace{}%
\AgdaOperator{\AgdaFunction{⟦\AgdaUnderscore{}⟧}}\AgdaSpace{}%
\AgdaSymbol{)}\<%
\\
%
\\[\AgdaEmptyExtraSkip]%
%
\>[1]\AgdaFunction{AllEqual⊆ker𝔽}\AgdaSpace{}%
\AgdaSymbol{:}\AgdaSpace{}%
\AgdaSymbol{∀}\AgdaSpace{}%
\AgdaSymbol{\{}\AgdaBound{p}\AgdaSpace{}%
\AgdaBound{q}\AgdaSymbol{\}}\AgdaSpace{}%
\AgdaSymbol{→}\AgdaSpace{}%
\AgdaSymbol{(∀}\AgdaSpace{}%
\AgdaBound{i}\AgdaSpace{}%
\AgdaSymbol{→}\AgdaSpace{}%
\AgdaFunction{skEqual}\AgdaSpace{}%
\AgdaBound{i}\AgdaSpace{}%
\AgdaSymbol{\{}\AgdaBound{p}\AgdaSymbol{\}\{}\AgdaBound{q}\AgdaSymbol{\})}\AgdaSpace{}%
\AgdaSymbol{→}\AgdaSpace{}%
\AgdaSymbol{(}\AgdaBound{p}\AgdaSpace{}%
\AgdaOperator{\AgdaInductiveConstructor{,}}\AgdaSpace{}%
\AgdaBound{q}\AgdaSymbol{)}\AgdaSpace{}%
\AgdaOperator{\AgdaFunction{∈}}\AgdaSpace{}%
\AgdaFunction{ker}\AgdaSpace{}%
\AgdaOperator{\AgdaFunction{∣}}\AgdaSpace{}%
\AgdaOperator{\AgdaFunction{hom𝔽[}}\AgdaSpace{}%
\AgdaBound{X}\AgdaSpace{}%
\AgdaOperator{\AgdaFunction{]}}\AgdaSpace{}%
\AgdaOperator{\AgdaFunction{∣}}\<%
\\
%
\>[1]\AgdaFunction{AllEqual⊆ker𝔽}\AgdaSpace{}%
\AgdaSymbol{\{}\AgdaBound{p}\AgdaSymbol{\}}\AgdaSpace{}%
\AgdaSymbol{\{}\AgdaBound{q}\AgdaSymbol{\}}\AgdaSpace{}%
\AgdaBound{x}\AgdaSpace{}%
\AgdaSymbol{=}\AgdaSpace{}%
\AgdaFunction{Goal}\<%
\\
\>[1][@{}l@{\AgdaIndent{0}}]%
\>[2]\AgdaKeyword{where}\<%
\\
%
\>[2]\AgdaKeyword{open}\AgdaSpace{}%
\AgdaModule{Setoid}\AgdaSpace{}%
\AgdaOperator{\AgdaFunction{𝔻[}}\AgdaSpace{}%
\AgdaOperator{\AgdaFunction{𝔽[}}\AgdaSpace{}%
\AgdaBound{X}\AgdaSpace{}%
\AgdaOperator{\AgdaFunction{]}}\AgdaSpace{}%
\AgdaOperator{\AgdaFunction{]}}\AgdaSpace{}%
\AgdaKeyword{using}\AgdaSpace{}%
\AgdaSymbol{(}\AgdaSpace{}%
\AgdaOperator{\AgdaField{\AgdaUnderscore{}≈\AgdaUnderscore{}}}\AgdaSpace{}%
\AgdaSymbol{)}\<%
\\
%
\>[2]\AgdaFunction{S𝒦⊫pq}\AgdaSpace{}%
\AgdaSymbol{:}\AgdaSpace{}%
\AgdaFunction{S}\AgdaSymbol{\{}\AgdaArgument{β}\AgdaSpace{}%
\AgdaSymbol{=}\AgdaSpace{}%
\AgdaBound{α}\AgdaSymbol{\}\{}\AgdaBound{ρᵃ}\AgdaSymbol{\}}\AgdaSpace{}%
\AgdaBound{ℓ}\AgdaSpace{}%
\AgdaBound{𝒦}\AgdaSpace{}%
\AgdaOperator{\AgdaFunction{⊫}}\AgdaSpace{}%
\AgdaSymbol{(}\AgdaBound{p}\AgdaSpace{}%
\AgdaOperator{\AgdaInductiveConstructor{≈̇}}\AgdaSpace{}%
\AgdaBound{q}\AgdaSymbol{)}\<%
\\
%
\>[2]\AgdaFunction{S𝒦⊫pq}\AgdaSpace{}%
\AgdaBound{𝑨}\AgdaSpace{}%
\AgdaBound{sA}\AgdaSpace{}%
\AgdaBound{ρ}\AgdaSpace{}%
\AgdaSymbol{=}\AgdaSpace{}%
\AgdaBound{x}\AgdaSpace{}%
\AgdaSymbol{(}\AgdaBound{𝑨}\AgdaSpace{}%
\AgdaOperator{\AgdaInductiveConstructor{,}}\AgdaSpace{}%
\AgdaBound{sA}\AgdaSpace{}%
\AgdaOperator{\AgdaInductiveConstructor{,}}\AgdaSpace{}%
\AgdaBound{ρ}\AgdaSymbol{)}\<%
\\
%
\>[2]\AgdaFunction{Goal}\AgdaSpace{}%
\AgdaSymbol{:}\AgdaSpace{}%
\AgdaBound{p}\AgdaSpace{}%
\AgdaOperator{\AgdaFunction{≈}}\AgdaSpace{}%
\AgdaBound{q}\<%
\\
%
\>[2]\AgdaFunction{Goal}\AgdaSpace{}%
\AgdaSymbol{=}\AgdaSpace{}%
\AgdaFunction{𝒦⊫→ℰ⊢}\AgdaSpace{}%
\AgdaSymbol{(}\AgdaFunction{S-id2}\AgdaSymbol{\{}\AgdaArgument{ℓ}\AgdaSpace{}%
\AgdaSymbol{=}\AgdaSpace{}%
\AgdaBound{ℓ}\AgdaSymbol{\}\{}\AgdaArgument{p}\AgdaSpace{}%
\AgdaSymbol{=}\AgdaSpace{}%
\AgdaBound{p}\AgdaSymbol{\}\{}\AgdaBound{q}\AgdaSymbol{\}}\AgdaSpace{}%
\AgdaFunction{S𝒦⊫pq}\AgdaSymbol{)}\<%
\\
%
\\[\AgdaEmptyExtraSkip]%
%
\>[1]\AgdaFunction{homℭ}\AgdaSpace{}%
\AgdaSymbol{:}\AgdaSpace{}%
\AgdaFunction{hom}\AgdaSpace{}%
\AgdaSymbol{(}\AgdaFunction{𝑻}\AgdaSpace{}%
\AgdaBound{X}\AgdaSymbol{)}\AgdaSpace{}%
\AgdaFunction{ℭ}\<%
\\
%
\>[1]\AgdaFunction{homℭ}\AgdaSpace{}%
\AgdaSymbol{=}\AgdaSpace{}%
\AgdaFunction{⨅-hom-co}\AgdaSpace{}%
\AgdaFunction{𝔄⁺}\AgdaSpace{}%
\AgdaFunction{h}\<%
\\
\>[1][@{}l@{\AgdaIndent{0}}]%
\>[2]\AgdaKeyword{where}\<%
\\
%
\>[2]\AgdaFunction{h}\AgdaSpace{}%
\AgdaSymbol{:}\AgdaSpace{}%
\AgdaSymbol{∀}\AgdaSpace{}%
\AgdaBound{i}\AgdaSpace{}%
\AgdaSymbol{→}\AgdaSpace{}%
\AgdaFunction{hom}\AgdaSpace{}%
\AgdaSymbol{(}\AgdaFunction{𝑻}\AgdaSpace{}%
\AgdaBound{X}\AgdaSymbol{)}\AgdaSpace{}%
\AgdaSymbol{(}\AgdaFunction{𝔄⁺}\AgdaSpace{}%
\AgdaBound{i}\AgdaSymbol{)}\<%
\\
%
\>[2]\AgdaFunction{h}\AgdaSpace{}%
\AgdaBound{i}\AgdaSpace{}%
\AgdaSymbol{=}\AgdaSpace{}%
\AgdaFunction{lift-hom}\AgdaSpace{}%
\AgdaSymbol{(}\AgdaField{snd}\AgdaSpace{}%
\AgdaOperator{\AgdaFunction{∥}}\AgdaSpace{}%
\AgdaBound{i}\AgdaSpace{}%
\AgdaOperator{\AgdaFunction{∥}}\AgdaSymbol{)}\<%
\\
%
\\[\AgdaEmptyExtraSkip]%
%
\>[1]\AgdaFunction{ker𝔽⊆kerℭ}\AgdaSpace{}%
\AgdaSymbol{:}\AgdaSpace{}%
\AgdaFunction{ker}\AgdaSpace{}%
\AgdaOperator{\AgdaFunction{∣}}\AgdaSpace{}%
\AgdaOperator{\AgdaFunction{hom𝔽[}}\AgdaSpace{}%
\AgdaBound{X}\AgdaSpace{}%
\AgdaOperator{\AgdaFunction{]}}\AgdaSpace{}%
\AgdaOperator{\AgdaFunction{∣}}\AgdaSpace{}%
\AgdaOperator{\AgdaFunction{⊆}}\AgdaSpace{}%
\AgdaFunction{ker}\AgdaSpace{}%
\AgdaOperator{\AgdaFunction{∣}}\AgdaSpace{}%
\AgdaFunction{homℭ}\AgdaSpace{}%
\AgdaOperator{\AgdaFunction{∣}}\<%
\\
%
\>[1]\AgdaFunction{ker𝔽⊆kerℭ}\AgdaSpace{}%
\AgdaSymbol{\{}\AgdaBound{p}\AgdaSpace{}%
\AgdaOperator{\AgdaInductiveConstructor{,}}\AgdaSpace{}%
\AgdaBound{q}\AgdaSymbol{\}}\AgdaSpace{}%
\AgdaBound{pKq}\AgdaSpace{}%
\AgdaSymbol{(}\AgdaBound{𝑨}\AgdaSpace{}%
\AgdaOperator{\AgdaInductiveConstructor{,}}\AgdaSpace{}%
\AgdaBound{sA}\AgdaSpace{}%
\AgdaOperator{\AgdaInductiveConstructor{,}}\AgdaSpace{}%
\AgdaBound{ρ}\AgdaSymbol{)}\AgdaSpace{}%
\AgdaSymbol{=}\AgdaSpace{}%
\AgdaFunction{Goal}\<%
\\
\>[1][@{}l@{\AgdaIndent{0}}]%
\>[2]\AgdaKeyword{where}\<%
\\
%
\>[2]\AgdaKeyword{open}\AgdaSpace{}%
\AgdaModule{Setoid}\AgdaSpace{}%
\AgdaOperator{\AgdaFunction{𝔻[}}\AgdaSpace{}%
\AgdaBound{𝑨}\AgdaSpace{}%
\AgdaOperator{\AgdaFunction{]}}\AgdaSpace{}%
\AgdaKeyword{using}\AgdaSpace{}%
\AgdaSymbol{(}\AgdaSpace{}%
\AgdaOperator{\AgdaField{\AgdaUnderscore{}≈\AgdaUnderscore{}}}\AgdaSpace{}%
\AgdaSymbol{;}\AgdaSpace{}%
\AgdaFunction{sym}\AgdaSpace{}%
\AgdaSymbol{;}\AgdaSpace{}%
\AgdaFunction{trans}\AgdaSpace{}%
\AgdaSymbol{)}\<%
\\
%
\>[2]\AgdaKeyword{open}\AgdaSpace{}%
\AgdaModule{Environment}\AgdaSpace{}%
\AgdaBound{𝑨}\AgdaSpace{}%
\AgdaKeyword{using}\AgdaSpace{}%
\AgdaSymbol{(}\AgdaSpace{}%
\AgdaOperator{\AgdaFunction{⟦\AgdaUnderscore{}⟧}}\AgdaSpace{}%
\AgdaSymbol{)}\<%
\\
%
\>[2]\AgdaFunction{fl}\AgdaSpace{}%
\AgdaSymbol{:}\AgdaSpace{}%
\AgdaSymbol{∀}\AgdaSpace{}%
\AgdaBound{t}\AgdaSpace{}%
\AgdaSymbol{→}\AgdaSpace{}%
\AgdaOperator{\AgdaFunction{⟦}}\AgdaSpace{}%
\AgdaBound{t}\AgdaSpace{}%
\AgdaOperator{\AgdaFunction{⟧}}\AgdaSpace{}%
\AgdaOperator{\AgdaField{⟨\$⟩}}\AgdaSpace{}%
\AgdaBound{ρ}\AgdaSpace{}%
\AgdaOperator{\AgdaFunction{≈}}\AgdaSpace{}%
\AgdaFunction{free-lift}\AgdaSpace{}%
\AgdaBound{ρ}\AgdaSpace{}%
\AgdaBound{t}\<%
\\
%
\>[2]\AgdaFunction{fl}\AgdaSpace{}%
\AgdaBound{t}\AgdaSpace{}%
\AgdaSymbol{=}\AgdaSpace{}%
\AgdaFunction{free-lift-interp}\AgdaSpace{}%
\AgdaSymbol{\{}\AgdaArgument{𝑨}\AgdaSpace{}%
\AgdaSymbol{=}\AgdaSpace{}%
\AgdaBound{𝑨}\AgdaSymbol{\}}\AgdaSpace{}%
\AgdaBound{ρ}\AgdaSpace{}%
\AgdaBound{t}\<%
\\
%
\>[2]\AgdaFunction{subgoal}\AgdaSpace{}%
\AgdaSymbol{:}\AgdaSpace{}%
\AgdaOperator{\AgdaFunction{⟦}}\AgdaSpace{}%
\AgdaBound{p}\AgdaSpace{}%
\AgdaOperator{\AgdaFunction{⟧}}\AgdaSpace{}%
\AgdaOperator{\AgdaField{⟨\$⟩}}\AgdaSpace{}%
\AgdaBound{ρ}\AgdaSpace{}%
\AgdaOperator{\AgdaFunction{≈}}\AgdaSpace{}%
\AgdaOperator{\AgdaFunction{⟦}}\AgdaSpace{}%
\AgdaBound{q}\AgdaSpace{}%
\AgdaOperator{\AgdaFunction{⟧}}\AgdaSpace{}%
\AgdaOperator{\AgdaField{⟨\$⟩}}\AgdaSpace{}%
\AgdaBound{ρ}\<%
\\
%
\>[2]\AgdaFunction{subgoal}\AgdaSpace{}%
\AgdaSymbol{=}\AgdaSpace{}%
\AgdaFunction{ker𝔽⊆Equal}\AgdaSymbol{\{}\AgdaArgument{𝑨}\AgdaSpace{}%
\AgdaSymbol{=}\AgdaSpace{}%
\AgdaBound{𝑨}\AgdaSymbol{\}\{}\AgdaBound{sA}\AgdaSymbol{\}}\AgdaSpace{}%
\AgdaBound{pKq}\AgdaSpace{}%
\AgdaBound{ρ}\<%
\\
%
\>[2]\AgdaFunction{Goal}\AgdaSpace{}%
\AgdaSymbol{:}\AgdaSpace{}%
\AgdaSymbol{(}\AgdaFunction{free-lift}\AgdaSymbol{\{}\AgdaArgument{𝑨}\AgdaSpace{}%
\AgdaSymbol{=}\AgdaSpace{}%
\AgdaBound{𝑨}\AgdaSymbol{\}}\AgdaSpace{}%
\AgdaBound{ρ}\AgdaSpace{}%
\AgdaBound{p}\AgdaSymbol{)}\AgdaSpace{}%
\AgdaOperator{\AgdaFunction{≈}}\AgdaSpace{}%
\AgdaSymbol{(}\AgdaFunction{free-lift}\AgdaSymbol{\{}\AgdaArgument{𝑨}\AgdaSpace{}%
\AgdaSymbol{=}\AgdaSpace{}%
\AgdaBound{𝑨}\AgdaSymbol{\}}\AgdaSpace{}%
\AgdaBound{ρ}\AgdaSpace{}%
\AgdaBound{q}\AgdaSymbol{)}\<%
\\
%
\>[2]\AgdaFunction{Goal}\AgdaSpace{}%
\AgdaSymbol{=}\AgdaSpace{}%
\AgdaFunction{trans}\AgdaSpace{}%
\AgdaSymbol{(}\AgdaFunction{sym}\AgdaSpace{}%
\AgdaSymbol{(}\AgdaFunction{fl}\AgdaSpace{}%
\AgdaBound{p}\AgdaSymbol{))}\AgdaSpace{}%
\AgdaSymbol{(}\AgdaFunction{trans}\AgdaSpace{}%
\AgdaFunction{subgoal}\AgdaSpace{}%
\AgdaSymbol{(}\AgdaFunction{fl}\AgdaSpace{}%
\AgdaBound{q}\AgdaSymbol{))}\<%
\\
%
\\[\AgdaEmptyExtraSkip]%
%
\\[\AgdaEmptyExtraSkip]%
%
\>[1]\AgdaFunction{hom𝔽ℭ}\AgdaSpace{}%
\AgdaSymbol{:}\AgdaSpace{}%
\AgdaFunction{hom}\AgdaSpace{}%
\AgdaOperator{\AgdaFunction{𝔽[}}\AgdaSpace{}%
\AgdaBound{X}\AgdaSpace{}%
\AgdaOperator{\AgdaFunction{]}}\AgdaSpace{}%
\AgdaFunction{ℭ}\<%
\\
%
\>[1]\AgdaFunction{hom𝔽ℭ}\AgdaSpace{}%
\AgdaSymbol{=}\AgdaSpace{}%
\AgdaOperator{\AgdaFunction{∣}}\AgdaSpace{}%
\AgdaFunction{HomFactor}\AgdaSpace{}%
\AgdaFunction{ℭ}\AgdaSpace{}%
\AgdaFunction{homℭ}\AgdaSpace{}%
\AgdaOperator{\AgdaFunction{hom𝔽[}}\AgdaSpace{}%
\AgdaBound{X}\AgdaSpace{}%
\AgdaOperator{\AgdaFunction{]}}\AgdaSpace{}%
\AgdaFunction{ker𝔽⊆kerℭ}\AgdaSpace{}%
\AgdaOperator{\AgdaFunction{hom𝔽[}}\AgdaSpace{}%
\AgdaBound{X}\AgdaSpace{}%
\AgdaOperator{\AgdaFunction{]-is-epic}}\AgdaSpace{}%
\AgdaOperator{\AgdaFunction{∣}}\<%
\\
%
\\[\AgdaEmptyExtraSkip]%
%
\>[1]\AgdaFunction{kerℭ⊆ker𝔽}\AgdaSpace{}%
\AgdaSymbol{:}\AgdaSpace{}%
\AgdaSymbol{∀\{}\AgdaBound{p}\AgdaSpace{}%
\AgdaBound{q}\AgdaSymbol{\}}\AgdaSpace{}%
\AgdaSymbol{→}\AgdaSpace{}%
\AgdaSymbol{(}\AgdaBound{p}\AgdaSpace{}%
\AgdaOperator{\AgdaInductiveConstructor{,}}\AgdaSpace{}%
\AgdaBound{q}\AgdaSymbol{)}\AgdaSpace{}%
\AgdaOperator{\AgdaFunction{∈}}\AgdaSpace{}%
\AgdaFunction{ker}\AgdaSpace{}%
\AgdaOperator{\AgdaFunction{∣}}\AgdaSpace{}%
\AgdaFunction{homℭ}\AgdaSpace{}%
\AgdaOperator{\AgdaFunction{∣}}\AgdaSpace{}%
\AgdaSymbol{→}\AgdaSpace{}%
\AgdaSymbol{(}\AgdaBound{p}\AgdaSpace{}%
\AgdaOperator{\AgdaInductiveConstructor{,}}\AgdaSpace{}%
\AgdaBound{q}\AgdaSymbol{)}\AgdaSpace{}%
\AgdaOperator{\AgdaFunction{∈}}\AgdaSpace{}%
\AgdaFunction{ker}\AgdaSpace{}%
\AgdaOperator{\AgdaFunction{∣}}\AgdaSpace{}%
\AgdaOperator{\AgdaFunction{hom𝔽[}}\AgdaSpace{}%
\AgdaBound{X}\AgdaSpace{}%
\AgdaOperator{\AgdaFunction{]}}\AgdaSpace{}%
\AgdaOperator{\AgdaFunction{∣}}\<%
\\
%
\>[1]\AgdaFunction{kerℭ⊆ker𝔽}\AgdaSpace{}%
\AgdaSymbol{\{}\AgdaBound{p}\AgdaSymbol{\}\{}\AgdaBound{q}\AgdaSymbol{\}}\AgdaSpace{}%
\AgdaBound{pKq}\AgdaSpace{}%
\AgdaSymbol{=}\AgdaSpace{}%
\AgdaFunction{E⊢pq}\<%
\\
\>[1][@{}l@{\AgdaIndent{0}}]%
\>[2]\AgdaKeyword{where}\<%
\\
%
\>[2]\AgdaFunction{pqEqual}\AgdaSpace{}%
\AgdaSymbol{:}\AgdaSpace{}%
\AgdaSymbol{∀}\AgdaSpace{}%
\AgdaBound{i}\AgdaSpace{}%
\AgdaSymbol{→}\AgdaSpace{}%
\AgdaFunction{skEqual}\AgdaSpace{}%
\AgdaBound{i}\AgdaSpace{}%
\AgdaSymbol{\{}\AgdaBound{p}\AgdaSymbol{\}\{}\AgdaBound{q}\AgdaSymbol{\}}\<%
\\
%
\>[2]\AgdaFunction{pqEqual}\AgdaSpace{}%
\AgdaBound{i}\AgdaSpace{}%
\AgdaSymbol{=}\AgdaSpace{}%
\AgdaFunction{goal}\<%
\\
\>[2][@{}l@{\AgdaIndent{0}}]%
\>[3]\AgdaKeyword{where}\<%
\\
%
\>[3]\AgdaKeyword{open}\AgdaSpace{}%
\AgdaModule{Environment}\AgdaSpace{}%
\AgdaSymbol{(}\AgdaFunction{𝔄⁺}\AgdaSpace{}%
\AgdaBound{i}\AgdaSymbol{)}\AgdaSpace{}%
\AgdaKeyword{using}\AgdaSpace{}%
\AgdaSymbol{(}\AgdaSpace{}%
\AgdaOperator{\AgdaFunction{⟦\AgdaUnderscore{}⟧}}\AgdaSpace{}%
\AgdaSymbol{)}\<%
\\
%
\>[3]\AgdaKeyword{open}\AgdaSpace{}%
\AgdaModule{Setoid}\AgdaSpace{}%
\AgdaOperator{\AgdaFunction{𝔻[}}\AgdaSpace{}%
\AgdaFunction{𝔄⁺}\AgdaSpace{}%
\AgdaBound{i}\AgdaSpace{}%
\AgdaOperator{\AgdaFunction{]}}\AgdaSpace{}%
\AgdaKeyword{using}\AgdaSpace{}%
\AgdaSymbol{(}\AgdaSpace{}%
\AgdaOperator{\AgdaField{\AgdaUnderscore{}≈\AgdaUnderscore{}}}\AgdaSpace{}%
\AgdaSymbol{;}\AgdaSpace{}%
\AgdaFunction{sym}\AgdaSpace{}%
\AgdaSymbol{;}\AgdaSpace{}%
\AgdaFunction{trans}\AgdaSpace{}%
\AgdaSymbol{)}\<%
\\
%
\>[3]\AgdaFunction{goal}\AgdaSpace{}%
\AgdaSymbol{:}\AgdaSpace{}%
\AgdaOperator{\AgdaFunction{⟦}}\AgdaSpace{}%
\AgdaBound{p}\AgdaSpace{}%
\AgdaOperator{\AgdaFunction{⟧}}\AgdaSpace{}%
\AgdaOperator{\AgdaField{⟨\$⟩}}\AgdaSpace{}%
\AgdaField{snd}\AgdaSpace{}%
\AgdaOperator{\AgdaFunction{∥}}\AgdaSpace{}%
\AgdaBound{i}\AgdaSpace{}%
\AgdaOperator{\AgdaFunction{∥}}\AgdaSpace{}%
\AgdaOperator{\AgdaFunction{≈}}\AgdaSpace{}%
\AgdaOperator{\AgdaFunction{⟦}}\AgdaSpace{}%
\AgdaBound{q}\AgdaSpace{}%
\AgdaOperator{\AgdaFunction{⟧}}\AgdaSpace{}%
\AgdaOperator{\AgdaField{⟨\$⟩}}\AgdaSpace{}%
\AgdaField{snd}\AgdaSpace{}%
\AgdaOperator{\AgdaFunction{∥}}\AgdaSpace{}%
\AgdaBound{i}\AgdaSpace{}%
\AgdaOperator{\AgdaFunction{∥}}\<%
\\
%
\>[3]\AgdaFunction{goal}\AgdaSpace{}%
\AgdaSymbol{=}%
\>[7659I]\AgdaFunction{trans}\AgdaSpace{}%
\AgdaSymbol{(}\AgdaFunction{free-lift-interp}\AgdaSymbol{\{}\AgdaArgument{𝑨}\AgdaSpace{}%
\AgdaSymbol{=}\AgdaSpace{}%
\AgdaOperator{\AgdaFunction{∣}}\AgdaSpace{}%
\AgdaBound{i}\AgdaSpace{}%
\AgdaOperator{\AgdaFunction{∣}}\AgdaSymbol{\}(}\AgdaField{snd}\AgdaSpace{}%
\AgdaOperator{\AgdaFunction{∥}}\AgdaSpace{}%
\AgdaBound{i}\AgdaSpace{}%
\AgdaOperator{\AgdaFunction{∥}}\AgdaSymbol{)}\AgdaSpace{}%
\AgdaBound{p}\AgdaSymbol{)}\<%
\\
\>[7659I][@{}l@{\AgdaIndent{0}}]%
\>[11]\AgdaSymbol{(}\AgdaFunction{trans}\AgdaSpace{}%
\AgdaSymbol{(}\AgdaBound{pKq}\AgdaSpace{}%
\AgdaBound{i}\AgdaSymbol{)(}\AgdaFunction{sym}\AgdaSpace{}%
\AgdaSymbol{(}\AgdaFunction{free-lift-interp}\AgdaSymbol{\{}\AgdaArgument{𝑨}\AgdaSpace{}%
\AgdaSymbol{=}\AgdaSpace{}%
\AgdaOperator{\AgdaFunction{∣}}\AgdaSpace{}%
\AgdaBound{i}\AgdaSpace{}%
\AgdaOperator{\AgdaFunction{∣}}\AgdaSymbol{\}}\AgdaSpace{}%
\AgdaSymbol{(}\AgdaField{snd}\AgdaSpace{}%
\AgdaOperator{\AgdaFunction{∥}}\AgdaSpace{}%
\AgdaBound{i}\AgdaSpace{}%
\AgdaOperator{\AgdaFunction{∥}}\AgdaSymbol{)}\AgdaSpace{}%
\AgdaBound{q}\AgdaSymbol{)))}\<%
\\
%
\>[2]\AgdaFunction{E⊢pq}\AgdaSpace{}%
\AgdaSymbol{:}\AgdaSpace{}%
\AgdaFunction{ℰ}\AgdaSpace{}%
\AgdaOperator{\AgdaDatatype{⊢}}\AgdaSpace{}%
\AgdaBound{X}\AgdaSpace{}%
\AgdaOperator{\AgdaDatatype{▹}}\AgdaSpace{}%
\AgdaBound{p}\AgdaSpace{}%
\AgdaOperator{\AgdaDatatype{≈}}\AgdaSpace{}%
\AgdaBound{q}\<%
\\
%
\>[2]\AgdaFunction{E⊢pq}\AgdaSpace{}%
\AgdaSymbol{=}\AgdaSpace{}%
\AgdaFunction{AllEqual⊆ker𝔽}\AgdaSpace{}%
\AgdaFunction{pqEqual}\<%
\\
%
\\[\AgdaEmptyExtraSkip]%
%
\>[1]\AgdaFunction{mon𝔽ℭ}\AgdaSpace{}%
\AgdaSymbol{:}\AgdaSpace{}%
\AgdaFunction{mon}\AgdaSpace{}%
\AgdaOperator{\AgdaFunction{𝔽[}}\AgdaSpace{}%
\AgdaBound{X}\AgdaSpace{}%
\AgdaOperator{\AgdaFunction{]}}\AgdaSpace{}%
\AgdaFunction{ℭ}\<%
\\
%
\>[1]\AgdaFunction{mon𝔽ℭ}\AgdaSpace{}%
\AgdaSymbol{=}\AgdaSpace{}%
\AgdaOperator{\AgdaFunction{∣}}\AgdaSpace{}%
\AgdaFunction{hom𝔽ℭ}\AgdaSpace{}%
\AgdaOperator{\AgdaFunction{∣}}\AgdaSpace{}%
\AgdaOperator{\AgdaInductiveConstructor{,}}\AgdaSpace{}%
\AgdaFunction{isMon}\<%
\\
\>[1][@{}l@{\AgdaIndent{0}}]%
\>[2]\AgdaKeyword{where}\<%
\\
%
\>[2]\AgdaFunction{isMon}\AgdaSpace{}%
\AgdaSymbol{:}\AgdaSpace{}%
\AgdaRecord{IsMon}\AgdaSpace{}%
\AgdaOperator{\AgdaFunction{𝔽[}}\AgdaSpace{}%
\AgdaBound{X}\AgdaSpace{}%
\AgdaOperator{\AgdaFunction{]}}\AgdaSpace{}%
\AgdaFunction{ℭ}\AgdaSpace{}%
\AgdaOperator{\AgdaFunction{∣}}\AgdaSpace{}%
\AgdaFunction{hom𝔽ℭ}\AgdaSpace{}%
\AgdaOperator{\AgdaFunction{∣}}\<%
\\
%
\>[2]\AgdaField{isHom}\AgdaSpace{}%
\AgdaFunction{isMon}\AgdaSpace{}%
\AgdaSymbol{=}\AgdaSpace{}%
\AgdaOperator{\AgdaFunction{∥}}\AgdaSpace{}%
\AgdaFunction{hom𝔽ℭ}\AgdaSpace{}%
\AgdaOperator{\AgdaFunction{∥}}\<%
\\
%
\>[2]\AgdaField{isInjective}\AgdaSpace{}%
\AgdaFunction{isMon}\AgdaSpace{}%
\AgdaSymbol{\{}\AgdaBound{p}\AgdaSymbol{\}}\AgdaSpace{}%
\AgdaSymbol{\{}\AgdaBound{q}\AgdaSymbol{\}}\AgdaSpace{}%
\AgdaBound{φpq}\AgdaSpace{}%
\AgdaSymbol{=}\AgdaSpace{}%
\AgdaFunction{kerℭ⊆ker𝔽}\AgdaSpace{}%
\AgdaBound{φpq}\<%
\\
\>[0]\<%
\end{code}

Now that we have proved the existence of a monomorphism from
\texttt{𝔽{[}\ X\ {]}} to \texttt{ℭ} we are in a position to prove that
\texttt{𝔽{[}\ X\ {]}} is a subalgebra of ℭ, so belongs to
\texttt{S\ (P\ 𝒦)}. In fact, we will show that \texttt{𝔽{[}\ X\ {]}} is
a subalgebra of the \emph{lift} of \texttt{ℭ}, denoted \texttt{ℓℭ}.

\begin{code}%
\>[0]\<%
\\
\>[0][@{}l@{\AgdaIndent{1}}]%
\>[1]\AgdaFunction{𝔽≤ℭ}\AgdaSpace{}%
\AgdaSymbol{:}\AgdaSpace{}%
\AgdaOperator{\AgdaFunction{𝔽[}}\AgdaSpace{}%
\AgdaBound{X}\AgdaSpace{}%
\AgdaOperator{\AgdaFunction{]}}\AgdaSpace{}%
\AgdaOperator{\AgdaFunction{≤}}\AgdaSpace{}%
\AgdaFunction{ℭ}\<%
\\
%
\>[1]\AgdaFunction{𝔽≤ℭ}\AgdaSpace{}%
\AgdaSymbol{=}\AgdaSpace{}%
\AgdaFunction{mon→≤}\AgdaSpace{}%
\AgdaFunction{mon𝔽ℭ}\<%
\\
%
\\[\AgdaEmptyExtraSkip]%
%
\>[1]\AgdaFunction{SP𝔽}\AgdaSpace{}%
\AgdaSymbol{:}\AgdaSpace{}%
\AgdaOperator{\AgdaFunction{𝔽[}}\AgdaSpace{}%
\AgdaBound{X}\AgdaSpace{}%
\AgdaOperator{\AgdaFunction{]}}\AgdaSpace{}%
\AgdaOperator{\AgdaFunction{∈}}\AgdaSpace{}%
\AgdaFunction{S}\AgdaSpace{}%
\AgdaFunction{ι}\AgdaSpace{}%
\AgdaSymbol{(}\AgdaFunction{P}\AgdaSpace{}%
\AgdaBound{ℓ}\AgdaSpace{}%
\AgdaFunction{ι}\AgdaSpace{}%
\AgdaBound{𝒦}\AgdaSymbol{)}\<%
\\
%
\>[1]\AgdaFunction{SP𝔽}\AgdaSpace{}%
\AgdaSymbol{=}\AgdaSpace{}%
\AgdaFunction{S-idem}\AgdaSpace{}%
\AgdaFunction{SSP𝔽}\<%
\\
\>[1][@{}l@{\AgdaIndent{0}}]%
\>[2]\AgdaKeyword{where}\<%
\\
%
\>[2]\AgdaFunction{PSℭ}\AgdaSpace{}%
\AgdaSymbol{:}\AgdaSpace{}%
\AgdaFunction{ℭ}\AgdaSpace{}%
\AgdaOperator{\AgdaFunction{∈}}\AgdaSpace{}%
\AgdaFunction{P}\AgdaSpace{}%
\AgdaSymbol{(}\AgdaBound{α}\AgdaSpace{}%
\AgdaOperator{\AgdaPrimitive{⊔}}\AgdaSpace{}%
\AgdaBound{ρᵃ}\AgdaSpace{}%
\AgdaOperator{\AgdaPrimitive{⊔}}\AgdaSpace{}%
\AgdaBound{ℓ}\AgdaSymbol{)}\AgdaSpace{}%
\AgdaFunction{ι}\AgdaSpace{}%
\AgdaSymbol{(}\AgdaFunction{S}\AgdaSpace{}%
\AgdaBound{ℓ}\AgdaSpace{}%
\AgdaBound{𝒦}\AgdaSymbol{)}\<%
\\
%
\>[2]\AgdaFunction{PSℭ}\AgdaSpace{}%
\AgdaSymbol{=}\AgdaSpace{}%
\AgdaFunction{ℑ⁺}\AgdaSpace{}%
\AgdaOperator{\AgdaInductiveConstructor{,}}\AgdaSpace{}%
\AgdaSymbol{(}\AgdaFunction{𝔄⁺}\AgdaSpace{}%
\AgdaOperator{\AgdaInductiveConstructor{,}}\AgdaSpace{}%
\AgdaSymbol{((λ}\AgdaSpace{}%
\AgdaBound{i}\AgdaSpace{}%
\AgdaSymbol{→}\AgdaSpace{}%
\AgdaField{fst}\AgdaSpace{}%
\AgdaOperator{\AgdaFunction{∥}}\AgdaSpace{}%
\AgdaBound{i}\AgdaSpace{}%
\AgdaOperator{\AgdaFunction{∥}}\AgdaSymbol{)}\AgdaSpace{}%
\AgdaOperator{\AgdaInductiveConstructor{,}}\AgdaSpace{}%
\AgdaFunction{≅-refl}\AgdaSymbol{))}\<%
\\
%
\>[2]\AgdaFunction{SPℭ}\AgdaSpace{}%
\AgdaSymbol{:}\AgdaSpace{}%
\AgdaFunction{ℭ}\AgdaSpace{}%
\AgdaOperator{\AgdaFunction{∈}}\AgdaSpace{}%
\AgdaFunction{S}\AgdaSpace{}%
\AgdaFunction{ι}\AgdaSpace{}%
\AgdaSymbol{(}\AgdaFunction{P}\AgdaSpace{}%
\AgdaBound{ℓ}\AgdaSpace{}%
\AgdaFunction{ι}\AgdaSpace{}%
\AgdaBound{𝒦}\AgdaSymbol{)}\<%
\\
%
\>[2]\AgdaFunction{SPℭ}\AgdaSpace{}%
\AgdaSymbol{=}\AgdaSpace{}%
\AgdaFunction{PS⊆SP}\AgdaSpace{}%
\AgdaSymbol{\{}\AgdaArgument{ℓ}\AgdaSpace{}%
\AgdaSymbol{=}\AgdaSpace{}%
\AgdaBound{ℓ}\AgdaSymbol{\}}\AgdaSpace{}%
\AgdaFunction{PSℭ}\<%
\\
%
\>[2]\AgdaFunction{SSP𝔽}\AgdaSpace{}%
\AgdaSymbol{:}\AgdaSpace{}%
\AgdaOperator{\AgdaFunction{𝔽[}}\AgdaSpace{}%
\AgdaBound{X}\AgdaSpace{}%
\AgdaOperator{\AgdaFunction{]}}\AgdaSpace{}%
\AgdaOperator{\AgdaFunction{∈}}\AgdaSpace{}%
\AgdaFunction{S}\AgdaSpace{}%
\AgdaFunction{ι}\AgdaSpace{}%
\AgdaSymbol{(}\AgdaFunction{S}\AgdaSpace{}%
\AgdaFunction{ι}\AgdaSpace{}%
\AgdaSymbol{(}\AgdaFunction{P}\AgdaSpace{}%
\AgdaBound{ℓ}\AgdaSpace{}%
\AgdaFunction{ι}\AgdaSpace{}%
\AgdaBound{𝒦}\AgdaSymbol{))}\<%
\\
%
\>[2]\AgdaFunction{SSP𝔽}\AgdaSpace{}%
\AgdaSymbol{=}\AgdaSpace{}%
\AgdaFunction{ℭ}\AgdaSpace{}%
\AgdaOperator{\AgdaInductiveConstructor{,}}\AgdaSpace{}%
\AgdaSymbol{(}\AgdaFunction{SPℭ}\AgdaSpace{}%
\AgdaOperator{\AgdaInductiveConstructor{,}}\AgdaSpace{}%
\AgdaFunction{𝔽≤ℭ}\AgdaSymbol{)}\<%
\end{code}

\hypertarget{the-hsp-theorem}{%
\subsection{The HSP Theorem}\label{the-hsp-theorem}}

Finally, we are in a position to prove Birkhoff's celebrated variety
theorem.

\begin{code}%
\>[0]\<%
\\
\>[0]\AgdaKeyword{module}\AgdaSpace{}%
\AgdaModule{\AgdaUnderscore{}}\AgdaSpace{}%
\AgdaSymbol{\{}\AgdaBound{𝒦}\AgdaSpace{}%
\AgdaSymbol{:}\AgdaSpace{}%
\AgdaFunction{Pred}\AgdaSymbol{(}\AgdaRecord{Algebra}\AgdaSpace{}%
\AgdaGeneralizable{α}\AgdaSpace{}%
\AgdaGeneralizable{ρᵃ}\AgdaSymbol{)}\AgdaSpace{}%
\AgdaSymbol{(}\AgdaGeneralizable{α}\AgdaSpace{}%
\AgdaOperator{\AgdaPrimitive{⊔}}\AgdaSpace{}%
\AgdaGeneralizable{ρᵃ}\AgdaSpace{}%
\AgdaOperator{\AgdaPrimitive{⊔}}\AgdaSpace{}%
\AgdaFunction{ov}\AgdaSpace{}%
\AgdaGeneralizable{ℓ}\AgdaSymbol{)\}}\AgdaSpace{}%
\AgdaKeyword{where}\<%
\\
\>[0][@{}l@{\AgdaIndent{0}}]%
\>[1]\AgdaKeyword{private}\AgdaSpace{}%
\AgdaFunction{ι}\AgdaSpace{}%
\AgdaSymbol{=}\AgdaSpace{}%
\AgdaFunction{ov}\AgdaSymbol{(}\AgdaBound{α}\AgdaSpace{}%
\AgdaOperator{\AgdaPrimitive{⊔}}\AgdaSpace{}%
\AgdaBound{ρᵃ}\AgdaSpace{}%
\AgdaOperator{\AgdaPrimitive{⊔}}\AgdaSpace{}%
\AgdaBound{ℓ}\AgdaSymbol{)}\<%
\\
%
\>[1]\AgdaKeyword{open}\AgdaSpace{}%
\AgdaModule{FreeHom}\AgdaSpace{}%
\AgdaSymbol{\{}\AgdaArgument{ℓ}\AgdaSpace{}%
\AgdaSymbol{=}\AgdaSpace{}%
\AgdaBound{ℓ}\AgdaSymbol{\}(}\AgdaBound{α}\AgdaSpace{}%
\AgdaOperator{\AgdaPrimitive{⊔}}\AgdaSpace{}%
\AgdaBound{ρᵃ}\AgdaSpace{}%
\AgdaOperator{\AgdaPrimitive{⊔}}\AgdaSpace{}%
\AgdaBound{ℓ}\AgdaSymbol{)\{}\AgdaBound{𝒦}\AgdaSymbol{\}}\<%
\\
%
\>[1]\AgdaKeyword{open}\AgdaSpace{}%
\AgdaModule{FreeAlgebra}\AgdaSpace{}%
\AgdaSymbol{\{}\AgdaArgument{ι}\AgdaSpace{}%
\AgdaSymbol{=}\AgdaSpace{}%
\AgdaFunction{ι}\AgdaSymbol{\}\{}\AgdaArgument{I}\AgdaSpace{}%
\AgdaSymbol{=}\AgdaSpace{}%
\AgdaFunction{ℐ}\AgdaSymbol{\}}\AgdaSpace{}%
\AgdaFunction{ℰ}\AgdaSpace{}%
\AgdaKeyword{using}\AgdaSpace{}%
\AgdaSymbol{(}\AgdaSpace{}%
\AgdaOperator{\AgdaFunction{𝔽[\AgdaUnderscore{}]}}\AgdaSpace{}%
\AgdaSymbol{)}\<%
\\
%
\\[\AgdaEmptyExtraSkip]%
%
\>[1]\AgdaFunction{Birkhoff}\AgdaSpace{}%
\AgdaSymbol{:}\AgdaSpace{}%
\AgdaSymbol{∀}\AgdaSpace{}%
\AgdaBound{𝑨}\AgdaSpace{}%
\AgdaSymbol{→}\AgdaSpace{}%
\AgdaBound{𝑨}\AgdaSpace{}%
\AgdaOperator{\AgdaFunction{∈}}\AgdaSpace{}%
\AgdaFunction{Mod}\AgdaSpace{}%
\AgdaSymbol{(}\AgdaFunction{Th}\AgdaSpace{}%
\AgdaSymbol{(}\AgdaFunction{V}\AgdaSpace{}%
\AgdaBound{ℓ}\AgdaSpace{}%
\AgdaFunction{ι}\AgdaSpace{}%
\AgdaBound{𝒦}\AgdaSymbol{))}\AgdaSpace{}%
\AgdaSymbol{→}\AgdaSpace{}%
\AgdaBound{𝑨}\AgdaSpace{}%
\AgdaOperator{\AgdaFunction{∈}}\AgdaSpace{}%
\AgdaFunction{V}\AgdaSpace{}%
\AgdaBound{ℓ}\AgdaSpace{}%
\AgdaFunction{ι}\AgdaSpace{}%
\AgdaBound{𝒦}\<%
\\
%
\>[1]\AgdaFunction{Birkhoff}\AgdaSpace{}%
\AgdaBound{𝑨}\AgdaSpace{}%
\AgdaBound{ModThA}\AgdaSpace{}%
\AgdaSymbol{=}\AgdaSpace{}%
\AgdaFunction{V-≅-lc}\AgdaSymbol{\{}\AgdaBound{α}\AgdaSymbol{\}\{}\AgdaBound{ρᵃ}\AgdaSymbol{\}\{}\AgdaBound{ℓ}\AgdaSymbol{\}}\AgdaSpace{}%
\AgdaBound{𝒦}\AgdaSpace{}%
\AgdaBound{𝑨}\AgdaSpace{}%
\AgdaFunction{VlA}\<%
\\
\>[1][@{}l@{\AgdaIndent{0}}]%
\>[2]\AgdaKeyword{where}\<%
\\
%
\>[2]\AgdaKeyword{open}\AgdaSpace{}%
\AgdaModule{Setoid}\AgdaSpace{}%
\AgdaOperator{\AgdaFunction{𝔻[}}\AgdaSpace{}%
\AgdaBound{𝑨}\AgdaSpace{}%
\AgdaOperator{\AgdaFunction{]}}\AgdaSpace{}%
\AgdaKeyword{using}\AgdaSpace{}%
\AgdaSymbol{()}\AgdaSpace{}%
\AgdaKeyword{renaming}\AgdaSpace{}%
\AgdaSymbol{(}\AgdaSpace{}%
\AgdaField{Carrier}\AgdaSpace{}%
\AgdaSymbol{to}\AgdaSpace{}%
\AgdaField{A}\AgdaSpace{}%
\AgdaSymbol{)}\<%
\\
%
\>[2]\AgdaFunction{sp𝔽A}\AgdaSpace{}%
\AgdaSymbol{:}\AgdaSpace{}%
\AgdaOperator{\AgdaFunction{𝔽[}}\AgdaSpace{}%
\AgdaFunction{A}\AgdaSpace{}%
\AgdaOperator{\AgdaFunction{]}}\AgdaSpace{}%
\AgdaOperator{\AgdaFunction{∈}}\AgdaSpace{}%
\AgdaFunction{S}\AgdaSymbol{\{}\AgdaFunction{ι}\AgdaSymbol{\}}\AgdaSpace{}%
\AgdaFunction{ι}\AgdaSpace{}%
\AgdaSymbol{(}\AgdaFunction{P}\AgdaSpace{}%
\AgdaBound{ℓ}\AgdaSpace{}%
\AgdaFunction{ι}\AgdaSpace{}%
\AgdaBound{𝒦}\AgdaSymbol{)}\<%
\\
%
\>[2]\AgdaFunction{sp𝔽A}\AgdaSpace{}%
\AgdaSymbol{=}\AgdaSpace{}%
\AgdaFunction{SP𝔽}\AgdaSymbol{\{}\AgdaArgument{ℓ}\AgdaSpace{}%
\AgdaSymbol{=}\AgdaSpace{}%
\AgdaBound{ℓ}\AgdaSymbol{\}}\AgdaSpace{}%
\AgdaBound{𝒦}\<%
\\
%
\>[2]\AgdaFunction{epi𝔽lA}\AgdaSpace{}%
\AgdaSymbol{:}\AgdaSpace{}%
\AgdaFunction{epi}\AgdaSpace{}%
\AgdaOperator{\AgdaFunction{𝔽[}}\AgdaSpace{}%
\AgdaFunction{A}\AgdaSpace{}%
\AgdaOperator{\AgdaFunction{]}}\AgdaSpace{}%
\AgdaSymbol{(}\AgdaFunction{Lift-Alg}\AgdaSpace{}%
\AgdaBound{𝑨}\AgdaSpace{}%
\AgdaFunction{ι}\AgdaSpace{}%
\AgdaFunction{ι}\AgdaSymbol{)}\<%
\\
%
\>[2]\AgdaFunction{epi𝔽lA}\AgdaSpace{}%
\AgdaSymbol{=}\AgdaSpace{}%
\AgdaFunction{𝔽-ModTh-epi-lift}\AgdaSymbol{\{}\AgdaArgument{ℓ}\AgdaSpace{}%
\AgdaSymbol{=}\AgdaSpace{}%
\AgdaBound{ℓ}\AgdaSymbol{\}}\AgdaSpace{}%
\AgdaSymbol{(λ}\AgdaSpace{}%
\AgdaSymbol{\{}\AgdaBound{p}\AgdaSpace{}%
\AgdaBound{q}\AgdaSymbol{\}}\AgdaSpace{}%
\AgdaSymbol{→}\AgdaSpace{}%
\AgdaBound{ModThA}\AgdaSymbol{\{}\AgdaArgument{p}\AgdaSpace{}%
\AgdaSymbol{=}\AgdaSpace{}%
\AgdaBound{p}\AgdaSymbol{\}\{}\AgdaBound{q}\AgdaSymbol{\})}\<%
\\
%
\>[2]\AgdaFunction{lAimg𝔽A}\AgdaSpace{}%
\AgdaSymbol{:}\AgdaSpace{}%
\AgdaFunction{Lift-Alg}\AgdaSpace{}%
\AgdaBound{𝑨}\AgdaSpace{}%
\AgdaFunction{ι}\AgdaSpace{}%
\AgdaFunction{ι}\AgdaSpace{}%
\AgdaOperator{\AgdaFunction{IsHomImageOf}}\AgdaSpace{}%
\AgdaOperator{\AgdaFunction{𝔽[}}\AgdaSpace{}%
\AgdaFunction{A}\AgdaSpace{}%
\AgdaOperator{\AgdaFunction{]}}\<%
\\
%
\>[2]\AgdaFunction{lAimg𝔽A}\AgdaSpace{}%
\AgdaSymbol{=}\AgdaSpace{}%
\AgdaFunction{epi→ontohom}\AgdaSpace{}%
\AgdaOperator{\AgdaFunction{𝔽[}}\AgdaSpace{}%
\AgdaFunction{A}\AgdaSpace{}%
\AgdaOperator{\AgdaFunction{]}}\AgdaSpace{}%
\AgdaSymbol{(}\AgdaFunction{Lift-Alg}\AgdaSpace{}%
\AgdaBound{𝑨}\AgdaSpace{}%
\AgdaFunction{ι}\AgdaSpace{}%
\AgdaFunction{ι}\AgdaSymbol{)}\AgdaSpace{}%
\AgdaFunction{epi𝔽lA}\<%
\\
%
\>[2]\AgdaFunction{VlA}\AgdaSpace{}%
\AgdaSymbol{:}\AgdaSpace{}%
\AgdaFunction{Lift-Alg}\AgdaSpace{}%
\AgdaBound{𝑨}\AgdaSpace{}%
\AgdaFunction{ι}\AgdaSpace{}%
\AgdaFunction{ι}\AgdaSpace{}%
\AgdaOperator{\AgdaFunction{∈}}\AgdaSpace{}%
\AgdaFunction{V}\AgdaSpace{}%
\AgdaBound{ℓ}\AgdaSpace{}%
\AgdaFunction{ι}\AgdaSpace{}%
\AgdaBound{𝒦}\<%
\\
%
\>[2]\AgdaFunction{VlA}\AgdaSpace{}%
\AgdaSymbol{=}\AgdaSpace{}%
\AgdaOperator{\AgdaFunction{𝔽[}}\AgdaSpace{}%
\AgdaFunction{A}\AgdaSpace{}%
\AgdaOperator{\AgdaFunction{]}}\AgdaSpace{}%
\AgdaOperator{\AgdaInductiveConstructor{,}}\AgdaSpace{}%
\AgdaFunction{sp𝔽A}\AgdaSpace{}%
\AgdaOperator{\AgdaInductiveConstructor{,}}\AgdaSpace{}%
\AgdaFunction{lAimg𝔽A}\<%
\\
\>[0]\<%
\end{code}

The converse inclusion, \texttt{V\ 𝒦\ ⊆\ Mod\ (Th\ (V\ 𝒦))}, is a simple
consequence of the fact that \texttt{Mod\ Th} is a closure operator.
Nonetheless, completeness demands that we formalize this inclusion as
well, however trivial the proof.

\begin{code}%
\>[0]\<%
\\
\>[0][@{}l@{\AgdaIndent{1}}]%
\>[1]\AgdaKeyword{module}\AgdaSpace{}%
\AgdaModule{\AgdaUnderscore{}}\AgdaSpace{}%
\AgdaSymbol{\{}\AgdaBound{𝑨}\AgdaSpace{}%
\AgdaSymbol{:}\AgdaSpace{}%
\AgdaRecord{Algebra}\AgdaSpace{}%
\AgdaBound{α}\AgdaSpace{}%
\AgdaBound{ρᵃ}\AgdaSymbol{\}}\AgdaSpace{}%
\AgdaKeyword{where}\<%
\\
\>[1][@{}l@{\AgdaIndent{0}}]%
\>[2]\AgdaFunction{Birkhoff-converse}\AgdaSpace{}%
\AgdaSymbol{:}\AgdaSpace{}%
\AgdaBound{𝑨}\AgdaSpace{}%
\AgdaOperator{\AgdaFunction{∈}}\AgdaSpace{}%
\AgdaFunction{V}\AgdaSymbol{\{}\AgdaBound{α}\AgdaSymbol{\}\{}\AgdaBound{ρᵃ}\AgdaSymbol{\}\{}\AgdaBound{α}\AgdaSymbol{\}\{}\AgdaBound{ρᵃ}\AgdaSymbol{\}\{}\AgdaBound{α}\AgdaSymbol{\}\{}\AgdaBound{ρᵃ}\AgdaSymbol{\}}\AgdaSpace{}%
\AgdaBound{ℓ}\AgdaSpace{}%
\AgdaFunction{ι}\AgdaSpace{}%
\AgdaBound{𝒦}\AgdaSpace{}%
\AgdaSymbol{→}\AgdaSpace{}%
\AgdaBound{𝑨}\AgdaSpace{}%
\AgdaOperator{\AgdaFunction{∈}}\AgdaSpace{}%
\AgdaFunction{Mod}\AgdaSymbol{\{}\AgdaArgument{X}\AgdaSpace{}%
\AgdaSymbol{=}\AgdaSpace{}%
\AgdaOperator{\AgdaFunction{𝕌[}}\AgdaSpace{}%
\AgdaBound{𝑨}\AgdaSpace{}%
\AgdaOperator{\AgdaFunction{]}}\AgdaSymbol{\}}\AgdaSpace{}%
\AgdaSymbol{(}\AgdaFunction{Th}\AgdaSpace{}%
\AgdaSymbol{(}\AgdaFunction{V}\AgdaSpace{}%
\AgdaBound{ℓ}\AgdaSpace{}%
\AgdaFunction{ι}\AgdaSpace{}%
\AgdaBound{𝒦}\AgdaSymbol{))}\<%
\\
%
\>[2]\AgdaFunction{Birkhoff-converse}\AgdaSpace{}%
\AgdaBound{vA}\AgdaSpace{}%
\AgdaBound{pThq}\AgdaSpace{}%
\AgdaSymbol{=}\AgdaSpace{}%
\AgdaBound{pThq}\AgdaSpace{}%
\AgdaBound{𝑨}\AgdaSpace{}%
\AgdaBound{vA}\<%
\\
\>[0]\<%
\end{code}

We have thus proved that every variety is an equational class.

Readers familiar with the classical formulation of the Birkhoff HSP
theorem as an ``if and only if'' assertion might worry that the proof is
still incomplete. However, recall that in the
\href{https://ualib.github.io/agda-algebras/Varieties.Func.Preservation.html}{Varieties.Func.Preservation}
module we proved the following identity preservation lemma:

\texttt{V-id1\ :\ 𝒦\ ⊫\ p\ ≈̇\ q\ →\ V\ 𝒦\ ⊫\ p\ ≈̇\ q}

Thus, if \texttt{𝒦} is an equational class---that is, if \texttt{𝒦} is
the class of algebras satisfying all identities in some set---then
\texttt{V\ 𝒦} ⊆ 𝒦\texttt{.\ \ On\ the\ other\ hand,\ we\ proved\ that}V`
is expansive in the
\href{https://ualib.github.io/agda-algebras/Varieties.Func.Closure.html}{Varieties.Func.Closure}
module:

\texttt{V-expa\ :\ 𝒦\ ⊆\ V\ 𝒦}

so \texttt{𝒦} (= \texttt{V\ 𝒦} = \texttt{HSP\ 𝒦}) is a variety.

Taken together, \texttt{V-id1} and \texttt{V-expa} constitute formal
proof that every equational class is a variety.

This completes the formal proof of Birkhoff's variety theorem.

 %% for arXiv version (where no subdirectories are allowed)

% % -*- TeX-master: "ualib-part1.tex" -*-
%%% Local Variables: 
%%% mode: latex
%%% TeX-engine: 'xetex
%%% End:
\section{Introduction}\label{sec:introduction}
To support formalization in type theory of research level mathematics in universal algebra and related fields, we present the Agda Universal Algebra Library (\agdaualib), a software library containing formal statements and proofs of the core definitions and results of universal algebra. The \ualib is written in \agda~\cite{Norell:2009}, a programming language and proof assistant based on \MLTT (\mltt) that supports dependent and inductive types.


\subsection{Motivation}\label{sec:motivation}
The seminal idea for the \agdaualib project was the observation that, on the one hand, a number of fundamental constructions in universal algebra can be defined recursively, and theorems about them proved by structural induction, while, on the other hand, inductive and dependent types make possible very precise formal representations of recursively defined objects, which often admit elegant constructive proofs of properties of such objects.  An important feature of such proofs in type theory is that they are total functional programs and, as such, they are computable, composable, and machine-verifiable.
% These observations suggested that there would be much to gain from implementing universal algebra in a language, such as Martin-L\"of type theory, that features dependent and inductive types.

Finally, our own research experience has taught us that a proof assistant and programming language (like Agda), when equipped with specialized libraries and domain-specific tactics to automate the proof idioms of our field, can be an extremely powerful and effective asset. As such we believe that proof assistants and their supporting libraries will eventually become indispensable tools in the working mathematician's toolkit.

\subsection{Attributions and Contributions}\label{sec:contributions}
The mathematical results described in this paper have well known \emph{informal} proofs. Our main contribution is the formalization, mechanization, and verification of the statements and proofs of these results in dependent type theory using Agda.

Unless explicitly stated otherwise, the Agda source code described in this paper is due to the author, with the following caveat: the \ualib depends on the \typetopology library of \MartinEscardo~\cite{MHE}. For convenience, we refer to Escard\'o's library as \typtop throughout the paper. For the sake of completeness and clarity, and to keep the paper mostly self-contained, we repeat some definitions from \typtop, but in each instance we cite the original source.\footnote{In the \ualib, such instances occur only inside hidden modules that are never actually used, followed immediately by a statement that imports the code in question from its original source.}


\subsection{Prior art}\label{sec:prior-art}
There have been a number of efforts to formalize parts of universal algebra in type theory prior to ours, most notably
\begin{itemize}
  \item Capretta~\cite{Capretta:1999} (1999) formalized the basics of universal algebra in the Calculus of Inductive Constructions using the Coq proof assistant;
    \item Spitters and van der Weegen~\cite{Spitters:2011} (2011) formalized the basics of universal algebra and some classical algebraic structures, also in the Calculus of Inductive Constructions using the Coq proof assistant, promoting the use of type classes;
 \item Gunther, et al~\cite{Gunther:2018} (2018) developed what seems to be (prior to the \ualib) the most extensive library of formal universal algebra to date; in particular, this work includes a formalization of some basic equational logic; also (unlike the \ualib) it handles \emph{multisorted} algebraic structures; (like the \ualib) it is based on dependent type theory and the Agda proof assistant.
\end{itemize}
Some other projects aimed at formalizing mathematics generally, and algebra in particular, have developed into very extensive libraries that include definitions, theorems, and proofs about algebraic structures, such as groups, rings, modules, etc.  However, the goals of these efforts seem to be the formalization of special classical algebraic structures, as opposed to the general theory of (universal) algebras. Moreover, the part of universal algebra and equational logic formalized in the \ualib extends beyond the scope of prior efforts. In particular, the library now includes a proof of Birkhoff's variety theorem.  Most other proofs of this theorem that we know of are informal and nonconstructive.\footnote{After completing the formal proof in \agda, we learned about a constructive version of Birkhoff's theorem proved by Carlstr\"om in~\cite{Carlstrom:2008}.  The latter is presented in the informal style of standard mathematical writing, and as far as we know it was never formalized in type theory and type-checked with a proof assistant. Nonetheless, a comparison of Carlstr\"om's proof and the \ualib proof would be interesting.}
 %% We remark briefly on this in~\S\ref{sec:conclusions-and-future-work}.}


\subsection{Organization of the paper}\label{sec:organization}

In this paper we limit ourselves to the presentation of the core foundational modules of the \ualib so that we have space to discuss some of the more interesting type theoretic and foundational issues that arose when developing the library and attempting to represent advanced mathematical notions in type theory and formalize them in Agda.  This is the first in a series of three papers describing the \agdaualib. The second paper (\cite{DeMeo:2021-2}) covers \emph{homomorphisms}, \emph{terms}, and \emph{subalgebras}. The third paper (\cite{DeMeo:2021-3}) covers \emph{free algebras}, \emph{equational classes} of algebras (i.e., \emph{varieties}), and \emph{Birkhoff's HSP theorem}.

This present paper is organized into three parts as follows. The first part is~\S\ref{sec:overture} which introduces the basic concepts of type theory with special emphasis on the way such concepts are formalized in \agda. Specifically, \S\ref{preliminaries} introduces \emph{Sigma types} and Agda's hierarchy of \emph{universes}. The important topics of \emph{equality} and \emph{function extensionality} are discussed in \S\ref{equality} and \S\ref{function-extensionality}; \S\ref{sec:inverse-image-invers} covers inverses and inverse images of functions. In \S\ref{sec:lifts-altern-univ} we describe a technical problem that one frequently encounters when working in a \emph{noncumulative universe hierarchy} and offer some tools for resolving the type-checking errors that arise from this.

The second part is~\S\ref{sec:relat-quot-types} which covers \emph{relation types} and \emph{quotient types}. Specifically,~\S\ref{sec:discrete-relations} defines types that represent \emph{unary} and \emph{binary relations} as well as \emph{function kernels}. These ``discrete relation types,'' are all very standard.  In~\S\ref{sec:continuous-relations} we introduce the (less standard) types that we use to represent \emph{general} and \emph{dependent relations}. We call these ``continuous relations'' because they can have arbitrary arity (general relations) and they can be defined over arbitrary families of types (dependent relations).
In~\S\ref{sec:equiv-relat-quot} we cover standard types for equivalence relations and quotients, and in~\S\ref{sec:trunc-sets-prop} we discuss a family of concepts that are vital to the mechanization of mathematics using type theory; these are the closely related concepts of \emph{truncation}, \emph{sets}, \emph{propositions}, and \emph{proposition extensionality}.

The third part of the paper is~\S\ref{sec:algebra-types} which covers the basic domain-specific types offered by the \ualib. It is here that we finally get to see some types representing algebraic structures.  Specifically, we describe types for \emph{operations} and \emph{signatures}~(\S\ref{sec:oper-sign}), \emph{general algebras}~(\S\ref{sec:algebras}), and \emph{product algebras}~(\S\ref{sec:product-algebras}), including types for representing \emph{products over arbitrary classes of algebraic structures}. Finally, we define types for congruence relations and quotient algebras in~\S\ref{congruences}.

\newcommand\otherparta{\textit{Part 2: homomorphisms, terms, and subalgebras}~\cite{DeMeo:2021-2}.}
\newcommand\otherpartb{\textit{Part 3: free algebras, equational classes, and Birkhoff's theorem}~\cite{DeMeo:2021-3}.}

\subsection{Resources}
We conclude this introduction with some pointers to helpful reference materials.
For the required background in Universal Algebra, we recommend the textbook by Clifford Bergman~\cite{Bergman:2012}.  For the type theory background, we recommend the HoTT Book~\cite{HoTT} and Escard\'o's \ufcourse~\cite{MHE}.

The following are informed the development of the \ualib and are highly recommended.
\begin{itemize}
\item \textit{\ufcourse}, \escardo~\cite{MHE}
\item \href{http://www.cse.chalmers.se/~peterd/papers/DependentTypesAtWork.pdf}{\it Dependent Types at Work}, Bove and Dybjer~\cite{Bove:2009}
\item \href{http://www.cse.chalmers.se/~ulfn/papers/afp08/tutorial.pdf}{\it Dependently Typed Programming in Agda}, Norell and Chapman~\cite{Norell:2008}
\item \href{http://www.sciencedirect.com/science/article/pii/S1571066118300768}{\it Formalization of Universal Algebra in Agda}, Gunther, Gadea, Pagano~\cite{Gunther:2018}
\item \href{https://plfa.github.io/}{\it Programming Languages Foundations in Agda}~\cite{Wadler:2020}
\end{itemize}

More information about \agdaualib can be obtained from the following official sources.
\begin{itemize}
  \item \href{https://ualib.gitlab.io}{\texttt{ualib.org}} (the web site) documents every line of code in the library.
  \item \href{https://gitlab.com/ualib/ualib.gitlab.io}{\texttt{gitlab.com/ualib/ualib.gitlab.io}} (the source code) \agdaualib is open source.\footnote{License: \href{https://creativecommons.org/licenses/by-sa/4.0/}{Creative Commons Attribution-ShareAlike 4.0 International License.}}
  \item \emph{The Agda UALib}, \otherparta
  \item \emph{The Agda UALib}, \otherpartb
\end{itemize}
The first item links to the official \ualib html documentation which includes complete proofs of every theorem we mention here, and much more, including the Agda modules covered in the first and third installments of this series of papers on the \ualib.

Finally, readers will get much more out of the paper if they download \agdaualib from \url{https://gitlab.com/ualib/ualib.gitlab.io}, install the library, and try it out for themselves.

% % -*- TeX-master: "ualib-part1.tex" -*-
%%% Local Variables: 
%%% mode: latex
%%% TeX-engine: 'xetex
%%% End:
\section{Overture}\label{sec:overture}

\subsection{Preliminaries}\label{preliminaries}
% : logical foundations, universes, dependent types
\firstsentence{Overture}{Preliminaries}
% -*- TeX-master: "ualib-part1.tex" -*-
%%% Local Variables: 
%%% mode: latex
%%% TeX-engine: 'xetex
%%% End:
This module defines (or imports) the most basic and important types of \defn{Martin-Löf dependent type theory} (MLTT).  Although this is standard, we take this opportunity to highlight aspects of the \ualib syntax that may differ from that of ``standard Agda.''  Although we don't discuss all of the types on which the library depends here, Appendix Section~\ref{sec:imports-from-type} provides a comprehensive list of the components from \MartinEscardo's \typetopology library~\cite{MHE} that are imported at one place or another in the \ualib.

\subsubsection{Logical foundations}\label{sec:logical-foundations}
% Mainly for the benefit of readers who are not yet proficient in Agda or type theory, we briefly describe the type theoretic foundations of the \ualib, including the most important basic types and features that are used throughout the library. 
The \ualib is based on a minimal version of \mltt that is the same or very close to the type theory on which \MartinEscardo's \TypeTopology Agda library is based.
% This is also the type theory that \escardo taught us in the short course \ufcourse at the Midlands Graduate School in the Foundations of Computing Science at University of Birmingham in 2019.}
We won't go into great detail here because there are already other very nice resources available, such as the section \href{https://www.cs.bham.ac.uk/~mhe/HoTT-UF-in-Agda-Lecture-Notes/HoTT-UF-Agda.html\#mlttinagda}{A spartan Martin-Löf type theory} of the lecture notes by \escardo just mentioned, the \href{https://ncatlab.org/nlab/show/Martin-L\%C3\%B6f+dependent+type+theory}{ncatlab entry on Martin-Löf dependent type theory}, as well as the HoTT Book~\cite{HoTT}.

We begin by noting that only a very small collection of objects is assumed at the jumping-off point for MLTT. We have the \defn{primitive types} (\ad 𝟘, \ad 𝟙, and \ad ℕ, denoting the empty type, one-element type, and natural numbers), the \defn{type formers} (\ad +, \ad Π, \ad Σ, \ad{Id}, denoting \defn{binary sum}, \defn{product}, \defn{sum}, and the \defn{identity} type), and an infinite collection of \defn{type universes} (types of types) and universe variables to denote them.  Like Escard\'o's, our universe variables are typically upper-case caligraphic letters from the latter half of the English alphabet (e.g., \ab 𝓤, \ab 𝓥, \ab 𝓦, etc.).

\paragraph*{Specifying logical foundations in Agda}
An Agda program typically begins by setting some options and by importing types from existing Agda libraries.
Options are specified with the \AgdaKeyword{OPTIONS} \emph{pragma} and control the way Agda behaves by, for example, specifying the logical axioms and deduction rules we wish to assume when the program is type-checked to verify its correctness. Every Agda program in the \ualib begins with the following line.
\ccpad
\begin{code}[number=code:options]
\>[0]\AgdaSymbol{\{-\#}\AgdaSpace{}%
\AgdaKeyword{OPTIONS}\AgdaSpace{}%
\AgdaPragma{--without-K}\AgdaSpace{}%
\AgdaPragma{--exact-split}\AgdaSpace{}%
\AgdaPragma{--safe}\AgdaSpace{}%
\AgdaSymbol{\#-\}}\<%
\end{code}
\ccpad
These options control certain foundational assumptions that Agda makes when type-checking the program to verify its correctness.
\begin{itemize}
\item \AgdaPragma{--without-K} disables \axiomk; see~\cite{agdaref-axiomk};
\item \AgdaPragma{--exact-split} makes Agda accept only definitions that are \emph{judgmental} equalities; see~\cite{agdatools-patternmatching};
\item \AgdaPragma{--safe} ensures that nothing is postulated outright---every non-MLTT axiom has to be an explicit assumption (e.g., an argument to a function or module); see~\cite{agdaref-safeagda} and~\cite{agdatools-patternmatching}.
\end{itemize}
%% \end{enumerate}
Throughout this paper we take assumptions 1--3 for granted without mentioning them explicitly.


\paragraph*{Agda Modules}
The \ak{OPTIONS} pragma is usually followed by the start of a module.  For example, the \ualibhtml{Prelude.Preliminaries} module begins with the following line.
\ccpad
\begin{code}%
\>[0]\AgdaKeyword{module}\AgdaSpace{}%
\AgdaModule{Prelude.Preliminaries}\AgdaSpace{}%
\AgdaKeyword{where}\<%
\end{code}
\ccpad
Sometimes we want to declare parameters that will be assumed throughout the module.  For instance, when working with algebras, we often assume they come from a particular fixed signature, and this signature is something we could fix as a parameter at the start of a module. Thus, we might start an \defn{anonymous submodule} of the main module with a line like\footnote{The \af{Signature} type will be defined in Section~\ref{sec:oper-sign}.}
\ccpad
\begin{code}%
\>[0]\AgdaKeyword{module}\AgdaSpace{}%
\AgdaUnderscore\AgdaSpace{}%
\AgdaSymbol{\{}\AgdaBound{𝑆}\AgdaSpace{}%
\AgdaSymbol{:}\AgdaSpace{}%
\AgdaFunction{Signature}\AgdaSpace{}%
\AgdaBound{𝓞}\AgdaSpace{}%
\AgdaBound{𝓥}\AgdaSymbol{\}}\AgdaSpace{}%
\AgdaKeyword{where}\<%
\end{code}
\ccpad
Such a module is called \emph{anonymous} because an underscore appears in place of a module name. Agda determines where a submodule ends by indentation.  This can take some getting used to, but after a short time it will feel very natural. The main module of a file must have the same name as the file, without the \texttt{.agda} or \texttt{.lagda} file extension.  The code inside the main module is not indented. Submodules are declared inside the main module and code inside these submodules must be indented to a fixed column.  As long as the code is indented, Agda considers it part of the submodule.  A submodule is exited as soon as a nonindented line of code appears.




\paragraph*{Agda's universe hierarchy}\label{ssec:agdas-universe-hierarchy}
The \agdaualib adopts the notation of Martin Escardo's \typetopology library~\cite{MHE}. In particular, \emph{universe levels}%
\footnote{See \url{https://agda.readthedocs.io/en/v2.6.1.2/language/universe-levels.html}.}
are denoted by capitalized script letters from the second half of the alphabet, e.g., \ab 𝓤, \ab 𝓥, \ab 𝓦, etc.  Also defined in \typetopology are the operators~\af ̇~and~\af ⁺. These map a universe level \ab 𝓤 to the universe \ab 𝓤\af ̇ := \Set \ab 𝓤 and the level \ab 𝓤 \af ⁺ \aod := \lsuc \ab 𝓤, respectively.  Thus, \ab 𝓤\af ̇ is simply an alias for the universe \Set \ab 𝓤, and we have \ab 𝓤\af ̇ \as : \ab 𝓤 \af ⁺\af ̇.
%% Table~\ref{tab:dictionary} translates between standard \agda syntax and \typetopology/\ualib notation.

To justify the introduction of this somewhat nonstandard notation for universe levels, \escardo points out that the Agda library uses \ak{Level} for universes (so what we write as \ab 𝓤\af ̇ is written \ak{Set}~\ab 𝓤 in standard Agda), but in univalent mathematics the types in \ab 𝓤\af ̇ need not be sets, so the standard Agda notation can be a bit confusing, especially to new users.


The hierarchy of universes in Agda is structured as \ab{𝓤}\af ̇ \as : \ab 𝓤 \af ⁺\af ̇, \hskip3mm
\ab{𝓤} \af ⁺\af ̇ \as : \ab 𝓤 \af ⁺\af ⁺\af ̇, etc. This means that the universe \ab 𝓤\af ̇ has type \ab 𝓤  \af ⁺\af ̇, and 𝓤 \af ⁺\af ̇ has type \ab 𝓤 \af ⁺\af ⁺\af ̇, and so on.  It is important to note, however, this does \emph{not} imply that \ab 𝓤\af ̇ \as : \ab 𝓤 \af ⁺\af ⁺\af ̇. In other words, Agda's universe hierarchy is \emph{noncummulative}. This makes it possible to treat universe levels more generally and precisely, which is nice. On the other hand, a noncummulative hierarchy can sometimes make for a nonfun proof assistant. Luckily, there are ways to circumvent noncummulativity without introducing logical inconsistencies into the type theory. Section~\ref{lifts-of-algebras} describes some domain-specific tools that we developed for this purpose.








\subsubsection{Dependent types}\label{sec:dependent-types}

\newcommand\FstUnder{\AgdaOperator{\AgdaFunction{∣\AgdaUnderscore{}∣}}\xspace}
\newcommand\SndUnder{\AgdaOperator{\AgdaFunction{∥\AgdaUnderscore{}∥}}\xspace}
% Convenient notations for the first and second projections out of a product are \FstUnder and \SndUnder, respectively. However, to improve readability or to avoid notation clashes with other modules, we sometimes use more standard alternatives, such as \AgdaFunction{pr₁} and \AgdaFunction{pr₂}, or \AgdaFunction{fst} and \AgdaFunction{snd}, or some combination of these. The definitions are standard so we omit them.\footnote{For details see~\cite{DeMeo:2021} or \ualiburl{Prelude.Preliminaries}.}








\paragraph*{Sigma types (dependent pairs)} %\label{ssec:dependent-pairs}

Given universes \ab 𝓤 and \ab 𝓥, a type \ab{X} \as : \ab 𝓤\af ̇, and a type family \ab{Y}~\as :~\ab X~\as →~\ab 𝓥\af ̇, the \defn{Sigma type} (aka \defn{dependent pair type}, aka \defn{dependent product type}) is denoted by \ar{Σ}(\ab x~\af ꞉~\ab X),~\ab Y~\ab x and generalizes the Cartesian product \ab{X} \af × \ab Y by allowing the type \ab{Y x} of the second argument of the ordered pair (\ab x \af , \ab y) to depend on the value \ab{x} of the first. That is, an inhabitant of the type \ar{Σ}(\ab x~\af ꞉~\ab X),~\ab Y~\ab x is a pair (\ab{x}~\af ,~\ab y) such that \abt{x}{X}
and \abt{y}{Y x}.

Agda's default syntax for a Sigma type is \AgdaRecord{Σ}\sP{3}\AgdaSymbol{λ}(\ab x\sP{3}꞉\sP{3}\ab X)\sP{3}\as →\sP{3}\ab Y, but we prefer the notation \AgdaRecord{Σ}~\ab x~꞉~\ab X~,~\ab Y, which is closer to the standard syntax described in the preceding paragraph. Fortunately, this preferred notation is available in the \typetopology library (see~\cite[Σ types]{MHE}).\footnote{\textbf{WARNING!} The symbol \as ꞉ in the expression \AgdaRecord{Σ}\sP{3}\ab x\sP{3}\as ꞉\sP{3}\ab X\sP{3}\AgdaComma\sP{3}\ab Y is not the ordinary colon; rather, it is the symbol obtained by typing \texttt{\textbackslash{}:4} in \agdatwomode.} For pedagogical purposes we repeat this definition here, inside a \defn{hidden module} so that it doesn't conflict with the original definition that we will import later.\footnote{To hide code from the rest of the development, we enclose it in a named module.  For example, the code inside the \af{hide-sigma} module will not conflict with the original definitions from the \typetopology library, even though we import the latter right after presenting the definition.  As long as we don't invoke \ak{open} \af{hide-sigma}, the code inside the \af{hide-sigma} module remains essentially hidden (though Agda will type-check the code for us). It may seem odd to both define things in the hidden module only to immediately import the definition that we actually use, but we do this in an attempt to exhibit all of the types on which the \ualib depends, in a clear and self-contained way, while also ensuring that this cannot be misinterpreted as a claim to originality.}
\ccpad
\begin{code}%
\>[0]\AgdaKeyword{module}\AgdaSpace{}%
\AgdaModule{hide-sigma}\AgdaSpace{}%
\AgdaKeyword{where}\<%
\\
%
\\[\AgdaEmptyExtraSkip]%
\>[0][@{}l@{\AgdaIndent{0}}]%
\>[1]\AgdaKeyword{record}\AgdaSpace{}%
\AgdaRecord{Σ}\AgdaSpace{}%
\AgdaSymbol{\{}\AgdaBound{𝓤}\AgdaSpace{}%
\AgdaBound{𝓥}\AgdaSymbol{\}}\AgdaSpace{}%
\AgdaSymbol{\{}\AgdaBound{X}\AgdaSpace{}%
\AgdaSymbol{:}\AgdaSpace{}%
\AgdaBound{𝓤}\AgdaSpace{}%
\AgdaOperator{\AgdaFunction{̇}}\AgdaSpace{}%
\AgdaSymbol{\}}\AgdaSpace{}%
\AgdaSymbol{(}\AgdaBound{Y}\AgdaSpace{}%
\AgdaSymbol{:}\AgdaSpace{}%
\AgdaBound{X}\AgdaSpace{}%
\AgdaSymbol{→}\AgdaSpace{}%
\AgdaBound{𝓥}\AgdaSpace{}%
\AgdaOperator{\AgdaFunction{̇}}\AgdaSpace{}%
\AgdaSymbol{)}\AgdaSpace{}%
\AgdaSymbol{:}\AgdaSpace{}%
\AgdaBound{𝓤}\AgdaSpace{}%
\AgdaOperator{\AgdaPrimitive{⊔}}\AgdaSpace{}%
\AgdaBound{𝓥}\AgdaSpace{}%
\AgdaOperator{\AgdaFunction{̇}}%
\>[53]\AgdaKeyword{where}\<%
\\
\>[1][@{}l@{\AgdaIndent{0}}]%
\>[2]\AgdaKeyword{constructor}\AgdaSpace{}%
\AgdaOperator{\AgdaInductiveConstructor{\AgdaUnderscore{},\AgdaUnderscore{}}}\<%
\\
%
\>[2]\AgdaKeyword{field}\<%
\\
\>[2][@{}l@{\AgdaIndent{0}}]%
\>[3]\AgdaField{pr₁}\AgdaSpace{}%
\AgdaSymbol{:}\AgdaSpace{}%
\AgdaBound{X}\<%
\\
%
\>[3]\AgdaField{pr₂}\AgdaSpace{}%
\AgdaSymbol{:}\AgdaSpace{}%
\AgdaBound{Y}\AgdaSpace{}%
\AgdaField{pr₁}\<%
\\
%
\\[\AgdaEmptyExtraSkip]%
%
\>[1]\AgdaKeyword{infixr}\AgdaSpace{}%
\AgdaNumber{50}\AgdaSpace{}%
\AgdaOperator{\AgdaInductiveConstructor{\AgdaUnderscore{},\AgdaUnderscore{}}}\<%
\end{code}
\scpad
% For this dependent pair type, we prefer the notation \ar{Σ}~\ab x~\af ꞉~\ab X~\af ,~\ab y, which is more pleasing and more standard than Agda's default syntax, \ar{Σ}~\as λ(\ab x~\as ꞉~\ab X)~\as →~\ab y. Escardó makes this preferred notation available in the \typetopology library by making the index type explicit, as follows.
\ccpad
\begin{code}%
\>[1]\AgdaFunction{-Σ}\AgdaSpace{}%
\AgdaSymbol{:}\AgdaSpace{}%
\AgdaSymbol{\{}\AgdaBound{𝓤}\AgdaSpace{}%
\AgdaBound{𝓥}\AgdaSpace{}%
\AgdaSymbol{:}\AgdaSpace{}%
\AgdaPostulate{Universe}\AgdaSymbol{\}}\AgdaSpace{}%
\AgdaSymbol{(}\AgdaBound{X}\AgdaSpace{}%
\AgdaSymbol{:}\AgdaSpace{}%
\AgdaBound{𝓤}\AgdaSpace{}%
\AgdaOperator{\AgdaFunction{̇}}\AgdaSpace{}%
\AgdaSymbol{)}\AgdaSpace{}%
\AgdaSymbol{(}\AgdaBound{Y}\AgdaSpace{}%
\AgdaSymbol{:}\AgdaSpace{}%
\AgdaBound{X}\AgdaSpace{}%
\AgdaSymbol{→}\AgdaSpace{}%
\AgdaBound{𝓥}\AgdaSpace{}%
\AgdaOperator{\AgdaFunction{̇}}\AgdaSpace{}%
\AgdaSymbol{)}\AgdaSpace{}%
\AgdaSymbol{→}\AgdaSpace{}%
\AgdaBound{𝓤}\AgdaSpace{}%
\AgdaOperator{\AgdaPrimitive{⊔}}\AgdaSpace{}%
\AgdaBound{𝓥}\AgdaSpace{}%
\AgdaOperator{\AgdaFunction{̇}}\<%
\\
%
\>[1]\AgdaFunction{-Σ}\AgdaSpace{}%
\AgdaBound{X}\AgdaSpace{}%
\AgdaBound{Y}\AgdaSpace{}%
\AgdaSymbol{=}\AgdaSpace{}%
\AgdaRecord{Σ}\AgdaSpace{}%
\AgdaBound{Y}\<%
\\
%
\\[\AgdaEmptyExtraSkip]%
%
\>[1]\AgdaKeyword{syntax}\AgdaSpace{}%
\AgdaFunction{-Σ}\AgdaSpace{}%
\AgdaBound{X}\AgdaSpace{}%
\AgdaSymbol{(λ}\AgdaSpace{}%
\AgdaBound{x}\AgdaSpace{}%
\AgdaSymbol{→}\AgdaSpace{}%
\AgdaBound{Y}\AgdaSymbol{)}\AgdaSpace{}%
\AgdaSymbol{=}\AgdaSpace{}%
\AgdaFunction{Σ}\AgdaSpace{}%
\AgdaBound{x}\AgdaSpace{}%
\AgdaFunction{꞉}\AgdaSpace{}%
\AgdaBound{X}\AgdaSpace{}%
\AgdaFunction{,}\AgdaSpace{}%
\AgdaBound{Y}\<%
\end{code}
\scpad
% \textbf{WARNING!} The symbol \af ꞉ is not the same as \as : despite how similar they may look. The correct colon in the expression \ar{Σ}~\ab x~\af ꞉~\ab X~\af ,~\ab y, above is obtained by typing \texttt{\textbackslash{}:4} in \agdamode.

A special case of the Sigma type is the one in which the type \ab{Y} doesn't depend on \ab{X}. This is the usual Cartesian product, defined in Agda as follows.
\ccpad
\begin{code}%
\>[1]\AgdaOperator{\AgdaFunction{\AgdaUnderscore{}×\AgdaUnderscore{}}}\AgdaSpace{}%
\AgdaSymbol{:}\AgdaSpace{}%
\AgdaGeneralizable{𝓤}\AgdaSpace{}%
\AgdaOperator{\AgdaFunction{̇}}\AgdaSpace{}%
\AgdaSymbol{→}\AgdaSpace{}%
\AgdaGeneralizable{𝓥}\AgdaSpace{}%
\AgdaOperator{\AgdaFunction{̇}}\AgdaSpace{}%
\AgdaSymbol{→}\AgdaSpace{}%
\AgdaGeneralizable{𝓤}\AgdaSpace{}%
\AgdaOperator{\AgdaPrimitive{⊔}}\AgdaSpace{}%
\AgdaGeneralizable{𝓥}\AgdaSpace{}%
\AgdaOperator{\AgdaFunction{̇}}\<%
\\
%
\>[1]\AgdaBound{X}\AgdaSpace{}%
\AgdaOperator{\AgdaFunction{×}}\AgdaSpace{}%
\AgdaBound{Y}\AgdaSpace{}%
\AgdaSymbol{=}\AgdaSpace{}%
\AgdaFunction{Σ}\AgdaSpace{}%
\AgdaBound{x}\AgdaSpace{}%
\AgdaFunction{꞉}\AgdaSpace{}%
\AgdaBound{X}\AgdaSpace{}%
\AgdaFunction{,}\AgdaSpace{}%
\AgdaBound{Y}\<%
\end{code}
\ccpad
Now that we have repeated these definitions from the \ualib for illustration purposes, let us import the original
definitions that we will use throughout the \ualib.
\ccpad
\begin{code}%
\>[0]\AgdaKeyword{open}\AgdaSpace{}%
\AgdaKeyword{import}\AgdaSpace{}%
\AgdaModule{Sigma-Type}\AgdaSpace{}%
\AgdaKeyword{renaming}\AgdaSpace{}%
\AgdaSymbol{(}\AgdaOperator{\AgdaInductiveConstructor{\AgdaUnderscore{},\AgdaUnderscore{}}}\AgdaSpace{}%
\AgdaSymbol{to}\AgdaSpace{}%
\AgdaKeyword{infixr}\AgdaSpace{}%
\AgdaNumber{50}\AgdaSpace{}%
\AgdaOperator{\AgdaInductiveConstructor{\AgdaUnderscore{},\AgdaUnderscore{}}}\AgdaSymbol{)}\<%
\\
\>[0]\AgdaKeyword{open}\AgdaSpace{}%
\AgdaKeyword{import}\AgdaSpace{}%
\AgdaModule{MGS-MLTT}\AgdaSpace{}%
\AgdaKeyword{using}\AgdaSpace{}%
\AgdaSymbol{(}\AgdaFunction{pr₁}\AgdaSymbol{;}\AgdaSpace{}%
\AgdaFunction{pr₂}\AgdaSymbol{;}\AgdaSpace{}%
\AgdaOperator{\AgdaFunction{\AgdaUnderscore{}×\AgdaUnderscore{}}}\AgdaSymbol{;}\AgdaSpace{}%
\AgdaFunction{-Σ}\AgdaSymbol{)}\<%
\end{code}
\ccpad
The definition of \ar Σ (and thus, of \af ×) includes the fields  \af{pr₁} and \af{pr₂} representing the first and second projections out of the product.  Sometimes we prefer to denote these projections with the notation \af{∣\_∣} and \af{∥\_∥}, respectively. However, for emphasis or readability we alternate between these and the following standard notations: \af{pr₁} and \af{fst} for the first projection, \af{pr₂} and \af{snd} for the second.  We define these alternative notations for projections out of pairs as follows.
% \footnote{We have made a concerted effort to avoid duplicating types that are already provided elsewhere, such as the \typetopology library.  We occasionally repeat the definitions of such types for pedagogical purposes, but we always import and work with the original definitions in order to make the sources known and to credit the original authors.}
\ccpad
\begin{code}%
\>[0]\AgdaKeyword{module}\AgdaSpace{}%
\AgdaModule{\AgdaUnderscore{}}\AgdaSpace{}%
\AgdaSymbol{\{}\AgdaBound{𝓤}\AgdaSpace{}%
\AgdaSymbol{:}\AgdaSpace{}%
\AgdaPostulate{Universe}\AgdaSymbol{\}}\AgdaSpace{}%
\AgdaKeyword{where}\<%
\\
%
\\[\AgdaEmptyExtraSkip]%
\>[0][@{}l@{\AgdaIndent{0}}]%
\>[1]\AgdaOperator{\AgdaFunction{∣\AgdaUnderscore{}∣}}\AgdaSpace{}%
\AgdaFunction{fst}\AgdaSpace{}%
\AgdaSymbol{:}\AgdaSpace{}%
\AgdaSymbol{\{}\AgdaBound{X}\AgdaSpace{}%
\AgdaSymbol{:}\AgdaSpace{}%
\AgdaBound{𝓤}\AgdaSpace{}%
\AgdaOperator{\AgdaFunction{̇}}\AgdaSpace{}%
\AgdaSymbol{\}\{}\AgdaBound{Y}\AgdaSpace{}%
\AgdaSymbol{:}\AgdaSpace{}%
\AgdaBound{X}\AgdaSpace{}%
\AgdaSymbol{→}\AgdaSpace{}%
\AgdaGeneralizable{𝓥}\AgdaSpace{}%
\AgdaOperator{\AgdaFunction{̇}}\AgdaSymbol{\}}\AgdaSpace{}%
\AgdaSymbol{→}\AgdaSpace{}%
\AgdaRecord{Σ}\AgdaSpace{}%
\AgdaBound{Y}\AgdaSpace{}%
\AgdaSymbol{→}\AgdaSpace{}%
\AgdaBound{X}\<%
\\
%
\>[1]\AgdaOperator{\AgdaFunction{∣}}\AgdaSpace{}%
\AgdaBound{x}\AgdaSpace{}%
\AgdaOperator{\AgdaInductiveConstructor{,}}\AgdaSpace{}%
\AgdaBound{y}\AgdaSpace{}%
\AgdaOperator{\AgdaFunction{∣}}\AgdaSpace{}%
\AgdaSymbol{=}\AgdaSpace{}%
\AgdaBound{x}\<%
\\
%
\>[1]\AgdaFunction{fst}\AgdaSpace{}%
\AgdaSymbol{(}\AgdaBound{x}\AgdaSpace{}%
\AgdaOperator{\AgdaInductiveConstructor{,}}\AgdaSpace{}%
\AgdaBound{y}\AgdaSymbol{)}\AgdaSpace{}%
\AgdaSymbol{=}\AgdaSpace{}%
\AgdaBound{x}\<%
\\
%
\\[\AgdaEmptyExtraSkip]%
%
\>[1]\AgdaOperator{\AgdaFunction{∥\AgdaUnderscore{}∥}}\AgdaSpace{}%
\AgdaFunction{snd}\AgdaSpace{}%
\AgdaSymbol{:}\AgdaSpace{}%
\AgdaSymbol{\{}\AgdaBound{X}\AgdaSpace{}%
\AgdaSymbol{:}\AgdaSpace{}%
\AgdaBound{𝓤}\AgdaSpace{}%
\AgdaOperator{\AgdaFunction{̇}}\AgdaSpace{}%
\AgdaSymbol{\}\{}\AgdaBound{Y}\AgdaSpace{}%
\AgdaSymbol{:}\AgdaSpace{}%
\AgdaBound{X}\AgdaSpace{}%
\AgdaSymbol{→}\AgdaSpace{}%
\AgdaGeneralizable{𝓥}\AgdaSpace{}%
\AgdaOperator{\AgdaFunction{̇}}\AgdaSpace{}%
\AgdaSymbol{\}}\AgdaSpace{}%
\AgdaSymbol{→}\AgdaSpace{}%
\AgdaSymbol{(}\AgdaBound{z}\AgdaSpace{}%
\AgdaSymbol{:}\AgdaSpace{}%
\AgdaRecord{Σ}\AgdaSpace{}%
\AgdaBound{Y}\AgdaSymbol{)}\AgdaSpace{}%
\AgdaSymbol{→}\AgdaSpace{}%
\AgdaBound{Y}\AgdaSpace{}%
\AgdaSymbol{(}\AgdaFunction{pr₁}\AgdaSpace{}%
\AgdaBound{z}\AgdaSymbol{)}\<%
\\
%
\>[1]\AgdaOperator{\AgdaFunction{∥}}\AgdaSpace{}%
\AgdaBound{x}\AgdaSpace{}%
\AgdaOperator{\AgdaInductiveConstructor{,}}\AgdaSpace{}%
\AgdaBound{y}\AgdaSpace{}%
\AgdaOperator{\AgdaFunction{∥}}\AgdaSpace{}%
\AgdaSymbol{=}\AgdaSpace{}%
\AgdaBound{y}\<%
\\
%
\>[1]\AgdaFunction{snd}\AgdaSpace{}%
\AgdaSymbol{(}\AgdaBound{x}\AgdaSpace{}%
\AgdaOperator{\AgdaInductiveConstructor{,}}\AgdaSpace{}%
\AgdaBound{y}\AgdaSymbol{)}\AgdaSpace{}%
\AgdaSymbol{=}\AgdaSpace{}%
\AgdaBound{y}\<%
\end{code}
\ccpad
Note that we put the definitions above inside an \defn{anonymous module}, which starts with the \ak{module} keyword followed by an underscore (instead of a module name). The purpose is simply to move the postulated typing judgments---the ``parameters'' of the module (e.g., `𝓤 : Universe`)---out of the way so they don't obfuscate the definitions inside the module.





\paragraph*{Pi types (dependent functions)} %\label{ssec:dependent-functions}
Given universes \ab 𝓤 and \ab 𝓥, a type \ab{X} \as : \ab 𝓤\af ̇, and a type family \ab{Y}~\as :~\ab X~\as →~\ab 𝓥\af ̇, the \defn{Pi type} (aka \defn{dependent function type}) is denoted by \ar Π(\ab x~\af :~\ab X),~\ab Y~\ab x and generalizes the function type \ab X \as → \ab Y by letting the type \ab Y~\ab x of the codomain depend on the value \ab x of the domain type. The dependent function type is defined in the \typetopology in a standard way.  For the reader's benefit, however, we repeat the definition here inside a hidden module.
\ccpad
\begin{code}%
\>[0]\AgdaKeyword{module}\AgdaSpace{}%
\AgdaModule{hide-pi}\AgdaSpace{}%
\AgdaSymbol{\{}\AgdaBound{𝓤}\AgdaSpace{}%
\AgdaBound{𝓦}\AgdaSpace{}%
\AgdaSymbol{:}\AgdaSpace{}%
\AgdaPostulate{Universe}\AgdaSymbol{\}}\AgdaSpace{}%
\AgdaKeyword{where}\<%
\\
%
\\[\AgdaEmptyExtraSkip]%
\>[0][@{}l@{\AgdaIndent{0}}]%
\>[1]\AgdaFunction{Π}\AgdaSpace{}%
\AgdaSymbol{:}\AgdaSpace{}%
\AgdaSymbol{\{}\AgdaBound{X}\AgdaSpace{}%
\AgdaSymbol{:}\AgdaSpace{}%
\AgdaBound{𝓤}\AgdaSpace{}%
\AgdaOperator{\AgdaFunction{̇}}\AgdaSpace{}%
\AgdaSymbol{\}}\AgdaSpace{}%
\AgdaSymbol{(}\AgdaBound{A}\AgdaSpace{}%
\AgdaSymbol{:}\AgdaSpace{}%
\AgdaBound{X}\AgdaSpace{}%
\AgdaSymbol{→}\AgdaSpace{}%
\AgdaBound{𝓦}\AgdaSpace{}%
\AgdaOperator{\AgdaFunction{̇}}\AgdaSpace{}%
\AgdaSymbol{)}\AgdaSpace{}%
\AgdaSymbol{→}\AgdaSpace{}%
\AgdaBound{𝓤}\AgdaSpace{}%
\AgdaOperator{\AgdaPrimitive{⊔}}\AgdaSpace{}%
\AgdaBound{𝓦}\AgdaSpace{}%
\AgdaOperator{\AgdaFunction{̇}}\<%
\\
%
\>[1]\AgdaFunction{Π}\AgdaSpace{}%
\AgdaSymbol{\{}\AgdaBound{X}\AgdaSymbol{\}}\AgdaSpace{}%
\AgdaBound{A}\AgdaSpace{}%
\AgdaSymbol{=}\AgdaSpace{}%
\AgdaSymbol{(}\AgdaBound{x}\AgdaSpace{}%
\AgdaSymbol{:}\AgdaSpace{}%
\AgdaBound{X}\AgdaSymbol{)}\AgdaSpace{}%
\AgdaSymbol{→}\AgdaSpace{}%
\AgdaBound{A}\AgdaSpace{}%
\AgdaBound{x}\<%
\end{code}
\ccpad
To make the syntax for \af{Π} conform to the standard notation for Pi types, \escardo uses the same trick as the one used above for Sigma types.\footnote{\textbf{WARNING!} The symbol \af ꞉ is not the same as \as : despite how similar they may look. The correct colon in the expression \af{Π}~\ab x~\as ꞉~\ab X~\af ,~\ab y above is obtained by typing \texttt{\textbackslash{}:4} in \agdamode.}
\ccpad
\begin{code}
\>[1]\AgdaFunction{-Π}\AgdaSpace{}%
\AgdaSymbol{:}\AgdaSpace{}%
\AgdaSymbol{(}\AgdaBound{X}\AgdaSpace{}%
\AgdaSymbol{:}\AgdaSpace{}%
\AgdaBound{𝓤}\AgdaSpace{}%
\AgdaOperator{\AgdaFunction{̇}}\AgdaSpace{}%
\AgdaSymbol{)(}\AgdaBound{Y}\AgdaSpace{}%
\AgdaSymbol{:}\AgdaSpace{}%
\AgdaBound{X}\AgdaSpace{}%
\AgdaSymbol{→}\AgdaSpace{}%
\AgdaBound{𝓦}\AgdaSpace{}%
\AgdaOperator{\AgdaFunction{̇}}\AgdaSpace{}%
\AgdaSymbol{)}\AgdaSpace{}%
\AgdaSymbol{→}\AgdaSpace{}%
\AgdaBound{𝓤}\AgdaSpace{}%
\AgdaOperator{\AgdaPrimitive{⊔}}\AgdaSpace{}%
\AgdaBound{𝓦}\AgdaSpace{}%
\AgdaOperator{\AgdaFunction{̇}}\<%
\\
%
\>[1]\AgdaFunction{-Π}\AgdaSpace{}%
\AgdaBound{X}\AgdaSpace{}%
\AgdaBound{Y}\AgdaSpace{}%
\AgdaSymbol{=}\AgdaSpace{}%
\AgdaFunction{Π}\AgdaSpace{}%
\AgdaBound{Y}\<%
\\
%
\\[\AgdaEmptyExtraSkip]%
%
\>[1]\AgdaKeyword{infixr}\AgdaSpace{}%
\AgdaNumber{-1}\AgdaSpace{}%
\AgdaFunction{-Π}\<%
\\
%
\>[1]\AgdaKeyword{syntax}\AgdaSpace{}%
\AgdaFunction{-Π}\AgdaSpace{}%
\AgdaBound{A}\AgdaSpace{}%
\AgdaSymbol{(λ}\AgdaSpace{}%
\AgdaBound{x}\AgdaSpace{}%
\AgdaSymbol{→}\AgdaSpace{}%
\AgdaBound{b}\AgdaSymbol{)}\AgdaSpace{}%
\AgdaSymbol{=}\AgdaSpace{}%
\AgdaFunction{Π}\AgdaSpace{}%
\AgdaBound{x}\AgdaSpace{}%
\AgdaFunction{꞉}\AgdaSpace{}%
\AgdaBound{A}\AgdaSpace{}%
\AgdaFunction{,}\AgdaSpace{}%
\AgdaBound{b}\<%
\end{code}




\subsection{Equality}\label{equality} %: definitional equality and transport
\firstsentence{Overture}{Equality}
% -*- TeX-master: "ualib-part1.tex" -*-
%%% Local Variables: 
%%% mode: latex
%%% TeX-engine: 'xetex
%%% End:
\subsubsection{Definitional equality}\label{sec:defin-equal}
Here we discuss what is probably the most important type in \mltt. It is called \defn{definitional equality}. This concept is most understood, at least heuristically, with the following slogan: ``Definitional equality is the substitution-preserving equivalence relation generated by definitions.''
We will make this precise below, but first let us quote from a primary source. Per Martin-L\"of offers the following definition in~\cite[\S1.11]{MR0387009} (italics added):\footnote{The \defn{definiendum} is the left-hand side of a defining equation, the \defn{definiens} is the right-hand side.\\ For readers who have never generated an equivalence relation: the \defn{reflexive closure} of \ab R \af ⊆ \ab A \ad × \ab A is the union of \ab R and all pairs of the form (\ab a , \ab a); the \defn{symmetric closure} is the union of \ab R and its inverse \{(\ab y , \ab x) : (\ab x , \ab y) ∈ \ab R\}; we leave it to the reader to come up with the correct definition of transitive closure.}
\begin{quote}
  \defn{Definitional equality} is defined to be the equivalence relation, that is, reflexive, symmetric and transitive relation, which is generated by the principles that a definiendum is always definitionally equal to its definiens and that definitional equality is preserved under substitution.
\end{quote}
To be sure we understand what this means, let $:=$ denote the relation with respect to which $x$ is related to $y$ (denoted $x := y$) if and only if $y$ \emph{is the definition of} $x$.  Then the definitional equality relation \ad ≡ is the reflexive, symmetric, transitive, substitutive closure of $:=$. By \defn{subsitutive closure} we mean closure under the following \defn{substitution rule}.
% \{\ab A \as : \ab 𝓤\af ̇\}\{\ab B \as : \ab A \as → \ab 𝓦\af ̇\}\{\ab x \abt{y}{A}\} \as → \ab x \ad ≡ \ab y \as → \ab B \ab x \ad ≡ \ab B \ab y\\[4pt]
\vskip3mm\hskip1cm \infer[(subst)]{\ab B~\ab x~\ad ≡~\ab B~\ab y}{\{\ab A~\as :~\ab 𝓤\af ̇\}~\{\ab B~\as :~\ab A~\as →~\ab 𝓦\af ̇\}~\{\ab x~\abt{y}{A}\}  & \ab x~\ad ≡~\ab y}\\[-5pt]


% In~\cite[\S1.11, page 85]{MR0387009}, Per Martin-L\"of describes definitional equality as follows:
% \begin{quote}
% ``Definitional equality is intensional equality, or equality of meaning (synonymy)... Definitional equality ≡ is a relation between linguistic expressions; it should not be confused with equality between objects (sets, elements of a set etc.)... Definitional equality is the equivalence relation generated by abbreviatory definitions, changes of bound variables and the principle of substituting equals for equals. Therefore it is decidable, but not in the sense that a ≡ b ∨ ¬(a ≡ b) holds, simply because a ≡ b is not a proposition in the sense of the present theory.''\\[-30pt]
% \begin{flushright}Per Martin-L\"of,\\
% \textit{Padua Lecture Notes}~\cite{MR769301}
% \end{flushright}
% \end{quote}

The datatype used in the \ualib to represent definitional equality is imported from the \am{Identity-Type} module of \typtop, but apart from superficial syntactic differences, it is equivalent to the identity type used in all other Agda libraries we know of.  We repeat the definition here for easy reference.
\ccpad
\begin{code}%
% \>[0]\AgdaKeyword{module}\AgdaSpace{}%
% \AgdaModule{hide-refl}\AgdaSpace{}%
% \AgdaSymbol{\{}\AgdaBound{𝓤}\AgdaSpace{}%
% \AgdaSymbol{:}\AgdaSpace{}%
% \AgdaPostulate{Universe}\AgdaSymbol{\}}\AgdaSpace{}%
% \AgdaKeyword{where}\<%
% \\
% %
% \\[\AgdaEmptyExtraSkip]%
% \>[0][@{}l@{\AgdaIndent{0}}]%
\>[1]\AgdaKeyword{data}\AgdaSpace{}%
\AgdaOperator{\AgdaDatatype{\AgdaUnderscore{}≡\AgdaUnderscore{}}}\AgdaSpace{}%
\AgdaSymbol{\{}\AgdaBound{𝓤}\AgdaSymbol{\}}\AgdaSpace{}%
\AgdaSymbol{\{}\AgdaBound{A}\AgdaSpace{}%
\AgdaSymbol{:}\AgdaSpace{}%
\AgdaBound{𝓤}\AgdaSpace{}%
\AgdaOperator{\AgdaFunction{̇}}\AgdaSpace{}%
\AgdaSymbol{\}}\AgdaSpace{}%
\AgdaSymbol{:}\AgdaSpace{}%
\AgdaBound{A}\AgdaSpace{}%
\AgdaSymbol{→}\AgdaSpace{}%
\AgdaBound{A}\AgdaSpace{}%
\AgdaSymbol{→}\AgdaSpace{}%
\AgdaBound{𝓤}\AgdaSpace{}%
\AgdaOperator{\AgdaFunction{̇}}\AgdaSpace{}%
\AgdaKeyword{where}\AgdaSpace{}%
\AgdaInductiveConstructor{refl}\AgdaSpace{}%
\AgdaSymbol{:}\AgdaSpace{}%
\AgdaSymbol{\{}\AgdaBound{x}\AgdaSpace{}%
\AgdaSymbol{:}\AgdaSpace{}%
\AgdaBound{A}\AgdaSymbol{\}}\AgdaSpace{}%
\AgdaSymbol{→}\AgdaSpace{}%
\AgdaBound{x}\AgdaSpace{}%
\AgdaOperator{\AgdaDatatype{≡}}\AgdaSpace{}%
\AgdaBound{x}\<%
% \\
% %
% \\[\AgdaEmptyExtraSkip]%
% \>[0]\AgdaKeyword{open}\AgdaSpace{}%
% \AgdaKeyword{import}\AgdaSpace{}%
% \AgdaModule{Identity-Type}\AgdaSpace{}%
% \AgdaKeyword{renaming}\AgdaSpace{}%
% \AgdaSymbol{(}\AgdaOperator{\AgdaDatatype{\AgdaUnderscore{}≡\AgdaUnderscore{}}}\AgdaSpace{}%
% \AgdaSymbol{to}\AgdaSpace{}%
% \AgdaKeyword{infix}\AgdaSpace{}%
% \AgdaNumber{0}\AgdaSpace{}%
% \AgdaOperator{\AgdaDatatype{\AgdaUnderscore{}≡\AgdaUnderscore{}}}\AgdaSymbol{)}\AgdaSpace{}%
% \AgdaKeyword{public}\<%
% \\
% \>[0]\<%
\end{code}
\ccpad
Whenever we need to complete a proof by simply asserting that \ab{x} is \defn{definitionally equal} to itself, we invoke \aic{refl}. If we need to make explicit the implicit argument \ab x, then we use \aic{refl}~\{\ab x~=~\ab x\}.

\paragraph*{Assumed module contexts}
Before proceeding, a word about a special convention we adopt in the sequel is in order. To reduce reader strain, we will often omit easily inferred typing judgments which would normally appear in the list of parameters of a module or at the start of a type definition, and we sometimes make an announcement like the following (which applies to the present section):
\context{\{\ab{𝓤}~\as :~\ap{Universe}\} \{\ab{A}~\as :~\ab 𝓤\af ̇\}}
\paragraph*{Definitional equality is an equivalence}
The relation \ad{≡} just defined is naturally an equivalence relation, and the formal proof of this fact is trivial. Indeed, we don't need to prove reflexivity, since that is the defining property of \ad{≡}, and the proofs of symmetry and transitivity are also immediate.
\ccpad
\begin{code}%
\>[1]\AgdaFunction{≡-symmetric}\AgdaSpace{}%
\AgdaSymbol{:}\AgdaSpace{}%
\AgdaSymbol{(}\AgdaBound{x}\AgdaSpace{}%
\AgdaBound{y}\AgdaSpace{}%
\AgdaSymbol{:}\AgdaSpace{}%
\AgdaBound{A}\AgdaSymbol{)}\AgdaSpace{}%
\AgdaSymbol{→}\AgdaSpace{}%
\AgdaBound{x}\AgdaSpace{}%
\AgdaOperator{\AgdaDatatype{≡}}\AgdaSpace{}%
\AgdaBound{y}\AgdaSpace{}%
\AgdaSymbol{→}\AgdaSpace{}%
\AgdaBound{y}\AgdaSpace{}%
\AgdaOperator{\AgdaDatatype{≡}}\AgdaSpace{}%
\AgdaBound{x}\<%
\\
%
\>[1]\AgdaFunction{≡-symmetric}\AgdaSpace{}%
\AgdaSymbol{\AgdaUnderscore{}}\AgdaSpace{}%
\AgdaSymbol{\AgdaUnderscore{}}\AgdaSpace{}%
\AgdaInductiveConstructor{refl}\AgdaSpace{}%
\AgdaSymbol{=}\AgdaSpace{}%
\AgdaInductiveConstructor{refl}\<%
\\
%
\\[\AgdaEmptyExtraSkip]%
%
\>[1]\AgdaFunction{≡-sym}\AgdaSpace{}%
\AgdaSymbol{:}\AgdaSpace{}%
\AgdaSymbol{\{}\AgdaBound{x}\AgdaSpace{}%
\AgdaBound{y}\AgdaSpace{}%
\AgdaSymbol{:}\AgdaSpace{}%
\AgdaBound{A}\AgdaSymbol{\}}\AgdaSpace{}%
\AgdaSymbol{→}\AgdaSpace{}%
\AgdaBound{x}\AgdaSpace{}%
\AgdaOperator{\AgdaDatatype{≡}}\AgdaSpace{}%
\AgdaBound{y}\AgdaSpace{}%
\AgdaSymbol{→}\AgdaSpace{}%
\AgdaBound{y}\AgdaSpace{}%
\AgdaOperator{\AgdaDatatype{≡}}\AgdaSpace{}%
\AgdaBound{x}\<%
\\
%
\>[1]\AgdaFunction{≡-sym}\AgdaSpace{}%
\AgdaInductiveConstructor{refl}\AgdaSpace{}%
\AgdaSymbol{=}\AgdaSpace{}%
\AgdaInductiveConstructor{refl}\<%
\\
%
\\[\AgdaEmptyExtraSkip]%
%
\>[1]\AgdaFunction{≡-transitive}\AgdaSpace{}%
\AgdaSymbol{:}\AgdaSpace{}%
\AgdaSymbol{(}\AgdaBound{x}\AgdaSpace{}%
\AgdaBound{y}\AgdaSpace{}%
\AgdaBound{z}\AgdaSpace{}%
\AgdaSymbol{:}\AgdaSpace{}%
\AgdaBound{A}\AgdaSymbol{)}\AgdaSpace{}%
\AgdaSymbol{→}\AgdaSpace{}%
\AgdaBound{x}\AgdaSpace{}%
\AgdaOperator{\AgdaDatatype{≡}}\AgdaSpace{}%
\AgdaBound{y}\AgdaSpace{}%
\AgdaSymbol{→}\AgdaSpace{}%
\AgdaBound{y}\AgdaSpace{}%
\AgdaOperator{\AgdaDatatype{≡}}\AgdaSpace{}%
\AgdaBound{z}\AgdaSpace{}%
\AgdaSymbol{→}\AgdaSpace{}%
\AgdaBound{x}\AgdaSpace{}%
\AgdaOperator{\AgdaDatatype{≡}}\AgdaSpace{}%
\AgdaBound{z}\<%
\\
%
\>[1]\AgdaFunction{≡-transitive}\AgdaSpace{}%
\AgdaSymbol{\AgdaUnderscore{}}\AgdaSpace{}%
\AgdaSymbol{\AgdaUnderscore{}}\AgdaSpace{}%
\AgdaSymbol{\AgdaUnderscore{}}\AgdaSpace{}%
\AgdaInductiveConstructor{refl}\AgdaSpace{}%
\AgdaInductiveConstructor{refl}\AgdaSpace{}%
\AgdaSymbol{=}\AgdaSpace{}%
\AgdaInductiveConstructor{refl}\<%
\\
%
\\[\AgdaEmptyExtraSkip]%
%
\>[1]\AgdaFunction{≡-trans}\AgdaSpace{}%
\AgdaSymbol{:}\AgdaSpace{}%
\AgdaSymbol{\{}\AgdaBound{x}\AgdaSpace{}%
\AgdaBound{y}\AgdaSpace{}%
\AgdaBound{z}\AgdaSpace{}%
\AgdaSymbol{:}\AgdaSpace{}%
\AgdaBound{A}\AgdaSymbol{\}}\AgdaSpace{}%
\AgdaSymbol{→}\AgdaSpace{}%
\AgdaBound{x}\AgdaSpace{}%
\AgdaOperator{\AgdaDatatype{≡}}\AgdaSpace{}%
\AgdaBound{y}\AgdaSpace{}%
\AgdaSymbol{→}\AgdaSpace{}%
\AgdaBound{y}\AgdaSpace{}%
\AgdaOperator{\AgdaDatatype{≡}}\AgdaSpace{}%
\AgdaBound{z}\AgdaSpace{}%
\AgdaSymbol{→}\AgdaSpace{}%
\AgdaBound{x}\AgdaSpace{}%
\AgdaOperator{\AgdaDatatype{≡}}\AgdaSpace{}%
\AgdaBound{z}\<%
\\
%
\>[1]\AgdaFunction{≡-trans}\AgdaSpace{}%
\AgdaInductiveConstructor{refl}\AgdaSpace{}%
\AgdaInductiveConstructor{refl}\AgdaSpace{}%
\AgdaSymbol{=}\AgdaSpace{}%
\AgdaInductiveConstructor{refl}\<%
\end{code}
\ccpad
The only difference between \af{≡-symmetric} and \af{≡-sym} (resp., \af{≡-transitive} and \af{≡-trans}) is that \af{≡-sym} (resp., \af{≡-trans}) has fewer explicit arguments, which is sometimes convenient.

We prove that \ad ≡ obeys the substitution rule (subst) in the next section (see \af{ap}~\S\ref{transport}), but first we define some syntactic sugar that will make it easier to apply symmetry and transitivity of \ad ≡ in proofs.\footnote{%
\textbf{Unicode Hints} (\agdamode): \texttt{\textbackslash{}\^{}-\textbackslash{}\^{}1} ↝ \af{⁻¹}; \texttt{\textbackslash{}Mii\textbackslash{}Mid} ↝ \af{𝑖𝑑}; \texttt{\textbackslash{}.}↝ \af{∙}. In general, for information about a character, place the cursor on the character and type \texttt{M-x\ describe-char} (or \texttt{M-x\ h\ d\ c}).}
\ccpad
\begin{code}
% \>[0]\AgdaKeyword{module}\AgdaSpace{}%
% \AgdaModule{hide-sym-trans}\AgdaSpace{}%
% \AgdaSymbol{\{}\AgdaBound{𝓤}\AgdaSpace{}%
% \AgdaSymbol{:}\AgdaSpace{}%
% \AgdaPostulate{Universe}\AgdaSymbol{\}}\AgdaSpace{}%
% \AgdaKeyword{where}\<%
% \\
% %
% \\[\AgdaEmptyExtraSkip]%
% \>[0][@{}l@{\AgdaIndent{0}}]%
\>[1]\AgdaOperator{\AgdaFunction{\AgdaUnderscore{}⁻¹}}\AgdaSpace{}%
\AgdaSymbol{:}\AgdaSpace{}%
% \AgdaSymbol{\{}\AgdaBound{A}\AgdaSpace{}%
% \AgdaSymbol{:}\AgdaSpace{}%
% \AgdaBound{𝓤}\AgdaSpace{}%
% \AgdaOperator{\AgdaFunction{̇}}\AgdaSpace{}%
% \AgdaSymbol{\}}\AgdaSpace{}%
% \AgdaSymbol{→}\AgdaSpace{}%
\AgdaSymbol{\{}\AgdaBound{x}\AgdaSpace{}%
\AgdaBound{y}\AgdaSpace{}%
\AgdaSymbol{:}\AgdaSpace{}%
\AgdaBound{A}\AgdaSymbol{\}}\AgdaSpace{}%
\AgdaSymbol{→}\AgdaSpace{}%
\AgdaBound{x}\AgdaSpace{}%
\AgdaOperator{\AgdaDatatype{≡}}\AgdaSpace{}%
\AgdaBound{y}\AgdaSpace{}%
\AgdaSymbol{→}\AgdaSpace{}%
\AgdaBound{y}\AgdaSpace{}%
\AgdaOperator{\AgdaDatatype{≡}}\AgdaSpace{}%
\AgdaBound{x}\<%
\\
%
\>[1]\AgdaBound{p}\AgdaSpace{}%
\AgdaOperator{\AgdaFunction{⁻¹}}\AgdaSpace{}%
\AgdaSymbol{=}\AgdaSpace{}%
\AgdaFunction{≡-sym}\AgdaSpace{}%
\AgdaBound{p}\<%
\end{code}
\ccpad
If we have a proof \ab p \as :~\ab x \aod ≡ \ab y, and we need a proof of \ab y \aod ≡ \ab x, then instead of \af{≡-sym} \ab p we can use the more intuitive \ab p \af{⁻¹}. Similarly, the following syntactic sugar makes abundant appeals to transitivity easier to stomach.
\ccpad
\begin{code}%
\>[1]\AgdaOperator{\AgdaFunction{\AgdaUnderscore{}∙\AgdaUnderscore{}}}\AgdaSpace{}%
\AgdaSymbol{:}\AgdaSpace{}%
% \AgdaSymbol{\{}\AgdaBound{A}\AgdaSpace{}%
% \AgdaSymbol{:}\AgdaSpace{}%
% \AgdaBound{𝓤}\AgdaSpace{}%
% \AgdaOperator{\AgdaFunction{̇}}\AgdaSpace{}%
% \AgdaSymbol{\}}\AgdaSpace{}%
\AgdaSymbol{\{}\AgdaBound{x}\AgdaSpace{}%
\AgdaBound{y}\AgdaSpace{}%
\AgdaBound{z}\AgdaSpace{}%
\AgdaSymbol{:}\AgdaSpace{}%
\AgdaBound{A}\AgdaSymbol{\}}\AgdaSpace{}%
\AgdaSymbol{→}\AgdaSpace{}%
\AgdaBound{x}\AgdaSpace{}%
\AgdaOperator{\AgdaDatatype{≡}}\AgdaSpace{}%
\AgdaBound{y}\AgdaSpace{}%
\AgdaSymbol{→}\AgdaSpace{}%
\AgdaBound{y}\AgdaSpace{}%
\AgdaOperator{\AgdaDatatype{≡}}\AgdaSpace{}%
\AgdaBound{z}\AgdaSpace{}%
\AgdaSymbol{→}\AgdaSpace{}%
\AgdaBound{x}\AgdaSpace{}%
\AgdaOperator{\AgdaDatatype{≡}}\AgdaSpace{}%
\AgdaBound{z}\<%
\\
%
\>[1]\AgdaBound{p}\AgdaSpace{}%
\AgdaOperator{\AgdaFunction{∙}}\AgdaSpace{}%
\AgdaBound{q}\AgdaSpace{}%
\AgdaSymbol{=}\AgdaSpace{}%
\AgdaFunction{≡-trans}\AgdaSpace{}%
\AgdaBound{p}\AgdaSpace{}%
\AgdaBound{q}\<%
% \\
% %
% \\[\AgdaEmptyExtraSkip]%
% \>[0]\AgdaKeyword{open}\AgdaSpace{}%
% \AgdaKeyword{import}\AgdaSpace{}%
% \AgdaModule{MGS-MLTT}\AgdaSpace{}%
% \AgdaKeyword{using}\AgdaSpace{}%
% \AgdaSymbol{(}\AgdaOperator{\AgdaFunction{\AgdaUnderscore{}⁻¹}}\AgdaSymbol{;}\AgdaSpace{}%
% \AgdaOperator{\AgdaFunction{\AgdaUnderscore{}∙\AgdaUnderscore{}}}\AgdaSymbol{)}\AgdaSpace{}%
% \AgdaKeyword{public}\<%
\end{code}

\subsubsection{Transport}\label{transport}

Alonzo Church characterized equality by declaring two things equal if and only if no property (predicate) can distinguish them (see \cite{Church:1940}). In other terms, \ab x and \ab y are equal if and only if for all \ab{P} we have \ab{P} \ab x \as → \ab P \ab y. One direction of this implication is sometimes called \emph{substitution} or \emph{transport} or \emph{transport along an identity}. It asserts the following: if two objects are equal and one of them satisfies a given predicate, then so does the other. A type representing this notion is defined, along with the (polymorphic) identity function, in the \am{MGS-MLTT} module of \typtop as follows.\footnote{Including every line of code of the \agdaualib in this paper would result in an unbearable reading experience. We include all significant sections of code from the first 13 modules, but we omit lines indicating that redundant definitions of functions (e.g., \af{transport} and \af{ap}) occur inside named ``hidden'' modules. We also omit lines importing the original definitions of such duplicate definitions from \typtop library.}
\ccpad
\begin{code}%
\>[1]\AgdaFunction{𝑖𝑑}\AgdaSpace{}%
\AgdaSymbol{:}\AgdaSpace{}%
\AgdaSymbol{\{}\AgdaBound{𝓤}\AgdaSpace{}%
\AgdaSymbol{:}\AgdaSpace{}%
\AgdaPostulate{Universe}\AgdaSymbol{\}}\AgdaSpace{}%
\AgdaSymbol{(}\AgdaBound{A}\AgdaSpace{}%
\AgdaSymbol{:}\AgdaSpace{}%
\AgdaBound{𝓤}\AgdaSpace{}%
\AgdaOperator{\AgdaFunction{̇}}\AgdaSpace{}%
\AgdaSymbol{)}\AgdaSpace{}%
\AgdaSymbol{→}\AgdaSpace{}%
\AgdaBound{A}\AgdaSpace{}%
\AgdaSymbol{→}\AgdaSpace{}%
\AgdaBound{A}\<%
\\
%
\>[1]\AgdaFunction{𝑖𝑑}\AgdaSpace{}%
\AgdaBound{A}\AgdaSpace{}%
\AgdaSymbol{=}\AgdaSpace{}%
\AgdaSymbol{λ}\AgdaSpace{}%
\AgdaBound{x}\AgdaSpace{}%
\AgdaSymbol{→}\AgdaSpace{}%
\AgdaBound{x}\<%
\end{code}
\scpad
\begin{code}
\>[1]\AgdaFunction{transport}\AgdaSpace{}%
\AgdaSymbol{:}\AgdaSpace{}%
\AgdaSymbol{\{}\AgdaBound{A}\AgdaSpace{}%
\AgdaSymbol{:}\AgdaSpace{}%
\AgdaBound{𝓤}\AgdaSpace{}%
\AgdaOperator{\AgdaFunction{̇}}\AgdaSpace{}%
\AgdaSymbol{\}}\AgdaSpace{}%
\AgdaSymbol{(}\AgdaBound{B}\AgdaSpace{}%
\AgdaSymbol{:}\AgdaSpace{}%
\AgdaBound{A}\AgdaSpace{}%
\AgdaSymbol{→}\AgdaSpace{}%
\AgdaBound{𝓦}\AgdaSpace{}%
\AgdaOperator{\AgdaFunction{̇}}\AgdaSpace{}%
\AgdaSymbol{)}\AgdaSpace{}%
\AgdaSymbol{\{}\AgdaBound{x}\AgdaSpace{}%
\AgdaBound{y}\AgdaSpace{}%
\AgdaSymbol{:}\AgdaSpace{}%
\AgdaBound{A}\AgdaSymbol{\}}\AgdaSpace{}%
\AgdaSymbol{→}\AgdaSpace{}%
\AgdaBound{x}\AgdaSpace{}%
\AgdaOperator{\AgdaDatatype{≡}}\AgdaSpace{}%
\AgdaBound{y}\AgdaSpace{}%
\AgdaSymbol{→}\AgdaSpace{}%
\AgdaBound{B}\AgdaSpace{}%
\AgdaBound{x}\AgdaSpace{}%
\AgdaSymbol{→}\AgdaSpace{}%
\AgdaBound{B}\AgdaSpace{}%
\AgdaBound{y}\<%
\\
%
\>[1]\AgdaFunction{transport}\AgdaSpace{}%
\AgdaBound{B}\AgdaSpace{}%
\AgdaSymbol{(}\AgdaInductiveConstructor{refl}\AgdaSpace{}%
\AgdaSymbol{\{}\AgdaArgument{x}\AgdaSpace{}%
\AgdaSymbol{=}\AgdaSpace{}%
\AgdaBound{x}\AgdaSymbol{\})}\AgdaSpace{}%
\AgdaSymbol{=}\AgdaSpace{}%
\AgdaFunction{𝑖𝑑}\AgdaSpace{}%
\AgdaSymbol{(}\AgdaBound{B}\AgdaSpace{}%
\AgdaBound{x}\AgdaSymbol{)}\<%
\end{code}
\scpad
% See~\cite{MHE} for a discussion of transport.\footnote{cf.~transport in \href{https://github.com/HoTT/HoTT-Agda/blob/master/core/lib/Base.agda}{HoTT-Agda}: \url{https://github.com/HoTT/HoTT-Agda/blob/master/core/lib/Base.agda}.}

A function is well-defined if and only if it maps equivalent elements to a single element and we often use this nature of functions in Agda proofs.  It is equivalent to the substitution rule (subst) we defined in the last section. If we have a function \ab{f} \as : \ab A \as → \ab B, two elements \ab{x} \ab{y} \as : \ab A of the domain, and an identity proof \ab{p} \as : \ab x \ad ≡ \ab{y}, then we obtain a proof of \ab{f} \ab x \ad ≡ \ab f \ab{y} by simply applying the \af{ap} function like so, \af{ap} \ab f \ab p \as : \ab f \ab x \ad ≡ \ab f \ab{y}. Escardó defines \af{ap} in \typtop as follows.
\ccpad
\begin{code}%
% \>[0]\AgdaKeyword{module}\AgdaSpace{}%
% \AgdaModule{hide-ap}%
% \>[16]\AgdaSymbol{\{}\AgdaBound{𝓤}\AgdaSpace{}%
% \AgdaSymbol{:}\AgdaSpace{}%
% \AgdaPostulate{Universe}\AgdaSymbol{\}}\AgdaSpace{}%
% \AgdaKeyword{where}\<%
% \\
% %
% \\[\AgdaEmptyExtraSkip]%
% \>[0][@{}l@{\AgdaIndent{0}}]%
\>[1]\AgdaFunction{ap}\AgdaSpace{}%
\AgdaSymbol{:}\AgdaSpace{}%
\AgdaSymbol{\{}\AgdaBound{A}\AgdaSpace{}%
\AgdaSymbol{:}\AgdaSpace{}%
\AgdaBound{𝓤}\AgdaSpace{}%
\AgdaOperator{\AgdaFunction{̇}}\AgdaSymbol{\}\{}\AgdaBound{B}\AgdaSpace{}%
\AgdaSymbol{:}\AgdaSpace{}%
\AgdaGeneralizable{𝓥}\AgdaSpace{}%
\AgdaOperator{\AgdaFunction{̇}}\AgdaSymbol{\}}
\AgdaSymbol{(}\AgdaBound{f}\AgdaSpace{}%
\AgdaSymbol{:}\AgdaSpace{}%
\AgdaBound{A}\AgdaSpace{}%
\AgdaSymbol{→}\AgdaSpace{}%
\AgdaBound{B}\AgdaSymbol{)\{}\AgdaBound{a}\AgdaSpace{}%
\AgdaBound{b}\AgdaSpace{}%
\AgdaSymbol{:}\AgdaSpace{}%
\AgdaBound{A}\AgdaSymbol{\}}\AgdaSpace{}%
\AgdaSymbol{→}\AgdaSpace{}%
\AgdaBound{a}\AgdaSpace{}%
\AgdaOperator{\AgdaDatatype{≡}}\AgdaSpace{}%
\AgdaBound{b}\AgdaSpace{}%
\AgdaSymbol{→}\AgdaSpace{}%
\AgdaBound{f}\AgdaSpace{}%
\AgdaBound{a}\AgdaSpace{}%
\AgdaOperator{\AgdaDatatype{≡}}\AgdaSpace{}%
\AgdaBound{f}\AgdaSpace{}%
\AgdaBound{b}\<%
\\
%
\>[1]\AgdaFunction{ap}\AgdaSpace{}%
\AgdaBound{f}\AgdaSpace{}%
\AgdaSymbol{\{}\AgdaBound{a}\AgdaSymbol{\}}\AgdaSpace{}%
\AgdaBound{p}\AgdaSpace{}%
\AgdaSymbol{=}\AgdaSpace{}%
\AgdaFunction{transport}\AgdaSpace{}%
\AgdaSymbol{(λ}\AgdaSpace{}%
\AgdaBound{-}\AgdaSpace{}%
\AgdaSymbol{→}\AgdaSpace{}%
\AgdaBound{f}\AgdaSpace{}%
\AgdaBound{a}\AgdaSpace{}%
\AgdaOperator{\AgdaDatatype{≡}}\AgdaSpace{}%
\AgdaBound{f}\AgdaSpace{}%
\AgdaBound{-}\AgdaSymbol{)}\AgdaSpace{}%
\AgdaBound{p}\AgdaSpace{}%
\AgdaSymbol{(}\AgdaInductiveConstructor{refl}\AgdaSpace{}%
\AgdaSymbol{\{}\AgdaArgument{x}\AgdaSpace{}%
\AgdaSymbol{=}\AgdaSpace{}%
\AgdaBound{f}\AgdaSpace{}%
\AgdaBound{a}\AgdaSymbol{\})}\<%
% \\
% %
% \\[\AgdaEmptyExtraSkip]%
% \>[0]\AgdaKeyword{open}\AgdaSpace{}%
% \AgdaKeyword{import}\AgdaSpace{}%
% \AgdaModule{MGS-MLTT}\AgdaSpace{}%
% \AgdaKeyword{using}\AgdaSpace{}%
% \AgdaSymbol{(}\AgdaFunction{ap}\AgdaSymbol{)}\AgdaSpace{}%
% \AgdaKeyword{public}\<%
\end{code}
\ccpad
This establishes that our definitional equality satisfies the substitution rule (subst).

Here's a useful variation of \af{ap} that we borrow from the \texttt{Relation/Binary/Core.agda} module of the \agdastdlib (transcribed into \typtop/\ualib).
\ccpad
\begin{code}%
\>[0]\AgdaFunction{cong-app}\AgdaSpace{}%
\AgdaSymbol{:}%
\>[101I]\AgdaSymbol{\{}\AgdaBound{A}\AgdaSpace{}%
\AgdaSymbol{:}\AgdaSpace{}%
\AgdaBound{𝓤}\AgdaSpace{}%
\AgdaOperator{\AgdaFunction{̇}}\AgdaSymbol{\}\{}\AgdaBound{B}\AgdaSpace{}%
\AgdaSymbol{:}\AgdaSpace{}%
\AgdaBound{A}\AgdaSpace{}%
\AgdaSymbol{→}\AgdaSpace{}%
\AgdaBound{𝓦}\AgdaSpace{}%
\AgdaOperator{\AgdaFunction{̇}}\AgdaSymbol{\}\{}\AgdaBound{f}\AgdaSpace{}%
\AgdaBound{g}\AgdaSpace{}%
\AgdaSymbol{:}\AgdaSpace{}%
\AgdaFunction{Π}\AgdaSpace{}%
\AgdaBound{B}\AgdaSymbol{\}}\AgdaSpace{}%
\AgdaSymbol{→}\AgdaSpace{}%
\AgdaBound{f}\AgdaSpace{}%
% \\
% \>[0][@{}l@{\AgdaIndent{0}}]%
% \>[1]\AgdaSymbol{→}%
% \>[.][@{}l@{}]\<[101I]%
% \>[12]\AgdaBound{f}\AgdaSpace{}%
\AgdaOperator{\AgdaDatatype{≡}}\AgdaSpace{}%
\AgdaBound{g}\AgdaSpace{}%
\AgdaSymbol{→}\AgdaSpace{}%
\AgdaSymbol{(}\AgdaBound{a}\AgdaSpace{}%
\AgdaSymbol{:}\AgdaSpace{}%
\AgdaBound{A}\AgdaSymbol{)}\AgdaSpace{}%
\AgdaSymbol{→}\AgdaSpace{}%
\AgdaBound{f}\AgdaSpace{}%
\AgdaBound{a}\AgdaSpace{}%
\AgdaOperator{\AgdaDatatype{≡}}\AgdaSpace{}%
\AgdaBound{g}\AgdaSpace{}%
\AgdaBound{a}\<%
\\
%
\>[0]\AgdaFunction{cong-app}\AgdaSpace{}%
\AgdaInductiveConstructor{refl}\AgdaSpace{}%
\AgdaSymbol{\AgdaUnderscore{}}\AgdaSpace{}%
\AgdaSymbol{=}\AgdaSpace{}%
\AgdaInductiveConstructor{refl}\<%
\end{code}

% \subsubsection{≡-intro and ≡-elim for nondependent pairs}\label{intro-and-elim-for-nondependent-pairs}

% We conclude our presentation of the \ualibhtml{Overture.Equality} module with some occasionally useful introduction and elimination rules for the equality relation on (nondependent) pair types.
% \ccpad
% \begin{code}%
% \>[0]\AgdaFunction{≡-elim-left}\AgdaSpace{}%
% \AgdaSymbol{:}%
% \>[102I]\AgdaSymbol{\{}\AgdaBound{A₁}\AgdaSpace{}%
% \AgdaBound{A₂}\AgdaSpace{}%
% \AgdaSymbol{:}\AgdaSpace{}%
% \AgdaBound{𝓤}\AgdaSpace{}%
% \AgdaOperator{\AgdaFunction{̇}}\AgdaSymbol{\}\{}\AgdaBound{B₁}\AgdaSpace{}%
% \AgdaBound{B₂}\AgdaSpace{}%
% \AgdaSymbol{:}\AgdaSpace{}%
% \AgdaBound{𝓦}\AgdaSpace{}%
% \AgdaOperator{\AgdaFunction{̇}}\AgdaSymbol{\}}\AgdaSpace{}%
% \AgdaSymbol{→}\AgdaSpace{}%
% \AgdaSymbol{(}\AgdaBound{A₁}\AgdaSpace{}%
% % \\
% % \>[0][@{}l@{\AgdaIndent{0}}]%
% % \>[1]\AgdaSymbol{→}%
% % \>[.][@{}l@{}]\<[102I]%
% % \>[15]\AgdaSymbol{(}\AgdaBound{A₁}\AgdaSpace{}%
% \AgdaOperator{\AgdaInductiveConstructor{,}}\AgdaSpace{}%
% \AgdaBound{B₁}\AgdaSymbol{)}\AgdaSpace{}%
% \AgdaOperator{\AgdaDatatype{≡}}\AgdaSpace{}%
% \AgdaSymbol{(}\AgdaBound{A₂}\AgdaSpace{}%
% \AgdaOperator{\AgdaInductiveConstructor{,}}\AgdaSpace{}%
% \AgdaBound{B₂}\AgdaSymbol{)}\AgdaSpace{}%
% \AgdaSymbol{→}\AgdaSpace{}%
% \AgdaBound{A₁}\AgdaSpace{}%
% \AgdaOperator{\AgdaDatatype{≡}}\AgdaSpace{}%
% \AgdaBound{A₂}\<%
% \\
% %
% \>[0]\AgdaFunction{≡-elim-left}\AgdaSpace{}%
% \AgdaBound{e}\AgdaSpace{}%
% \AgdaSymbol{=}\AgdaSpace{}%
% \AgdaFunction{ap}\AgdaSpace{}%
% \AgdaFunction{pr₁}\AgdaSpace{}%
% \AgdaBound{e}\<%
% \end{code}
% \scpad
% \begin{code}%
% \>[0]\AgdaFunction{≡-elim-right}\AgdaSpace{}%
% \AgdaSymbol{:}%
% \>[105I]\AgdaSymbol{\{}\AgdaBound{A₁}\AgdaSpace{}%
% \AgdaBound{A₂}\AgdaSpace{}%
% \AgdaSymbol{:}\AgdaSpace{}%
% \AgdaBound{𝓤}\AgdaSpace{}%
% \AgdaOperator{\AgdaFunction{̇}}\AgdaSymbol{\}\{}\AgdaBound{B₁}\AgdaSpace{}%
% \AgdaBound{B₂}\AgdaSpace{}%
% \AgdaSymbol{:}\AgdaSpace{}%
% \AgdaBound{𝓦}\AgdaSpace{}%
% \AgdaOperator{\AgdaFunction{̇}}\AgdaSymbol{\}}\AgdaSpace{}%
% \AgdaSymbol{→}\AgdaSpace{}%
% \AgdaSymbol{(}\AgdaBound{A₁}\AgdaSpace{}%
% % \\
% % \>[0][@{}l@{\AgdaIndent{0}}]%
% % \>[1]\AgdaSymbol{→}%
% % \>[.][@{}l@{}]\<[105I]%
% % \>[16]\AgdaSymbol{(}\AgdaBound{A₁}\AgdaSpace{}%
% \AgdaOperator{\AgdaInductiveConstructor{,}}\AgdaSpace{}%
% \AgdaBound{B₁}\AgdaSymbol{)}\AgdaSpace{}%
% \AgdaOperator{\AgdaDatatype{≡}}\AgdaSpace{}%
% \AgdaSymbol{(}\AgdaBound{A₂}\AgdaSpace{}%
% \AgdaOperator{\AgdaInductiveConstructor{,}}\AgdaSpace{}%
% \AgdaBound{B₂}\AgdaSymbol{)}\AgdaSpace{}%
% \AgdaSymbol{→}\AgdaSpace{}%
% \AgdaBound{B₁}\AgdaSpace{}%
% \AgdaOperator{\AgdaDatatype{≡}}\AgdaSpace{}%
% \AgdaBound{B₂}\<%
% \\
% %
% \>[0]\AgdaFunction{≡-elim-right}\AgdaSpace{}%
% \AgdaBound{e}\AgdaSpace{}%
% \AgdaSymbol{=}\AgdaSpace{}%
% \AgdaFunction{ap}\AgdaSpace{}%
% \AgdaFunction{pr₂}\AgdaSpace{}%
% \AgdaBound{e}\<%
% \end{code}
% \scpad
% \begin{code}
% \>[0]\AgdaFunction{≡-×-intro}\AgdaSpace{}%
% \AgdaSymbol{:}%
% \>[106I]\AgdaSymbol{\{}\AgdaBound{A₁}\AgdaSpace{}%
% \AgdaBound{A₂}\AgdaSpace{}%
% \AgdaSymbol{:}\AgdaSpace{}%
% \AgdaBound{𝓤}\AgdaSpace{}%
% \AgdaOperator{\AgdaFunction{̇}}\AgdaSymbol{\}}\AgdaSpace{}%
% \AgdaSymbol{\{}\AgdaBound{B₁}\AgdaSpace{}%
% \AgdaBound{B₂}\AgdaSpace{}%
% \AgdaSymbol{:}\AgdaSpace{}%
% \AgdaBound{𝓦}\AgdaSpace{}%
% \AgdaOperator{\AgdaFunction{̇}}\AgdaSymbol{\}}\AgdaSpace{}%
% % \\
% % \>[0][@{}l@{\AgdaIndent{0}}]%
% % \>[1]\AgdaSymbol{→}%
% % \>[.][@{}l@{}]\<[106I]%
% % \>[13]\AgdaBound{A₁}\AgdaSpace{}%
% \AgdaSymbol{→}\AgdaSpace{}%
% \AgdaBound{A₁}\AgdaSpace{}%
% \AgdaOperator{\AgdaDatatype{≡}}\AgdaSpace{}%
% \AgdaBound{A₂}\AgdaSpace{}%
% \AgdaSymbol{→}\AgdaSpace{}%
% \AgdaBound{B₁}\AgdaSpace{}%
% \AgdaOperator{\AgdaDatatype{≡}}\AgdaSpace{}%
% \AgdaBound{B₂}\AgdaSpace{}%
% \AgdaSymbol{→}\AgdaSpace{}%
% \AgdaSymbol{(}\AgdaBound{A₁}\AgdaSpace{}%
% \AgdaOperator{\AgdaInductiveConstructor{,}}\AgdaSpace{}%
% \AgdaBound{B₁}\AgdaSymbol{)}\AgdaSpace{}%
% \AgdaOperator{\AgdaDatatype{≡}}\AgdaSpace{}%
% \AgdaSymbol{(}\AgdaBound{A₂}\AgdaSpace{}%
% \AgdaOperator{\AgdaInductiveConstructor{,}}\AgdaSpace{}%
% \AgdaBound{B₂}\AgdaSymbol{)}\<%
% \\
% %
% \>[0]\AgdaFunction{≡-×-intro}\AgdaSpace{}%
% \AgdaInductiveConstructor{refl}\AgdaSpace{}%
% \AgdaInductiveConstructor{refl}\AgdaSpace{}%
% \AgdaSymbol{=}\AgdaSpace{}%
% \AgdaInductiveConstructor{refl}\<%
% \end{code}
% \scpad
% \begin{code}
% \>[1]\AgdaFunction{≡-×-int}\AgdaSpace{}%
% \AgdaSymbol{:}%
% \>[99I]\AgdaSymbol{\{}\AgdaBound{A}\AgdaSpace{}%
% \AgdaSymbol{:}\AgdaSpace{}%
% \AgdaBound{𝓤}\AgdaSpace{}%
% \AgdaOperator{\AgdaFunction{̇}}\AgdaSymbol{\}\{}\AgdaBound{B}\AgdaSpace{}%
% \AgdaSymbol{:}\AgdaSpace{}%
% \AgdaBound{𝓦}\AgdaSpace{}%
% \AgdaOperator{\AgdaFunction{̇}}\AgdaSymbol{\}\{}\AgdaBound{a}\AgdaSpace{}%
% \AgdaBound{a'}\AgdaSpace{}%
% \AgdaSymbol{:}\AgdaSpace{}%
% \AgdaBound{A}\AgdaSymbol{\}\{}\AgdaBound{b}\AgdaSpace{}%
% \AgdaBound{b'}\AgdaSpace{}%
% \AgdaSymbol{:}\AgdaSpace{}%
% \AgdaBound{B}\AgdaSymbol{\}}\AgdaSpace{}%
% \AgdaSymbol{→}\AgdaSpace{}%
% \AgdaBound{a}\AgdaSpace{}%
% % \\
% % \>[1][@{}l@{\AgdaIndent{0}}]%
% % \>[2]\AgdaSymbol{→}%
% % \>[.][@{}l@{}]\<[99I]%
% % \>[11]\AgdaBound{a}\AgdaSpace{}%
% \AgdaOperator{\AgdaDatatype{≡}}\AgdaSpace{}%
% \AgdaBound{a'}%
% \>[19]\AgdaSymbol{→}%
% \>[22]\AgdaBound{b}\AgdaSpace{}%
% \AgdaOperator{\AgdaDatatype{≡}}\AgdaSpace{}%
% \AgdaBound{b'}\AgdaSpace{}%
% \AgdaSymbol{→}\AgdaSpace{}%
% \AgdaSymbol{(}\AgdaBound{a}\AgdaSpace{}%
% \AgdaOperator{\AgdaInductiveConstructor{,}}\AgdaSpace{}%
% \AgdaBound{b}\AgdaSymbol{)}\AgdaSpace{}%
% \AgdaOperator{\AgdaDatatype{≡}}\AgdaSpace{}%
% \AgdaSymbol{(}\AgdaBound{a'}\AgdaSpace{}%
% \AgdaOperator{\AgdaInductiveConstructor{,}}\AgdaSpace{}%
% \AgdaBound{b'}\AgdaSymbol{)}\<%
% \\
% %
% \>[1]\AgdaFunction{≡-×-int}\AgdaSpace{}%
% \AgdaInductiveConstructor{refl}\AgdaSpace{}%
% \AgdaInductiveConstructor{refl}\AgdaSpace{}%
% \AgdaSymbol{=}\AgdaSpace{}%
% \AgdaInductiveConstructor{refl}\<%
% \end{code}

% \begin{code}%
% \>[0]\AgdaFunction{≡-×-int}\AgdaSpace{}%
% \AgdaSymbol{:}%
% \[111I]\AgdaSymbol{\{}\AgdaBound{A}\AgdaSpace{}%
% \AgdaSymbol{:}\AgdaSpace{}%
% \AgdaBound{𝓤}\AgdaSpace{}%
% \AgdaOperator{\AgdaFunction{̇}}\AgdaSymbol{\}\{}\AgdaBound{B}\AgdaSpace{}%
% \AgdaSymbol{:}\AgdaSpace{}%
% \AgdaBound{𝓦}\AgdaSpace{}%
% \AgdaOperator{\AgdaFunction{̇}}\AgdaSymbol{\}\{}\AgdaBound{a}\AgdaSpace{}%
% \AgdaBound{u}\AgdaSpace{}%
% \AgdaSymbol{:}\AgdaSpace{}%
% \AgdaBound{A}\AgdaSymbol{\}\{}\AgdaBound{b}\AgdaSpace{}%
% \AgdaBound{v}\AgdaSpace{}%
% \AgdaSymbol{:}\AgdaSpace{}%
% \AgdaBound{B}\AgdaSymbol{\}}\<%
% \\
% \\
% \>[0][@{}l@{\AgdaIndent{0}}]%
% \>[1]\AgdaSymbol{→}%
% \>[.][@{}l@{}]\<[111I]%
% \>[11]\AgdaBound{a}\AgdaSpace{}%
% \AgdaOperator{\AgdaDatatype{≡}}\AgdaSpace{}%
% \AgdaBound{u}\AgdaSpace{}%
% \AgdaSymbol{→}\AgdaSpace{}%
% \AgdaBound{b}\AgdaSpace{}%
% \AgdaOperator{\AgdaDatatype{≡}}\AgdaSpace{}%
% \AgdaBound{v}\<%
% \\
% %
% \>[11]\AgdaComment{---------------------------}\<%
% \\
% %
% \>[1]\AgdaSymbol{→}%
% \>[11]\AgdaSymbol{(}\AgdaBound{a}\AgdaSpace{}%
% \AgdaOperator{\AgdaInductiveConstructor{,}}\AgdaSpace{}%
% \AgdaBound{b}\AgdaSymbol{)}\AgdaSpace{}%
% \AgdaOperator{\AgdaDatatype{≡}}\AgdaSpace{}%
% \AgdaSymbol{(}\AgdaBound{u}\AgdaSpace{}%
% \AgdaOperator{\AgdaInductiveConstructor{,}}\AgdaSpace{}%
% \AgdaBound{v}\AgdaSymbol{)}\<%
% \\
% %
% \\[\AgdaEmptyExtraSkip]%
% %
% \>[0]\AgdaFunction{≡-×-int}\AgdaSpace{}%
% \AgdaInductiveConstructor{refl}\AgdaSpace{}%
% \AgdaInductiveConstructor{refl}\AgdaSpace{}%
% \AgdaSymbol{=}\AgdaSpace{}%
% \AgdaInductiveConstructor{refl}\<%
% \end{code}


\subsection{Function extensionality}\label{function-extensionality}\firstsentence{Overture}{Extensionality}
% : types for postulating function extensionality
% \context{\{\ab 𝓤 \ab 𝓦 \as : \ap{Universe}\}}
% -*- TeX-master: "ualib-part1.tex" -*-
%%% Local Variables: 
%%% mode: latex
%%% TeX-engine: 'xetex
%%% End:
\subsubsection{Background and motivation}\label{background-and-motivation}
This brief introduction to \emph{function extensionality} is intended for novices. Those already familiar with the concept might wish to skip to the next subsection.

What does it mean to say that two functions \(f\ g \colon X → Y\) are equal? Suppose $f$ and \(g\) are defined on \(X = ℤ\) (the integers) as follows: \(f x := x + 2\) and \(g x := ((2 * x) - 8)/2 + 6\). Should we call \(f\) and \(g\) equal?  Are they the ``same'' function?  What does that even mean?

It's hard to resist the urge to reduce \(g\) to \(x + 2\) and proclaim that \(f\) and \(g\) are equal. Indeed, this is often an acceptable answer and the discussion normally ends there.  In the science of computing, however, more attention is paid to equality, and with good reason.

We can probably all agree that the functions \(f\) and \(g\) above, while not syntactically equal, do produce the same output when given the same input so it seems fine to think of the functions as the same, for all intents and purposes. But we should ask ourselves at what point do we notice or care about the difference in the way functions are defined? What if we had started out this discussion with two functions \(f\) and \(g\) both of which take a list as argument and produce as output a correctly sorted version of the input list?  Suppose \(f\) is defined using the \href{https://en.wikipedia.org/wiki/Merge_sort}{merge sort} algorithm, while \(g\) uses \href{https://en.wikipedia.org/wiki/Quicksort}{quick sort}. Probably few of us would call \(f\) and \(g\) the ``same'' in this case.

In the examples above, it is common to say that the two functions are \href{https://en.wikipedia.org/wiki/Extensionality}{\defn{extensionally equal}}, since they produce the same \emph{external} output when given the same input, but they are not \href{https://en.wikipedia.org/wiki/Intension}{\defn{intensionally equal}}, since their \emph{internal} definitions differ. In the next subsection we describe types that manifest this idea of \emph{extensional equality of functions}, or \emph{function extensionality}.\footnote{Most of these types are already defined in \typtop, so the \ualib imports the definitions from there; as usual, we redefine some of these types here for the purpose of explication.}

\subsubsection{Definition of function extensionality}\label{sec:defin-funct-extens}

As explained above, a natural notion of function equality is defined as follows: \(f\) and \(g\) are said to be \emph{pointwise equal} provided \(∀ x → f x ≡ g x\). Here is how this is expressed in type theory (e.g., in \typtop).
\ccpad
\begin{code}%
\>[1]\AgdaOperator{\AgdaFunction{\AgdaUnderscore{}∼\AgdaUnderscore{}}}\AgdaSpace{}%
\AgdaSymbol{:}\AgdaSpace{}%
\AgdaSymbol{\{}\AgdaBound{A}\AgdaSpace{}%
\AgdaSymbol{:}\AgdaSpace{}%
\AgdaBound{𝓤}\AgdaSpace{}%
\AgdaOperator{\AgdaFunction{̇}}\AgdaSpace{}%
\AgdaSymbol{\}}\AgdaSpace{}%
\AgdaSymbol{\{}\AgdaBound{B}\AgdaSpace{}%
\AgdaSymbol{:}\AgdaSpace{}%
\AgdaBound{A}\AgdaSpace{}%
\AgdaSymbol{→}\AgdaSpace{}%
\AgdaBound{𝓦}\AgdaSpace{}%
\AgdaOperator{\AgdaFunction{̇}}\AgdaSpace{}%
\AgdaSymbol{\}}\AgdaSpace{}%
\AgdaSymbol{→}\AgdaSpace{}%
\AgdaFunction{Π}\AgdaSpace{}%
\AgdaBound{B}\AgdaSpace{}%
\AgdaSymbol{→}\AgdaSpace{}%
\AgdaFunction{Π}\AgdaSpace{}%
\AgdaBound{B}\AgdaSpace{}%
\AgdaSymbol{→}\AgdaSpace{}%
\AgdaBound{𝓤}\AgdaSpace{}%
\AgdaOperator{\AgdaPrimitive{⊔}}\AgdaSpace{}%
\AgdaBound{𝓦}\AgdaSpace{}%
\AgdaOperator{\AgdaFunction{̇}}\<%
\\
%
\>[1]\AgdaBound{f}\AgdaSpace{}%
\AgdaOperator{\AgdaFunction{∼}}\AgdaSpace{}%
\AgdaBound{g}\AgdaSpace{}%
\AgdaSymbol{=}\AgdaSpace{}%
\AgdaSymbol{∀}\AgdaSpace{}%
\AgdaBound{x}\AgdaSpace{}%
\AgdaSymbol{→}\AgdaSpace{}%
\AgdaBound{f}\AgdaSpace{}%
\AgdaBound{x}\AgdaSpace{}%
\AgdaOperator{\AgdaDatatype{≡}}\AgdaSpace{}%
\AgdaBound{g}\AgdaSpace{}%
\AgdaBound{x}\<%
\end{code}
\ccpad
\defn{Function extensionality} is the principle that pointwise equal functions are \defn{definitionally equal}; that is, \(\forall\ f\ g\ (f ∼ g → f ≡ g)\). In type theory we represent this principle by \af{funext} (for nondependent functions) and \af{dfunext} (for dependent functions), which are defined as follows.
\ccpad
\begin{code}%
\>[1]\AgdaFunction{funext}\AgdaSpace{}%
\AgdaSymbol{:}\AgdaSpace{}%
\AgdaSymbol{∀}\AgdaSpace{}%
\AgdaBound{𝓤}\AgdaSpace{}%
\AgdaBound{𝓦}%
\>[17]\AgdaSymbol{→}\AgdaSpace{}%
\as (\AgdaBound{𝓤}\AgdaSpace{}%
\AgdaOperator{\AgdaPrimitive{⊔}}\AgdaSpace{}%
\AgdaBound{𝓦}\as )\AgdaSpace{}%
\AgdaOperator{\AgdaPrimitive{⁺}}\AgdaSpace{}%
\AgdaOperator{\AgdaFunction{̇}}\<%
\\
%
\>[1]\AgdaFunction{funext}\AgdaSpace{}%
\AgdaBound{𝓤}\AgdaSpace{}%
\AgdaBound{𝓦}\AgdaSpace{}%
\AgdaSymbol{=}\AgdaSpace{}%
\AgdaSymbol{\{}\AgdaBound{A}\AgdaSpace{}%
\AgdaSymbol{:}\AgdaSpace{}%
\AgdaBound{𝓤}\AgdaSpace{}%
\AgdaOperator{\AgdaFunction{̇}}\AgdaSpace{}%
\AgdaSymbol{\}}\AgdaSpace{}%
\AgdaSymbol{\{}\AgdaBound{B}\AgdaSpace{}%
\AgdaSymbol{:}\AgdaSpace{}%
\AgdaBound{𝓦}\AgdaSpace{}%
\AgdaOperator{\AgdaFunction{̇}}\AgdaSpace{}%
\AgdaSymbol{\}}\AgdaSpace{}%
\AgdaSymbol{\{}\AgdaBound{f}\AgdaSpace{}%
\AgdaBound{g}\AgdaSpace{}%
\AgdaSymbol{:}\AgdaSpace{}%
\AgdaBound{A}\AgdaSpace{}%
\AgdaSymbol{→}\AgdaSpace{}%
\AgdaBound{B}\AgdaSymbol{\}}\AgdaSpace{}%
\AgdaSymbol{→}\AgdaSpace{}%
\AgdaBound{f}\AgdaSpace{}%
\AgdaOperator{\AgdaFunction{∼}}\AgdaSpace{}%
\AgdaBound{g}\AgdaSpace{}%
\AgdaSymbol{→}\AgdaSpace{}%
\AgdaBound{f}\AgdaSpace{}%
\AgdaOperator{\AgdaDatatype{≡}}\AgdaSpace{}%
\AgdaBound{g}\<%
\end{code}
\scpad
\begin{code}%
\>[1]\AgdaFunction{dfunext}\AgdaSpace{}%
\AgdaSymbol{:}\AgdaSpace{}%
\AgdaSymbol{∀}\AgdaSpace{}%
\AgdaBound{𝓤}\AgdaSpace{}%
\AgdaBound{𝓦}\AgdaSpace{}%
\AgdaSymbol{→}\AgdaSpace{}%
\as (\AgdaBound{𝓤}\AgdaSpace{}%
\AgdaOperator{\AgdaPrimitive{⊔}}\AgdaSpace{}%
\AgdaBound{𝓦}\as )\AgdaSpace{}%
\AgdaOperator{\AgdaPrimitive{⁺}}\AgdaSpace{}%
\AgdaOperator{\AgdaFunction{̇}}\<%
\\
%
\>[1]\AgdaFunction{dfunext}\AgdaSpace{}%
\AgdaBound{𝓤}\AgdaSpace{}%
\AgdaBound{𝓦}\AgdaSpace{}%
\AgdaSymbol{=}\AgdaSpace{}%
\AgdaSymbol{\{}\AgdaBound{A}\AgdaSpace{}%
\AgdaSymbol{:}\AgdaSpace{}%
\AgdaBound{𝓤}\AgdaSpace{}%
\AgdaOperator{\AgdaFunction{̇}}\AgdaSpace{}%
\AgdaSymbol{\}\{}\AgdaBound{B}\AgdaSpace{}%
\AgdaSymbol{:}\AgdaSpace{}%
\AgdaBound{A}\AgdaSpace{}%
\AgdaSymbol{→}\AgdaSpace{}%
\AgdaBound{𝓦}\AgdaSpace{}%
\AgdaOperator{\AgdaFunction{̇}}\AgdaSpace{}%
\AgdaSymbol{\}\{}\AgdaBound{f}\AgdaSpace{}%
\AgdaBound{g}\AgdaSpace{}%
\AgdaSymbol{:}\AgdaSpace{}%
\AgdaSymbol{∀(}\AgdaBound{x}\AgdaSpace{}%
\AgdaSymbol{:}\AgdaSpace{}%
\AgdaBound{A}\AgdaSymbol{)}\AgdaSpace{}%
\AgdaSymbol{→}\AgdaSpace{}%
\AgdaBound{B}\AgdaSpace{}%
\AgdaBound{x}\AgdaSymbol{\}}\AgdaSpace{}%
\AgdaSymbol{→}%
\>[65]\AgdaBound{f}\AgdaSpace{}%
\AgdaOperator{\AgdaFunction{∼}}\AgdaSpace{}%
\AgdaBound{g}%
\>[72]\AgdaSymbol{→}%
\>[75]\AgdaBound{f}\AgdaSpace{}%
\AgdaOperator{\AgdaDatatype{≡}}\AgdaSpace{}%
\AgdaBound{g}\<%
\end{code}
\ccpad
In informal settings, this so-called \emph{point-wise equality of functions} is typically what one means when one
asserts that two functions are ``equal.''\footnote{In fact, if one assumes the \emph{univalence axiom} of Homotopy Type Theory~\cite{HoTT}, then point-wise equality of functions is equivalent to definitional equality of functions. See the section \href{https://www.cs.bham.ac.uk/~mhe/HoTT-UF-in-Agda-Lecture-Notes/HoTT-UF-Agda.html\#funextfromua}{``Function extensionality from univalence''} of~\cite{MHE}.} However, it is important to keep in mind the following fact: \textit{function extensionality is known to be neither provable nor disprovable in Martin-Löf type theory. It is an independent statement}.~\cite{MHE}

\subsubsection{Global function extensionality}\label{sec:glob-funct-extens}

An assumption that we adopt throughout much of the current version of the \ualib is a \emph{global function extensionality principle}. This asserts that function extensionality holds at all universe levels. Agda is capable of expressing types representing global principles as the language has a special universe level for such types.
Following Escardó, we denote this universe by \AgdaPrimitive{𝓤ω} (which is just an alias for Agda's \ad{Setω} universe).\footnote{More details about the \ad{𝓤ω} type are available at
\href{https://agda.readthedocs.io/en/latest/language/universe-levels.html\#expressions-of-kind-set}{agda.readthedocs.io}.}
The types \af{global-funext} and \af{global-dfunext} are defined in \typtop as follows.
\ccpad
\begin{code}%
\>[1]\AgdaFunction{global-funext}\AgdaSpace{}%
\AgdaSymbol{:}\AgdaSpace{}%
\AgdaPrimitive{𝓤ω}\<%
\\
%
\>[1]\AgdaFunction{global-funext}\AgdaSpace{}%
\AgdaSymbol{=}\AgdaSpace{}%
\AgdaSymbol{∀}%
\>[20]\AgdaSymbol{\{}\AgdaBound{𝓤}\AgdaSpace{}%
\AgdaBound{𝓥}\AgdaSymbol{\}}\AgdaSpace{}%
\AgdaSymbol{→}\AgdaSpace{}%
\AgdaFunction{funext}\AgdaSpace{}%
\AgdaBound{𝓤}\AgdaSpace{}%
\AgdaBound{𝓥}\<%
\\
%
\\[\AgdaEmptyExtraSkip]%
%
\>[1]\AgdaFunction{global-dfunext}\AgdaSpace{}%
\AgdaSymbol{:}\AgdaSpace{}%
\AgdaPrimitive{𝓤ω}\<%
\\
%
\>[1]\AgdaFunction{global-dfunext}\AgdaSpace{}%
\AgdaSymbol{=}\AgdaSpace{}%
\AgdaSymbol{∀}\AgdaSpace{}%
\AgdaSymbol{\{}\AgdaBound{𝓤}\AgdaSpace{}%
\AgdaBound{𝓥}\AgdaSymbol{\}}\AgdaSpace{}%
\AgdaSymbol{→}\AgdaSpace{}%
\AgdaFunction{dfunext}\AgdaSpace{}%
\AgdaBound{𝓤}\AgdaSpace{}%
\AgdaBound{𝓥}\<%
\end{code}
\ccpad
The next two types define the converse of function extensionality.
\ccpad
\begin{code}%
\>[1]\AgdaFunction{extfun}\AgdaSpace{}%
\AgdaSymbol{:}\AgdaSpace{}%
\AgdaSymbol{\{}\AgdaBound{A}\AgdaSpace{}%
\AgdaSymbol{:}\AgdaSpace{}%
\AgdaBound{𝓤}\AgdaSpace{}%
\AgdaOperator{\AgdaFunction{̇}}\AgdaSymbol{\}\{}\AgdaBound{B}\AgdaSpace{}%
\AgdaSymbol{:}\AgdaSpace{}%
\AgdaBound{𝓦}\AgdaSpace{}%
\AgdaOperator{\AgdaFunction{̇}}\AgdaSymbol{\}\{}\AgdaBound{f}\AgdaSpace{}%
\AgdaBound{g}\AgdaSpace{}%
\AgdaSymbol{:}\AgdaSpace{}%
\AgdaBound{A}\AgdaSpace{}%
\AgdaSymbol{→}\AgdaSpace{}%
\AgdaBound{B}\AgdaSymbol{\}}\AgdaSpace{}%
\AgdaSymbol{→}\AgdaSpace{}%
\AgdaBound{f}\AgdaSpace{}%
\AgdaOperator{\AgdaDatatype{≡}}\AgdaSpace{}%
\AgdaBound{g}%
\>[51]\AgdaSymbol{→}%
\>[54]\AgdaBound{f}\AgdaSpace{}%
\AgdaOperator{\AgdaFunction{∼}}\AgdaSpace{}%
\AgdaBound{g}\<%
\\
%
\>[1]\AgdaFunction{extfun}\AgdaSpace{}%
\AgdaInductiveConstructor{refl}\AgdaSpace{}%
\AgdaSymbol{\AgdaUnderscore{}}\AgdaSpace{}%
\AgdaSymbol{=}\AgdaSpace{}%
\AgdaInductiveConstructor{refl}\<%
\end{code}
\scpad
\begin{code}%
\>[1]\AgdaFunction{extdfun}\AgdaSpace{}%
\AgdaSymbol{:}\AgdaSpace{}%
\AgdaSymbol{\{}\AgdaBound{A}\AgdaSpace{}%
\AgdaSymbol{:}\AgdaSpace{}%
\AgdaBound{𝓤}\AgdaSpace{}%
\AgdaOperator{\AgdaFunction{̇}}\AgdaSpace{}%
\AgdaSymbol{\}\{}\AgdaBound{B}\AgdaSpace{}%
\AgdaSymbol{:}\AgdaSpace{}%
\AgdaBound{A}\AgdaSpace{}%
\AgdaSymbol{→}\AgdaSpace{}%
\AgdaBound{𝓦}\AgdaSpace{}%
\AgdaOperator{\AgdaFunction{̇}}\AgdaSpace{}%
\AgdaSymbol{\}(}\AgdaBound{f}\AgdaSpace{}%
\AgdaBound{g}\AgdaSpace{}%
\AgdaSymbol{:}\AgdaSpace{}%
\AgdaFunction{Π}\AgdaSpace{}%
\AgdaBound{B}\AgdaSymbol{)}\AgdaSpace{}%
\AgdaSymbol{→}\AgdaSpace{}%
\AgdaBound{f}\AgdaSpace{}%
\AgdaOperator{\AgdaDatatype{≡}}\AgdaSpace{}%
\AgdaBound{g}\AgdaSpace{}%
\AgdaSymbol{→}\AgdaSpace{}%
\AgdaBound{f}\AgdaSpace{}%
\AgdaOperator{\AgdaFunction{∼}}\AgdaSpace{}%
\AgdaBound{g}\<%
\\
%
\>[1]\AgdaFunction{extdfun}\AgdaSpace{}%
\AgdaSymbol{\AgdaUnderscore{}}\AgdaSpace{}%
\AgdaSymbol{\AgdaUnderscore{}}\AgdaSpace{}%
\AgdaInductiveConstructor{refl}\AgdaSpace{}%
\AgdaSymbol{\AgdaUnderscore{}}\AgdaSpace{}%
\AgdaSymbol{=}\AgdaSpace{}%
\AgdaInductiveConstructor{refl}\<%
\end{code}
\scpad

Though it may seem obvious to some readers, we wish to emphasize the important conceptual distinction between two flavors of type definition.  We do so by comparing the definitions of \af{funext} and \af{extfun}.

In the definition of \af{funext}, the codomain is a parameterized type, namely, \ab 𝓤\af ⁺~\apr ⊔~\ab 𝓥\af ⁺\af ̇, and the right-hand side of the defining equation of \af{funext} is an assertion (which may or may not hold). In the definition of \af{extfun}, the codomain is an assertion, namely, \ab f \af ∼ \ab g, and the right-hand side of the defining equation is a proof of this assertion. As such, \af{extfun} is a \defn{proof object}; it \emph{proves} (or \emph{inhabits the type that represents}) the proposition asserting that definitionally equivalent functions are pointwise equal. In contrast, \af{funext} is a type, and we may or may not wish to postulate an inhabitant of this type. That is, we could postulate that function extensionality holds by assuming we have a witness, say, \ab{fe}~\as :~\af{funext}~\ab 𝓤~\ab 𝓥, but as noted above the existence of such a witness cannot be proved in \mltt.

\subsubsection{An alternative extensionality type}\label{alternative-extensionality-type}
Finally, a useful alternative for expressing dependent function extensionality, which is essentially equivalent to \af{dfunext}, is to assert that \af{extdfun} is actually an \emph{equivalence} in a sense that we now describe. This requires a few definitions from the \am{MGS-Equivalences} module of \typtop. First, a type is a \defn{singleton} if it has exactly one inhabitant and a \defn{subsingleton} if it has at most one inhabitant.
 % These properties are represented in the \typetopology library by the following type.
\ccpad
\begin{code}%
\>[1]\AgdaFunction{is-center}\AgdaSpace{}%
\AgdaSymbol{:}\AgdaSpace{}%
\AgdaSymbol{(}\AgdaBound{A}\AgdaSpace{}%
\AgdaSymbol{:}\AgdaSpace{}%
\AgdaBound{𝓤}\AgdaSpace{}%
\AgdaOperator{\AgdaFunction{̇}}\AgdaSpace{}%
\AgdaSymbol{)}\AgdaSpace{}%
\AgdaSymbol{→}\AgdaSpace{}%
\AgdaBound{A}\AgdaSpace{}%
\AgdaSymbol{→}\AgdaSpace{}%
\AgdaBound{𝓤}\AgdaSpace{}%
\AgdaOperator{\AgdaFunction{̇}}\<%
\\
%
\>[1]\AgdaFunction{is-center}\AgdaSpace{}%
\AgdaBound{A}\AgdaSpace{}%
\AgdaBound{c}\AgdaSpace{}%
\AgdaSymbol{=}\AgdaSpace{}%
\AgdaSymbol{(}\AgdaBound{x}\AgdaSpace{}%
\AgdaSymbol{:}\AgdaSpace{}%
\AgdaBound{A}\AgdaSymbol{)}\AgdaSpace{}%
\AgdaSymbol{→}\AgdaSpace{}%
\AgdaBound{c}\AgdaSpace{}%
\AgdaOperator{\AgdaDatatype{≡}}\AgdaSpace{}%
\AgdaBound{x}\<%
\end{code}
\scpad
\begin{code}
\>[1]\AgdaFunction{is-singleton}\AgdaSpace{}%
\AgdaSymbol{:}\AgdaSpace{}%
\AgdaBound{𝓤}\AgdaSpace{}%
\AgdaOperator{\AgdaFunction{̇}}\AgdaSpace{}%
\AgdaSymbol{→}\AgdaSpace{}%
\AgdaBound{𝓤}\AgdaSpace{}%
\AgdaOperator{\AgdaFunction{̇}}\<%
\\
%
\>[1]\AgdaFunction{is-singleton}\AgdaSpace{}%
\AgdaBound{A}\AgdaSpace{}%
\AgdaSymbol{=}\AgdaSpace{}%
\AgdaFunction{Σ}\AgdaSpace{}%
\AgdaBound{c}\AgdaSpace{}%
\AgdaFunction{꞉}\AgdaSpace{}%
\AgdaBound{A}\AgdaSpace{}%
\AgdaFunction{,}\AgdaSpace{}%
\AgdaFunction{is-center}\AgdaSpace{}%
\AgdaBound{A}\AgdaSpace{}%
\AgdaBound{c}\<%
\end{code}
\scpad
\begin{code}
\>[1]\AgdaFunction{is-subsingleton}\AgdaSpace{}%
\AgdaSymbol{:}\AgdaSpace{}%
\AgdaBound{𝓤}\AgdaSpace{}%
\AgdaOperator{\AgdaFunction{̇}}\AgdaSpace{}%
\AgdaSymbol{→}\AgdaSpace{}%
\AgdaBound{𝓤}\AgdaSpace{}%
\AgdaOperator{\AgdaFunction{̇}}\<%
\\
%
\>[1]\AgdaFunction{is-subsingleton}\AgdaSpace{}%
\AgdaBound{A}\AgdaSpace{}%
\AgdaSymbol{=}\AgdaSpace{}%
\AgdaSymbol{(}\AgdaBound{x}\AgdaSpace{}%
\AgdaBound{y}\AgdaSpace{}%
\AgdaSymbol{:}\AgdaSpace{}%
\AgdaBound{A}\AgdaSymbol{)}\AgdaSpace{}%
\AgdaSymbol{→}\AgdaSpace{}%
\AgdaBound{x}\AgdaSpace{}%
\AgdaOperator{\AgdaDatatype{≡}}\AgdaSpace{}%
\AgdaBound{y}\<%
\end{code}
\ccpad
Next, we consider the type \af{is-equiv} which is used to assert that a function is an equivalence in the sense that we now describe. This requires the concept of a \href{https://ncatlab.org/nlab/show/fiber}{\defn{fiber}} of a function, which can be represented as a Sigma type whose inhabitants denote inverse images of points in the codomain of the given function.
\ccpad
\begin{code}%
\>[1]\AgdaFunction{fiber}\AgdaSpace{}%
\AgdaSymbol{:}\AgdaSpace{}%
\AgdaSymbol{\{}\AgdaBound{A}\AgdaSpace{}%
\AgdaSymbol{:}\AgdaSpace{}%
\AgdaBound{𝓤}\AgdaSpace{}%
\AgdaOperator{\AgdaFunction{̇}}%
\AgdaSymbol{\}}\AgdaSpace{}%
\AgdaSymbol{\{}\AgdaBound{B}\AgdaSpace{}%
\AgdaSymbol{:}\AgdaSpace{}%
\AgdaBound{𝓦}\AgdaSpace{}%
\AgdaOperator{\AgdaFunction{̇}}%
\AgdaSymbol{\}}\AgdaSpace{}%
\AgdaSymbol{(}\AgdaBound{f}\AgdaSpace{}%
\AgdaSymbol{:}\AgdaSpace{}%
\AgdaBound{A}\AgdaSpace{}%
\AgdaSymbol{→}\AgdaSpace{}%
\AgdaBound{B}\AgdaSymbol{)}\AgdaSpace{}%
\AgdaSymbol{→}\AgdaSpace{}%
\AgdaBound{B}\AgdaSpace{}%
\AgdaSymbol{→}\AgdaSpace{}%
\AgdaBound{𝓤}\AgdaSpace{}%
\AgdaOperator{\AgdaPrimitive{⊔}}\AgdaSpace{}%
\AgdaBound{𝓦}\AgdaSpace{}%
\AgdaOperator{\AgdaFunction{̇}}\<%
\\
%
\>[1]\AgdaFunction{fiber}\AgdaSpace{}%
\AgdaSymbol{\{}\AgdaBound{A}\AgdaSymbol{\}}\AgdaSpace{}%
\AgdaBound{f}\AgdaSpace{}%
\AgdaBound{y}\AgdaSpace{}%
\AgdaSymbol{=}\AgdaSpace{}%
\AgdaFunction{Σ}\AgdaSpace{}%
\AgdaBound{x}\AgdaSpace{}%
\AgdaFunction{꞉}\AgdaSpace{}%
\AgdaBound{A}\AgdaSpace{}%
\AgdaFunction{,}\AgdaSpace{}%
\AgdaBound{f}\AgdaSpace{}%
\AgdaBound{x}\AgdaSpace{}%
\AgdaOperator{\AgdaDatatype{≡}}\AgdaSpace{}%
\AgdaBound{y}\<%
\end{code}
\ccpad
A function is called an \defn{equivalence} if all of its fibers are singletons.
\ccpad
\begin{code}%
\>[1]\AgdaFunction{is-equiv}\AgdaSpace{}%
\AgdaSymbol{:}\AgdaSpace{}%
\AgdaSymbol{\{}\AgdaBound{A}\AgdaSpace{}%
\AgdaSymbol{:}\AgdaSpace{}%
\AgdaBound{𝓤}\AgdaSpace{}%
\AgdaOperator{\AgdaFunction{̇}}%
\AgdaSymbol{\}}\AgdaSpace{}%
\AgdaSymbol{\{}\AgdaBound{B}\AgdaSpace{}%
\AgdaSymbol{:}\AgdaSpace{}%
\AgdaBound{𝓦}\AgdaSpace{}%
\AgdaOperator{\AgdaFunction{̇}}%
\AgdaSymbol{\}}\AgdaSpace{}%
\AgdaSymbol{→}\AgdaSpace{}%
\AgdaSymbol{(}\AgdaBound{A}\AgdaSpace{}%
\AgdaSymbol{→}\AgdaSpace{}%
\AgdaBound{B}\AgdaSymbol{)}\AgdaSpace{}%
\AgdaSymbol{→}\AgdaSpace{}%
\AgdaBound{𝓤}\AgdaSpace{}%
\AgdaOperator{\AgdaPrimitive{⊔}}\AgdaSpace{}%
\AgdaBound{𝓦}\AgdaSpace{}%
\AgdaOperator{\AgdaFunction{̇}}\<%
\\
%
\>[1]\AgdaFunction{is-equiv}\AgdaSpace{}%
\AgdaBound{f}\AgdaSpace{}%
\AgdaSymbol{=}\AgdaSpace{}%
\AgdaSymbol{∀}\AgdaSpace{}%
\AgdaBound{y}\AgdaSpace{}%
\AgdaSymbol{→}\AgdaSpace{}%
\AgdaFunction{is-singleton}\AgdaSpace{}%
\AgdaSymbol{(}\AgdaFunction{fiber}\AgdaSpace{}%
\AgdaBound{f}\AgdaSpace{}%
\AgdaBound{y}\AgdaSymbol{)}\<%
\end{code}
\ccpad
Finally we are ready to fulfill the promise of a type that provides an alternative means of postulating function extensionality.
\ccpad
\begin{code}%
\>[1]\AgdaFunction{hfunext}\AgdaSpace{}%
\AgdaSymbol{:}\AgdaSpace{}%
\AgdaSymbol{∀}\AgdaSpace{}
\AgdaBound{𝓤}\AgdaSpace{}%
\AgdaBound{𝓦}\AgdaSpace{}%
\AgdaSymbol{→}\AgdaSpace{}%
\AgdaSymbol{(}\AgdaBound{𝓤}\AgdaSpace{}%
\AgdaOperator{\AgdaPrimitive{⊔}}\AgdaSpace{}%
\AgdaBound{𝓦}\AgdaSymbol{)}\AgdaOperator{\AgdaPrimitive{⁺}}\AgdaSpace{}%
\AgdaOperator{\AgdaFunction{̇}}\<%
\\
%
\>[1]\AgdaFunction{hfunext}\AgdaSpace{}%
\AgdaBound{𝓤}\AgdaSpace{}%
\AgdaBound{𝓦}\AgdaSpace{}%
\AgdaSymbol{=}\AgdaSpace{}%
\AgdaSymbol{\{}\AgdaBound{A}\AgdaSpace{}%
\AgdaSymbol{:}\AgdaSpace{}%
\AgdaBound{𝓤}\AgdaSpace{}%
\AgdaOperator{\AgdaFunction{̇}}\AgdaSymbol{\}\{}\AgdaBound{B}\AgdaSpace{}%
\AgdaSymbol{:}\AgdaSpace{}%
\AgdaBound{A}\AgdaSpace{}%
\AgdaSymbol{→}\AgdaSpace{}%
\AgdaBound{𝓦}\AgdaSpace{}%
\AgdaOperator{\AgdaFunction{̇}}\AgdaSymbol{\}}\AgdaSpace{}%
\AgdaSymbol{(}\AgdaBound{f}\AgdaSpace{}%
\AgdaBound{g}\AgdaSpace{}%
\AgdaSymbol{:}\AgdaSpace{}%
\AgdaFunction{Π}\AgdaSpace{}%
\AgdaBound{B}\AgdaSymbol{)}\AgdaSpace{}%
\AgdaSymbol{→}\AgdaSpace{}%
\AgdaFunction{is-equiv}\AgdaSpace{}%
\AgdaSymbol{(}\AgdaFunction{extdfun}\AgdaSpace{}%
\AgdaBound{f}\AgdaSpace{}%
\AgdaBound{g}\AgdaSymbol{)}\<%
\end{code}

\subsection{Inverses}\label{sec:inverse-image-invers}\firstsentence{Overture}{Inverses}
% : epics, monics, embeddings, inverse images
% \contextshort{\{\ab 𝓤 \ab 𝓦 \as : \ap{Universe}\}\{\ab A \as : \ab 𝓤\af ̇\}\{\ab B \as : \ab 𝓦\af ̇\}}
We begin by defining an inductive type that represents the semantic concept of \defn{inverse image} of a function.
\ccpad
\begin{code}%
\>[0][@{}l@{\AgdaIndent{0}}]%
\>[1]\AgdaKeyword{data}\AgdaSpace{}%
\AgdaOperator{\AgdaDatatype{Image\AgdaUnderscore{}∋\AgdaUnderscore{}}}\AgdaSpace{}%
\AgdaSymbol{\{}\AgdaBound{A}\AgdaSpace{}%
\AgdaSymbol{:}\AgdaSpace{}%
\AgdaBound{𝓤}\AgdaSpace{}%
\AgdaOperator{\AgdaFunction{̇}}\AgdaSpace{}%
\AgdaSymbol{\}\{}\AgdaBound{B}\AgdaSpace{}%
\AgdaSymbol{:}\AgdaSpace{}%
\AgdaBound{𝓦}\AgdaSpace{}%
\AgdaOperator{\AgdaFunction{̇}}\AgdaSpace{}%
\AgdaSymbol{\}(}\AgdaBound{f}\AgdaSpace{}%
\AgdaSymbol{:}\AgdaSpace{}%
\AgdaBound{A}\AgdaSpace{}%
\AgdaSymbol{→}\AgdaSpace{}%
\AgdaBound{B}\AgdaSymbol{)}\AgdaSpace{}%
\AgdaSymbol{:}\AgdaSpace{}%
\AgdaBound{B}\AgdaSpace{}%
\AgdaSymbol{→}\AgdaSpace{}%
\AgdaBound{𝓤}\AgdaSpace{}%
\AgdaOperator{\AgdaPrimitive{⊔}}\AgdaSpace{}%
\AgdaBound{𝓦}\AgdaSpace{}%
\AgdaOperator{\AgdaFunction{̇}}\<%
\\
\>[1][@{}l@{\AgdaIndent{0}}]%
\>[2]\AgdaKeyword{where}\<%
\\
%
\>[2]\AgdaInductiveConstructor{im}\AgdaSpace{}%
\AgdaSymbol{:}\AgdaSpace{}%
\AgdaSymbol{(}\AgdaBound{x}\AgdaSpace{}%
\AgdaSymbol{:}\AgdaSpace{}%
\AgdaBound{A}\AgdaSymbol{)}\AgdaSpace{}%
\AgdaSymbol{→}\AgdaSpace{}%
\AgdaOperator{\AgdaDatatype{Image}}\AgdaSpace{}%
\AgdaBound{f}\AgdaSpace{}%
\AgdaOperator{\AgdaDatatype{∋}}\AgdaSpace{}%
\AgdaBound{f}\AgdaSpace{}%
\AgdaBound{x}\<%
\\
%
\>[2]\AgdaInductiveConstructor{eq}\AgdaSpace{}%
\AgdaSymbol{:}\AgdaSpace{}%
\AgdaSymbol{(}\AgdaBound{b}\AgdaSpace{}%
\AgdaSymbol{:}\AgdaSpace{}%
\AgdaBound{B}\AgdaSymbol{)}\AgdaSpace{}%
\AgdaSymbol{→}\AgdaSpace{}%
\AgdaSymbol{(}\AgdaBound{a}\AgdaSpace{}%
\AgdaSymbol{:}\AgdaSpace{}%
\AgdaBound{A}\AgdaSymbol{)}\AgdaSpace{}%
\AgdaSymbol{→}\AgdaSpace{}%
\AgdaBound{b}\AgdaSpace{}%
\AgdaOperator{\AgdaDatatype{≡}}\AgdaSpace{}%
\AgdaBound{f}\AgdaSpace{}%
\AgdaBound{a}\AgdaSpace{}%
\AgdaSymbol{→}\AgdaSpace{}%
\AgdaOperator{\AgdaDatatype{Image}}\AgdaSpace{}%
\AgdaBound{f}\AgdaSpace{}%
\AgdaOperator{\AgdaDatatype{∋}}\AgdaSpace{}%
\AgdaBound{b}\<%
\end{code}
\ccpad
Next we verify that the type just defined is what we expect.
\ccpad
\begin{code}
\>[1]\AgdaFunction{ImageIsImage}\AgdaSpace{}%
\AgdaSymbol{:}%
\>[72I]\AgdaSymbol{\{}\AgdaBound{A}\AgdaSpace{}%
\AgdaSymbol{:}\AgdaSpace{}%
\AgdaBound{𝓤}\AgdaSpace{}%
\AgdaOperator{\AgdaFunction{̇}}\AgdaSymbol{\}\{}\AgdaBound{B}\AgdaSpace{}%
\AgdaSymbol{:}\AgdaSpace{}%
\AgdaBound{𝓦}\AgdaSpace{}%
\AgdaOperator{\AgdaFunction{̇}}\AgdaSymbol{\}(}\AgdaBound{f}\AgdaSpace{}%
\AgdaSymbol{:}\AgdaSpace{}%
\AgdaBound{A}\AgdaSpace{}%
\AgdaSymbol{→}\AgdaSpace{}%
\AgdaBound{B}\AgdaSymbol{)(}\AgdaBound{b}\AgdaSpace{}%
\AgdaSymbol{:}\AgdaSpace{}%
\AgdaBound{B}\AgdaSymbol{)(}\AgdaBound{a}\AgdaSpace{}%
\AgdaSymbol{:}\AgdaSpace{}%
\AgdaBound{A}\AgdaSymbol{)}\<%
\\
\>[.][@{}l@{}]\<[72I]%
\>[16]\AgdaComment{---------------------------------------------}\<%
\\
\>[1][@{}l@{\AgdaIndent{0}}]%
\>[2]\AgdaSymbol{→}%
\>[16]\AgdaBound{b}\AgdaSpace{}%
\AgdaOperator{\AgdaDatatype{≡}}\AgdaSpace{}%
\AgdaBound{f}\AgdaSpace{}%
\AgdaBound{a}\AgdaSpace{}%
\AgdaSymbol{→}\AgdaSpace{}%
\AgdaOperator{\AgdaDatatype{Image}}\AgdaSpace{}%
\AgdaBound{f}\AgdaSpace{}%
\AgdaOperator{\AgdaDatatype{∋}}\AgdaSpace{}%
\AgdaBound{b}\<%
\\
%
\\[\AgdaEmptyExtraSkip]%
%
\>[1]\AgdaFunction{ImageIsImage}\AgdaSpace{}%
\AgdaBound{f}\AgdaSpace{}%
\AgdaBound{b}\AgdaSpace{}%
\AgdaBound{a}\AgdaSpace{}%
\AgdaBound{b≡fa}\AgdaSpace{}%
\AgdaSymbol{=}\AgdaSpace{}%
\AgdaInductiveConstructor{eq}\AgdaSpace{}%
\AgdaBound{b}\AgdaSpace{}%
\AgdaBound{a}\AgdaSpace{}%
\AgdaBound{b≡fa}\<%
\end{code}
\ccpad
Note that an inhabitant of \af{Image} \ab f \af ∋ \ab b is a dependent pair (\ab a , \ab p), where \ab a \as : \ab A and \ab p \as : \ab b \ad ≡ \af f \ab a is a proof that \ab f maps \ab a to \ab b. Since the proof that \ab b belongs to the image of \ab f is always accompanied by a ``witness'' \ab a \as : \ab A, we can actually \emph{compute} a \emph{pseudoinverse} of \ab f. For convenience, we define this inverse function, which we call \af{Inv}, and which takes an arbitrary \ab b \as : \ab B and a witness-proof pair, (\ab a , \ab p) \as : \af{Image} \ab f \af ∋ \ab b, and returns \ab a.
\ccpad
\begin{code}%
\>[0][@{}l@{\AgdaIndent{1}}]%
\>[1]\AgdaFunction{Inv}\AgdaSpace{}%
\AgdaSymbol{:}\AgdaSpace{}%
\AgdaSymbol{\{}\AgdaBound{A}\AgdaSpace{}%
\AgdaSymbol{:}\AgdaSpace{}%
\AgdaBound{𝓤}\AgdaSpace{}%
\AgdaOperator{\AgdaFunction{̇}}\AgdaSpace{}%
\AgdaSymbol{\}\{}\AgdaBound{B}\AgdaSpace{}%
\AgdaSymbol{:}\AgdaSpace{}%
\AgdaBound{𝓦}\AgdaSpace{}%
\AgdaOperator{\AgdaFunction{̇}}\AgdaSpace{}%
\AgdaSymbol{\}(}\AgdaBound{f}\AgdaSpace{}%
\AgdaSymbol{:}\AgdaSpace{}%
\AgdaBound{A}\AgdaSpace{}%
\AgdaSymbol{→}\AgdaSpace{}%
\AgdaBound{B}\AgdaSymbol{)\{}\AgdaBound{b}\AgdaSpace{}%
\AgdaSymbol{:}\AgdaSpace{}%
\AgdaBound{B}\AgdaSymbol{\}}\AgdaSpace{}%
\AgdaSymbol{→}\AgdaSpace{}%
\AgdaOperator{\AgdaDatatype{Image}}\AgdaSpace{}%
\AgdaBound{f}\AgdaSpace{}%
\AgdaOperator{\AgdaDatatype{∋}}\AgdaSpace{}%
\AgdaBound{b}%
\>[61]\AgdaSymbol{→}%
\>[64]\AgdaBound{A}\<%
\\
%
\>[1]\AgdaFunction{Inv}\AgdaSpace{}%
\AgdaBound{f}\AgdaSpace{}%
\AgdaSymbol{\{}\AgdaDottedPattern{\AgdaSymbol{.(}}\AgdaDottedPattern{\AgdaBound{f}}\AgdaSpace{}%
\AgdaDottedPattern{\AgdaBound{a}}\AgdaDottedPattern{\AgdaSymbol{)}}\AgdaSymbol{\}}\AgdaSpace{}%
\AgdaSymbol{(}\AgdaInductiveConstructor{im}\AgdaSpace{}%
\AgdaBound{a}\AgdaSymbol{)}\AgdaSpace{}%
\AgdaSymbol{=}\AgdaSpace{}%
\AgdaBound{a}\<%
\\
%
\>[1]\AgdaFunction{Inv}\AgdaSpace{}%
\AgdaBound{f}\AgdaSpace{}%
\AgdaSymbol{(}\AgdaInductiveConstructor{eq}\AgdaSpace{}%
\AgdaSymbol{\AgdaUnderscore{}}\AgdaSpace{}%
\AgdaBound{a}\AgdaSpace{}%
\AgdaSymbol{\AgdaUnderscore{})}\AgdaSpace{}%
\AgdaSymbol{=}\AgdaSpace{}%
\AgdaBound{a}\<%
\end{code}
\ccpad
We can prove that \af{Inv} \ab f is the right-inverse of \ab f, as follows.
\ccpad
\begin{code}%
\>[0][@{}l@{\AgdaIndent{1}}]%
\>[1]\AgdaFunction{InvIsInv}\AgdaSpace{}%
\AgdaSymbol{:}\AgdaSpace{}%
\AgdaSymbol{\{}\AgdaBound{A}\AgdaSpace{}%
\AgdaSymbol{:}\AgdaSpace{}%
\AgdaBound{𝓤}\AgdaSpace{}%
\AgdaOperator{\AgdaFunction{̇}}\AgdaSymbol{\}\{}\AgdaBound{B}\AgdaSpace{}%
\AgdaSymbol{:}\AgdaSpace{}%
\AgdaBound{𝓦}\AgdaSpace{}%
\AgdaOperator{\AgdaFunction{̇}}\AgdaSymbol{\}(}\AgdaBound{f}\AgdaSpace{}%
\AgdaSymbol{:}\AgdaSpace{}%
\AgdaBound{A}\AgdaSpace{}%
\AgdaSymbol{→}\AgdaSpace{}%
\AgdaBound{B}\AgdaSymbol{)\{}\AgdaBound{b}\AgdaSpace{}%
\AgdaSymbol{:}\AgdaSpace{}%
\AgdaBound{B}\AgdaSymbol{\}(}\AgdaBound{b∈Imgf}\AgdaSpace{}%
\AgdaSymbol{:}\AgdaSpace{}%
\AgdaOperator{\AgdaDatatype{Image}}\AgdaSpace{}%
\AgdaBound{f}\AgdaSpace{}%
\AgdaOperator{\AgdaDatatype{∋}}\AgdaSpace{}%
\AgdaBound{b}\AgdaSymbol{)}\AgdaSpace{}%
\AgdaSymbol{→}\AgdaSpace{}%
\AgdaBound{f}\AgdaSymbol{(}\AgdaFunction{Inv}\AgdaSpace{}%
\AgdaBound{f}\AgdaSpace{}%
\AgdaBound{b∈Imgf}\AgdaSymbol{)}\AgdaSpace{}%
\AgdaOperator{\AgdaDatatype{≡}}\AgdaSpace{}%
\AgdaBound{b}\<%
\\
%
\>[1]\AgdaFunction{InvIsInv}\AgdaSpace{}%
\AgdaBound{f}\AgdaSpace{}%
\AgdaSymbol{\{}\AgdaDottedPattern{\AgdaSymbol{.(}}\AgdaDottedPattern{\AgdaBound{f}}\AgdaSpace{}%
\AgdaDottedPattern{\AgdaBound{a}}\AgdaDottedPattern{\AgdaSymbol{)}}\AgdaSymbol{\}}\AgdaSpace{}%
\AgdaSymbol{(}\AgdaInductiveConstructor{im}\AgdaSpace{}%
\AgdaBound{a}\AgdaSymbol{)}\AgdaSpace{}%
\AgdaSymbol{=}\AgdaSpace{}%
\AgdaInductiveConstructor{refl}\AgdaSpace{}%
\AgdaSymbol{\AgdaUnderscore{}}\<%
\\
%
\>[1]\AgdaFunction{InvIsInv}\AgdaSpace{}%
\AgdaBound{f}\AgdaSpace{}%
\AgdaSymbol{(}\AgdaInductiveConstructor{eq}\AgdaSpace{}%
\AgdaSymbol{\AgdaUnderscore{}}\AgdaSpace{}%
\AgdaSymbol{\AgdaUnderscore{}}\AgdaSpace{}%
\AgdaBound{p}\AgdaSymbol{)}\AgdaSpace{}%
\AgdaSymbol{=}\AgdaSpace{}%
\AgdaBound{p}\AgdaSpace{}%
\AgdaOperator{\AgdaFunction{⁻¹}}\<%
\end{code}

\subsubsection{Surjective functions}\label{surjective-functions}

An \defn{epic} (or \defn{surjective}) function from type \ab A \as : \ab 𝓤\af ̇ to type \ab B \as : \ab 𝓦\af ̇ is as an inhabitant of the \af{Epic} type, which we define as follows.
\ccpad
\begin{code}%
\>[0][@{}l@{\AgdaIndent{1}}]%
\>[1]\AgdaFunction{Epic}\AgdaSpace{}%
\AgdaSymbol{:}\AgdaSpace{}%
\AgdaSymbol{\{}\AgdaBound{A}\AgdaSpace{}%
\AgdaSymbol{:}\AgdaSpace{}%
\AgdaBound{𝓤}\AgdaSpace{}%
\AgdaOperator{\AgdaFunction{̇}}\AgdaSpace{}%
\AgdaSymbol{\}}\AgdaSpace{}%
\AgdaSymbol{\{}\AgdaBound{B}\AgdaSpace{}%
\AgdaSymbol{:}\AgdaSpace{}%
\AgdaBound{𝓦}\AgdaSpace{}%
\AgdaOperator{\AgdaFunction{̇}}\AgdaSpace{}%
\AgdaSymbol{\}}\AgdaSpace{}%
\AgdaSymbol{(}\AgdaBound{g}\AgdaSpace{}%
\AgdaSymbol{:}\AgdaSpace{}%
\AgdaBound{A}\AgdaSpace{}%
\AgdaSymbol{→}\AgdaSpace{}%
\AgdaBound{B}\AgdaSymbol{)}\AgdaSpace{}%
\AgdaSymbol{→}%
\>[45]\AgdaBound{𝓤}\AgdaSpace{}%
\AgdaOperator{\AgdaPrimitive{⊔}}\AgdaSpace{}%
\AgdaBound{𝓦}\AgdaSpace{}%
\AgdaOperator{\AgdaFunction{̇}}\<%
\\
%
\>[1]\AgdaFunction{Epic}\AgdaSpace{}%
\AgdaBound{g}\AgdaSpace{}%
\AgdaSymbol{=}\AgdaSpace{}%
\AgdaSymbol{∀}\AgdaSpace{}%
\AgdaBound{y}\AgdaSpace{}%
\AgdaSymbol{→}\AgdaSpace{}%
\AgdaOperator{\AgdaDatatype{Image}}\AgdaSpace{}%
\AgdaBound{g}\AgdaSpace{}%
\AgdaOperator{\AgdaDatatype{∋}}\AgdaSpace{}%
\AgdaBound{y}\<%
\end{code}
\ccpad
We obtain the right-inverse (or pseudoinverse) of an epic function \ab f by applying the function \af{EpicInv} (which we now define) to the function \ab f along with a proof, \ab{fepi} \as : \af{Epic} \ab f, that \ab f is surjective.
\ccpad
\begin{code}%
\>[0][@{}l@{\AgdaIndent{1}}]%
\>[1]\AgdaFunction{EpicInv}\AgdaSpace{}%
\AgdaSymbol{:}%
\>[242I]\AgdaSymbol{\{}\AgdaBound{A}\AgdaSpace{}%
\AgdaSymbol{:}\AgdaSpace{}%
\AgdaBound{𝓤}\AgdaSpace{}%
\AgdaOperator{\AgdaFunction{̇}}\AgdaSpace{}%
\AgdaSymbol{\}}\AgdaSpace{}%
\AgdaSymbol{\{}\AgdaBound{B}\AgdaSpace{}%
\AgdaSymbol{:}\AgdaSpace{}%
\AgdaBound{𝓦}\AgdaSpace{}%
\AgdaOperator{\AgdaFunction{̇}}\AgdaSpace{}%
\AgdaSymbol{\}}\AgdaSpace{}%
% \<%
% \\
% \>[.][@{}l@{}]\<[242I]%
% \>[11]
\AgdaSymbol{(}\AgdaBound{f}\AgdaSpace{}%
\AgdaSymbol{:}\AgdaSpace{}%
\AgdaBound{A}\AgdaSpace{}%
\AgdaSymbol{→}\AgdaSpace{}%
\AgdaBound{B}\AgdaSymbol{)}\AgdaSpace{}%
\AgdaSymbol{→}\AgdaSpace{}%
\AgdaFunction{Epic}\AgdaSpace{}%
\AgdaBound{f}\AgdaSpace{}%
% \<%
% \\
% %
% \>[11]\AgdaComment{--------------------}\<%
% \\
% \>[1][@{}l@{\AgdaIndent{0}}]%
% \>[2]
\AgdaSymbol{→}\AgdaSpace{}%%
% \>[11]
\AgdaBound{B}\AgdaSpace{}%
\AgdaSymbol{→}\AgdaSpace{}%
\AgdaBound{A}\<%
\\
%
\>[1]\AgdaFunction{EpicInv}\AgdaSpace{}%
\AgdaBound{f}\AgdaSpace{}%
\AgdaBound{fepi}\AgdaSpace{}%
\AgdaBound{b}\AgdaSpace{}%
\AgdaSymbol{=}\AgdaSpace{}%
\AgdaFunction{Inv}\AgdaSpace{}%
\AgdaBound{f}\AgdaSpace{}%
\AgdaSymbol{(}\AgdaBound{fepi}\AgdaSpace{}%
\AgdaBound{b}\AgdaSymbol{)}\<%
\end{code}
\ccpad
The function defined by \af{EpicInv} \ab f \ab{fepi} is indeed the right-inverse of \ab f. To state this, we'll use the function composition operation \af{∘}, which is already defined in the \typetopology library, as follows.
\ccpad
\begin{code}%
\>[1]\AgdaOperator{\AgdaFunction{\AgdaUnderscore{}∘\AgdaUnderscore{}}}\AgdaSpace{}%
\AgdaSymbol{:}\AgdaSpace{}%
\AgdaSymbol{\{}\AgdaBound{X}\AgdaSpace{}%
\AgdaSymbol{:}\AgdaSpace{}%
\AgdaBound{𝓤}\AgdaSpace{}%
\AgdaOperator{\AgdaFunction{̇}}\AgdaSpace{}%
\AgdaSymbol{\}}\AgdaSpace{}%
\AgdaSymbol{\{}\AgdaBound{Y}\AgdaSpace{}%
\AgdaSymbol{:}\AgdaSpace{}%
\AgdaBound{𝓦}\AgdaSpace{}%
\AgdaOperator{\AgdaFunction{̇}}\AgdaSymbol{\}\{}\AgdaBound{Z}\AgdaSpace{}%
\AgdaSymbol{:}\AgdaSpace{}%
\AgdaBound{Y}\AgdaSpace{}%
\AgdaSymbol{→}\AgdaSpace{}%
\AgdaBound{𝓦}\AgdaSpace{}%
\AgdaOperator{\AgdaFunction{̇}}\AgdaSpace{}%
\AgdaSymbol{\}}\AgdaSpace{}%
% \<%
% \\
% \>[1][@{}l@{\AgdaIndent{0}}]%
% \>[2]
\AgdaSymbol{→}\AgdaSpace{}%
%
% \>[7]
\AgdaFunction{Π}\AgdaSpace{}%
\AgdaBound{Z}\AgdaSpace{}%
\AgdaSymbol{→}\AgdaSpace{}%
\AgdaSymbol{(}\AgdaBound{f}\AgdaSpace{}%
\AgdaSymbol{:}\AgdaSpace{}%
\AgdaBound{X}\AgdaSpace{}%
\AgdaSymbol{→}\AgdaSpace{}%
\AgdaBound{Y}\AgdaSymbol{)}\AgdaSpace{}%
\AgdaSymbol{→}\AgdaSpace{}%
\AgdaSymbol{(}\AgdaBound{x}\AgdaSpace{}%
\AgdaSymbol{:}\AgdaSpace{}%
\AgdaBound{X}\AgdaSymbol{)}\AgdaSpace{}%
\AgdaSymbol{→}\AgdaSpace{}%
\AgdaBound{Z}\AgdaSpace{}%
\AgdaSymbol{(}\AgdaBound{f}\AgdaSpace{}%
\AgdaBound{x}\AgdaSymbol{)}\<%
\\
%
\>[1]\AgdaBound{g}\AgdaSpace{}%
\AgdaOperator{\AgdaFunction{∘}}\AgdaSpace{}%
\AgdaBound{f}\AgdaSpace{}%
\AgdaSymbol{=}\AgdaSpace{}%
\AgdaSymbol{λ}\AgdaSpace{}%
\AgdaBound{x}\AgdaSpace{}%
\AgdaSymbol{→}\AgdaSpace{}%
\AgdaBound{g}\AgdaSpace{}%
\AgdaSymbol{(}\AgdaBound{f}\AgdaSpace{}%
\AgdaBound{x}\AgdaSymbol{)}\<%
\end{code}
\scpad
% \\
% %
% \\[\AgdaEmptyExtraSkip]%
% \>[0]\AgdaKeyword{open}\AgdaSpace{}%
% \AgdaKeyword{import}\AgdaSpace{}%
% \AgdaModule{MGS-MLTT}\AgdaSpace{}%
% \AgdaKeyword{using}\AgdaSpace{}%
% \AgdaSymbol{(}\AgdaOperator{\AgdaFunction{\AgdaUnderscore{}∘\AgdaUnderscore{}}}\AgdaSymbol{)}\AgdaSpace{}%
% \AgdaKeyword{public}\<%
% \\
% %
% \\[\AgdaEmptyExtraSkip]%
% %
% \\[\AgdaEmptyExtraSkip]%
% \>[0]\AgdaKeyword{module}\AgdaSpace{}%
% \AgdaModule{\AgdaUnderscore{}}\AgdaSpace{}%
% \AgdaSymbol{\{}\AgdaBound{𝓤}\AgdaSpace{}%
% \AgdaBound{𝓦}\AgdaSpace{}%
% \AgdaSymbol{:}\AgdaSpace{}%
% \AgdaPostulate{Universe}\AgdaSymbol{\}}\AgdaSpace{}%
% \AgdaKeyword{where}\<%
% \\
% %
% \\[\AgdaEmptyExtraSkip]%
\begin{code}
\>[1]\AgdaFunction{EpicInvIsRightInv}\AgdaSpace{}%
\AgdaSymbol{:}%
\>[327I]\AgdaFunction{funext}\AgdaSpace{}%
\AgdaBound{𝓦}\AgdaSpace{}%
\AgdaBound{𝓦}\AgdaSpace{}%
\AgdaSymbol{→}\AgdaSpace{}%
\AgdaSymbol{\{}\AgdaBound{A}\AgdaSpace{}%
\AgdaSymbol{:}\AgdaSpace{}%
\AgdaBound{𝓤}\AgdaSpace{}%
\AgdaOperator{\AgdaFunction{̇}}\AgdaSpace{}%
\AgdaSymbol{\}}\AgdaSpace{}%
\AgdaSymbol{\{}\AgdaBound{B}\AgdaSpace{}%
\AgdaSymbol{:}\AgdaSpace{}%
\AgdaBound{𝓦}\AgdaSpace{}%
\AgdaOperator{\AgdaFunction{̇}}\AgdaSpace{}%
\AgdaSymbol{\}}\<%
\\
\>[.][@{}l@{}]\<[327I]%
\>[21]\AgdaSymbol{(}\AgdaBound{f}\AgdaSpace{}%
\AgdaSymbol{:}\AgdaSpace{}%
\AgdaBound{A}\AgdaSpace{}%
\AgdaSymbol{→}\AgdaSpace{}%
\AgdaBound{B}\AgdaSymbol{)}%
\>[34]\AgdaSymbol{(}\AgdaBound{fepi}\AgdaSpace{}%
\AgdaSymbol{:}\AgdaSpace{}%
\AgdaFunction{Epic}\AgdaSpace{}%
\AgdaBound{f}\AgdaSymbol{)}\<%
\\
%
\>[21]\AgdaComment{--------------------------}\<%
\\
\>[1][@{}l@{\AgdaIndent{0}}]%
\>[2]\AgdaSymbol{→}%
\>[21]\AgdaBound{f}\AgdaSpace{}%
\AgdaOperator{\AgdaFunction{∘}}\AgdaSpace{}%
\AgdaSymbol{(}\AgdaFunction{EpicInv}\AgdaSpace{}%
\AgdaBound{f}\AgdaSpace{}%
\AgdaBound{fepi}\AgdaSymbol{)}\AgdaSpace{}%
\AgdaOperator{\AgdaDatatype{≡}}\AgdaSpace{}%
\AgdaFunction{𝑖𝑑}\AgdaSpace{}%
\AgdaBound{B}\<%
\\
%
\\[\AgdaEmptyExtraSkip]%
%
\>[1]\AgdaFunction{EpicInvIsRightInv}\AgdaSpace{}%
\AgdaBound{fe}\AgdaSpace{}%
\AgdaBound{f}\AgdaSpace{}%
\AgdaBound{fepi}\AgdaSpace{}%
\AgdaSymbol{=}\AgdaSpace{}%
\AgdaBound{fe}\AgdaSpace{}%
\AgdaSymbol{(λ}\AgdaSpace{}%
\AgdaBound{x}\AgdaSpace{}%
\AgdaSymbol{→}\AgdaSpace{}%
\AgdaFunction{InvIsInv}\AgdaSpace{}%
\AgdaBound{f}\AgdaSpace{}%
\AgdaSymbol{(}\AgdaBound{fepi}\AgdaSpace{}%
\AgdaBound{x}\AgdaSymbol{))}\<%
\end{code}

\subsubsection{Injective functions}\label{injective-functions}

We say that a function \ab g \as : \ab A \as → \ab B is \textbf{monic} (or \textbf{injective} or \textbf{one-to-one}) if it doesn't map distinct elements to a common point. This property is formalized quite naturally using the \af{Monic} type, which we now define.
\ccpad
\begin{code}%
\>[0][@{}l@{\AgdaIndent{1}}]%
\>[1]\AgdaFunction{Monic}\AgdaSpace{}%
\AgdaSymbol{:}\AgdaSpace{}%
\AgdaSymbol{\{}\AgdaBound{A}\AgdaSpace{}%
\AgdaSymbol{:}\AgdaSpace{}%
\AgdaBound{𝓤}\AgdaSpace{}%
\AgdaOperator{\AgdaFunction{̇}}\AgdaSpace{}%
\AgdaSymbol{\}}\AgdaSpace{}%
\AgdaSymbol{\{}\AgdaBound{B}\AgdaSpace{}%
\AgdaSymbol{:}\AgdaSpace{}%
\AgdaBound{𝓦}\AgdaSpace{}%
\AgdaOperator{\AgdaFunction{̇}}\AgdaSpace{}%
\AgdaSymbol{\}(}\AgdaBound{g}\AgdaSpace{}%
\AgdaSymbol{:}\AgdaSpace{}%
\AgdaBound{A}\AgdaSpace{}%
\AgdaSymbol{→}\AgdaSpace{}%
\AgdaBound{B}\AgdaSymbol{)}\AgdaSpace{}%
\AgdaSymbol{→}\AgdaSpace{}%
\AgdaBound{𝓤}\AgdaSpace{}%
\AgdaOperator{\AgdaPrimitive{⊔}}\AgdaSpace{}%
\AgdaBound{𝓦}\AgdaSpace{}%
\AgdaOperator{\AgdaFunction{̇}}\<%
\\
%
\>[1]\AgdaFunction{Monic}\AgdaSpace{}%
\AgdaBound{g}\AgdaSpace{}%
\AgdaSymbol{=}\AgdaSpace{}%
\AgdaSymbol{∀}\AgdaSpace{}%
\AgdaBound{a₁}\AgdaSpace{}%
\AgdaBound{a₂}\AgdaSpace{}%
\AgdaSymbol{→}\AgdaSpace{}%
\AgdaBound{g}\AgdaSpace{}%
\AgdaBound{a₁}\AgdaSpace{}%
\AgdaOperator{\AgdaDatatype{≡}}\AgdaSpace{}%
\AgdaBound{g}\AgdaSpace{}%
\AgdaBound{a₂}\AgdaSpace{}%
\AgdaSymbol{→}\AgdaSpace{}%
\AgdaBound{a₁}\AgdaSpace{}%
\AgdaOperator{\AgdaDatatype{≡}}\AgdaSpace{}%
\AgdaBound{a₂}\<%
\end{code}
\ccpad
Again, we obtain a pseudoinverse. Here it is obtained by applying the function \af{MonicInv} to \ab g and a proof that \ab g is monic.
\ccpad
\begin{code}%
\>[1]\AgdaFunction{MonicInv}\AgdaSpace{}%
\AgdaSymbol{:}\AgdaSpace{}%
\AgdaSymbol{\{}\AgdaBound{A}\AgdaSpace{}%
\AgdaSymbol{:}\AgdaSpace{}%
\AgdaBound{𝓤}\AgdaSpace{}%
\AgdaOperator{\AgdaFunction{̇}}\AgdaSpace{}%
\AgdaSymbol{\}\{}\AgdaBound{B}\AgdaSpace{}%
\AgdaSymbol{:}\AgdaSpace{}%
\AgdaBound{𝓦}\AgdaSpace{}%
\AgdaOperator{\AgdaFunction{̇}}\AgdaSpace{}%
\AgdaSymbol{\}(}\AgdaBound{f}\AgdaSpace{}%
\AgdaSymbol{:}\AgdaSpace{}%
\AgdaBound{A}\AgdaSpace{}%
\AgdaSymbol{→}\AgdaSpace{}%
\AgdaBound{B}\AgdaSymbol{)}\AgdaSpace{}%
\AgdaSymbol{→}\AgdaSpace{}%
\AgdaFunction{Monic}\AgdaSpace{}%
\AgdaBound{f}\AgdaSpace{}%
% \<%
% \\
% \>[1][@{}l@{\AgdaIndent{0}}]%
% \>[2]
\AgdaSymbol{→}\AgdaSpace{}%
% \>[12]
\AgdaSymbol{(}\AgdaBound{b}\AgdaSpace{}%
\AgdaSymbol{:}\AgdaSpace{}%
\AgdaBound{B}\AgdaSymbol{)}\AgdaSpace{}%
\AgdaSymbol{→}\AgdaSpace{}%
\AgdaOperator{\AgdaDatatype{Image}}\AgdaSpace{}%
\AgdaBound{f}\AgdaSpace{}%
\AgdaOperator{\AgdaDatatype{∋}}\AgdaSpace{}%
\AgdaBound{b}\AgdaSpace{}%
\AgdaSymbol{→}\AgdaSpace{}%
\AgdaBound{A}\<%
\\
%
\>[1]\AgdaFunction{MonicInv}\AgdaSpace{}%
\AgdaBound{f}\AgdaSpace{}%
\AgdaSymbol{\AgdaUnderscore{}}\AgdaSpace{}%
\AgdaSymbol{=}\AgdaSpace{}%
\AgdaSymbol{λ}\AgdaSpace{}%
\AgdaBound{b}\AgdaSpace{}%
\AgdaBound{Imf∋b}\AgdaSpace{}%
\AgdaSymbol{→}\AgdaSpace{}%
\AgdaFunction{Inv}\AgdaSpace{}%
\AgdaBound{f}\AgdaSpace{}%
\AgdaBound{Imf∋b}\<%
\end{code}
\ccpad
The function defined by \af{MonicInv} \ab f \ab{fM} is the left-inverse of \ab f.
\ccpad
\begin{code}%
\>[1]\AgdaFunction{MonicInvIsLeftInv}\AgdaSpace{}%
\AgdaSymbol{:}%
\>[100I]\AgdaSymbol{\{}\AgdaBound{A}\AgdaSpace{}%
\AgdaSymbol{:}\AgdaSpace{}%
\AgdaBound{𝓤}\AgdaSpace{}%
\AgdaOperator{\AgdaFunction{̇}}\AgdaSpace{}%
\AgdaSymbol{\}\{}\AgdaBound{B}\AgdaSpace{}%
\AgdaSymbol{:}\AgdaSpace{}%
\AgdaBound{𝓦}\AgdaSpace{}%
\AgdaOperator{\AgdaFunction{̇}}\AgdaSpace{}%
\AgdaSymbol{\}(}\AgdaBound{f}\AgdaSpace{}%
\AgdaSymbol{:}\AgdaSpace{}%
\AgdaBound{A}\AgdaSpace{}%
\AgdaSymbol{→}\AgdaSpace{}%
\AgdaBound{B}\AgdaSymbol{)(}\AgdaBound{fmonic}\AgdaSpace{}%
\AgdaSymbol{:}\AgdaSpace{}%
\AgdaFunction{Monic}\AgdaSpace{}%
\AgdaBound{f}\AgdaSymbol{)(}\AgdaBound{x}\AgdaSpace{}%
\AgdaSymbol{:}\AgdaSpace{}%
\AgdaBound{A}\AgdaSymbol{)}\<%
\\
\>[1][@{}l@{\AgdaIndent{0}}]%
\>[3]\AgdaSymbol{→}%
\>[.][@{}l@{}]\<[100I]%
\>[21]\AgdaSymbol{(}\AgdaFunction{MonicInv}\AgdaSpace{}%
\AgdaBound{f}\AgdaSpace{}%
\AgdaBound{fmonic}\AgdaSymbol{)(}\AgdaBound{f}\AgdaSpace{}%
\AgdaBound{x}\AgdaSymbol{)(}\AgdaInductiveConstructor{im}\AgdaSpace{}%
\AgdaBound{x}\AgdaSymbol{)}\AgdaSpace{}%
\AgdaOperator{\AgdaDatatype{≡}}\AgdaSpace{}%
\AgdaBound{x}\<%
\\
%
\>[1]\AgdaFunction{MonicInvIsLeftInv}\AgdaSpace{}%
\AgdaBound{f}\AgdaSpace{}%
\AgdaBound{fmonic}\AgdaSpace{}%
\AgdaBound{x}\AgdaSpace{}%
\AgdaSymbol{=}\AgdaSpace{}%
\AgdaInductiveConstructor{𝓇ℯ𝒻𝓁}\<%
\end{code}

% \subsubsection{Composition laws}\label{composition-laws}

% \begin{code}%
% \>[0]\AgdaKeyword{module}\AgdaSpace{}%
% \AgdaModule{\AgdaUnderscore{}}\AgdaSpace{}%
% \AgdaSymbol{\{}\AgdaBound{𝓧}\AgdaSpace{}%
% \AgdaBound{𝓨}\AgdaSpace{}%
% \AgdaBound{𝓩}\AgdaSpace{}%
% \AgdaSymbol{:}\AgdaSpace{}%
% \AgdaPostulate{Universe}\AgdaSymbol{\}}\AgdaSpace{}%
% \AgdaKeyword{where}\<%
% \\
% %
% \\[\AgdaEmptyExtraSkip]%
% \>[0][@{}l@{\AgdaIndent{0}}]%
% \>[1]\AgdaFunction{epic-factor}\AgdaSpace{}%
% \AgdaSymbol{:}%
% \>[477I]\AgdaFunction{funext}\AgdaSpace{}%
% \AgdaBound{𝓨}\AgdaSpace{}%
% \AgdaBound{𝓨}\AgdaSpace{}%
% \AgdaSymbol{→}\AgdaSpace{}%
% \AgdaSymbol{\{}\AgdaBound{A}\AgdaSpace{}%
% \AgdaSymbol{:}\AgdaSpace{}%
% \AgdaBound{𝓧}\AgdaSpace{}%
% \AgdaOperator{\AgdaFunction{̇}}\AgdaSymbol{\}\{}\AgdaBound{B}\AgdaSpace{}%
% \AgdaSymbol{:}\AgdaSpace{}%
% \AgdaBound{𝓨}\AgdaSpace{}%
% \AgdaOperator{\AgdaFunction{̇}}\AgdaSymbol{\}\{}\AgdaBound{C}\AgdaSpace{}%
% \AgdaSymbol{:}\AgdaSpace{}%
% \AgdaBound{𝓩}\AgdaSpace{}%
% \AgdaOperator{\AgdaFunction{̇}}\AgdaSymbol{\}}\<%
% \\
% \>[.][@{}l@{}]\<[477I]%
% \>[15]\AgdaSymbol{(}\AgdaBound{β}\AgdaSpace{}%
% \AgdaSymbol{:}\AgdaSpace{}%
% \AgdaBound{A}\AgdaSpace{}%
% \AgdaSymbol{→}\AgdaSpace{}%
% \AgdaBound{B}\AgdaSymbol{)(}\AgdaBound{ξ}\AgdaSpace{}%
% \AgdaSymbol{:}\AgdaSpace{}%
% \AgdaBound{A}\AgdaSpace{}%
% \AgdaSymbol{→}\AgdaSpace{}%
% \AgdaBound{C}\AgdaSymbol{)(}\AgdaBound{ϕ}\AgdaSpace{}%
% \AgdaSymbol{:}\AgdaSpace{}%
% \AgdaBound{C}\AgdaSpace{}%
% \AgdaSymbol{→}\AgdaSpace{}%
% \AgdaBound{B}\AgdaSymbol{)}\<%
% \\
% \>[1][@{}l@{\AgdaIndent{0}}]%
% \>[2]\AgdaSymbol{→}%
% \>[15]\AgdaBound{β}\AgdaSpace{}%
% \AgdaOperator{\AgdaDatatype{≡}}\AgdaSpace{}%
% \AgdaBound{ϕ}\AgdaSpace{}%
% \AgdaOperator{\AgdaFunction{∘}}\AgdaSpace{}%
% \AgdaBound{ξ}\AgdaSpace{}%
% \AgdaSymbol{→}%
% \>[28]\AgdaFunction{Epic}\AgdaSpace{}%
% \AgdaBound{β}\AgdaSpace{}%
% \AgdaSymbol{→}\AgdaSpace{}%
% \AgdaFunction{Epic}\AgdaSpace{}%
% \AgdaBound{ϕ}\<%
% \\
% %
% \\[\AgdaEmptyExtraSkip]%
% %
% \>[1]\AgdaFunction{epic-factor}\AgdaSpace{}%
% \AgdaBound{fe}\AgdaSpace{}%
% \AgdaSymbol{\{}\AgdaBound{A}\AgdaSymbol{\}\{}\AgdaBound{B}\AgdaSymbol{\}\{}\AgdaBound{C}\AgdaSymbol{\}}\AgdaSpace{}%
% \AgdaBound{β}\AgdaSpace{}%
% \AgdaBound{ξ}\AgdaSpace{}%
% \AgdaBound{ϕ}\AgdaSpace{}%
% \AgdaBound{compId}\AgdaSpace{}%
% \AgdaBound{βe}\AgdaSpace{}%
% \AgdaBound{y}\AgdaSpace{}%
% \AgdaSymbol{=}\AgdaSpace{}%
% \AgdaFunction{γ}\<%
% \\
% \>[1][@{}l@{\AgdaIndent{0}}]%
% \>[2]\AgdaKeyword{where}\<%
% \\
% \>[2][@{}l@{\AgdaIndent{0}}]%
% \>[3]\AgdaFunction{βinv}\AgdaSpace{}%
% \AgdaSymbol{:}\AgdaSpace{}%
% \AgdaBound{B}\AgdaSpace{}%
% \AgdaSymbol{→}\AgdaSpace{}%
% \AgdaBound{A}\<%
% \\
% %
% \>[3]\AgdaFunction{βinv}\AgdaSpace{}%
% \AgdaSymbol{=}\AgdaSpace{}%
% \AgdaFunction{EpicInv}\AgdaSpace{}%
% \AgdaBound{β}\AgdaSpace{}%
% \AgdaBound{βe}\<%
% \\
% %
% \\[\AgdaEmptyExtraSkip]%
% %
% \>[3]\AgdaFunction{ζ}\AgdaSpace{}%
% \AgdaSymbol{:}\AgdaSpace{}%
% \AgdaBound{β}\AgdaSpace{}%
% \AgdaSymbol{(}\AgdaFunction{βinv}\AgdaSpace{}%
% \AgdaBound{y}\AgdaSymbol{)}\AgdaSpace{}%
% \AgdaOperator{\AgdaDatatype{≡}}\AgdaSpace{}%
% \AgdaBound{y}\<%
% \\
% %
% \>[3]\AgdaFunction{ζ}\AgdaSpace{}%
% \AgdaSymbol{=}\AgdaSpace{}%
% \AgdaFunction{ap}\AgdaSpace{}%
% \AgdaSymbol{(λ}\AgdaSpace{}%
% \AgdaBound{-}\AgdaSpace{}%
% \AgdaSymbol{→}\AgdaSpace{}%
% \AgdaBound{-}\AgdaSpace{}%
% \AgdaBound{y}\AgdaSymbol{)}\AgdaSpace{}%
% \AgdaSymbol{(}\AgdaFunction{EpicInvIsRightInv}\AgdaSpace{}%
% \AgdaBound{fe}\AgdaSpace{}%
% \AgdaBound{β}\AgdaSpace{}%
% \AgdaBound{βe}\AgdaSymbol{)}\<%
% \\
% %
% \>[3]\AgdaFunction{η}\AgdaSpace{}%
% \AgdaSymbol{:}\AgdaSpace{}%
% \AgdaSymbol{(}\AgdaBound{ϕ}\AgdaSpace{}%
% \AgdaOperator{\AgdaFunction{∘}}\AgdaSpace{}%
% \AgdaBound{ξ}\AgdaSymbol{)}\AgdaSpace{}%
% \AgdaSymbol{(}\AgdaFunction{βinv}\AgdaSpace{}%
% \AgdaBound{y}\AgdaSymbol{)}\AgdaSpace{}%
% \AgdaOperator{\AgdaDatatype{≡}}\AgdaSpace{}%
% \AgdaBound{y}\<%
% \\
% %
% \>[3]\AgdaFunction{η}\AgdaSpace{}%
% \AgdaSymbol{=}\AgdaSpace{}%
% \AgdaSymbol{(}\AgdaFunction{ap}\AgdaSpace{}%
% \AgdaSymbol{(λ}\AgdaSpace{}%
% \AgdaBound{-}\AgdaSpace{}%
% \AgdaSymbol{→}\AgdaSpace{}%
% \AgdaBound{-}\AgdaSpace{}%
% \AgdaSymbol{(}\AgdaFunction{βinv}\AgdaSpace{}%
% \AgdaBound{y}\AgdaSymbol{))}\AgdaSpace{}%
% \AgdaSymbol{(}\AgdaBound{compId}\AgdaSpace{}%
% \AgdaOperator{\AgdaFunction{⁻¹}}\AgdaSymbol{))}\AgdaSpace{}%
% \AgdaOperator{\AgdaFunction{∙}}\AgdaSpace{}%
% \AgdaFunction{ζ}\<%
% \\
% %
% \>[3]\AgdaFunction{γ}\AgdaSpace{}%
% \AgdaSymbol{:}\AgdaSpace{}%
% \AgdaOperator{\AgdaDatatype{Image}}\AgdaSpace{}%
% \AgdaBound{ϕ}\AgdaSpace{}%
% \AgdaOperator{\AgdaDatatype{∋}}\AgdaSpace{}%
% \AgdaBound{y}\<%
% \\
% %
% \>[3]\AgdaFunction{γ}\AgdaSpace{}%
% \AgdaSymbol{=}\AgdaSpace{}%
% \AgdaInductiveConstructor{eq}\AgdaSpace{}%
% \AgdaBound{y}\AgdaSpace{}%
% \AgdaSymbol{(}\AgdaBound{ξ}\AgdaSpace{}%
% \AgdaSymbol{(}\AgdaFunction{βinv}\AgdaSpace{}%
% \AgdaBound{y}\AgdaSymbol{))}\AgdaSpace{}%
% \AgdaSymbol{(}\AgdaFunction{η}\AgdaSpace{}%
% \AgdaOperator{\AgdaFunction{⁻¹}}\AgdaSymbol{)}\<%
% \end{code}

\subsubsection{Embeddings}\label{embeddings}

% This is the first point at which \href{UALib.Preface.html\#truncation}{truncation} comes into play. An
% \href{https://www.cs.bham.ac.uk/~mhe/HoTT-UF-in-Agda-Lecture-Notes/HoTT-UF-Agda.html\#embeddings}{embedding}
% is defined in the \typetopology library, using the \af{is-subsingleton} type \href{Prelude.Extensionality.html\#alternative-extensionality-type}{described earlier}, as follows.

% Finally, t
The type \af{is-embedding} \ab f denotes the assertion that \ab f is a function all of whose fibers are subsingletons.
\ccpad
\begin{code}%
\>[1]\AgdaFunction{is-embedding}\AgdaSpace{}%
\AgdaSymbol{:}\AgdaSpace{}%
\AgdaSymbol{\{}\AgdaBound{X}\AgdaSpace{}%
\AgdaSymbol{:}\AgdaSpace{}%
\AgdaBound{𝓤}\AgdaSpace{}%
\AgdaOperator{\AgdaFunction{̇}}\AgdaSpace{}%
\AgdaSymbol{\}}\AgdaSpace{}%
\AgdaSymbol{\{}\AgdaBound{Y}\AgdaSpace{}%
\AgdaSymbol{:}\AgdaSpace{}%
\AgdaBound{𝓦}\AgdaSpace{}%
\AgdaOperator{\AgdaFunction{̇}}\AgdaSpace{}%
\AgdaSymbol{\}}\AgdaSpace{}%
\AgdaSymbol{→}\AgdaSpace{}%
\AgdaSymbol{(}\AgdaBound{X}\AgdaSpace{}%
\AgdaSymbol{→}\AgdaSpace{}%
\AgdaBound{Y}\AgdaSymbol{)}\AgdaSpace{}%
\AgdaSymbol{→}\AgdaSpace{}%
\AgdaBound{𝓤}\AgdaSpace{}%
\AgdaOperator{\AgdaPrimitive{⊔}}\AgdaSpace{}%
\AgdaBound{𝓦}\AgdaSpace{}%
\AgdaOperator{\AgdaFunction{̇}}\<%
\\
%
\>[1]\AgdaFunction{is-embedding}\AgdaSpace{}%
\AgdaBound{f}\AgdaSpace{}%
\AgdaSymbol{=}\AgdaSpace{}%
\AgdaSymbol{∀}\AgdaSpace{}%
\AgdaBound{y}\AgdaSpace{}%
\AgdaSymbol{→}\AgdaSpace{}%
\AgdaFunction{is-subsingleton}\AgdaSpace{}%
\AgdaSymbol{(}\AgdaFunction{fiber}\AgdaSpace{}%
\AgdaBound{f}\AgdaSpace{}%
\AgdaBound{y}\AgdaSymbol{)}\<%
\end{code}
\ccpad
This is a natural way to represent what we usually mean in mathematics by embedding. Observe that an embedding does not simply correspond to an injective map. However, if we assume that the codomain \ab B has unique identity proofs (i.e., \ab B is a \emph{set}), then we can prove that a monic function into \ab B is an embedding. We postpone this until we arrive at the \ualibTruncation module and take up the topic of sets.

It should be clear that embeddings are monic; from a proof \ab p \as : \af{is-embedding} \ab f that \ab f is an embedding we can construct a proof of \af{Monic} \ab f. We verify this as follows.
\ccpad
\begin{code}%
\>[0]\AgdaFunction{embedding-is-monic}\AgdaSpace{}%
\AgdaSymbol{:}%
\>[650I]\AgdaSymbol{\{}\AgdaBound{𝓧}\AgdaSpace{}%
\AgdaBound{𝓨}\AgdaSpace{}%
\AgdaSymbol{:}\AgdaSpace{}%
\AgdaPostulate{Universe}\AgdaSymbol{\}}\AgdaSpace{}%
\AgdaSymbol{\{}\AgdaBound{X}\AgdaSpace{}%
\AgdaSymbol{:}\AgdaSpace{}%
\AgdaBound{𝓧}\AgdaSpace{}%
\AgdaOperator{\AgdaFunction{̇}}\AgdaSymbol{\}\{}\AgdaBound{Y}\AgdaSpace{}%
\AgdaSymbol{:}\AgdaSpace{}%
\AgdaBound{𝓨}\AgdaSpace{}%
\AgdaOperator{\AgdaFunction{̇}}\AgdaSymbol{\}}\<%
\\
\>[.][@{}l@{}]\<[650I]%
\>[21]\AgdaSymbol{(}\AgdaBound{f}\AgdaSpace{}%
\AgdaSymbol{:}\AgdaSpace{}%
\AgdaBound{X}\AgdaSpace{}%
\AgdaSymbol{→}\AgdaSpace{}%
\AgdaBound{Y}\AgdaSymbol{)}\AgdaSpace{}%
\AgdaSymbol{→}\AgdaSpace{}%
\AgdaFunction{is-embedding}\AgdaSpace{}%
\AgdaBound{f}\AgdaSpace{}%
\AgdaSymbol{→}\AgdaSpace{}%
\AgdaFunction{Monic}\AgdaSpace{}%
\AgdaBound{f}\<%
\\
%
\\[\AgdaEmptyExtraSkip]%
\>[0]\AgdaFunction{embedding-is-monic}\AgdaSpace{}%
\AgdaBound{f}\AgdaSpace{}%
\AgdaBound{femb}\AgdaSpace{}%
\AgdaBound{x}\AgdaSpace{}%
\AgdaBound{x'}\AgdaSpace{}%
\AgdaBound{fxfx'}\AgdaSpace{}%
\AgdaSymbol{=}\AgdaSpace{}%
\AgdaFunction{ap}\AgdaSpace{}%
\AgdaFunction{pr₁}\AgdaSpace{}%
\AgdaSymbol{((}\AgdaBound{femb}\AgdaSpace{}%
\AgdaSymbol{(}\AgdaBound{f}\AgdaSpace{}%
\AgdaBound{x}\AgdaSymbol{))}\AgdaSpace{}%
\AgdaFunction{fa}\AgdaSpace{}%
\AgdaFunction{fb}\AgdaSymbol{)}\<%
\\
\>[0][@{}l@{\AgdaIndent{0}}]%
\>[1]\AgdaKeyword{where}\<%
\\
%
\>[1]\AgdaFunction{fa}\AgdaSpace{}%
\AgdaSymbol{:}\AgdaSpace{}%
\AgdaFunction{fiber}\AgdaSpace{}%
\AgdaBound{f}\AgdaSpace{}%
\AgdaSymbol{(}\AgdaBound{f}\AgdaSpace{}%
\AgdaBound{x}\AgdaSymbol{)}\<%
\\
%
\>[1]\AgdaFunction{fa}\AgdaSpace{}%
\AgdaSymbol{=}\AgdaSpace{}%
\AgdaBound{x}\AgdaSpace{}%
\AgdaOperator{\AgdaInductiveConstructor{,}}\AgdaSpace{}%
\AgdaInductiveConstructor{𝓇ℯ𝒻𝓁}\<%
\\
%
\\[\AgdaEmptyExtraSkip]%
%
\>[1]\AgdaFunction{fb}\AgdaSpace{}%
\AgdaSymbol{:}\AgdaSpace{}%
\AgdaFunction{fiber}\AgdaSpace{}%
\AgdaBound{f}\AgdaSpace{}%
\AgdaSymbol{(}\AgdaBound{f}\AgdaSpace{}%
\AgdaBound{x}\AgdaSymbol{)}\<%
\\
%
\>[1]\AgdaFunction{fb}\AgdaSpace{}%
\AgdaSymbol{=}\AgdaSpace{}%
\AgdaBound{x'}\AgdaSpace{}%
\AgdaOperator{\AgdaInductiveConstructor{,}}\AgdaSpace{}%
\AgdaSymbol{(}\AgdaBound{fxfx'}\AgdaSpace{}%
\AgdaOperator{\AgdaFunction{⁻¹}}\AgdaSymbol{)}\<%
\end{code}
\scpad

Finally, one way to show that a function is an embedding is to first prove it is invertible and then invoke the following theorem.
\ccpad
\begin{code}%
\>[0]\AgdaFunction{invertibles-are-embeddings}\AgdaSpace{}%
\AgdaSymbol{:}%
\>[621I]\AgdaSymbol{\{}\AgdaBound{𝓧}\AgdaSpace{}%
\AgdaBound{𝓨}\AgdaSpace{}%
\AgdaSymbol{:}\AgdaSpace{}%
\AgdaPostulate{Universe}\AgdaSymbol{\}}\AgdaSpace{}%
\AgdaSymbol{\{}\AgdaBound{X}\AgdaSpace{}%
\AgdaSymbol{:}\AgdaSpace{}%
\AgdaBound{𝓧}\AgdaSpace{}%
\AgdaOperator{\AgdaFunction{̇}}\AgdaSymbol{\}}\AgdaSpace{}%
\AgdaSymbol{\{}\AgdaBound{Y}\AgdaSpace{}%
\AgdaSymbol{:}\AgdaSpace{}%
\AgdaBound{𝓨}\AgdaSpace{}%
\AgdaOperator{\AgdaFunction{̇}}\AgdaSymbol{\}}\AgdaSpace{}%
\AgdaSymbol{(}\AgdaBound{f}\AgdaSpace{}%
\AgdaSymbol{:}\AgdaSpace{}%
\AgdaBound{X}\AgdaSpace{}%
\AgdaSymbol{→}\AgdaSpace{}%
\AgdaBound{Y}\AgdaSymbol{)}\<%
\\
\>[0][@{}l@{\AgdaIndent{0}}]%
\>[1]\AgdaSymbol{→}%
\>[.][@{}l@{}]\<[621I]%
\>[29]\AgdaFunction{invertible}\AgdaSpace{}%
\AgdaBound{f}\AgdaSpace{}%
\AgdaSymbol{→}\AgdaSpace{}%
\AgdaFunction{is-embedding}\AgdaSpace{}%
\AgdaBound{f}\<%
\\
%
\\[\AgdaEmptyExtraSkip]%
\>[0]\AgdaFunction{invertibles-are-embeddings}\AgdaSpace{}%
\AgdaBound{f}\AgdaSpace{}%
\AgdaBound{fi}\AgdaSpace{}%
\AgdaSymbol{=}\AgdaSpace{}%
\AgdaFunction{equivs-are-embeddings}\AgdaSpace{}%
\AgdaBound{f}\AgdaSpace{}%
\AgdaSymbol{(}\AgdaFunction{invertibles-are-equivs}\AgdaSpace{}%
\AgdaBound{f}\AgdaSpace{}%
\AgdaBound{fi}\AgdaSymbol{)}\<%
\end{code}

%%% Local Variables:
%%% mode: latex
%%% TeX-master: "ualib-part1-20210304.tex"
%%% End:


\subsection{Lifts}\label{sec:lifts-altern-univ}
% : making peace with a noncumulative universe hierarchy
\firstsentence{Overture}{Lifts}
\subsubsection{The noncumulative universe hierarchy}\label{the-noncumulative-hierarchy}

The hierarchy of universe levels in Agda looks like this:\\[-4pt]

\ab{𝓤₀} \as : \ab{𝓤₁}\AgdaOperator{\AgdaFunction{̇}} , ~ ~ \ab{𝓤₁} \as : \ab{𝓤₂}\AgdaOperator{\AgdaFunction{̇}} , ~ ~ \ab{𝓤₂} \as : \ab{𝓤₃}\AgdaOperator{\AgdaFunction{̇}} , \ldots{}\\[4pt]
This means that the type level of \ab{𝓤₀} is \ab{𝓤₁}\AgdaOperator{\AgdaFunction{̇}}, and for each \ab n the type level of \ab{𝓤ₙ} is \ab{𝓤ₙ₊₁}\AgdaOperator{\AgdaFunction{̇}}.
It is important to note, however, this does \emph{not} imply that \ab{𝓤₀} \as : \ab{𝓤₂}\AgdaOperator{\AgdaFunction{̇}} and \ab{𝓤₀} \as : \ab{𝓤₃}\AgdaOperator{\AgdaFunction{̇}}, and so on. In other words, Agda's universe hierarchy is \defn{noncummulative}. This makes it possible to treat universe levels more generally and precisely, which is nice. On the other hand (in this author's experience) a noncummulative hierarchy can sometimes make for a nonfun proof assistant.

Luckily, there are ways to overcome this technical issue. We describe general lifting and lowering functions below, and then later, in \S\ref{lifts-of-algebras},
 % \href{https://ualib.gitlab.io/Algebras.Algebras.html\#lifts-of-algebras}{Lifts
we'll see the domain-specific analogs of these tools which turn out to have some nice properties that make them very effective for resolving universe level problems when working with algebra datatypes.

\subsubsection{Lifting and lowering}\label{lifting-and-lowering}

Let us be more concrete about what is at issue here by giving an example. Agda frequently encounters errors during the type-checking process and responds by printing an error message. Often the message has the following form.
{\color{red}{\small
\begin{verbatim}
  Algebras.lagda:498,20-23 𝓤 != 𝓞 ⊔ 𝓥 ⊔ 𝓤 ⁺ when checking that... has type...
\end{verbatim}}}
\noindent This error message means that Agda encountered the universe \ab{𝓤} on line 498 (columns 20--23) of the file \af{Algebras.lagda}, but was expecting to find the universe \ab{𝓞} \aop ⊔ \ab 𝓥 \aop ⊔ \ab 𝓤\af ⁺ instead.

To make these situations easier to deal with, we have developed some domain specific tools for the lifting and lowering of universe levels inhabited by algebra types. (Later we do the same for other domain specific types like homomorphisms, subalgebras, products, etc). This must be done carefully to avoid making the type theory inconsistent. In particular, we cannot lower the level of a type unless it was previously lifted to a (higher than necessary) universe level.

A general \af{Lift} record type, similar to the one found in the \am{Level} module of the \agdastdlib, is defined as follows.
\ccpad
\begin{code}%
\>[0]\AgdaKeyword{record}\AgdaSpace{}%
\AgdaRecord{Lift}\AgdaSpace{}%
\AgdaSymbol{\{}\AgdaBound{𝓦}\AgdaSpace{}%
\AgdaBound{𝓤}\AgdaSpace{}%
\AgdaSymbol{:}\AgdaSpace{}%
\AgdaPostulate{Universe}\AgdaSymbol{\}}\AgdaSpace{}%
\AgdaSymbol{(}\AgdaBound{X}\AgdaSpace{}%
\AgdaSymbol{:}\AgdaSpace{}%
\AgdaBound{𝓤}\AgdaSpace{}%
\AgdaOperator{\AgdaFunction{̇}}\AgdaSymbol{)}\AgdaSpace{}%
\AgdaSymbol{:}\AgdaSpace{}%
\AgdaBound{𝓤}\AgdaSpace{}%
\AgdaOperator{\AgdaPrimitive{⊔}}\AgdaSpace{}%
\AgdaBound{𝓦}\AgdaSpace{}%
\AgdaOperator{\AgdaFunction{̇}}%
\>[50]\AgdaKeyword{where}\<%
\\
\>[0][@{}l@{\AgdaIndent{0}}]%
\>[1]\AgdaKeyword{constructor}\AgdaSpace{}%
\AgdaInductiveConstructor{lift}\<%
\\
%
\>[1]\AgdaKeyword{field}\AgdaSpace{}%
\AgdaField{lower}\AgdaSpace{}%
\AgdaSymbol{:}\AgdaSpace{}%
\AgdaBound{X}\<%
\\
\>[0]\AgdaKeyword{open}\AgdaSpace{}%
\AgdaModule{Lift}\<%
\end{code}
\ccpad
Using the \af{Lift} type, we define two functions that lift the domain (respectively, codomain) type of a function to a higher universe.
\ccpad
\begin{code}%
\>[1]\AgdaFunction{lift-dom}\AgdaSpace{}%
\AgdaSymbol{:}\AgdaSpace{}%
\AgdaSymbol{\{}\AgdaBound{X}\AgdaSpace{}%
\AgdaSymbol{:}\AgdaSpace{}%
\AgdaBound{𝓧}\AgdaSpace{}%
\AgdaOperator{\AgdaFunction{̇}}\AgdaSymbol{\}\{}\AgdaBound{Y}\AgdaSpace{}%
\AgdaSymbol{:}\AgdaSpace{}%
\AgdaBound{𝓨}\AgdaSpace{}%
\AgdaOperator{\AgdaFunction{̇}}\AgdaSymbol{\}}\AgdaSpace{}%
\AgdaSymbol{→}\AgdaSpace{}%
\AgdaSymbol{(}\AgdaBound{X}\AgdaSpace{}%
\AgdaSymbol{→}\AgdaSpace{}%
\AgdaBound{Y}\AgdaSymbol{)}\AgdaSpace{}%
\AgdaSymbol{→}\AgdaSpace{}%
\AgdaSymbol{(}\AgdaRecord{Lift}\AgdaSymbol{\{}\AgdaBound{𝓦}\AgdaSymbol{\}\{}\AgdaBound{𝓧}\AgdaSymbol{\}}\AgdaSpace{}%
\AgdaBound{X}\AgdaSpace{}%
\AgdaSymbol{→}\AgdaSpace{}%
\AgdaBound{Y}\AgdaSymbol{)}\<%
\\
%
\>[1]\AgdaFunction{lift-dom}\AgdaSpace{}%
\AgdaBound{f}\AgdaSpace{}%
\AgdaSymbol{=}\AgdaSpace{}%
\AgdaSymbol{λ}\AgdaSpace{}%
\AgdaBound{x}\AgdaSpace{}%
\AgdaSymbol{→}\AgdaSpace{}%
\AgdaSymbol{(}\AgdaBound{f}\AgdaSpace{}%
\AgdaSymbol{(}\AgdaField{lower}\AgdaSpace{}%
\AgdaBound{x}\AgdaSymbol{))}\<%
\\
%
\\[\AgdaEmptyExtraSkip]%
%
\>[1]\AgdaFunction{lift-cod}\AgdaSpace{}%
\AgdaSymbol{:}\AgdaSpace{}%
\AgdaSymbol{\{}\AgdaBound{X}\AgdaSpace{}%
\AgdaSymbol{:}\AgdaSpace{}%
\AgdaBound{𝓧}\AgdaSpace{}%
\AgdaOperator{\AgdaFunction{̇}}\AgdaSymbol{\}\{}\AgdaBound{Y}\AgdaSpace{}%
\AgdaSymbol{:}\AgdaSpace{}%
\AgdaBound{𝓨}\AgdaSpace{}%
\AgdaOperator{\AgdaFunction{̇}}\AgdaSymbol{\}}\AgdaSpace{}%
\AgdaSymbol{→}\AgdaSpace{}%
\AgdaSymbol{(}\AgdaBound{X}\AgdaSpace{}%
\AgdaSymbol{→}\AgdaSpace{}%
\AgdaBound{Y}\AgdaSymbol{)}\AgdaSpace{}%
\AgdaSymbol{→}\AgdaSpace{}%
\AgdaSymbol{(}\AgdaBound{X}\AgdaSpace{}%
\AgdaSymbol{→}\AgdaSpace{}%
\AgdaRecord{Lift}\AgdaSymbol{\{}\AgdaBound{𝓦}\AgdaSymbol{\}\{}\AgdaBound{𝓨}\AgdaSymbol{\}}\AgdaSpace{}%
\AgdaBound{Y}\AgdaSymbol{)}\<%
\\
%
\>[1]\AgdaFunction{lift-cod}\AgdaSpace{}%
\AgdaBound{f}\AgdaSpace{}%
\AgdaSymbol{=}\AgdaSpace{}%
\AgdaSymbol{λ}\AgdaSpace{}%
\AgdaBound{x}\AgdaSpace{}%
\AgdaSymbol{→}\AgdaSpace{}%
\AgdaInductiveConstructor{lift}\AgdaSpace{}%
\AgdaSymbol{(}\AgdaBound{f}\AgdaSpace{}%
\AgdaBound{x}\AgdaSymbol{)}\<%
\end{code}
\ccpad
Naturally, we can combine these and define a function that lifts both the domain and codomain type.
\ccpad
\begin{code}
\>[1]\AgdaFunction{lift-fun}\AgdaSpace{}%
\AgdaSymbol{:}\AgdaSpace{}%
\AgdaSymbol{\{}\AgdaBound{X}\AgdaSpace{}%
\AgdaSymbol{:}\AgdaSpace{}%
\AgdaBound{𝓧}\AgdaSpace{}%
\AgdaOperator{\AgdaFunction{̇}}\AgdaSymbol{\}\{}\AgdaBound{Y}\AgdaSpace{}%
\AgdaSymbol{:}\AgdaSpace{}%
\AgdaBound{𝓨}\AgdaSpace{}%
\AgdaOperator{\AgdaFunction{̇}}\AgdaSymbol{\}}\AgdaSpace{}%
\AgdaSymbol{→}\AgdaSpace{}%
\AgdaSymbol{(}\AgdaBound{X}\AgdaSpace{}%
\AgdaSymbol{→}\AgdaSpace{}%
\AgdaBound{Y}\AgdaSymbol{)}\AgdaSpace{}%
\AgdaSymbol{→}\AgdaSpace{}%
\AgdaSymbol{(}\AgdaRecord{Lift}\AgdaSymbol{\{}\AgdaBound{𝓦}\AgdaSymbol{\}\{}\AgdaBound{𝓧}\AgdaSymbol{\}}\AgdaSpace{}%
\AgdaBound{X}\AgdaSpace{}%
\AgdaSymbol{→}\AgdaSpace{}%
\AgdaRecord{Lift}\AgdaSymbol{\{}\AgdaBound{𝓩}\AgdaSymbol{\}\{}\AgdaBound{𝓨}\AgdaSymbol{\}}\AgdaSpace{}%
\AgdaBound{Y}\AgdaSymbol{)}\<%
\\
%
\>[1]\AgdaFunction{lift-fun}\AgdaSpace{}%
\AgdaBound{f}\AgdaSpace{}%
\AgdaSymbol{=}\AgdaSpace{}%
\AgdaSymbol{λ}\AgdaSpace{}%
\AgdaBound{x}\AgdaSpace{}%
\AgdaSymbol{→}\AgdaSpace{}%
\AgdaInductiveConstructor{lift}\AgdaSpace{}%
\AgdaSymbol{(}\AgdaBound{f}\AgdaSpace{}%
\AgdaSymbol{(}\AgdaField{lower}\AgdaSpace{}%
\AgdaBound{x}\AgdaSymbol{))}\<%
\end{code}
\ccpad
For example, \af{lift-dom} takes a function \ab f defined on the domain \AgdaBound{X}\AgdaSpace{}\AgdaSymbol{:}\AgdaSpace{}\AgdaBound{𝓧}\AgdaSpace{}\AgdaOperator{\AgdaFunction{̇}}\AgdaSpace{} and returns a function defined on the domain \ar{Lift}\as{\{}\ab 𝓦\as{\}}\as{\{}\ab 𝓧\as{\}} \ab X \as : \ab 𝓧 \aop ⊔ \ab 𝓦\aof ̇.

The point of a ramified hierarchy of universes is to avoid Russell's paradox, and this would be subverted if we lower the universe of a type that hasn't already be lifted.  Fortunately, however, we can prove that \aic{lift} and \afld{lower} compose to the identity. Later, there will be some situations that require these facts, so we formalize them and their (trivial) proofs.
\ccpad
\begin{code}%
\>[0]\AgdaFunction{lower∼lift}\AgdaSpace{}%
\AgdaSymbol{:}\AgdaSpace{}%
\AgdaSymbol{\{}\AgdaBound{𝓦}\AgdaSpace{}%
\AgdaBound{𝓧}\AgdaSpace{}%
\AgdaSymbol{:}\AgdaSpace{}%
\AgdaPostulate{Universe}\AgdaSymbol{\}\{}\AgdaBound{X}\AgdaSpace{}%
\AgdaSymbol{:}\AgdaSpace{}%
\AgdaBound{𝓧}\AgdaSpace{}%
\AgdaOperator{\AgdaFunction{̇}}\AgdaSymbol{\}}\AgdaSpace{}%
\AgdaSymbol{→}\AgdaSpace{}%
\AgdaField{lower}\AgdaSymbol{\{}\AgdaBound{𝓦}\AgdaSymbol{\}\{}\AgdaBound{𝓧}\AgdaSymbol{\}}\AgdaSpace{}%
\AgdaOperator{\AgdaFunction{∘}}\AgdaSpace{}%
\AgdaInductiveConstructor{lift}\AgdaSpace{}%
\AgdaOperator{\AgdaDatatype{≡}}\AgdaSpace{}%
\AgdaFunction{𝑖𝑑}\AgdaSpace{}%
\AgdaBound{X}\<%
\\
\>[0]\AgdaFunction{lower∼lift}\AgdaSpace{}%
\AgdaSymbol{=}\AgdaSpace{}%
\AgdaInductiveConstructor{refl}\AgdaSpace{}%
\AgdaSymbol{\AgdaUnderscore{}}\<%
\\
%
\\[\AgdaEmptyExtraSkip]%
\>[0]\AgdaFunction{lift∼lower}\AgdaSpace{}%
\AgdaSymbol{:}\AgdaSpace{}%
\AgdaSymbol{\{}\AgdaBound{𝓦}\AgdaSpace{}%
\AgdaBound{𝓧}\AgdaSpace{}%
\AgdaSymbol{:}\AgdaSpace{}%
\AgdaPostulate{Universe}\AgdaSymbol{\}\{}\AgdaBound{X}\AgdaSpace{}%
\AgdaSymbol{:}\AgdaSpace{}%
\AgdaBound{𝓧}\AgdaSpace{}%
\AgdaOperator{\AgdaFunction{̇}}\AgdaSymbol{\}}\AgdaSpace{}%
\AgdaSymbol{→}\AgdaSpace{}%
\AgdaInductiveConstructor{lift}\AgdaSpace{}%
\AgdaOperator{\AgdaFunction{∘}}\AgdaSpace{}%
\AgdaField{lower}\AgdaSpace{}%
\AgdaOperator{\AgdaDatatype{≡}}\AgdaSpace{}%
\AgdaFunction{𝑖𝑑}\AgdaSpace{}%
\AgdaSymbol{(}\AgdaRecord{Lift}\AgdaSymbol{\{}\AgdaBound{𝓦}\AgdaSymbol{\}\{}\AgdaBound{𝓧}\AgdaSymbol{\}}\AgdaSpace{}%
\AgdaBound{X}\AgdaSymbol{)}\<%
\\
\>[0]\AgdaFunction{lift∼lower}\AgdaSpace{}%
\AgdaSymbol{=}\AgdaSpace{}%
\AgdaInductiveConstructor{refl}\AgdaSpace{}%
\AgdaSymbol{\AgdaUnderscore{}}\<%
\end{code}


%%% Local Variables:
%%% mode: latex
%%% TeX-master: "ualib-part1-20210304.tex"
%%% End:





% % -*- TeX-master: "ualib-part1.tex" -*-
%%% Local Variables: 
%%% mode: latex
%%% TeX-engine: 'xetex
%%% End:
\section{Relation Types}\label{sec:relat-quot-types}
We now present the \agdaualib's \ualibhtml{Relations} module and its submodules. In \S\ref{sec:discrete-relations} we define types that represent \emph{unary} and \emph{binary relations}, which we refer to as ``discrete relations'' to contrast them with the (``continuous'') \emph{general} and \emph{dependent relations} that we introduce in \S\ref{sec:continuous-relations}. We call the latter ``continuous relations'' because they can have arbitrary arity (general relations) and they can be defined over arbitrary families of types (dependent relations).

\subsection{Discrete relations}\label{sec:discrete-relations}
% : predicates, axiom of extensionality, compatibility
\firstsentence{Relations}{Discrete}
% \context{%
% \{\ab 𝓤 \ab 𝓦 \ab 𝓧 \ab 𝓨 \ab 𝓩 \as : \AgdaPostulate{Universe}\}%
% \{\ab A \as : \ab 𝓤\AgdaOperator{\AgdaFunction{̇}}\}%
% \{\ab B \as : \ab 𝓦\AgdaOperator{\AgdaFunction{̇}}\}%
% \{\ab C \as : \ab 𝓩\AgdaOperator{\AgdaFunction{̇}}\}%
% }
% -*- TeX-master: "ualib-part1.tex" -*-
%%% Local Variables: 
%%% mode: latex
%%% TeX-engine: 'xetex
%%% End:
\paragraph*{Unary relations}

In set theory, given two sets \ab A and \ab P, we say that \ab P is a \defn{subset} of \ab A, and we write \ab P \af ⊆ \ab A, just in case \as ∀~\ab x~(\ab x~\af ∈~\ab P~\as →~\ab x~\af ∈~\ab A). We need a mechanism for representing this notion in Agda. A typical approach is to use a \emph{unary predicate} type which we will denote by \af{Pred} and define as follows.  Given two universes \ab 𝓤 \ab 𝓦 and a type \ab A~\as :~\ab 𝓤\af ̇, the type \af{Pred}~\ab A~\ab 𝓦 represents \emph{properties} that inhabitants of \ab A may or may not satisfy.  We write \ab P~\as :~\af{Pred}~\ab A~\ab 𝓤 to represent the % semantic concept of the 
collection of inhabitants of \ab A that satisfy (or belong to) \ab P. Here is the definition.\footnote{\label{relunary}cf.~\texttt{Relation/Unary.agda} in the \agdastdlib.}
\ccpad
\begin{code}%
\>[0]\AgdaFunction{Pred}\AgdaSpace{}%
\AgdaSymbol{:}\AgdaSpace{}%
% \AgdaSymbol{\{}\AgdaBound{𝓤}\AgdaSpace{}%
% \AgdaSymbol{:}\AgdaSpace{}%
% \AgdaPostulate{Universe}\AgdaSymbol{\}}\AgdaSpace{}%
% \AgdaSymbol{→}\AgdaSpace{}%
\AgdaBound{𝓤}\AgdaSpace{}%
\AgdaOperator{\AgdaFunction{̇}}\AgdaSpace{}%
\AgdaSymbol{→}\AgdaSpace{}%
\AgdaSymbol{(}\AgdaBound{𝓦}\AgdaSpace{}%
\AgdaSymbol{:}\AgdaSpace{}%
\AgdaPostulate{Universe}\AgdaSymbol{)}\AgdaSpace{}%
\AgdaSymbol{→}\AgdaSpace{}%
\AgdaBound{𝓤}\AgdaSpace{}%
\AgdaOperator{\AgdaPrimitive{⊔}}\AgdaSpace{}%
\AgdaBound{𝓦}\AgdaSpace{}%
\AgdaOperator{\AgdaPrimitive{⁺}}\AgdaSpace{}%
\AgdaOperator{\AgdaFunction{̇}}\<%
\\
\>[0]\AgdaFunction{Pred}\AgdaSpace{}%
\AgdaBound{A}\AgdaSpace{}%
\AgdaBound{𝓦}\AgdaSpace{}%
\AgdaSymbol{=}\AgdaSpace{}%
\AgdaBound{A}\AgdaSpace{}%
\AgdaSymbol{→}\AgdaSpace{}%
\AgdaBound{𝓦}\AgdaSpace{}%
\AgdaOperator{\AgdaFunction{̇}}\<%
\end{code}
\ccpad
This is a general unary predicate type, but by taking the codomain to be \af{Bool} = \{0, 1\}, we obtain the usual interpretation of set membership; that is,
% \footnote{The type \af 𝟚 is defined in the \AgdaModule{MGS-MLTT} module of \typtop.}  
for \ab P~\as :~\af{Pred}~\ab A~\af{Bool} and for each \abt{x}{A}, we would interpret \ab P~\ab x~\ad ≡ 0 to mean \ab x ∉ \ab A, and \ab P~\ab x~\ad ≡ 1 to mean \ab x ∈ \ab A.


% To reiterate, given a type \ab A \as : \ab 𝓤\af ̇, we think of \af{Pred} \ab A \ab 𝓦 as the type of a property that inhabitants of \ab A may or may not satisfy. If \ab P \as : \af{Pred} \ab A \ab 𝓦, then we view \ab P as a collection of inhabitants of type \ab A that ``satisfy property \ab P,'' or that ``belong to the subset \ab P of \ab A.''
% Later we consider predicates over the class of algebras in a given signature. In the \ualibhtml{Algebras} module we will define the type \af{Algebra} \ab 𝓤 \ab 𝑆 of \ab 𝑆-algebras with domain type \ab 𝓤\af ̇, and the type \af{Pred} (\af{Algebra} \ab 𝓤 \ab 𝑆) \ab 𝓦 will represent classes of \ab 𝑆-algebras with certain properties.





%\subsubsection{Membership and inclusion relations}\label{membership-and-inclusion-relations}
% Like the \agdastdlib, t
The \ualib includes types that represent the \emph{element inclusion} and \emph{subset inclusion} relations from set theory. For example, given a predicate \af P, we may represent that  ``\ab x belongs to \af P'' or that ``\ab x has property \af{P},'' by writing either \ab x \af ∈ \af P or \af P \ab x.  The definition of \af ∈ is standard, as is the definition of \af{⊆} for the \defn{subset} relation; nonetheless, here they are.\cref{relunary}
\ccpad
\begin{code}%
\>[0]\AgdaOperator{\AgdaFunction{\AgdaUnderscore{}∈\AgdaUnderscore{}}}\AgdaSpace{}%
\AgdaSymbol{:}\AgdaSpace{}%
\AgdaBound{A}\AgdaSpace{}%
\AgdaSymbol{→}\AgdaSpace{}%
\AgdaFunction{Pred}\AgdaSpace{}%
\AgdaBound{A}\AgdaSpace{}%
\AgdaBound{𝓦}\AgdaSpace{}%
\AgdaSymbol{→}\AgdaSpace{}%
\AgdaBound{𝓦}\AgdaSpace{}%
\AgdaOperator{\AgdaFunction{̇}}\<%
\\
\>[0]\AgdaBound{x}\AgdaSpace{}%
\AgdaOperator{\AgdaFunction{∈}}\AgdaSpace{}%
\AgdaBound{P}\AgdaSpace{}%
\AgdaSymbol{=}\AgdaSpace{}%
\AgdaBound{P}\AgdaSpace{}%
\AgdaBound{x}\<%
\end{code}
\scpad
\begin{code}%
\>[1]\AgdaOperator{\AgdaFunction{\AgdaUnderscore{}⊆\AgdaUnderscore{}}}\AgdaSpace{}%
\AgdaSymbol{:}\AgdaSpace{}%
\AgdaFunction{Pred}\AgdaSpace{}%
\AgdaBound{A}\AgdaSpace{}%
\AgdaBound{𝓦}\AgdaSpace{}%
\AgdaSymbol{→}\AgdaSpace{}%
\AgdaFunction{Pred}\AgdaSpace{}%
\AgdaBound{A}\AgdaSpace{}%
\AgdaBound{𝓩}\AgdaSpace{}%
\AgdaSymbol{→}\AgdaSpace{}%
\AgdaBound{𝓤}\AgdaSpace{}%
\AgdaOperator{\AgdaPrimitive{⊔}}\AgdaSpace{}%
\AgdaBound{𝓦}\AgdaSpace{}%
\AgdaOperator{\AgdaPrimitive{⊔}}\AgdaSpace{}%
\AgdaBound{𝓩}\AgdaSpace{}%
\AgdaOperator{\AgdaFunction{̇}}\<%
\\
%
\>[1]\AgdaBound{P}\AgdaSpace{}%
\AgdaOperator{\AgdaFunction{⊆}}\AgdaSpace{}%
\AgdaBound{Q}\AgdaSpace{}%
\AgdaSymbol{=}\AgdaSpace{}%
\AgdaSymbol{∀}\AgdaSpace{}%
\AgdaSymbol{\{}\AgdaBound{x}\AgdaSymbol{\}}\AgdaSpace{}%
\AgdaSymbol{→}\AgdaSpace{}%
\AgdaBound{x}\AgdaSpace{}%
\AgdaOperator{\AgdaFunction{∈}}\AgdaSpace{}%
\AgdaBound{P}\AgdaSpace{}%
\AgdaSymbol{→}\AgdaSpace{}%
\AgdaBound{x}\AgdaSpace{}%
\AgdaOperator{\AgdaFunction{∈}}\AgdaSpace{}%
\AgdaBound{Q}\<%
\end{code}


\paragraph*{Predicates toolbox}
Here is a small collection of tools that will come in handy later. The first is an inductive type that represents \defn{disjoint union}.\footnote{\label{uhints}%
  \textbf{Unicode Hints}. In \agdamode, \texttt{\textbackslash{}u+} ↝ \af{⊎}, \texttt{\textbackslash{}b0} ↝ \af 𝟘, \texttt{\textbackslash{}B0} ↝ \af 𝟎.}
\ccpad % \texttt{\textbackslash{}.=} ↝ \af{≐}, , \texttt{\textbackslash{}b1} ↝ \af 𝟙, \texttt{\textbackslash{}B1} ↝ \af 𝟏
\begin{code}
\>[0]\AgdaKeyword{data}\AgdaSpace{}%
\AgdaOperator{\AgdaDatatype{\AgdaUnderscore{}⊎\AgdaUnderscore{}}}\AgdaSpace{}%
% \AgdaSymbol{\{}\AgdaBound{𝓤}\AgdaSpace{}%
% \AgdaBound{𝓦}\AgdaSpace{}%
% \AgdaSymbol{:}\AgdaSpace{}%
% \AgdaPostulate{Universe}\AgdaSymbol{\}}
\AgdaSymbol{(}\AgdaBound{A}\AgdaSpace{}%
\AgdaSymbol{:}\AgdaSpace{}%
\AgdaBound{𝓤}\AgdaSpace{}%
\AgdaOperator{\AgdaFunction{̇}}\AgdaSymbol{)}\AgdaSpace{}%
\AgdaSymbol{(}\AgdaBound{B}\AgdaSpace{}%
\AgdaSymbol{:}\AgdaSpace{}%
\AgdaBound{𝓦}\AgdaSpace{}%
\AgdaOperator{\AgdaFunction{̇}}\AgdaSymbol{)}\AgdaSpace{}%
\AgdaSymbol{:}\AgdaSpace{}%
\AgdaBound{𝓤}\AgdaSpace{}%
\AgdaOperator{\AgdaPrimitive{⊔}}\AgdaSpace{}%
\AgdaBound{𝓦}\AgdaSpace{}%
\AgdaOperator{\AgdaFunction{̇}}\AgdaSpace{}%
\AgdaKeyword{where}\<%
\\
\>[0][@{}l@{\AgdaIndent{0}}]%
\>[1]\AgdaInductiveConstructor{inj₁}\AgdaSpace{}%
\AgdaSymbol{:}\AgdaSpace{}%
\AgdaSymbol{(}\AgdaBound{x}\AgdaSpace{}%
\AgdaSymbol{:}\AgdaSpace{}%
\AgdaBound{A}\AgdaSymbol{)}\AgdaSpace{}%
\AgdaSymbol{→}\AgdaSpace{}%
\AgdaBound{A}\AgdaSpace{}%
\AgdaOperator{\AgdaDatatype{⊎}}\AgdaSpace{}%
\AgdaBound{B}\<%
\\
%
\>[1]\AgdaInductiveConstructor{inj₂}\AgdaSpace{}%
\AgdaSymbol{:}\AgdaSpace{}%
\AgdaSymbol{(}\AgdaBound{y}\AgdaSpace{}%
\AgdaSymbol{:}\AgdaSpace{}%
\AgdaBound{B}\AgdaSymbol{)}\AgdaSpace{}%
\AgdaSymbol{→}\AgdaSpace{}%
\AgdaBound{A}\AgdaSpace{}%
\AgdaOperator{\AgdaDatatype{⊎}}\AgdaSpace{}%
\AgdaBound{B}\<%
\end{code}
\ccpad
And this can be used to define a type representing \defn{union}, as follows.
\ccpad
\begin{code}
\>[0]\AgdaOperator{\AgdaFunction{\AgdaUnderscore{}∪\AgdaUnderscore{}}}\AgdaSpace{}%
\AgdaSymbol{:}\AgdaSpace{}%
% \AgdaSymbol{\{}\AgdaBound{𝓤}\AgdaSpace{}%
% \AgdaBound{𝓦}\AgdaSpace{}%
% \AgdaBound{𝓩}\AgdaSpace{}%
% \AgdaSymbol{:}\AgdaSpace{}%
% \AgdaPostulate{Universe}\AgdaSymbol{\}\{}\AgdaBound{A}\AgdaSpace{}%
% \AgdaSymbol{:}\AgdaSpace{}%
% \AgdaBound{𝓤}\AgdaSpace{}%
% \AgdaOperator{\AgdaFunction{̇}}\AgdaSymbol{\}}\AgdaSpace{}%
% \AgdaSymbol{→}\AgdaSpace{}%
\AgdaFunction{Pred}\AgdaSpace{}%
\AgdaBound{A}\AgdaSpace{}%
\AgdaBound{𝓦}\AgdaSpace{}%
\AgdaSymbol{→}\AgdaSpace{}%
\AgdaFunction{Pred}\AgdaSpace{}%
\AgdaBound{A}\AgdaSpace{}%
\AgdaBound{𝓩}\AgdaSpace{}%
\AgdaSymbol{→}\AgdaSpace{}%
\AgdaFunction{Pred}\AgdaSpace{}%
\AgdaBound{A}\AgdaSpace{}%
\AgdaSymbol{(}\AgdaBound{𝓦}\AgdaSpace{}%
\AgdaOperator{\AgdaPrimitive{⊔}}\AgdaSpace{}%
\AgdaBound{𝓩}\AgdaSymbol{)}\<%
\\
\>[0]\AgdaBound{P}\AgdaSpace{}%
\AgdaOperator{\AgdaFunction{∪}}\AgdaSpace{}%
\AgdaBound{Q}\AgdaSpace{}%
\AgdaSymbol{=}\AgdaSpace{}%
\AgdaSymbol{λ}\AgdaSpace{}%
\AgdaBound{x}\AgdaSpace{}%
\AgdaSymbol{→}\AgdaSpace{}%
\AgdaBound{x}\AgdaSpace{}%
\AgdaOperator{\AgdaFunction{∈}}\AgdaSpace{}%
\AgdaBound{P}\AgdaSpace{}%
\AgdaOperator{\AgdaDatatype{⊎}}\AgdaSpace{}%
\AgdaBound{x}\AgdaSpace{}%
\AgdaOperator{\AgdaFunction{∈}}\AgdaSpace{}%
\AgdaBound{Q}\<%
\end{code}
\ccpad
Next we define convenient notation for asserting that the image of a function (the first argument) is contained in a predicate (the second argument).
\ccpad
\begin{code}
\>[1]\AgdaOperator{\AgdaFunction{Im\AgdaUnderscore{}⊆\AgdaUnderscore{}}}\AgdaSpace{}%
\AgdaSymbol{:}\AgdaSpace{}%
\AgdaSymbol{(}\AgdaBound{A}\AgdaSpace{}%
\AgdaSymbol{→}\AgdaSpace{}%
\AgdaBound{B}\AgdaSymbol{)}\AgdaSpace{}%
\AgdaSymbol{→}\AgdaSpace{}%
\AgdaFunction{Pred}\AgdaSpace{}%
\AgdaBound{B}\AgdaSpace{}%
\AgdaBound{𝓩}\AgdaSpace{}%
\AgdaSymbol{→}\AgdaSpace{}%
\AgdaBound{𝓤}\AgdaSpace{}%
\AgdaOperator{\AgdaPrimitive{⊔}}\AgdaSpace{}%
\AgdaBound{𝓩}\AgdaSpace{}%
\AgdaOperator{\AgdaFunction{̇}}\<%
\\
%
\>[1]\AgdaOperator{\AgdaFunction{Im}}\AgdaSpace{}%
\AgdaBound{f}\AgdaSpace{}%
\AgdaOperator{\AgdaFunction{⊆}}\AgdaSpace{}%
\AgdaBound{S}\AgdaSpace{}%
\AgdaSymbol{=}\AgdaSpace{}%
\AgdaSymbol{∀}\AgdaSpace{}%
\AgdaBound{x}\AgdaSpace{}%
\AgdaSymbol{→}\AgdaSpace{}%
\AgdaBound{f}\AgdaSpace{}%
\AgdaBound{x}\AgdaSpace{}%
\AgdaOperator{\AgdaFunction{∈}}\AgdaSpace{}%
\AgdaBound{S}\<%
\end{code}
\ccpad
The \defn{empty set} is naturally represented by the \defn{empty type}, \af 𝟘, and the latter is defined in the \am{Empty-Type} module of \typtop.\cref{uhints}$^, $\footnote{%
The empty type is an inductive type with no constructors; that is, \AgdaKeyword{data}\AgdaSpace{}%
\AgdaDatatype{𝟘}\AgdaSpace{}%
\AgdaSymbol{\{}\AgdaBound{𝓤}\AgdaSymbol{\}}\AgdaSpace{}%
\AgdaSymbol{:}\AgdaSpace{}%
\AgdaBound{𝓤}\AgdaSpace{}%
\AgdaOperator{\AgdaFunction{̇}}\AgdaSpace{}%
\AgdaKeyword{where}\AgdaSpace{}%
\AgdaComment{-- (empty body)}.}
\ccpad
\begin{code}
% \>[0]\AgdaKeyword{open}\AgdaSpace{}%
% \AgdaKeyword{import}\AgdaSpace{}%
% \AgdaModule{Empty-Type}\AgdaSpace{}%
% \AgdaKeyword{using}\AgdaSpace{}%
% \AgdaSymbol{(}\AgdaDatatype{𝟘}\AgdaSymbol{)}\<%
% \\
% %
% \\[\AgdaEmptyExtraSkip]%
\>[0]\AgdaFunction{∅}\AgdaSpace{}%
\AgdaSymbol{:}\AgdaSpace{}%
% \AgdaSymbol{\{}\AgdaBound{𝓤}\AgdaSpace{}%
% \AgdaSymbol{:}\AgdaSpace{}%
% \AgdaPostulate{Universe}\AgdaSymbol{\}\{}\AgdaBound{A}\AgdaSpace{}%
% \AgdaSymbol{:}\AgdaSpace{}%
% \AgdaBound{𝓤}\AgdaSpace{}%
% \AgdaOperator{\AgdaFunction{̇}}\AgdaSymbol{\}}\AgdaSpace{}%
% \AgdaSymbol{→}\AgdaSpace{}%
\AgdaFunction{Pred}\AgdaSpace{}%
\AgdaBound{A}\AgdaSpace{}%
\AgdaPrimitive{𝓤₀}\<%
\\
\>[0]\AgdaFunction{∅}\AgdaSpace{}%
\AgdaSymbol{\AgdaUnderscore{}}\AgdaSpace{}%
\AgdaSymbol{=}\AgdaSpace{}%
\AgdaDatatype{𝟘}\<%
\end{code}
\ccpad
We close our little predicates toolbox with a natural way to encode \defn{singletons}.
\ccpad
\begin{code}
\>[0]\AgdaOperator{\AgdaFunction{{\AgdaUnderscore{}}}}\AgdaSpace{}%
\AgdaSymbol{:}\AgdaSpace{}%
% \AgdaSymbol{\{}\AgdaBound{𝓤}\AgdaSpace{}%
% \AgdaSymbol{:}\AgdaSpace{}%
% \AgdaPostulate{Universe}\AgdaSymbol{\}\{}\AgdaBound{A}\AgdaSpace{}%
% \AgdaSymbol{:}\AgdaSpace{}%
% \AgdaBound{𝓤}\AgdaSpace{}%
% \AgdaOperator{\AgdaFunction{̇}}\AgdaSymbol{\}}\AgdaSpace{}%
% \AgdaSymbol{→}\AgdaSpace{}%
\AgdaBound{A}\AgdaSpace{}%
\AgdaSymbol{→}\AgdaSpace{}%
\AgdaFunction{Pred}\AgdaSpace{}%
\AgdaBound{A}\AgdaSpace{}%
\AgdaSymbol{\AgdaUnderscore{}}\<%
\\
\>[0]\AgdaOperator{\AgdaFunction{{}}\AgdaSpace{}%
\AgdaBound{x}\AgdaSpace{}%
\AgdaOperator{\AgdaFunction{}}}\AgdaSpace{}%
\AgdaSymbol{=}\AgdaSpace{}%
\AgdaBound{x}\AgdaSpace{}%
\AgdaOperator{\AgdaDatatype{≡\AgdaUnderscore{}}}\<%
\end{code}





\paragraph*{Binary Relations} %\label{sec:binary-relations}
In set theory, a binary relation on a set \ab{A} is simply a subset of the Cartesian product \ab A × \ab A. As such, we could model such a relation as a (unary) predicate over the product type \ab A \af × \ab A, or as an inhabitant of the function type \ab A \as → \ab A \as → \ab 𝓦\af ̇ (for some universe \ab 𝓦).  Note, however, this is not the same as a unary predicate over the function type \ab A \as → \ab A since the latter has type  (\ab A~\as →~\ab A)~\as →~\ab 𝓦\af ̇, while a binary relation should have type \ab A~\as →~(\ab A~\as →~\ab 𝓦\af ̇). 
% \footnote{Note that a binary relation from \ab A to \ab B is not simply a unary predicate over the type \ab A~\as →~\ab B.  The binary relation has type \ab A~\as →~(\ab B~\as →~\ab 𝓩\af ̇) whereas a unary predicate over \ab A~\as →~\ab B has type (\ab A~\as →~\ab B)~\as →~\ab 𝓩\af ̇.}

A generalization of the notion of binary relation is a \emph{relation from} \ab{A} \emph{to} \ab{B}, which we define first and treat binary relations on a single \ab{A} as a special case.
\ccpad
\begin{code}
\>[1]\AgdaFunction{REL}\AgdaSpace{}%
\AgdaSymbol{:}\AgdaSpace{}%
\AgdaBound{𝓤}%
\AgdaOperator{\AgdaFunction{̇}}\AgdaSpace{}%
\AgdaSymbol{→}\AgdaSpace{}%
\AgdaBound{𝓦}%
\AgdaOperator{\AgdaFunction{̇}}\AgdaSpace{}%
\AgdaSymbol{→}\AgdaSpace{}%
\AgdaSymbol{(}\AgdaBound{𝓩}\AgdaSpace{}%
\AgdaSymbol{:}\AgdaSpace{}%
\AgdaPostulate{Universe}\AgdaSymbol{)}\AgdaSpace{}%
\AgdaSymbol{→}\AgdaSpace{}%
\AgdaBound{𝓤}\AgdaSpace{}%
\AgdaOperator{\AgdaPrimitive{⊔}}\AgdaSpace{}%
\AgdaBound{𝓦}\AgdaSpace{}%
\AgdaOperator{\AgdaPrimitive{⊔}}\AgdaSpace{}%
\AgdaBound{𝓩}\AgdaSpace{}%
\AgdaOperator{\AgdaPrimitive{⁺}}\AgdaSpace{}%
\AgdaOperator{\AgdaFunction{̇}}\<%
\\
%
\>[1]\AgdaFunction{REL}\AgdaSpace{}%
\AgdaBound{A}\AgdaSpace{}%
\AgdaBound{B}\AgdaSpace{}%
\AgdaBound{𝓩}\AgdaSpace{}%
\AgdaSymbol{=}\AgdaSpace{}%
\AgdaBound{A}\AgdaSpace{}%
\AgdaSymbol{→}\AgdaSpace{}%
\AgdaBound{B}\AgdaSpace{}%
\AgdaSymbol{→}\AgdaSpace{}%
\AgdaBound{𝓩}%
\AgdaOperator{\AgdaFunction{̇}}\<%
\\
%
\\[\AgdaEmptyExtraSkip]%
%
\>[1]\AgdaFunction{Rel}\AgdaSpace{}%
\AgdaSymbol{:}\AgdaSpace{}%
\AgdaBound{𝓤}%
\AgdaOperator{\AgdaFunction{̇}}\AgdaSpace{}%
\AgdaSymbol{→}\AgdaSpace{}%
\AgdaSymbol{(}\AgdaBound{𝓩}\AgdaSpace{}%
\AgdaSymbol{:}\AgdaSpace{}%
\AgdaPostulate{Universe}\AgdaSymbol{)}\AgdaSpace{}%
\AgdaSymbol{→}\AgdaSpace{}%
\AgdaBound{𝓤}\AgdaSpace{}%
\AgdaOperator{\AgdaPrimitive{⊔}}\AgdaSpace{}%
\AgdaBound{𝓩}\AgdaSpace{}%
\AgdaOperator{\AgdaPrimitive{⁺}}%
\AgdaOperator{\AgdaFunction{̇}}\<%
\\
%
\>[1]\AgdaFunction{Rel}\AgdaSpace{}%
\AgdaBound{A}\AgdaSpace{}%
\AgdaBound{𝓩}\AgdaSpace{}%
\AgdaSymbol{=}\AgdaSpace{}%
\AgdaFunction{REL}\AgdaSpace{}%
\AgdaBound{A}\AgdaSpace{}%
\AgdaBound{A}\AgdaSpace{}%
\AgdaBound{𝓩}\<%
\end{code}

\paragraph*{The kernel of a function}
The \defn{kernel} of a function \ab f \as : \ab A \AgdaSymbol{→} \ab B is defined informally by \{(\ab x , \ab y) ∈ \ab A × \ab A : \ab f \ab x = \ab f \ab y\}. This can be represented in type theory in a number of ways, each of which may be useful in a particular context. For example, we could define the kernel to be an inhabitant of a (binary) relation type, a (unary) predicate type, a (curried) Sigma type, or an (uncurried) Sigma type. Since the first two alternatives are the ones we use thoughout the \ualib, we present them here.
\ccpad
\begin{code}%
\>[1]\AgdaFunction{ker}\AgdaSpace{}%
\AgdaSymbol{:}\AgdaSpace{}%
\AgdaSymbol{(}\AgdaBound{A}\AgdaSpace{}%
\AgdaSymbol{→}\AgdaSpace{}%
\AgdaBound{B}\AgdaSymbol{)}\AgdaSpace{}%
\AgdaSymbol{→}\AgdaSpace{}%
\AgdaFunction{Rel}\AgdaSpace{}%
\AgdaBound{A}\AgdaSpace{}%
\AgdaBound{𝓦}\<%
\\
%
\>[1]\AgdaFunction{ker}\AgdaSpace{}%
\AgdaBound{g}\AgdaSpace{}%
\AgdaBound{x}\AgdaSpace{}%
\AgdaBound{y}\AgdaSpace{}%
\AgdaSymbol{=}\AgdaSpace{}%
\AgdaBound{g}\AgdaSpace{}%
\AgdaBound{x}\AgdaSpace{}%
\AgdaOperator{\AgdaDatatype{≡}}\AgdaSpace{}%
\AgdaBound{g}\AgdaSpace{}%
\AgdaBound{y}\<%
\\
%
\\[\AgdaEmptyExtraSkip]%
%
\>[1]\AgdaFunction{kernel}\AgdaSpace{}%
\AgdaSymbol{:}\AgdaSpace{}%
\AgdaSymbol{(}\AgdaBound{A}\AgdaSpace{}%
\AgdaSymbol{→}\AgdaSpace{}%
\AgdaBound{B}\AgdaSymbol{)}\AgdaSpace{}%
\AgdaSymbol{→}\AgdaSpace{}%
\AgdaFunction{Pred}\AgdaSpace{}%
\AgdaSymbol{(}\AgdaBound{A}\AgdaSpace{}%
\AgdaOperator{\AgdaFunction{×}}\AgdaSpace{}%
\AgdaBound{A}\AgdaSymbol{)}\AgdaSpace{}%
\AgdaBound{𝓦}\<%
\\
%
\>[1]\AgdaFunction{kernel}\AgdaSpace{}%
\AgdaBound{g}\AgdaSpace{}%
\AgdaSymbol{(}\AgdaBound{x}\AgdaSpace{}%
\AgdaOperator{\AgdaInductiveConstructor{,}}\AgdaSpace{}%
\AgdaBound{y}\AgdaSymbol{)}\AgdaSpace{}%
\AgdaSymbol{=}\AgdaSpace{}%
\AgdaBound{g}\AgdaSpace{}%
\AgdaBound{x}\AgdaSpace{}%
\AgdaOperator{\AgdaDatatype{≡}}\AgdaSpace{}%
\AgdaBound{g}\AgdaSpace{}%
\AgdaBound{y}\<%
% \\
% %
% \\[\AgdaEmptyExtraSkip]%
% %
% \>[1]\AgdaFunction{ker-sigma}\AgdaSpace{}%
% \AgdaSymbol{:}\AgdaSpace{}%
% \AgdaSymbol{(}\AgdaBound{A}\AgdaSpace{}%
% \AgdaSymbol{→}\AgdaSpace{}%
% \AgdaBound{B}\AgdaSymbol{)}\AgdaSpace{}%
% \AgdaSymbol{→}\AgdaSpace{}%
% \AgdaBound{𝓤}\AgdaSpace{}%
% \AgdaOperator{\AgdaPrimitive{⊔}}\AgdaSpace{}%
% \AgdaBound{𝓦}\AgdaSpace{}%
% \AgdaOperator{\AgdaFunction{̇}}\<%
% \\
% %
% \>[1]\AgdaFunction{ker-sigma}\AgdaSpace{}%
% \AgdaBound{g}\AgdaSpace{}%
% \AgdaSymbol{=}\AgdaSpace{}%
% \AgdaFunction{Σ}\AgdaSpace{}%
% \AgdaBound{x}\AgdaSpace{}%
% \AgdaFunction{꞉}\AgdaSpace{}%
% \AgdaBound{A}\AgdaSpace{}%
% \AgdaFunction{,}\AgdaSpace{}%
% \AgdaFunction{Σ}\AgdaSpace{}%
% \AgdaBound{y}\AgdaSpace{}%
% \AgdaFunction{꞉}\AgdaSpace{}%
% \AgdaBound{A}\AgdaSpace{}%
% \AgdaFunction{,}\AgdaSpace{}%
% \AgdaBound{g}\AgdaSpace{}%
% \AgdaBound{x}\AgdaSpace{}%
% \AgdaOperator{\AgdaDatatype{≡}}\AgdaSpace{}%
% \AgdaBound{g}\AgdaSpace{}%
% \AgdaBound{y}\<%
% \\
% %
% \\[\AgdaEmptyExtraSkip]%
% %
% \>[1]\AgdaFunction{ker-sigma'}\AgdaSpace{}%
% \AgdaSymbol{:}\AgdaSpace{}%
% \AgdaSymbol{(}\AgdaBound{A}\AgdaSpace{}%
% \AgdaSymbol{→}\AgdaSpace{}%
% \AgdaBound{B}\AgdaSymbol{)}\AgdaSpace{}%
% \AgdaSymbol{→}\AgdaSpace{}%
% \AgdaBound{𝓤}\AgdaSpace{}%
% \AgdaOperator{\AgdaPrimitive{⊔}}\AgdaSpace{}%
% \AgdaBound{𝓦}\AgdaSpace{}%
% \AgdaOperator{\AgdaFunction{̇}}\<%
% \\
% %
% \>[1]\AgdaFunction{ker-sigma'}\AgdaSpace{}%
% \AgdaBound{g}\AgdaSpace{}%
% \AgdaSymbol{=}\AgdaSpace{}%
% \AgdaFunction{Σ}\AgdaSpace{}%
% \AgdaBound{(x}\AgdaSpace{}%
% \AgdaBound{,}\AgdaSpace{}%
% \AgdaBound{y)}\AgdaSpace{}%
% \AgdaFunction{꞉}\AgdaSpace{}%
% \AgdaSymbol{(}\AgdaBound{A}\AgdaSpace{}%
% \AgdaOperator{\AgdaFunction{×}}\AgdaSpace{}%
% \AgdaBound{A}\AgdaSymbol{)}\AgdaSpace{}%
% \AgdaFunction{,}\AgdaSpace{}%
% \AgdaBound{g}\AgdaSpace{}%
% \AgdaBound{x}\AgdaSpace{}%
% \AgdaOperator{\AgdaDatatype{≡}}\AgdaSpace{}%
% \AgdaBound{g}\AgdaSpace{}%
% \AgdaBound{y}\<%
\end{code}
\ccpad
Similarly, the \defn{identity relation} (which is equivalent to the kernel of an injective function) can be represented by a number of different types. Here we only show the representation that we use later to construct the zero congruence. The notation we use here is close to that of conventional algebra notation, where $0_A$ is used to denote the identity relation $\{(x, y) \in A \times A : x = y\}$.\cref{uhints}
\ccpad
\begin{code}%
\>[1]\AgdaFunction{𝟎}\AgdaSpace{}%
\AgdaSymbol{:}\AgdaSpace{}%
\AgdaFunction{Rel}\AgdaSpace{}%
\AgdaBound{A}\AgdaSpace{}%
\AgdaBound{𝓤}\<%
\\
%
\>[1]\AgdaFunction{𝟎}\AgdaSpace{}%
\AgdaBound{x}\AgdaSpace{}%
\AgdaBound{y}\AgdaSpace{}%
\AgdaSymbol{=}\AgdaSpace{}%
\AgdaBound{x}\AgdaSpace{}%
\AgdaOperator{\AgdaDatatype{≡}}\AgdaSpace{}%
\AgdaBound{y}\<%
% \\
% %
% \\[\AgdaEmptyExtraSkip]%
% %
% \>[1]\AgdaFunction{𝟎-pred}\AgdaSpace{}%
% \AgdaSymbol{:}\AgdaSpace{}%
% \AgdaFunction{Pred}\AgdaSpace{}%
% \AgdaSymbol{(}\AgdaBound{A}\AgdaSpace{}%
% \AgdaOperator{\AgdaFunction{×}}\AgdaSpace{}%
% \AgdaBound{A}\AgdaSymbol{)}\AgdaSpace{}%
% \AgdaBound{𝓤}\<%
% \\
% %
% \>[1]\AgdaFunction{𝟎-pred}\AgdaSpace{}%
% \AgdaSymbol{(}\AgdaBound{x}\AgdaSpace{}%
% \AgdaOperator{\AgdaInductiveConstructor{,}}\AgdaSpace{}%
% \AgdaBound{y}\AgdaSymbol{)}\AgdaSpace{}%
% \AgdaSymbol{=}\AgdaSpace{}%
% \AgdaBound{x}\AgdaSpace{}%
% \AgdaOperator{\AgdaDatatype{≡}}\AgdaSpace{}%
% \AgdaBound{y}\<%
% \\
% %
% \\[\AgdaEmptyExtraSkip]%
% %
% \>[1]\AgdaFunction{𝟎-sigma}\AgdaSpace{}%
% \AgdaSymbol{:}\AgdaSpace{}%
% \AgdaBound{𝓤}\AgdaSpace{}%
% \AgdaOperator{\AgdaFunction{̇}}\<%
% \\
% %
% \>[1]\AgdaFunction{𝟎-sigma}\AgdaSpace{}%
% \AgdaSymbol{=}\AgdaSpace{}%
% \AgdaFunction{Σ}\AgdaSpace{}%
% \AgdaBound{x}\AgdaSpace{}%
% \AgdaFunction{꞉}\AgdaSpace{}%
% \AgdaBound{A}\AgdaSpace{}%
% \AgdaFunction{,}\AgdaSpace{}%
% \AgdaFunction{Σ}\AgdaSpace{}%
% \AgdaBound{y}\AgdaSpace{}%
% \AgdaFunction{꞉}\AgdaSpace{}%
% \AgdaBound{A}\AgdaSpace{}%
% \AgdaFunction{,}\AgdaSpace{}%
% \AgdaBound{x}\AgdaSpace{}%
% \AgdaOperator{\AgdaDatatype{≡}}\AgdaSpace{}%
% \AgdaBound{y}\<%
% \\
% %
% \\[\AgdaEmptyExtraSkip]%
% %
% \>[1]\AgdaFunction{𝟎-sigma'}\AgdaSpace{}%
% \AgdaSymbol{:}\AgdaSpace{}%
% \AgdaBound{𝓤}\AgdaSpace{}%
% \AgdaOperator{\AgdaFunction{̇}}\<%
% \\
% %
% \>[1]\AgdaFunction{𝟎-sigma'}\AgdaSpace{}%
% \AgdaSymbol{=}\AgdaSpace{}%
% \AgdaFunction{Σ}\AgdaSpace{}%
% \AgdaBound{(x}\AgdaSpace{}%
% \AgdaBound{,}\AgdaSpace{}%
% \AgdaBound{y)}\AgdaSpace{}%
% \AgdaFunction{꞉}\AgdaSpace{}%
% \AgdaSymbol{(}\AgdaBound{A}\AgdaSpace{}%
% \AgdaOperator{\AgdaFunction{×}}\AgdaSpace{}%
% \AgdaBound{A}\AgdaSymbol{)}\AgdaSpace{}%
% \AgdaFunction{,}\AgdaSpace{}%
% \AgdaBound{x}\AgdaSpace{}%
% \AgdaOperator{\AgdaDatatype{≡}}\AgdaSpace{}%
% \AgdaBound{y}\<%
\end{code}
% \ccpad
% Finally, the \defn{total relation} over \ab A, which in set theory is the full Cartesian product \ab A~\af ×~\ab A, can be represented using the one-element type from the \am{Unit-Type} module of \typtop as follows.\cref{uhints}$^, $\footnote{%
% The one-element type is an inductive type with a single constructor, denoted \AgdaInductiveConstructor{⋆}, as follows: \AgdaKeyword{data}\AgdaSpace{}%
% \AgdaDatatype{𝟙}\AgdaSpace{}%
% \AgdaSymbol{\{}\AgdaBound{𝓤}\AgdaSymbol{\}}\AgdaSpace{}%
% \AgdaSymbol{:}\AgdaSpace{}%
% \AgdaBound{𝓤}\AgdaSpace{}%
% \AgdaOperator{\AgdaFunction{̇}}\AgdaSpace{}%
% \AgdaKeyword{where}\AgdaSpace{}\AgdaInductiveConstructor{⋆}\AgdaSpace{}%
% \AgdaSymbol{:}\AgdaSpace{}%
% \AgdaDatatype{𝟙}.}
% \ccpad
% \begin{code}%
% \>[1]\AgdaFunction{𝟏}\AgdaSpace{}%
% \AgdaSymbol{:}\AgdaSpace{}%
% \AgdaFunction{Rel}\AgdaSpace{}%
% \AgdaBound{A}\AgdaSpace{}%
% \AgdaPrimitive{𝓤₀}\<%
% \\
% %
% \>[1]\AgdaFunction{𝟏}\AgdaSpace{}%
% \AgdaBound{a}\AgdaSpace{}%
% \AgdaBound{b}\AgdaSpace{}%
% \AgdaSymbol{=}\AgdaSpace{}%
% \AgdaFunction{𝟙}\<%
% \end{code}


\paragraph*{The implication relation\protect\footnotemark}
\footnotetext{The definitions here are from the \agdastdlib, translated into \typtop/\ualib notation.}

The following types represent \defn{implication} for binary relations.
\ccpad
\begin{code}%
\>[0]\AgdaOperator{\AgdaFunction{\AgdaUnderscore{}on\AgdaUnderscore{}}}\AgdaSpace{}%
\AgdaSymbol{:}\AgdaSpace{}%
\AgdaSymbol{(}\AgdaBound{B}\AgdaSpace{}%
\AgdaSymbol{→}\AgdaSpace{}%
\AgdaBound{B}\AgdaSpace{}%
\AgdaSymbol{→}\AgdaSpace{}%
\AgdaBound{C}\AgdaSymbol{)}\AgdaSpace{}%
\AgdaSymbol{→}\AgdaSpace{}%
\AgdaSymbol{(}\AgdaBound{A}\AgdaSpace{}%
\AgdaSymbol{→}\AgdaSpace{}%
\AgdaBound{B}\AgdaSymbol{)}\AgdaSpace{}%
\AgdaSymbol{→}\AgdaSpace{}%
\AgdaSymbol{(}\AgdaBound{A}\AgdaSpace{}%
\AgdaSymbol{→}\AgdaSpace{}%
\AgdaBound{A}\AgdaSpace{}%
\AgdaSymbol{→}\AgdaSpace{}%
\AgdaBound{C}\AgdaSymbol{)}\<%
\\
\>[0]\AgdaBound{R}\AgdaSpace{}%
\AgdaOperator{\AgdaFunction{on}}\AgdaSpace{}%
\AgdaBound{g}\AgdaSpace{}%
\AgdaSymbol{=}\AgdaSpace{}%
\AgdaSymbol{λ}\AgdaSpace{}%
\AgdaBound{x}\AgdaSpace{}%
\AgdaBound{y}\AgdaSpace{}%
\AgdaSymbol{→}\AgdaSpace{}%
\AgdaBound{R}\AgdaSpace{}%
\AgdaSymbol{(}\AgdaBound{g}\AgdaSpace{}%
\AgdaBound{x}\AgdaSymbol{)}\AgdaSpace{}%
\AgdaSymbol{(}\AgdaBound{g}\AgdaSpace{}%
\AgdaBound{y}\AgdaSymbol{)}\<%
\end{code}
\scpad
\begin{code}
\>[1]\AgdaOperator{\AgdaFunction{\AgdaUnderscore{}⇒\AgdaUnderscore{}}}\AgdaSpace{}%
\AgdaSymbol{:}\AgdaSpace{}%
\AgdaFunction{REL}\AgdaSpace{}%
\AgdaBound{A}\AgdaSpace{}%
\AgdaBound{B}\AgdaSpace{}%
\AgdaBound{𝓧}\AgdaSpace{}%
\AgdaSymbol{→}\AgdaSpace{}%
\AgdaFunction{REL}\AgdaSpace{}%
\AgdaBound{A}\AgdaSpace{}%
\AgdaBound{B}\AgdaSpace{}%
\AgdaBound{𝓨}\AgdaSpace{}%
\AgdaSymbol{→}\AgdaSpace{}%
\AgdaBound{𝓤}\AgdaSpace{}%
\AgdaOperator{\AgdaPrimitive{⊔}}\AgdaSpace{}%
\AgdaBound{𝓦}\AgdaSpace{}%
\AgdaOperator{\AgdaPrimitive{⊔}}\AgdaSpace{}%
\AgdaBound{𝓧}\AgdaSpace{}%
\AgdaOperator{\AgdaPrimitive{⊔}}\AgdaSpace{}%
\AgdaBound{𝓨}\AgdaSpace{}%
\AgdaOperator{\AgdaFunction{̇}}\<%
\\
%
\>[1]\AgdaBound{P}\AgdaSpace{}%
\AgdaOperator{\AgdaFunction{⇒}}\AgdaSpace{}%
\AgdaBound{Q}\AgdaSpace{}%
\AgdaSymbol{=}\AgdaSpace{}%
\AgdaSymbol{∀}\AgdaSpace{}%
\AgdaSymbol{\{}\AgdaBound{i}\AgdaSpace{}%
\AgdaBound{j}\AgdaSymbol{\}}\AgdaSpace{}%
\AgdaSymbol{→}\AgdaSpace{}%
\AgdaBound{P}\AgdaSpace{}%
\AgdaBound{i}\AgdaSpace{}%
\AgdaBound{j}\AgdaSpace{}%
\AgdaSymbol{→}\AgdaSpace{}%
\AgdaBound{Q}\AgdaSpace{}%
\AgdaBound{i}\AgdaSpace{}%
\AgdaBound{j}\<%
\end{code}
\ccpad
These combine to give a nice, general implication operation.
\ccpad
\begin{code}%
\>[1]\AgdaOperator{\AgdaFunction{\AgdaUnderscore{}=[\AgdaUnderscore{}]⇒\AgdaUnderscore{}}}\AgdaSpace{}%
\AgdaSymbol{:}\AgdaSpace{}%
\AgdaFunction{Rel}\AgdaSpace{}%
\AgdaBound{A}\AgdaSpace{}%
\AgdaBound{𝓧}\AgdaSpace{}%
\AgdaSymbol{→}\AgdaSpace{}%
\AgdaSymbol{(}\AgdaBound{A}\AgdaSpace{}%
\AgdaSymbol{→}\AgdaSpace{}%
\AgdaBound{B}\AgdaSymbol{)}\AgdaSpace{}%
\AgdaSymbol{→}\AgdaSpace{}%
\AgdaFunction{Rel}\AgdaSpace{}%
\AgdaBound{B}\AgdaSpace{}%
\AgdaBound{𝓨}\AgdaSpace{}%
\AgdaSymbol{→}\AgdaSpace{}%
\AgdaBound{𝓤}\AgdaSpace{}%
\AgdaOperator{\AgdaPrimitive{⊔}}\AgdaSpace{}%
\AgdaBound{𝓧}\AgdaSpace{}%
\AgdaOperator{\AgdaPrimitive{⊔}}\AgdaSpace{}%
\AgdaBound{𝓨}\AgdaSpace{}%
\AgdaOperator{\AgdaFunction{̇}}\<%
\\
%
\>[1]\AgdaBound{P}\AgdaSpace{}%
\AgdaOperator{\AgdaFunction{=[}}\AgdaSpace{}%
\AgdaBound{g}\AgdaSpace{}%
\AgdaOperator{\AgdaFunction{]⇒}}\AgdaSpace{}%
\AgdaBound{Q}\AgdaSpace{}%
\AgdaSymbol{=}\AgdaSpace{}%
\AgdaBound{P}\AgdaSpace{}%
\AgdaOperator{\AgdaFunction{⇒}}\AgdaSpace{}%
\AgdaSymbol{(}\AgdaBound{Q}\AgdaSpace{}%
\AgdaOperator{\AgdaFunction{on}}\AgdaSpace{}%
\AgdaBound{g}\AgdaSymbol{)}\<%
\end{code}

\paragraph*{Operation type and compatibility}%\label{compatibility-of-functions-and-binary-relations}

\textbf{Notation}. For consistency and readability, throughout the \ualib we reserve two universe variables for special purposes.  The first of these is \ab 𝓞 which shall be reserved for types that represent \defn{operation symbols} (see \ualibhtml{Algebras.Signatures}). The second is \ab 𝓥 which we reserve for types representing \defn{arities} of relations or operations.

Below we will define types that are useful for asserting and proving facts about \defn{compatibility} of operations and relations, but first we need a general type with which to represent operations.  Here is the definition, which we justify below.
\ccpad
\begin{code}
% \>[0]\AgdaComment{--The type of operations}\<%
% \\
\>[0]\AgdaFunction{Op}\AgdaSpace{}%
\AgdaSymbol{:}\AgdaSpace{}%
\AgdaGeneralizable{𝓥}\AgdaSpace{}%
\AgdaOperator{\AgdaFunction{̇}}\AgdaSpace{}%
\AgdaSymbol{→}\AgdaSpace{}%
\AgdaGeneralizable{𝓤}\AgdaSpace{}%
\AgdaOperator{\AgdaFunction{̇}}\AgdaSpace{}%
\AgdaSymbol{→}\AgdaSpace{}%
\AgdaGeneralizable{𝓤}\AgdaSpace{}%
\AgdaOperator{\AgdaPrimitive{⊔}}\AgdaSpace{}%
\AgdaGeneralizable{𝓥}\AgdaSpace{}%
\AgdaOperator{\AgdaFunction{̇}}\<%
\\
\>[0]\AgdaFunction{Op}\AgdaSpace{}%
\AgdaBound{I}\AgdaSpace{}%
\AgdaBound{A}\AgdaSpace{}%
\AgdaSymbol{=}\AgdaSpace{}%
\AgdaSymbol{(}\AgdaBound{I}\AgdaSpace{}%
\AgdaSymbol{→}\AgdaSpace{}%
\AgdaBound{A}\AgdaSymbol{)}\AgdaSpace{}%
\AgdaSymbol{→}\AgdaSpace{}%
\AgdaBound{A}\<%
\end{code}
\ccpad
The definition of \af{Op} codifies the arity of an operation as an arbitrary type \ab I~\as :~\ab 𝓥\af ̇, which gives us a very general way to represent an operation as a function type with domain \ab I~\as →~\ab A (the type of ``\ab I-tuples'') and codomain \ab A. For example, the \ab I-\defn{ary projection operations} on \ab A are represented as inhabitants of the type \af{Op}~\ab I~\ab A as follows.
\ccpad
\begin{code}
\>[0]\AgdaFunction{π}\AgdaSpace{}%
\AgdaSymbol{:}\AgdaSpace{}%
\AgdaSymbol{\{}\AgdaBound{I}\AgdaSpace{}%
\AgdaSymbol{:}\AgdaSpace{}%
\AgdaGeneralizable{𝓥}\AgdaSpace{}%
\AgdaOperator{\AgdaFunction{̇}}\AgdaSpace{}%
\AgdaSymbol{\}}\AgdaSpace{}%
\AgdaSymbol{\{}\AgdaBound{A}\AgdaSpace{}%
\AgdaSymbol{:}\AgdaSpace{}%
\AgdaGeneralizable{𝓤}\AgdaSpace{}%
\AgdaOperator{\AgdaFunction{̇}}\AgdaSpace{}%
\AgdaSymbol{\}}\AgdaSpace{}%
\AgdaSymbol{→}\AgdaSpace{}%
\AgdaBound{I}\AgdaSpace{}%
\AgdaSymbol{→}\AgdaSpace{}%
\AgdaFunction{Op}\AgdaSpace{}%
\AgdaBound{I}\AgdaSpace{}%
\AgdaBound{A}\<%
\\
\>[0]\AgdaFunction{π}\AgdaSpace{}%
\AgdaBound{i}\AgdaSpace{}%
\AgdaBound{x}\AgdaSpace{}%
\AgdaSymbol{=}\AgdaSpace{}%
\AgdaBound{x}\AgdaSpace{}%
\AgdaBound{i}\<%
\end{code}
\scpad
% Before discussing general and dependent relations, we pause to define some types that are useful for asserting and proving facts about \emph{compatibility} of functions with binary relations. 

Let us review the informal definition of compatibility. Suppose \ab{A} and \ab{I} are types and fix \ab{𝑓}~\as :~\AgdaFunction{Op}~\AgdaBound{I}~\AgdaBound{A} and \ab{R}~\as :~\af{Rel}~\ab A~\ab 𝓦 (an \ab{I}-ary operation and a binary relation on \ab{A}, respectively). We say that \ab{𝑓} and \ab{R} are \emph{compatible} and we write\footnote{The symbol \af{\textbar{}:} denoting compatibility comes from Cliff Bergman's universal algebra textbook~\cite{Bergman:2012}.} \ab{𝑓}~\af{\textbar{}:}~\ab R just in case \as{∀}~\ab u~\ab v~\as :~\ab I~\as →~\ab A,\\[-8pt]

\ad{Π}~\ab i~\as ꞉ \ab I \af , \ab R (\ab u \ab i) (\ab v \ab i) ~ \as{→} ~ \ab{R} (\ab f \ab u) (\ab f \ab v).\\[4pt]
To implement this in Agda, we first define a function \af{eval-rel} which ``lifts'' a binary relation to the corresponding \ab{I}-ary relation, and we use this to define the function \AgdaFunction{|:} representing \defn{compatibility of an \ab{I}-ary operation with a binary relation}.
\ccpad
\begin{code}%
\>[0]\AgdaFunction{eval-rel}\AgdaSpace{}%
\AgdaSymbol{:}\AgdaSpace{}%
\AgdaSymbol{\{}\AgdaBound{A}\AgdaSpace{}%
\AgdaSymbol{:}\AgdaSpace{}%
\AgdaGeneralizable{𝓤}\AgdaSpace{}%
\AgdaOperator{\AgdaFunction{̇}}\AgdaSymbol{\}\{}\AgdaBound{I}\AgdaSpace{}%
\AgdaSymbol{:}\AgdaSpace{}%
\AgdaGeneralizable{𝓥}\AgdaSpace{}%
\AgdaOperator{\AgdaFunction{̇}}\AgdaSymbol{\}}\AgdaSpace{}%
\AgdaSymbol{→}\AgdaSpace{}%
\AgdaFunction{Rel}\AgdaSpace{}%
\AgdaBound{A}\AgdaSpace{}%
\AgdaGeneralizable{𝓦}\AgdaSpace{}%
\AgdaSymbol{→}\AgdaSpace{}%
\AgdaFunction{Rel}\AgdaSpace{}%
\AgdaSymbol{(}\AgdaBound{I}\AgdaSpace{}%
\AgdaSymbol{→}\AgdaSpace{}%
\AgdaBound{A}\AgdaSymbol{)(}\AgdaGeneralizable{𝓥}\AgdaSpace{}%
\AgdaOperator{\AgdaPrimitive{⊔}}\AgdaSpace{}%
\AgdaGeneralizable{𝓦}\AgdaSymbol{)}\<%
\\
\>[0]\AgdaFunction{eval-rel}\AgdaSpace{}%
\AgdaBound{R}\AgdaSpace{}%
\AgdaBound{u}\AgdaSpace{}%
\AgdaBound{v}\AgdaSpace{}%
\AgdaSymbol{=}\AgdaSpace{}%
\AgdaFunction{Π}\AgdaSpace{}%
\AgdaBound{i}\AgdaSpace{}%
\AgdaFunction{꞉}\AgdaSpace{}%
\AgdaSymbol{\AgdaUnderscore{}}\AgdaSpace{}%
\AgdaFunction{,}\AgdaSpace{}%
\AgdaBound{R}\AgdaSpace{}%
\AgdaSymbol{(}\AgdaBound{u}\AgdaSpace{}%
\AgdaBound{i}\AgdaSymbol{)}\AgdaSpace{}%
\AgdaSymbol{(}\AgdaBound{v}\AgdaSpace{}%
\AgdaBound{i}\AgdaSymbol{)}\<%
\\
%
\\[\AgdaEmptyExtraSkip]%
\>[0]\AgdaOperator{\AgdaFunction{\AgdaUnderscore{}|:\AgdaUnderscore{}}}\AgdaSpace{}%
\AgdaSymbol{:}\AgdaSpace{}%
\AgdaSymbol{\{}\AgdaBound{A}\AgdaSpace{}%
\AgdaSymbol{:}\AgdaSpace{}%
\AgdaGeneralizable{𝓤}\AgdaSpace{}%
\AgdaOperator{\AgdaFunction{̇}}\AgdaSymbol{\}\{}\AgdaBound{I}\AgdaSpace{}%
\AgdaSymbol{:}\AgdaSpace{}%
\AgdaGeneralizable{𝓥}\AgdaSpace{}%
\AgdaOperator{\AgdaFunction{̇}}\AgdaSymbol{\}}\AgdaSpace{}%
\AgdaSymbol{→}\AgdaSpace{}%
\AgdaFunction{Op}\AgdaSpace{}%
\AgdaBound{I}\AgdaSpace{}%
\AgdaBound{A}\AgdaSpace{}%
\AgdaSymbol{→}\AgdaSpace{}%
\AgdaFunction{Rel}\AgdaSpace{}%
\AgdaBound{A}\AgdaSpace{}%
\AgdaGeneralizable{𝓦}\AgdaSpace{}%
\AgdaSymbol{→}\AgdaSpace{}%
\AgdaGeneralizable{𝓤}\AgdaSpace{}%
\AgdaOperator{\AgdaPrimitive{⊔}}\AgdaSpace{}%
\AgdaGeneralizable{𝓥}\AgdaSpace{}%
\AgdaOperator{\AgdaPrimitive{⊔}}\AgdaSpace{}%
\AgdaGeneralizable{𝓦}\AgdaSpace{}%
\AgdaOperator{\AgdaFunction{̇}}\<%
\\
\>[0]\AgdaBound{f}\AgdaSpace{}%
\AgdaOperator{\AgdaFunction{|:}}\AgdaSpace{}%
\AgdaBound{R}%
\>[8]\AgdaSymbol{=}\AgdaSpace{}%
\AgdaSymbol{(}\AgdaFunction{eval-rel}\AgdaSpace{}%
\AgdaBound{R}\AgdaSymbol{)}\AgdaSpace{}%
\AgdaOperator{\AgdaFunction{=[}}\AgdaSpace{}%
\AgdaBound{f}\AgdaSpace{}%
\AgdaOperator{\AgdaFunction{]⇒}}\AgdaSpace{}%
\AgdaBound{R}\<%
\end{code}
\ccpad
% \footnote{Initially we called the first function
% \af{lift-rel} because it ``lifts'' a binary relation on \ab A to a binary relation on tuples of type \ab I~\as →~\ab A. However, we renamed it \af{eval-rel} to avoid confusion with the universe level \AgdaRecord{Lift} type defined in the \ualibhtml{Overture.Lifts} module, or with \af{free-lift} (\ualibhtml{Terms.Basic}) which lifts a map defined on generators to a map on the thing being generated. Also, o
% (Here we silently added the type \ab I~\as :~\ab 𝓥\af ̇, representing \emph{relation arity}, to the context; we will have more to say about relation arities in the next section (\S\ref{sec:continuous-relations}).)

%%%%%%%%%%%%%%%%%%%%%%%%%%%%%%%%%%%%%%%%%%%%%%%%%%%%%%%%%%%%%%%%%%%%%%%%%%%%%%%%%%%%%%%%%%%%%%%%%%
\subsection{Continuous relations*\protect\footnotemark}\label{sec:continuous-relations}\footnotetext{\label{starred}%
Sections marked with an asterisk include new types that are more abstract and general than those presented elsewhere, but they are used only very rarely in other parts of the \ualib. Therefore, the sections marked * may be safely skimmed or skipped when first encountering them.}
% : arbitrary-sorted relations of arbitrary arity
\firstsentencefull{Relations}{Continuous}
% \context{%
% \{\ab 𝓤 \ab 𝓥 \ab 𝓦 \as : \AgdaPostulate{Universe}\}%
% \{\ab I \ab J \as : \ab 𝓥\AgdaOperator{\AgdaFunction{̇}}\}%
% \{\ab A \as : \ab 𝓤\AgdaOperator{\AgdaFunction{̇}}\}%
% }
% -*- TeX-master: "ualib-part1.tex" -*-
%%% Local Variables: 
%%% mode: latex
%%% TeX-engine: 'xetex
%%% End:
\subsubsection{Motivation}\label{sec:motivation-1}
In set theory, an \ab n-ary relation on a set \ab{A} is a subset of the \ab n-fold product \ab{A} \ad × \as ⋯ \ad × \ab A. We could try to model these as predicates over a product type representing \ab{A} \ad × \as ⋯ \ad × \ab A, or as relations of type \ab{A} \as → \ab A \as → \as ⋯ \as → \ab A \as → \ab 𝓦\af ̇ (for some universe \ab 𝓦). To implement such types requires knowing the arity in advance, and then form an \ab n-fold product or \ab n-fold arrow. It turns out to be both easier and more general if we define an arity type \ab{I}~\as :~\ab 𝓥\af ̇ , and define the type representing \ab{I}-ary relations on \ab{A} as the function type (\ab{I} \as → \ab A) \as → \ab 𝓦\af ̇. Then, if we happen to be interested in \ab n-ary relations for some natural number \ab{n}, we could take \ab{I} to be the \ab n-element type \ad{Fin}~\ab n,~\cite{agda-fin}.

Below we will define \af{ContRel} to be the type (\ab{I}~\as →~\ab A)~\as →~\ab 𝓦\af ̇ and we will call this the type of \defn{continuous relations}. This generalizes the discrete relations we defined in \ualibhtml{Relations.Discrete} (unary, binary, etc.) since continuous relations can be of arbitrary arity.  They are not completely general, however, since they are defined over a single type. Said another way, these are ``single-sorted'' relations. We will remove this limitation when we define the type of \emph{dependent continuous relations}. Just as \af{Rel} \ab A \ab 𝓦 was the single-sorted special case of the multisorted \af{REL} \ab A \ab B \ab 𝓦 type, so too will \af{ContRel} \ab I \ab A \ab 𝓦 be the single-sorted version of dependent continuous relations. The latter will represent relations that not only have arbitrary arities, but also are defined over arbitrary families of types.

To be more concrete, given an arbitrary family \ab{A}~\as :~\ab I~\as →~\ab 𝓤\af ̇ of types, we may have a relation from \ab{A}~\ab i to \ab{A}~\ab j to \ab{A}~\ab k to $\ldots$, where the collection represented by the indexing type \ab I might not even be enumerable.\footnote{\label{uncountable}Because the collection represented by the indexing type \ab I might not even be enumerable, technically speaking, instead of ``\ab{A} \ab i to \ab{A} \ab j to \ab{A} \ab k to $\ldots$,'' we should have written something like ``$\mathsf{\af{TO}}$ (\abt{i}{I}) , \ab{A} \ab i.''}
We will refer to such relations as \defn{dependent continuous relations} (or \defn{dependent relations}) because the definition of a type that represents them requires dependent types. The \af{DepRel} type that we define below manifests this completely general notion of relation.

\subsubsection{Continuous relation types}\label{continuous-relation-types}

We now define the type \af{ContRel} which represents predicates of arbitrary arity over a single type \ab{A}. We call this the type of \defn{continuous relations}.\footnote{%
For consistency and readability, throughout the \ualib we reserve two universe variables for special purposes. The first of these is 𝓞 which shall be reserved for types that represent \emph{operation symbols} (see \ualibhtml{Algebras.Signatures}). The second is 𝓥 which we reserve for types representing \emph{arities} of relations or operations.}
\ccpad
\begin{code}%
\>[0]\AgdaFunction{ContRel}\AgdaSpace{}%
\AgdaSymbol{:}\AgdaSpace{}%
\AgdaGeneralizable{𝓥}\AgdaSpace{}%
\AgdaOperator{\AgdaFunction{̇}}\AgdaSpace{}%
\AgdaSymbol{→}\AgdaSpace{}%
\AgdaGeneralizable{𝓤}\AgdaSpace{}%
\AgdaOperator{\AgdaFunction{̇}}\AgdaSpace{}%
\AgdaSymbol{→}\AgdaSpace{}%
\AgdaSymbol{(}\AgdaBound{𝓦}\AgdaSpace{}%
\AgdaSymbol{:}\AgdaSpace{}%
\AgdaPostulate{Universe}\AgdaSymbol{)}\AgdaSpace{}%
\AgdaSymbol{→}\AgdaSpace{}%
\AgdaGeneralizable{𝓥}\AgdaSpace{}%
\AgdaOperator{\AgdaPrimitive{⊔}}\AgdaSpace{}%
\AgdaGeneralizable{𝓤}\AgdaSpace{}%
\AgdaOperator{\AgdaPrimitive{⊔}}\AgdaSpace{}%
\AgdaBound{𝓦}\AgdaSpace{}%
\AgdaOperator{\AgdaPrimitive{⁺}}\AgdaSpace{}%
\AgdaOperator{\AgdaFunction{̇}}\<%
\\
\>[0]\AgdaFunction{ContRel}\AgdaSpace{}%
\AgdaBound{I}\AgdaSpace{}%
\AgdaBound{A}\AgdaSpace{}%
\AgdaBound{𝓦}\AgdaSpace{}%
\AgdaSymbol{=}\AgdaSpace{}%
\AgdaSymbol{(}\AgdaBound{I}\AgdaSpace{}%
\AgdaSymbol{→}\AgdaSpace{}%
\AgdaBound{A}\AgdaSymbol{)}\AgdaSpace{}%
\AgdaSymbol{→}\AgdaSpace{}%
\AgdaBound{𝓦}\AgdaSpace{}%
\AgdaOperator{\AgdaFunction{̇}}\<%
\end{code}
\ccpad
% \subsubsection{Compatibility with continuous relations}\label{compatibility-with-continuous-relations}
Next we present types that are useful for asserting and proving facts about the compatibility of functions with continuous relations. The first is an \defn{evaluation} function which ``lifts'' an \ab I-ary relation to an (\ab I~\as →~\ab J)-ary relation. The lifted relation will relate an \ab I-tuple of \ab J-tuples when the ``\ab I-slices'' (or ``rows'') of the \ab J-tuples belong to the original relation. The second definition is a type that represents compatibility of an operation with a continuous relation. (We'll dissect these definitions below.)
\ccpad
\begin{code}%
% \>[0]\AgdaKeyword{module}\AgdaSpace{}%
% \AgdaModule{\AgdaUnderscore{}}\AgdaSpace{}%
% \AgdaSymbol{\{}\AgdaBound{I}\AgdaSpace{}%
% \AgdaBound{J}\AgdaSpace{}%
% \AgdaSymbol{:}\AgdaSpace{}%
% \AgdaGeneralizable{𝓥}\AgdaSpace{}%
% \AgdaOperator{\AgdaFunction{̇}}\AgdaSymbol{\}}\AgdaSpace{}%
% \AgdaSymbol{\{}\AgdaBound{A}\AgdaSpace{}%
% \AgdaSymbol{:}\AgdaSpace{}%
% \AgdaGeneralizable{𝓤}\AgdaSpace{}%
% \AgdaOperator{\AgdaFunction{̇}}\AgdaSymbol{\}}\AgdaSpace{}%
% \AgdaKeyword{where}\<%
% \\
% %
% \\[\AgdaEmptyExtraSkip]%
\>[0][@{}l@{\AgdaIndent{0}}]%
\>[1]\AgdaFunction{eval-cont-rel}\AgdaSpace{}%
\AgdaSymbol{:}\AgdaSpace{}%
\AgdaFunction{ContRel}\AgdaSpace{}%
\AgdaBound{I}\AgdaSpace{}%
\AgdaBound{A}\AgdaSpace{}%
\AgdaGeneralizable{𝓦}\AgdaSpace{}%
\AgdaSymbol{→}\AgdaSpace{}%
\AgdaSymbol{(}\AgdaBound{I}\AgdaSpace{}%
\AgdaSymbol{→}\AgdaSpace{}%
\AgdaBound{J}\AgdaSpace{}%
\AgdaSymbol{→}\AgdaSpace{}%
\AgdaBound{A}\AgdaSymbol{)}\AgdaSpace{}%
\AgdaSymbol{→}\AgdaSpace{}%
\AgdaBound{𝓥}\AgdaSpace{}%
\AgdaOperator{\AgdaPrimitive{⊔}}\AgdaSpace{}%
\AgdaGeneralizable{𝓦}\AgdaSpace{}%
\AgdaOperator{\AgdaFunction{̇}}\<%
\\
%
\>[1]\AgdaFunction{eval-cont-rel}\AgdaSpace{}%
\AgdaBound{R}\AgdaSpace{}%
\AgdaBound{𝒶}\AgdaSpace{}%
\AgdaSymbol{=}\AgdaSpace{}%
\AgdaFunction{Π}\AgdaSpace{}%
\AgdaBound{j}\AgdaSpace{}%
\AgdaFunction{꞉}\AgdaSpace{}%
\AgdaBound{J}\AgdaSpace{}%
\AgdaFunction{,}\AgdaSpace{}%
\AgdaBound{R}\AgdaSpace{}%
\AgdaSymbol{λ}\AgdaSpace{}%
\AgdaBound{i}\AgdaSpace{}%
\AgdaSymbol{→}\AgdaSpace{}%
\AgdaBound{𝒶}\AgdaSpace{}%
\AgdaBound{i}\AgdaSpace{}%
\AgdaBound{j}\<%
\\
%
\\[\AgdaEmptyExtraSkip]%
%
\>[1]\AgdaFunction{cont-compatible-fun}\AgdaSpace{}%
\AgdaSymbol{:}\AgdaSpace{}%
\AgdaSymbol{((}\AgdaBound{J}\AgdaSpace{}%
\AgdaSymbol{→}\AgdaSpace{}%
\AgdaBound{A}\AgdaSymbol{)}\AgdaSpace{}%
\AgdaSymbol{→}\AgdaSpace{}%
\AgdaBound{A}\AgdaSymbol{)}\AgdaSpace{}%
\AgdaSymbol{→}\AgdaSpace{}%
\AgdaFunction{ContRel}\AgdaSpace{}%
\AgdaBound{I}\AgdaSpace{}%
\AgdaBound{A}\AgdaSpace{}%
\AgdaGeneralizable{𝓦}\AgdaSpace{}%
\AgdaSymbol{→}\AgdaSpace{}%
\AgdaBound{𝓥}\AgdaSpace{}%
\AgdaOperator{\AgdaPrimitive{⊔}}\AgdaSpace{}%
\AgdaBound{𝓤}\AgdaSpace{}%
\AgdaOperator{\AgdaPrimitive{⊔}}\AgdaSpace{}%
\AgdaGeneralizable{𝓦}\AgdaSpace{}%
\AgdaOperator{\AgdaFunction{̇}}\<%
\\
%
\>[1]\AgdaFunction{cont-compatible-fun}\AgdaSpace{}%
\AgdaBound{𝑓}\AgdaSpace{}%
\AgdaBound{R}%
\>[26]\AgdaSymbol{=}\AgdaSpace{}%
\AgdaFunction{Π}\AgdaSpace{}%
\AgdaBound{𝒶}\AgdaSpace{}%
\AgdaFunction{꞉}\AgdaSpace{}%
\AgdaSymbol{(}\AgdaBound{I}\AgdaSpace{}%
\AgdaSymbol{→}\AgdaSpace{}%
\AgdaBound{J}\AgdaSpace{}%
\AgdaSymbol{→}\AgdaSpace{}%
\AgdaBound{A}\AgdaSymbol{)}\AgdaSpace{}%
\AgdaFunction{,}\AgdaSpace{}%
\AgdaSymbol{(}\AgdaFunction{eval-cont-rel}\AgdaSpace{}%
\AgdaBound{R}\AgdaSpace{}%
\AgdaBound{𝒶}\AgdaSpace{}%
\AgdaSymbol{→}\AgdaSpace{}%
\AgdaBound{R}\AgdaSpace{}%
\AgdaSymbol{λ}\AgdaSpace{}%
\AgdaBound{i}\AgdaSpace{}%
\AgdaSymbol{→}\AgdaSpace{}%
\AgdaSymbol{(}\AgdaBound{𝑓}\AgdaSpace{}%
\AgdaSymbol{(}\AgdaBound{𝒶}\AgdaSpace{}%
\AgdaBound{i}\AgdaSymbol{)))}\<%
\end{code}
\ccpad
% In the definition of \af{cont-compatible-fun}, we let Agda infer the type \ab{I}
% \as → (\ab J \as → \ab A) of \ab{𝒶}.
To readers who find the syntax of the last two definitions nauseating, we recommend focusing on the semantics. First, internalize the fact that \ab 𝒶~\as :~\ab I~\as →~\ab J~\as →~\ab A denotes an \ab{I}-tuple of \ab{J}-tuples of inhabitants of \ab{A}. Next, recall that a continuous relation \ab{R} represents a certain collection of \ab{I}-tuples. Specifically, if \ab x~\as :~\ab I~\as →~\ab A is an \ab{I}-tuple, then \ab{R}~\ab x denotes the assertion that ``\ab{x} belongs to \ab{R}'' or ``\ab{x} satisfies \ab{R}.''  For each continuous relation \ab{R}, the type \af{eval-cont-rel}~\ab R represents a certain collection of \ab{I}-tuples of \ab{J}-tuples, namely, the tuples \ab{𝒶}~\as :~\ab I~\as →~\ab J~\as →~\ab A for which \af{eval-cont-rel}~\ab R~\ab 𝒶 holds. For simplicity, pretend that \ab{J} is a finite set, say, \{1, 2, ...,~\ab J\}, so that we can write down a couple of the \ab{J}-tuples as columns. For example, here are the \ab i-th and \ab k-th columns (for some \ab i~\ab k~\as :~\ab I).
% ~
% \vskip-10mm
% ~
\begin{center}
\begin{tabular}{rll}
\ab 𝒶 \ab i 1 & \ab 𝒶 \ab k 1 & \\
\ab 𝒶 \ab i 2 &    \ab  𝒶 \ab k 2 & \\
\ab 𝒶 \ab i 3 & \ab 𝒶 \ab k 3 &   ← (if there are \ab I columns, then each row forms an \ab I-tuple)  \\
  ⋮     &     ⋮  & \\
\ab 𝒶 \ab i \ab J  &   \ab 𝒶 \ab k \ab J  & \\
\end{tabular}
\end{center}
% ~
% \vskip-5mm
% ~
Now \af{eval-cont-rel}~\ab R~\ab 𝒶 is defined by \as{∀}~\ab j~\as →~\ab R~(\as λ~\ab i~\as →~(\ab 𝒶~\ab i)~\ab j) which represents the assertion that each row of the \ab{I} columns shown above %(which evidently is an \ab{I}-tuple)
 belongs to the original relation \ab{R}. Finally, \af{cont-compatible-fun} takes a \ab{J}-ary operation on \ab{A}, say, \ab 𝑓~\as :~(\ab J~\as →~\ab A)~\as →~\ab A, and an \ab I-tuple \ab{𝒶}~\ab i~\as :~\ab J~\as →~\ab A of \ab{J}-tuples, and determines whether the \ab I-tuple \as λ~\ab i~\as →~\ab 𝑓~(\ab 𝒶~\ab i) belongs to \ab R.
% Finally, digest all the types involved in these definitions and note how nicely they align (as they must if type-checking is to succeed!). For example, \ab 𝒶~\as :~\ab I~\as →~\ab J~\as →~\ab A is precisely the type on which the relation \af{eval-cont-rel}~\ab R is defined.

\subsubsection{Dependent relations}\label{dependent-relations}

In this section we exploit the power of dependent types to define a completely general relation type. Specifically, we let the tuples inhabit a dependent function type, where the codomain may depend upon the input coordinate \ab i \as : \ab I of the domain. 
Heuristically, think of the inhabitants of the following type as relations from \ab{A} \ab{i} to \ab{A} \ab{j} to \ab{A} \ab{k} to …. (This is only an heuristic since \ab I can represent an uncountable collection.\cref{uncountable})
\ccpad
\begin{code}%
\>[0]\AgdaFunction{DepRel}\AgdaSpace{}%
\AgdaSymbol{:}\AgdaSpace{}%
\AgdaSymbol{(}\AgdaBound{I}\AgdaSpace{}%
\AgdaSymbol{:}\AgdaSpace{}%
\AgdaGeneralizable{𝓥}\AgdaSpace{}%
\AgdaOperator{\AgdaFunction{̇}}\AgdaSymbol{)(}\AgdaBound{A}\AgdaSpace{}%
\AgdaSymbol{:}\AgdaSpace{}%
\AgdaBound{I}\AgdaSpace{}%
\AgdaSymbol{→}\AgdaSpace{}%
\AgdaGeneralizable{𝓤}\AgdaSpace{}%
\AgdaOperator{\AgdaFunction{̇}}\AgdaSymbol{)(}\AgdaBound{𝓦}\AgdaSpace{}%
\AgdaSymbol{:}\AgdaSpace{}%
\AgdaPostulate{Universe}\AgdaSymbol{)}\AgdaSpace{}%
\AgdaSymbol{→}\AgdaSpace{}%
\AgdaGeneralizable{𝓥}\AgdaSpace{}%
\AgdaOperator{\AgdaPrimitive{⊔}}\AgdaSpace{}%
\AgdaGeneralizable{𝓤}\AgdaSpace{}%
\AgdaOperator{\AgdaPrimitive{⊔}}\AgdaSpace{}%
\AgdaBound{𝓦}\AgdaSpace{}%
\AgdaOperator{\AgdaPrimitive{⁺}}\AgdaSpace{}%
\AgdaOperator{\AgdaFunction{̇}}\<%
\\
\>[0]\AgdaFunction{DepRel}\AgdaSpace{}%
\AgdaBound{I}\AgdaSpace{}%
\AgdaBound{A}\AgdaSpace{}%
\AgdaBound{𝓦}\AgdaSpace{}%
\AgdaSymbol{=}\AgdaSpace{}%
\AgdaFunction{Π}\AgdaSpace{}%
\AgdaBound{A}\AgdaSpace{}%
\AgdaSymbol{→}\AgdaSpace{}%
\AgdaBound{𝓦}\AgdaSpace{}%
\AgdaOperator{\AgdaFunction{̇}}\<%
\end{code}
\ccpad
We call \af{DepRel} the type of \defn{dependent relations}.

% \subsubsection{Compatibility with dependent relations}\label{compatibility-with-dependent-relations}

Above we saw lifts of continuous relations and what it means for such relations to be compatible with functions. We conclude this module by defining the (only slightly more complicated) lift of dependent relations, and the type that represents compatibility of a tuple of operations with a dependent relation.
\ccpad
\begin{code}%
\>[0]\AgdaKeyword{module}\AgdaSpace{}%
\AgdaModule{\AgdaUnderscore{}}\AgdaSpace{}%
\AgdaSymbol{\{}\AgdaBound{I}\AgdaSpace{}%
\AgdaBound{J}\AgdaSpace{}%
\AgdaSymbol{:}\AgdaSpace{}%
\AgdaGeneralizable{𝓥}\AgdaSpace{}%
\AgdaOperator{\AgdaFunction{̇}}\AgdaSymbol{\}}\AgdaSpace{}%
\AgdaSymbol{\{}\AgdaBound{𝒜}\AgdaSpace{}%
\AgdaSymbol{:}\AgdaSpace{}%
\AgdaBound{I}\AgdaSpace{}%
\AgdaSymbol{→}\AgdaSpace{}%
\AgdaGeneralizable{𝓤}\AgdaSpace{}%
\AgdaOperator{\AgdaFunction{̇}}\AgdaSymbol{\}}\AgdaSpace{}%
\AgdaKeyword{where}\<%
\\
%
\\[\AgdaEmptyExtraSkip]%
\>[0][@{}l@{\AgdaIndent{0}}]%
\>[1]\AgdaFunction{eval-dep-rel}\AgdaSpace{}%
\AgdaSymbol{:}\AgdaSpace{}%
\AgdaFunction{DepRel}\AgdaSpace{}%
\AgdaBound{I}\AgdaSpace{}%
\AgdaBound{𝒜}\AgdaSpace{}%
\AgdaGeneralizable{𝓦}\AgdaSpace{}%
\AgdaSymbol{→}\AgdaSpace{}%
\AgdaSymbol{(∀}\AgdaSpace{}%
\AgdaBound{i}\AgdaSpace{}%
\AgdaSymbol{→}\AgdaSpace{}%
\AgdaBound{J}\AgdaSpace{}%
\AgdaSymbol{→}\AgdaSpace{}%
\AgdaBound{𝒜}\AgdaSpace{}%
\AgdaBound{i}\AgdaSymbol{)}\AgdaSpace{}%
\AgdaSymbol{→}\AgdaSpace{}%
\AgdaBound{𝓥}\AgdaSpace{}%
\AgdaOperator{\AgdaPrimitive{⊔}}\AgdaSpace{}%
\AgdaGeneralizable{𝓦}\AgdaSpace{}%
\AgdaOperator{\AgdaFunction{̇}}\<%
\\
%
\>[1]\AgdaFunction{eval-dep-rel}\AgdaSpace{}%
\AgdaBound{R}\AgdaSpace{}%
\AgdaBound{𝒶}\AgdaSpace{}%
\AgdaSymbol{=}\AgdaSpace{}%
\AgdaSymbol{∀}\AgdaSpace{}%
\AgdaSymbol{(}\AgdaBound{j}\AgdaSpace{}%
\AgdaSymbol{:}\AgdaSpace{}%
\AgdaBound{J}\AgdaSymbol{)}\AgdaSpace{}%
\AgdaSymbol{→}\AgdaSpace{}%
\AgdaBound{R}\AgdaSpace{}%
\AgdaSymbol{(λ}\AgdaSpace{}%
\AgdaBound{i}\AgdaSpace{}%
\AgdaSymbol{→}\AgdaSpace{}%
\AgdaSymbol{(}\AgdaBound{𝒶}\AgdaSpace{}%
\AgdaBound{i}\AgdaSymbol{)}\AgdaSpace{}%
\AgdaBound{j}\AgdaSymbol{)}\<%
\\
%
\\[\AgdaEmptyExtraSkip]%
%
\>[1]\AgdaFunction{dep-compatible-fun}\AgdaSpace{}%
\AgdaSymbol{:}\AgdaSpace{}%
\AgdaSymbol{(∀}\AgdaSpace{}%
\AgdaBound{i}\AgdaSpace{}%
\AgdaSymbol{→}\AgdaSpace{}%
\AgdaSymbol{(}\AgdaBound{J}\AgdaSpace{}%
\AgdaSymbol{→}\AgdaSpace{}%
\AgdaBound{𝒜}\AgdaSpace{}%
\AgdaBound{i}\AgdaSymbol{)}\AgdaSpace{}%
\AgdaSymbol{→}\AgdaSpace{}%
\AgdaBound{𝒜}\AgdaSpace{}%
\AgdaBound{i}\AgdaSymbol{)}\AgdaSpace{}%
\AgdaSymbol{→}\AgdaSpace{}%
\AgdaFunction{DepRel}\AgdaSpace{}%
\AgdaBound{I}\AgdaSpace{}%
\AgdaBound{𝒜}\AgdaSpace{}%
\AgdaGeneralizable{𝓦}\AgdaSpace{}%
\AgdaSymbol{→}\AgdaSpace{}%
\AgdaBound{𝓥}\AgdaSpace{}%
\AgdaOperator{\AgdaPrimitive{⊔}}\AgdaSpace{}%
\AgdaBound{𝓤}\AgdaSpace{}%
\AgdaOperator{\AgdaPrimitive{⊔}}\AgdaSpace{}%
\AgdaGeneralizable{𝓦}\AgdaSpace{}%
\AgdaOperator{\AgdaFunction{̇}}\<%
\\
%
\>[1]\AgdaFunction{dep-compatible-fun}\AgdaSpace{}%
\AgdaBound{𝑓}\AgdaSpace{}%
\AgdaBound{R}%
\>[25]\AgdaSymbol{=}\AgdaSpace{}%
\AgdaSymbol{∀}\AgdaSpace{}%
\AgdaBound{𝒶}\AgdaSpace{}%
\AgdaSymbol{→}\AgdaSpace{}%
\AgdaSymbol{(}\AgdaFunction{eval-dep-rel}\AgdaSpace{}%
\AgdaBound{R}\AgdaSymbol{)}\AgdaSpace{}%
\AgdaBound{𝒶}\AgdaSpace{}%
\AgdaSymbol{→}\AgdaSpace{}%
\AgdaBound{R}\AgdaSpace{}%
\AgdaSymbol{λ}\AgdaSpace{}%
\AgdaBound{i}\AgdaSpace{}%
\AgdaSymbol{→}\AgdaSpace{}%
\AgdaSymbol{(}\AgdaBound{𝑓}\AgdaSpace{}%
\AgdaBound{i}\AgdaSymbol{)(}\AgdaBound{𝒶}\AgdaSpace{}%
\AgdaBound{i}\AgdaSymbol{)}\<%
\end{code}
\ccpad
In the definition of \af{dep-compatible-fun}, we let Agda infer the type of \ab{𝒶}, which is \af Π~\ab i~\af ꞉~\ab I~\af ,~(\ab J~\as →~\ab 𝒜~\ab i) in this case; equivalently, \AgdaSymbol{∀}\AgdaSpace{}%
\AgdaBound{i}\AgdaSpace{}%
\AgdaSymbol{→}\AgdaSpace{}%
\AgdaSymbol{(}\AgdaBound{J}\AgdaSpace{}%
\AgdaSymbol{→}\AgdaSpace{}%
\AgdaBound{𝒜}\AgdaSpace{}%
\AgdaBound{i}\AgdaSymbol{)}.


%%%%%%%%%%%%%%%%%%%%%%%%%%%%%%%%%%%%%%%%%%%%%%%%%%%%%%%%%%%%%%%%%%%%%%%%%%%%
\subsection{Quotients}\label{sec:equiv-relat-quot}\firstsentence{Relations}{Quotients}
% \context{\{\ab 𝓤 𝓦 \as : \AgdaPostulate{Universe}\}\{\ab A \as : \ab 𝓤\AgdaOperator{\AgdaFunction{̇}}\}}
%: equivalences, class representatives, quotient types
% -*- TeX-master: "ualib-part1.tex" -*-
%%% Local Variables: 
%%% mode: latex
%%% TeX-engine: 'xetex
%%% End:
\subsubsection{Properties of binary relations}
\label{props-of-bin-rels}
In the \ualibhtml{Relations.Discrete} module we defined types for representing and reasoning about binary relations on \ab A. In this module we will define types for binary relations that have special properties. The most important special properties of relations are the ones we now define.
\ccpad
\begin{code}%
% \>[0]\AgdaKeyword{module}\AgdaSpace{}%
% \AgdaModule{\AgdaUnderscore{}}\AgdaSpace{}%
% \AgdaSymbol{\{}\AgdaBound{𝓤}\AgdaSpace{}%
% \AgdaSymbol{:}\AgdaSpace{}%
% \AgdaPostulate{Universe}\AgdaSymbol{\}\{}\AgdaBound{A}\AgdaSpace{}%
% \AgdaSymbol{:}\AgdaSpace{}%
% \AgdaBound{𝓤}\AgdaSpace{}%
% \AgdaOperator{\AgdaFunction{̇}}\AgdaSpace{}%
% \AgdaSymbol{\}\{}\AgdaBound{𝓦}\AgdaSpace{}%
% \AgdaSymbol{:}\AgdaSpace{}%
% \AgdaPostulate{Universe}\AgdaSymbol{\}}\AgdaSpace{}%
% \AgdaKeyword{where}\<%
% \\
% %
% \\[\AgdaEmptyExtraSkip]%
% \>[0][@{}l@{\AgdaIndent{0}}]%
\>[1]\AgdaFunction{reflexive}\AgdaSpace{}%
\AgdaSymbol{:}\AgdaSpace{}%
\AgdaFunction{Rel}\AgdaSpace{}%
\AgdaBound{A}\AgdaSpace{}%
\AgdaBound{𝓦}\AgdaSpace{}%
\AgdaSymbol{→}\AgdaSpace{}%
\AgdaBound{𝓤}\AgdaSpace{}%
\AgdaOperator{\AgdaPrimitive{⊔}}\AgdaSpace{}%
\AgdaBound{𝓦}\AgdaSpace{}%
\AgdaOperator{\AgdaFunction{̇}}\<%
\\
%
\>[1]\AgdaFunction{reflexive}\AgdaSpace{}%
\AgdaOperator{\AgdaBound{\AgdaUnderscore{}≈\AgdaUnderscore{}}}\AgdaSpace{}%
\AgdaSymbol{=}\AgdaSpace{}%
\AgdaSymbol{∀}\AgdaSpace{}%
\AgdaBound{x}\AgdaSpace{}%
\AgdaSymbol{→}\AgdaSpace{}%
\AgdaBound{x}\AgdaSpace{}%
\AgdaOperator{\AgdaBound{≈}}\AgdaSpace{}%
\AgdaBound{x}\<%
\\
%
\\[\AgdaEmptyExtraSkip]%
%
\>[1]\AgdaFunction{symmetric}\AgdaSpace{}%
\AgdaSymbol{:}\AgdaSpace{}%
\AgdaFunction{Rel}\AgdaSpace{}%
\AgdaBound{A}\AgdaSpace{}%
\AgdaBound{𝓦}\AgdaSpace{}%
\AgdaSymbol{→}\AgdaSpace{}%
\AgdaBound{𝓤}\AgdaSpace{}%
\AgdaOperator{\AgdaPrimitive{⊔}}\AgdaSpace{}%
\AgdaBound{𝓦}\AgdaSpace{}%
\AgdaOperator{\AgdaFunction{̇}}\<%
\\
%
\>[1]\AgdaFunction{symmetric}\AgdaSpace{}%
\AgdaOperator{\AgdaBound{\AgdaUnderscore{}≈\AgdaUnderscore{}}}\AgdaSpace{}%
\AgdaSymbol{=}\AgdaSpace{}%
\AgdaSymbol{∀}\AgdaSpace{}%
\AgdaBound{x}\AgdaSpace{}%
\AgdaBound{y}\AgdaSpace{}%
\AgdaSymbol{→}\AgdaSpace{}%
\AgdaBound{x}\AgdaSpace{}%
\AgdaOperator{\AgdaBound{≈}}\AgdaSpace{}%
\AgdaBound{y}\AgdaSpace{}%
\AgdaSymbol{→}\AgdaSpace{}%
\AgdaBound{y}\AgdaSpace{}%
\AgdaOperator{\AgdaBound{≈}}\AgdaSpace{}%
\AgdaBound{x}\<%
\\
%
\\[\AgdaEmptyExtraSkip]%
%
\>[1]\AgdaFunction{antisymmetric}\AgdaSpace{}%
\AgdaSymbol{:}\AgdaSpace{}%
\AgdaFunction{Rel}\AgdaSpace{}%
\AgdaBound{A}\AgdaSpace{}%
\AgdaBound{𝓦}\AgdaSpace{}%
\AgdaSymbol{→}\AgdaSpace{}%
\AgdaBound{𝓤}\AgdaSpace{}%
\AgdaOperator{\AgdaPrimitive{⊔}}\AgdaSpace{}%
\AgdaBound{𝓦}\AgdaSpace{}%
\AgdaOperator{\AgdaFunction{̇}}\<%
\\
%
\>[1]\AgdaFunction{antisymmetric}\AgdaSpace{}%
\AgdaOperator{\AgdaBound{\AgdaUnderscore{}≈\AgdaUnderscore{}}}\AgdaSpace{}%
\AgdaSymbol{=}\AgdaSpace{}%
\AgdaSymbol{∀}\AgdaSpace{}%
\AgdaBound{x}\AgdaSpace{}%
\AgdaBound{y}\AgdaSpace{}%
\AgdaSymbol{→}\AgdaSpace{}%
\AgdaBound{x}\AgdaSpace{}%
\AgdaOperator{\AgdaBound{≈}}\AgdaSpace{}%
\AgdaBound{y}\AgdaSpace{}%
\AgdaSymbol{→}\AgdaSpace{}%
\AgdaBound{y}\AgdaSpace{}%
\AgdaOperator{\AgdaBound{≈}}\AgdaSpace{}%
\AgdaBound{x}\AgdaSpace{}%
\AgdaSymbol{→}\AgdaSpace{}%
\AgdaBound{x}\AgdaSpace{}%
\AgdaOperator{\AgdaDatatype{≡}}\AgdaSpace{}%
\AgdaBound{y}\<%
\\
%
\\[\AgdaEmptyExtraSkip]%
%
\>[1]\AgdaFunction{transitive}\AgdaSpace{}%
\AgdaSymbol{:}\AgdaSpace{}%
\AgdaFunction{Rel}\AgdaSpace{}%
\AgdaBound{A}\AgdaSpace{}%
\AgdaBound{𝓦}\AgdaSpace{}%
\AgdaSymbol{→}\AgdaSpace{}%
\AgdaBound{𝓤}\AgdaSpace{}%
\AgdaOperator{\AgdaPrimitive{⊔}}\AgdaSpace{}%
\AgdaBound{𝓦}\AgdaSpace{}%
\AgdaOperator{\AgdaFunction{̇}}\<%
\\
%
\>[1]\AgdaFunction{transitive}\AgdaSpace{}%
\AgdaOperator{\AgdaBound{\AgdaUnderscore{}≈\AgdaUnderscore{}}}\AgdaSpace{}%
\AgdaSymbol{=}\AgdaSpace{}%
\AgdaSymbol{∀}\AgdaSpace{}%
\AgdaBound{x}\AgdaSpace{}%
\AgdaBound{y}\AgdaSpace{}%
\AgdaBound{z}\AgdaSpace{}%
\AgdaSymbol{→}\AgdaSpace{}%
\AgdaBound{x}\AgdaSpace{}%
\AgdaOperator{\AgdaBound{≈}}\AgdaSpace{}%
\AgdaBound{y}\AgdaSpace{}%
\AgdaSymbol{→}\AgdaSpace{}%
\AgdaBound{y}\AgdaSpace{}%
\AgdaOperator{\AgdaBound{≈}}\AgdaSpace{}%
\AgdaBound{z}\AgdaSpace{}%
\AgdaSymbol{→}\AgdaSpace{}%
\AgdaBound{x}\AgdaSpace{}%
\AgdaOperator{\AgdaBound{≈}}\AgdaSpace{}%
\AgdaBound{z}\<%
\end{code}
\ccpad
The \typetopology library defines a \defn{uniqueness-of-proofs} principle for binary relations.
\ccpad
\begin{code}%
% \>[0]\AgdaKeyword{module}\AgdaSpace{}%
% \AgdaModule{hide-is-subsingleton-valued}\AgdaSpace{}%
% \AgdaSymbol{\{}\AgdaBound{𝓤}\AgdaSpace{}%
% \AgdaBound{𝓦}\AgdaSpace{}%
% \AgdaSymbol{:}\AgdaSpace{}%
% \AgdaPostulate{Universe}\AgdaSymbol{\}\{}\AgdaBound{A}\AgdaSpace{}%
% \AgdaSymbol{:}\AgdaSpace{}%
% \AgdaBound{𝓤}\AgdaSpace{}%
% \AgdaOperator{\AgdaFunction{̇}}\AgdaSpace{}%
% \AgdaSymbol{\}}\AgdaSpace{}%
% \AgdaKeyword{where}\<%
% \\
% %
% \\[\AgdaEmptyExtraSkip]%
% \>[0][@{}l@{\AgdaIndent{0}}]%
\>[1]\AgdaFunction{is-subsingleton-valued}\AgdaSpace{}%
\AgdaSymbol{:}\AgdaSpace{}%
\AgdaFunction{Rel}\AgdaSpace{}%
\AgdaBound{A}\AgdaSpace{}%
\AgdaBound{𝓦}\AgdaSpace{}%
\AgdaSymbol{→}\AgdaSpace{}%
\AgdaBound{𝓤}\AgdaSpace{}%
\AgdaOperator{\AgdaPrimitive{⊔}}\AgdaSpace{}%
\AgdaBound{𝓦}\AgdaSpace{}%
\AgdaOperator{\AgdaFunction{̇}}\<%
\\
%
\>[1]\AgdaFunction{is-subsingleton-valued}%
\>[25]\AgdaOperator{\AgdaBound{\AgdaUnderscore{}≈\AgdaUnderscore{}}}\AgdaSpace{}%
\AgdaSymbol{=}\AgdaSpace{}%
\AgdaSymbol{∀}\AgdaSpace{}%
\AgdaBound{x}\AgdaSpace{}%
\AgdaBound{y}\AgdaSpace{}%
\AgdaSymbol{→}\AgdaSpace{}%
\AgdaFunction{is-subsingleton}\AgdaSpace{}%
\AgdaSymbol{(}\AgdaBound{x}\AgdaSpace{}%
\AgdaOperator{\AgdaBound{≈}}\AgdaSpace{}%
\AgdaBound{y}\AgdaSymbol{)}\<%
\end{code}
\ccpad
Thus, if \ab R \as : \af{Rel} \ab A \ab 𝓦, then \AgdaFunction{is-subsingleton-valued} \ab R asserts that for each pair \ab x \ab y \as : \ab A there is \emph{at most one proof} of \ab R \ab x \ab y. In the section on \emph{Truncation} below (\S~\ref{sec:trunc-sets-prop}) we introduce a number of similar but more general types to represent uniqueness-of-proofs principles for relations of arbitrary arity, over arbitrary types.







\subsubsection{Equivalence classes}\label{equivalence-classes}

A binary relation is called a \defn{preorder} if it is reflexive and transitive. An \defn{equivalence relation} is a symmetric preorder. Here are the types we use to represent these concepts in the \ualib.
\ccpad
\begin{code}%
% \>[0]\AgdaKeyword{module}\AgdaSpace{}%
% \AgdaModule{\AgdaUnderscore{}}\AgdaSpace{}%
% \AgdaSymbol{\{}\AgdaBound{𝓤}\AgdaSpace{}%
% \AgdaBound{𝓦}\AgdaSpace{}%
% \AgdaSymbol{:}\AgdaSpace{}%
% \AgdaPostulate{Universe}\AgdaSymbol{\}\{}\AgdaBound{A}\AgdaSpace{}%
% \AgdaSymbol{:}\AgdaSpace{}%
% \AgdaBound{𝓤}\AgdaSpace{}%
% \AgdaOperator{\AgdaFunction{̇}}\AgdaSpace{}%
% \AgdaSymbol{\}}\AgdaSpace{}%
% \AgdaKeyword{where}\<%
% \\
% %
% \\[\AgdaEmptyExtraSkip]%
% \>[0][@{}l@{\AgdaIndent{0}}]%
\>[1]\AgdaFunction{is-preorder}\AgdaSpace{}%
\AgdaSymbol{:}\AgdaSpace{}%
\AgdaFunction{Rel}\AgdaSpace{}%
\AgdaBound{A}\AgdaSpace{}%
\AgdaBound{𝓦}\AgdaSpace{}%
\AgdaSymbol{→}\AgdaSpace{}%
\AgdaBound{𝓤}\AgdaSpace{}%
\AgdaOperator{\AgdaPrimitive{⊔}}\AgdaSpace{}%
\AgdaBound{𝓦}\AgdaSpace{}%
\AgdaOperator{\AgdaFunction{̇}}\<%
\\
%
\>[1]\AgdaFunction{is-preorder}\AgdaSpace{}%
\AgdaOperator{\AgdaBound{\AgdaUnderscore{}≈\AgdaUnderscore{}}}\AgdaSpace{}%
\AgdaSymbol{=}\AgdaSpace{}%
\AgdaFunction{is-subsingleton-valued}\AgdaSpace{}%
\AgdaOperator{\AgdaBound{\AgdaUnderscore{}≈\AgdaUnderscore{}}}\AgdaSpace{}%
\AgdaOperator{\AgdaFunction{×}}\AgdaSpace{}%
\AgdaFunction{reflexive}\AgdaSpace{}%
\AgdaOperator{\AgdaBound{\AgdaUnderscore{}≈\AgdaUnderscore{}}}\AgdaSpace{}%
\AgdaOperator{\AgdaFunction{×}}\AgdaSpace{}%
\AgdaFunction{transitive}\AgdaSpace{}%
\AgdaOperator{\AgdaBound{\AgdaUnderscore{}≈\AgdaUnderscore{}}}\<%
\\
%
\\[\AgdaEmptyExtraSkip]%
%
\>[1]\AgdaKeyword{record}\AgdaSpace{}%
\AgdaRecord{IsEquivalence}\AgdaSpace{}%
\AgdaSymbol{(}\AgdaOperator{\AgdaBound{\AgdaUnderscore{}≈\AgdaUnderscore{}}}\AgdaSpace{}%
\AgdaSymbol{:}\AgdaSpace{}%
\AgdaFunction{Rel}\AgdaSpace{}%
\AgdaBound{A}\AgdaSpace{}%
\AgdaBound{𝓦}\AgdaSymbol{)}\AgdaSpace{}%
\AgdaSymbol{:}\AgdaSpace{}%
\AgdaBound{𝓤}\AgdaSpace{}%
\AgdaOperator{\AgdaPrimitive{⊔}}\AgdaSpace{}%
\AgdaBound{𝓦}\AgdaSpace{}%
\AgdaOperator{\AgdaFunction{̇}}\AgdaSpace{}%
\AgdaKeyword{where}\<%
\\
\>[1][@{}l@{\AgdaIndent{0}}]%
\>[2]\AgdaKeyword{field}\<%
\\
\>[2][@{}l@{\AgdaIndent{0}}]%
\>[3]\AgdaField{rfl}%
\>[9]\AgdaSymbol{:}\AgdaSpace{}%
\AgdaFunction{reflexive}\AgdaSpace{}%
\AgdaOperator{\AgdaBound{\AgdaUnderscore{}≈\AgdaUnderscore{}}}\<%
\\
%
\>[3]\AgdaField{sym}%
\>[9]\AgdaSymbol{:}\AgdaSpace{}%
\AgdaFunction{symmetric}\AgdaSpace{}%
\AgdaOperator{\AgdaBound{\AgdaUnderscore{}≈\AgdaUnderscore{}}}\<%
\\
%
\>[3]\AgdaField{trans}\AgdaSpace{}%
\AgdaSymbol{:}\AgdaSpace{}%
\AgdaFunction{transitive}\AgdaSpace{}%
\AgdaOperator{\AgdaBound{\AgdaUnderscore{}≈\AgdaUnderscore{}}}\<%
\\
%
\\[\AgdaEmptyExtraSkip]%
%
\>[1]\AgdaFunction{is-equivalence}\AgdaSpace{}%
\AgdaSymbol{:}\AgdaSpace{}%
\AgdaFunction{Rel}\AgdaSpace{}%
\AgdaBound{A}\AgdaSpace{}%
\AgdaBound{𝓦}\AgdaSpace{}%
\AgdaSymbol{→}\AgdaSpace{}%
\AgdaBound{𝓤}\AgdaSpace{}%
\AgdaOperator{\AgdaPrimitive{⊔}}\AgdaSpace{}%
\AgdaBound{𝓦}\AgdaSpace{}%
\AgdaOperator{\AgdaFunction{̇}}\<%
\\
%
\>[1]\AgdaFunction{is-equivalence}\AgdaSpace{}%
\AgdaOperator{\AgdaBound{\AgdaUnderscore{}≈\AgdaUnderscore{}}}\AgdaSpace{}%
\AgdaSymbol{=}\AgdaSpace{}%
\AgdaFunction{is-preorder}\AgdaSpace{}%
\AgdaOperator{\AgdaBound{\AgdaUnderscore{}≈\AgdaUnderscore{}}}\AgdaSpace{}%
\AgdaOperator{\AgdaFunction{×}}\AgdaSpace{}%
\AgdaFunction{symmetric}\AgdaSpace{}%
\AgdaOperator{\AgdaBound{\AgdaUnderscore{}≈\AgdaUnderscore{}}}\<%
\end{code}
\scpad

An easy first example of an equivalence relation is the kernel of any function. Here is how we prove that the kernel of a function is, indeed, an equivalence relation on the domain of the function.
\ccpad
\begin{code}%
% \>[0]\AgdaKeyword{module}\AgdaSpace{}%
% \AgdaModule{\AgdaUnderscore{}}\AgdaSpace{}%
% \AgdaSymbol{\{}\AgdaBound{𝓤}\AgdaSpace{}%
% \AgdaBound{𝓦}\AgdaSpace{}%
% \AgdaSymbol{:}\AgdaSpace{}%
% \AgdaPostulate{Universe}\AgdaSymbol{\}\{}\AgdaBound{A}\AgdaSpace{}%
% \AgdaSymbol{:}\AgdaSpace{}%
% \AgdaBound{𝓤}\AgdaSpace{}%
% \AgdaOperator{\AgdaFunction{̇}}\AgdaSymbol{\}\{}\AgdaBound{B}\AgdaSpace{}%
% \AgdaSymbol{:}\AgdaSpace{}%
% \AgdaBound{𝓦}\AgdaSpace{}%
% \AgdaOperator{\AgdaFunction{̇}}\AgdaSymbol{\}}\AgdaSpace{}%
% \AgdaKeyword{where}\<%
% \\
% %
% \\[\AgdaEmptyExtraSkip]%
% \>[0][@{}l@{\AgdaIndent{0}}]%
\>[1]\AgdaFunction{map-kernel-IsEquivalence}\AgdaSpace{}%
\AgdaSymbol{:}\AgdaSpace{}%
\AgdaSymbol{(}\AgdaBound{f}\AgdaSpace{}%
\AgdaSymbol{:}\AgdaSpace{}%
\AgdaBound{A}\AgdaSpace{}%
\AgdaSymbol{→}\AgdaSpace{}%
\AgdaBound{B}\AgdaSymbol{)}\AgdaSpace{}%
\AgdaSymbol{→}\AgdaSpace{}%
\AgdaRecord{IsEquivalence}\AgdaSpace{}%
\AgdaSymbol{(}\AgdaFunction{ker}\AgdaSymbol{\{}\AgdaBound{𝓤}\AgdaSymbol{\}\{}\AgdaBound{𝓦}\AgdaSymbol{\}}\AgdaSpace{}%
\AgdaBound{f}\AgdaSymbol{)}\<%
\\
%
\>[1]\AgdaFunction{map-kernel-IsEquivalence}\AgdaSpace{}%
\AgdaBound{f}\AgdaSpace{}%
\AgdaSymbol{=}\AgdaSpace{}%
\AgdaKeyword{record}%
\>[235I]\AgdaSymbol{\{}\AgdaSpace{}%
\AgdaField{rfl}\AgdaSpace{}%
\AgdaSymbol{=}\AgdaSpace{}%
\AgdaSymbol{λ}\AgdaSpace{}%
\AgdaBound{x}\AgdaSpace{}%
\AgdaSymbol{→}\AgdaSpace{}%
\AgdaInductiveConstructor{refl}\<%
\\
\>[.][@{}l@{}]\<[235I]%
\>[37]\AgdaSymbol{;}\AgdaSpace{}%
\AgdaField{sym}\AgdaSpace{}%
\AgdaSymbol{=}\AgdaSpace{}%
\AgdaSymbol{λ}\AgdaSpace{}%
\AgdaBound{x}\AgdaSpace{}%
\AgdaBound{y}\AgdaSpace{}%
\AgdaBound{x₁}\AgdaSpace{}%
\AgdaSymbol{→}\AgdaSpace{}%
\AgdaFunction{≡-sym}\AgdaSymbol{\{}\AgdaBound{𝓦}\AgdaSymbol{\}}\AgdaSpace{}%
\AgdaBound{x₁}\<%
\\
%
\>[37]\AgdaSymbol{;}\AgdaSpace{}%
\AgdaField{trans}\AgdaSpace{}%
\AgdaSymbol{=}\AgdaSpace{}%
\AgdaSymbol{λ}\AgdaSpace{}%
\AgdaBound{x}\AgdaSpace{}%
\AgdaBound{y}\AgdaSpace{}%
\AgdaBound{z}\AgdaSpace{}%
\AgdaBound{x₁}\AgdaSpace{}%
\AgdaBound{x₂}\AgdaSpace{}%
\AgdaSymbol{→}\AgdaSpace{}%
\AgdaFunction{≡-trans}\AgdaSpace{}%
\AgdaBound{x₁}\AgdaSpace{}%
\AgdaBound{x₂}\AgdaSpace{}%
\AgdaSymbol{\}}\<%
\end{code}

\subsubsection{Equivalence classes}\label{equivalence-classes-1}

If \ab R is an equivalence relation on \ab A, then for each \abt{𝑎}{A}, there is an \defn{equivalence class} containing \ab 𝑎, which we denote and define by \af{[} \ab 𝑎 \af{]} \ab R \as{:=} all \abt{𝑏}{A} such that \ab R \ab 𝑎 \ab 𝑏.
\ccpad
\begin{code}%
% \>[0]\AgdaKeyword{module}\AgdaSpace{}%
% \AgdaModule{\AgdaUnderscore{}}\AgdaSpace{}%
% \AgdaSymbol{\{}\AgdaBound{𝓤}\AgdaSpace{}%
% \AgdaBound{𝓦}\AgdaSpace{}%
% \AgdaSymbol{:}\AgdaSpace{}%
% \AgdaPostulate{Universe}\AgdaSymbol{\}\{}\AgdaBound{A}\AgdaSpace{}%
% \AgdaSymbol{:}\AgdaSpace{}%
% \AgdaBound{𝓤}\AgdaSpace{}%
% \AgdaOperator{\AgdaFunction{̇}}\AgdaSpace{}%
% \AgdaSymbol{\}}\AgdaSpace{}%
% \AgdaKeyword{where}\<%
% \\
% %
% \\[\AgdaEmptyExtraSkip]%
% \>[0][@{}l@{\AgdaIndent{0}}]%
\>[1]\AgdaOperator{\AgdaFunction{[\AgdaUnderscore{}]\AgdaUnderscore{}}}\AgdaSpace{}%
\AgdaSymbol{:}\AgdaSpace{}%
\AgdaBound{A}\AgdaSpace{}%
\AgdaSymbol{→}\AgdaSpace{}%
\AgdaFunction{Rel}\AgdaSpace{}%
\AgdaBound{A}\AgdaSpace{}%
\AgdaBound{𝓦}\AgdaSpace{}%
\AgdaSymbol{→}\AgdaSpace{}%
\AgdaFunction{Pred}\AgdaSpace{}%
\AgdaBound{A}\AgdaSpace{}%
\AgdaBound{𝓦}\<%
\\
%
\>[1]\AgdaOperator{\AgdaFunction{[}}\AgdaSpace{}%
\AgdaBound{𝑎}\AgdaSpace{}%
\AgdaOperator{\AgdaFunction{]}}\AgdaSpace{}%
\AgdaBound{R}\AgdaSpace{}%
\AgdaSymbol{=}\AgdaSpace{}%
\AgdaSymbol{λ}\AgdaSpace{}%
\AgdaBound{x}\AgdaSpace{}%
\AgdaSymbol{→}\AgdaSpace{}%
\AgdaBound{R}\AgdaSpace{}%
\AgdaBound{𝑎}\AgdaSpace{}%
\AgdaBound{x}\<%
\end{code}
\ccpad
Thus, \ab{x} \af ∈ \as{[} \ab a \as{]} \ab R if and only if \ab{R} \ab a \ab x, as desired. We often refer to \af{[} \ab 𝑎 \af{]} \ab R as the \emph{R-class containing} \ab 𝑎, and we represent the collection of all such \ab R-classes by the following type.
\ccpad
\begin{code}%
\>[1]\AgdaFunction{𝒞}\AgdaSpace{}%
\AgdaSymbol{:}\AgdaSpace{}%
\AgdaSymbol{\{}\AgdaBound{R}\AgdaSpace{}%
\AgdaSymbol{:}\AgdaSpace{}%
\AgdaFunction{Rel}\AgdaSpace{}%
\AgdaBound{A}\AgdaSpace{}%
\AgdaBound{𝓦}\AgdaSymbol{\}}\AgdaSpace{}%
\AgdaSymbol{→}\AgdaSpace{}%
\AgdaFunction{Pred}\AgdaSpace{}%
\AgdaBound{A}\AgdaSpace{}%
\AgdaBound{𝓦}\AgdaSpace{}%
\AgdaSymbol{→}\AgdaSpace{}%
\AgdaSymbol{(}\AgdaBound{𝓤}\AgdaSpace{}%
\AgdaOperator{\AgdaPrimitive{⊔}}\AgdaSpace{}%
\AgdaBound{𝓦}\AgdaSpace{}%
\AgdaOperator{\AgdaPrimitive{⁺}}\AgdaSymbol{)}\AgdaSpace{}%
\AgdaOperator{\AgdaFunction{̇}}\<%
\\
%
\>[1]\AgdaFunction{𝒞}%
\>[4]\AgdaSymbol{\{}\AgdaBound{R}\AgdaSymbol{\}}\AgdaSpace{}%
\AgdaBound{C}\AgdaSpace{}%
\AgdaSymbol{=}\AgdaSpace{}%
\AgdaFunction{Σ}\AgdaSpace{}%
\AgdaBound{a}\AgdaSpace{}%
\AgdaFunction{꞉}\AgdaSpace{}%
\AgdaBound{A}\AgdaSpace{}%
\AgdaFunction{,}\AgdaSpace{}%
\AgdaBound{C}\AgdaSpace{}%
\AgdaOperator{\AgdaDatatype{≡}}\AgdaSpace{}%
\AgdaSymbol{(}\AgdaSpace{}%
\AgdaOperator{\AgdaFunction{[}}\AgdaSpace{}%
\AgdaBound{a}\AgdaSpace{}%
\AgdaOperator{\AgdaFunction{]}}\AgdaSpace{}%
\AgdaBound{R}\AgdaSymbol{)}\<%
\end{code}
\ccpad
If \ab{R} is an equivalence relation on \ab{A}, then the \defn{quotient} of \ab{A} modulo \ab{R}, denoted by
\ab{A} \aof / \ab R, is defined as the collection \as{\{}\af{[} \ab 𝑎 \af{]} \ab R \as{∣}  \abt{𝑎}{A}\as{\}} of equivalence classes of \ab{R}. There are a few ways we could represent the quotient with respect to a relation as a type, but we find the following to be the most useful.
\ccpad
\begin{code}%
% \>[0]\AgdaKeyword{module}\AgdaSpace{}%
% \AgdaModule{\AgdaUnderscore{}}\AgdaSpace{}%
% \AgdaSymbol{\{}\AgdaBound{𝓤}\AgdaSpace{}%
% \AgdaBound{𝓦}\AgdaSpace{}%
% \AgdaSymbol{:}\AgdaSpace{}%
% \AgdaPostulate{Universe}\AgdaSymbol{\}}\AgdaSpace{}%
% \AgdaKeyword{where}\<%
% \\
% %
% \\[\AgdaEmptyExtraSkip]%
% \>[0][@{}l@{\AgdaIndent{0}}]%
\>[1]\AgdaOperator{\AgdaFunction{\AgdaUnderscore{}/\AgdaUnderscore{}}}\AgdaSpace{}%
\AgdaSymbol{:}\AgdaSpace{}%
\AgdaSymbol{(}\AgdaBound{A}\AgdaSpace{}%
\AgdaSymbol{:}\AgdaSpace{}%
\AgdaBound{𝓤}\AgdaSpace{}%
\AgdaOperator{\AgdaFunction{̇}}\AgdaSpace{}%
\AgdaSymbol{)}\AgdaSpace{}%
\AgdaSymbol{→}\AgdaSpace{}%
\AgdaFunction{Rel}\AgdaSpace{}%
\AgdaBound{A}\AgdaSpace{}%
\AgdaBound{𝓦}\AgdaSpace{}%
\AgdaSymbol{→}\AgdaSpace{}%
\AgdaBound{𝓤}\AgdaSpace{}%
\AgdaOperator{\AgdaPrimitive{⊔}}\AgdaSpace{}%
\AgdaSymbol{(}\AgdaBound{𝓦}\AgdaSpace{}%
\AgdaOperator{\AgdaPrimitive{⁺}}\AgdaSymbol{)}\AgdaSpace{}%
\AgdaOperator{\AgdaFunction{̇}}\<%
\\
%
\>[1]\AgdaBound{A}\AgdaSpace{}%
\AgdaOperator{\AgdaFunction{/}}\AgdaSpace{}%
\AgdaBound{R}\AgdaSpace{}%
\AgdaSymbol{=}\AgdaSpace{}%
\AgdaFunction{Σ}\AgdaSpace{}%
\AgdaBound{C}\AgdaSpace{}%
\AgdaFunction{꞉}\AgdaSpace{}%
\AgdaFunction{Pred}\AgdaSpace{}%
\AgdaBound{A}\AgdaSpace{}%
\AgdaBound{𝓦}\AgdaSpace{}%
\AgdaFunction{,}%
\>[27]\AgdaFunction{𝒞}\AgdaSpace{}%
\AgdaSymbol{\{}\AgdaArgument{R}\AgdaSpace{}%
\AgdaSymbol{=}\AgdaSpace{}%
\AgdaBound{R}\AgdaSymbol{\}}\AgdaSpace{}%
\AgdaBound{C}\<%
\end{code}
\ccpad
The next type is used to represent an `R`-class with a designated representative.
\ccpad
\begin{code}%
\>[0]\AgdaKeyword{module}\AgdaSpace{}%
\AgdaModule{\AgdaUnderscore{}}\AgdaSpace{}%
\AgdaSymbol{\{}\AgdaBound{𝓤}\AgdaSpace{}%
\AgdaBound{𝓦}\AgdaSpace{}%
\AgdaSymbol{:}\AgdaSpace{}%
\AgdaPostulate{Universe}\AgdaSymbol{\}\{}\AgdaBound{A}\AgdaSpace{}%
\AgdaSymbol{:}\AgdaSpace{}%
\AgdaBound{𝓤}\AgdaSpace{}%
\AgdaOperator{\AgdaFunction{̇}}\AgdaSymbol{\}}\AgdaSpace{}%
\AgdaKeyword{where}\<%
\\
%
\\[\AgdaEmptyExtraSkip]%
\>[0][@{}l@{\AgdaIndent{0}}]%
\>[1]\AgdaOperator{\AgdaFunction{⟦\AgdaUnderscore{}⟧}}\AgdaSpace{}%
\AgdaSymbol{:}\AgdaSpace{}%
\AgdaBound{A}\AgdaSpace{}%
\AgdaSymbol{→}\AgdaSpace{}%
\AgdaSymbol{\{}\AgdaBound{R}\AgdaSpace{}%
\AgdaSymbol{:}\AgdaSpace{}%
\AgdaFunction{Rel}\AgdaSpace{}%
\AgdaBound{A}\AgdaSpace{}%
\AgdaBound{𝓦}\AgdaSymbol{\}}\AgdaSpace{}%
\AgdaSymbol{→}\AgdaSpace{}%
\AgdaBound{A}\AgdaSpace{}%
\AgdaOperator{\AgdaFunction{/}}\AgdaSpace{}%
\AgdaBound{R}\<%
\\
%
\>[1]\AgdaOperator{\AgdaFunction{⟦}}\AgdaSpace{}%
\AgdaBound{a}\AgdaSpace{}%
\AgdaOperator{\AgdaFunction{⟧}}\AgdaSpace{}%
\AgdaSymbol{\{}\AgdaBound{R}\AgdaSymbol{\}}\AgdaSpace{}%
\AgdaSymbol{=}\AgdaSpace{}%
\AgdaOperator{\AgdaFunction{[}}\AgdaSpace{}%
\AgdaBound{a}\AgdaSpace{}%
\AgdaOperator{\AgdaFunction{]}}\AgdaSpace{}%
\AgdaBound{R}\AgdaSpace{}%
\AgdaOperator{\AgdaInductiveConstructor{,}}\AgdaSpace{}%
\AgdaBound{a}\AgdaSpace{}%
\AgdaOperator{\AgdaInductiveConstructor{,}}\AgdaSpace{}%
\AgdaInductiveConstructor{refl}\<%
\end{code}
\ccpad
This serves as a kind of \emph{introduction rule}.  Dually, the next type provides an *elimination rule*.\footnote{%
\textbf{Unicode Hint}. Type \af ⌜ and \af ⌝ as \texttt{\textbackslash{}cul} and \texttt{\textbackslash{}cur} in \agdamode.}
\ccpad
\begin{code}%
\>[1]\AgdaOperator{\AgdaFunction{⌜\AgdaUnderscore{}⌝}}\AgdaSpace{}%
\AgdaSymbol{:}\AgdaSpace{}%
\AgdaSymbol{\{}\AgdaBound{R}\AgdaSpace{}%
\AgdaSymbol{:}\AgdaSpace{}%
\AgdaFunction{Rel}\AgdaSpace{}%
\AgdaBound{A}\AgdaSpace{}%
\AgdaBound{𝓦}\AgdaSymbol{\}}\AgdaSpace{}%
\AgdaSymbol{→}\AgdaSpace{}%
\AgdaBound{A}\AgdaSpace{}%
\AgdaOperator{\AgdaFunction{/}}\AgdaSpace{}%
\AgdaBound{R}%
\>[30]\AgdaSymbol{→}\AgdaSpace{}%
\AgdaBound{A}\<%
\\
%
\\[\AgdaEmptyExtraSkip]%
%
\>[1]\AgdaOperator{\AgdaFunction{⌜}}\AgdaSpace{}%
\AgdaBound{𝕔}\AgdaSpace{}%
\AgdaOperator{\AgdaFunction{⌝}}\AgdaSpace{}%
\AgdaSymbol{=}\AgdaSpace{}%
\AgdaFunction{fst}\AgdaSpace{}%
\AgdaOperator{\AgdaFunction{∥}}\AgdaSpace{}%
\AgdaBound{𝕔}\AgdaSpace{}%
\AgdaOperator{\AgdaFunction{∥}}\<%
\end{code}
\scpad

Later we will need the following tools for working with the quotient types defined above.
\ccpad
\begin{code}%
\>[1]\AgdaFunction{/-subset}\AgdaSpace{}%
\AgdaSymbol{:}\AgdaSpace{}%
\AgdaSymbol{\{}\AgdaBound{x}\AgdaSpace{}%
\AgdaBound{y}\AgdaSpace{}%
\AgdaSymbol{:}\AgdaSpace{}%
\AgdaBound{A}\AgdaSymbol{\}\{}\AgdaBound{R}\AgdaSpace{}%
\AgdaSymbol{:}\AgdaSpace{}%
\AgdaFunction{Rel}\AgdaSpace{}%
\AgdaBound{A}\AgdaSpace{}%
\AgdaBound{𝓦}\AgdaSymbol{\}}\AgdaSpace{}%
\AgdaSymbol{→}\AgdaSpace{}%
\AgdaRecord{IsEquivalence}\AgdaSpace{}%
\AgdaBound{R}\AgdaSpace{}%
\AgdaSymbol{→}\AgdaSpace{}%
\AgdaBound{R}\AgdaSpace{}%
\AgdaBound{x}\AgdaSpace{}%
\AgdaBound{y}\AgdaSpace{}%
\AgdaSymbol{→}%
\>[64]\AgdaOperator{\AgdaFunction{[}}\AgdaSpace{}%
\AgdaBound{x}\AgdaSpace{}%
\AgdaOperator{\AgdaFunction{]}}\AgdaSpace{}%
\AgdaBound{R}%
\>[73]\AgdaOperator{\AgdaFunction{⊆}}%
\>[76]\AgdaOperator{\AgdaFunction{[}}\AgdaSpace{}%
\AgdaBound{y}\AgdaSpace{}%
\AgdaOperator{\AgdaFunction{]}}\AgdaSpace{}%
\AgdaBound{R}\<%
\\
%
\>[1]\AgdaFunction{/-subset}\AgdaSpace{}%
\AgdaSymbol{\{}\AgdaBound{x}\AgdaSymbol{\}\{}\AgdaBound{y}\AgdaSymbol{\}}\AgdaSpace{}%
\AgdaBound{Req}\AgdaSpace{}%
\AgdaBound{Rxy}\AgdaSpace{}%
\AgdaSymbol{\{}\AgdaBound{z}\AgdaSymbol{\}}\AgdaSpace{}%
\AgdaBound{Rxz}\AgdaSpace{}%
\AgdaSymbol{=}\AgdaSpace{}%
\AgdaSymbol{(}\AgdaField{trans}\AgdaSpace{}%
\AgdaBound{Req}\AgdaSymbol{)}\AgdaSpace{}%
\AgdaBound{y}\AgdaSpace{}%
\AgdaBound{x}\AgdaSpace{}%
\AgdaBound{z}\AgdaSpace{}%
\AgdaSymbol{(}\AgdaField{sym}\AgdaSpace{}%
\AgdaBound{Req}\AgdaSpace{}%
\AgdaBound{x}\AgdaSpace{}%
\AgdaBound{y}\AgdaSpace{}%
\AgdaBound{Rxy}\AgdaSymbol{)}\AgdaSpace{}%
\AgdaBound{Rxz}\<%
\\
%
\\[\AgdaEmptyExtraSkip]%
%
\>[1]\AgdaFunction{/-supset}\AgdaSpace{}%
\AgdaSymbol{:}\AgdaSpace{}%
\AgdaSymbol{\{}\AgdaBound{x}\AgdaSpace{}%
\AgdaBound{y}\AgdaSpace{}%
\AgdaSymbol{:}\AgdaSpace{}%
\AgdaBound{A}\AgdaSymbol{\}\{}\AgdaBound{R}\AgdaSpace{}%
\AgdaSymbol{:}\AgdaSpace{}%
\AgdaFunction{Rel}\AgdaSpace{}%
\AgdaBound{A}\AgdaSpace{}%
\AgdaBound{𝓦}\AgdaSymbol{\}}\AgdaSpace{}%
\AgdaSymbol{→}\AgdaSpace{}%
\AgdaRecord{IsEquivalence}\AgdaSpace{}%
\AgdaBound{R}\AgdaSpace{}%
\AgdaSymbol{→}\AgdaSpace{}%
\AgdaBound{R}\AgdaSpace{}%
\AgdaBound{x}\AgdaSpace{}%
\AgdaBound{y}\AgdaSpace{}%
\AgdaSymbol{→}%
\>[64]\AgdaOperator{\AgdaFunction{[}}\AgdaSpace{}%
\AgdaBound{y}\AgdaSpace{}%
\AgdaOperator{\AgdaFunction{]}}\AgdaSpace{}%
\AgdaBound{R}\AgdaSpace{}%
\AgdaOperator{\AgdaFunction{⊆}}\AgdaSpace{}%
\AgdaOperator{\AgdaFunction{[}}\AgdaSpace{}%
\AgdaBound{x}\AgdaSpace{}%
\AgdaOperator{\AgdaFunction{]}}\AgdaSpace{}%
\AgdaBound{R}\<%
\\
%
\>[1]\AgdaFunction{/-supset}\AgdaSpace{}%
\AgdaSymbol{\{}\AgdaBound{x}\AgdaSymbol{\}\{}\AgdaBound{y}\AgdaSymbol{\}}\AgdaSpace{}%
\AgdaBound{Req}\AgdaSpace{}%
\AgdaBound{Rxy}\AgdaSpace{}%
\AgdaSymbol{\{}\AgdaBound{z}\AgdaSymbol{\}}\AgdaSpace{}%
\AgdaBound{Ryz}\AgdaSpace{}%
\AgdaSymbol{=}\AgdaSpace{}%
\AgdaSymbol{(}\AgdaField{trans}\AgdaSpace{}%
\AgdaBound{Req}\AgdaSymbol{)}\AgdaSpace{}%
\AgdaBound{x}\AgdaSpace{}%
\AgdaBound{y}\AgdaSpace{}%
\AgdaBound{z}\AgdaSpace{}%
\AgdaBound{Rxy}\AgdaSpace{}%
\AgdaBound{Ryz}\<%
\\
%
\\[\AgdaEmptyExtraSkip]%
%
\>[1]\AgdaFunction{/-≐}\AgdaSpace{}%
\AgdaSymbol{:}\AgdaSpace{}%
\AgdaSymbol{\{}\AgdaBound{x}\AgdaSpace{}%
\AgdaBound{y}\AgdaSpace{}%
\AgdaSymbol{:}\AgdaSpace{}%
\AgdaBound{A}\AgdaSymbol{\}\{}\AgdaBound{R}\AgdaSpace{}%
\AgdaSymbol{:}\AgdaSpace{}%
\AgdaFunction{Rel}\AgdaSpace{}%
\AgdaBound{A}\AgdaSpace{}%
\AgdaBound{𝓦}\AgdaSymbol{\}}\AgdaSpace{}%
\AgdaSymbol{→}\AgdaSpace{}%
\AgdaRecord{IsEquivalence}\AgdaSpace{}%
\AgdaBound{R}\AgdaSpace{}%
\AgdaSymbol{→}\AgdaSpace{}%
\AgdaBound{R}\AgdaSpace{}%
\AgdaBound{x}\AgdaSpace{}%
\AgdaBound{y}\AgdaSpace{}%
\AgdaSymbol{→}%
\>[60]\AgdaOperator{\AgdaFunction{[}}\AgdaSpace{}%
\AgdaBound{x}\AgdaSpace{}%
\AgdaOperator{\AgdaFunction{]}}\AgdaSpace{}%
\AgdaBound{R}%
\>[69]\AgdaOperator{\AgdaFunction{≐}}%
\>[72]\AgdaOperator{\AgdaFunction{[}}\AgdaSpace{}%
\AgdaBound{y}\AgdaSpace{}%
\AgdaOperator{\AgdaFunction{]}}\AgdaSpace{}%
\AgdaBound{R}\<%
\\
%
\>[1]\AgdaFunction{/-≐}\AgdaSpace{}%
\AgdaBound{Req}\AgdaSpace{}%
\AgdaBound{Rxy}\AgdaSpace{}%
\AgdaSymbol{=}\AgdaSpace{}%
\AgdaFunction{/-subset}\AgdaSpace{}%
\AgdaBound{Req}\AgdaSpace{}%
\AgdaBound{Rxy}\AgdaSpace{}%
\AgdaOperator{\AgdaInductiveConstructor{,}}\AgdaSpace{}%
\AgdaFunction{/-supset}\AgdaSpace{}%
\AgdaBound{Req}\AgdaSpace{}%
\AgdaBound{Rxy}\<%
\end{code}


\subsection{Truncation}\label{sec:trunc-sets-prop}
%: continuous propositions, quotient extensionality
\firstsentence{Relations}{Truncation}
% -*- TeX-master: "ualib-part1.tex" -*-
%%% Local Variables: 
%%% mode: latex
%%% TeX-engine: 'xetex
%%% End:
Here we discuss \defn{truncation} and \defn{h-sets} (which we just call \defn{sets}). We first give a brief discussion of standard notions of \emph{truncation} from a viewpoint that seems useful for formalizing mathematics in Agda.\footnote{Readers wishing to learn more about truncation may wish to consult~\cite[\S34]{MHE},~\cite{Brunerie:2012}, or~\cite[\S7.1]{HoTT}.}

\subsubsection*{Uniqueness of identity proofs} %\label{sec:uniq-ident-proofs}
This brief introduction is intended for novices; those already familiar with the concept of \emph{truncation} and \emph{uniqueness of identity proofs} may want to skip to the next subsection.

In general, we may have multiple inhabitants of a given type, hence (via Curry-Howard) multiple proofs of a given proposition. For instance, suppose we have a type \ab{X} and an identity relation \aod{\_≡₀\_} on \ab{X} so that, given two inhabitants of \ab{X}, say, \ab{a} \ab b \as : \ab X, we can form the type \ab{a} \aod{≡₀} \ab b. Suppose \ab{p} and \ab{q} inhabit the type \ab{a} \aod{≡₀} \ab b; that is, \ab{p} and \ab{q} are proofs of \ab{a} \aod{≡₀} \ab b, in which case we write \ab{p} \ab q \as : \ab a \aod{≡₀} \ab b. We might then wonder whether the two proofs \ab{p} and \ab{q} are equivalent.

We are asking about an identity type on the identity type \aod{≡₀}, and whether there is some inhabitant, say, \ab{r} of this type; i.e., whether there is a proof \ab{r} \as : \ab p \aod{≡₁} \ab q that the proofs of \ab{a} \aod{≡₀} \ab{b} are the same. If such a proof exists for all \ab{p} \ab q \as : \ab{a} \aod{≡₀} \ab b, then the proof of \ab a \aod{≡₀} \ab b is unique; as a property of the types \ab{X} and \ad{≡₀}, this is sometimes called \defn{uniqueness of identity proofs}.

Now, perhaps we have two proofs, say, \ab{r} \ab s \as : \ab p \aod{≡₁} \ab q that the proofs \ab{p} and \ab{q} are equivalent. Then of course we wonder whether \ab{r} \aod{≡₂} \ab s has a proof!  But at some level we may decide that the potential to distinguish two proofs of an identity in a meaningful way (so-called \emph{proof-relevance}) is not useful or desirable. At that point, say, at level \ab{k}, we would be naturally inclined to assume that there is at most one proof of any identity of the form \ab{p} \aod{≡ₖ} \ab q. This is called \href{https://www.cs.bham.ac.uk/~mhe/HoTT-UF-in-Agda-Lecture-Notes/HoTT-UF-Agda.html\#truncation}{truncation} (at level \ab{k}).

\paragraph*{Sets} %\label{sec:sets}
In \href{https://homotopytypetheory.org}{homotopy type theory}, a type \ab{X} with an identity relation \ad{≡₀} is called a \defn{set} (or \defn{0-groupoid}) if for every pair \ab{x} \abt{y}{X} there is at most one proof of \ab{x} \aod{≡₀} \ab y. In other words, the type \ab{X}, along with it's equality type \ad{≡ₓ}, form a \emph{set} if for all \ab{x} \abt{y}{X} there is at most one proof of \ab{x} \aod{≡₀} \ab y.

This notion is formalized in \typtop using the type \af{is-set} which is defined using the \af{is-subsingleton} type (\S\ref{sec:inverse-image-invers}) as follows.
\ccpad
\begin{code}%
\>[1]\AgdaFunction{is-set}\AgdaSpace{}%
\AgdaSymbol{:}\AgdaSpace{}%
\AgdaBound{𝓤}\AgdaSpace{}%
\AgdaOperator{\AgdaFunction{̇}}\AgdaSpace{}%
\AgdaSymbol{→}\AgdaSpace{}%
\AgdaBound{𝓤}\AgdaSpace{}%
\AgdaOperator{\AgdaFunction{̇}}\<%
\\
%
\>[1]\AgdaFunction{is-set}\AgdaSpace{}%
\AgdaBound{A}\AgdaSpace{}%
\AgdaSymbol{=}\AgdaSpace{}%
\AgdaSymbol{(}\AgdaBound{x}\AgdaSpace{}%
\AgdaBound{y}\AgdaSpace{}%
\AgdaSymbol{:}\AgdaSpace{}%
\AgdaBound{A}\AgdaSymbol{)}\AgdaSpace{}%
\AgdaSymbol{→}\AgdaSpace{}%
\AgdaFunction{is-subsingleton}\AgdaSpace{}%
\AgdaSymbol{(}\AgdaBound{x}\AgdaSpace{}%
\AgdaOperator{\AgdaDatatype{≡}}\AgdaSpace{}%
\AgdaBound{y}\AgdaSymbol{)}\<%
\end{code}
\ccpad
Thus, the pair (\ab{X} , \ad{≡₀}) forms a set iff it satisfies \as{∀} \ab x \abt{y}{X} \as → \af{is-subsingleton} (\ab x \aod{≡₀} \ab y).

We will also need the function
\href{https://www.cs.bham.ac.uk/~mhe/HoTT-UF-in-Agda-Lecture-Notes/HoTT-UF-Agda.html\#sigmaequality}{to-Σ-≡},
which is part of Escardó's characterization of \emph{equality in Sigma types} in~\cite{MHE} and is defined as follows.
\ccpad
\begin{code}%
\>[1]\AgdaFunction{to-Σ-≡}\AgdaSpace{}%
\AgdaSymbol{:}\AgdaSpace{}%
\>[111I]\AgdaSymbol{\{}\AgdaBound{A}\AgdaSpace{}%
\AgdaSymbol{:}\AgdaSpace{}%
\AgdaBound{𝓤}\AgdaSpace{}%
\AgdaOperator{\AgdaFunction{̇}}%
\AgdaSymbol{\}}\AgdaSpace{}%
\AgdaSymbol{\{}\AgdaBound{B}\AgdaSpace{}%
\AgdaSymbol{:}\AgdaSpace{}%
\AgdaBound{A}\AgdaSpace{}%
\AgdaSymbol{→}\AgdaSpace{}%
\AgdaBound{𝓦}\AgdaSpace{}%
\AgdaOperator{\AgdaFunction{̇}}%
\AgdaSymbol{\}}\AgdaSpace{}%
\AgdaSymbol{\{}\AgdaBound{σ}\AgdaSpace{}%
\AgdaBound{τ}\AgdaSpace{}%
\AgdaSymbol{:}\AgdaSpace{}%
\AgdaRecord{Σ}\AgdaSpace{}%
\AgdaBound{B}\AgdaSymbol{\}}\<%
\\
\>[1][@{}l@{\AgdaIndent{0}}]%
\>[2]\AgdaSymbol{→}%
\>[.][@{}l@{}]\<[111I]%
\>[10]\AgdaFunction{Σ}\AgdaSpace{}%
\AgdaBound{p}\AgdaSpace{}%
\AgdaFunction{꞉}\AgdaSpace{}%
\AgdaOperator{\AgdaFunction{∣}}\AgdaSpace{}%
\AgdaBound{σ}\AgdaSpace{}%
\AgdaOperator{\AgdaFunction{∣}}\AgdaSpace{}%
\AgdaOperator{\AgdaDatatype{≡}}\AgdaSpace{}%
\AgdaOperator{\AgdaFunction{∣}}\AgdaSpace{}%
\AgdaBound{τ}\AgdaSpace{}%
\AgdaOperator{\AgdaFunction{∣}}\AgdaSpace{}%
\AgdaFunction{,}\AgdaSpace{}%
\AgdaSymbol{(}\AgdaFunction{transport}\AgdaSpace{}%
\AgdaBound{B}\AgdaSpace{}%
\AgdaBound{p}\AgdaSpace{}%
\AgdaOperator{\AgdaFunction{∥}}\AgdaSpace{}%
\AgdaBound{σ}\AgdaSpace{}%
\AgdaOperator{\AgdaFunction{∥}}\AgdaSymbol{)}\AgdaSpace{}%
\AgdaOperator{\AgdaDatatype{≡}}\AgdaSpace{}%
\AgdaOperator{\AgdaFunction{∥}}\AgdaSpace{}%
\AgdaBound{τ}\AgdaSpace{}%
\AgdaOperator{\AgdaFunction{∥}}\<%
\\
% \>[.][@{}l@{}]\<[111I]%
% \>[10]\AgdaComment{--------------------------------------------------------------}\<%
% \\
\>[2]\AgdaSymbol{→}%
\>[10]\AgdaBound{σ}\AgdaSpace{}%
\AgdaOperator{\AgdaDatatype{≡}}\AgdaSpace{}%
\AgdaBound{τ}\<%
\\
%
\\[\AgdaEmptyExtraSkip]%
%
\>[1]\AgdaFunction{to-Σ-≡}\AgdaSpace{}%
\AgdaSymbol{(}\AgdaInductiveConstructor{refl}\AgdaSpace{}%
\AgdaSymbol{\{}\AgdaArgument{x}\AgdaSpace{}%
\AgdaSymbol{=}\AgdaSpace{}%
\AgdaBound{x}\AgdaSymbol{\}}\AgdaSpace{}%
\AgdaOperator{\AgdaInductiveConstructor{,}}\AgdaSpace{}%
\AgdaInductiveConstructor{refl}\AgdaSpace{}%
\AgdaSymbol{\{}\AgdaArgument{x}\AgdaSpace{}%
\AgdaSymbol{=}\AgdaSpace{}%
\AgdaBound{a}\AgdaSymbol{\})}\AgdaSpace{}%
\AgdaSymbol{=}\AgdaSpace{}%
\AgdaInductiveConstructor{refl}\AgdaSpace{}%
\AgdaSymbol{\{}\AgdaArgument{x}\AgdaSpace{}%
\AgdaSymbol{=}\AgdaSpace{}%
\AgdaSymbol{(}\AgdaBound{x}\AgdaSpace{}%
\AgdaOperator{\AgdaInductiveConstructor{,}}\AgdaSpace{}%
\AgdaBound{a}\AgdaSymbol{)\}}\<%
\end{code}
\ccpad
We will use \af{is-embedding}, \af{is-set}, and \af{to-Σ-≡} in the next subsection to prove that a monic function into a set is an embedding.

\paragraph*{Injective functions are set embeddings} %\label{injective-functions-are-set-embeddings}
Before moving on to define propositions, we discharge an obligation mentioned but left unfulfilled in the
\href{https://ualib.gitlab.io/Overture.Inverses.html\#embeddings}{embeddings} section of the \ualibhtml{Overture.Inverses} module. Recall, we described and imported the \af{is-embedding} type, and we remarked that an embedding is not simply a monic function. However, if we assume that the codomain is truncated so as to have unique identity proofs, then we can prove that every monic function into that codomain will be an embedding. On the other hand, embeddings are always monic, so we will end up with an equivalence.
% Assume the context contains the following typing judgments: \AgdaSymbol{\{}\AgdaBound{𝓤}\AgdaSpace{}\AgdaBound{𝓦}\AgdaSpace{}%
% \AgdaSymbol{:}\AgdaSpace{}%
% \AgdaPostulate{Universe}\AgdaSymbol{\}\{}\AgdaBound{A}\AgdaSpace{}%
% \AgdaSymbol{:}\AgdaSpace{}%
% \AgdaBound{𝓤}\AgdaSpace{}%
% \AgdaOperator{\AgdaFunction{̇}}\AgdaSymbol{\}\{}\AgdaBound{B}\AgdaSpace{}%
% \AgdaSymbol{:}\AgdaSpace{}%
% \AgdaBound{𝓦}\AgdaSpace{}%
% \AgdaOperator{\AgdaFunction{̇}}\AgdaSymbol{\}}.
\ccpad
\begin{code}%
\>[1]\AgdaFunction{monic-is-embedding|Set}\AgdaSpace{}%
\AgdaSymbol{:}%
\>[143I]\AgdaSymbol{(}\AgdaBound{f}\AgdaSpace{}%
\AgdaSymbol{:}\AgdaSpace{}%
\AgdaBound{A}\AgdaSpace{}%
\AgdaSymbol{→}\AgdaSpace{}%
\AgdaBound{B}\AgdaSymbol{)}\AgdaSpace{}%
\AgdaSymbol{→}\AgdaSpace{}%
\AgdaFunction{is-set}\AgdaSpace{}%
\AgdaBound{B}\AgdaSpace{}%
\AgdaSymbol{→}\AgdaSpace{}%
\AgdaFunction{Monic}\AgdaSpace{}%
\AgdaBound{f}\AgdaSpace{}%
\AgdaSymbol{→}\AgdaSpace{}%
\AgdaFunction{is-embedding}\AgdaSpace{}%
\AgdaBound{f}\<%
\\
%
\\[\AgdaEmptyExtraSkip]%
%
\>[1]\AgdaFunction{monic-is-embedding|Set}\AgdaSpace{}%
\AgdaBound{f}\AgdaSpace{}%
\AgdaBound{Bset}\AgdaSpace{}%
\AgdaBound{fmon}\AgdaSpace{}%
\AgdaBound{b}\AgdaSpace{}%
\AgdaSymbol{(}\AgdaBound{u}\AgdaSpace{}%
\AgdaOperator{\AgdaInductiveConstructor{,}}\AgdaSpace{}%
\AgdaBound{fu≡b}\AgdaSymbol{)}\AgdaSpace{}%
\AgdaSymbol{(}\AgdaBound{v}\AgdaSpace{}%
\AgdaOperator{\AgdaInductiveConstructor{,}}\AgdaSpace{}%
\AgdaBound{fv≡b}\AgdaSymbol{)}\AgdaSpace{}%
\AgdaSymbol{=}\AgdaSpace{}%
\AgdaFunction{γ}\<%
\\
\>[1][@{}l@{\AgdaIndent{0}}]%
\>[2]\AgdaKeyword{where}\<%
\\
%
\>[2]\AgdaFunction{fuv}\AgdaSpace{}%
\AgdaSymbol{:}\AgdaSpace{}%
\AgdaBound{f}\AgdaSpace{}%
\AgdaBound{u}\AgdaSpace{}%
\AgdaOperator{\AgdaDatatype{≡}}\AgdaSpace{}%
\AgdaBound{f}\AgdaSpace{}%
\AgdaBound{v}\<%
\\
%
\>[2]\AgdaFunction{fuv}\AgdaSpace{}%
\AgdaSymbol{=}\AgdaSpace{}%
\AgdaFunction{≡-trans}\AgdaSpace{}%
\AgdaBound{fu≡b}\AgdaSpace{}%
\AgdaSymbol{(}\AgdaBound{fv≡b}\AgdaSpace{}%
\AgdaOperator{\AgdaFunction{⁻¹}}\AgdaSymbol{)}\<%
\\
%
\\[\AgdaEmptyExtraSkip]%
%
\>[2]\AgdaFunction{uv}\AgdaSpace{}%
\AgdaSymbol{:}\AgdaSpace{}%
\AgdaBound{u}\AgdaSpace{}%
\AgdaOperator{\AgdaDatatype{≡}}\AgdaSpace{}%
\AgdaBound{v}\<%
\\
%
\>[2]\AgdaFunction{uv}\AgdaSpace{}%
\AgdaSymbol{=}\AgdaSpace{}%
\AgdaBound{fmon}\AgdaSpace{}%
\AgdaBound{u}\AgdaSpace{}%
\AgdaBound{v}\AgdaSpace{}%
\AgdaFunction{fuv}\<%
\\
%
\\[\AgdaEmptyExtraSkip]%
%
\>[2]\AgdaFunction{arg1}\AgdaSpace{}%
\AgdaSymbol{:}\AgdaSpace{}%
\AgdaFunction{Σ}\AgdaSpace{}%
\AgdaBound{p}\AgdaSpace{}%
\AgdaFunction{꞉}\AgdaSpace{}%
\AgdaSymbol{(}\AgdaBound{u}\AgdaSpace{}%
\AgdaOperator{\AgdaDatatype{≡}}\AgdaSpace{}%
\AgdaBound{v}\AgdaSymbol{)}\AgdaSpace{}%
\AgdaFunction{,}\AgdaSpace{}%
\AgdaSymbol{(}\AgdaFunction{transport}\AgdaSpace{}%
\AgdaSymbol{(λ}\AgdaSpace{}%
\AgdaBound{a}\AgdaSpace{}%
\AgdaSymbol{→}\AgdaSpace{}%
\AgdaBound{f}\AgdaSpace{}%
\AgdaBound{a}\AgdaSpace{}%
\AgdaOperator{\AgdaDatatype{≡}}\AgdaSpace{}%
\AgdaBound{b}\AgdaSymbol{)}\AgdaSpace{}%
\AgdaBound{p}\AgdaSpace{}%
\AgdaBound{fu≡b}\AgdaSymbol{)}\AgdaSpace{}%
\AgdaOperator{\AgdaDatatype{≡}}\AgdaSpace{}%
\AgdaBound{fv≡b}\<%
\\
%
\>[2]\AgdaFunction{arg1}\AgdaSpace{}%
\AgdaSymbol{=}\AgdaSpace{}%
\AgdaFunction{uv}\AgdaSpace{}%
\AgdaOperator{\AgdaInductiveConstructor{,}}\AgdaSpace{}%
\AgdaBound{Bset}\AgdaSpace{}%
\AgdaSymbol{(}\AgdaBound{f}\AgdaSpace{}%
\AgdaBound{v}\AgdaSymbol{)}\AgdaSpace{}%
\AgdaBound{b}\AgdaSpace{}%
\AgdaSymbol{(}\AgdaFunction{transport}\AgdaSpace{}%
\AgdaSymbol{(λ}\AgdaSpace{}%
\AgdaBound{a}\AgdaSpace{}%
\AgdaSymbol{→}\AgdaSpace{}%
\AgdaBound{f}\AgdaSpace{}%
\AgdaBound{a}\AgdaSpace{}%
\AgdaOperator{\AgdaDatatype{≡}}\AgdaSpace{}%
\AgdaBound{b}\AgdaSymbol{)}\AgdaSpace{}%
\AgdaFunction{uv}\AgdaSpace{}%
\AgdaBound{fu≡b}\AgdaSymbol{)}\AgdaSpace{}%
\AgdaBound{fv≡b}\<%
\\
%
\\[\AgdaEmptyExtraSkip]%
%
\>[2]\AgdaFunction{γ}\AgdaSpace{}%
\AgdaSymbol{:}\AgdaSpace{}%
\AgdaBound{u}\AgdaSpace{}%
\AgdaOperator{\AgdaInductiveConstructor{,}}\AgdaSpace{}%
\AgdaBound{fu≡b}\AgdaSpace{}%
\AgdaOperator{\AgdaDatatype{≡}}\AgdaSpace{}%
\AgdaBound{v}\AgdaSpace{}%
\AgdaOperator{\AgdaInductiveConstructor{,}}\AgdaSpace{}%
\AgdaBound{fv≡b}\<%
\\
%
\>[2]\AgdaFunction{γ}\AgdaSpace{}%
\AgdaSymbol{=}\AgdaSpace{}%
\AgdaFunction{to-Σ-≡}\AgdaSpace{}%
\AgdaFunction{arg1}\<%
\end{code}
\ccpad
In stating the previous result, we introduce a new convention to which we will try to adhere. If the antecedent of a theorem includes the assumption that one of the types involved is a set, then we add to the name of the theorem the suffix \af{\textbar{}sets}, which calls to mind the standard notation for the restriction of a function to a subset of its domain.

Embeddings are always monic, so we conclude that when a function's codomain is a set, then that function is an embedding if and only if it is monic.
\ccpad
\begin{code}%
\>[0][@{}l@{\AgdaIndent{1}}]%
\>[1]\AgdaFunction{embedding-iff-monic|Set}\AgdaSpace{}%
\AgdaSymbol{:}%
\>[234I]\AgdaSymbol{(}\AgdaBound{f}\AgdaSpace{}%
\AgdaSymbol{:}\AgdaSpace{}%
\AgdaBound{A}\AgdaSpace{}%
\AgdaSymbol{→}\AgdaSpace{}%
\AgdaBound{B}\AgdaSymbol{)}\AgdaSpace{}%
\AgdaSymbol{→}\AgdaSpace{}%
\AgdaFunction{is-set}\AgdaSpace{}%
\AgdaBound{B}\AgdaSpace{}%
\AgdaSymbol{→}\AgdaSpace{}%
\AgdaFunction{is-embedding}\AgdaSpace{}%
\AgdaBound{f}\AgdaSpace{}%
\AgdaOperator{\AgdaFunction{⇔}}\AgdaSpace{}%
\AgdaFunction{Monic}\AgdaSpace{}%
\AgdaBound{f}\<%
\\
\>[1]\AgdaFunction{embedding-iff-monic|Set}\AgdaSpace{}%
\AgdaBound{f}\AgdaSpace{}%
\AgdaBound{Bset}\AgdaSpace{}%
\AgdaSymbol{=}\AgdaSpace{}%
\AgdaSymbol{(}\AgdaFunction{embedding-is-monic}\AgdaSpace{}%
\AgdaBound{f}\AgdaSymbol{)}\AgdaOperator{\AgdaInductiveConstructor{,}}\AgdaSpace{}%
\AgdaSymbol{(}\AgdaFunction{monic-is-embedding|Set}\AgdaSpace{}%
\AgdaBound{f}\AgdaSpace{}%
\AgdaBound{Bset}\AgdaSymbol{)}\<%
\end{code}


\subsection{Relation extensionality}\label{sec:relation-extensionality}
%: continuous propositions, quotient extensionality
\firstsentence{Relations}{Extensionality}
\input{35Extensionality}

% % -*- TeX-master: "ualib-part1.tex" -*-
%%% Local Variables: 
%%% mode: latex
%%% TeX-engine: 'xetex
%%% End:
\section{Algebra Types}\label{sec:algebra-types}
Finally we are ready to present the \ualibhtml{Algebras} module of the \agdaualib. Here we use type theory and Agda to codify the most basic objects of universal algebra, such as \emph{operations} and \emph{signatures} (\S\ref{sec:oper-sign}), \emph{algebras} (\S\ref{sec:algebras}), \emph{product algebras} (\S\ref{sec:product-algebras}), \emph{congruence relations} and \emph{quotient algebras}~(\S\ref{congruences}).

A popular way to represent algebraic structures in type theory is with record types. The Sigma type provides an equivalent alternative that we happen to prefer and we use it throughout the library, both for consistency and because of its direct connection to the existential quantifier of logic. Recall that inhabitants of the type \ad Σ~\ab x~\af ꞉~\ab X~\af ,~\ab P are pairs (\ab x,~\ab p) such that \abt{x}{X} and \ab p~\as : \ab P~\ab x. In this sense, when such a Sigma type is inhabited we conclude, ``there exists \ab x in \ab X such that \ab P~\ab x holds;'' in symbols, \as ∃~\ab x~\af ∈~\ab X~\af ,~\ab P~\ab x.  % Indeed, an inhabitant of  \ad Σ~\ab x~\af ꞉~\ab X~\af ,~\ab P~\ab x is a pair (\ab x , \ab p) such that \ab x inhabits \ab X and \ab p is a proof of \ab P \ab x.
Moreover, the pair (\ab x,~\ab p) is not merely a proof of the logical sentence \as ∃~\ab x~\af ∈~\ab X~\af ,~\ab P~\ab x; it is also a \emph{witness} of the truth of this sentence. We sometimes say that a proof of an existentially quantified sentence has ``computational content'' if it provides such a witness, or a function that can extract a witness from the proof.

\subsection{Signatures: types for operations \& signatures}\label{sec:oper-sign}
\firstsentence{Algebras}{Signatures}
% -*- TeX-master: "ualib-part1.tex" -*-
%%% Local Variables: 
%%% mode: latex
%%% TeX-engine: 'xetex
%%% End:
% \subsubsection{Operation type}\label{operation-type}
% We begin by defining the type of \defn{operations} as follows.
% \ccpad
% \begin{code}%
% \>[0][@{}l@{\AgdaIndent{0}}]%
% \>[1]\AgdaFunction{Op}\AgdaSpace{}%
% \AgdaSymbol{:}\AgdaSpace{}%
% \AgdaGeneralizable{𝓥}\AgdaSpace{}%
% \AgdaOperator{\AgdaFunction{̇}}\AgdaSpace{}%
% \AgdaSymbol{→}\AgdaSpace{}%
% \AgdaBound{𝓤}\AgdaSpace{}%
% \AgdaOperator{\AgdaFunction{̇}}\AgdaSpace{}%
% \AgdaSymbol{→}\AgdaSpace{}%
% \AgdaBound{𝓤}\AgdaSpace{}%
% \AgdaOperator{\AgdaPrimitive{⊔}}\AgdaSpace{}%
% \AgdaGeneralizable{𝓥}\AgdaSpace{}%
% \AgdaOperator{\AgdaFunction{̇}}\<%
% \\
% %
% \>[1]\AgdaFunction{Op}\AgdaSpace{}%
% \AgdaBound{I}\AgdaSpace{}%
% \AgdaBound{A}\AgdaSpace{}%
% \AgdaSymbol{=}\AgdaSpace{}%
% \AgdaSymbol{(}\AgdaBound{I}\AgdaSpace{}%
% \AgdaSymbol{→}\AgdaSpace{}%
% \AgdaBound{A}\AgdaSymbol{)}\AgdaSpace{}%
% \AgdaSymbol{→}\AgdaSpace{}%
% \AgdaBound{A}\<%
% \end{code}
% \ccpad
% The type \af{Op} encodes the arity of an operation as an arbitrary type \abt{I}{𝓥}, which gives us a very general way to represent an operation as a function type with domain \ab I \as → \ab A (the type of ``tuples'') and codomain \ab{A}. For example, the \ab{I}-\defn{ary projection operations} on \ab{A} can be codified as inhabitants of the type \af{Op} \ab I \af A in the following way.
% \ccpad
% \begin{code}%
% \>[0]\AgdaFunction{π}\AgdaSpace{}%
% \AgdaSymbol{:}\AgdaSpace{}%
% \AgdaSymbol{\{}\AgdaBound{I}\AgdaSpace{}%
% \AgdaSymbol{:}\AgdaSpace{}%
% \AgdaGeneralizable{𝓥}\AgdaSpace{}%
% \AgdaOperator{\AgdaFunction{̇}}%
% \AgdaSymbol{\}}\AgdaSpace{}%
% \AgdaSymbol{\{}\AgdaBound{A}\AgdaSpace{}%
% \AgdaSymbol{:}\AgdaSpace{}%
% \AgdaGeneralizable{𝓤}\AgdaSpace{}%
% \AgdaOperator{\AgdaFunction{̇}}\AgdaSpace{}%
% \AgdaSymbol{\}}\AgdaSpace{}%
% \AgdaSymbol{→}\AgdaSpace{}%
% \AgdaBound{I}\AgdaSpace{}%
% \AgdaSymbol{→}\AgdaSpace{}%
% \AgdaFunction{Op}\AgdaSpace{}%
% \AgdaBound{I}\AgdaSpace{}%
% \AgdaBound{A}\<%
% \\
% \>[0]\AgdaFunction{π}\AgdaSpace{}%
% \AgdaBound{i}\AgdaSpace{}%
% \AgdaBound{x}\AgdaSpace{}%
% \AgdaSymbol{=}\AgdaSpace{}%
% \AgdaBound{x}\AgdaSpace{}%
% \AgdaBound{i}\<%
% \end{code}

% \subsubsection{Signature type}\label{signature-type}

We define the signature of an algebraic structure in Agda like this.
\ccpad
\begin{code}%
\>[0]\AgdaFunction{Signature}\AgdaSpace{}%
\AgdaSymbol{:}\AgdaSpace{}%
\AgdaSymbol{(}\AgdaBound{𝓞}\AgdaSpace{}%
\AgdaBound{𝓥}\AgdaSpace{}%
\AgdaSymbol{:}\AgdaSpace{}%
\AgdaPostulate{Universe}\AgdaSymbol{)}\AgdaSpace{}%
\AgdaSymbol{→}\AgdaSpace{}%
\AgdaSymbol{(}\AgdaBound{𝓞}\AgdaSpace{}%
\AgdaOperator{\AgdaPrimitive{⊔}}\AgdaSpace{}%
\AgdaBound{𝓥}\AgdaSymbol{)}\AgdaSpace{}%
\AgdaOperator{\AgdaPrimitive{⁺}}\AgdaSpace{}%
\AgdaOperator{\AgdaFunction{̇}}\<%
\\
\>[0]\AgdaFunction{Signature}\AgdaSpace{}%
\AgdaBound{𝓞}\AgdaSpace{}%
\AgdaBound{𝓥}\AgdaSpace{}%
\AgdaSymbol{=}\AgdaSpace{}%
\AgdaFunction{Σ}\AgdaSpace{}%
\AgdaBound{F}\AgdaSpace{}%
\AgdaFunction{꞉}\AgdaSpace{}%
\AgdaBound{𝓞}\AgdaSpace{}%
\AgdaOperator{\AgdaFunction{̇}}\AgdaSpace{}%
\AgdaFunction{,}\AgdaSpace{}%
\AgdaSymbol{(}\AgdaBound{F}\AgdaSpace{}%
\AgdaSymbol{→}\AgdaSpace{}%
\AgdaBound{𝓥}\AgdaSpace{}%
\AgdaOperator{\AgdaFunction{̇}}\AgdaSymbol{)}\<%
\end{code}
\ccpad
As mentioned in \S\ref{sec:discrete-relations}, the symbol 𝓞 always denotes the universe of \emph{operation symbol} types, while 𝓥 is always the universe of \emph{arity} types.

In \S\ref{preliminaries} we defined special syntax for the first and second projections---namely, \af ∣\au\af ∣ and \af{∥}\au\af{∥}, respectively. Consequently, if \ab 𝑆 \as : \af{Signature} \ab 𝓞 \ab 𝓥 is a signature, then \af ∣~\ab 𝑆~\af ∣ denotes the set of operation symbols, and \af{∥}~\ab 𝑆~\af{∥} denotes the arity function. If \ab 𝑓 \as : \af ∣~\ab 𝑆~\af ∣ is an operation symbol in the signature \ab 𝑆, then \af{∥}~\ab 𝑆~\af{∥} \ab 𝑓 is the arity of \ab 𝑓.

% For reference, we recall the definition of the Sigma type, \af{Σ}, which is

% \begin{Shaded}
% \begin{Highlighting}[]
% \NormalTok{-Σ }\OtherTok{:} \OtherTok{\{}\NormalTok{𝓤 𝓥 }\OtherTok{:}\NormalTok{ Universe}\OtherTok{\}(}\NormalTok{X }\OtherTok{:}\NormalTok{ 𝓤 ̇}\OtherTok{)(}\NormalTok{Y }\OtherTok{:}\NormalTok{ X }\OtherTok{→}\NormalTok{ 𝓥 ̇}\OtherTok{)} \OtherTok{→}\NormalTok{ 𝓤 ⊔ 𝓥 ̇}
% \NormalTok{-Σ X Y }\OtherTok{=}\NormalTok{ Σ Y}
% \end{Highlighting}
% \end{Shaded}

\paragraph*{Example of a signature} %\label{example}
Here is how we could define the signature for \defn{monoids} as an inhabitant of the type \af{Signature} \ab 𝓞 \ab 𝓥.
\ccpad
\begin{code}%
\>[1]\AgdaKeyword{data}\AgdaSpace{}%
\AgdaDatatype{monoid-op}\AgdaSpace{}%
\AgdaSymbol{:}\AgdaSpace{}%
\AgdaBound{𝓞}\AgdaSpace{}%
\AgdaOperator{\AgdaFunction{̇}}\AgdaSpace{}%
\AgdaKeyword{where}\<%
\\
\>[1][@{}l@{\AgdaIndent{0}}]%
\>[2]\AgdaInductiveConstructor{e}\AgdaSpace{}%
\AgdaSymbol{:}\AgdaSpace{}%
\AgdaDatatype{monoid-op}\<%
\\
%
\>[2]\AgdaInductiveConstructor{·}\AgdaSpace{}%
\AgdaSymbol{:}\AgdaSpace{}%
\AgdaDatatype{monoid-op}\<%
\\
%
\\[\AgdaEmptyExtraSkip]%
%
\>[1]\AgdaFunction{monoid-sig}\AgdaSpace{}%
\AgdaSymbol{:}\AgdaSpace{}%
\AgdaFunction{Signature}\AgdaSpace{}%
\AgdaBound{𝓞}\AgdaSpace{}%
\AgdaPrimitive{𝓤₀}\<%
\\
%
\>[1]\AgdaFunction{monoid-sig}\AgdaSpace{}%
\AgdaSymbol{=}\AgdaSpace{}%
\AgdaDatatype{monoid-op}\AgdaSpace{}%
\AgdaOperator{\AgdaInductiveConstructor{,}}\AgdaSpace{}%
\AgdaSymbol{λ}\AgdaSpace{}%
\AgdaSymbol{\{}\AgdaSpace{}%
\AgdaInductiveConstructor{e}\AgdaSpace{}%
\AgdaSymbol{→}\AgdaSpace{}%
\AgdaFunction{𝟘}\AgdaSymbol{;}\AgdaSpace{}%
\AgdaInductiveConstructor{·}\AgdaSpace{}%
\AgdaSymbol{→}\AgdaSpace{}%
\AgdaFunction{𝟚}\AgdaSpace{}%
\AgdaSymbol{\}}\<%
\end{code}
\ccpad
Thus, the signature for a monoid consists of two operation symbols, \aic{e} and \aic{·}, and a function
\as λ \as{\{} \aic e \as → \af 𝟘 \as ; \aic · \as → \af 𝟚 \as{\}} which maps \aic{e} to the empty type \af 𝟘 (since \aic{e} is the nullary identity) and maps \aic{·} to the two element type \af 𝟚 (since \aic{·} is binary).\footnote{The types \af 𝟘 and \af 𝟚 are defined in the \AgdaModule{MGS-MLTT} module of \typtop.}


\subsection{Algebras: types for algebras, operation interpretation \& compatibility}\label{sec:algebras}
\firstsentence{Algebras}{Algebras}
% -*- TeX-master: "ualib-part1.tex" -*-
%%% Local Variables: 
%%% mode: latex
%%% TeX-engine: 'xetex
%%% End:
% \subsubsection{The Algebra type}\label{algebra-types}
For a fixed signature \ab{𝑆} \as : \af{Signature} \ab 𝓞 \ab 𝓥 and universe \ab 𝓤, we define the type of \defn{algebras in the signature} \ab 𝑆 (or \ab 𝑆-\defn{algebras}) with \defn{domain} (or \defn{carrier} or \defn{universe}) \abt{𝐴}{𝓤} as follows.
\ccpad
\begin{code}%
\>[0]\AgdaFunction{Algebra}\AgdaSpace{}%
\AgdaSymbol{:}\AgdaSpace{}%
\AgdaSymbol{(}\AgdaBound{𝓤}\AgdaSpace{}%
\AgdaSymbol{:}\AgdaSpace{}%
\AgdaPostulate{Universe}\AgdaSymbol{)(}\AgdaBound{𝑆}\AgdaSpace{}%
\AgdaSymbol{:}\AgdaSpace{}%
\AgdaFunction{Signature}\AgdaSpace{}%
\AgdaGeneralizable{𝓞}\AgdaSpace{}%
\AgdaGeneralizable{𝓥}\AgdaSymbol{)}\AgdaSpace{}%
\AgdaSymbol{→}\AgdaSpace{}%
\AgdaGeneralizable{𝓞}\AgdaSpace{}%
\AgdaOperator{\AgdaPrimitive{⊔}}\AgdaSpace{}%
\AgdaGeneralizable{𝓥}\AgdaSpace{}%
\AgdaOperator{\AgdaPrimitive{⊔}}\AgdaSpace{}%
\AgdaBound{𝓤}\AgdaSpace{}%
\AgdaOperator{\AgdaPrimitive{⁺}}\AgdaSpace{}%
\AgdaOperator{\AgdaFunction{̇}}\<%
\\
%
\\[\AgdaEmptyExtraSkip]%
\>[0]\AgdaFunction{Algebra}\AgdaSpace{}%
\AgdaBound{𝓤}\AgdaSpace{}%
\AgdaBound{𝑆}\AgdaSpace{}%
\AgdaSymbol{=}%
\>[29I]\AgdaFunction{Σ}\AgdaSpace{}%
\AgdaBound{A}\AgdaSpace{}%
\AgdaFunction{꞉}\AgdaSpace{}%
\AgdaBound{𝓤}\AgdaSpace{}%
\AgdaOperator{\AgdaFunction{̇}}\AgdaSpace{}%
\AgdaFunction{,}%
\>[50]\AgdaComment{-- the domain}\<%
\\
\>[.][@{}l@{}]\<[29I]%
\>[14]\AgdaFunction{Π}\AgdaSpace{}%
\AgdaBound{f}\AgdaSpace{}%
\AgdaFunction{꞉}\AgdaSpace{}%
\AgdaOperator{\AgdaFunction{∣}}\AgdaSpace{}%
\AgdaBound{𝑆}\AgdaSpace{}%
\AgdaOperator{\AgdaFunction{∣}}\AgdaSpace{}%
\AgdaFunction{,}\AgdaSpace{}%
\AgdaFunction{Op}\AgdaSpace{}%
\AgdaSymbol{(}\AgdaOperator{\AgdaFunction{∥}}\AgdaSpace{}%
\AgdaBound{𝑆}\AgdaSpace{}%
\AgdaOperator{\AgdaFunction{∥}}\AgdaSpace{}%
\AgdaBound{f}\AgdaSymbol{)}\AgdaSpace{}%
\AgdaBound{A}%
\>[50]\AgdaComment{-- the basic operations}\<%
\end{code}
\ccpad
To be more precise, we might call an inhabitant of this type an ``∞-algebra'' because its domain can be an arbitrary type, say, \abt{A}{𝓤} and need not be truncated at some level. (In particular, \ab{A} need not be a \emph{set}; see the discussion of sets in~\S~\ref{sec:trunc-sets-prop}.) We could then proceed to define the type of ``0-algebras'' as algebras whose domains are sets (which may be closer to what most of us have in mind when doing informal universal algebra). However, we have found that the domains of our algebras need to be sets in just a few places in the \ualib, and it seems preferable to work with general (∞-)algebras throughout and then assume \emph{uniqueness of identity proofs} (uip) explicitly and only where needed.  This makes any dependence on uip more transparent (which is also the reason \AgdaPragma{--without-K} appears at the top of every module in the \ualib).

\paragraph*{Algebras as record types} %\label{algebras-as-record-types}

Some people prefer to represent algebraic structures in type theory using records, and for those folks we offer the following (equivalent) definition.
\ccpad
\begin{code}%
\>[0]\AgdaKeyword{record}\AgdaSpace{}%
\AgdaRecord{algebra}\AgdaSpace{}%
\AgdaSymbol{(}\AgdaBound{𝓤}\AgdaSpace{}%
\AgdaSymbol{:}\AgdaSpace{}%
\AgdaPostulate{Universe}\AgdaSymbol{)}\AgdaSpace{}%
\AgdaSymbol{(}\AgdaBound{𝑆}\AgdaSpace{}%
\AgdaSymbol{:}\AgdaSpace{}%
\AgdaFunction{Signature}\AgdaSpace{}%
\AgdaGeneralizable{𝓞}\AgdaSpace{}%
\AgdaGeneralizable{𝓥}\AgdaSymbol{)}\AgdaSpace{}%
\AgdaSymbol{:}\AgdaSpace{}%
\AgdaSymbol{(}\AgdaBound{𝓞}\AgdaSpace{}%
\AgdaOperator{\AgdaPrimitive{⊔}}\AgdaSpace{}%
\AgdaBound{𝓥}\AgdaSpace{}%
\AgdaOperator{\AgdaPrimitive{⊔}}\AgdaSpace{}%
\AgdaBound{𝓤}\AgdaSymbol{)}\AgdaSpace{}%
\AgdaOperator{\AgdaPrimitive{⁺}}\AgdaSpace{}%
\AgdaOperator{\AgdaFunction{̇}}\AgdaSpace{}%
\AgdaKeyword{where}\<%
\\
\>[0][@{}l@{\AgdaIndent{0}}]%
\>[1]\AgdaKeyword{constructor}\AgdaSpace{}%
\AgdaInductiveConstructor{mkalg}\<%
\\
%
\>[1]\AgdaKeyword{field}\<%
\\
\>[1][@{}l@{\AgdaIndent{0}}]%
\>[2]\AgdaField{univ}\AgdaSpace{}%
\AgdaSymbol{:}\AgdaSpace{}%
\AgdaBound{𝓤}\AgdaSpace{}%
\AgdaOperator{\AgdaFunction{̇}}\<%
\\
%
\>[2]\AgdaField{op}\AgdaSpace{}%
\AgdaSymbol{:}\AgdaSpace{}%
\AgdaSymbol{(}\AgdaBound{f}\AgdaSpace{}%
\AgdaSymbol{:}\AgdaSpace{}%
\AgdaOperator{\AgdaFunction{∣}}\AgdaSpace{}%
\AgdaBound{𝑆}\AgdaSpace{}%
\AgdaOperator{\AgdaFunction{∣}}\AgdaSymbol{)}\AgdaSpace{}%
\AgdaSymbol{→}\AgdaSpace{}%
\AgdaSymbol{((}\AgdaOperator{\AgdaFunction{∥}}\AgdaSpace{}%
\AgdaBound{𝑆}\AgdaSpace{}%
\AgdaOperator{\AgdaFunction{∥}}\AgdaSpace{}%
\AgdaBound{f}\AgdaSymbol{)}\AgdaSpace{}%
\AgdaSymbol{→}\AgdaSpace{}%
\AgdaField{univ}\AgdaSymbol{)}\AgdaSpace{}%
\AgdaSymbol{→}\AgdaSpace{}%
\AgdaField{univ}\<%
\end{code}
\ccpad
This representation of algebras as inhabitants of a record type is equivalent to the representation using Sigma types in the sense that there is a bi-implication between the two representations. The proofs are immediate, so we omit them (see the \ualibhtml{Algebras.Algebras} module for details).
% \ccpad
% \begin{code}%
% \>[1]\AgdaFunction{algebra→Algebra}\AgdaSpace{}%
% \AgdaSymbol{:}\AgdaSpace{}%
% \AgdaRecord{algebra}\AgdaSpace{}%
% \AgdaGeneralizable{𝓤}\AgdaSpace{}%
% \AgdaBound{𝑆}\AgdaSpace{}%
% \AgdaSymbol{→}\AgdaSpace{}%
% \AgdaFunction{Algebra}\AgdaSpace{}%
% \AgdaGeneralizable{𝓤}\AgdaSpace{}%
% \AgdaBound{𝑆}\<%
% \\
% %
% \>[1]\AgdaFunction{algebra→Algebra}\AgdaSpace{}%
% \AgdaBound{𝑨}\AgdaSpace{}%
% \AgdaSymbol{=}\AgdaSpace{}%
% \AgdaSymbol{(}\AgdaField{univ}\AgdaSpace{}%
% \AgdaBound{𝑨}\AgdaSpace{}%
% \AgdaOperator{\AgdaInductiveConstructor{,}}\AgdaSpace{}%
% \AgdaField{op}\AgdaSpace{}%
% \AgdaBound{𝑨}\AgdaSymbol{)}\<%
% \\
% %
% \\[\AgdaEmptyExtraSkip]%
% %
% \>[1]\AgdaFunction{Algebra→algebra}\AgdaSpace{}%
% \AgdaSymbol{:}\AgdaSpace{}%
% \AgdaFunction{Algebra}\AgdaSpace{}%
% \AgdaGeneralizable{𝓤}\AgdaSpace{}%
% \AgdaBound{𝑆}\AgdaSpace{}%
% \AgdaSymbol{→}\AgdaSpace{}%
% \AgdaRecord{algebra}\AgdaSpace{}%
% \AgdaGeneralizable{𝓤}\AgdaSpace{}%
% \AgdaBound{𝑆}\<%
% \\
% %
% \>[1]\AgdaFunction{Algebra→algebra}\AgdaSpace{}%
% \AgdaBound{𝑨}\AgdaSpace{}%
% \AgdaSymbol{=}\AgdaSpace{}%
% \AgdaInductiveConstructor{mkalg}\AgdaSpace{}%
% \AgdaOperator{\AgdaFunction{∣}}\AgdaSpace{}%
% \AgdaBound{𝑨}\AgdaSpace{}%
% \AgdaOperator{\AgdaFunction{∣}}\AgdaSpace{}%
% \AgdaOperator{\AgdaFunction{∥}}\AgdaSpace{}%
% \AgdaBound{𝑨}\AgdaSpace{}%
% \AgdaOperator{\AgdaFunction{∥}}\<%
% \end{code}
% \ccpad

In developing the more advanced modules of the \agdaualib, we have used the Sigma type representation of algebras exclusively, though we occasionally use record types to represent other basic objects (congruences, subalgebras, and others).



\paragraph*{Operation interpretation syntax}\label{operation-interpretation-syntax}

We now define a convenient shorthand for the interpretation of an operation symbol. This looks more similar to the standard notation one finds in the literature as compared to the double bar notation we started with, so we will use this new notation almost exclusively in the remaining modules of the \ualib.
\ccpad
\begin{code}%
\>[1]\AgdaOperator{\AgdaFunction{\AgdaUnderscore{}̂\AgdaUnderscore{}}}\AgdaSpace{}%
\AgdaSymbol{:}\AgdaSpace{}%
\AgdaSymbol{(}\AgdaBound{𝑓}\AgdaSpace{}%
\AgdaSymbol{:}\AgdaSpace{}%
\AgdaOperator{\AgdaFunction{∣}}\AgdaSpace{}%
\AgdaBound{𝑆}\AgdaSpace{}%
\AgdaOperator{\AgdaFunction{∣}}\AgdaSymbol{)(}\AgdaBound{𝑨}\AgdaSpace{}%
\AgdaSymbol{:}\AgdaSpace{}%
\AgdaFunction{Algebra}\AgdaSpace{}%
\AgdaBound{𝓤}\AgdaSpace{}%
\AgdaBound{𝑆}\AgdaSymbol{)}\AgdaSpace{}%
\AgdaSymbol{→}\AgdaSpace{}%
\AgdaSymbol{(}\AgdaOperator{\AgdaFunction{∥}}\AgdaSpace{}%
\AgdaBound{𝑆}\AgdaSpace{}%
\AgdaOperator{\AgdaFunction{∥}}\AgdaSpace{}%
\AgdaBound{𝑓}%
\>[48]\AgdaSymbol{→}%
\>[51]\AgdaOperator{\AgdaFunction{∣}}\AgdaSpace{}%
\AgdaBound{𝑨}\AgdaSpace{}%
\AgdaOperator{\AgdaFunction{∣}}\AgdaSymbol{)}\AgdaSpace{}%
\AgdaSymbol{→}\AgdaSpace{}%
\AgdaOperator{\AgdaFunction{∣}}\AgdaSpace{}%
\AgdaBound{𝑨}\AgdaSpace{}%
\AgdaOperator{\AgdaFunction{∣}}\<%
\\
%
\\[\AgdaEmptyExtraSkip]%
%
\>[1]\AgdaBound{𝑓}\AgdaSpace{}%
\AgdaOperator{\AgdaFunction{̂}}\AgdaSpace{}%
\AgdaBound{𝑨}\AgdaSpace{}%
\AgdaSymbol{=}\AgdaSpace{}%
\AgdaSymbol{λ}\AgdaSpace{}%
\AgdaBound{𝑎}\AgdaSpace{}%
\AgdaSymbol{→}\AgdaSpace{}%
\AgdaSymbol{(}\AgdaOperator{\AgdaFunction{∥}}\AgdaSpace{}%
\AgdaBound{𝑨}\AgdaSpace{}%
\AgdaOperator{\AgdaFunction{∥}}\AgdaSpace{}%
\AgdaBound{𝑓}\AgdaSymbol{)}\AgdaSpace{}%
\AgdaBound{𝑎}\<%
\end{code}
\ccpad
Thus, if \ab{𝑓} \as : \af ∣ \ab 𝑆 \af ∣ is an operation symbol in the signature \ab{𝑆} and if \ab{𝑎} \as : \af ∥ \ab 𝑆 \af ∥ \ab 𝑓 \as → \af ∣ \ab 𝑨 \af ∣ is a tuple of the same arity, then (\ab{𝑓} \af ̂ \ab 𝑨) \ab 𝑎 denotes the operation \ab{𝑓} interpreted in \ab{𝑨} and evaluated at \ab{𝑎}.

% \paragraph*{Arbitrarily many variable symbols}\label{arbitrarily-many-variable-symbols}

% We sometimes want to assume that we have at our disposal an arbitrary collection \ab X of variable symbols such that, for every algebra \ab 𝑨, no matter the type of its domain, we have a surjective map \ab{h} \as : \ab X \as → \af ∣ \ab 𝑨 \af ∣ from variables onto the domain of \ab 𝑨. We may use the following definition to express this assumption when we need it.
% \ccpad
% \begin{code}%
% \>[0][@{}l@{\AgdaIndent{1}}]%
% \>[1]\AgdaOperator{\AgdaFunction{\AgdaUnderscore{}↠\AgdaUnderscore{}}}\AgdaSpace{}%
% \AgdaSymbol{:}\AgdaSpace{}%
% \AgdaSymbol{\{}\AgdaBound{𝓧}\AgdaSpace{}%
% \AgdaSymbol{:}\AgdaSpace{}%
% \AgdaPostulate{Universe}\AgdaSymbol{\}}\AgdaSpace{}%
% \AgdaSymbol{→}\AgdaSpace{}%
% \AgdaBound{𝓧}\AgdaSpace{}%
% \AgdaOperator{\AgdaFunction{̇}}\AgdaSpace{}%
% \AgdaSymbol{→}\AgdaSpace{}%
% \AgdaFunction{Algebra}\AgdaSpace{}%
% \AgdaGeneralizable{𝓤}\AgdaSpace{}%
% \AgdaBound{𝑆}\AgdaSpace{}%
% \AgdaSymbol{→}\AgdaSpace{}%
% \AgdaBound{𝓧}\AgdaSpace{}%
% \AgdaOperator{\AgdaPrimitive{⊔}}\AgdaSpace{}%
% \AgdaGeneralizable{𝓤}\AgdaSpace{}%
% \AgdaOperator{\AgdaFunction{̇}}\<%
% \\
% %
% \>[1]\AgdaBound{X}\AgdaSpace{}%
% \AgdaOperator{\AgdaFunction{↠}}\AgdaSpace{}%
% \AgdaBound{𝑨}\AgdaSpace{}%
% \AgdaSymbol{=}\AgdaSpace{}%
% \AgdaFunction{Σ}\AgdaSpace{}%
% \AgdaBound{h}\AgdaSpace{}%
% \AgdaFunction{꞉}\AgdaSpace{}%
% \AgdaSymbol{(}\AgdaBound{X}\AgdaSpace{}%
% \AgdaSymbol{→}\AgdaSpace{}%
% \AgdaOperator{\AgdaFunction{∣}}\AgdaSpace{}%
% \AgdaBound{𝑨}\AgdaSpace{}%
% \AgdaOperator{\AgdaFunction{∣}}\AgdaSymbol{)}\AgdaSpace{}%
% \AgdaFunction{,}\AgdaSpace{}%
% \AgdaFunction{Epic}\AgdaSpace{}%
% \AgdaBound{h}\<%
% \end{code}
% \ccpad
% Now we can assert, in a specific module, the existence of the surjective map described above by including the following line in that module's declaration, like so.
% \ccpad
% \begin{code}%
% \>[0]\AgdaKeyword{module}\AgdaSpace{}%
% \AgdaModule{\AgdaUnderscore{}}%
% \>[187I]\AgdaSymbol{\{}\AgdaBound{𝓧}\AgdaSpace{}%
% \AgdaSymbol{:}\AgdaSpace{}%
% \AgdaPostulate{Universe}\AgdaSymbol{\}\{}\AgdaBound{X}\AgdaSpace{}%
% \AgdaSymbol{:}\AgdaSpace{}%
% \AgdaBound{𝓧}\AgdaSpace{}%
% \AgdaOperator{\AgdaFunction{̇}}\AgdaSymbol{\}\{}\AgdaBound{𝑆}\AgdaSpace{}%
% \AgdaSymbol{:}\AgdaSpace{}%
% \AgdaFunction{Signature}\AgdaSpace{}%
% \AgdaGeneralizable{𝓞}\AgdaSpace{}%
% \AgdaGeneralizable{𝓥}\AgdaSymbol{\}}\<%
% \\
% \>[.][@{}l@{}]\<[187I]%
% \>[9]\AgdaSymbol{\{}\AgdaBound{𝕏}\AgdaSpace{}%
% \AgdaSymbol{:}\AgdaSpace{}%
% \AgdaSymbol{(}\AgdaBound{𝑨}\AgdaSpace{}%
% \AgdaSymbol{:}\AgdaSpace{}%
% \AgdaFunction{Algebra}\AgdaSpace{}%
% \AgdaGeneralizable{𝓤}\AgdaSpace{}%
% \AgdaBound{𝑆}\AgdaSymbol{)}\AgdaSpace{}%
% \AgdaSymbol{→}\AgdaSpace{}%
% \AgdaBound{X}\AgdaSpace{}%
% \AgdaOperator{\AgdaFunction{↠}}\AgdaSpace{}%
% \AgdaBound{𝑨}\AgdaSymbol{\}}\AgdaSpace{}%
% \AgdaKeyword{where}\<%
% \end{code}
% \ccpad
% Then \af{fst}(\af 𝕏 \ab 𝑨) will denote the surjective map from \ab X onto ∣ \ab 𝑨 \af ∣, and \af{snd}(\af 𝕏 \ab 𝑨) will be a proof that the map is surjective.








\paragraph*{Lifts of algebras}\label{lifts-of-algebras}

Recall, in \S\ref{sec:lifts-altern-univ} we described a common difficulty one encounters when working with a noncumulative universe hierarchy. We made a promise to provide some domain-specific level lifting and lowering methods. Here we fulfill this promise by supplying a couple of bespoke tools designed specifically for our operation and algebra types.
\ccpad
\begin{code}%
% \>[0]\AgdaKeyword{module}\AgdaSpace{}%
% \AgdaModule{\AgdaUnderscore{}}\AgdaSpace{}%
% \AgdaSymbol{\{}\AgdaBound{I}\AgdaSpace{}%
% \AgdaSymbol{:}\AgdaSpace{}%
% \AgdaGeneralizable{𝓥}\AgdaSpace{}%
% \AgdaOperator{\AgdaFunction{̇}}\AgdaSymbol{\}\{}\AgdaBound{A}\AgdaSpace{}%
% \AgdaSymbol{:}\AgdaSpace{}%
% \AgdaGeneralizable{𝓤}\AgdaSpace{}%
% \AgdaOperator{\AgdaFunction{̇}}\AgdaSymbol{\}}\AgdaSpace{}%
% \AgdaKeyword{where}\<%
% \\
% %
% \\[\AgdaEmptyExtraSkip]%
\>[0][@{}l@{\AgdaIndent{0}}]%
% \>[1]\AgdaKeyword{open}\AgdaSpace{}%
% \AgdaModule{Lift}\<%
% \\
% %
% \\[\AgdaEmptyExtraSkip]%
% %
\>[1]\AgdaFunction{Lift-op}\AgdaSpace{}%
\AgdaSymbol{:}\AgdaSpace{}%
\AgdaSymbol{((}\AgdaBound{I}\AgdaSpace{}%
\AgdaSymbol{→}\AgdaSpace{}%
\AgdaBound{A}\AgdaSymbol{)}\AgdaSpace{}%
\AgdaSymbol{→}\AgdaSpace{}%
\AgdaBound{A}\AgdaSymbol{)}\AgdaSpace{}%
\AgdaSymbol{→}\AgdaSpace{}%
\AgdaSymbol{(}\AgdaBound{𝓦}\AgdaSpace{}%
\AgdaSymbol{:}\AgdaSpace{}%
\AgdaPostulate{Universe}\AgdaSymbol{)}\AgdaSpace{}%
\AgdaSymbol{→}\AgdaSpace{}%
\AgdaSymbol{((}\AgdaBound{I}\AgdaSpace{}%
\AgdaSymbol{→}\AgdaSpace{}%
\AgdaRecord{Lift}\AgdaSymbol{\{}\AgdaBound{𝓦}\AgdaSymbol{\}}\AgdaSpace{}%
\AgdaBound{A}\AgdaSymbol{)}\AgdaSpace{}%
\AgdaSymbol{→}\AgdaSpace{}%
\AgdaRecord{Lift}\AgdaSpace{}%
\AgdaSymbol{\{}\AgdaBound{𝓦}\AgdaSymbol{\}}\AgdaSpace{}%
\AgdaBound{A}\AgdaSymbol{)}\<%
\\
%
\>[1]\AgdaFunction{Lift-op}\AgdaSpace{}%
\AgdaBound{f}\AgdaSpace{}%
\AgdaBound{𝓦}\AgdaSpace{}%
\AgdaSymbol{=}\AgdaSpace{}%
\AgdaSymbol{λ}\AgdaSpace{}%
\AgdaBound{x}\AgdaSpace{}%
\AgdaSymbol{→}\AgdaSpace{}%
\AgdaInductiveConstructor{lift}\AgdaSpace{}%
\AgdaSymbol{(}\AgdaBound{f}\AgdaSpace{}%
\AgdaSymbol{(λ}\AgdaSpace{}%
\AgdaBound{i}\AgdaSpace{}%
\AgdaSymbol{→}\AgdaSpace{}%
\AgdaField{lower}\AgdaSpace{}%
\AgdaSymbol{(}\AgdaBound{x}\AgdaSpace{}%
\AgdaBound{i}\AgdaSymbol{)))}\<%
\\
%
\\[\AgdaEmptyExtraSkip]%
% \>[0]\AgdaKeyword{module}\AgdaSpace{}%
% \AgdaModule{\AgdaUnderscore{}}\AgdaSpace{}%
% \AgdaSymbol{\{}\AgdaBound{𝑆}\AgdaSpace{}%
% \AgdaSymbol{:}\AgdaSpace{}%
% \AgdaFunction{Signature}\AgdaSpace{}%
% \AgdaGeneralizable{𝓞}\AgdaSpace{}%
% \AgdaGeneralizable{𝓥}\AgdaSymbol{\}}%
% \>[30]\AgdaKeyword{where}\<%
% \\
% %
% \\[\AgdaEmptyExtraSkip]%
\>[0][@{}l@{\AgdaIndent{0}}]%
% \>[1]\AgdaKeyword{open}\AgdaSpace{}%
% \AgdaModule{algebra}\<%
% \\
% %
% \\[\AgdaEmptyExtraSkip]%
% %
\>[1]\AgdaFunction{Lift-alg}\AgdaSpace{}%
\AgdaSymbol{:}\AgdaSpace{}%
\AgdaFunction{Algebra}\AgdaSpace{}%
\AgdaGeneralizable{𝓤}\AgdaSpace{}%
\AgdaBound{𝑆}\AgdaSpace{}%
\AgdaSymbol{→}\AgdaSpace{}%
\AgdaSymbol{(}\AgdaBound{𝓦}\AgdaSpace{}%
\AgdaSymbol{:}\AgdaSpace{}%
\AgdaPostulate{Universe}\AgdaSymbol{)}\AgdaSpace{}%
\AgdaSymbol{→}\AgdaSpace{}%
\AgdaFunction{Algebra}\AgdaSpace{}%
\AgdaSymbol{(}\AgdaGeneralizable{𝓤}\AgdaSpace{}%
\AgdaOperator{\AgdaPrimitive{⊔}}\AgdaSpace{}%
\AgdaBound{𝓦}\AgdaSymbol{)}\AgdaSpace{}%
\AgdaBound{𝑆}\<%
\\
%
\>[1]\AgdaFunction{Lift-alg}\AgdaSpace{}%
\AgdaBound{𝑨}\AgdaSpace{}%
\AgdaBound{𝓦}\AgdaSpace{}%
\AgdaSymbol{=}\AgdaSpace{}%
\AgdaRecord{Lift}\AgdaSpace{}%
\AgdaOperator{\AgdaFunction{∣}}\AgdaSpace{}%
\AgdaBound{𝑨}\AgdaSpace{}%
\AgdaOperator{\AgdaFunction{∣}}\AgdaSpace{}%
\AgdaOperator{\AgdaInductiveConstructor{,}}\AgdaSpace{}%
\AgdaSymbol{(λ}\AgdaSpace{}%
\AgdaSymbol{(}\AgdaBound{𝑓}\AgdaSpace{}%
\AgdaSymbol{:}\AgdaSpace{}%
\AgdaOperator{\AgdaFunction{∣}}\AgdaSpace{}%
\AgdaBound{𝑆}\AgdaSpace{}%
\AgdaOperator{\AgdaFunction{∣}}\AgdaSymbol{)}\AgdaSpace{}%
\AgdaSymbol{→}\AgdaSpace{}%
\AgdaFunction{Lift-op}\AgdaSpace{}%
\AgdaSymbol{(}\AgdaBound{𝑓}\AgdaSpace{}%
\AgdaOperator{\AgdaFunction{̂}}\AgdaSpace{}%
\AgdaBound{𝑨}\AgdaSymbol{)}\AgdaSpace{}%
\AgdaBound{𝓦}\AgdaSymbol{)}\<%
\\
%
\\[\AgdaEmptyExtraSkip]%
%
\>[1]\AgdaFunction{Lift-alg-record-type}\AgdaSpace{}%
\AgdaSymbol{:}\AgdaSpace{}%
\AgdaRecord{algebra}\AgdaSpace{}%
\AgdaGeneralizable{𝓤}\AgdaSpace{}%
\AgdaBound{𝑆}\AgdaSpace{}%
\AgdaSymbol{→}\AgdaSpace{}%
\AgdaSymbol{(}\AgdaBound{𝓦}\AgdaSpace{}%
\AgdaSymbol{:}\AgdaSpace{}%
\AgdaPostulate{Universe}\AgdaSymbol{)}\AgdaSpace{}%
\AgdaSymbol{→}\AgdaSpace{}%
\AgdaRecord{algebra}\AgdaSpace{}%
\AgdaSymbol{(}\AgdaGeneralizable{𝓤}\AgdaSpace{}%
\AgdaOperator{\AgdaPrimitive{⊔}}\AgdaSpace{}%
\AgdaBound{𝓦}\AgdaSymbol{)}\AgdaSpace{}%
\AgdaBound{𝑆}\<%
\\
%
\>[1]\AgdaFunction{Lift-alg-record-type}\AgdaSpace{}%
\AgdaBound{𝑨}\AgdaSpace{}%
\AgdaBound{𝓦}\AgdaSpace{}%
\AgdaSymbol{=}\AgdaSpace{}%
\AgdaInductiveConstructor{mkalg}\AgdaSpace{}%
\AgdaSymbol{(}\AgdaRecord{Lift}\AgdaSpace{}%
\AgdaSymbol{(}\AgdaField{univ}\AgdaSpace{}%
\AgdaBound{𝑨}\AgdaSymbol{))}\AgdaSpace{}%
\AgdaSymbol{(λ}\AgdaSpace{}%
\AgdaSymbol{(}\AgdaBound{f}\AgdaSpace{}%
\AgdaSymbol{:}\AgdaSpace{}%
\AgdaOperator{\AgdaFunction{∣}}\AgdaSpace{}%
\AgdaBound{𝑆}\AgdaSpace{}%
\AgdaOperator{\AgdaFunction{∣}}\AgdaSymbol{)}\AgdaSpace{}%
\AgdaSymbol{→}\AgdaSpace{}%
\AgdaFunction{Lift-op}\AgdaSpace{}%
\AgdaSymbol{((}\AgdaField{op}\AgdaSpace{}%
\AgdaBound{𝑨}\AgdaSymbol{)}\AgdaSpace{}%
\AgdaBound{f}\AgdaSymbol{)}\AgdaSpace{}%
\AgdaBound{𝓦}\AgdaSymbol{)}\<%
\end{code}
\ccpad
What makes the \AgdaFunction{Lift-alg} type so useful for resolving type level errors is the nice properties it possesses. Indeed, in the \ualib we prove that \af{Lift-alg} preserves term identities and is a \emph{homomorphism}, an \emph{algebraic invariant}, and a \emph{sublagebraic invariant}.\footnote{%
See \href{https://ualib.gitlab.io/Varieties.EquationalLogic.html\#lift-invariance}{EquationalLogic.html}, \href{https://ualib.gitlab.io/Homomorphisms.Basic.html\#exmples-of-homomorphisms}{Homomorphisms.Basic.html}, \href{https://ualib.gitlab.io/Homomorphisms.Isomorphisms.html\#lift-is-an-algebraic-invariant}{Isomorphisms.html}, and \href{https://ualib.gitlab.io/Subalgebras.Subalgebras.html\#lifts-of-subalgebras}{Subalgebras.html}, resp.}
% Our experience has shown \af{lift-alg} to be a perfectly adequate tool for resolving any universe level unification errors that arise when working with the \af{Algebra} type in Agda. We will see some examples of its effectiveness later.
% 's noncumulative hierarchy of type universes.






\paragraph*{Compatibility}\label{compatibility-of-binary-relations}

We now define the function \af{compatible} so that, if \ab{𝑨} is an algebra and \ab{R} a binary relation, then \af{compatible} \ab 𝑨 \ab R will denote the assertion that \ab{R} is \emph{compatible} with all basic operations of \ab{𝑨}; that is, for every operation symbol \ab{𝑓} \as : \af ∣ \ab 𝑆 \af ∣ and all pairs \ab u \ab v~\as :~\af{∥} \ab 𝑆 \af{∥} \ab 𝑓 \as → \af ∣ \ab 𝑨 \af ∣ of tuples from the domain of \ab 𝑨, the following implication holds:\\[-8pt]

(\af Π(\ab i \af ꞉ \ab I)~\af ,~\ab{R} (\ab u \ab i) (\ab v \ab i)) \hskip2mm \as →  \hskip2mm \ab{R} ((\ab 𝑓 \af ̂ \ab 𝑨) \ab u) ((\ab 𝑓 \af ̂ \ab 𝑨) \ab v).\\[4pt]
% ; in symbols,\\[-8pt]
% (\af Π \ab i \af ꞉ \ab I \af , \ab R (\ab u \ab i) (\ab u \ab i)) \hskip2mm \as →  \hskip2mm \ab R ((\ab 𝑓 \af ̂ \ab 𝑨) \ab u) ((\ab 𝑓 \af ̂ \ab 𝑨) \ab v).\\[4pt]
Using the relation |: (defined in \S\ref{sec:discrete-relations}) this implication is expressed compactly as (\ab 𝑓~\af ̂~\ab 𝑨)~\af{|:}~\ab R, yielding a compact representation of compatibility of algebraic operations and binary relations.
\ccpad
\begin{code}%
\>[0][@{}l@{\AgdaIndent{1}}]%
\>[1]\AgdaFunction{compatible}\AgdaSpace{}%
\AgdaSymbol{:}\AgdaSpace{}%
\AgdaSymbol{(}\AgdaBound{𝑨}\AgdaSpace{}%
\AgdaSymbol{:}\AgdaSpace{}%
\AgdaFunction{Algebra}\AgdaSpace{}%
\AgdaGeneralizable{𝓤}\AgdaSpace{}%
\AgdaBound{𝑆}\AgdaSymbol{)}\AgdaSpace{}%
\AgdaSymbol{→}\AgdaSpace{}%
\AgdaFunction{Rel}\AgdaSpace{}%
\AgdaOperator{\AgdaFunction{∣}}\AgdaSpace{}%
\AgdaBound{𝑨}\AgdaSpace{}%
\AgdaOperator{\AgdaFunction{∣}}\AgdaSpace{}%
\AgdaGeneralizable{𝓦}\AgdaSpace{}%
\AgdaSymbol{→}\AgdaSpace{}%
\AgdaBound{𝓞}\AgdaSpace{}%
\AgdaOperator{\AgdaPrimitive{⊔}}\AgdaSpace{}%
\AgdaGeneralizable{𝓤}\AgdaSpace{}%
\AgdaOperator{\AgdaPrimitive{⊔}}\AgdaSpace{}%
\AgdaBound{𝓥}\AgdaSpace{}%
\AgdaOperator{\AgdaPrimitive{⊔}}\AgdaSpace{}%
\AgdaGeneralizable{𝓦}\AgdaSpace{}%
\AgdaOperator{\AgdaFunction{̇}}\<%
\\
%
\>[1]\AgdaFunction{compatible}%
\>[13]\AgdaBound{𝑨}\AgdaSpace{}%
\AgdaBound{R}\AgdaSpace{}%
\AgdaSymbol{=}\AgdaSpace{}%
\AgdaSymbol{∀}\AgdaSpace{}%
\AgdaBound{𝑓}\AgdaSpace{}%
\AgdaSymbol{→}\AgdaSpace{}%
\AgdaSymbol{(}\AgdaBound{𝑓}\AgdaSpace{}%
\AgdaOperator{\AgdaFunction{̂}}\AgdaSpace{}%
\AgdaBound{𝑨}\AgdaSymbol{)}\AgdaSpace{}%
\AgdaFunction{|:}\AgdaSpace{}%
\AgdaBound{R}\<%
\end{code}
\scpad
% \paragraph*{Compatibility of continuous relations★}

In the \ualibhtml{Relations.Continuous} module we defined a function called \af{cont-compatible-op} to represent the assertion that a given continuous relation is compatible with a given operation. With that, it is easy to define a function, which we call \af{cont-compatible}, representing compatibility of a continuous relation with all operations of an algebra. Similarly, we define the analogous \af{dep-compatible} function for the (even more general)
type of dependent relations.
\ccpad
\begin{code}%
% \>[0]\AgdaKeyword{module}\AgdaSpace{}%
% \AgdaModule{continuous-compatibility}\AgdaSpace{}%
% \AgdaSymbol{\{}\AgdaBound{𝓤}\AgdaSpace{}%
% \AgdaBound{𝓦}\AgdaSpace{}%
% \AgdaSymbol{:}\AgdaSpace{}%
% \AgdaPostulate{Universe}\AgdaSymbol{\}\{}\AgdaBound{𝑆}\AgdaSpace{}%
% \AgdaSymbol{:}\AgdaSpace{}%
% \AgdaFunction{Signature}\AgdaSpace{}%
% \AgdaGeneralizable{𝓞}\AgdaSpace{}%
% \AgdaGeneralizable{𝓥}\AgdaSymbol{\}}\AgdaSpace{}%
% \AgdaKeyword{where}\<%
% \\
% \>[0][@{}l@{\AgdaIndent{0}}]%
% \>[1]\AgdaKeyword{open}\AgdaSpace{}%
% \AgdaKeyword{import}\AgdaSpace{}%
% \AgdaModule{Relations.Continuous}\AgdaSpace{}%
% \AgdaKeyword{using}\AgdaSpace{}%
% \AgdaSymbol{(}\AgdaFunction{ContRel}\AgdaSymbol{;}\AgdaSpace{}%
% \AgdaFunction{DepRel}\AgdaSymbol{;}\AgdaSpace{}%
% \AgdaFunction{cont-compatible-op}\AgdaSymbol{;}\AgdaSpace{}%
% \AgdaFunction{dep-compatible-op}\AgdaSymbol{)}\<%
% \\
% %
% \\[\AgdaEmptyExtraSkip]%
% %
\>[1]\AgdaFunction{cont-compatible}\AgdaSpace{}%
\AgdaSymbol{:}\AgdaSpace{}%
\AgdaSymbol{\{}\AgdaBound{I}\AgdaSpace{}%
\AgdaSymbol{:}\AgdaSpace{}%
\AgdaBound{𝓥}\AgdaSpace{}%
\AgdaOperator{\AgdaFunction{̇}}\AgdaSymbol{\}(}\AgdaBound{𝑨}\AgdaSpace{}%
\AgdaSymbol{:}\AgdaSpace{}%
\AgdaFunction{Algebra}\AgdaSpace{}%
\AgdaBound{𝓤}\AgdaSpace{}%
\AgdaBound{𝑆}\AgdaSymbol{)}\AgdaSpace{}%
\AgdaSymbol{→}\AgdaSpace{}%
\AgdaFunction{ContRel}\AgdaSpace{}%
\AgdaBound{I}\AgdaSpace{}%
\AgdaOperator{\AgdaFunction{∣}}\AgdaSpace{}%
\AgdaBound{𝑨}\AgdaSpace{}%
\AgdaOperator{\AgdaFunction{∣}}\AgdaSpace{}%
\AgdaBound{𝓦}\AgdaSpace{}%
\AgdaSymbol{→}\AgdaSpace{}%
\AgdaBound{𝓞}\AgdaSpace{}%
\AgdaOperator{\AgdaPrimitive{⊔}}\AgdaSpace{}%
\AgdaBound{𝓤}\AgdaSpace{}%
\AgdaOperator{\AgdaPrimitive{⊔}}\AgdaSpace{}%
\AgdaBound{𝓥}\AgdaSpace{}%
\AgdaOperator{\AgdaPrimitive{⊔}}\AgdaSpace{}%
\AgdaBound{𝓦}\AgdaSpace{}%
\AgdaOperator{\AgdaFunction{̇}}\<%
\\
%
\>[1]\AgdaFunction{cont-compatible}\AgdaSpace{}%
\AgdaBound{𝑨}\AgdaSpace{}%
\AgdaBound{R}\AgdaSpace{}%
\AgdaSymbol{=}\AgdaSpace{}%
\AgdaFunction{Π}\AgdaSpace{}%
\AgdaBound{𝑓}\AgdaSpace{}%
\AgdaFunction{꞉}\AgdaSpace{}%
\AgdaOperator{\AgdaFunction{∣}}\AgdaSpace{}%
\AgdaBound{𝑆}\AgdaSpace{}%
\AgdaOperator{\AgdaFunction{∣}}\AgdaSpace{}%
\AgdaFunction{,}\AgdaSpace{}%
\AgdaFunction{cont-compatible-op}\AgdaSpace{}%
\AgdaSymbol{(}\AgdaBound{𝑓}\AgdaSpace{}%
\AgdaOperator{\AgdaFunction{̂}}\AgdaSpace{}%
\AgdaBound{𝑨}\AgdaSymbol{)}\AgdaSpace{}%
\AgdaBound{R}\<%
\\
%
\\[\AgdaEmptyExtraSkip]%
%
\>[1]\AgdaFunction{dep-compatible}\AgdaSpace{}%
\AgdaSymbol{:}\AgdaSpace{}%
\AgdaSymbol{\{}\AgdaBound{I}\AgdaSpace{}%
\AgdaSymbol{:}\AgdaSpace{}%
\AgdaBound{𝓥}\AgdaSpace{}%
\AgdaOperator{\AgdaFunction{̇}}\AgdaSymbol{\}(}\AgdaBound{𝒜}\AgdaSpace{}%
\AgdaSymbol{:}\AgdaSpace{}%
\AgdaBound{I}\AgdaSpace{}%
\AgdaSymbol{→}\AgdaSpace{}%
\AgdaFunction{Algebra}\AgdaSpace{}%
\AgdaBound{𝓤}\AgdaSpace{}%
\AgdaBound{𝑆}\AgdaSymbol{)}\AgdaSpace{}%
\AgdaSymbol{→}\AgdaSpace{}%
\AgdaFunction{DepRel}\AgdaSpace{}%
\AgdaBound{I}\AgdaSpace{}%
\AgdaSymbol{(λ}\AgdaSpace{}%
\AgdaBound{i}\AgdaSpace{}%
\AgdaSymbol{→}\AgdaSpace{}%
\AgdaOperator{\AgdaFunction{∣}}\AgdaSpace{}%
\AgdaBound{𝒜}%
\>[72]\AgdaBound{i}\AgdaSpace{}%
\AgdaOperator{\AgdaFunction{∣}}\AgdaSymbol{)}\AgdaSpace{}%
\AgdaBound{𝓦}\AgdaSpace{}%
\AgdaSymbol{→}\AgdaSpace{}%
\AgdaBound{𝓞}\AgdaSpace{}%
\AgdaOperator{\AgdaPrimitive{⊔}}\AgdaSpace{}%
\AgdaBound{𝓤}\AgdaSpace{}%
\AgdaOperator{\AgdaPrimitive{⊔}}\AgdaSpace{}%
\AgdaBound{𝓥}\AgdaSpace{}%
\AgdaOperator{\AgdaPrimitive{⊔}}\AgdaSpace{}%
\AgdaBound{𝓦}\AgdaSpace{}%
\AgdaOperator{\AgdaFunction{̇}}\<%
\\
%
\>[1]\AgdaFunction{dep-compatible}\AgdaSpace{}%
\AgdaBound{𝒜}\AgdaSpace{}%
\AgdaBound{R}\AgdaSpace{}%
\AgdaSymbol{=}\AgdaSpace{}%
\AgdaFunction{Π}\AgdaSpace{}%
\AgdaBound{𝑓}\AgdaSpace{}%
\AgdaFunction{꞉}\AgdaSpace{}%
\AgdaOperator{\AgdaFunction{∣}}\AgdaSpace{}%
\AgdaBound{𝑆}\AgdaSpace{}%
\AgdaOperator{\AgdaFunction{∣}}\AgdaSpace{}%
\AgdaFunction{,}\AgdaSpace{}%
\AgdaFunction{dep-compatible-op}\AgdaSpace{}%
\AgdaSymbol{(λ}\AgdaSpace{}%
\AgdaBound{i}\AgdaSpace{}%
\AgdaSymbol{→}\AgdaSpace{}%
\AgdaBound{𝑓}\AgdaSpace{}%
\AgdaOperator{\AgdaFunction{̂}}\AgdaSpace{}%
\AgdaSymbol{(}\AgdaBound{𝒜}\AgdaSpace{}%
\AgdaBound{i}\AgdaSymbol{))}\AgdaSpace{}%
\AgdaBound{R}\<%
\end{code}


% \paragraph*{Compatibility of continuous relations*\protect\cref{starred}}\label{compatibility-of-continuous-relations}

% Similarly, a type representing \emph{compatibility of a continuous relation} with a single operation of an algebra is defined using \AgdaFunction{cont-compatible-fun}~(\S\ref{continuous-relation-types}).
% \ccpad
% \begin{code}%
% % \>[0][@{}l@{\AgdaIndent{0}}]%
% \>[1]\AgdaFunction{cont-compatible-op}\AgdaSpace{}%
% \AgdaSymbol{:}\AgdaSpace{}%
% \AgdaOperator{\AgdaFunction{∣}}\AgdaSpace{}%
% \AgdaBound{𝑆}\AgdaSpace{}%
% \AgdaOperator{\AgdaFunction{∣}}\AgdaSpace{}%
% \AgdaSymbol{→}\AgdaSpace{}%
% \AgdaFunction{ContRel}\AgdaSpace{}%
% \AgdaBound{I}\AgdaSpace{}%
% \AgdaOperator{\AgdaFunction{∣}}\AgdaSpace{}%
% \AgdaBound{𝑨}\AgdaSpace{}%
% \AgdaOperator{\AgdaFunction{∣}}\AgdaSpace{}%
% \AgdaGeneralizable{𝓦}\AgdaSpace{}%
% \AgdaSymbol{→}\AgdaSpace{}%
% \AgdaBound{𝓤}\AgdaSpace{}%
% \AgdaOperator{\AgdaPrimitive{⊔}}\AgdaSpace{}%
% \AgdaBound{𝓥}\AgdaSpace{}%
% \AgdaOperator{\AgdaPrimitive{⊔}}\AgdaSpace{}%
% \AgdaGeneralizable{𝓦}\AgdaSpace{}%
% \AgdaOperator{\AgdaFunction{̇}}\<%
% \\
% %
% \>[1]\AgdaFunction{cont-compatible-op}\AgdaSpace{}%
% \AgdaBound{𝑓}\AgdaSpace{}%
% \AgdaBound{R}\AgdaSpace{}%
% \AgdaSymbol{=}\AgdaSpace{}%
% \AgdaFunction{cont-compatible-fun}\AgdaSpace{}%
% \AgdaSymbol{(}\AgdaBound{𝑓}\AgdaSpace{}%
% \AgdaOperator{\AgdaFunction{̂}}\AgdaSpace{}%
% \AgdaBound{𝑨}\AgdaSymbol{)}\AgdaSpace{}%
% \AgdaBound{R}\<%
% \end{code}
% \ccpad
% Whence, we obtain a type representing \emph{compatibility of a continuous relation with an algebra}.
% \ccpad
% \begin{code}%
% \>[0][@{}l@{\AgdaIndent{1}}]%
% \>[1]\AgdaFunction{cont-compatible}\AgdaSpace{}%
% \AgdaSymbol{:}\AgdaSpace{}%
% \AgdaFunction{ContRel}\AgdaSpace{}%
% \AgdaBound{I}\AgdaSpace{}%
% \AgdaOperator{\AgdaFunction{∣}}\AgdaSpace{}%
% \AgdaBound{𝑨}\AgdaSpace{}%
% \AgdaOperator{\AgdaFunction{∣}}\AgdaSpace{}%
% \AgdaGeneralizable{𝓦}\AgdaSpace{}%
% \AgdaSymbol{→}\AgdaSpace{}%
% \AgdaBound{𝓞}\AgdaSpace{}%
% \AgdaOperator{\AgdaPrimitive{⊔}}\AgdaSpace{}%
% \AgdaBound{𝓤}\AgdaSpace{}%
% \AgdaOperator{\AgdaPrimitive{⊔}}\AgdaSpace{}%
% \AgdaBound{𝓥}\AgdaSpace{}%
% \AgdaOperator{\AgdaPrimitive{⊔}}\AgdaSpace{}%
% \AgdaGeneralizable{𝓦}\AgdaSpace{}%
% \AgdaOperator{\AgdaFunction{̇}}\<%
% \\
% %
% \>[1]\AgdaFunction{cont-compatible}\AgdaSpace{}%
% \AgdaBound{R}\AgdaSpace{}%
% \AgdaSymbol{=}\AgdaSpace{}%
% \AgdaFunction{Π}\AgdaSpace{}%
% \AgdaBound{𝑓}\AgdaSpace{}%
% \AgdaFunction{꞉}\AgdaSpace{}%
% \AgdaOperator{\AgdaFunction{∣}}\AgdaSpace{}%
% \AgdaBound{𝑆}\AgdaSpace{}%
% \AgdaOperator{\AgdaFunction{∣}}\AgdaSpace{}%
% \AgdaFunction{,}\AgdaSpace{}%
% \AgdaFunction{cont-compatible-op}\AgdaSpace{}%
% \AgdaBound{𝑓}\AgdaSpace{}%
% \AgdaBound{R}\<%
% \end{code}
% \protect\hypertarget{fn2}{}{2 This voilates the ``don't repeat yourself'' (dry) principle of programming, but it might make it easier for readers to see what's going on. (In the {[}UALib{]}{[}{]} we try to put transparency before elegance.)}



\subsection{Products: types for products over arbitrary classes}\label{sec:product-algebras}
\firstsentence{Algebras}{Products}
% -*- TeX-master: "ualib-part1.tex" -*-
%%% Local Variables: 
%%% mode: latex
%%% TeX-engine: 'xetex
%%% End:
% We assume a fixed signature \ab{𝑆} \as : \af{Signature} \ab 𝓞 \ab 𝓥 throughout the module by starting with the line
% \AgdaKeyword{module}\AgdaSpace{}%
% \AgdaModule{Algebras.Products}\AgdaSpace{}%
% \AgdaSymbol{\{}\AgdaBound{𝑆}\AgdaSpace{}%
% \AgdaSymbol{:}\AgdaSpace{}%
% \AgdaFunction{Signature}\AgdaSpace{}%
% \AgdaGeneralizable{𝓞}\AgdaSpace{}%
% \AgdaGeneralizable{𝓥}\AgdaSymbol{\}}\AgdaSpace{}%
% \AgdaKeyword{where}.%
% We begin this module by assuming a signature \ab{𝑆} \as : \af{Signature} \ab 𝓞 \ab 𝓥 which is then present and available throughout the module.  Because of this, in contrast to descriptions of previous modules, we present the first few lines of the \ualibhtml{Algebras.Products} module in full.
% \ccpad
% \begin{code}%
% \>[0]\AgdaSymbol{\{-\#}\AgdaSpace{}%
% \AgdaKeyword{OPTIONS}\AgdaSpace{}%
% \AgdaPragma{--without-K}\AgdaSpace{}%
% \AgdaPragma{--exact-split}\AgdaSpace{}%
% \AgdaPragma{--safe}\AgdaSpace{}%
% \AgdaSymbol{\#-\}}\<%
% \\
% %
% \\[\AgdaEmptyExtraSkip]%
% \>[0]\AgdaKeyword{open}\AgdaSpace{}%
% \AgdaKeyword{import}\AgdaSpace{}%
% \AgdaModule{Algebras.Signatures}\AgdaSpace{}%
% \AgdaKeyword{using}\AgdaSpace{}%
% \AgdaSymbol{(}\AgdaFunction{Signature}\AgdaSymbol{;}\AgdaSpace{}%
% \AgdaGeneralizable{𝓞}\AgdaSymbol{;}\AgdaSpace{}%
% \AgdaGeneralizable{𝓥}\AgdaSymbol{)}\<%
% \\
% %
% % \\[\AgdaEmptyExtraSkip]%
% \>[0]\AgdaKeyword{module}\AgdaSpace{}%
% \AgdaModule{Algebras.Products}\AgdaSpace{}%
% \AgdaSymbol{\{}\AgdaBound{𝑆}\AgdaSpace{}%
% \AgdaSymbol{:}\AgdaSpace{}%
% \AgdaFunction{Signature}\AgdaSpace{}%
% \AgdaGeneralizable{𝓞}\AgdaSpace{}%
% \AgdaGeneralizable{𝓥}\AgdaSymbol{\}}\AgdaSpace{}%
% \AgdaKeyword{where}\<%
% \\
% %
% % \\[\AgdaEmptyExtraSkip]%
% \>[0]\AgdaKeyword{open}\AgdaSpace{}%
% \AgdaKeyword{import}\AgdaSpace{}%
% \AgdaModule{Algebras.Algebras}\AgdaSpace{}%
% \AgdaKeyword{hiding}\AgdaSpace{}%
% \AgdaSymbol{(}\AgdaGeneralizable{𝓞}\AgdaSymbol{;}\AgdaSpace{}%
% \AgdaGeneralizable{𝓥}\AgdaSymbol{)}\AgdaSpace{}%
% \AgdaKeyword{public}\<%
% \end{code}
% \ccpad
% Notice that we must import the \af{Signature} type from \ualibhtml{Algebras.Signatures} first so that we can use it to declare the signature \AgdaBound{𝑆} as a parameter of the \ualibhtml{Algebras.Products} module.
% \>[0]\AgdaKeyword{module}\AgdaSpace{}%
% \AgdaModule{Algebras.Products}\AgdaSpace{}%
% \AgdaSymbol{\{}\AgdaBound{𝑆}\AgdaSpace{}%
% \AgdaSymbol{:}\AgdaSpace{}%
% \AgdaFunction{Signature}\AgdaSpace{}%
% \AgdaGeneralizable{𝓞}\AgdaSpace{}%
% \AgdaGeneralizable{𝓥}\AgdaSymbol{\}}\AgdaSpace{}%
% \AgdaKeyword{where}\<%
% \end{code}

Recall the informal definition of a \defn{product} of a family of \ab 𝑆-algebras. Given a type \ab I~\as :~\ab 𝓘\af ̇ and a family \ab 𝒜~\as :~\ab I~\as →~\ab{Algebra}~\ab 𝓤~\ab 𝑆, the \defn{product} \af ⨅~\ab 𝒜 is the algebra whose domain is the Cartesian product \af Π~\ab 𝑖~\as ꞉~\ab I~\af ,~\af ∣~\ab 𝒜~\ab 𝑖~\af ∣ of the domains of the algebras in \ab 𝒜, and the operation symbols are interpreted point-wise in the following sense: if \ab 𝑓 is a \ab J-ary operation symbol and if \ab 𝑎~\as :~\af Π~\ab 𝑖~\as ꞉~\ab I~\af ,~\ab J~\as →~\ab 𝒜~\ab 𝑖 gives, for each \ab 𝑖~\as :~\ab I, a \ab J-tuple of elements of \ab 𝒜~\ab 𝑖, then we define (\ab 𝑓~\af ̂~\af ⨅~\ab 𝒜)~\ab 𝑎 := (\ab i~\as :~\ab I)~\as → (\ab 𝑓~\af ̂~\ab 𝒜~\ab i)(\ab 𝑎~\ab i). We now define a type that codifies this informal definition of product algebra.
% \footnote{Alternative equivalent notation for the domain of the product is \as ∀~\ab i~\as →~\af ∣~\ab 𝒜~\ab i~\af ∣.}
\ccpad
\begin{code}%
% \>[0]\AgdaKeyword{module}\AgdaSpace{}%
% \AgdaModule{\AgdaUnderscore{}}\AgdaSpace{}%
% \AgdaSymbol{\{}\AgdaBound{𝓤}\AgdaSpace{}%
% \AgdaBound{𝓘}\AgdaSpace{}%
% \AgdaSymbol{:}\AgdaSpace{}%
% \AgdaPostulate{Universe}\AgdaSymbol{\}\{}\AgdaBound{I}\AgdaSpace{}%
% \AgdaSymbol{:}\AgdaSpace{}%
% \AgdaBound{𝓘}\AgdaSpace{}%
% \AgdaOperator{\AgdaFunction{̇}}\AgdaSpace{}%
% \AgdaSymbol{\}}\AgdaSpace{}%
% \AgdaKeyword{where}\<%
% \\
% %
% \\[\AgdaEmptyExtraSkip]%
\>[0][@{}l@{\AgdaIndent{0}}]%
\>[1]\AgdaFunction{⨅}\AgdaSpace{}%
\AgdaSymbol{:}\AgdaSpace{}%
\AgdaSymbol{(}\AgdaBound{𝒜}\AgdaSpace{}%
\AgdaSymbol{:}\AgdaSpace{}%
\AgdaBound{I}\AgdaSpace{}%
\AgdaSymbol{→}\AgdaSpace{}%
\AgdaFunction{Algebra}\AgdaSpace{}%
\AgdaBound{𝓤}\AgdaSpace{}%
\AgdaBound{𝑆}\AgdaSpace{}%
\AgdaSymbol{)}\AgdaSpace{}%
\AgdaSymbol{→}\AgdaSpace{}%
\AgdaFunction{Algebra}\AgdaSpace{}%
\AgdaSymbol{(}\AgdaBound{𝓘}\AgdaSpace{}%
\AgdaOperator{\AgdaPrimitive{⊔}}\AgdaSpace{}%
\AgdaBound{𝓤}\AgdaSymbol{)}\AgdaSpace{}%
\AgdaBound{𝑆}\<%
\\
%
\\[\AgdaEmptyExtraSkip]%
%
\>[1]\AgdaFunction{⨅}\AgdaSpace{}%
\AgdaBound{𝒜}\AgdaSpace{}%
\AgdaSymbol{=}%
\>[51I]\AgdaSymbol{(}\AgdaFunction{Π}\AgdaSpace{}%
\AgdaBound{i}\AgdaSpace{}%
\AgdaFunction{꞉}\AgdaSpace{}%
\AgdaBound{I}\AgdaSpace{}%
\AgdaFunction{,}\AgdaSpace{}%
\AgdaOperator{\AgdaFunction{∣}}\AgdaSpace{}%
\AgdaBound{𝒜}\AgdaSpace{}%
\AgdaBound{i}\AgdaSpace{}%
\AgdaOperator{\AgdaFunction{∣}}\AgdaSymbol{)}\AgdaSpace{}%
\AgdaOperator{\AgdaInductiveConstructor{,}}%
\>[43]\AgdaComment{-- domain of the product algebra}\<%
\\
\>[.][@{}l@{}]\<[51I]%
\>[7]\AgdaSymbol{λ}\AgdaSpace{}%
\AgdaBound{𝑓}\AgdaSpace{}%
\AgdaBound{𝑎}\AgdaSpace{}%
\AgdaBound{i}\AgdaSpace{}%
\AgdaSymbol{→}\AgdaSpace{}%
\AgdaSymbol{(}\AgdaBound{𝑓}\AgdaSpace{}%
\AgdaOperator{\AgdaFunction{̂}}\AgdaSpace{}%
\AgdaBound{𝒜}\AgdaSpace{}%
\AgdaBound{i}\AgdaSymbol{)}\AgdaSpace{}%
\AgdaSymbol{λ}\AgdaSpace{}%
\AgdaBound{x}\AgdaSpace{}%
\AgdaSymbol{→}\AgdaSpace{}%
\AgdaBound{𝑎}\AgdaSpace{}%
\AgdaBound{x}\AgdaSpace{}%
\AgdaBound{i}%
\>[43]\AgdaComment{-- basic operations of the product algebra}\<%
\end{code}
\ccpad
% The type just defined is the one we use whenever the product of a collection of
% algebras (of type \af{Algebra}) is required.
%%%%%%%%%%%%%%%%%%%%%%%%%%%%%%%%%%%%%%%%%%%%%%%%%%%%%%%%%%%%%%%%%%%%%%%%%%%%%%%%%%%%%%%%%
%% COMMENTED OUT %%%%%%%%%%%%%%%%%%%%%%%%%%%%%%%%%%%%%%%%%%%%%%%%%%%%%%%%%%%%%%%%%%%%%%%%
\begin{comment}
However, for the sake of completeness, here is how one could define a type representing the product of algebras inhabiting the record type \AgdaRecord{algebra}.
\ccpad
\begin{code}%
\>[0][@{}l@{\AgdaIndent{0}}]%
\>[1]\AgdaFunction{⨅'}\AgdaSpace{}%
\AgdaSymbol{:}\AgdaSpace{}%
\AgdaSymbol{(}\AgdaBound{𝒜}\AgdaSpace{}%
\AgdaSymbol{:}\AgdaSpace{}%
\AgdaBound{I}\AgdaSpace{}%
\AgdaSymbol{→}\AgdaSpace{}%
\AgdaRecord{algebra}\AgdaSpace{}%
\AgdaBound{𝓤}\AgdaSpace{}%
\AgdaBound{𝑆}\AgdaSymbol{)}\AgdaSpace{}%
\AgdaSymbol{→}\AgdaSpace{}%
\AgdaRecord{algebra}\AgdaSpace{}%
\AgdaSymbol{(}\AgdaBound{𝓘}\AgdaSpace{}%
\AgdaOperator{\AgdaPrimitive{⊔}}\AgdaSpace{}%
\AgdaBound{𝓤}\AgdaSymbol{)}\AgdaSpace{}%
\AgdaBound{𝑆}\<%
\\
\>[1]\AgdaFunction{⨅'}\AgdaSpace{}%
\AgdaBound{𝒜}\AgdaSpace{}%
\AgdaSymbol{=}\AgdaSpace{}%
\AgdaKeyword{record}%
\>[100I]\AgdaSymbol{\{}\AgdaSpace{}%
\AgdaField{univ}\AgdaSpace{}%
\AgdaSymbol{=}\AgdaSpace{}%
\AgdaSymbol{∀}\AgdaSpace{}%
\AgdaBound{i}\AgdaSpace{}%
\AgdaSymbol{→}\AgdaSpace{}%
\AgdaField{univ}\AgdaSpace{}%
\AgdaSymbol{(}\AgdaBound{𝒜}\AgdaSpace{}%
\AgdaBound{i}\AgdaSymbol{)}\AgdaSpace{}\AgdaSymbol{;}%
\>[55]\AgdaComment{-- domain}\<%
\\
\>[100I][@{}l@{\AgdaIndent{0}}]%
\>[15]\AgdaField{op}\AgdaSpace{}%
\AgdaSymbol{=}\AgdaSpace{}%
\AgdaSymbol{λ}\AgdaSpace{}%
\AgdaBound{𝑓}\AgdaSpace{}%
\AgdaBound{𝑎}\AgdaSpace{}%
\AgdaBound{i}\AgdaSpace{}%
\AgdaSymbol{→}\AgdaSpace{}%
\AgdaSymbol{(}\AgdaField{op}\AgdaSpace{}%
\AgdaSymbol{(}\AgdaBound{𝒜}\AgdaSpace{}%
\AgdaBound{i}\AgdaSymbol{))}\AgdaSpace{}%
\AgdaBound{𝑓}\AgdaSpace{}%
\AgdaSymbol{λ}\AgdaSpace{}%
\AgdaBound{x}\AgdaSpace{}%
\AgdaSymbol{→}\AgdaSpace{}%
\AgdaBound{𝑎}\AgdaSpace{}%
\AgdaBound{x}\AgdaSpace{}%
\AgdaBound{i}\AgdaSpace{}\AgdaSymbol{\}}%
\>[55]\AgdaComment{-- basic operations}\<%
\end{code}
\ccpad
\end{comment}
%% END COMMENTED OUT %%%%%%%%%%%%%%%%%%%%%%%%%%%%%%%%%%%%%%%%%%%%%%%%%%%%%%%%%%%%%%%%%%%%
%%%%%%%%%%%%%%%%%%%%%%%%%%%%%%%%%%%%%%%%%%%%%%%%%%%%%%%%%%%%%%%%%%%%%%%%%%%%%%%%%%%%%%%%%
Before going further, let us agree on another convenient notational convention, which is used in many of the later modules of the \ualib. Given a signature \ab{𝑆}~\as :~\af{Signature}~\ab 𝓞~\ab 𝓥, the type \af{Algebra}~\ab 𝓤~\ab 𝑆 has type \ab{𝓞}~\ap ⊔~\ab 𝓥~\ap ⊔~\ab 𝓤~\af ⁺\af ̇, and \ab{𝓞} \ap ⊔ \ab 𝓥 remains fixed since \ab{𝓞} and \ab 𝓥 always denote the universes of operation and arity types, respectively. Such levels occur so often in the \ualib that we define the following shorthand for their universes: \AgdaFunction{ov}\AgdaSpace{}%
\AgdaBound{𝓤}\AgdaSpace{}%
\AgdaSymbol{:=}\AgdaSpace{}%
\AgdaBound{𝓞}\AgdaSpace{}%
\AgdaOperator{\AgdaPrimitive{⊔}}\AgdaSpace{}%
\AgdaBound{𝓥}\AgdaSpace{}%
\AgdaOperator{\AgdaPrimitive{⊔}}\AgdaSpace{}%
\AgdaBound{𝓤}\AgdaSpace{}%
\AgdaOperator{\AgdaPrimitive{⁺}}.


\paragraph*{Products of classes of algebras}\label{products-of-classes-of-algebras}
% \context{\AgdaSymbol{\{}\AgdaBound{𝓤}\AgdaSpace{}%
% \AgdaBound{𝓧}\AgdaSpace{}%
% \AgdaSymbol{:}\AgdaSpace{}%
% \AgdaPostulate{Universe}\AgdaSymbol{\}\{}\AgdaBound{X}\AgdaSpace{}%
% \AgdaSymbol{:}\AgdaSpace{}%
% \AgdaBound{𝓧}\AgdaSpace{}%
% \AgdaOperator{\AgdaFunction{̇}}\AgdaSymbol{\}}}

An arbitrary class \ab 𝒦 of algebras is represented as a predicate over the type \AgdaFunction{Algebra}\AgdaSpace{}\AgdaBound{𝓤}\AgdaSpace{}\AgdaBound{𝑆}, for some universe level \AgdaBound{𝓤} and signature \AgdaBound{𝑆}.
That is, \ab 𝒦~\as :~\af{Pred}(\af{Algebra}~\ab 𝓤~\ab 𝑆)~𝓦 for some 𝓦.
Later we will formally state and prove that the product of all subalgebras of algebras in \ab 𝒦  belongs to the class \af{SP}(\ab 𝒦) of subalgebras of products of algebras in \ab 𝒦. That is, \af ⨅ \af S(\ab 𝒦) \af ∈ \af{SP}(\ab 𝒦). This turns out to be a nontrivial exercise.

To begin, we need to define types that represent products over arbitrary (non-indexed) families such as \ab 𝒦 or \af S(\ab 𝒦). Observe that \ad Π~\ab 𝒦 is certainly not what we want.  For recall that \af{Pred}(\ab{Algebra}~\ab 𝓤~\ab 𝑆)~\ab 𝓦 is just an alias for the function type \af{Algebra}~\ab 𝓤~\ab 𝑆~\as →~\ab 𝓦\af ̇, and the semantics of the latter takes \ab 𝒦~\ab 𝑨 to mean that \ab 𝑨 belongs to the class \ab 𝒦. Therefore, by definition\\[-10pt]

% \vskip-5mm
% \begin{align*}
% \ad Π~\ab 𝒦 &= \ad Π~\ab 𝑨~\af ꞉~(\af{Algebra}~\ab 𝓤~\ab 𝑆)~,~\ab 𝒦 𝑨\\
%              &= \ad Π~\ab 𝑨~\af ꞉~(\af{Algebra}~\ab 𝓤~\ab 𝑆)~,~\ab 𝑨~\af ∈~\ab 𝒦.
% \end{align*}
\ad Π~\ab 𝒦 \hskip1mm = \hskip1mm \ad Π~\ab 𝑨~\af ꞉~(\af{Algebra}~\ab 𝓤~\ab 𝑆)~,~\ab 𝒦 𝑨
             \hskip1mm = \hskip1mm \as ∀ (\ab 𝑨~\as :~\af{Algebra}~\ab 𝓤~\ab 𝑆)~\as →~\ab 𝑨~\af ∈~\ab 𝒦,\\[4pt]
which denotes the assertion that every inhabitant of the type \af{Algebra} \ab 𝓤 \ab 𝑆 belongs to \ab 𝒦. .  Evidently this is not the product algebra that we seek.

What we need is a type that serves to index the class \ab 𝒦, and a function \af 𝔄 that maps an index to the inhabitant of \ab 𝒦 at that index. But \ab 𝒦 is a predicate (of type (\af{Algebra}~\ab 𝓤~\ab 𝑆)~\as →~\ab 𝓦\af ̇) and the type \af{Algebra}~\ab 𝓤~\ab 𝑆 seems rather nebulous in that there is no natural indexing class with which to ``enumerate'' all inhabitants of \af{Algebra}~\ab 𝓤~\ab 𝑆 that belong to \ab 𝒦.\footnote{%
If you haven't seen this before, give it some thought and see if the correct type comes to you organically.}


The solution is to essentially take \ab 𝒦 itself to be the index type; at least heuristically that is how one can think of the type \af ℑ that we now define.\footnote{\textbf{Unicode Hints}. Some of our types are denoted with Gothic (``mathfrak'') symbols. To produce them in \agdamode, type \texttt{\textbackslash{}Mf} followed by a letter. For example, \texttt{\textbackslash{}MfI} ↝ \af ℑ.}
\ccpad
\begin{code}%
% \>[0]\AgdaKeyword{module}\AgdaSpace{}%
% \AgdaModule{class-products}\AgdaSpace{}%
% \AgdaSymbol{\{}\AgdaBound{𝓤}\AgdaSpace{}%
% \AgdaSymbol{:}\AgdaSpace{}%
% \AgdaPostulate{Universe}\AgdaSymbol{\}}\AgdaSpace{}%
% \AgdaSymbol{(}\AgdaBound{𝒦}\AgdaSpace{}%
% \AgdaSymbol{:}\AgdaSpace{}%
% \AgdaFunction{Pred}\AgdaSpace{}%
% \AgdaSymbol{(}\AgdaFunction{Algebra}\AgdaSpace{}%
% \AgdaBound{𝓤}\AgdaSpace{}%
% \AgdaBound{𝑆}\AgdaSymbol{)(}\AgdaFunction{ov}\AgdaSpace{}%
% \AgdaBound{𝓤}\AgdaSymbol{))}\AgdaSpace{}%
% \AgdaKeyword{where}\<%
% \\
% %
% \\[\AgdaEmptyExtraSkip]%
\>[0][@{}l@{\AgdaIndent{0}}]%
\>[1]\AgdaFunction{ℑ}\AgdaSpace{}%
\AgdaSymbol{:}\AgdaSpace{}%
\AgdaFunction{ov}\AgdaSpace{}%
\AgdaBound{𝓤}\AgdaSpace{}%
\AgdaOperator{\AgdaFunction{̇}}\<%
\\
%
\>[1]\AgdaFunction{ℑ}\AgdaSpace{}%
\AgdaSymbol{=}\AgdaSpace{}%
\AgdaFunction{Σ}\AgdaSpace{}%
\AgdaBound{𝑨}\AgdaSpace{}%
\AgdaFunction{꞉}\AgdaSpace{}%
\AgdaSymbol{(}\AgdaFunction{Algebra}\AgdaSpace{}%
\AgdaBound{𝓤}\AgdaSpace{}%
\AgdaBound{𝑆}\AgdaSymbol{)}\AgdaSpace{}%
\AgdaFunction{,}\AgdaSpace{}%
\AgdaSymbol{(}\AgdaBound{𝑨}\AgdaSpace{}%
\AgdaOperator{\AgdaFunction{∈}}\AgdaSpace{}%
\AgdaBound{𝒦}\AgdaSymbol{)}\<%
\end{code}
\ccpad
Taking the product over the index type \af ℑ requires a function that maps an index \abt{i}{ℑ} to the corresponding algebra.  Each \abt{i}{ℑ} denotes a pair, (\ab 𝑨 , \ab p), where \ab 𝑨 is an algebra and \ab p is a proof that \ab 𝑨 belongs to \ab 𝒦, so the function mapping such an index to the corresponding algebra is simply the first projection.
\ccpad
\begin{code}
\>[0][@{}l@{\AgdaIndent{0}}]%
\>[1]\AgdaFunction{𝔄}\AgdaSpace{}%
\AgdaSymbol{:}\AgdaSpace{}%
\AgdaFunction{ℑ}\AgdaSpace{}%
\AgdaSymbol{→}\AgdaSpace{}%
\AgdaFunction{Algebra}\AgdaSpace{}%
\AgdaBound{𝓤}\AgdaSpace{}%
\AgdaBound{𝑆}\<%
\\
%
\>[1]\AgdaFunction{𝔄}\AgdaSpace{}%
\AgdaSymbol{=}\AgdaSpace{}%
\AgdaSymbol{λ}\AgdaSpace{}%
\AgdaSymbol{(}\AgdaBound{i}\AgdaSpace{}%
\AgdaSymbol{:}\AgdaSpace{}%
\AgdaFunction{ℑ}\AgdaSymbol{)}\AgdaSpace{}%
\AgdaSymbol{→}\AgdaSpace{}%
\AgdaOperator{\AgdaFunction{∣}}\AgdaSpace{}%
\AgdaBound{i}\AgdaSpace{}%
\AgdaOperator{\AgdaFunction{∣}}\<%
\end{code}
\ccpad
Finally, we represent the product of all members of the class \ab 𝒦 by the following type.
\ccpad
\begin{code}%
\>[0][@{}l@{\AgdaIndent{0}}]%
\>[1]\AgdaFunction{class-product}\AgdaSpace{}%
\AgdaSymbol{:}\AgdaSpace{}%
\AgdaFunction{Algebra}\AgdaSpace{}%
\AgdaSymbol{(}\AgdaFunction{ov}\AgdaSpace{}%
\AgdaBound{𝓤}\AgdaSymbol{)}\AgdaSpace{}%
\AgdaBound{𝑆}\<%
\\
%
\>[1]\AgdaFunction{class-product}\AgdaSpace{}%
\AgdaSymbol{=}\AgdaSpace{}%
\AgdaFunction{⨅}\AgdaSpace{}%
\AgdaFunction{𝔄}\<%
\end{code}
\ccpad
Observe that the application % \af 𝔄 (\ab 𝑨 , \ab p) 
of \af 𝔄 to the pair (\ab 𝑨 , \ab p) (the result of which is simply the algebra \ab{𝑨}) may be viewed as the projection out of the product \af ⨅~\af 𝔄 and onto the ``(\ab 𝑨,~\ab p)-th component'' of that product.



\subsection{Congruences: types for congruences \& quotient algebras}\label{congruences}
\firstsentence{Algebras}{Congruences}
% -*- TeX-master: "ualib-part1.tex" -*-
%%% Local Variables: 
%%% mode: latex
%%% TeX-engine: 'xetex
%%% End:
A \defn{congruence relation} of an algebra \ab{𝑨} is defined to be an equivalence relation that is compatible with the basic operations of \ab 𝑨. This concept can be represented in a number of alternative but equivalent ways.  Informally, a relation is a congruence if and only if it is both an equivalence relation on the domain of \ab 𝑨 and a subalgebra of the square of \ab 𝑨.  Formally, we represent a congruence as an inhabitant of either a the Sigma type \af{Con} or the record type \af{Congruence}, which we now define.
\ccpad
\begin{code}%
\>[0]\AgdaFunction{Con}\AgdaSpace{}%
\AgdaSymbol{:}\AgdaSpace{}%
\AgdaSymbol{\{}\AgdaBound{𝓤}\AgdaSpace{}%
\AgdaSymbol{:}\AgdaSpace{}%
\AgdaFunction{Universe}\AgdaSymbol{\}(}\AgdaBound{𝑨}\AgdaSpace{}%
\AgdaSymbol{:}\AgdaSpace{}%
\AgdaFunction{Algebra}\AgdaSpace{}%
\AgdaBound{𝓤}\AgdaSpace{}%
\AgdaBound{𝑆}\AgdaSymbol{)}\AgdaSpace{}%
\AgdaSymbol{→}\AgdaSpace{}%
\AgdaFunction{ov}\AgdaSpace{}%
\AgdaBound{𝓤}\AgdaSpace{}%
\AgdaOperator{\AgdaFunction{̇}}\<%
\\
\>[0]\AgdaFunction{Con}\AgdaSpace{}%
\AgdaSymbol{\{}\AgdaBound{𝓤}\AgdaSymbol{\}}\AgdaSpace{}%
\AgdaBound{𝑨}\AgdaSpace{}%
\AgdaSymbol{=}\AgdaSpace{}%
\AgdaFunction{Σ}\AgdaSpace{}%
\AgdaBound{θ}\AgdaSpace{}%
\AgdaFunction{꞉}\AgdaSpace{}%
\AgdaSymbol{(}\AgdaSpace{}%
\AgdaFunction{Rel}\AgdaSpace{}%
\AgdaOperator{\AgdaFunction{∣}}\AgdaSpace{}%
\AgdaBound{𝑨}\AgdaSpace{}%
\AgdaOperator{\AgdaFunction{∣}}\AgdaSpace{}%
\AgdaBound{𝓤}\AgdaSpace{}%
\AgdaSymbol{)}\AgdaSpace{}%
\AgdaFunction{,}\AgdaSpace{}%
\AgdaRecord{IsEquivalence}\AgdaSpace{}%
\AgdaBound{θ}\AgdaSpace{}%
\AgdaOperator{\AgdaFunction{×}}\AgdaSpace{}%
\AgdaFunction{compatible}\AgdaSpace{}%
\AgdaBound{𝑨}\AgdaSpace{}%
\AgdaBound{θ}\<%
\\
%
\\[\AgdaEmptyExtraSkip]%
\>[0]\AgdaKeyword{record}\AgdaSpace{}%
\AgdaRecord{Congruence}\AgdaSpace{}%
\AgdaSymbol{\{}\AgdaBound{𝓤}\AgdaSpace{}%
\AgdaBound{𝓦}\AgdaSpace{}%
\AgdaSymbol{:}\AgdaSpace{}%
\AgdaFunction{Universe}\AgdaSymbol{\}}\AgdaSpace{}%
\AgdaSymbol{(}\AgdaBound{𝑨}\AgdaSpace{}%
\AgdaSymbol{:}\AgdaSpace{}%
\AgdaFunction{Algebra}\AgdaSpace{}%
\AgdaBound{𝓤}\AgdaSpace{}%
\AgdaBound{𝑆}\AgdaSymbol{)}\AgdaSpace{}%
\AgdaSymbol{:}\AgdaSpace{}%
\AgdaFunction{ov}\AgdaSpace{}%
\AgdaBound{𝓦}\AgdaSpace{}%
\AgdaOperator{\AgdaFunction{⊔}}\AgdaSpace{}%
\AgdaBound{𝓤}\AgdaSpace{}%
\AgdaOperator{\AgdaFunction{̇}}%
\>[67]\AgdaKeyword{where}\<%
\\
\>[0][@{}l@{\AgdaIndent{0}}]%
\>[1]\AgdaKeyword{constructor}\AgdaSpace{}%
\AgdaInductiveConstructor{mkcon}\<%
\\
%
\>[1]\AgdaKeyword{field}\<%
\\
\>[1][@{}l@{\AgdaIndent{0}}]%
\>[2]\AgdaOperator{\AgdaField{⟨\AgdaUnderscore{}⟩}}\AgdaSpace{}%
\AgdaSymbol{:}\AgdaSpace{}%
\AgdaFunction{Rel}\AgdaSpace{}%
\AgdaOperator{\AgdaFunction{∣}}\AgdaSpace{}%
\AgdaBound{𝑨}\AgdaSpace{}%
\AgdaOperator{\AgdaFunction{∣}}\AgdaSpace{}%
\AgdaBound{𝓦}\<%
\\
%
\>[2]\AgdaField{Compatible}\AgdaSpace{}%
\AgdaSymbol{:}\AgdaSpace{}%
\AgdaFunction{compatible}\AgdaSpace{}%
\AgdaBound{𝑨}\AgdaSpace{}%
\AgdaOperator{\AgdaField{⟨\AgdaUnderscore{}⟩}}\<%
\\
%
\>[2]\AgdaField{IsEquiv}\AgdaSpace{}%
\AgdaSymbol{:}\AgdaSpace{}%
\AgdaRecord{IsEquivalence}\AgdaSpace{}%
\AgdaOperator{\AgdaField{⟨\AgdaUnderscore{}⟩}}\<%
\end{code}
\ccpad
Each of these types captures the informal notion of congruence, and the only real difference between them is that \af{Congruence} includes the extra universe parameter \ab 𝓦 to accommodate more general underlying relations.   Otherwise, the two definitions are equivalent in the sense that each implies the other, as we now prove.
\ccpad
\begin{code}
% \>[0]\AgdaKeyword{module}\AgdaSpace{}%
% \AgdaModule{\AgdaUnderscore{}}\AgdaSpace{}%
% \AgdaSymbol{\{}\AgdaBound{𝑨}\AgdaSpace{}%
% \AgdaSymbol{:}\AgdaSpace{}%
% \AgdaFunction{Algebra}\AgdaSpace{}%
% \AgdaGeneralizable{𝓤}\AgdaSpace{}%
% \AgdaBound{𝑆}\AgdaSymbol{\}}\AgdaSpace{}%
% \AgdaKeyword{where}\<%
% \\
% %
% \\[\AgdaEmptyExtraSkip]%
\>[0][@{}l@{\AgdaIndent{0}}]%
\>[1]\AgdaFunction{Con→Congruence}\AgdaSpace{}%
\AgdaSymbol{:}\AgdaSpace{}%
\AgdaFunction{Con}\AgdaSpace{}%
\AgdaBound{𝑨}\AgdaSpace{}%
\AgdaSymbol{→}\AgdaSpace{}%
\AgdaRecord{Congruence}\AgdaSymbol{\{}\AgdaBound{𝓤}\AgdaSymbol{\}}\AgdaSpace{}%
\AgdaBound{𝑨}\<%
\\
%
\>[1]\AgdaFunction{Con→Congruence}\AgdaSpace{}%
\AgdaBound{θ}\AgdaSpace{}%
\AgdaSymbol{=}\AgdaSpace{}%
\AgdaInductiveConstructor{mkcon}\AgdaSpace{}%
\AgdaOperator{\AgdaFunction{∣}}\AgdaSpace{}%
\AgdaBound{θ}\AgdaSpace{}%
\AgdaOperator{\AgdaFunction{∣}}\AgdaSpace{}%
\AgdaSymbol{(}\AgdaFunction{fst}\AgdaSpace{}%
\AgdaOperator{\AgdaFunction{∥}}\AgdaSpace{}%
\AgdaBound{θ}\AgdaSpace{}%
\AgdaOperator{\AgdaFunction{∥}}\AgdaSymbol{)}\AgdaSpace{}%
\AgdaSymbol{(}\AgdaFunction{snd}\AgdaSpace{}%
\AgdaOperator{\AgdaFunction{∥}}\AgdaSpace{}%
\AgdaBound{θ}\AgdaSpace{}%
\AgdaOperator{\AgdaFunction{∥}}\AgdaSymbol{)}\<%
\\
%
% \\[\AgdaEmptyExtraSkip]%
% %
% \>[1]\AgdaKeyword{open}\AgdaSpace{}%
% \AgdaModule{Congruence}\<%
% \\
% %
\\[\AgdaEmptyExtraSkip]%
% %
\>[1]\AgdaFunction{Congruence→Con}\AgdaSpace{}%
\AgdaSymbol{:}\AgdaSpace{}%
\AgdaRecord{Congruence}\AgdaSymbol{\{}\AgdaBound{𝓤}\AgdaSymbol{\}}\AgdaSpace{}%
\AgdaBound{𝑨}\AgdaSpace{}%
\AgdaSymbol{→}%
\>[37]\AgdaFunction{Con}\AgdaSpace{}%
\AgdaBound{𝑨}\<%
\\
%
\>[1]\AgdaFunction{Congruence→Con}\AgdaSpace{}%
\AgdaBound{θ}\AgdaSpace{}%
\AgdaSymbol{=}\AgdaSpace{}%
\AgdaOperator{\AgdaField{⟨}}\AgdaSpace{}%
\AgdaBound{θ}\AgdaSpace{}%
\AgdaOperator{\AgdaField{⟩}}\AgdaSpace{}%
\AgdaOperator{\AgdaInductiveConstructor{,}}\AgdaSpace{}%
\AgdaField{IsEquiv}\AgdaSpace{}%
\AgdaBound{θ}\AgdaSpace{}%
\AgdaOperator{\AgdaInductiveConstructor{,}}\AgdaSpace{}%
\AgdaField{Compatible}\AgdaSpace{}%
\AgdaBound{θ}\<%
\end{code}
\ccpad
We defined the \defn{zero relation} \af{𝟎-rel} in the \ualibhtml{Relations.Discrete} module. We defined the zero relation `𝟎-rel` in the [Relations.Discrete][] module.  We now build the \defn{trivial congruence}, which has \af{𝟎-rel} as its underlying relation. Observe that \af{𝟎-rel} is equivalent to the identity relation \ad{≡} and these are obviously both equivalences. In fact, we already proved this of \ad{≡} in the \ualibhtml{Overture.Equality} module, so we simply apply the corresponding proofs.
\ccpad
\begin{code}%
\>[0]\AgdaFunction{𝟎-IsEquivalence}\AgdaSpace{}%
\AgdaSymbol{:}\AgdaSpace{}%
\AgdaSymbol{\{}\AgdaBound{A}\AgdaSpace{}%
\AgdaSymbol{:}\AgdaSpace{}%
\AgdaGeneralizable{𝓤}\AgdaSpace{}%
\AgdaOperator{\AgdaFunction{̇}}\AgdaSymbol{\}}\AgdaSpace{}%
\AgdaSymbol{→}%
\>[31]\AgdaRecord{IsEquivalence}\AgdaSpace{}%
\AgdaSymbol{\{}\AgdaArgument{A}\AgdaSpace{}%
\AgdaSymbol{=}\AgdaSpace{}%
\AgdaBound{A}\AgdaSymbol{\}}\AgdaSpace{}%
\AgdaFunction{𝟎}\<%
\\
\>[0]\AgdaFunction{𝟎-IsEquivalence}\AgdaSpace{}%
\AgdaSymbol{=}\AgdaSpace{}%
\AgdaKeyword{record}\AgdaSpace{}%
\AgdaSymbol{\{}\AgdaField{rfl}\AgdaSpace{}%
\AgdaSymbol{=}\AgdaSpace{}%
\AgdaSymbol{λ}\AgdaSpace{}%
\AgdaBound{x}\AgdaSpace{}%
\AgdaSymbol{→}\AgdaSpace{}%
\AgdaInductiveConstructor{refl}\AgdaSymbol{\{}\AgdaArgument{x}\AgdaSpace{}%
\AgdaSymbol{=}\AgdaSpace{}%
\AgdaBound{x}\AgdaSymbol{\};}\AgdaSpace{}%
\AgdaField{sym}\AgdaSpace{}%
\AgdaSymbol{=}\AgdaSpace{}%
\AgdaFunction{≡-symmetric}\AgdaSymbol{;}\AgdaSpace{}%
\AgdaField{trans}\AgdaSpace{}%
\AgdaSymbol{=}\AgdaSpace{}%
\AgdaFunction{≡-transitive}\AgdaSymbol{\}}\<%
\end{code}
\ccpad
Next we formally record another obvious fact---namely, that \af{𝟎-rel} is compatible with all operations of all algebras.
\ccpad
\begin{code}%
\>[0]\AgdaFunction{𝟎-compatible-op}\AgdaSpace{}%
\AgdaSymbol{:}\AgdaSpace{}%
\AgdaFunction{funext}\AgdaSpace{}%
\AgdaBound{𝓥}\AgdaSpace{}%
\AgdaGeneralizable{𝓤}\AgdaSpace{}%
\AgdaSymbol{→}\AgdaSpace{}%
\AgdaSymbol{\{}\AgdaBound{𝑨}\AgdaSpace{}%
\AgdaSymbol{:}\AgdaSpace{}%
\AgdaFunction{Algebra}\AgdaSpace{}%
\AgdaGeneralizable{𝓤}\AgdaSpace{}%
\AgdaBound{𝑆}\AgdaSymbol{\}}\AgdaSpace{}%
\AgdaSymbol{(}\AgdaBound{𝑓}\AgdaSpace{}%
\AgdaSymbol{:}\AgdaSpace{}%
\AgdaOperator{\AgdaFunction{∣}}\AgdaSpace{}%
\AgdaBound{𝑆}\AgdaSpace{}%
\AgdaOperator{\AgdaFunction{∣}}\AgdaSymbol{)}\AgdaSpace{}%
\AgdaSymbol{→}\AgdaSpace{}%
\AgdaFunction{compatible-fun}\AgdaSpace{}%
\AgdaSymbol{(}\AgdaBound{𝑓}\AgdaSpace{}%
\AgdaOperator{\AgdaFunction{̂}}\AgdaSpace{}%
\AgdaBound{𝑨}\AgdaSymbol{)}\AgdaSpace{}%
\AgdaFunction{𝟎}\<%
\\
\>[0]\AgdaFunction{𝟎-compatible-op}\AgdaSpace{}%
\AgdaBound{fe}\AgdaSpace{}%
\AgdaSymbol{\{}\AgdaBound{𝑨}\AgdaSymbol{\}}\AgdaSpace{}%
\AgdaBound{𝑓}\AgdaSpace{}%
\AgdaBound{ptws0}%
\>[32]\AgdaSymbol{=}\AgdaSpace{}%
\AgdaFunction{ap}\AgdaSpace{}%
\AgdaSymbol{(}\AgdaBound{𝑓}\AgdaSpace{}%
\AgdaOperator{\AgdaFunction{̂}}\AgdaSpace{}%
\AgdaBound{𝑨}\AgdaSymbol{)}\AgdaSpace{}%
\AgdaSymbol{(}\AgdaBound{fe}\AgdaSpace{}%
\AgdaSymbol{(λ}\AgdaSpace{}%
\AgdaBound{x}\AgdaSpace{}%
\AgdaSymbol{→}\AgdaSpace{}%
\AgdaBound{ptws0}\AgdaSpace{}%
\AgdaBound{x}\AgdaSymbol{))}\<%
\\
%
\\[\AgdaEmptyExtraSkip]%
\>[0]\AgdaFunction{𝟎-compatible}\AgdaSpace{}%
\AgdaSymbol{:}\AgdaSpace{}%
\AgdaFunction{funext}\AgdaSpace{}%
\AgdaBound{𝓥}\AgdaSpace{}%
\AgdaGeneralizable{𝓤}\AgdaSpace{}%
\AgdaSymbol{→}\AgdaSpace{}%
\AgdaSymbol{\{}\AgdaBound{𝑨}\AgdaSpace{}%
\AgdaSymbol{:}\AgdaSpace{}%
\AgdaFunction{Algebra}\AgdaSpace{}%
\AgdaGeneralizable{𝓤}\AgdaSpace{}%
\AgdaBound{𝑆}\AgdaSymbol{\}}\AgdaSpace{}%
\AgdaSymbol{→}\AgdaSpace{}%
\AgdaFunction{compatible}\AgdaSpace{}%
\AgdaBound{𝑨}\AgdaSpace{}%
\AgdaFunction{𝟎}\<%
\\
\>[0]\AgdaFunction{𝟎-compatible}\AgdaSpace{}%
\AgdaBound{fe}\AgdaSpace{}%
\AgdaSymbol{\{}\AgdaBound{𝑨}\AgdaSymbol{\}}\AgdaSpace{}%
\AgdaSymbol{=}\AgdaSpace{}%
\AgdaSymbol{λ}\AgdaSpace{}%
\AgdaBound{𝑓}\AgdaSpace{}%
\AgdaBound{args}\AgdaSpace{}%
\AgdaSymbol{→}\AgdaSpace{}%
\AgdaFunction{𝟎-compatible-op}\AgdaSpace{}%
\AgdaBound{fe}\AgdaSpace{}%
\AgdaSymbol{\{}\AgdaBound{𝑨}\AgdaSymbol{\}}\AgdaSpace{}%
\AgdaBound{𝑓}\AgdaSpace{}%
\AgdaBound{args}\<%
\end{code}
\ccpad
Finally, we have the ingredients need to construct the zero congruence of any algebra we like.
\ccpad
\begin{code}%
\>[0]\AgdaFunction{Δ}\AgdaSpace{}%
\AgdaSymbol{:}\AgdaSpace{}%
\AgdaFunction{funext}\AgdaSpace{}%
\AgdaBound{𝓥}\AgdaSpace{}%
\AgdaGeneralizable{𝓤}\AgdaSpace{}%
\AgdaSymbol{→}\AgdaSpace{}%
\AgdaSymbol{\{}\AgdaBound{𝑨}\AgdaSpace{}%
\AgdaSymbol{:}\AgdaSpace{}%
\AgdaFunction{Algebra}\AgdaSpace{}%
\AgdaGeneralizable{𝓤}\AgdaSpace{}%
\AgdaBound{𝑆}\AgdaSymbol{\}}\AgdaSpace{}%
\AgdaSymbol{→}\AgdaSpace{}%
\AgdaRecord{Congruence}\AgdaSpace{}%
\AgdaBound{𝑨}\<%
\\
\>[0]\AgdaFunction{Δ}\AgdaSpace{}%
\AgdaBound{fe}\AgdaSpace{}%
\AgdaSymbol{=}\AgdaSpace{}%
\AgdaInductiveConstructor{mkcon}\AgdaSpace{}%
\AgdaFunction{𝟎}\AgdaSpace{}%
\AgdaSymbol{(}\AgdaFunction{𝟎-compatible}\AgdaSpace{}%
\AgdaBound{fe}\AgdaSymbol{)}\AgdaSpace{}%
\AgdaFunction{𝟎-IsEquivalence}\<%
\end{code}
\ccpad
A concrete example is \af{⟪𝟎⟫}~\ab 𝑨~\af ╱~\ab θ, presented in the next subsection.

\subsubsection{Quotient Algebras}\label{quotient-algebras}
In many areas of abstract mathematics the \defn{quotient} of an algebra \ab 𝑨 with respect to a congruence relation \ab θ of \ab 𝑨 plays an important role. This quotient is typically denoted by \ab 𝑨 \af / \ab θ and Agda allows us to define and express quotients using this standard notation.\footnote{%
\textbf{Unicode Hints}. Produce the ╱ symbol in \agdamode by typing \texttt{\textbackslash{}-\/-\/-} and then \texttt{C-f} a number of times.}
\ccpad
\begin{code}%
\>[0][@{}l@{\AgdaIndent{0}}]%
\>[1]\AgdaOperator{\AgdaFunction{\AgdaUnderscore{}╱\AgdaUnderscore{}}}\AgdaSpace{}%
\AgdaSymbol{:}\AgdaSpace{}%
\AgdaSymbol{(}\AgdaBound{𝑨}\AgdaSpace{}%
\AgdaSymbol{:}\AgdaSpace{}%
\AgdaFunction{Algebra}\AgdaSpace{}%
\AgdaBound{𝓤}\AgdaSpace{}%
\AgdaBound{𝑆}\AgdaSymbol{)}\AgdaSpace{}%
\AgdaSymbol{→}\AgdaSpace{}%
\AgdaRecord{Congruence}\AgdaSymbol{\{}\AgdaBound{𝓦}\AgdaSymbol{\}}\AgdaSpace{}%
\AgdaBound{𝑨}\AgdaSpace{}%
\AgdaSymbol{→}\AgdaSpace{}%
\AgdaFunction{Algebra}\AgdaSpace{}%
\AgdaSymbol{(}\AgdaBound{𝓤}\AgdaSpace{}%
\AgdaOperator{\AgdaFunction{⊔}}\AgdaSpace{}%
\AgdaBound{𝓦}\AgdaSpace{}%
\AgdaOperator{\AgdaFunction{⁺}}\AgdaSymbol{)}\AgdaSpace{}%
\AgdaBound{𝑆}\<%
\\
%
\\[\AgdaEmptyExtraSkip]%
%
\>[1]\AgdaBound{𝑨}\AgdaSpace{}%
\AgdaOperator{\AgdaFunction{╱}}\AgdaSpace{}%
\AgdaBound{θ}\AgdaSpace{}%
\AgdaSymbol{=}%
\>[258I]\AgdaSymbol{(}\AgdaSpace{}%
\AgdaOperator{\AgdaFunction{∣}}\AgdaSpace{}%
\AgdaBound{𝑨}\AgdaSpace{}%
\AgdaOperator{\AgdaFunction{∣}}\AgdaSpace{}%
\AgdaOperator{\AgdaFunction{/}}\AgdaSpace{}%
\AgdaOperator{\AgdaField{⟨}}\AgdaSpace{}%
\AgdaBound{θ}\AgdaSpace{}%
\AgdaOperator{\AgdaField{⟩}}\AgdaSpace{}%
\AgdaSymbol{)}\AgdaSpace{}%
\AgdaOperator{\AgdaInductiveConstructor{,}}%
\>[49]\AgdaComment{-- the domain of the quotient algebra}\<%
\\
%
\\[\AgdaEmptyExtraSkip]%
\>[.][@{}l@{}]\<[258I]%
\>[9]\AgdaSymbol{λ}\AgdaSpace{}%
\AgdaBound{𝑓}\AgdaSpace{}%
\AgdaBound{𝒂}\AgdaSpace{}%
\AgdaSymbol{→}\AgdaSpace{}%
\AgdaOperator{\AgdaFunction{⟪}}\AgdaSymbol{(}\AgdaBound{𝑓}\AgdaSpace{}%
\AgdaOperator{\AgdaFunction{̂}}\AgdaSpace{}%
\AgdaBound{𝑨}\AgdaSymbol{)}\AgdaSpace{}%
\AgdaSymbol{(λ}\AgdaSpace{}%
\AgdaBound{i}\AgdaSpace{}%
\AgdaSymbol{→}\AgdaSpace{}%
\AgdaOperator{\AgdaFunction{∣}}\AgdaSpace{}%
\AgdaOperator{\AgdaFunction{∥}}\AgdaSpace{}%
\AgdaBound{𝒂}\AgdaSpace{}%
\AgdaBound{i}\AgdaSpace{}%
\AgdaOperator{\AgdaFunction{∥}}\AgdaSpace{}%
\AgdaOperator{\AgdaFunction{∣}}\AgdaSymbol{)}\AgdaOperator{\AgdaFunction{⟫}}%
\>[49]\AgdaComment{-- the basic operations of the quotient algebra}\<%
\end{code}
\ccpad
\textbf{Example}.
Denote by \af{𝟎[} \ab 𝑨~\af ╱~\ab θ~\af ] the zero relation on the quotient algebra \ab 𝑨~\af ╱~\ab θ, which is defined as follows.
\ccpad
\begin{code}%
\>[0][@{}l@{\AgdaIndent{0}}]%
\>[1]\AgdaOperator{\AgdaFunction{𝟎[\AgdaUnderscore{}╱\AgdaUnderscore{}]}}\AgdaSpace{}%
\AgdaSymbol{:}\AgdaSpace{}%
\AgdaSymbol{(}\AgdaBound{𝑨}\AgdaSpace{}%
\AgdaSymbol{:}\AgdaSpace{}%
\AgdaFunction{Algebra}\AgdaSpace{}%
\AgdaBound{𝓤}\AgdaSpace{}%
\AgdaBound{𝑆}\AgdaSymbol{)(}\AgdaBound{θ}\AgdaSpace{}%
\AgdaSymbol{:}\AgdaSpace{}%
\AgdaRecord{Congruence}\AgdaSymbol{\{}\AgdaBound{𝓦}\AgdaSymbol{\}}\AgdaSpace{}%
\AgdaBound{𝑨}\AgdaSymbol{)}\AgdaSpace{}%
\AgdaSymbol{→}\AgdaSpace{}%
\AgdaFunction{Rel}\AgdaSpace{}%
\AgdaSymbol{(}\AgdaOperator{\AgdaFunction{∣}}\AgdaSpace{}%
\AgdaBound{𝑨}\AgdaSpace{}%
\AgdaOperator{\AgdaFunction{∣}}\AgdaSpace{}%
\AgdaOperator{\AgdaFunction{/}}\AgdaSpace{}%
\AgdaOperator{\AgdaField{⟨}}\AgdaSpace{}%
\AgdaBound{θ}\AgdaSpace{}%
\AgdaOperator{\AgdaField{⟩}}\AgdaSymbol{)(}\AgdaBound{𝓤}\AgdaSpace{}%
\AgdaOperator{\AgdaFunction{⊔}}\AgdaSpace{}%
\AgdaBound{𝓦}\AgdaSpace{}%
\AgdaOperator{\AgdaFunction{⁺}}\AgdaSymbol{)}\<%
\\
%
\>[1]\AgdaOperator{\AgdaFunction{𝟎[}}\AgdaSpace{}%
\AgdaBound{𝑨}\AgdaSpace{}%
\AgdaOperator{\AgdaFunction{╱}}\AgdaSpace{}%
\AgdaBound{θ}\AgdaSpace{}%
\AgdaOperator{\AgdaFunction{]}}\AgdaSpace{}%
\AgdaSymbol{=}\AgdaSpace{}%
\AgdaSymbol{λ}\AgdaSpace{}%
\AgdaBound{u}\AgdaSpace{}%
\AgdaBound{v}\AgdaSpace{}%
\AgdaSymbol{→}\AgdaSpace{}%
\AgdaBound{u}\AgdaSpace{}%
\AgdaOperator{\AgdaDatatype{≡}}\AgdaSpace{}%
\AgdaBound{v}\<%
\end{code}
\ccpad
From this we obtain the zero congruence of \ab 𝑨~\af ╱~\ab θ by applying the \af Δ function defined above.
\ccpad
\begin{code}%
\>[0][@{}l@{\AgdaIndent{0}}]%
\>[1]\AgdaOperator{\AgdaFunction{⟪𝟎⟫\AgdaUnderscore{}╱\AgdaUnderscore{}}}\AgdaSpace{}%
\AgdaSymbol{:}\AgdaSpace{}%
\>[259I]\AgdaSymbol{(}\AgdaBound{𝑨}\AgdaSpace{}%
\AgdaSymbol{:}\AgdaSpace{}%
\AgdaFunction{Algebra}\AgdaSpace{}%
\AgdaBound{𝓤}\AgdaSpace{}%
\AgdaBound{𝑆}\AgdaSymbol{)(}\AgdaBound{θ}\AgdaSpace{}%
\AgdaSymbol{:}\AgdaSpace{}%
\AgdaRecord{Congruence}\AgdaSymbol{\{}\AgdaBound{𝓦}\AgdaSymbol{\}}\AgdaSpace{}%
\AgdaBound{𝑨}\AgdaSymbol{)\{}\AgdaBound{fe}\AgdaSpace{}%
\AgdaSymbol{:}\AgdaSpace{}%
\AgdaFunction{funext}\AgdaSpace{}%
\AgdaBound{𝓥}\AgdaSpace{}%
\AgdaSymbol{(}\AgdaBound{𝓤}\AgdaSpace{}%
\AgdaOperator{\AgdaFunction{⊔}}\AgdaSpace{}%
\AgdaBound{𝓦}\AgdaSpace{}%
\AgdaOperator{\AgdaFunction{⁺}}\AgdaSymbol{)\}}\<%
\\
\>[1][@{}l@{\AgdaIndent{0}}]%
\>[2]\AgdaSymbol{→}%
\>[.][@{}l@{}]\<[259I]%
\>[14]\AgdaRecord{Congruence}\AgdaSpace{}%
\AgdaSymbol{(}\AgdaBound{𝑨}\AgdaSpace{}%
\AgdaOperator{\AgdaFunction{╱}}\AgdaSpace{}%
\AgdaBound{θ}\AgdaSymbol{)}\<%
\\
%
\>[1]\AgdaSymbol{(}\AgdaOperator{\AgdaFunction{⟪𝟎⟫}}\AgdaSpace{}%
\AgdaBound{𝑨}\AgdaSpace{}%
\AgdaOperator{\AgdaFunction{╱}}\AgdaSpace{}%
\AgdaBound{θ}\AgdaSymbol{)}\AgdaSpace{}%
\AgdaSymbol{\{}\AgdaBound{fe}\AgdaSymbol{\}}\AgdaSpace{}%
\AgdaSymbol{=}\AgdaSpace{}%
\AgdaFunction{Δ}\AgdaSpace{}%
\AgdaBound{fe}\<%
\end{code}
\ccpad
Finally, the following \defn{elimination rule} is sometimes useful.
\ccpad
\begin{code}%
% \>[0]\AgdaKeyword{module}\AgdaSpace{}%
% \AgdaModule{\AgdaUnderscore{}}\AgdaSpace{}%
% \AgdaSymbol{\{}\AgdaBound{𝓤}\AgdaSpace{}%
% \AgdaBound{𝓦}\AgdaSpace{}%
% \AgdaSymbol{:}\AgdaSpace{}%
% \AgdaFunction{Universe}\AgdaSymbol{\}\{}\AgdaBound{𝑨}\AgdaSpace{}%
% \AgdaSymbol{:}\AgdaSpace{}%
% \AgdaFunction{Algebra}\AgdaSpace{}%
% \AgdaBound{𝓤}\AgdaSpace{}%
% \AgdaBound{𝑆}\AgdaSymbol{\}}\AgdaSpace{}%
% \AgdaKeyword{where}\<%
% \\
% %
% \\[\AgdaEmptyExtraSkip]%
\>[0][@{}l@{\AgdaIndent{0}}]%
\>[1]\AgdaFunction{╱-≡}\AgdaSpace{}%
\AgdaSymbol{:}\AgdaSpace{}%
\AgdaSymbol{(}\AgdaBound{θ}\AgdaSpace{}%
\AgdaSymbol{:}\AgdaSpace{}%
\AgdaRecord{Congruence}\AgdaSymbol{\{}\AgdaBound{𝓦}\AgdaSymbol{\}}\AgdaSpace{}%
\AgdaBound{𝑨}\AgdaSymbol{)\{}\AgdaBound{u}\AgdaSpace{}%
\AgdaBound{v}\AgdaSpace{}%
\AgdaSymbol{:}\AgdaSpace{}%
\AgdaOperator{\AgdaFunction{∣}}\AgdaSpace{}%
\AgdaBound{𝑨}\AgdaSpace{}%
\AgdaOperator{\AgdaFunction{∣}}\AgdaSymbol{\}}\AgdaSpace{}%
\AgdaSymbol{→}\AgdaSpace{}%
\AgdaOperator{\AgdaFunction{⟪}}\AgdaSpace{}%
\AgdaBound{u}\AgdaSpace{}%
\AgdaOperator{\AgdaFunction{⟫}}\AgdaSymbol{\{}\AgdaOperator{\AgdaField{⟨}}\AgdaSpace{}%
\AgdaBound{θ}\AgdaSpace{}%
\AgdaOperator{\AgdaField{⟩}}\AgdaSymbol{\}}\AgdaSpace{}%
\AgdaOperator{\AgdaDatatype{≡}}\AgdaSpace{}%
\AgdaOperator{\AgdaFunction{⟪}}\AgdaSpace{}%
\AgdaBound{v}\AgdaSpace{}%
\AgdaOperator{\AgdaFunction{⟫}}\AgdaSpace{}%
\AgdaSymbol{→}\AgdaSpace{}%
\AgdaOperator{\AgdaField{⟨}}\AgdaSpace{}%
\AgdaBound{θ}\AgdaSpace{}%
\AgdaOperator{\AgdaField{⟩}}\AgdaSpace{}%
\AgdaBound{u}\AgdaSpace{}%
\AgdaBound{v}\<%
\\
%
\>[1]\AgdaFunction{╱-≡}\AgdaSpace{}%
\AgdaBound{θ}\AgdaSpace{}%
\AgdaInductiveConstructor{refl}\AgdaSpace{}%
\AgdaSymbol{=}\AgdaSpace{}%
\AgdaField{IsEquivalence.rfl}\AgdaSpace{}%
\AgdaSymbol{(}\AgdaField{IsEquiv}\AgdaSpace{}%
\AgdaBound{θ}\AgdaSymbol{)}\AgdaSpace{}%
\AgdaSymbol{\AgdaUnderscore{}}\<%
\end{code}


% % -*- TeX-master: "ualib-part1.tex" -*-
%%% Local Variables: 
%%% mode: latex
%%% TeX-engine: 'xetex
%%% End:
\section{Concluding Remarks}\label{sec:concluding-remarks}
We've reached the end of Part 1 of our three-part series describing the \agdaualib.  Part 2 will cover homomorphism, terms, and subalgebras, and Part 3 will cover free algebras, equational classes of algebras (i.e., varieties), and Birkhoff's HSP theorem.

% We completed the first major milestone of this project in January 2021, which was the first formal, machine-checked proof of Birkhoff's HSP theorem.  Our next goal is to support current mathematical research by formalizing some recently proved theorems. For example, we intend to formalize theorems about the computational complexity of decidable properties of algebraic structures. Part of this effort will naturally involve further work on the library itself, and may lead to new observations or discoveries in dependent type theory or homotopy type theory.

%% One natural question is whether there are any objects of our research that cannot be represented constructively in type theory. % and, if so, whether these have suitable constructive substitutes.
%% %Along these lines, %% we should revisit and perhaps refine our proof of Birkhoff's theorem. A
%% As mentioned in \S~\ref{sec:contributions}, a constructive version of Birkhoff's theorem was presented by Carlstr\"om in~\cite{Carlstrom:2008}. We would like to know, for example, how the two new hypotheses required by Carlstr\"om %% adds to the classical theorem in order to make it constructive 
%% compare with the assumptions we make in our Agda proof of this result.

We conclude by noting that one of our goals is to make computer formalization of mathematics more accessible to mathematicians working in universal algebra and model theory. We welcome feedback from the community and are happy to field questions about the \ualib, how it is installed, and how it can be used to prove theorems that are not yet part of the library.  Merge requests submitted to the UALib's main gitlab repository are especially welcomed.  Please visit the repository at \url{https://gitlab.com/ualib/ualib.gitlab.io/} and help us improve it.



% % -*- TeX-master: "ualib-part1.tex" -*-
%%% Local Variables: 
%%% mode: latex
%%% TeX-engine: 'xetex
%%% End:
\section{Introduction}\label{sec:introduction}
To support formalization in type theory of research level mathematics in universal algebra and related fields, we present the Agda Universal Algebra Library (\agdaualib), a software library containing formal statements and proofs of the core definitions and results of universal algebra. The \ualib is written in \agda~\cite{Norell:2009}, a programming language and proof assistant based on \MLTT (\mltt) that supports dependent and inductive types.


\subsection{Motivation}\label{sec:motivation}
The seminal idea for the \agdaualib project was the observation that, on the one hand, a number of fundamental constructions in universal algebra can be defined recursively, and theorems about them proved by structural induction, while, on the other hand, inductive and dependent types make possible very precise formal representations of recursively defined objects, which often admit elegant constructive proofs of properties of such objects.  An important feature of such proofs in type theory is that they are total functional programs and, as such, they are computable, composable, and machine-verifiable.
% These observations suggested that there would be much to gain from implementing universal algebra in a language, such as Martin-L\"of type theory, that features dependent and inductive types.

Finally, our own research experience has taught us that a proof assistant and programming language (like Agda), when equipped with specialized libraries and domain-specific tactics to automate the proof idioms of our field, can be an extremely powerful and effective asset. As such we believe that proof assistants and their supporting libraries will eventually become indispensable tools in the working mathematician's toolkit.

\subsection{Attributions and Contributions}\label{sec:contributions}
The mathematical results described in this paper have well known \emph{informal} proofs. Our main contribution is the formalization, mechanization, and verification of the statements and proofs of these results in dependent type theory using Agda.

Unless explicitly stated otherwise, the Agda source code described in this paper is due to the author, with the following caveat: the \ualib depends on the \typetopology library of \MartinEscardo~\cite{MHE}. For convenience, we refer to Escard\'o's library as \typtop throughout the paper. For the sake of completeness and clarity, and to keep the paper mostly self-contained, we repeat some definitions from \typtop, but in each instance we cite the original source.\footnote{In the \ualib, such instances occur only inside hidden modules that are never actually used, followed immediately by a statement that imports the code in question from its original source.}


\subsection{Prior art}\label{sec:prior-art}
There have been a number of efforts to formalize parts of universal algebra in type theory prior to ours, most notably
\begin{itemize}
  \item Capretta~\cite{Capretta:1999} (1999) formalized the basics of universal algebra in the Calculus of Inductive Constructions using the Coq proof assistant;
    \item Spitters and van der Weegen~\cite{Spitters:2011} (2011) formalized the basics of universal algebra and some classical algebraic structures, also in the Calculus of Inductive Constructions using the Coq proof assistant, promoting the use of type classes;
 \item Gunther, et al~\cite{Gunther:2018} (2018) developed what seems to be (prior to the \ualib) the most extensive library of formal universal algebra to date; in particular, this work includes a formalization of some basic equational logic; also (unlike the \ualib) it handles \emph{multisorted} algebraic structures; (like the \ualib) it is based on dependent type theory and the Agda proof assistant.
\end{itemize}
Some other projects aimed at formalizing mathematics generally, and algebra in particular, have developed into very extensive libraries that include definitions, theorems, and proofs about algebraic structures, such as groups, rings, modules, etc.  However, the goals of these efforts seem to be the formalization of special classical algebraic structures, as opposed to the general theory of (universal) algebras. Moreover, the part of universal algebra and equational logic formalized in the \ualib extends beyond the scope of prior efforts. In particular, the library now includes a proof of Birkhoff's variety theorem.  Most other proofs of this theorem that we know of are informal and nonconstructive.\footnote{After completing the formal proof in \agda, we learned about a constructive version of Birkhoff's theorem proved by Carlstr\"om in~\cite{Carlstrom:2008}.  The latter is presented in the informal style of standard mathematical writing, and as far as we know it was never formalized in type theory and type-checked with a proof assistant. Nonetheless, a comparison of Carlstr\"om's proof and the \ualib proof would be interesting.}
 %% We remark briefly on this in~\S\ref{sec:conclusions-and-future-work}.}


\subsection{Organization of the paper}\label{sec:organization}

In this paper we limit ourselves to the presentation of the core foundational modules of the \ualib so that we have space to discuss some of the more interesting type theoretic and foundational issues that arose when developing the library and attempting to represent advanced mathematical notions in type theory and formalize them in Agda.  This is the first in a series of three papers describing the \agdaualib. The second paper (\cite{DeMeo:2021-2}) covers \emph{homomorphisms}, \emph{terms}, and \emph{subalgebras}. The third paper (\cite{DeMeo:2021-3}) covers \emph{free algebras}, \emph{equational classes} of algebras (i.e., \emph{varieties}), and \emph{Birkhoff's HSP theorem}.

This present paper is organized into three parts as follows. The first part is~\S\ref{sec:overture} which introduces the basic concepts of type theory with special emphasis on the way such concepts are formalized in \agda. Specifically, \S\ref{preliminaries} introduces \emph{Sigma types} and Agda's hierarchy of \emph{universes}. The important topics of \emph{equality} and \emph{function extensionality} are discussed in \S\ref{equality} and \S\ref{function-extensionality}; \S\ref{sec:inverse-image-invers} covers inverses and inverse images of functions. In \S\ref{sec:lifts-altern-univ} we describe a technical problem that one frequently encounters when working in a \emph{noncumulative universe hierarchy} and offer some tools for resolving the type-checking errors that arise from this.

The second part is~\S\ref{sec:relat-quot-types} which covers \emph{relation types} and \emph{quotient types}. Specifically,~\S\ref{sec:discrete-relations} defines types that represent \emph{unary} and \emph{binary relations} as well as \emph{function kernels}. These ``discrete relation types,'' are all very standard.  In~\S\ref{sec:continuous-relations} we introduce the (less standard) types that we use to represent \emph{general} and \emph{dependent relations}. We call these ``continuous relations'' because they can have arbitrary arity (general relations) and they can be defined over arbitrary families of types (dependent relations).
In~\S\ref{sec:equiv-relat-quot} we cover standard types for equivalence relations and quotients, and in~\S\ref{sec:trunc-sets-prop} we discuss a family of concepts that are vital to the mechanization of mathematics using type theory; these are the closely related concepts of \emph{truncation}, \emph{sets}, \emph{propositions}, and \emph{proposition extensionality}.

The third part of the paper is~\S\ref{sec:algebra-types} which covers the basic domain-specific types offered by the \ualib. It is here that we finally get to see some types representing algebraic structures.  Specifically, we describe types for \emph{operations} and \emph{signatures}~(\S\ref{sec:oper-sign}), \emph{general algebras}~(\S\ref{sec:algebras}), and \emph{product algebras}~(\S\ref{sec:product-algebras}), including types for representing \emph{products over arbitrary classes of algebraic structures}. Finally, we define types for congruence relations and quotient algebras in~\S\ref{congruences}.

\newcommand\otherparta{\textit{Part 2: homomorphisms, terms, and subalgebras}~\cite{DeMeo:2021-2}.}
\newcommand\otherpartb{\textit{Part 3: free algebras, equational classes, and Birkhoff's theorem}~\cite{DeMeo:2021-3}.}

\subsection{Resources}
We conclude this introduction with some pointers to helpful reference materials.
For the required background in Universal Algebra, we recommend the textbook by Clifford Bergman~\cite{Bergman:2012}.  For the type theory background, we recommend the HoTT Book~\cite{HoTT} and Escard\'o's \ufcourse~\cite{MHE}.

The following are informed the development of the \ualib and are highly recommended.
\begin{itemize}
\item \textit{\ufcourse}, \escardo~\cite{MHE}
\item \href{http://www.cse.chalmers.se/~peterd/papers/DependentTypesAtWork.pdf}{\it Dependent Types at Work}, Bove and Dybjer~\cite{Bove:2009}
\item \href{http://www.cse.chalmers.se/~ulfn/papers/afp08/tutorial.pdf}{\it Dependently Typed Programming in Agda}, Norell and Chapman~\cite{Norell:2008}
\item \href{http://www.sciencedirect.com/science/article/pii/S1571066118300768}{\it Formalization of Universal Algebra in Agda}, Gunther, Gadea, Pagano~\cite{Gunther:2018}
\item \href{https://plfa.github.io/}{\it Programming Languages Foundations in Agda}~\cite{Wadler:2020}
\end{itemize}

More information about \agdaualib can be obtained from the following official sources.
\begin{itemize}
  \item \href{https://ualib.gitlab.io}{\texttt{ualib.org}} (the web site) documents every line of code in the library.
  \item \href{https://gitlab.com/ualib/ualib.gitlab.io}{\texttt{gitlab.com/ualib/ualib.gitlab.io}} (the source code) \agdaualib is open source.\footnote{License: \href{https://creativecommons.org/licenses/by-sa/4.0/}{Creative Commons Attribution-ShareAlike 4.0 International License.}}
  \item \emph{The Agda UALib}, \otherparta
  \item \emph{The Agda UALib}, \otherpartb
\end{itemize}
The first item links to the official \ualib html documentation which includes complete proofs of every theorem we mention here, and much more, including the Agda modules covered in the first and third installments of this series of papers on the \ualib.

Finally, readers will get much more out of the paper if they download \agdaualib from \url{https://gitlab.com/ualib/ualib.gitlab.io}, install the library, and try it out for themselves.

% % -*- TeX-master: "ualib-part1.tex" -*-
%%% Local Variables: 
%%% mode: latex
%%% TeX-engine: 'xetex
%%% End:
\section{Overture}\label{sec:overture}

\subsection{Preliminaries}\label{preliminaries}
% : logical foundations, universes, dependent types
\firstsentence{Overture}{Preliminaries}
% -*- TeX-master: "ualib-part1.tex" -*-
%%% Local Variables: 
%%% mode: latex
%%% TeX-engine: 'xetex
%%% End:
This module defines (or imports) the most basic and important types of \defn{Martin-Löf dependent type theory} (MLTT).  Although this is standard, we take this opportunity to highlight aspects of the \ualib syntax that may differ from that of ``standard Agda.''  Although we don't discuss all of the types on which the library depends here, Appendix Section~\ref{sec:imports-from-type} provides a comprehensive list of the components from \MartinEscardo's \typetopology library~\cite{MHE} that are imported at one place or another in the \ualib.

\subsubsection{Logical foundations}\label{sec:logical-foundations}
% Mainly for the benefit of readers who are not yet proficient in Agda or type theory, we briefly describe the type theoretic foundations of the \ualib, including the most important basic types and features that are used throughout the library. 
The \ualib is based on a minimal version of \mltt that is the same or very close to the type theory on which \MartinEscardo's \TypeTopology Agda library is based.
% This is also the type theory that \escardo taught us in the short course \ufcourse at the Midlands Graduate School in the Foundations of Computing Science at University of Birmingham in 2019.}
We won't go into great detail here because there are already other very nice resources available, such as the section \href{https://www.cs.bham.ac.uk/~mhe/HoTT-UF-in-Agda-Lecture-Notes/HoTT-UF-Agda.html\#mlttinagda}{A spartan Martin-Löf type theory} of the lecture notes by \escardo just mentioned, the \href{https://ncatlab.org/nlab/show/Martin-L\%C3\%B6f+dependent+type+theory}{ncatlab entry on Martin-Löf dependent type theory}, as well as the HoTT Book~\cite{HoTT}.

We begin by noting that only a very small collection of objects is assumed at the jumping-off point for MLTT. We have the \defn{primitive types} (\ad 𝟘, \ad 𝟙, and \ad ℕ, denoting the empty type, one-element type, and natural numbers), the \defn{type formers} (\ad +, \ad Π, \ad Σ, \ad{Id}, denoting \defn{binary sum}, \defn{product}, \defn{sum}, and the \defn{identity} type), and an infinite collection of \defn{type universes} (types of types) and universe variables to denote them.  Like Escard\'o's, our universe variables are typically upper-case caligraphic letters from the latter half of the English alphabet (e.g., \ab 𝓤, \ab 𝓥, \ab 𝓦, etc.).

\paragraph*{Specifying logical foundations in Agda}
An Agda program typically begins by setting some options and by importing types from existing Agda libraries.
Options are specified with the \AgdaKeyword{OPTIONS} \emph{pragma} and control the way Agda behaves by, for example, specifying the logical axioms and deduction rules we wish to assume when the program is type-checked to verify its correctness. Every Agda program in the \ualib begins with the following line.
\ccpad
\begin{code}[number=code:options]
\>[0]\AgdaSymbol{\{-\#}\AgdaSpace{}%
\AgdaKeyword{OPTIONS}\AgdaSpace{}%
\AgdaPragma{--without-K}\AgdaSpace{}%
\AgdaPragma{--exact-split}\AgdaSpace{}%
\AgdaPragma{--safe}\AgdaSpace{}%
\AgdaSymbol{\#-\}}\<%
\end{code}
\ccpad
These options control certain foundational assumptions that Agda makes when type-checking the program to verify its correctness.
\begin{itemize}
\item \AgdaPragma{--without-K} disables \axiomk; see~\cite{agdaref-axiomk};
\item \AgdaPragma{--exact-split} makes Agda accept only definitions that are \emph{judgmental} equalities; see~\cite{agdatools-patternmatching};
\item \AgdaPragma{--safe} ensures that nothing is postulated outright---every non-MLTT axiom has to be an explicit assumption (e.g., an argument to a function or module); see~\cite{agdaref-safeagda} and~\cite{agdatools-patternmatching}.
\end{itemize}
%% \end{enumerate}
Throughout this paper we take assumptions 1--3 for granted without mentioning them explicitly.


\paragraph*{Agda Modules}
The \ak{OPTIONS} pragma is usually followed by the start of a module.  For example, the \ualibhtml{Prelude.Preliminaries} module begins with the following line.
\ccpad
\begin{code}%
\>[0]\AgdaKeyword{module}\AgdaSpace{}%
\AgdaModule{Prelude.Preliminaries}\AgdaSpace{}%
\AgdaKeyword{where}\<%
\end{code}
\ccpad
Sometimes we want to declare parameters that will be assumed throughout the module.  For instance, when working with algebras, we often assume they come from a particular fixed signature, and this signature is something we could fix as a parameter at the start of a module. Thus, we might start an \defn{anonymous submodule} of the main module with a line like\footnote{The \af{Signature} type will be defined in Section~\ref{sec:oper-sign}.}
\ccpad
\begin{code}%
\>[0]\AgdaKeyword{module}\AgdaSpace{}%
\AgdaUnderscore\AgdaSpace{}%
\AgdaSymbol{\{}\AgdaBound{𝑆}\AgdaSpace{}%
\AgdaSymbol{:}\AgdaSpace{}%
\AgdaFunction{Signature}\AgdaSpace{}%
\AgdaBound{𝓞}\AgdaSpace{}%
\AgdaBound{𝓥}\AgdaSymbol{\}}\AgdaSpace{}%
\AgdaKeyword{where}\<%
\end{code}
\ccpad
Such a module is called \emph{anonymous} because an underscore appears in place of a module name. Agda determines where a submodule ends by indentation.  This can take some getting used to, but after a short time it will feel very natural. The main module of a file must have the same name as the file, without the \texttt{.agda} or \texttt{.lagda} file extension.  The code inside the main module is not indented. Submodules are declared inside the main module and code inside these submodules must be indented to a fixed column.  As long as the code is indented, Agda considers it part of the submodule.  A submodule is exited as soon as a nonindented line of code appears.




\paragraph*{Agda's universe hierarchy}\label{ssec:agdas-universe-hierarchy}
The \agdaualib adopts the notation of Martin Escardo's \typetopology library~\cite{MHE}. In particular, \emph{universe levels}%
\footnote{See \url{https://agda.readthedocs.io/en/v2.6.1.2/language/universe-levels.html}.}
are denoted by capitalized script letters from the second half of the alphabet, e.g., \ab 𝓤, \ab 𝓥, \ab 𝓦, etc.  Also defined in \typetopology are the operators~\af ̇~and~\af ⁺. These map a universe level \ab 𝓤 to the universe \ab 𝓤\af ̇ := \Set \ab 𝓤 and the level \ab 𝓤 \af ⁺ \aod := \lsuc \ab 𝓤, respectively.  Thus, \ab 𝓤\af ̇ is simply an alias for the universe \Set \ab 𝓤, and we have \ab 𝓤\af ̇ \as : \ab 𝓤 \af ⁺\af ̇.
%% Table~\ref{tab:dictionary} translates between standard \agda syntax and \typetopology/\ualib notation.

To justify the introduction of this somewhat nonstandard notation for universe levels, \escardo points out that the Agda library uses \ak{Level} for universes (so what we write as \ab 𝓤\af ̇ is written \ak{Set}~\ab 𝓤 in standard Agda), but in univalent mathematics the types in \ab 𝓤\af ̇ need not be sets, so the standard Agda notation can be a bit confusing, especially to new users.


The hierarchy of universes in Agda is structured as \ab{𝓤}\af ̇ \as : \ab 𝓤 \af ⁺\af ̇, \hskip3mm
\ab{𝓤} \af ⁺\af ̇ \as : \ab 𝓤 \af ⁺\af ⁺\af ̇, etc. This means that the universe \ab 𝓤\af ̇ has type \ab 𝓤  \af ⁺\af ̇, and 𝓤 \af ⁺\af ̇ has type \ab 𝓤 \af ⁺\af ⁺\af ̇, and so on.  It is important to note, however, this does \emph{not} imply that \ab 𝓤\af ̇ \as : \ab 𝓤 \af ⁺\af ⁺\af ̇. In other words, Agda's universe hierarchy is \emph{noncummulative}. This makes it possible to treat universe levels more generally and precisely, which is nice. On the other hand, a noncummulative hierarchy can sometimes make for a nonfun proof assistant. Luckily, there are ways to circumvent noncummulativity without introducing logical inconsistencies into the type theory. Section~\ref{lifts-of-algebras} describes some domain-specific tools that we developed for this purpose.








\subsubsection{Dependent types}\label{sec:dependent-types}

\newcommand\FstUnder{\AgdaOperator{\AgdaFunction{∣\AgdaUnderscore{}∣}}\xspace}
\newcommand\SndUnder{\AgdaOperator{\AgdaFunction{∥\AgdaUnderscore{}∥}}\xspace}
% Convenient notations for the first and second projections out of a product are \FstUnder and \SndUnder, respectively. However, to improve readability or to avoid notation clashes with other modules, we sometimes use more standard alternatives, such as \AgdaFunction{pr₁} and \AgdaFunction{pr₂}, or \AgdaFunction{fst} and \AgdaFunction{snd}, or some combination of these. The definitions are standard so we omit them.\footnote{For details see~\cite{DeMeo:2021} or \ualiburl{Prelude.Preliminaries}.}








\paragraph*{Sigma types (dependent pairs)} %\label{ssec:dependent-pairs}

Given universes \ab 𝓤 and \ab 𝓥, a type \ab{X} \as : \ab 𝓤\af ̇, and a type family \ab{Y}~\as :~\ab X~\as →~\ab 𝓥\af ̇, the \defn{Sigma type} (aka \defn{dependent pair type}, aka \defn{dependent product type}) is denoted by \ar{Σ}(\ab x~\af ꞉~\ab X),~\ab Y~\ab x and generalizes the Cartesian product \ab{X} \af × \ab Y by allowing the type \ab{Y x} of the second argument of the ordered pair (\ab x \af , \ab y) to depend on the value \ab{x} of the first. That is, an inhabitant of the type \ar{Σ}(\ab x~\af ꞉~\ab X),~\ab Y~\ab x is a pair (\ab{x}~\af ,~\ab y) such that \abt{x}{X}
and \abt{y}{Y x}.

Agda's default syntax for a Sigma type is \AgdaRecord{Σ}\sP{3}\AgdaSymbol{λ}(\ab x\sP{3}꞉\sP{3}\ab X)\sP{3}\as →\sP{3}\ab Y, but we prefer the notation \AgdaRecord{Σ}~\ab x~꞉~\ab X~,~\ab Y, which is closer to the standard syntax described in the preceding paragraph. Fortunately, this preferred notation is available in the \typetopology library (see~\cite[Σ types]{MHE}).\footnote{\textbf{WARNING!} The symbol \as ꞉ in the expression \AgdaRecord{Σ}\sP{3}\ab x\sP{3}\as ꞉\sP{3}\ab X\sP{3}\AgdaComma\sP{3}\ab Y is not the ordinary colon; rather, it is the symbol obtained by typing \texttt{\textbackslash{}:4} in \agdatwomode.} For pedagogical purposes we repeat this definition here, inside a \defn{hidden module} so that it doesn't conflict with the original definition that we will import later.\footnote{To hide code from the rest of the development, we enclose it in a named module.  For example, the code inside the \af{hide-sigma} module will not conflict with the original definitions from the \typetopology library, even though we import the latter right after presenting the definition.  As long as we don't invoke \ak{open} \af{hide-sigma}, the code inside the \af{hide-sigma} module remains essentially hidden (though Agda will type-check the code for us). It may seem odd to both define things in the hidden module only to immediately import the definition that we actually use, but we do this in an attempt to exhibit all of the types on which the \ualib depends, in a clear and self-contained way, while also ensuring that this cannot be misinterpreted as a claim to originality.}
\ccpad
\begin{code}%
\>[0]\AgdaKeyword{module}\AgdaSpace{}%
\AgdaModule{hide-sigma}\AgdaSpace{}%
\AgdaKeyword{where}\<%
\\
%
\\[\AgdaEmptyExtraSkip]%
\>[0][@{}l@{\AgdaIndent{0}}]%
\>[1]\AgdaKeyword{record}\AgdaSpace{}%
\AgdaRecord{Σ}\AgdaSpace{}%
\AgdaSymbol{\{}\AgdaBound{𝓤}\AgdaSpace{}%
\AgdaBound{𝓥}\AgdaSymbol{\}}\AgdaSpace{}%
\AgdaSymbol{\{}\AgdaBound{X}\AgdaSpace{}%
\AgdaSymbol{:}\AgdaSpace{}%
\AgdaBound{𝓤}\AgdaSpace{}%
\AgdaOperator{\AgdaFunction{̇}}\AgdaSpace{}%
\AgdaSymbol{\}}\AgdaSpace{}%
\AgdaSymbol{(}\AgdaBound{Y}\AgdaSpace{}%
\AgdaSymbol{:}\AgdaSpace{}%
\AgdaBound{X}\AgdaSpace{}%
\AgdaSymbol{→}\AgdaSpace{}%
\AgdaBound{𝓥}\AgdaSpace{}%
\AgdaOperator{\AgdaFunction{̇}}\AgdaSpace{}%
\AgdaSymbol{)}\AgdaSpace{}%
\AgdaSymbol{:}\AgdaSpace{}%
\AgdaBound{𝓤}\AgdaSpace{}%
\AgdaOperator{\AgdaPrimitive{⊔}}\AgdaSpace{}%
\AgdaBound{𝓥}\AgdaSpace{}%
\AgdaOperator{\AgdaFunction{̇}}%
\>[53]\AgdaKeyword{where}\<%
\\
\>[1][@{}l@{\AgdaIndent{0}}]%
\>[2]\AgdaKeyword{constructor}\AgdaSpace{}%
\AgdaOperator{\AgdaInductiveConstructor{\AgdaUnderscore{},\AgdaUnderscore{}}}\<%
\\
%
\>[2]\AgdaKeyword{field}\<%
\\
\>[2][@{}l@{\AgdaIndent{0}}]%
\>[3]\AgdaField{pr₁}\AgdaSpace{}%
\AgdaSymbol{:}\AgdaSpace{}%
\AgdaBound{X}\<%
\\
%
\>[3]\AgdaField{pr₂}\AgdaSpace{}%
\AgdaSymbol{:}\AgdaSpace{}%
\AgdaBound{Y}\AgdaSpace{}%
\AgdaField{pr₁}\<%
\\
%
\\[\AgdaEmptyExtraSkip]%
%
\>[1]\AgdaKeyword{infixr}\AgdaSpace{}%
\AgdaNumber{50}\AgdaSpace{}%
\AgdaOperator{\AgdaInductiveConstructor{\AgdaUnderscore{},\AgdaUnderscore{}}}\<%
\end{code}
\scpad
% For this dependent pair type, we prefer the notation \ar{Σ}~\ab x~\af ꞉~\ab X~\af ,~\ab y, which is more pleasing and more standard than Agda's default syntax, \ar{Σ}~\as λ(\ab x~\as ꞉~\ab X)~\as →~\ab y. Escardó makes this preferred notation available in the \typetopology library by making the index type explicit, as follows.
\ccpad
\begin{code}%
\>[1]\AgdaFunction{-Σ}\AgdaSpace{}%
\AgdaSymbol{:}\AgdaSpace{}%
\AgdaSymbol{\{}\AgdaBound{𝓤}\AgdaSpace{}%
\AgdaBound{𝓥}\AgdaSpace{}%
\AgdaSymbol{:}\AgdaSpace{}%
\AgdaPostulate{Universe}\AgdaSymbol{\}}\AgdaSpace{}%
\AgdaSymbol{(}\AgdaBound{X}\AgdaSpace{}%
\AgdaSymbol{:}\AgdaSpace{}%
\AgdaBound{𝓤}\AgdaSpace{}%
\AgdaOperator{\AgdaFunction{̇}}\AgdaSpace{}%
\AgdaSymbol{)}\AgdaSpace{}%
\AgdaSymbol{(}\AgdaBound{Y}\AgdaSpace{}%
\AgdaSymbol{:}\AgdaSpace{}%
\AgdaBound{X}\AgdaSpace{}%
\AgdaSymbol{→}\AgdaSpace{}%
\AgdaBound{𝓥}\AgdaSpace{}%
\AgdaOperator{\AgdaFunction{̇}}\AgdaSpace{}%
\AgdaSymbol{)}\AgdaSpace{}%
\AgdaSymbol{→}\AgdaSpace{}%
\AgdaBound{𝓤}\AgdaSpace{}%
\AgdaOperator{\AgdaPrimitive{⊔}}\AgdaSpace{}%
\AgdaBound{𝓥}\AgdaSpace{}%
\AgdaOperator{\AgdaFunction{̇}}\<%
\\
%
\>[1]\AgdaFunction{-Σ}\AgdaSpace{}%
\AgdaBound{X}\AgdaSpace{}%
\AgdaBound{Y}\AgdaSpace{}%
\AgdaSymbol{=}\AgdaSpace{}%
\AgdaRecord{Σ}\AgdaSpace{}%
\AgdaBound{Y}\<%
\\
%
\\[\AgdaEmptyExtraSkip]%
%
\>[1]\AgdaKeyword{syntax}\AgdaSpace{}%
\AgdaFunction{-Σ}\AgdaSpace{}%
\AgdaBound{X}\AgdaSpace{}%
\AgdaSymbol{(λ}\AgdaSpace{}%
\AgdaBound{x}\AgdaSpace{}%
\AgdaSymbol{→}\AgdaSpace{}%
\AgdaBound{Y}\AgdaSymbol{)}\AgdaSpace{}%
\AgdaSymbol{=}\AgdaSpace{}%
\AgdaFunction{Σ}\AgdaSpace{}%
\AgdaBound{x}\AgdaSpace{}%
\AgdaFunction{꞉}\AgdaSpace{}%
\AgdaBound{X}\AgdaSpace{}%
\AgdaFunction{,}\AgdaSpace{}%
\AgdaBound{Y}\<%
\end{code}
\scpad
% \textbf{WARNING!} The symbol \af ꞉ is not the same as \as : despite how similar they may look. The correct colon in the expression \ar{Σ}~\ab x~\af ꞉~\ab X~\af ,~\ab y, above is obtained by typing \texttt{\textbackslash{}:4} in \agdamode.

A special case of the Sigma type is the one in which the type \ab{Y} doesn't depend on \ab{X}. This is the usual Cartesian product, defined in Agda as follows.
\ccpad
\begin{code}%
\>[1]\AgdaOperator{\AgdaFunction{\AgdaUnderscore{}×\AgdaUnderscore{}}}\AgdaSpace{}%
\AgdaSymbol{:}\AgdaSpace{}%
\AgdaGeneralizable{𝓤}\AgdaSpace{}%
\AgdaOperator{\AgdaFunction{̇}}\AgdaSpace{}%
\AgdaSymbol{→}\AgdaSpace{}%
\AgdaGeneralizable{𝓥}\AgdaSpace{}%
\AgdaOperator{\AgdaFunction{̇}}\AgdaSpace{}%
\AgdaSymbol{→}\AgdaSpace{}%
\AgdaGeneralizable{𝓤}\AgdaSpace{}%
\AgdaOperator{\AgdaPrimitive{⊔}}\AgdaSpace{}%
\AgdaGeneralizable{𝓥}\AgdaSpace{}%
\AgdaOperator{\AgdaFunction{̇}}\<%
\\
%
\>[1]\AgdaBound{X}\AgdaSpace{}%
\AgdaOperator{\AgdaFunction{×}}\AgdaSpace{}%
\AgdaBound{Y}\AgdaSpace{}%
\AgdaSymbol{=}\AgdaSpace{}%
\AgdaFunction{Σ}\AgdaSpace{}%
\AgdaBound{x}\AgdaSpace{}%
\AgdaFunction{꞉}\AgdaSpace{}%
\AgdaBound{X}\AgdaSpace{}%
\AgdaFunction{,}\AgdaSpace{}%
\AgdaBound{Y}\<%
\end{code}
\ccpad
Now that we have repeated these definitions from the \ualib for illustration purposes, let us import the original
definitions that we will use throughout the \ualib.
\ccpad
\begin{code}%
\>[0]\AgdaKeyword{open}\AgdaSpace{}%
\AgdaKeyword{import}\AgdaSpace{}%
\AgdaModule{Sigma-Type}\AgdaSpace{}%
\AgdaKeyword{renaming}\AgdaSpace{}%
\AgdaSymbol{(}\AgdaOperator{\AgdaInductiveConstructor{\AgdaUnderscore{},\AgdaUnderscore{}}}\AgdaSpace{}%
\AgdaSymbol{to}\AgdaSpace{}%
\AgdaKeyword{infixr}\AgdaSpace{}%
\AgdaNumber{50}\AgdaSpace{}%
\AgdaOperator{\AgdaInductiveConstructor{\AgdaUnderscore{},\AgdaUnderscore{}}}\AgdaSymbol{)}\<%
\\
\>[0]\AgdaKeyword{open}\AgdaSpace{}%
\AgdaKeyword{import}\AgdaSpace{}%
\AgdaModule{MGS-MLTT}\AgdaSpace{}%
\AgdaKeyword{using}\AgdaSpace{}%
\AgdaSymbol{(}\AgdaFunction{pr₁}\AgdaSymbol{;}\AgdaSpace{}%
\AgdaFunction{pr₂}\AgdaSymbol{;}\AgdaSpace{}%
\AgdaOperator{\AgdaFunction{\AgdaUnderscore{}×\AgdaUnderscore{}}}\AgdaSymbol{;}\AgdaSpace{}%
\AgdaFunction{-Σ}\AgdaSymbol{)}\<%
\end{code}
\ccpad
The definition of \ar Σ (and thus, of \af ×) includes the fields  \af{pr₁} and \af{pr₂} representing the first and second projections out of the product.  Sometimes we prefer to denote these projections with the notation \af{∣\_∣} and \af{∥\_∥}, respectively. However, for emphasis or readability we alternate between these and the following standard notations: \af{pr₁} and \af{fst} for the first projection, \af{pr₂} and \af{snd} for the second.  We define these alternative notations for projections out of pairs as follows.
% \footnote{We have made a concerted effort to avoid duplicating types that are already provided elsewhere, such as the \typetopology library.  We occasionally repeat the definitions of such types for pedagogical purposes, but we always import and work with the original definitions in order to make the sources known and to credit the original authors.}
\ccpad
\begin{code}%
\>[0]\AgdaKeyword{module}\AgdaSpace{}%
\AgdaModule{\AgdaUnderscore{}}\AgdaSpace{}%
\AgdaSymbol{\{}\AgdaBound{𝓤}\AgdaSpace{}%
\AgdaSymbol{:}\AgdaSpace{}%
\AgdaPostulate{Universe}\AgdaSymbol{\}}\AgdaSpace{}%
\AgdaKeyword{where}\<%
\\
%
\\[\AgdaEmptyExtraSkip]%
\>[0][@{}l@{\AgdaIndent{0}}]%
\>[1]\AgdaOperator{\AgdaFunction{∣\AgdaUnderscore{}∣}}\AgdaSpace{}%
\AgdaFunction{fst}\AgdaSpace{}%
\AgdaSymbol{:}\AgdaSpace{}%
\AgdaSymbol{\{}\AgdaBound{X}\AgdaSpace{}%
\AgdaSymbol{:}\AgdaSpace{}%
\AgdaBound{𝓤}\AgdaSpace{}%
\AgdaOperator{\AgdaFunction{̇}}\AgdaSpace{}%
\AgdaSymbol{\}\{}\AgdaBound{Y}\AgdaSpace{}%
\AgdaSymbol{:}\AgdaSpace{}%
\AgdaBound{X}\AgdaSpace{}%
\AgdaSymbol{→}\AgdaSpace{}%
\AgdaGeneralizable{𝓥}\AgdaSpace{}%
\AgdaOperator{\AgdaFunction{̇}}\AgdaSymbol{\}}\AgdaSpace{}%
\AgdaSymbol{→}\AgdaSpace{}%
\AgdaRecord{Σ}\AgdaSpace{}%
\AgdaBound{Y}\AgdaSpace{}%
\AgdaSymbol{→}\AgdaSpace{}%
\AgdaBound{X}\<%
\\
%
\>[1]\AgdaOperator{\AgdaFunction{∣}}\AgdaSpace{}%
\AgdaBound{x}\AgdaSpace{}%
\AgdaOperator{\AgdaInductiveConstructor{,}}\AgdaSpace{}%
\AgdaBound{y}\AgdaSpace{}%
\AgdaOperator{\AgdaFunction{∣}}\AgdaSpace{}%
\AgdaSymbol{=}\AgdaSpace{}%
\AgdaBound{x}\<%
\\
%
\>[1]\AgdaFunction{fst}\AgdaSpace{}%
\AgdaSymbol{(}\AgdaBound{x}\AgdaSpace{}%
\AgdaOperator{\AgdaInductiveConstructor{,}}\AgdaSpace{}%
\AgdaBound{y}\AgdaSymbol{)}\AgdaSpace{}%
\AgdaSymbol{=}\AgdaSpace{}%
\AgdaBound{x}\<%
\\
%
\\[\AgdaEmptyExtraSkip]%
%
\>[1]\AgdaOperator{\AgdaFunction{∥\AgdaUnderscore{}∥}}\AgdaSpace{}%
\AgdaFunction{snd}\AgdaSpace{}%
\AgdaSymbol{:}\AgdaSpace{}%
\AgdaSymbol{\{}\AgdaBound{X}\AgdaSpace{}%
\AgdaSymbol{:}\AgdaSpace{}%
\AgdaBound{𝓤}\AgdaSpace{}%
\AgdaOperator{\AgdaFunction{̇}}\AgdaSpace{}%
\AgdaSymbol{\}\{}\AgdaBound{Y}\AgdaSpace{}%
\AgdaSymbol{:}\AgdaSpace{}%
\AgdaBound{X}\AgdaSpace{}%
\AgdaSymbol{→}\AgdaSpace{}%
\AgdaGeneralizable{𝓥}\AgdaSpace{}%
\AgdaOperator{\AgdaFunction{̇}}\AgdaSpace{}%
\AgdaSymbol{\}}\AgdaSpace{}%
\AgdaSymbol{→}\AgdaSpace{}%
\AgdaSymbol{(}\AgdaBound{z}\AgdaSpace{}%
\AgdaSymbol{:}\AgdaSpace{}%
\AgdaRecord{Σ}\AgdaSpace{}%
\AgdaBound{Y}\AgdaSymbol{)}\AgdaSpace{}%
\AgdaSymbol{→}\AgdaSpace{}%
\AgdaBound{Y}\AgdaSpace{}%
\AgdaSymbol{(}\AgdaFunction{pr₁}\AgdaSpace{}%
\AgdaBound{z}\AgdaSymbol{)}\<%
\\
%
\>[1]\AgdaOperator{\AgdaFunction{∥}}\AgdaSpace{}%
\AgdaBound{x}\AgdaSpace{}%
\AgdaOperator{\AgdaInductiveConstructor{,}}\AgdaSpace{}%
\AgdaBound{y}\AgdaSpace{}%
\AgdaOperator{\AgdaFunction{∥}}\AgdaSpace{}%
\AgdaSymbol{=}\AgdaSpace{}%
\AgdaBound{y}\<%
\\
%
\>[1]\AgdaFunction{snd}\AgdaSpace{}%
\AgdaSymbol{(}\AgdaBound{x}\AgdaSpace{}%
\AgdaOperator{\AgdaInductiveConstructor{,}}\AgdaSpace{}%
\AgdaBound{y}\AgdaSymbol{)}\AgdaSpace{}%
\AgdaSymbol{=}\AgdaSpace{}%
\AgdaBound{y}\<%
\end{code}
\ccpad
Note that we put the definitions above inside an \defn{anonymous module}, which starts with the \ak{module} keyword followed by an underscore (instead of a module name). The purpose is simply to move the postulated typing judgments---the ``parameters'' of the module (e.g., `𝓤 : Universe`)---out of the way so they don't obfuscate the definitions inside the module.





\paragraph*{Pi types (dependent functions)} %\label{ssec:dependent-functions}
Given universes \ab 𝓤 and \ab 𝓥, a type \ab{X} \as : \ab 𝓤\af ̇, and a type family \ab{Y}~\as :~\ab X~\as →~\ab 𝓥\af ̇, the \defn{Pi type} (aka \defn{dependent function type}) is denoted by \ar Π(\ab x~\af :~\ab X),~\ab Y~\ab x and generalizes the function type \ab X \as → \ab Y by letting the type \ab Y~\ab x of the codomain depend on the value \ab x of the domain type. The dependent function type is defined in the \typetopology in a standard way.  For the reader's benefit, however, we repeat the definition here inside a hidden module.
\ccpad
\begin{code}%
\>[0]\AgdaKeyword{module}\AgdaSpace{}%
\AgdaModule{hide-pi}\AgdaSpace{}%
\AgdaSymbol{\{}\AgdaBound{𝓤}\AgdaSpace{}%
\AgdaBound{𝓦}\AgdaSpace{}%
\AgdaSymbol{:}\AgdaSpace{}%
\AgdaPostulate{Universe}\AgdaSymbol{\}}\AgdaSpace{}%
\AgdaKeyword{where}\<%
\\
%
\\[\AgdaEmptyExtraSkip]%
\>[0][@{}l@{\AgdaIndent{0}}]%
\>[1]\AgdaFunction{Π}\AgdaSpace{}%
\AgdaSymbol{:}\AgdaSpace{}%
\AgdaSymbol{\{}\AgdaBound{X}\AgdaSpace{}%
\AgdaSymbol{:}\AgdaSpace{}%
\AgdaBound{𝓤}\AgdaSpace{}%
\AgdaOperator{\AgdaFunction{̇}}\AgdaSpace{}%
\AgdaSymbol{\}}\AgdaSpace{}%
\AgdaSymbol{(}\AgdaBound{A}\AgdaSpace{}%
\AgdaSymbol{:}\AgdaSpace{}%
\AgdaBound{X}\AgdaSpace{}%
\AgdaSymbol{→}\AgdaSpace{}%
\AgdaBound{𝓦}\AgdaSpace{}%
\AgdaOperator{\AgdaFunction{̇}}\AgdaSpace{}%
\AgdaSymbol{)}\AgdaSpace{}%
\AgdaSymbol{→}\AgdaSpace{}%
\AgdaBound{𝓤}\AgdaSpace{}%
\AgdaOperator{\AgdaPrimitive{⊔}}\AgdaSpace{}%
\AgdaBound{𝓦}\AgdaSpace{}%
\AgdaOperator{\AgdaFunction{̇}}\<%
\\
%
\>[1]\AgdaFunction{Π}\AgdaSpace{}%
\AgdaSymbol{\{}\AgdaBound{X}\AgdaSymbol{\}}\AgdaSpace{}%
\AgdaBound{A}\AgdaSpace{}%
\AgdaSymbol{=}\AgdaSpace{}%
\AgdaSymbol{(}\AgdaBound{x}\AgdaSpace{}%
\AgdaSymbol{:}\AgdaSpace{}%
\AgdaBound{X}\AgdaSymbol{)}\AgdaSpace{}%
\AgdaSymbol{→}\AgdaSpace{}%
\AgdaBound{A}\AgdaSpace{}%
\AgdaBound{x}\<%
\end{code}
\ccpad
To make the syntax for \af{Π} conform to the standard notation for Pi types, \escardo uses the same trick as the one used above for Sigma types.\footnote{\textbf{WARNING!} The symbol \af ꞉ is not the same as \as : despite how similar they may look. The correct colon in the expression \af{Π}~\ab x~\as ꞉~\ab X~\af ,~\ab y above is obtained by typing \texttt{\textbackslash{}:4} in \agdamode.}
\ccpad
\begin{code}
\>[1]\AgdaFunction{-Π}\AgdaSpace{}%
\AgdaSymbol{:}\AgdaSpace{}%
\AgdaSymbol{(}\AgdaBound{X}\AgdaSpace{}%
\AgdaSymbol{:}\AgdaSpace{}%
\AgdaBound{𝓤}\AgdaSpace{}%
\AgdaOperator{\AgdaFunction{̇}}\AgdaSpace{}%
\AgdaSymbol{)(}\AgdaBound{Y}\AgdaSpace{}%
\AgdaSymbol{:}\AgdaSpace{}%
\AgdaBound{X}\AgdaSpace{}%
\AgdaSymbol{→}\AgdaSpace{}%
\AgdaBound{𝓦}\AgdaSpace{}%
\AgdaOperator{\AgdaFunction{̇}}\AgdaSpace{}%
\AgdaSymbol{)}\AgdaSpace{}%
\AgdaSymbol{→}\AgdaSpace{}%
\AgdaBound{𝓤}\AgdaSpace{}%
\AgdaOperator{\AgdaPrimitive{⊔}}\AgdaSpace{}%
\AgdaBound{𝓦}\AgdaSpace{}%
\AgdaOperator{\AgdaFunction{̇}}\<%
\\
%
\>[1]\AgdaFunction{-Π}\AgdaSpace{}%
\AgdaBound{X}\AgdaSpace{}%
\AgdaBound{Y}\AgdaSpace{}%
\AgdaSymbol{=}\AgdaSpace{}%
\AgdaFunction{Π}\AgdaSpace{}%
\AgdaBound{Y}\<%
\\
%
\\[\AgdaEmptyExtraSkip]%
%
\>[1]\AgdaKeyword{infixr}\AgdaSpace{}%
\AgdaNumber{-1}\AgdaSpace{}%
\AgdaFunction{-Π}\<%
\\
%
\>[1]\AgdaKeyword{syntax}\AgdaSpace{}%
\AgdaFunction{-Π}\AgdaSpace{}%
\AgdaBound{A}\AgdaSpace{}%
\AgdaSymbol{(λ}\AgdaSpace{}%
\AgdaBound{x}\AgdaSpace{}%
\AgdaSymbol{→}\AgdaSpace{}%
\AgdaBound{b}\AgdaSymbol{)}\AgdaSpace{}%
\AgdaSymbol{=}\AgdaSpace{}%
\AgdaFunction{Π}\AgdaSpace{}%
\AgdaBound{x}\AgdaSpace{}%
\AgdaFunction{꞉}\AgdaSpace{}%
\AgdaBound{A}\AgdaSpace{}%
\AgdaFunction{,}\AgdaSpace{}%
\AgdaBound{b}\<%
\end{code}




\subsection{Equality}\label{equality} %: definitional equality and transport
\firstsentence{Overture}{Equality}
% -*- TeX-master: "ualib-part1.tex" -*-
%%% Local Variables: 
%%% mode: latex
%%% TeX-engine: 'xetex
%%% End:
\subsubsection{Definitional equality}\label{sec:defin-equal}
Here we discuss what is probably the most important type in \mltt. It is called \defn{definitional equality}. This concept is most understood, at least heuristically, with the following slogan: ``Definitional equality is the substitution-preserving equivalence relation generated by definitions.''
We will make this precise below, but first let us quote from a primary source. Per Martin-L\"of offers the following definition in~\cite[\S1.11]{MR0387009} (italics added):\footnote{The \defn{definiendum} is the left-hand side of a defining equation, the \defn{definiens} is the right-hand side.\\ For readers who have never generated an equivalence relation: the \defn{reflexive closure} of \ab R \af ⊆ \ab A \ad × \ab A is the union of \ab R and all pairs of the form (\ab a , \ab a); the \defn{symmetric closure} is the union of \ab R and its inverse \{(\ab y , \ab x) : (\ab x , \ab y) ∈ \ab R\}; we leave it to the reader to come up with the correct definition of transitive closure.}
\begin{quote}
  \defn{Definitional equality} is defined to be the equivalence relation, that is, reflexive, symmetric and transitive relation, which is generated by the principles that a definiendum is always definitionally equal to its definiens and that definitional equality is preserved under substitution.
\end{quote}
To be sure we understand what this means, let $:=$ denote the relation with respect to which $x$ is related to $y$ (denoted $x := y$) if and only if $y$ \emph{is the definition of} $x$.  Then the definitional equality relation \ad ≡ is the reflexive, symmetric, transitive, substitutive closure of $:=$. By \defn{subsitutive closure} we mean closure under the following \defn{substitution rule}.
% \{\ab A \as : \ab 𝓤\af ̇\}\{\ab B \as : \ab A \as → \ab 𝓦\af ̇\}\{\ab x \abt{y}{A}\} \as → \ab x \ad ≡ \ab y \as → \ab B \ab x \ad ≡ \ab B \ab y\\[4pt]
\vskip3mm\hskip1cm \infer[(subst)]{\ab B~\ab x~\ad ≡~\ab B~\ab y}{\{\ab A~\as :~\ab 𝓤\af ̇\}~\{\ab B~\as :~\ab A~\as →~\ab 𝓦\af ̇\}~\{\ab x~\abt{y}{A}\}  & \ab x~\ad ≡~\ab y}\\[-5pt]


% In~\cite[\S1.11, page 85]{MR0387009}, Per Martin-L\"of describes definitional equality as follows:
% \begin{quote}
% ``Definitional equality is intensional equality, or equality of meaning (synonymy)... Definitional equality ≡ is a relation between linguistic expressions; it should not be confused with equality between objects (sets, elements of a set etc.)... Definitional equality is the equivalence relation generated by abbreviatory definitions, changes of bound variables and the principle of substituting equals for equals. Therefore it is decidable, but not in the sense that a ≡ b ∨ ¬(a ≡ b) holds, simply because a ≡ b is not a proposition in the sense of the present theory.''\\[-30pt]
% \begin{flushright}Per Martin-L\"of,\\
% \textit{Padua Lecture Notes}~\cite{MR769301}
% \end{flushright}
% \end{quote}

The datatype used in the \ualib to represent definitional equality is imported from the \am{Identity-Type} module of \typtop, but apart from superficial syntactic differences, it is equivalent to the identity type used in all other Agda libraries we know of.  We repeat the definition here for easy reference.
\ccpad
\begin{code}%
% \>[0]\AgdaKeyword{module}\AgdaSpace{}%
% \AgdaModule{hide-refl}\AgdaSpace{}%
% \AgdaSymbol{\{}\AgdaBound{𝓤}\AgdaSpace{}%
% \AgdaSymbol{:}\AgdaSpace{}%
% \AgdaPostulate{Universe}\AgdaSymbol{\}}\AgdaSpace{}%
% \AgdaKeyword{where}\<%
% \\
% %
% \\[\AgdaEmptyExtraSkip]%
% \>[0][@{}l@{\AgdaIndent{0}}]%
\>[1]\AgdaKeyword{data}\AgdaSpace{}%
\AgdaOperator{\AgdaDatatype{\AgdaUnderscore{}≡\AgdaUnderscore{}}}\AgdaSpace{}%
\AgdaSymbol{\{}\AgdaBound{𝓤}\AgdaSymbol{\}}\AgdaSpace{}%
\AgdaSymbol{\{}\AgdaBound{A}\AgdaSpace{}%
\AgdaSymbol{:}\AgdaSpace{}%
\AgdaBound{𝓤}\AgdaSpace{}%
\AgdaOperator{\AgdaFunction{̇}}\AgdaSpace{}%
\AgdaSymbol{\}}\AgdaSpace{}%
\AgdaSymbol{:}\AgdaSpace{}%
\AgdaBound{A}\AgdaSpace{}%
\AgdaSymbol{→}\AgdaSpace{}%
\AgdaBound{A}\AgdaSpace{}%
\AgdaSymbol{→}\AgdaSpace{}%
\AgdaBound{𝓤}\AgdaSpace{}%
\AgdaOperator{\AgdaFunction{̇}}\AgdaSpace{}%
\AgdaKeyword{where}\AgdaSpace{}%
\AgdaInductiveConstructor{refl}\AgdaSpace{}%
\AgdaSymbol{:}\AgdaSpace{}%
\AgdaSymbol{\{}\AgdaBound{x}\AgdaSpace{}%
\AgdaSymbol{:}\AgdaSpace{}%
\AgdaBound{A}\AgdaSymbol{\}}\AgdaSpace{}%
\AgdaSymbol{→}\AgdaSpace{}%
\AgdaBound{x}\AgdaSpace{}%
\AgdaOperator{\AgdaDatatype{≡}}\AgdaSpace{}%
\AgdaBound{x}\<%
% \\
% %
% \\[\AgdaEmptyExtraSkip]%
% \>[0]\AgdaKeyword{open}\AgdaSpace{}%
% \AgdaKeyword{import}\AgdaSpace{}%
% \AgdaModule{Identity-Type}\AgdaSpace{}%
% \AgdaKeyword{renaming}\AgdaSpace{}%
% \AgdaSymbol{(}\AgdaOperator{\AgdaDatatype{\AgdaUnderscore{}≡\AgdaUnderscore{}}}\AgdaSpace{}%
% \AgdaSymbol{to}\AgdaSpace{}%
% \AgdaKeyword{infix}\AgdaSpace{}%
% \AgdaNumber{0}\AgdaSpace{}%
% \AgdaOperator{\AgdaDatatype{\AgdaUnderscore{}≡\AgdaUnderscore{}}}\AgdaSymbol{)}\AgdaSpace{}%
% \AgdaKeyword{public}\<%
% \\
% \>[0]\<%
\end{code}
\ccpad
Whenever we need to complete a proof by simply asserting that \ab{x} is \defn{definitionally equal} to itself, we invoke \aic{refl}. If we need to make explicit the implicit argument \ab x, then we use \aic{refl}~\{\ab x~=~\ab x\}.

\paragraph*{Assumed module contexts}
Before proceeding, a word about a special convention we adopt in the sequel is in order. To reduce reader strain, we will often omit easily inferred typing judgments which would normally appear in the list of parameters of a module or at the start of a type definition, and we sometimes make an announcement like the following (which applies to the present section):
\context{\{\ab{𝓤}~\as :~\ap{Universe}\} \{\ab{A}~\as :~\ab 𝓤\af ̇\}}
\paragraph*{Definitional equality is an equivalence}
The relation \ad{≡} just defined is naturally an equivalence relation, and the formal proof of this fact is trivial. Indeed, we don't need to prove reflexivity, since that is the defining property of \ad{≡}, and the proofs of symmetry and transitivity are also immediate.
\ccpad
\begin{code}%
\>[1]\AgdaFunction{≡-symmetric}\AgdaSpace{}%
\AgdaSymbol{:}\AgdaSpace{}%
\AgdaSymbol{(}\AgdaBound{x}\AgdaSpace{}%
\AgdaBound{y}\AgdaSpace{}%
\AgdaSymbol{:}\AgdaSpace{}%
\AgdaBound{A}\AgdaSymbol{)}\AgdaSpace{}%
\AgdaSymbol{→}\AgdaSpace{}%
\AgdaBound{x}\AgdaSpace{}%
\AgdaOperator{\AgdaDatatype{≡}}\AgdaSpace{}%
\AgdaBound{y}\AgdaSpace{}%
\AgdaSymbol{→}\AgdaSpace{}%
\AgdaBound{y}\AgdaSpace{}%
\AgdaOperator{\AgdaDatatype{≡}}\AgdaSpace{}%
\AgdaBound{x}\<%
\\
%
\>[1]\AgdaFunction{≡-symmetric}\AgdaSpace{}%
\AgdaSymbol{\AgdaUnderscore{}}\AgdaSpace{}%
\AgdaSymbol{\AgdaUnderscore{}}\AgdaSpace{}%
\AgdaInductiveConstructor{refl}\AgdaSpace{}%
\AgdaSymbol{=}\AgdaSpace{}%
\AgdaInductiveConstructor{refl}\<%
\\
%
\\[\AgdaEmptyExtraSkip]%
%
\>[1]\AgdaFunction{≡-sym}\AgdaSpace{}%
\AgdaSymbol{:}\AgdaSpace{}%
\AgdaSymbol{\{}\AgdaBound{x}\AgdaSpace{}%
\AgdaBound{y}\AgdaSpace{}%
\AgdaSymbol{:}\AgdaSpace{}%
\AgdaBound{A}\AgdaSymbol{\}}\AgdaSpace{}%
\AgdaSymbol{→}\AgdaSpace{}%
\AgdaBound{x}\AgdaSpace{}%
\AgdaOperator{\AgdaDatatype{≡}}\AgdaSpace{}%
\AgdaBound{y}\AgdaSpace{}%
\AgdaSymbol{→}\AgdaSpace{}%
\AgdaBound{y}\AgdaSpace{}%
\AgdaOperator{\AgdaDatatype{≡}}\AgdaSpace{}%
\AgdaBound{x}\<%
\\
%
\>[1]\AgdaFunction{≡-sym}\AgdaSpace{}%
\AgdaInductiveConstructor{refl}\AgdaSpace{}%
\AgdaSymbol{=}\AgdaSpace{}%
\AgdaInductiveConstructor{refl}\<%
\\
%
\\[\AgdaEmptyExtraSkip]%
%
\>[1]\AgdaFunction{≡-transitive}\AgdaSpace{}%
\AgdaSymbol{:}\AgdaSpace{}%
\AgdaSymbol{(}\AgdaBound{x}\AgdaSpace{}%
\AgdaBound{y}\AgdaSpace{}%
\AgdaBound{z}\AgdaSpace{}%
\AgdaSymbol{:}\AgdaSpace{}%
\AgdaBound{A}\AgdaSymbol{)}\AgdaSpace{}%
\AgdaSymbol{→}\AgdaSpace{}%
\AgdaBound{x}\AgdaSpace{}%
\AgdaOperator{\AgdaDatatype{≡}}\AgdaSpace{}%
\AgdaBound{y}\AgdaSpace{}%
\AgdaSymbol{→}\AgdaSpace{}%
\AgdaBound{y}\AgdaSpace{}%
\AgdaOperator{\AgdaDatatype{≡}}\AgdaSpace{}%
\AgdaBound{z}\AgdaSpace{}%
\AgdaSymbol{→}\AgdaSpace{}%
\AgdaBound{x}\AgdaSpace{}%
\AgdaOperator{\AgdaDatatype{≡}}\AgdaSpace{}%
\AgdaBound{z}\<%
\\
%
\>[1]\AgdaFunction{≡-transitive}\AgdaSpace{}%
\AgdaSymbol{\AgdaUnderscore{}}\AgdaSpace{}%
\AgdaSymbol{\AgdaUnderscore{}}\AgdaSpace{}%
\AgdaSymbol{\AgdaUnderscore{}}\AgdaSpace{}%
\AgdaInductiveConstructor{refl}\AgdaSpace{}%
\AgdaInductiveConstructor{refl}\AgdaSpace{}%
\AgdaSymbol{=}\AgdaSpace{}%
\AgdaInductiveConstructor{refl}\<%
\\
%
\\[\AgdaEmptyExtraSkip]%
%
\>[1]\AgdaFunction{≡-trans}\AgdaSpace{}%
\AgdaSymbol{:}\AgdaSpace{}%
\AgdaSymbol{\{}\AgdaBound{x}\AgdaSpace{}%
\AgdaBound{y}\AgdaSpace{}%
\AgdaBound{z}\AgdaSpace{}%
\AgdaSymbol{:}\AgdaSpace{}%
\AgdaBound{A}\AgdaSymbol{\}}\AgdaSpace{}%
\AgdaSymbol{→}\AgdaSpace{}%
\AgdaBound{x}\AgdaSpace{}%
\AgdaOperator{\AgdaDatatype{≡}}\AgdaSpace{}%
\AgdaBound{y}\AgdaSpace{}%
\AgdaSymbol{→}\AgdaSpace{}%
\AgdaBound{y}\AgdaSpace{}%
\AgdaOperator{\AgdaDatatype{≡}}\AgdaSpace{}%
\AgdaBound{z}\AgdaSpace{}%
\AgdaSymbol{→}\AgdaSpace{}%
\AgdaBound{x}\AgdaSpace{}%
\AgdaOperator{\AgdaDatatype{≡}}\AgdaSpace{}%
\AgdaBound{z}\<%
\\
%
\>[1]\AgdaFunction{≡-trans}\AgdaSpace{}%
\AgdaInductiveConstructor{refl}\AgdaSpace{}%
\AgdaInductiveConstructor{refl}\AgdaSpace{}%
\AgdaSymbol{=}\AgdaSpace{}%
\AgdaInductiveConstructor{refl}\<%
\end{code}
\ccpad
The only difference between \af{≡-symmetric} and \af{≡-sym} (resp., \af{≡-transitive} and \af{≡-trans}) is that \af{≡-sym} (resp., \af{≡-trans}) has fewer explicit arguments, which is sometimes convenient.

We prove that \ad ≡ obeys the substitution rule (subst) in the next section (see \af{ap}~\S\ref{transport}), but first we define some syntactic sugar that will make it easier to apply symmetry and transitivity of \ad ≡ in proofs.\footnote{%
\textbf{Unicode Hints} (\agdamode): \texttt{\textbackslash{}\^{}-\textbackslash{}\^{}1} ↝ \af{⁻¹}; \texttt{\textbackslash{}Mii\textbackslash{}Mid} ↝ \af{𝑖𝑑}; \texttt{\textbackslash{}.}↝ \af{∙}. In general, for information about a character, place the cursor on the character and type \texttt{M-x\ describe-char} (or \texttt{M-x\ h\ d\ c}).}
\ccpad
\begin{code}
% \>[0]\AgdaKeyword{module}\AgdaSpace{}%
% \AgdaModule{hide-sym-trans}\AgdaSpace{}%
% \AgdaSymbol{\{}\AgdaBound{𝓤}\AgdaSpace{}%
% \AgdaSymbol{:}\AgdaSpace{}%
% \AgdaPostulate{Universe}\AgdaSymbol{\}}\AgdaSpace{}%
% \AgdaKeyword{where}\<%
% \\
% %
% \\[\AgdaEmptyExtraSkip]%
% \>[0][@{}l@{\AgdaIndent{0}}]%
\>[1]\AgdaOperator{\AgdaFunction{\AgdaUnderscore{}⁻¹}}\AgdaSpace{}%
\AgdaSymbol{:}\AgdaSpace{}%
% \AgdaSymbol{\{}\AgdaBound{A}\AgdaSpace{}%
% \AgdaSymbol{:}\AgdaSpace{}%
% \AgdaBound{𝓤}\AgdaSpace{}%
% \AgdaOperator{\AgdaFunction{̇}}\AgdaSpace{}%
% \AgdaSymbol{\}}\AgdaSpace{}%
% \AgdaSymbol{→}\AgdaSpace{}%
\AgdaSymbol{\{}\AgdaBound{x}\AgdaSpace{}%
\AgdaBound{y}\AgdaSpace{}%
\AgdaSymbol{:}\AgdaSpace{}%
\AgdaBound{A}\AgdaSymbol{\}}\AgdaSpace{}%
\AgdaSymbol{→}\AgdaSpace{}%
\AgdaBound{x}\AgdaSpace{}%
\AgdaOperator{\AgdaDatatype{≡}}\AgdaSpace{}%
\AgdaBound{y}\AgdaSpace{}%
\AgdaSymbol{→}\AgdaSpace{}%
\AgdaBound{y}\AgdaSpace{}%
\AgdaOperator{\AgdaDatatype{≡}}\AgdaSpace{}%
\AgdaBound{x}\<%
\\
%
\>[1]\AgdaBound{p}\AgdaSpace{}%
\AgdaOperator{\AgdaFunction{⁻¹}}\AgdaSpace{}%
\AgdaSymbol{=}\AgdaSpace{}%
\AgdaFunction{≡-sym}\AgdaSpace{}%
\AgdaBound{p}\<%
\end{code}
\ccpad
If we have a proof \ab p \as :~\ab x \aod ≡ \ab y, and we need a proof of \ab y \aod ≡ \ab x, then instead of \af{≡-sym} \ab p we can use the more intuitive \ab p \af{⁻¹}. Similarly, the following syntactic sugar makes abundant appeals to transitivity easier to stomach.
\ccpad
\begin{code}%
\>[1]\AgdaOperator{\AgdaFunction{\AgdaUnderscore{}∙\AgdaUnderscore{}}}\AgdaSpace{}%
\AgdaSymbol{:}\AgdaSpace{}%
% \AgdaSymbol{\{}\AgdaBound{A}\AgdaSpace{}%
% \AgdaSymbol{:}\AgdaSpace{}%
% \AgdaBound{𝓤}\AgdaSpace{}%
% \AgdaOperator{\AgdaFunction{̇}}\AgdaSpace{}%
% \AgdaSymbol{\}}\AgdaSpace{}%
\AgdaSymbol{\{}\AgdaBound{x}\AgdaSpace{}%
\AgdaBound{y}\AgdaSpace{}%
\AgdaBound{z}\AgdaSpace{}%
\AgdaSymbol{:}\AgdaSpace{}%
\AgdaBound{A}\AgdaSymbol{\}}\AgdaSpace{}%
\AgdaSymbol{→}\AgdaSpace{}%
\AgdaBound{x}\AgdaSpace{}%
\AgdaOperator{\AgdaDatatype{≡}}\AgdaSpace{}%
\AgdaBound{y}\AgdaSpace{}%
\AgdaSymbol{→}\AgdaSpace{}%
\AgdaBound{y}\AgdaSpace{}%
\AgdaOperator{\AgdaDatatype{≡}}\AgdaSpace{}%
\AgdaBound{z}\AgdaSpace{}%
\AgdaSymbol{→}\AgdaSpace{}%
\AgdaBound{x}\AgdaSpace{}%
\AgdaOperator{\AgdaDatatype{≡}}\AgdaSpace{}%
\AgdaBound{z}\<%
\\
%
\>[1]\AgdaBound{p}\AgdaSpace{}%
\AgdaOperator{\AgdaFunction{∙}}\AgdaSpace{}%
\AgdaBound{q}\AgdaSpace{}%
\AgdaSymbol{=}\AgdaSpace{}%
\AgdaFunction{≡-trans}\AgdaSpace{}%
\AgdaBound{p}\AgdaSpace{}%
\AgdaBound{q}\<%
% \\
% %
% \\[\AgdaEmptyExtraSkip]%
% \>[0]\AgdaKeyword{open}\AgdaSpace{}%
% \AgdaKeyword{import}\AgdaSpace{}%
% \AgdaModule{MGS-MLTT}\AgdaSpace{}%
% \AgdaKeyword{using}\AgdaSpace{}%
% \AgdaSymbol{(}\AgdaOperator{\AgdaFunction{\AgdaUnderscore{}⁻¹}}\AgdaSymbol{;}\AgdaSpace{}%
% \AgdaOperator{\AgdaFunction{\AgdaUnderscore{}∙\AgdaUnderscore{}}}\AgdaSymbol{)}\AgdaSpace{}%
% \AgdaKeyword{public}\<%
\end{code}

\subsubsection{Transport}\label{transport}

Alonzo Church characterized equality by declaring two things equal if and only if no property (predicate) can distinguish them (see \cite{Church:1940}). In other terms, \ab x and \ab y are equal if and only if for all \ab{P} we have \ab{P} \ab x \as → \ab P \ab y. One direction of this implication is sometimes called \emph{substitution} or \emph{transport} or \emph{transport along an identity}. It asserts the following: if two objects are equal and one of them satisfies a given predicate, then so does the other. A type representing this notion is defined, along with the (polymorphic) identity function, in the \am{MGS-MLTT} module of \typtop as follows.\footnote{Including every line of code of the \agdaualib in this paper would result in an unbearable reading experience. We include all significant sections of code from the first 13 modules, but we omit lines indicating that redundant definitions of functions (e.g., \af{transport} and \af{ap}) occur inside named ``hidden'' modules. We also omit lines importing the original definitions of such duplicate definitions from \typtop library.}
\ccpad
\begin{code}%
\>[1]\AgdaFunction{𝑖𝑑}\AgdaSpace{}%
\AgdaSymbol{:}\AgdaSpace{}%
\AgdaSymbol{\{}\AgdaBound{𝓤}\AgdaSpace{}%
\AgdaSymbol{:}\AgdaSpace{}%
\AgdaPostulate{Universe}\AgdaSymbol{\}}\AgdaSpace{}%
\AgdaSymbol{(}\AgdaBound{A}\AgdaSpace{}%
\AgdaSymbol{:}\AgdaSpace{}%
\AgdaBound{𝓤}\AgdaSpace{}%
\AgdaOperator{\AgdaFunction{̇}}\AgdaSpace{}%
\AgdaSymbol{)}\AgdaSpace{}%
\AgdaSymbol{→}\AgdaSpace{}%
\AgdaBound{A}\AgdaSpace{}%
\AgdaSymbol{→}\AgdaSpace{}%
\AgdaBound{A}\<%
\\
%
\>[1]\AgdaFunction{𝑖𝑑}\AgdaSpace{}%
\AgdaBound{A}\AgdaSpace{}%
\AgdaSymbol{=}\AgdaSpace{}%
\AgdaSymbol{λ}\AgdaSpace{}%
\AgdaBound{x}\AgdaSpace{}%
\AgdaSymbol{→}\AgdaSpace{}%
\AgdaBound{x}\<%
\end{code}
\scpad
\begin{code}
\>[1]\AgdaFunction{transport}\AgdaSpace{}%
\AgdaSymbol{:}\AgdaSpace{}%
\AgdaSymbol{\{}\AgdaBound{A}\AgdaSpace{}%
\AgdaSymbol{:}\AgdaSpace{}%
\AgdaBound{𝓤}\AgdaSpace{}%
\AgdaOperator{\AgdaFunction{̇}}\AgdaSpace{}%
\AgdaSymbol{\}}\AgdaSpace{}%
\AgdaSymbol{(}\AgdaBound{B}\AgdaSpace{}%
\AgdaSymbol{:}\AgdaSpace{}%
\AgdaBound{A}\AgdaSpace{}%
\AgdaSymbol{→}\AgdaSpace{}%
\AgdaBound{𝓦}\AgdaSpace{}%
\AgdaOperator{\AgdaFunction{̇}}\AgdaSpace{}%
\AgdaSymbol{)}\AgdaSpace{}%
\AgdaSymbol{\{}\AgdaBound{x}\AgdaSpace{}%
\AgdaBound{y}\AgdaSpace{}%
\AgdaSymbol{:}\AgdaSpace{}%
\AgdaBound{A}\AgdaSymbol{\}}\AgdaSpace{}%
\AgdaSymbol{→}\AgdaSpace{}%
\AgdaBound{x}\AgdaSpace{}%
\AgdaOperator{\AgdaDatatype{≡}}\AgdaSpace{}%
\AgdaBound{y}\AgdaSpace{}%
\AgdaSymbol{→}\AgdaSpace{}%
\AgdaBound{B}\AgdaSpace{}%
\AgdaBound{x}\AgdaSpace{}%
\AgdaSymbol{→}\AgdaSpace{}%
\AgdaBound{B}\AgdaSpace{}%
\AgdaBound{y}\<%
\\
%
\>[1]\AgdaFunction{transport}\AgdaSpace{}%
\AgdaBound{B}\AgdaSpace{}%
\AgdaSymbol{(}\AgdaInductiveConstructor{refl}\AgdaSpace{}%
\AgdaSymbol{\{}\AgdaArgument{x}\AgdaSpace{}%
\AgdaSymbol{=}\AgdaSpace{}%
\AgdaBound{x}\AgdaSymbol{\})}\AgdaSpace{}%
\AgdaSymbol{=}\AgdaSpace{}%
\AgdaFunction{𝑖𝑑}\AgdaSpace{}%
\AgdaSymbol{(}\AgdaBound{B}\AgdaSpace{}%
\AgdaBound{x}\AgdaSymbol{)}\<%
\end{code}
\scpad
% See~\cite{MHE} for a discussion of transport.\footnote{cf.~transport in \href{https://github.com/HoTT/HoTT-Agda/blob/master/core/lib/Base.agda}{HoTT-Agda}: \url{https://github.com/HoTT/HoTT-Agda/blob/master/core/lib/Base.agda}.}

A function is well-defined if and only if it maps equivalent elements to a single element and we often use this nature of functions in Agda proofs.  It is equivalent to the substitution rule (subst) we defined in the last section. If we have a function \ab{f} \as : \ab A \as → \ab B, two elements \ab{x} \ab{y} \as : \ab A of the domain, and an identity proof \ab{p} \as : \ab x \ad ≡ \ab{y}, then we obtain a proof of \ab{f} \ab x \ad ≡ \ab f \ab{y} by simply applying the \af{ap} function like so, \af{ap} \ab f \ab p \as : \ab f \ab x \ad ≡ \ab f \ab{y}. Escardó defines \af{ap} in \typtop as follows.
\ccpad
\begin{code}%
% \>[0]\AgdaKeyword{module}\AgdaSpace{}%
% \AgdaModule{hide-ap}%
% \>[16]\AgdaSymbol{\{}\AgdaBound{𝓤}\AgdaSpace{}%
% \AgdaSymbol{:}\AgdaSpace{}%
% \AgdaPostulate{Universe}\AgdaSymbol{\}}\AgdaSpace{}%
% \AgdaKeyword{where}\<%
% \\
% %
% \\[\AgdaEmptyExtraSkip]%
% \>[0][@{}l@{\AgdaIndent{0}}]%
\>[1]\AgdaFunction{ap}\AgdaSpace{}%
\AgdaSymbol{:}\AgdaSpace{}%
\AgdaSymbol{\{}\AgdaBound{A}\AgdaSpace{}%
\AgdaSymbol{:}\AgdaSpace{}%
\AgdaBound{𝓤}\AgdaSpace{}%
\AgdaOperator{\AgdaFunction{̇}}\AgdaSymbol{\}\{}\AgdaBound{B}\AgdaSpace{}%
\AgdaSymbol{:}\AgdaSpace{}%
\AgdaGeneralizable{𝓥}\AgdaSpace{}%
\AgdaOperator{\AgdaFunction{̇}}\AgdaSymbol{\}}
\AgdaSymbol{(}\AgdaBound{f}\AgdaSpace{}%
\AgdaSymbol{:}\AgdaSpace{}%
\AgdaBound{A}\AgdaSpace{}%
\AgdaSymbol{→}\AgdaSpace{}%
\AgdaBound{B}\AgdaSymbol{)\{}\AgdaBound{a}\AgdaSpace{}%
\AgdaBound{b}\AgdaSpace{}%
\AgdaSymbol{:}\AgdaSpace{}%
\AgdaBound{A}\AgdaSymbol{\}}\AgdaSpace{}%
\AgdaSymbol{→}\AgdaSpace{}%
\AgdaBound{a}\AgdaSpace{}%
\AgdaOperator{\AgdaDatatype{≡}}\AgdaSpace{}%
\AgdaBound{b}\AgdaSpace{}%
\AgdaSymbol{→}\AgdaSpace{}%
\AgdaBound{f}\AgdaSpace{}%
\AgdaBound{a}\AgdaSpace{}%
\AgdaOperator{\AgdaDatatype{≡}}\AgdaSpace{}%
\AgdaBound{f}\AgdaSpace{}%
\AgdaBound{b}\<%
\\
%
\>[1]\AgdaFunction{ap}\AgdaSpace{}%
\AgdaBound{f}\AgdaSpace{}%
\AgdaSymbol{\{}\AgdaBound{a}\AgdaSymbol{\}}\AgdaSpace{}%
\AgdaBound{p}\AgdaSpace{}%
\AgdaSymbol{=}\AgdaSpace{}%
\AgdaFunction{transport}\AgdaSpace{}%
\AgdaSymbol{(λ}\AgdaSpace{}%
\AgdaBound{-}\AgdaSpace{}%
\AgdaSymbol{→}\AgdaSpace{}%
\AgdaBound{f}\AgdaSpace{}%
\AgdaBound{a}\AgdaSpace{}%
\AgdaOperator{\AgdaDatatype{≡}}\AgdaSpace{}%
\AgdaBound{f}\AgdaSpace{}%
\AgdaBound{-}\AgdaSymbol{)}\AgdaSpace{}%
\AgdaBound{p}\AgdaSpace{}%
\AgdaSymbol{(}\AgdaInductiveConstructor{refl}\AgdaSpace{}%
\AgdaSymbol{\{}\AgdaArgument{x}\AgdaSpace{}%
\AgdaSymbol{=}\AgdaSpace{}%
\AgdaBound{f}\AgdaSpace{}%
\AgdaBound{a}\AgdaSymbol{\})}\<%
% \\
% %
% \\[\AgdaEmptyExtraSkip]%
% \>[0]\AgdaKeyword{open}\AgdaSpace{}%
% \AgdaKeyword{import}\AgdaSpace{}%
% \AgdaModule{MGS-MLTT}\AgdaSpace{}%
% \AgdaKeyword{using}\AgdaSpace{}%
% \AgdaSymbol{(}\AgdaFunction{ap}\AgdaSymbol{)}\AgdaSpace{}%
% \AgdaKeyword{public}\<%
\end{code}
\ccpad
This establishes that our definitional equality satisfies the substitution rule (subst).

Here's a useful variation of \af{ap} that we borrow from the \texttt{Relation/Binary/Core.agda} module of the \agdastdlib (transcribed into \typtop/\ualib).
\ccpad
\begin{code}%
\>[0]\AgdaFunction{cong-app}\AgdaSpace{}%
\AgdaSymbol{:}%
\>[101I]\AgdaSymbol{\{}\AgdaBound{A}\AgdaSpace{}%
\AgdaSymbol{:}\AgdaSpace{}%
\AgdaBound{𝓤}\AgdaSpace{}%
\AgdaOperator{\AgdaFunction{̇}}\AgdaSymbol{\}\{}\AgdaBound{B}\AgdaSpace{}%
\AgdaSymbol{:}\AgdaSpace{}%
\AgdaBound{A}\AgdaSpace{}%
\AgdaSymbol{→}\AgdaSpace{}%
\AgdaBound{𝓦}\AgdaSpace{}%
\AgdaOperator{\AgdaFunction{̇}}\AgdaSymbol{\}\{}\AgdaBound{f}\AgdaSpace{}%
\AgdaBound{g}\AgdaSpace{}%
\AgdaSymbol{:}\AgdaSpace{}%
\AgdaFunction{Π}\AgdaSpace{}%
\AgdaBound{B}\AgdaSymbol{\}}\AgdaSpace{}%
\AgdaSymbol{→}\AgdaSpace{}%
\AgdaBound{f}\AgdaSpace{}%
% \\
% \>[0][@{}l@{\AgdaIndent{0}}]%
% \>[1]\AgdaSymbol{→}%
% \>[.][@{}l@{}]\<[101I]%
% \>[12]\AgdaBound{f}\AgdaSpace{}%
\AgdaOperator{\AgdaDatatype{≡}}\AgdaSpace{}%
\AgdaBound{g}\AgdaSpace{}%
\AgdaSymbol{→}\AgdaSpace{}%
\AgdaSymbol{(}\AgdaBound{a}\AgdaSpace{}%
\AgdaSymbol{:}\AgdaSpace{}%
\AgdaBound{A}\AgdaSymbol{)}\AgdaSpace{}%
\AgdaSymbol{→}\AgdaSpace{}%
\AgdaBound{f}\AgdaSpace{}%
\AgdaBound{a}\AgdaSpace{}%
\AgdaOperator{\AgdaDatatype{≡}}\AgdaSpace{}%
\AgdaBound{g}\AgdaSpace{}%
\AgdaBound{a}\<%
\\
%
\>[0]\AgdaFunction{cong-app}\AgdaSpace{}%
\AgdaInductiveConstructor{refl}\AgdaSpace{}%
\AgdaSymbol{\AgdaUnderscore{}}\AgdaSpace{}%
\AgdaSymbol{=}\AgdaSpace{}%
\AgdaInductiveConstructor{refl}\<%
\end{code}

% \subsubsection{≡-intro and ≡-elim for nondependent pairs}\label{intro-and-elim-for-nondependent-pairs}

% We conclude our presentation of the \ualibhtml{Overture.Equality} module with some occasionally useful introduction and elimination rules for the equality relation on (nondependent) pair types.
% \ccpad
% \begin{code}%
% \>[0]\AgdaFunction{≡-elim-left}\AgdaSpace{}%
% \AgdaSymbol{:}%
% \>[102I]\AgdaSymbol{\{}\AgdaBound{A₁}\AgdaSpace{}%
% \AgdaBound{A₂}\AgdaSpace{}%
% \AgdaSymbol{:}\AgdaSpace{}%
% \AgdaBound{𝓤}\AgdaSpace{}%
% \AgdaOperator{\AgdaFunction{̇}}\AgdaSymbol{\}\{}\AgdaBound{B₁}\AgdaSpace{}%
% \AgdaBound{B₂}\AgdaSpace{}%
% \AgdaSymbol{:}\AgdaSpace{}%
% \AgdaBound{𝓦}\AgdaSpace{}%
% \AgdaOperator{\AgdaFunction{̇}}\AgdaSymbol{\}}\AgdaSpace{}%
% \AgdaSymbol{→}\AgdaSpace{}%
% \AgdaSymbol{(}\AgdaBound{A₁}\AgdaSpace{}%
% % \\
% % \>[0][@{}l@{\AgdaIndent{0}}]%
% % \>[1]\AgdaSymbol{→}%
% % \>[.][@{}l@{}]\<[102I]%
% % \>[15]\AgdaSymbol{(}\AgdaBound{A₁}\AgdaSpace{}%
% \AgdaOperator{\AgdaInductiveConstructor{,}}\AgdaSpace{}%
% \AgdaBound{B₁}\AgdaSymbol{)}\AgdaSpace{}%
% \AgdaOperator{\AgdaDatatype{≡}}\AgdaSpace{}%
% \AgdaSymbol{(}\AgdaBound{A₂}\AgdaSpace{}%
% \AgdaOperator{\AgdaInductiveConstructor{,}}\AgdaSpace{}%
% \AgdaBound{B₂}\AgdaSymbol{)}\AgdaSpace{}%
% \AgdaSymbol{→}\AgdaSpace{}%
% \AgdaBound{A₁}\AgdaSpace{}%
% \AgdaOperator{\AgdaDatatype{≡}}\AgdaSpace{}%
% \AgdaBound{A₂}\<%
% \\
% %
% \>[0]\AgdaFunction{≡-elim-left}\AgdaSpace{}%
% \AgdaBound{e}\AgdaSpace{}%
% \AgdaSymbol{=}\AgdaSpace{}%
% \AgdaFunction{ap}\AgdaSpace{}%
% \AgdaFunction{pr₁}\AgdaSpace{}%
% \AgdaBound{e}\<%
% \end{code}
% \scpad
% \begin{code}%
% \>[0]\AgdaFunction{≡-elim-right}\AgdaSpace{}%
% \AgdaSymbol{:}%
% \>[105I]\AgdaSymbol{\{}\AgdaBound{A₁}\AgdaSpace{}%
% \AgdaBound{A₂}\AgdaSpace{}%
% \AgdaSymbol{:}\AgdaSpace{}%
% \AgdaBound{𝓤}\AgdaSpace{}%
% \AgdaOperator{\AgdaFunction{̇}}\AgdaSymbol{\}\{}\AgdaBound{B₁}\AgdaSpace{}%
% \AgdaBound{B₂}\AgdaSpace{}%
% \AgdaSymbol{:}\AgdaSpace{}%
% \AgdaBound{𝓦}\AgdaSpace{}%
% \AgdaOperator{\AgdaFunction{̇}}\AgdaSymbol{\}}\AgdaSpace{}%
% \AgdaSymbol{→}\AgdaSpace{}%
% \AgdaSymbol{(}\AgdaBound{A₁}\AgdaSpace{}%
% % \\
% % \>[0][@{}l@{\AgdaIndent{0}}]%
% % \>[1]\AgdaSymbol{→}%
% % \>[.][@{}l@{}]\<[105I]%
% % \>[16]\AgdaSymbol{(}\AgdaBound{A₁}\AgdaSpace{}%
% \AgdaOperator{\AgdaInductiveConstructor{,}}\AgdaSpace{}%
% \AgdaBound{B₁}\AgdaSymbol{)}\AgdaSpace{}%
% \AgdaOperator{\AgdaDatatype{≡}}\AgdaSpace{}%
% \AgdaSymbol{(}\AgdaBound{A₂}\AgdaSpace{}%
% \AgdaOperator{\AgdaInductiveConstructor{,}}\AgdaSpace{}%
% \AgdaBound{B₂}\AgdaSymbol{)}\AgdaSpace{}%
% \AgdaSymbol{→}\AgdaSpace{}%
% \AgdaBound{B₁}\AgdaSpace{}%
% \AgdaOperator{\AgdaDatatype{≡}}\AgdaSpace{}%
% \AgdaBound{B₂}\<%
% \\
% %
% \>[0]\AgdaFunction{≡-elim-right}\AgdaSpace{}%
% \AgdaBound{e}\AgdaSpace{}%
% \AgdaSymbol{=}\AgdaSpace{}%
% \AgdaFunction{ap}\AgdaSpace{}%
% \AgdaFunction{pr₂}\AgdaSpace{}%
% \AgdaBound{e}\<%
% \end{code}
% \scpad
% \begin{code}
% \>[0]\AgdaFunction{≡-×-intro}\AgdaSpace{}%
% \AgdaSymbol{:}%
% \>[106I]\AgdaSymbol{\{}\AgdaBound{A₁}\AgdaSpace{}%
% \AgdaBound{A₂}\AgdaSpace{}%
% \AgdaSymbol{:}\AgdaSpace{}%
% \AgdaBound{𝓤}\AgdaSpace{}%
% \AgdaOperator{\AgdaFunction{̇}}\AgdaSymbol{\}}\AgdaSpace{}%
% \AgdaSymbol{\{}\AgdaBound{B₁}\AgdaSpace{}%
% \AgdaBound{B₂}\AgdaSpace{}%
% \AgdaSymbol{:}\AgdaSpace{}%
% \AgdaBound{𝓦}\AgdaSpace{}%
% \AgdaOperator{\AgdaFunction{̇}}\AgdaSymbol{\}}\AgdaSpace{}%
% % \\
% % \>[0][@{}l@{\AgdaIndent{0}}]%
% % \>[1]\AgdaSymbol{→}%
% % \>[.][@{}l@{}]\<[106I]%
% % \>[13]\AgdaBound{A₁}\AgdaSpace{}%
% \AgdaSymbol{→}\AgdaSpace{}%
% \AgdaBound{A₁}\AgdaSpace{}%
% \AgdaOperator{\AgdaDatatype{≡}}\AgdaSpace{}%
% \AgdaBound{A₂}\AgdaSpace{}%
% \AgdaSymbol{→}\AgdaSpace{}%
% \AgdaBound{B₁}\AgdaSpace{}%
% \AgdaOperator{\AgdaDatatype{≡}}\AgdaSpace{}%
% \AgdaBound{B₂}\AgdaSpace{}%
% \AgdaSymbol{→}\AgdaSpace{}%
% \AgdaSymbol{(}\AgdaBound{A₁}\AgdaSpace{}%
% \AgdaOperator{\AgdaInductiveConstructor{,}}\AgdaSpace{}%
% \AgdaBound{B₁}\AgdaSymbol{)}\AgdaSpace{}%
% \AgdaOperator{\AgdaDatatype{≡}}\AgdaSpace{}%
% \AgdaSymbol{(}\AgdaBound{A₂}\AgdaSpace{}%
% \AgdaOperator{\AgdaInductiveConstructor{,}}\AgdaSpace{}%
% \AgdaBound{B₂}\AgdaSymbol{)}\<%
% \\
% %
% \>[0]\AgdaFunction{≡-×-intro}\AgdaSpace{}%
% \AgdaInductiveConstructor{refl}\AgdaSpace{}%
% \AgdaInductiveConstructor{refl}\AgdaSpace{}%
% \AgdaSymbol{=}\AgdaSpace{}%
% \AgdaInductiveConstructor{refl}\<%
% \end{code}
% \scpad
% \begin{code}
% \>[1]\AgdaFunction{≡-×-int}\AgdaSpace{}%
% \AgdaSymbol{:}%
% \>[99I]\AgdaSymbol{\{}\AgdaBound{A}\AgdaSpace{}%
% \AgdaSymbol{:}\AgdaSpace{}%
% \AgdaBound{𝓤}\AgdaSpace{}%
% \AgdaOperator{\AgdaFunction{̇}}\AgdaSymbol{\}\{}\AgdaBound{B}\AgdaSpace{}%
% \AgdaSymbol{:}\AgdaSpace{}%
% \AgdaBound{𝓦}\AgdaSpace{}%
% \AgdaOperator{\AgdaFunction{̇}}\AgdaSymbol{\}\{}\AgdaBound{a}\AgdaSpace{}%
% \AgdaBound{a'}\AgdaSpace{}%
% \AgdaSymbol{:}\AgdaSpace{}%
% \AgdaBound{A}\AgdaSymbol{\}\{}\AgdaBound{b}\AgdaSpace{}%
% \AgdaBound{b'}\AgdaSpace{}%
% \AgdaSymbol{:}\AgdaSpace{}%
% \AgdaBound{B}\AgdaSymbol{\}}\AgdaSpace{}%
% \AgdaSymbol{→}\AgdaSpace{}%
% \AgdaBound{a}\AgdaSpace{}%
% % \\
% % \>[1][@{}l@{\AgdaIndent{0}}]%
% % \>[2]\AgdaSymbol{→}%
% % \>[.][@{}l@{}]\<[99I]%
% % \>[11]\AgdaBound{a}\AgdaSpace{}%
% \AgdaOperator{\AgdaDatatype{≡}}\AgdaSpace{}%
% \AgdaBound{a'}%
% \>[19]\AgdaSymbol{→}%
% \>[22]\AgdaBound{b}\AgdaSpace{}%
% \AgdaOperator{\AgdaDatatype{≡}}\AgdaSpace{}%
% \AgdaBound{b'}\AgdaSpace{}%
% \AgdaSymbol{→}\AgdaSpace{}%
% \AgdaSymbol{(}\AgdaBound{a}\AgdaSpace{}%
% \AgdaOperator{\AgdaInductiveConstructor{,}}\AgdaSpace{}%
% \AgdaBound{b}\AgdaSymbol{)}\AgdaSpace{}%
% \AgdaOperator{\AgdaDatatype{≡}}\AgdaSpace{}%
% \AgdaSymbol{(}\AgdaBound{a'}\AgdaSpace{}%
% \AgdaOperator{\AgdaInductiveConstructor{,}}\AgdaSpace{}%
% \AgdaBound{b'}\AgdaSymbol{)}\<%
% \\
% %
% \>[1]\AgdaFunction{≡-×-int}\AgdaSpace{}%
% \AgdaInductiveConstructor{refl}\AgdaSpace{}%
% \AgdaInductiveConstructor{refl}\AgdaSpace{}%
% \AgdaSymbol{=}\AgdaSpace{}%
% \AgdaInductiveConstructor{refl}\<%
% \end{code}

% \begin{code}%
% \>[0]\AgdaFunction{≡-×-int}\AgdaSpace{}%
% \AgdaSymbol{:}%
% \[111I]\AgdaSymbol{\{}\AgdaBound{A}\AgdaSpace{}%
% \AgdaSymbol{:}\AgdaSpace{}%
% \AgdaBound{𝓤}\AgdaSpace{}%
% \AgdaOperator{\AgdaFunction{̇}}\AgdaSymbol{\}\{}\AgdaBound{B}\AgdaSpace{}%
% \AgdaSymbol{:}\AgdaSpace{}%
% \AgdaBound{𝓦}\AgdaSpace{}%
% \AgdaOperator{\AgdaFunction{̇}}\AgdaSymbol{\}\{}\AgdaBound{a}\AgdaSpace{}%
% \AgdaBound{u}\AgdaSpace{}%
% \AgdaSymbol{:}\AgdaSpace{}%
% \AgdaBound{A}\AgdaSymbol{\}\{}\AgdaBound{b}\AgdaSpace{}%
% \AgdaBound{v}\AgdaSpace{}%
% \AgdaSymbol{:}\AgdaSpace{}%
% \AgdaBound{B}\AgdaSymbol{\}}\<%
% \\
% \\
% \>[0][@{}l@{\AgdaIndent{0}}]%
% \>[1]\AgdaSymbol{→}%
% \>[.][@{}l@{}]\<[111I]%
% \>[11]\AgdaBound{a}\AgdaSpace{}%
% \AgdaOperator{\AgdaDatatype{≡}}\AgdaSpace{}%
% \AgdaBound{u}\AgdaSpace{}%
% \AgdaSymbol{→}\AgdaSpace{}%
% \AgdaBound{b}\AgdaSpace{}%
% \AgdaOperator{\AgdaDatatype{≡}}\AgdaSpace{}%
% \AgdaBound{v}\<%
% \\
% %
% \>[11]\AgdaComment{---------------------------}\<%
% \\
% %
% \>[1]\AgdaSymbol{→}%
% \>[11]\AgdaSymbol{(}\AgdaBound{a}\AgdaSpace{}%
% \AgdaOperator{\AgdaInductiveConstructor{,}}\AgdaSpace{}%
% \AgdaBound{b}\AgdaSymbol{)}\AgdaSpace{}%
% \AgdaOperator{\AgdaDatatype{≡}}\AgdaSpace{}%
% \AgdaSymbol{(}\AgdaBound{u}\AgdaSpace{}%
% \AgdaOperator{\AgdaInductiveConstructor{,}}\AgdaSpace{}%
% \AgdaBound{v}\AgdaSymbol{)}\<%
% \\
% %
% \\[\AgdaEmptyExtraSkip]%
% %
% \>[0]\AgdaFunction{≡-×-int}\AgdaSpace{}%
% \AgdaInductiveConstructor{refl}\AgdaSpace{}%
% \AgdaInductiveConstructor{refl}\AgdaSpace{}%
% \AgdaSymbol{=}\AgdaSpace{}%
% \AgdaInductiveConstructor{refl}\<%
% \end{code}


\subsection{Function extensionality}\label{function-extensionality}\firstsentence{Overture}{Extensionality}
% : types for postulating function extensionality
% \context{\{\ab 𝓤 \ab 𝓦 \as : \ap{Universe}\}}
% -*- TeX-master: "ualib-part1.tex" -*-
%%% Local Variables: 
%%% mode: latex
%%% TeX-engine: 'xetex
%%% End:
\subsubsection{Background and motivation}\label{background-and-motivation}
This brief introduction to \emph{function extensionality} is intended for novices. Those already familiar with the concept might wish to skip to the next subsection.

What does it mean to say that two functions \(f\ g \colon X → Y\) are equal? Suppose $f$ and \(g\) are defined on \(X = ℤ\) (the integers) as follows: \(f x := x + 2\) and \(g x := ((2 * x) - 8)/2 + 6\). Should we call \(f\) and \(g\) equal?  Are they the ``same'' function?  What does that even mean?

It's hard to resist the urge to reduce \(g\) to \(x + 2\) and proclaim that \(f\) and \(g\) are equal. Indeed, this is often an acceptable answer and the discussion normally ends there.  In the science of computing, however, more attention is paid to equality, and with good reason.

We can probably all agree that the functions \(f\) and \(g\) above, while not syntactically equal, do produce the same output when given the same input so it seems fine to think of the functions as the same, for all intents and purposes. But we should ask ourselves at what point do we notice or care about the difference in the way functions are defined? What if we had started out this discussion with two functions \(f\) and \(g\) both of which take a list as argument and produce as output a correctly sorted version of the input list?  Suppose \(f\) is defined using the \href{https://en.wikipedia.org/wiki/Merge_sort}{merge sort} algorithm, while \(g\) uses \href{https://en.wikipedia.org/wiki/Quicksort}{quick sort}. Probably few of us would call \(f\) and \(g\) the ``same'' in this case.

In the examples above, it is common to say that the two functions are \href{https://en.wikipedia.org/wiki/Extensionality}{\defn{extensionally equal}}, since they produce the same \emph{external} output when given the same input, but they are not \href{https://en.wikipedia.org/wiki/Intension}{\defn{intensionally equal}}, since their \emph{internal} definitions differ. In the next subsection we describe types that manifest this idea of \emph{extensional equality of functions}, or \emph{function extensionality}.\footnote{Most of these types are already defined in \typtop, so the \ualib imports the definitions from there; as usual, we redefine some of these types here for the purpose of explication.}

\subsubsection{Definition of function extensionality}\label{sec:defin-funct-extens}

As explained above, a natural notion of function equality is defined as follows: \(f\) and \(g\) are said to be \emph{pointwise equal} provided \(∀ x → f x ≡ g x\). Here is how this is expressed in type theory (e.g., in \typtop).
\ccpad
\begin{code}%
\>[1]\AgdaOperator{\AgdaFunction{\AgdaUnderscore{}∼\AgdaUnderscore{}}}\AgdaSpace{}%
\AgdaSymbol{:}\AgdaSpace{}%
\AgdaSymbol{\{}\AgdaBound{A}\AgdaSpace{}%
\AgdaSymbol{:}\AgdaSpace{}%
\AgdaBound{𝓤}\AgdaSpace{}%
\AgdaOperator{\AgdaFunction{̇}}\AgdaSpace{}%
\AgdaSymbol{\}}\AgdaSpace{}%
\AgdaSymbol{\{}\AgdaBound{B}\AgdaSpace{}%
\AgdaSymbol{:}\AgdaSpace{}%
\AgdaBound{A}\AgdaSpace{}%
\AgdaSymbol{→}\AgdaSpace{}%
\AgdaBound{𝓦}\AgdaSpace{}%
\AgdaOperator{\AgdaFunction{̇}}\AgdaSpace{}%
\AgdaSymbol{\}}\AgdaSpace{}%
\AgdaSymbol{→}\AgdaSpace{}%
\AgdaFunction{Π}\AgdaSpace{}%
\AgdaBound{B}\AgdaSpace{}%
\AgdaSymbol{→}\AgdaSpace{}%
\AgdaFunction{Π}\AgdaSpace{}%
\AgdaBound{B}\AgdaSpace{}%
\AgdaSymbol{→}\AgdaSpace{}%
\AgdaBound{𝓤}\AgdaSpace{}%
\AgdaOperator{\AgdaPrimitive{⊔}}\AgdaSpace{}%
\AgdaBound{𝓦}\AgdaSpace{}%
\AgdaOperator{\AgdaFunction{̇}}\<%
\\
%
\>[1]\AgdaBound{f}\AgdaSpace{}%
\AgdaOperator{\AgdaFunction{∼}}\AgdaSpace{}%
\AgdaBound{g}\AgdaSpace{}%
\AgdaSymbol{=}\AgdaSpace{}%
\AgdaSymbol{∀}\AgdaSpace{}%
\AgdaBound{x}\AgdaSpace{}%
\AgdaSymbol{→}\AgdaSpace{}%
\AgdaBound{f}\AgdaSpace{}%
\AgdaBound{x}\AgdaSpace{}%
\AgdaOperator{\AgdaDatatype{≡}}\AgdaSpace{}%
\AgdaBound{g}\AgdaSpace{}%
\AgdaBound{x}\<%
\end{code}
\ccpad
\defn{Function extensionality} is the principle that pointwise equal functions are \defn{definitionally equal}; that is, \(\forall\ f\ g\ (f ∼ g → f ≡ g)\). In type theory we represent this principle by \af{funext} (for nondependent functions) and \af{dfunext} (for dependent functions), which are defined as follows.
\ccpad
\begin{code}%
\>[1]\AgdaFunction{funext}\AgdaSpace{}%
\AgdaSymbol{:}\AgdaSpace{}%
\AgdaSymbol{∀}\AgdaSpace{}%
\AgdaBound{𝓤}\AgdaSpace{}%
\AgdaBound{𝓦}%
\>[17]\AgdaSymbol{→}\AgdaSpace{}%
\as (\AgdaBound{𝓤}\AgdaSpace{}%
\AgdaOperator{\AgdaPrimitive{⊔}}\AgdaSpace{}%
\AgdaBound{𝓦}\as )\AgdaSpace{}%
\AgdaOperator{\AgdaPrimitive{⁺}}\AgdaSpace{}%
\AgdaOperator{\AgdaFunction{̇}}\<%
\\
%
\>[1]\AgdaFunction{funext}\AgdaSpace{}%
\AgdaBound{𝓤}\AgdaSpace{}%
\AgdaBound{𝓦}\AgdaSpace{}%
\AgdaSymbol{=}\AgdaSpace{}%
\AgdaSymbol{\{}\AgdaBound{A}\AgdaSpace{}%
\AgdaSymbol{:}\AgdaSpace{}%
\AgdaBound{𝓤}\AgdaSpace{}%
\AgdaOperator{\AgdaFunction{̇}}\AgdaSpace{}%
\AgdaSymbol{\}}\AgdaSpace{}%
\AgdaSymbol{\{}\AgdaBound{B}\AgdaSpace{}%
\AgdaSymbol{:}\AgdaSpace{}%
\AgdaBound{𝓦}\AgdaSpace{}%
\AgdaOperator{\AgdaFunction{̇}}\AgdaSpace{}%
\AgdaSymbol{\}}\AgdaSpace{}%
\AgdaSymbol{\{}\AgdaBound{f}\AgdaSpace{}%
\AgdaBound{g}\AgdaSpace{}%
\AgdaSymbol{:}\AgdaSpace{}%
\AgdaBound{A}\AgdaSpace{}%
\AgdaSymbol{→}\AgdaSpace{}%
\AgdaBound{B}\AgdaSymbol{\}}\AgdaSpace{}%
\AgdaSymbol{→}\AgdaSpace{}%
\AgdaBound{f}\AgdaSpace{}%
\AgdaOperator{\AgdaFunction{∼}}\AgdaSpace{}%
\AgdaBound{g}\AgdaSpace{}%
\AgdaSymbol{→}\AgdaSpace{}%
\AgdaBound{f}\AgdaSpace{}%
\AgdaOperator{\AgdaDatatype{≡}}\AgdaSpace{}%
\AgdaBound{g}\<%
\end{code}
\scpad
\begin{code}%
\>[1]\AgdaFunction{dfunext}\AgdaSpace{}%
\AgdaSymbol{:}\AgdaSpace{}%
\AgdaSymbol{∀}\AgdaSpace{}%
\AgdaBound{𝓤}\AgdaSpace{}%
\AgdaBound{𝓦}\AgdaSpace{}%
\AgdaSymbol{→}\AgdaSpace{}%
\as (\AgdaBound{𝓤}\AgdaSpace{}%
\AgdaOperator{\AgdaPrimitive{⊔}}\AgdaSpace{}%
\AgdaBound{𝓦}\as )\AgdaSpace{}%
\AgdaOperator{\AgdaPrimitive{⁺}}\AgdaSpace{}%
\AgdaOperator{\AgdaFunction{̇}}\<%
\\
%
\>[1]\AgdaFunction{dfunext}\AgdaSpace{}%
\AgdaBound{𝓤}\AgdaSpace{}%
\AgdaBound{𝓦}\AgdaSpace{}%
\AgdaSymbol{=}\AgdaSpace{}%
\AgdaSymbol{\{}\AgdaBound{A}\AgdaSpace{}%
\AgdaSymbol{:}\AgdaSpace{}%
\AgdaBound{𝓤}\AgdaSpace{}%
\AgdaOperator{\AgdaFunction{̇}}\AgdaSpace{}%
\AgdaSymbol{\}\{}\AgdaBound{B}\AgdaSpace{}%
\AgdaSymbol{:}\AgdaSpace{}%
\AgdaBound{A}\AgdaSpace{}%
\AgdaSymbol{→}\AgdaSpace{}%
\AgdaBound{𝓦}\AgdaSpace{}%
\AgdaOperator{\AgdaFunction{̇}}\AgdaSpace{}%
\AgdaSymbol{\}\{}\AgdaBound{f}\AgdaSpace{}%
\AgdaBound{g}\AgdaSpace{}%
\AgdaSymbol{:}\AgdaSpace{}%
\AgdaSymbol{∀(}\AgdaBound{x}\AgdaSpace{}%
\AgdaSymbol{:}\AgdaSpace{}%
\AgdaBound{A}\AgdaSymbol{)}\AgdaSpace{}%
\AgdaSymbol{→}\AgdaSpace{}%
\AgdaBound{B}\AgdaSpace{}%
\AgdaBound{x}\AgdaSymbol{\}}\AgdaSpace{}%
\AgdaSymbol{→}%
\>[65]\AgdaBound{f}\AgdaSpace{}%
\AgdaOperator{\AgdaFunction{∼}}\AgdaSpace{}%
\AgdaBound{g}%
\>[72]\AgdaSymbol{→}%
\>[75]\AgdaBound{f}\AgdaSpace{}%
\AgdaOperator{\AgdaDatatype{≡}}\AgdaSpace{}%
\AgdaBound{g}\<%
\end{code}
\ccpad
In informal settings, this so-called \emph{point-wise equality of functions} is typically what one means when one
asserts that two functions are ``equal.''\footnote{In fact, if one assumes the \emph{univalence axiom} of Homotopy Type Theory~\cite{HoTT}, then point-wise equality of functions is equivalent to definitional equality of functions. See the section \href{https://www.cs.bham.ac.uk/~mhe/HoTT-UF-in-Agda-Lecture-Notes/HoTT-UF-Agda.html\#funextfromua}{``Function extensionality from univalence''} of~\cite{MHE}.} However, it is important to keep in mind the following fact: \textit{function extensionality is known to be neither provable nor disprovable in Martin-Löf type theory. It is an independent statement}.~\cite{MHE}

\subsubsection{Global function extensionality}\label{sec:glob-funct-extens}

An assumption that we adopt throughout much of the current version of the \ualib is a \emph{global function extensionality principle}. This asserts that function extensionality holds at all universe levels. Agda is capable of expressing types representing global principles as the language has a special universe level for such types.
Following Escardó, we denote this universe by \AgdaPrimitive{𝓤ω} (which is just an alias for Agda's \ad{Setω} universe).\footnote{More details about the \ad{𝓤ω} type are available at
\href{https://agda.readthedocs.io/en/latest/language/universe-levels.html\#expressions-of-kind-set}{agda.readthedocs.io}.}
The types \af{global-funext} and \af{global-dfunext} are defined in \typtop as follows.
\ccpad
\begin{code}%
\>[1]\AgdaFunction{global-funext}\AgdaSpace{}%
\AgdaSymbol{:}\AgdaSpace{}%
\AgdaPrimitive{𝓤ω}\<%
\\
%
\>[1]\AgdaFunction{global-funext}\AgdaSpace{}%
\AgdaSymbol{=}\AgdaSpace{}%
\AgdaSymbol{∀}%
\>[20]\AgdaSymbol{\{}\AgdaBound{𝓤}\AgdaSpace{}%
\AgdaBound{𝓥}\AgdaSymbol{\}}\AgdaSpace{}%
\AgdaSymbol{→}\AgdaSpace{}%
\AgdaFunction{funext}\AgdaSpace{}%
\AgdaBound{𝓤}\AgdaSpace{}%
\AgdaBound{𝓥}\<%
\\
%
\\[\AgdaEmptyExtraSkip]%
%
\>[1]\AgdaFunction{global-dfunext}\AgdaSpace{}%
\AgdaSymbol{:}\AgdaSpace{}%
\AgdaPrimitive{𝓤ω}\<%
\\
%
\>[1]\AgdaFunction{global-dfunext}\AgdaSpace{}%
\AgdaSymbol{=}\AgdaSpace{}%
\AgdaSymbol{∀}\AgdaSpace{}%
\AgdaSymbol{\{}\AgdaBound{𝓤}\AgdaSpace{}%
\AgdaBound{𝓥}\AgdaSymbol{\}}\AgdaSpace{}%
\AgdaSymbol{→}\AgdaSpace{}%
\AgdaFunction{dfunext}\AgdaSpace{}%
\AgdaBound{𝓤}\AgdaSpace{}%
\AgdaBound{𝓥}\<%
\end{code}
\ccpad
The next two types define the converse of function extensionality.
\ccpad
\begin{code}%
\>[1]\AgdaFunction{extfun}\AgdaSpace{}%
\AgdaSymbol{:}\AgdaSpace{}%
\AgdaSymbol{\{}\AgdaBound{A}\AgdaSpace{}%
\AgdaSymbol{:}\AgdaSpace{}%
\AgdaBound{𝓤}\AgdaSpace{}%
\AgdaOperator{\AgdaFunction{̇}}\AgdaSymbol{\}\{}\AgdaBound{B}\AgdaSpace{}%
\AgdaSymbol{:}\AgdaSpace{}%
\AgdaBound{𝓦}\AgdaSpace{}%
\AgdaOperator{\AgdaFunction{̇}}\AgdaSymbol{\}\{}\AgdaBound{f}\AgdaSpace{}%
\AgdaBound{g}\AgdaSpace{}%
\AgdaSymbol{:}\AgdaSpace{}%
\AgdaBound{A}\AgdaSpace{}%
\AgdaSymbol{→}\AgdaSpace{}%
\AgdaBound{B}\AgdaSymbol{\}}\AgdaSpace{}%
\AgdaSymbol{→}\AgdaSpace{}%
\AgdaBound{f}\AgdaSpace{}%
\AgdaOperator{\AgdaDatatype{≡}}\AgdaSpace{}%
\AgdaBound{g}%
\>[51]\AgdaSymbol{→}%
\>[54]\AgdaBound{f}\AgdaSpace{}%
\AgdaOperator{\AgdaFunction{∼}}\AgdaSpace{}%
\AgdaBound{g}\<%
\\
%
\>[1]\AgdaFunction{extfun}\AgdaSpace{}%
\AgdaInductiveConstructor{refl}\AgdaSpace{}%
\AgdaSymbol{\AgdaUnderscore{}}\AgdaSpace{}%
\AgdaSymbol{=}\AgdaSpace{}%
\AgdaInductiveConstructor{refl}\<%
\end{code}
\scpad
\begin{code}%
\>[1]\AgdaFunction{extdfun}\AgdaSpace{}%
\AgdaSymbol{:}\AgdaSpace{}%
\AgdaSymbol{\{}\AgdaBound{A}\AgdaSpace{}%
\AgdaSymbol{:}\AgdaSpace{}%
\AgdaBound{𝓤}\AgdaSpace{}%
\AgdaOperator{\AgdaFunction{̇}}\AgdaSpace{}%
\AgdaSymbol{\}\{}\AgdaBound{B}\AgdaSpace{}%
\AgdaSymbol{:}\AgdaSpace{}%
\AgdaBound{A}\AgdaSpace{}%
\AgdaSymbol{→}\AgdaSpace{}%
\AgdaBound{𝓦}\AgdaSpace{}%
\AgdaOperator{\AgdaFunction{̇}}\AgdaSpace{}%
\AgdaSymbol{\}(}\AgdaBound{f}\AgdaSpace{}%
\AgdaBound{g}\AgdaSpace{}%
\AgdaSymbol{:}\AgdaSpace{}%
\AgdaFunction{Π}\AgdaSpace{}%
\AgdaBound{B}\AgdaSymbol{)}\AgdaSpace{}%
\AgdaSymbol{→}\AgdaSpace{}%
\AgdaBound{f}\AgdaSpace{}%
\AgdaOperator{\AgdaDatatype{≡}}\AgdaSpace{}%
\AgdaBound{g}\AgdaSpace{}%
\AgdaSymbol{→}\AgdaSpace{}%
\AgdaBound{f}\AgdaSpace{}%
\AgdaOperator{\AgdaFunction{∼}}\AgdaSpace{}%
\AgdaBound{g}\<%
\\
%
\>[1]\AgdaFunction{extdfun}\AgdaSpace{}%
\AgdaSymbol{\AgdaUnderscore{}}\AgdaSpace{}%
\AgdaSymbol{\AgdaUnderscore{}}\AgdaSpace{}%
\AgdaInductiveConstructor{refl}\AgdaSpace{}%
\AgdaSymbol{\AgdaUnderscore{}}\AgdaSpace{}%
\AgdaSymbol{=}\AgdaSpace{}%
\AgdaInductiveConstructor{refl}\<%
\end{code}
\scpad

Though it may seem obvious to some readers, we wish to emphasize the important conceptual distinction between two flavors of type definition.  We do so by comparing the definitions of \af{funext} and \af{extfun}.

In the definition of \af{funext}, the codomain is a parameterized type, namely, \ab 𝓤\af ⁺~\apr ⊔~\ab 𝓥\af ⁺\af ̇, and the right-hand side of the defining equation of \af{funext} is an assertion (which may or may not hold). In the definition of \af{extfun}, the codomain is an assertion, namely, \ab f \af ∼ \ab g, and the right-hand side of the defining equation is a proof of this assertion. As such, \af{extfun} is a \defn{proof object}; it \emph{proves} (or \emph{inhabits the type that represents}) the proposition asserting that definitionally equivalent functions are pointwise equal. In contrast, \af{funext} is a type, and we may or may not wish to postulate an inhabitant of this type. That is, we could postulate that function extensionality holds by assuming we have a witness, say, \ab{fe}~\as :~\af{funext}~\ab 𝓤~\ab 𝓥, but as noted above the existence of such a witness cannot be proved in \mltt.

\subsubsection{An alternative extensionality type}\label{alternative-extensionality-type}
Finally, a useful alternative for expressing dependent function extensionality, which is essentially equivalent to \af{dfunext}, is to assert that \af{extdfun} is actually an \emph{equivalence} in a sense that we now describe. This requires a few definitions from the \am{MGS-Equivalences} module of \typtop. First, a type is a \defn{singleton} if it has exactly one inhabitant and a \defn{subsingleton} if it has at most one inhabitant.
 % These properties are represented in the \typetopology library by the following type.
\ccpad
\begin{code}%
\>[1]\AgdaFunction{is-center}\AgdaSpace{}%
\AgdaSymbol{:}\AgdaSpace{}%
\AgdaSymbol{(}\AgdaBound{A}\AgdaSpace{}%
\AgdaSymbol{:}\AgdaSpace{}%
\AgdaBound{𝓤}\AgdaSpace{}%
\AgdaOperator{\AgdaFunction{̇}}\AgdaSpace{}%
\AgdaSymbol{)}\AgdaSpace{}%
\AgdaSymbol{→}\AgdaSpace{}%
\AgdaBound{A}\AgdaSpace{}%
\AgdaSymbol{→}\AgdaSpace{}%
\AgdaBound{𝓤}\AgdaSpace{}%
\AgdaOperator{\AgdaFunction{̇}}\<%
\\
%
\>[1]\AgdaFunction{is-center}\AgdaSpace{}%
\AgdaBound{A}\AgdaSpace{}%
\AgdaBound{c}\AgdaSpace{}%
\AgdaSymbol{=}\AgdaSpace{}%
\AgdaSymbol{(}\AgdaBound{x}\AgdaSpace{}%
\AgdaSymbol{:}\AgdaSpace{}%
\AgdaBound{A}\AgdaSymbol{)}\AgdaSpace{}%
\AgdaSymbol{→}\AgdaSpace{}%
\AgdaBound{c}\AgdaSpace{}%
\AgdaOperator{\AgdaDatatype{≡}}\AgdaSpace{}%
\AgdaBound{x}\<%
\end{code}
\scpad
\begin{code}
\>[1]\AgdaFunction{is-singleton}\AgdaSpace{}%
\AgdaSymbol{:}\AgdaSpace{}%
\AgdaBound{𝓤}\AgdaSpace{}%
\AgdaOperator{\AgdaFunction{̇}}\AgdaSpace{}%
\AgdaSymbol{→}\AgdaSpace{}%
\AgdaBound{𝓤}\AgdaSpace{}%
\AgdaOperator{\AgdaFunction{̇}}\<%
\\
%
\>[1]\AgdaFunction{is-singleton}\AgdaSpace{}%
\AgdaBound{A}\AgdaSpace{}%
\AgdaSymbol{=}\AgdaSpace{}%
\AgdaFunction{Σ}\AgdaSpace{}%
\AgdaBound{c}\AgdaSpace{}%
\AgdaFunction{꞉}\AgdaSpace{}%
\AgdaBound{A}\AgdaSpace{}%
\AgdaFunction{,}\AgdaSpace{}%
\AgdaFunction{is-center}\AgdaSpace{}%
\AgdaBound{A}\AgdaSpace{}%
\AgdaBound{c}\<%
\end{code}
\scpad
\begin{code}
\>[1]\AgdaFunction{is-subsingleton}\AgdaSpace{}%
\AgdaSymbol{:}\AgdaSpace{}%
\AgdaBound{𝓤}\AgdaSpace{}%
\AgdaOperator{\AgdaFunction{̇}}\AgdaSpace{}%
\AgdaSymbol{→}\AgdaSpace{}%
\AgdaBound{𝓤}\AgdaSpace{}%
\AgdaOperator{\AgdaFunction{̇}}\<%
\\
%
\>[1]\AgdaFunction{is-subsingleton}\AgdaSpace{}%
\AgdaBound{A}\AgdaSpace{}%
\AgdaSymbol{=}\AgdaSpace{}%
\AgdaSymbol{(}\AgdaBound{x}\AgdaSpace{}%
\AgdaBound{y}\AgdaSpace{}%
\AgdaSymbol{:}\AgdaSpace{}%
\AgdaBound{A}\AgdaSymbol{)}\AgdaSpace{}%
\AgdaSymbol{→}\AgdaSpace{}%
\AgdaBound{x}\AgdaSpace{}%
\AgdaOperator{\AgdaDatatype{≡}}\AgdaSpace{}%
\AgdaBound{y}\<%
\end{code}
\ccpad
Next, we consider the type \af{is-equiv} which is used to assert that a function is an equivalence in the sense that we now describe. This requires the concept of a \href{https://ncatlab.org/nlab/show/fiber}{\defn{fiber}} of a function, which can be represented as a Sigma type whose inhabitants denote inverse images of points in the codomain of the given function.
\ccpad
\begin{code}%
\>[1]\AgdaFunction{fiber}\AgdaSpace{}%
\AgdaSymbol{:}\AgdaSpace{}%
\AgdaSymbol{\{}\AgdaBound{A}\AgdaSpace{}%
\AgdaSymbol{:}\AgdaSpace{}%
\AgdaBound{𝓤}\AgdaSpace{}%
\AgdaOperator{\AgdaFunction{̇}}%
\AgdaSymbol{\}}\AgdaSpace{}%
\AgdaSymbol{\{}\AgdaBound{B}\AgdaSpace{}%
\AgdaSymbol{:}\AgdaSpace{}%
\AgdaBound{𝓦}\AgdaSpace{}%
\AgdaOperator{\AgdaFunction{̇}}%
\AgdaSymbol{\}}\AgdaSpace{}%
\AgdaSymbol{(}\AgdaBound{f}\AgdaSpace{}%
\AgdaSymbol{:}\AgdaSpace{}%
\AgdaBound{A}\AgdaSpace{}%
\AgdaSymbol{→}\AgdaSpace{}%
\AgdaBound{B}\AgdaSymbol{)}\AgdaSpace{}%
\AgdaSymbol{→}\AgdaSpace{}%
\AgdaBound{B}\AgdaSpace{}%
\AgdaSymbol{→}\AgdaSpace{}%
\AgdaBound{𝓤}\AgdaSpace{}%
\AgdaOperator{\AgdaPrimitive{⊔}}\AgdaSpace{}%
\AgdaBound{𝓦}\AgdaSpace{}%
\AgdaOperator{\AgdaFunction{̇}}\<%
\\
%
\>[1]\AgdaFunction{fiber}\AgdaSpace{}%
\AgdaSymbol{\{}\AgdaBound{A}\AgdaSymbol{\}}\AgdaSpace{}%
\AgdaBound{f}\AgdaSpace{}%
\AgdaBound{y}\AgdaSpace{}%
\AgdaSymbol{=}\AgdaSpace{}%
\AgdaFunction{Σ}\AgdaSpace{}%
\AgdaBound{x}\AgdaSpace{}%
\AgdaFunction{꞉}\AgdaSpace{}%
\AgdaBound{A}\AgdaSpace{}%
\AgdaFunction{,}\AgdaSpace{}%
\AgdaBound{f}\AgdaSpace{}%
\AgdaBound{x}\AgdaSpace{}%
\AgdaOperator{\AgdaDatatype{≡}}\AgdaSpace{}%
\AgdaBound{y}\<%
\end{code}
\ccpad
A function is called an \defn{equivalence} if all of its fibers are singletons.
\ccpad
\begin{code}%
\>[1]\AgdaFunction{is-equiv}\AgdaSpace{}%
\AgdaSymbol{:}\AgdaSpace{}%
\AgdaSymbol{\{}\AgdaBound{A}\AgdaSpace{}%
\AgdaSymbol{:}\AgdaSpace{}%
\AgdaBound{𝓤}\AgdaSpace{}%
\AgdaOperator{\AgdaFunction{̇}}%
\AgdaSymbol{\}}\AgdaSpace{}%
\AgdaSymbol{\{}\AgdaBound{B}\AgdaSpace{}%
\AgdaSymbol{:}\AgdaSpace{}%
\AgdaBound{𝓦}\AgdaSpace{}%
\AgdaOperator{\AgdaFunction{̇}}%
\AgdaSymbol{\}}\AgdaSpace{}%
\AgdaSymbol{→}\AgdaSpace{}%
\AgdaSymbol{(}\AgdaBound{A}\AgdaSpace{}%
\AgdaSymbol{→}\AgdaSpace{}%
\AgdaBound{B}\AgdaSymbol{)}\AgdaSpace{}%
\AgdaSymbol{→}\AgdaSpace{}%
\AgdaBound{𝓤}\AgdaSpace{}%
\AgdaOperator{\AgdaPrimitive{⊔}}\AgdaSpace{}%
\AgdaBound{𝓦}\AgdaSpace{}%
\AgdaOperator{\AgdaFunction{̇}}\<%
\\
%
\>[1]\AgdaFunction{is-equiv}\AgdaSpace{}%
\AgdaBound{f}\AgdaSpace{}%
\AgdaSymbol{=}\AgdaSpace{}%
\AgdaSymbol{∀}\AgdaSpace{}%
\AgdaBound{y}\AgdaSpace{}%
\AgdaSymbol{→}\AgdaSpace{}%
\AgdaFunction{is-singleton}\AgdaSpace{}%
\AgdaSymbol{(}\AgdaFunction{fiber}\AgdaSpace{}%
\AgdaBound{f}\AgdaSpace{}%
\AgdaBound{y}\AgdaSymbol{)}\<%
\end{code}
\ccpad
Finally we are ready to fulfill the promise of a type that provides an alternative means of postulating function extensionality.
\ccpad
\begin{code}%
\>[1]\AgdaFunction{hfunext}\AgdaSpace{}%
\AgdaSymbol{:}\AgdaSpace{}%
\AgdaSymbol{∀}\AgdaSpace{}
\AgdaBound{𝓤}\AgdaSpace{}%
\AgdaBound{𝓦}\AgdaSpace{}%
\AgdaSymbol{→}\AgdaSpace{}%
\AgdaSymbol{(}\AgdaBound{𝓤}\AgdaSpace{}%
\AgdaOperator{\AgdaPrimitive{⊔}}\AgdaSpace{}%
\AgdaBound{𝓦}\AgdaSymbol{)}\AgdaOperator{\AgdaPrimitive{⁺}}\AgdaSpace{}%
\AgdaOperator{\AgdaFunction{̇}}\<%
\\
%
\>[1]\AgdaFunction{hfunext}\AgdaSpace{}%
\AgdaBound{𝓤}\AgdaSpace{}%
\AgdaBound{𝓦}\AgdaSpace{}%
\AgdaSymbol{=}\AgdaSpace{}%
\AgdaSymbol{\{}\AgdaBound{A}\AgdaSpace{}%
\AgdaSymbol{:}\AgdaSpace{}%
\AgdaBound{𝓤}\AgdaSpace{}%
\AgdaOperator{\AgdaFunction{̇}}\AgdaSymbol{\}\{}\AgdaBound{B}\AgdaSpace{}%
\AgdaSymbol{:}\AgdaSpace{}%
\AgdaBound{A}\AgdaSpace{}%
\AgdaSymbol{→}\AgdaSpace{}%
\AgdaBound{𝓦}\AgdaSpace{}%
\AgdaOperator{\AgdaFunction{̇}}\AgdaSymbol{\}}\AgdaSpace{}%
\AgdaSymbol{(}\AgdaBound{f}\AgdaSpace{}%
\AgdaBound{g}\AgdaSpace{}%
\AgdaSymbol{:}\AgdaSpace{}%
\AgdaFunction{Π}\AgdaSpace{}%
\AgdaBound{B}\AgdaSymbol{)}\AgdaSpace{}%
\AgdaSymbol{→}\AgdaSpace{}%
\AgdaFunction{is-equiv}\AgdaSpace{}%
\AgdaSymbol{(}\AgdaFunction{extdfun}\AgdaSpace{}%
\AgdaBound{f}\AgdaSpace{}%
\AgdaBound{g}\AgdaSymbol{)}\<%
\end{code}

\subsection{Inverses}\label{sec:inverse-image-invers}\firstsentence{Overture}{Inverses}
% : epics, monics, embeddings, inverse images
% \contextshort{\{\ab 𝓤 \ab 𝓦 \as : \ap{Universe}\}\{\ab A \as : \ab 𝓤\af ̇\}\{\ab B \as : \ab 𝓦\af ̇\}}
We begin by defining an inductive type that represents the semantic concept of \defn{inverse image} of a function.
\ccpad
\begin{code}%
\>[0][@{}l@{\AgdaIndent{0}}]%
\>[1]\AgdaKeyword{data}\AgdaSpace{}%
\AgdaOperator{\AgdaDatatype{Image\AgdaUnderscore{}∋\AgdaUnderscore{}}}\AgdaSpace{}%
\AgdaSymbol{\{}\AgdaBound{A}\AgdaSpace{}%
\AgdaSymbol{:}\AgdaSpace{}%
\AgdaBound{𝓤}\AgdaSpace{}%
\AgdaOperator{\AgdaFunction{̇}}\AgdaSpace{}%
\AgdaSymbol{\}\{}\AgdaBound{B}\AgdaSpace{}%
\AgdaSymbol{:}\AgdaSpace{}%
\AgdaBound{𝓦}\AgdaSpace{}%
\AgdaOperator{\AgdaFunction{̇}}\AgdaSpace{}%
\AgdaSymbol{\}(}\AgdaBound{f}\AgdaSpace{}%
\AgdaSymbol{:}\AgdaSpace{}%
\AgdaBound{A}\AgdaSpace{}%
\AgdaSymbol{→}\AgdaSpace{}%
\AgdaBound{B}\AgdaSymbol{)}\AgdaSpace{}%
\AgdaSymbol{:}\AgdaSpace{}%
\AgdaBound{B}\AgdaSpace{}%
\AgdaSymbol{→}\AgdaSpace{}%
\AgdaBound{𝓤}\AgdaSpace{}%
\AgdaOperator{\AgdaPrimitive{⊔}}\AgdaSpace{}%
\AgdaBound{𝓦}\AgdaSpace{}%
\AgdaOperator{\AgdaFunction{̇}}\<%
\\
\>[1][@{}l@{\AgdaIndent{0}}]%
\>[2]\AgdaKeyword{where}\<%
\\
%
\>[2]\AgdaInductiveConstructor{im}\AgdaSpace{}%
\AgdaSymbol{:}\AgdaSpace{}%
\AgdaSymbol{(}\AgdaBound{x}\AgdaSpace{}%
\AgdaSymbol{:}\AgdaSpace{}%
\AgdaBound{A}\AgdaSymbol{)}\AgdaSpace{}%
\AgdaSymbol{→}\AgdaSpace{}%
\AgdaOperator{\AgdaDatatype{Image}}\AgdaSpace{}%
\AgdaBound{f}\AgdaSpace{}%
\AgdaOperator{\AgdaDatatype{∋}}\AgdaSpace{}%
\AgdaBound{f}\AgdaSpace{}%
\AgdaBound{x}\<%
\\
%
\>[2]\AgdaInductiveConstructor{eq}\AgdaSpace{}%
\AgdaSymbol{:}\AgdaSpace{}%
\AgdaSymbol{(}\AgdaBound{b}\AgdaSpace{}%
\AgdaSymbol{:}\AgdaSpace{}%
\AgdaBound{B}\AgdaSymbol{)}\AgdaSpace{}%
\AgdaSymbol{→}\AgdaSpace{}%
\AgdaSymbol{(}\AgdaBound{a}\AgdaSpace{}%
\AgdaSymbol{:}\AgdaSpace{}%
\AgdaBound{A}\AgdaSymbol{)}\AgdaSpace{}%
\AgdaSymbol{→}\AgdaSpace{}%
\AgdaBound{b}\AgdaSpace{}%
\AgdaOperator{\AgdaDatatype{≡}}\AgdaSpace{}%
\AgdaBound{f}\AgdaSpace{}%
\AgdaBound{a}\AgdaSpace{}%
\AgdaSymbol{→}\AgdaSpace{}%
\AgdaOperator{\AgdaDatatype{Image}}\AgdaSpace{}%
\AgdaBound{f}\AgdaSpace{}%
\AgdaOperator{\AgdaDatatype{∋}}\AgdaSpace{}%
\AgdaBound{b}\<%
\end{code}
\ccpad
Next we verify that the type just defined is what we expect.
\ccpad
\begin{code}
\>[1]\AgdaFunction{ImageIsImage}\AgdaSpace{}%
\AgdaSymbol{:}%
\>[72I]\AgdaSymbol{\{}\AgdaBound{A}\AgdaSpace{}%
\AgdaSymbol{:}\AgdaSpace{}%
\AgdaBound{𝓤}\AgdaSpace{}%
\AgdaOperator{\AgdaFunction{̇}}\AgdaSymbol{\}\{}\AgdaBound{B}\AgdaSpace{}%
\AgdaSymbol{:}\AgdaSpace{}%
\AgdaBound{𝓦}\AgdaSpace{}%
\AgdaOperator{\AgdaFunction{̇}}\AgdaSymbol{\}(}\AgdaBound{f}\AgdaSpace{}%
\AgdaSymbol{:}\AgdaSpace{}%
\AgdaBound{A}\AgdaSpace{}%
\AgdaSymbol{→}\AgdaSpace{}%
\AgdaBound{B}\AgdaSymbol{)(}\AgdaBound{b}\AgdaSpace{}%
\AgdaSymbol{:}\AgdaSpace{}%
\AgdaBound{B}\AgdaSymbol{)(}\AgdaBound{a}\AgdaSpace{}%
\AgdaSymbol{:}\AgdaSpace{}%
\AgdaBound{A}\AgdaSymbol{)}\<%
\\
\>[.][@{}l@{}]\<[72I]%
\>[16]\AgdaComment{---------------------------------------------}\<%
\\
\>[1][@{}l@{\AgdaIndent{0}}]%
\>[2]\AgdaSymbol{→}%
\>[16]\AgdaBound{b}\AgdaSpace{}%
\AgdaOperator{\AgdaDatatype{≡}}\AgdaSpace{}%
\AgdaBound{f}\AgdaSpace{}%
\AgdaBound{a}\AgdaSpace{}%
\AgdaSymbol{→}\AgdaSpace{}%
\AgdaOperator{\AgdaDatatype{Image}}\AgdaSpace{}%
\AgdaBound{f}\AgdaSpace{}%
\AgdaOperator{\AgdaDatatype{∋}}\AgdaSpace{}%
\AgdaBound{b}\<%
\\
%
\\[\AgdaEmptyExtraSkip]%
%
\>[1]\AgdaFunction{ImageIsImage}\AgdaSpace{}%
\AgdaBound{f}\AgdaSpace{}%
\AgdaBound{b}\AgdaSpace{}%
\AgdaBound{a}\AgdaSpace{}%
\AgdaBound{b≡fa}\AgdaSpace{}%
\AgdaSymbol{=}\AgdaSpace{}%
\AgdaInductiveConstructor{eq}\AgdaSpace{}%
\AgdaBound{b}\AgdaSpace{}%
\AgdaBound{a}\AgdaSpace{}%
\AgdaBound{b≡fa}\<%
\end{code}
\ccpad
Note that an inhabitant of \af{Image} \ab f \af ∋ \ab b is a dependent pair (\ab a , \ab p), where \ab a \as : \ab A and \ab p \as : \ab b \ad ≡ \af f \ab a is a proof that \ab f maps \ab a to \ab b. Since the proof that \ab b belongs to the image of \ab f is always accompanied by a ``witness'' \ab a \as : \ab A, we can actually \emph{compute} a \emph{pseudoinverse} of \ab f. For convenience, we define this inverse function, which we call \af{Inv}, and which takes an arbitrary \ab b \as : \ab B and a witness-proof pair, (\ab a , \ab p) \as : \af{Image} \ab f \af ∋ \ab b, and returns \ab a.
\ccpad
\begin{code}%
\>[0][@{}l@{\AgdaIndent{1}}]%
\>[1]\AgdaFunction{Inv}\AgdaSpace{}%
\AgdaSymbol{:}\AgdaSpace{}%
\AgdaSymbol{\{}\AgdaBound{A}\AgdaSpace{}%
\AgdaSymbol{:}\AgdaSpace{}%
\AgdaBound{𝓤}\AgdaSpace{}%
\AgdaOperator{\AgdaFunction{̇}}\AgdaSpace{}%
\AgdaSymbol{\}\{}\AgdaBound{B}\AgdaSpace{}%
\AgdaSymbol{:}\AgdaSpace{}%
\AgdaBound{𝓦}\AgdaSpace{}%
\AgdaOperator{\AgdaFunction{̇}}\AgdaSpace{}%
\AgdaSymbol{\}(}\AgdaBound{f}\AgdaSpace{}%
\AgdaSymbol{:}\AgdaSpace{}%
\AgdaBound{A}\AgdaSpace{}%
\AgdaSymbol{→}\AgdaSpace{}%
\AgdaBound{B}\AgdaSymbol{)\{}\AgdaBound{b}\AgdaSpace{}%
\AgdaSymbol{:}\AgdaSpace{}%
\AgdaBound{B}\AgdaSymbol{\}}\AgdaSpace{}%
\AgdaSymbol{→}\AgdaSpace{}%
\AgdaOperator{\AgdaDatatype{Image}}\AgdaSpace{}%
\AgdaBound{f}\AgdaSpace{}%
\AgdaOperator{\AgdaDatatype{∋}}\AgdaSpace{}%
\AgdaBound{b}%
\>[61]\AgdaSymbol{→}%
\>[64]\AgdaBound{A}\<%
\\
%
\>[1]\AgdaFunction{Inv}\AgdaSpace{}%
\AgdaBound{f}\AgdaSpace{}%
\AgdaSymbol{\{}\AgdaDottedPattern{\AgdaSymbol{.(}}\AgdaDottedPattern{\AgdaBound{f}}\AgdaSpace{}%
\AgdaDottedPattern{\AgdaBound{a}}\AgdaDottedPattern{\AgdaSymbol{)}}\AgdaSymbol{\}}\AgdaSpace{}%
\AgdaSymbol{(}\AgdaInductiveConstructor{im}\AgdaSpace{}%
\AgdaBound{a}\AgdaSymbol{)}\AgdaSpace{}%
\AgdaSymbol{=}\AgdaSpace{}%
\AgdaBound{a}\<%
\\
%
\>[1]\AgdaFunction{Inv}\AgdaSpace{}%
\AgdaBound{f}\AgdaSpace{}%
\AgdaSymbol{(}\AgdaInductiveConstructor{eq}\AgdaSpace{}%
\AgdaSymbol{\AgdaUnderscore{}}\AgdaSpace{}%
\AgdaBound{a}\AgdaSpace{}%
\AgdaSymbol{\AgdaUnderscore{})}\AgdaSpace{}%
\AgdaSymbol{=}\AgdaSpace{}%
\AgdaBound{a}\<%
\end{code}
\ccpad
We can prove that \af{Inv} \ab f is the right-inverse of \ab f, as follows.
\ccpad
\begin{code}%
\>[0][@{}l@{\AgdaIndent{1}}]%
\>[1]\AgdaFunction{InvIsInv}\AgdaSpace{}%
\AgdaSymbol{:}\AgdaSpace{}%
\AgdaSymbol{\{}\AgdaBound{A}\AgdaSpace{}%
\AgdaSymbol{:}\AgdaSpace{}%
\AgdaBound{𝓤}\AgdaSpace{}%
\AgdaOperator{\AgdaFunction{̇}}\AgdaSymbol{\}\{}\AgdaBound{B}\AgdaSpace{}%
\AgdaSymbol{:}\AgdaSpace{}%
\AgdaBound{𝓦}\AgdaSpace{}%
\AgdaOperator{\AgdaFunction{̇}}\AgdaSymbol{\}(}\AgdaBound{f}\AgdaSpace{}%
\AgdaSymbol{:}\AgdaSpace{}%
\AgdaBound{A}\AgdaSpace{}%
\AgdaSymbol{→}\AgdaSpace{}%
\AgdaBound{B}\AgdaSymbol{)\{}\AgdaBound{b}\AgdaSpace{}%
\AgdaSymbol{:}\AgdaSpace{}%
\AgdaBound{B}\AgdaSymbol{\}(}\AgdaBound{b∈Imgf}\AgdaSpace{}%
\AgdaSymbol{:}\AgdaSpace{}%
\AgdaOperator{\AgdaDatatype{Image}}\AgdaSpace{}%
\AgdaBound{f}\AgdaSpace{}%
\AgdaOperator{\AgdaDatatype{∋}}\AgdaSpace{}%
\AgdaBound{b}\AgdaSymbol{)}\AgdaSpace{}%
\AgdaSymbol{→}\AgdaSpace{}%
\AgdaBound{f}\AgdaSymbol{(}\AgdaFunction{Inv}\AgdaSpace{}%
\AgdaBound{f}\AgdaSpace{}%
\AgdaBound{b∈Imgf}\AgdaSymbol{)}\AgdaSpace{}%
\AgdaOperator{\AgdaDatatype{≡}}\AgdaSpace{}%
\AgdaBound{b}\<%
\\
%
\>[1]\AgdaFunction{InvIsInv}\AgdaSpace{}%
\AgdaBound{f}\AgdaSpace{}%
\AgdaSymbol{\{}\AgdaDottedPattern{\AgdaSymbol{.(}}\AgdaDottedPattern{\AgdaBound{f}}\AgdaSpace{}%
\AgdaDottedPattern{\AgdaBound{a}}\AgdaDottedPattern{\AgdaSymbol{)}}\AgdaSymbol{\}}\AgdaSpace{}%
\AgdaSymbol{(}\AgdaInductiveConstructor{im}\AgdaSpace{}%
\AgdaBound{a}\AgdaSymbol{)}\AgdaSpace{}%
\AgdaSymbol{=}\AgdaSpace{}%
\AgdaInductiveConstructor{refl}\AgdaSpace{}%
\AgdaSymbol{\AgdaUnderscore{}}\<%
\\
%
\>[1]\AgdaFunction{InvIsInv}\AgdaSpace{}%
\AgdaBound{f}\AgdaSpace{}%
\AgdaSymbol{(}\AgdaInductiveConstructor{eq}\AgdaSpace{}%
\AgdaSymbol{\AgdaUnderscore{}}\AgdaSpace{}%
\AgdaSymbol{\AgdaUnderscore{}}\AgdaSpace{}%
\AgdaBound{p}\AgdaSymbol{)}\AgdaSpace{}%
\AgdaSymbol{=}\AgdaSpace{}%
\AgdaBound{p}\AgdaSpace{}%
\AgdaOperator{\AgdaFunction{⁻¹}}\<%
\end{code}

\subsubsection{Surjective functions}\label{surjective-functions}

An \defn{epic} (or \defn{surjective}) function from type \ab A \as : \ab 𝓤\af ̇ to type \ab B \as : \ab 𝓦\af ̇ is as an inhabitant of the \af{Epic} type, which we define as follows.
\ccpad
\begin{code}%
\>[0][@{}l@{\AgdaIndent{1}}]%
\>[1]\AgdaFunction{Epic}\AgdaSpace{}%
\AgdaSymbol{:}\AgdaSpace{}%
\AgdaSymbol{\{}\AgdaBound{A}\AgdaSpace{}%
\AgdaSymbol{:}\AgdaSpace{}%
\AgdaBound{𝓤}\AgdaSpace{}%
\AgdaOperator{\AgdaFunction{̇}}\AgdaSpace{}%
\AgdaSymbol{\}}\AgdaSpace{}%
\AgdaSymbol{\{}\AgdaBound{B}\AgdaSpace{}%
\AgdaSymbol{:}\AgdaSpace{}%
\AgdaBound{𝓦}\AgdaSpace{}%
\AgdaOperator{\AgdaFunction{̇}}\AgdaSpace{}%
\AgdaSymbol{\}}\AgdaSpace{}%
\AgdaSymbol{(}\AgdaBound{g}\AgdaSpace{}%
\AgdaSymbol{:}\AgdaSpace{}%
\AgdaBound{A}\AgdaSpace{}%
\AgdaSymbol{→}\AgdaSpace{}%
\AgdaBound{B}\AgdaSymbol{)}\AgdaSpace{}%
\AgdaSymbol{→}%
\>[45]\AgdaBound{𝓤}\AgdaSpace{}%
\AgdaOperator{\AgdaPrimitive{⊔}}\AgdaSpace{}%
\AgdaBound{𝓦}\AgdaSpace{}%
\AgdaOperator{\AgdaFunction{̇}}\<%
\\
%
\>[1]\AgdaFunction{Epic}\AgdaSpace{}%
\AgdaBound{g}\AgdaSpace{}%
\AgdaSymbol{=}\AgdaSpace{}%
\AgdaSymbol{∀}\AgdaSpace{}%
\AgdaBound{y}\AgdaSpace{}%
\AgdaSymbol{→}\AgdaSpace{}%
\AgdaOperator{\AgdaDatatype{Image}}\AgdaSpace{}%
\AgdaBound{g}\AgdaSpace{}%
\AgdaOperator{\AgdaDatatype{∋}}\AgdaSpace{}%
\AgdaBound{y}\<%
\end{code}
\ccpad
We obtain the right-inverse (or pseudoinverse) of an epic function \ab f by applying the function \af{EpicInv} (which we now define) to the function \ab f along with a proof, \ab{fepi} \as : \af{Epic} \ab f, that \ab f is surjective.
\ccpad
\begin{code}%
\>[0][@{}l@{\AgdaIndent{1}}]%
\>[1]\AgdaFunction{EpicInv}\AgdaSpace{}%
\AgdaSymbol{:}%
\>[242I]\AgdaSymbol{\{}\AgdaBound{A}\AgdaSpace{}%
\AgdaSymbol{:}\AgdaSpace{}%
\AgdaBound{𝓤}\AgdaSpace{}%
\AgdaOperator{\AgdaFunction{̇}}\AgdaSpace{}%
\AgdaSymbol{\}}\AgdaSpace{}%
\AgdaSymbol{\{}\AgdaBound{B}\AgdaSpace{}%
\AgdaSymbol{:}\AgdaSpace{}%
\AgdaBound{𝓦}\AgdaSpace{}%
\AgdaOperator{\AgdaFunction{̇}}\AgdaSpace{}%
\AgdaSymbol{\}}\AgdaSpace{}%
% \<%
% \\
% \>[.][@{}l@{}]\<[242I]%
% \>[11]
\AgdaSymbol{(}\AgdaBound{f}\AgdaSpace{}%
\AgdaSymbol{:}\AgdaSpace{}%
\AgdaBound{A}\AgdaSpace{}%
\AgdaSymbol{→}\AgdaSpace{}%
\AgdaBound{B}\AgdaSymbol{)}\AgdaSpace{}%
\AgdaSymbol{→}\AgdaSpace{}%
\AgdaFunction{Epic}\AgdaSpace{}%
\AgdaBound{f}\AgdaSpace{}%
% \<%
% \\
% %
% \>[11]\AgdaComment{--------------------}\<%
% \\
% \>[1][@{}l@{\AgdaIndent{0}}]%
% \>[2]
\AgdaSymbol{→}\AgdaSpace{}%%
% \>[11]
\AgdaBound{B}\AgdaSpace{}%
\AgdaSymbol{→}\AgdaSpace{}%
\AgdaBound{A}\<%
\\
%
\>[1]\AgdaFunction{EpicInv}\AgdaSpace{}%
\AgdaBound{f}\AgdaSpace{}%
\AgdaBound{fepi}\AgdaSpace{}%
\AgdaBound{b}\AgdaSpace{}%
\AgdaSymbol{=}\AgdaSpace{}%
\AgdaFunction{Inv}\AgdaSpace{}%
\AgdaBound{f}\AgdaSpace{}%
\AgdaSymbol{(}\AgdaBound{fepi}\AgdaSpace{}%
\AgdaBound{b}\AgdaSymbol{)}\<%
\end{code}
\ccpad
The function defined by \af{EpicInv} \ab f \ab{fepi} is indeed the right-inverse of \ab f. To state this, we'll use the function composition operation \af{∘}, which is already defined in the \typetopology library, as follows.
\ccpad
\begin{code}%
\>[1]\AgdaOperator{\AgdaFunction{\AgdaUnderscore{}∘\AgdaUnderscore{}}}\AgdaSpace{}%
\AgdaSymbol{:}\AgdaSpace{}%
\AgdaSymbol{\{}\AgdaBound{X}\AgdaSpace{}%
\AgdaSymbol{:}\AgdaSpace{}%
\AgdaBound{𝓤}\AgdaSpace{}%
\AgdaOperator{\AgdaFunction{̇}}\AgdaSpace{}%
\AgdaSymbol{\}}\AgdaSpace{}%
\AgdaSymbol{\{}\AgdaBound{Y}\AgdaSpace{}%
\AgdaSymbol{:}\AgdaSpace{}%
\AgdaBound{𝓦}\AgdaSpace{}%
\AgdaOperator{\AgdaFunction{̇}}\AgdaSymbol{\}\{}\AgdaBound{Z}\AgdaSpace{}%
\AgdaSymbol{:}\AgdaSpace{}%
\AgdaBound{Y}\AgdaSpace{}%
\AgdaSymbol{→}\AgdaSpace{}%
\AgdaBound{𝓦}\AgdaSpace{}%
\AgdaOperator{\AgdaFunction{̇}}\AgdaSpace{}%
\AgdaSymbol{\}}\AgdaSpace{}%
% \<%
% \\
% \>[1][@{}l@{\AgdaIndent{0}}]%
% \>[2]
\AgdaSymbol{→}\AgdaSpace{}%
%
% \>[7]
\AgdaFunction{Π}\AgdaSpace{}%
\AgdaBound{Z}\AgdaSpace{}%
\AgdaSymbol{→}\AgdaSpace{}%
\AgdaSymbol{(}\AgdaBound{f}\AgdaSpace{}%
\AgdaSymbol{:}\AgdaSpace{}%
\AgdaBound{X}\AgdaSpace{}%
\AgdaSymbol{→}\AgdaSpace{}%
\AgdaBound{Y}\AgdaSymbol{)}\AgdaSpace{}%
\AgdaSymbol{→}\AgdaSpace{}%
\AgdaSymbol{(}\AgdaBound{x}\AgdaSpace{}%
\AgdaSymbol{:}\AgdaSpace{}%
\AgdaBound{X}\AgdaSymbol{)}\AgdaSpace{}%
\AgdaSymbol{→}\AgdaSpace{}%
\AgdaBound{Z}\AgdaSpace{}%
\AgdaSymbol{(}\AgdaBound{f}\AgdaSpace{}%
\AgdaBound{x}\AgdaSymbol{)}\<%
\\
%
\>[1]\AgdaBound{g}\AgdaSpace{}%
\AgdaOperator{\AgdaFunction{∘}}\AgdaSpace{}%
\AgdaBound{f}\AgdaSpace{}%
\AgdaSymbol{=}\AgdaSpace{}%
\AgdaSymbol{λ}\AgdaSpace{}%
\AgdaBound{x}\AgdaSpace{}%
\AgdaSymbol{→}\AgdaSpace{}%
\AgdaBound{g}\AgdaSpace{}%
\AgdaSymbol{(}\AgdaBound{f}\AgdaSpace{}%
\AgdaBound{x}\AgdaSymbol{)}\<%
\end{code}
\scpad
% \\
% %
% \\[\AgdaEmptyExtraSkip]%
% \>[0]\AgdaKeyword{open}\AgdaSpace{}%
% \AgdaKeyword{import}\AgdaSpace{}%
% \AgdaModule{MGS-MLTT}\AgdaSpace{}%
% \AgdaKeyword{using}\AgdaSpace{}%
% \AgdaSymbol{(}\AgdaOperator{\AgdaFunction{\AgdaUnderscore{}∘\AgdaUnderscore{}}}\AgdaSymbol{)}\AgdaSpace{}%
% \AgdaKeyword{public}\<%
% \\
% %
% \\[\AgdaEmptyExtraSkip]%
% %
% \\[\AgdaEmptyExtraSkip]%
% \>[0]\AgdaKeyword{module}\AgdaSpace{}%
% \AgdaModule{\AgdaUnderscore{}}\AgdaSpace{}%
% \AgdaSymbol{\{}\AgdaBound{𝓤}\AgdaSpace{}%
% \AgdaBound{𝓦}\AgdaSpace{}%
% \AgdaSymbol{:}\AgdaSpace{}%
% \AgdaPostulate{Universe}\AgdaSymbol{\}}\AgdaSpace{}%
% \AgdaKeyword{where}\<%
% \\
% %
% \\[\AgdaEmptyExtraSkip]%
\begin{code}
\>[1]\AgdaFunction{EpicInvIsRightInv}\AgdaSpace{}%
\AgdaSymbol{:}%
\>[327I]\AgdaFunction{funext}\AgdaSpace{}%
\AgdaBound{𝓦}\AgdaSpace{}%
\AgdaBound{𝓦}\AgdaSpace{}%
\AgdaSymbol{→}\AgdaSpace{}%
\AgdaSymbol{\{}\AgdaBound{A}\AgdaSpace{}%
\AgdaSymbol{:}\AgdaSpace{}%
\AgdaBound{𝓤}\AgdaSpace{}%
\AgdaOperator{\AgdaFunction{̇}}\AgdaSpace{}%
\AgdaSymbol{\}}\AgdaSpace{}%
\AgdaSymbol{\{}\AgdaBound{B}\AgdaSpace{}%
\AgdaSymbol{:}\AgdaSpace{}%
\AgdaBound{𝓦}\AgdaSpace{}%
\AgdaOperator{\AgdaFunction{̇}}\AgdaSpace{}%
\AgdaSymbol{\}}\<%
\\
\>[.][@{}l@{}]\<[327I]%
\>[21]\AgdaSymbol{(}\AgdaBound{f}\AgdaSpace{}%
\AgdaSymbol{:}\AgdaSpace{}%
\AgdaBound{A}\AgdaSpace{}%
\AgdaSymbol{→}\AgdaSpace{}%
\AgdaBound{B}\AgdaSymbol{)}%
\>[34]\AgdaSymbol{(}\AgdaBound{fepi}\AgdaSpace{}%
\AgdaSymbol{:}\AgdaSpace{}%
\AgdaFunction{Epic}\AgdaSpace{}%
\AgdaBound{f}\AgdaSymbol{)}\<%
\\
%
\>[21]\AgdaComment{--------------------------}\<%
\\
\>[1][@{}l@{\AgdaIndent{0}}]%
\>[2]\AgdaSymbol{→}%
\>[21]\AgdaBound{f}\AgdaSpace{}%
\AgdaOperator{\AgdaFunction{∘}}\AgdaSpace{}%
\AgdaSymbol{(}\AgdaFunction{EpicInv}\AgdaSpace{}%
\AgdaBound{f}\AgdaSpace{}%
\AgdaBound{fepi}\AgdaSymbol{)}\AgdaSpace{}%
\AgdaOperator{\AgdaDatatype{≡}}\AgdaSpace{}%
\AgdaFunction{𝑖𝑑}\AgdaSpace{}%
\AgdaBound{B}\<%
\\
%
\\[\AgdaEmptyExtraSkip]%
%
\>[1]\AgdaFunction{EpicInvIsRightInv}\AgdaSpace{}%
\AgdaBound{fe}\AgdaSpace{}%
\AgdaBound{f}\AgdaSpace{}%
\AgdaBound{fepi}\AgdaSpace{}%
\AgdaSymbol{=}\AgdaSpace{}%
\AgdaBound{fe}\AgdaSpace{}%
\AgdaSymbol{(λ}\AgdaSpace{}%
\AgdaBound{x}\AgdaSpace{}%
\AgdaSymbol{→}\AgdaSpace{}%
\AgdaFunction{InvIsInv}\AgdaSpace{}%
\AgdaBound{f}\AgdaSpace{}%
\AgdaSymbol{(}\AgdaBound{fepi}\AgdaSpace{}%
\AgdaBound{x}\AgdaSymbol{))}\<%
\end{code}

\subsubsection{Injective functions}\label{injective-functions}

We say that a function \ab g \as : \ab A \as → \ab B is \textbf{monic} (or \textbf{injective} or \textbf{one-to-one}) if it doesn't map distinct elements to a common point. This property is formalized quite naturally using the \af{Monic} type, which we now define.
\ccpad
\begin{code}%
\>[0][@{}l@{\AgdaIndent{1}}]%
\>[1]\AgdaFunction{Monic}\AgdaSpace{}%
\AgdaSymbol{:}\AgdaSpace{}%
\AgdaSymbol{\{}\AgdaBound{A}\AgdaSpace{}%
\AgdaSymbol{:}\AgdaSpace{}%
\AgdaBound{𝓤}\AgdaSpace{}%
\AgdaOperator{\AgdaFunction{̇}}\AgdaSpace{}%
\AgdaSymbol{\}}\AgdaSpace{}%
\AgdaSymbol{\{}\AgdaBound{B}\AgdaSpace{}%
\AgdaSymbol{:}\AgdaSpace{}%
\AgdaBound{𝓦}\AgdaSpace{}%
\AgdaOperator{\AgdaFunction{̇}}\AgdaSpace{}%
\AgdaSymbol{\}(}\AgdaBound{g}\AgdaSpace{}%
\AgdaSymbol{:}\AgdaSpace{}%
\AgdaBound{A}\AgdaSpace{}%
\AgdaSymbol{→}\AgdaSpace{}%
\AgdaBound{B}\AgdaSymbol{)}\AgdaSpace{}%
\AgdaSymbol{→}\AgdaSpace{}%
\AgdaBound{𝓤}\AgdaSpace{}%
\AgdaOperator{\AgdaPrimitive{⊔}}\AgdaSpace{}%
\AgdaBound{𝓦}\AgdaSpace{}%
\AgdaOperator{\AgdaFunction{̇}}\<%
\\
%
\>[1]\AgdaFunction{Monic}\AgdaSpace{}%
\AgdaBound{g}\AgdaSpace{}%
\AgdaSymbol{=}\AgdaSpace{}%
\AgdaSymbol{∀}\AgdaSpace{}%
\AgdaBound{a₁}\AgdaSpace{}%
\AgdaBound{a₂}\AgdaSpace{}%
\AgdaSymbol{→}\AgdaSpace{}%
\AgdaBound{g}\AgdaSpace{}%
\AgdaBound{a₁}\AgdaSpace{}%
\AgdaOperator{\AgdaDatatype{≡}}\AgdaSpace{}%
\AgdaBound{g}\AgdaSpace{}%
\AgdaBound{a₂}\AgdaSpace{}%
\AgdaSymbol{→}\AgdaSpace{}%
\AgdaBound{a₁}\AgdaSpace{}%
\AgdaOperator{\AgdaDatatype{≡}}\AgdaSpace{}%
\AgdaBound{a₂}\<%
\end{code}
\ccpad
Again, we obtain a pseudoinverse. Here it is obtained by applying the function \af{MonicInv} to \ab g and a proof that \ab g is monic.
\ccpad
\begin{code}%
\>[1]\AgdaFunction{MonicInv}\AgdaSpace{}%
\AgdaSymbol{:}\AgdaSpace{}%
\AgdaSymbol{\{}\AgdaBound{A}\AgdaSpace{}%
\AgdaSymbol{:}\AgdaSpace{}%
\AgdaBound{𝓤}\AgdaSpace{}%
\AgdaOperator{\AgdaFunction{̇}}\AgdaSpace{}%
\AgdaSymbol{\}\{}\AgdaBound{B}\AgdaSpace{}%
\AgdaSymbol{:}\AgdaSpace{}%
\AgdaBound{𝓦}\AgdaSpace{}%
\AgdaOperator{\AgdaFunction{̇}}\AgdaSpace{}%
\AgdaSymbol{\}(}\AgdaBound{f}\AgdaSpace{}%
\AgdaSymbol{:}\AgdaSpace{}%
\AgdaBound{A}\AgdaSpace{}%
\AgdaSymbol{→}\AgdaSpace{}%
\AgdaBound{B}\AgdaSymbol{)}\AgdaSpace{}%
\AgdaSymbol{→}\AgdaSpace{}%
\AgdaFunction{Monic}\AgdaSpace{}%
\AgdaBound{f}\AgdaSpace{}%
% \<%
% \\
% \>[1][@{}l@{\AgdaIndent{0}}]%
% \>[2]
\AgdaSymbol{→}\AgdaSpace{}%
% \>[12]
\AgdaSymbol{(}\AgdaBound{b}\AgdaSpace{}%
\AgdaSymbol{:}\AgdaSpace{}%
\AgdaBound{B}\AgdaSymbol{)}\AgdaSpace{}%
\AgdaSymbol{→}\AgdaSpace{}%
\AgdaOperator{\AgdaDatatype{Image}}\AgdaSpace{}%
\AgdaBound{f}\AgdaSpace{}%
\AgdaOperator{\AgdaDatatype{∋}}\AgdaSpace{}%
\AgdaBound{b}\AgdaSpace{}%
\AgdaSymbol{→}\AgdaSpace{}%
\AgdaBound{A}\<%
\\
%
\>[1]\AgdaFunction{MonicInv}\AgdaSpace{}%
\AgdaBound{f}\AgdaSpace{}%
\AgdaSymbol{\AgdaUnderscore{}}\AgdaSpace{}%
\AgdaSymbol{=}\AgdaSpace{}%
\AgdaSymbol{λ}\AgdaSpace{}%
\AgdaBound{b}\AgdaSpace{}%
\AgdaBound{Imf∋b}\AgdaSpace{}%
\AgdaSymbol{→}\AgdaSpace{}%
\AgdaFunction{Inv}\AgdaSpace{}%
\AgdaBound{f}\AgdaSpace{}%
\AgdaBound{Imf∋b}\<%
\end{code}
\ccpad
The function defined by \af{MonicInv} \ab f \ab{fM} is the left-inverse of \ab f.
\ccpad
\begin{code}%
\>[1]\AgdaFunction{MonicInvIsLeftInv}\AgdaSpace{}%
\AgdaSymbol{:}%
\>[100I]\AgdaSymbol{\{}\AgdaBound{A}\AgdaSpace{}%
\AgdaSymbol{:}\AgdaSpace{}%
\AgdaBound{𝓤}\AgdaSpace{}%
\AgdaOperator{\AgdaFunction{̇}}\AgdaSpace{}%
\AgdaSymbol{\}\{}\AgdaBound{B}\AgdaSpace{}%
\AgdaSymbol{:}\AgdaSpace{}%
\AgdaBound{𝓦}\AgdaSpace{}%
\AgdaOperator{\AgdaFunction{̇}}\AgdaSpace{}%
\AgdaSymbol{\}(}\AgdaBound{f}\AgdaSpace{}%
\AgdaSymbol{:}\AgdaSpace{}%
\AgdaBound{A}\AgdaSpace{}%
\AgdaSymbol{→}\AgdaSpace{}%
\AgdaBound{B}\AgdaSymbol{)(}\AgdaBound{fmonic}\AgdaSpace{}%
\AgdaSymbol{:}\AgdaSpace{}%
\AgdaFunction{Monic}\AgdaSpace{}%
\AgdaBound{f}\AgdaSymbol{)(}\AgdaBound{x}\AgdaSpace{}%
\AgdaSymbol{:}\AgdaSpace{}%
\AgdaBound{A}\AgdaSymbol{)}\<%
\\
\>[1][@{}l@{\AgdaIndent{0}}]%
\>[3]\AgdaSymbol{→}%
\>[.][@{}l@{}]\<[100I]%
\>[21]\AgdaSymbol{(}\AgdaFunction{MonicInv}\AgdaSpace{}%
\AgdaBound{f}\AgdaSpace{}%
\AgdaBound{fmonic}\AgdaSymbol{)(}\AgdaBound{f}\AgdaSpace{}%
\AgdaBound{x}\AgdaSymbol{)(}\AgdaInductiveConstructor{im}\AgdaSpace{}%
\AgdaBound{x}\AgdaSymbol{)}\AgdaSpace{}%
\AgdaOperator{\AgdaDatatype{≡}}\AgdaSpace{}%
\AgdaBound{x}\<%
\\
%
\>[1]\AgdaFunction{MonicInvIsLeftInv}\AgdaSpace{}%
\AgdaBound{f}\AgdaSpace{}%
\AgdaBound{fmonic}\AgdaSpace{}%
\AgdaBound{x}\AgdaSpace{}%
\AgdaSymbol{=}\AgdaSpace{}%
\AgdaInductiveConstructor{𝓇ℯ𝒻𝓁}\<%
\end{code}

% \subsubsection{Composition laws}\label{composition-laws}

% \begin{code}%
% \>[0]\AgdaKeyword{module}\AgdaSpace{}%
% \AgdaModule{\AgdaUnderscore{}}\AgdaSpace{}%
% \AgdaSymbol{\{}\AgdaBound{𝓧}\AgdaSpace{}%
% \AgdaBound{𝓨}\AgdaSpace{}%
% \AgdaBound{𝓩}\AgdaSpace{}%
% \AgdaSymbol{:}\AgdaSpace{}%
% \AgdaPostulate{Universe}\AgdaSymbol{\}}\AgdaSpace{}%
% \AgdaKeyword{where}\<%
% \\
% %
% \\[\AgdaEmptyExtraSkip]%
% \>[0][@{}l@{\AgdaIndent{0}}]%
% \>[1]\AgdaFunction{epic-factor}\AgdaSpace{}%
% \AgdaSymbol{:}%
% \>[477I]\AgdaFunction{funext}\AgdaSpace{}%
% \AgdaBound{𝓨}\AgdaSpace{}%
% \AgdaBound{𝓨}\AgdaSpace{}%
% \AgdaSymbol{→}\AgdaSpace{}%
% \AgdaSymbol{\{}\AgdaBound{A}\AgdaSpace{}%
% \AgdaSymbol{:}\AgdaSpace{}%
% \AgdaBound{𝓧}\AgdaSpace{}%
% \AgdaOperator{\AgdaFunction{̇}}\AgdaSymbol{\}\{}\AgdaBound{B}\AgdaSpace{}%
% \AgdaSymbol{:}\AgdaSpace{}%
% \AgdaBound{𝓨}\AgdaSpace{}%
% \AgdaOperator{\AgdaFunction{̇}}\AgdaSymbol{\}\{}\AgdaBound{C}\AgdaSpace{}%
% \AgdaSymbol{:}\AgdaSpace{}%
% \AgdaBound{𝓩}\AgdaSpace{}%
% \AgdaOperator{\AgdaFunction{̇}}\AgdaSymbol{\}}\<%
% \\
% \>[.][@{}l@{}]\<[477I]%
% \>[15]\AgdaSymbol{(}\AgdaBound{β}\AgdaSpace{}%
% \AgdaSymbol{:}\AgdaSpace{}%
% \AgdaBound{A}\AgdaSpace{}%
% \AgdaSymbol{→}\AgdaSpace{}%
% \AgdaBound{B}\AgdaSymbol{)(}\AgdaBound{ξ}\AgdaSpace{}%
% \AgdaSymbol{:}\AgdaSpace{}%
% \AgdaBound{A}\AgdaSpace{}%
% \AgdaSymbol{→}\AgdaSpace{}%
% \AgdaBound{C}\AgdaSymbol{)(}\AgdaBound{ϕ}\AgdaSpace{}%
% \AgdaSymbol{:}\AgdaSpace{}%
% \AgdaBound{C}\AgdaSpace{}%
% \AgdaSymbol{→}\AgdaSpace{}%
% \AgdaBound{B}\AgdaSymbol{)}\<%
% \\
% \>[1][@{}l@{\AgdaIndent{0}}]%
% \>[2]\AgdaSymbol{→}%
% \>[15]\AgdaBound{β}\AgdaSpace{}%
% \AgdaOperator{\AgdaDatatype{≡}}\AgdaSpace{}%
% \AgdaBound{ϕ}\AgdaSpace{}%
% \AgdaOperator{\AgdaFunction{∘}}\AgdaSpace{}%
% \AgdaBound{ξ}\AgdaSpace{}%
% \AgdaSymbol{→}%
% \>[28]\AgdaFunction{Epic}\AgdaSpace{}%
% \AgdaBound{β}\AgdaSpace{}%
% \AgdaSymbol{→}\AgdaSpace{}%
% \AgdaFunction{Epic}\AgdaSpace{}%
% \AgdaBound{ϕ}\<%
% \\
% %
% \\[\AgdaEmptyExtraSkip]%
% %
% \>[1]\AgdaFunction{epic-factor}\AgdaSpace{}%
% \AgdaBound{fe}\AgdaSpace{}%
% \AgdaSymbol{\{}\AgdaBound{A}\AgdaSymbol{\}\{}\AgdaBound{B}\AgdaSymbol{\}\{}\AgdaBound{C}\AgdaSymbol{\}}\AgdaSpace{}%
% \AgdaBound{β}\AgdaSpace{}%
% \AgdaBound{ξ}\AgdaSpace{}%
% \AgdaBound{ϕ}\AgdaSpace{}%
% \AgdaBound{compId}\AgdaSpace{}%
% \AgdaBound{βe}\AgdaSpace{}%
% \AgdaBound{y}\AgdaSpace{}%
% \AgdaSymbol{=}\AgdaSpace{}%
% \AgdaFunction{γ}\<%
% \\
% \>[1][@{}l@{\AgdaIndent{0}}]%
% \>[2]\AgdaKeyword{where}\<%
% \\
% \>[2][@{}l@{\AgdaIndent{0}}]%
% \>[3]\AgdaFunction{βinv}\AgdaSpace{}%
% \AgdaSymbol{:}\AgdaSpace{}%
% \AgdaBound{B}\AgdaSpace{}%
% \AgdaSymbol{→}\AgdaSpace{}%
% \AgdaBound{A}\<%
% \\
% %
% \>[3]\AgdaFunction{βinv}\AgdaSpace{}%
% \AgdaSymbol{=}\AgdaSpace{}%
% \AgdaFunction{EpicInv}\AgdaSpace{}%
% \AgdaBound{β}\AgdaSpace{}%
% \AgdaBound{βe}\<%
% \\
% %
% \\[\AgdaEmptyExtraSkip]%
% %
% \>[3]\AgdaFunction{ζ}\AgdaSpace{}%
% \AgdaSymbol{:}\AgdaSpace{}%
% \AgdaBound{β}\AgdaSpace{}%
% \AgdaSymbol{(}\AgdaFunction{βinv}\AgdaSpace{}%
% \AgdaBound{y}\AgdaSymbol{)}\AgdaSpace{}%
% \AgdaOperator{\AgdaDatatype{≡}}\AgdaSpace{}%
% \AgdaBound{y}\<%
% \\
% %
% \>[3]\AgdaFunction{ζ}\AgdaSpace{}%
% \AgdaSymbol{=}\AgdaSpace{}%
% \AgdaFunction{ap}\AgdaSpace{}%
% \AgdaSymbol{(λ}\AgdaSpace{}%
% \AgdaBound{-}\AgdaSpace{}%
% \AgdaSymbol{→}\AgdaSpace{}%
% \AgdaBound{-}\AgdaSpace{}%
% \AgdaBound{y}\AgdaSymbol{)}\AgdaSpace{}%
% \AgdaSymbol{(}\AgdaFunction{EpicInvIsRightInv}\AgdaSpace{}%
% \AgdaBound{fe}\AgdaSpace{}%
% \AgdaBound{β}\AgdaSpace{}%
% \AgdaBound{βe}\AgdaSymbol{)}\<%
% \\
% %
% \>[3]\AgdaFunction{η}\AgdaSpace{}%
% \AgdaSymbol{:}\AgdaSpace{}%
% \AgdaSymbol{(}\AgdaBound{ϕ}\AgdaSpace{}%
% \AgdaOperator{\AgdaFunction{∘}}\AgdaSpace{}%
% \AgdaBound{ξ}\AgdaSymbol{)}\AgdaSpace{}%
% \AgdaSymbol{(}\AgdaFunction{βinv}\AgdaSpace{}%
% \AgdaBound{y}\AgdaSymbol{)}\AgdaSpace{}%
% \AgdaOperator{\AgdaDatatype{≡}}\AgdaSpace{}%
% \AgdaBound{y}\<%
% \\
% %
% \>[3]\AgdaFunction{η}\AgdaSpace{}%
% \AgdaSymbol{=}\AgdaSpace{}%
% \AgdaSymbol{(}\AgdaFunction{ap}\AgdaSpace{}%
% \AgdaSymbol{(λ}\AgdaSpace{}%
% \AgdaBound{-}\AgdaSpace{}%
% \AgdaSymbol{→}\AgdaSpace{}%
% \AgdaBound{-}\AgdaSpace{}%
% \AgdaSymbol{(}\AgdaFunction{βinv}\AgdaSpace{}%
% \AgdaBound{y}\AgdaSymbol{))}\AgdaSpace{}%
% \AgdaSymbol{(}\AgdaBound{compId}\AgdaSpace{}%
% \AgdaOperator{\AgdaFunction{⁻¹}}\AgdaSymbol{))}\AgdaSpace{}%
% \AgdaOperator{\AgdaFunction{∙}}\AgdaSpace{}%
% \AgdaFunction{ζ}\<%
% \\
% %
% \>[3]\AgdaFunction{γ}\AgdaSpace{}%
% \AgdaSymbol{:}\AgdaSpace{}%
% \AgdaOperator{\AgdaDatatype{Image}}\AgdaSpace{}%
% \AgdaBound{ϕ}\AgdaSpace{}%
% \AgdaOperator{\AgdaDatatype{∋}}\AgdaSpace{}%
% \AgdaBound{y}\<%
% \\
% %
% \>[3]\AgdaFunction{γ}\AgdaSpace{}%
% \AgdaSymbol{=}\AgdaSpace{}%
% \AgdaInductiveConstructor{eq}\AgdaSpace{}%
% \AgdaBound{y}\AgdaSpace{}%
% \AgdaSymbol{(}\AgdaBound{ξ}\AgdaSpace{}%
% \AgdaSymbol{(}\AgdaFunction{βinv}\AgdaSpace{}%
% \AgdaBound{y}\AgdaSymbol{))}\AgdaSpace{}%
% \AgdaSymbol{(}\AgdaFunction{η}\AgdaSpace{}%
% \AgdaOperator{\AgdaFunction{⁻¹}}\AgdaSymbol{)}\<%
% \end{code}

\subsubsection{Embeddings}\label{embeddings}

% This is the first point at which \href{UALib.Preface.html\#truncation}{truncation} comes into play. An
% \href{https://www.cs.bham.ac.uk/~mhe/HoTT-UF-in-Agda-Lecture-Notes/HoTT-UF-Agda.html\#embeddings}{embedding}
% is defined in the \typetopology library, using the \af{is-subsingleton} type \href{Prelude.Extensionality.html\#alternative-extensionality-type}{described earlier}, as follows.

% Finally, t
The type \af{is-embedding} \ab f denotes the assertion that \ab f is a function all of whose fibers are subsingletons.
\ccpad
\begin{code}%
\>[1]\AgdaFunction{is-embedding}\AgdaSpace{}%
\AgdaSymbol{:}\AgdaSpace{}%
\AgdaSymbol{\{}\AgdaBound{X}\AgdaSpace{}%
\AgdaSymbol{:}\AgdaSpace{}%
\AgdaBound{𝓤}\AgdaSpace{}%
\AgdaOperator{\AgdaFunction{̇}}\AgdaSpace{}%
\AgdaSymbol{\}}\AgdaSpace{}%
\AgdaSymbol{\{}\AgdaBound{Y}\AgdaSpace{}%
\AgdaSymbol{:}\AgdaSpace{}%
\AgdaBound{𝓦}\AgdaSpace{}%
\AgdaOperator{\AgdaFunction{̇}}\AgdaSpace{}%
\AgdaSymbol{\}}\AgdaSpace{}%
\AgdaSymbol{→}\AgdaSpace{}%
\AgdaSymbol{(}\AgdaBound{X}\AgdaSpace{}%
\AgdaSymbol{→}\AgdaSpace{}%
\AgdaBound{Y}\AgdaSymbol{)}\AgdaSpace{}%
\AgdaSymbol{→}\AgdaSpace{}%
\AgdaBound{𝓤}\AgdaSpace{}%
\AgdaOperator{\AgdaPrimitive{⊔}}\AgdaSpace{}%
\AgdaBound{𝓦}\AgdaSpace{}%
\AgdaOperator{\AgdaFunction{̇}}\<%
\\
%
\>[1]\AgdaFunction{is-embedding}\AgdaSpace{}%
\AgdaBound{f}\AgdaSpace{}%
\AgdaSymbol{=}\AgdaSpace{}%
\AgdaSymbol{∀}\AgdaSpace{}%
\AgdaBound{y}\AgdaSpace{}%
\AgdaSymbol{→}\AgdaSpace{}%
\AgdaFunction{is-subsingleton}\AgdaSpace{}%
\AgdaSymbol{(}\AgdaFunction{fiber}\AgdaSpace{}%
\AgdaBound{f}\AgdaSpace{}%
\AgdaBound{y}\AgdaSymbol{)}\<%
\end{code}
\ccpad
This is a natural way to represent what we usually mean in mathematics by embedding. Observe that an embedding does not simply correspond to an injective map. However, if we assume that the codomain \ab B has unique identity proofs (i.e., \ab B is a \emph{set}), then we can prove that a monic function into \ab B is an embedding. We postpone this until we arrive at the \ualibTruncation module and take up the topic of sets.

It should be clear that embeddings are monic; from a proof \ab p \as : \af{is-embedding} \ab f that \ab f is an embedding we can construct a proof of \af{Monic} \ab f. We verify this as follows.
\ccpad
\begin{code}%
\>[0]\AgdaFunction{embedding-is-monic}\AgdaSpace{}%
\AgdaSymbol{:}%
\>[650I]\AgdaSymbol{\{}\AgdaBound{𝓧}\AgdaSpace{}%
\AgdaBound{𝓨}\AgdaSpace{}%
\AgdaSymbol{:}\AgdaSpace{}%
\AgdaPostulate{Universe}\AgdaSymbol{\}}\AgdaSpace{}%
\AgdaSymbol{\{}\AgdaBound{X}\AgdaSpace{}%
\AgdaSymbol{:}\AgdaSpace{}%
\AgdaBound{𝓧}\AgdaSpace{}%
\AgdaOperator{\AgdaFunction{̇}}\AgdaSymbol{\}\{}\AgdaBound{Y}\AgdaSpace{}%
\AgdaSymbol{:}\AgdaSpace{}%
\AgdaBound{𝓨}\AgdaSpace{}%
\AgdaOperator{\AgdaFunction{̇}}\AgdaSymbol{\}}\<%
\\
\>[.][@{}l@{}]\<[650I]%
\>[21]\AgdaSymbol{(}\AgdaBound{f}\AgdaSpace{}%
\AgdaSymbol{:}\AgdaSpace{}%
\AgdaBound{X}\AgdaSpace{}%
\AgdaSymbol{→}\AgdaSpace{}%
\AgdaBound{Y}\AgdaSymbol{)}\AgdaSpace{}%
\AgdaSymbol{→}\AgdaSpace{}%
\AgdaFunction{is-embedding}\AgdaSpace{}%
\AgdaBound{f}\AgdaSpace{}%
\AgdaSymbol{→}\AgdaSpace{}%
\AgdaFunction{Monic}\AgdaSpace{}%
\AgdaBound{f}\<%
\\
%
\\[\AgdaEmptyExtraSkip]%
\>[0]\AgdaFunction{embedding-is-monic}\AgdaSpace{}%
\AgdaBound{f}\AgdaSpace{}%
\AgdaBound{femb}\AgdaSpace{}%
\AgdaBound{x}\AgdaSpace{}%
\AgdaBound{x'}\AgdaSpace{}%
\AgdaBound{fxfx'}\AgdaSpace{}%
\AgdaSymbol{=}\AgdaSpace{}%
\AgdaFunction{ap}\AgdaSpace{}%
\AgdaFunction{pr₁}\AgdaSpace{}%
\AgdaSymbol{((}\AgdaBound{femb}\AgdaSpace{}%
\AgdaSymbol{(}\AgdaBound{f}\AgdaSpace{}%
\AgdaBound{x}\AgdaSymbol{))}\AgdaSpace{}%
\AgdaFunction{fa}\AgdaSpace{}%
\AgdaFunction{fb}\AgdaSymbol{)}\<%
\\
\>[0][@{}l@{\AgdaIndent{0}}]%
\>[1]\AgdaKeyword{where}\<%
\\
%
\>[1]\AgdaFunction{fa}\AgdaSpace{}%
\AgdaSymbol{:}\AgdaSpace{}%
\AgdaFunction{fiber}\AgdaSpace{}%
\AgdaBound{f}\AgdaSpace{}%
\AgdaSymbol{(}\AgdaBound{f}\AgdaSpace{}%
\AgdaBound{x}\AgdaSymbol{)}\<%
\\
%
\>[1]\AgdaFunction{fa}\AgdaSpace{}%
\AgdaSymbol{=}\AgdaSpace{}%
\AgdaBound{x}\AgdaSpace{}%
\AgdaOperator{\AgdaInductiveConstructor{,}}\AgdaSpace{}%
\AgdaInductiveConstructor{𝓇ℯ𝒻𝓁}\<%
\\
%
\\[\AgdaEmptyExtraSkip]%
%
\>[1]\AgdaFunction{fb}\AgdaSpace{}%
\AgdaSymbol{:}\AgdaSpace{}%
\AgdaFunction{fiber}\AgdaSpace{}%
\AgdaBound{f}\AgdaSpace{}%
\AgdaSymbol{(}\AgdaBound{f}\AgdaSpace{}%
\AgdaBound{x}\AgdaSymbol{)}\<%
\\
%
\>[1]\AgdaFunction{fb}\AgdaSpace{}%
\AgdaSymbol{=}\AgdaSpace{}%
\AgdaBound{x'}\AgdaSpace{}%
\AgdaOperator{\AgdaInductiveConstructor{,}}\AgdaSpace{}%
\AgdaSymbol{(}\AgdaBound{fxfx'}\AgdaSpace{}%
\AgdaOperator{\AgdaFunction{⁻¹}}\AgdaSymbol{)}\<%
\end{code}
\scpad

Finally, one way to show that a function is an embedding is to first prove it is invertible and then invoke the following theorem.
\ccpad
\begin{code}%
\>[0]\AgdaFunction{invertibles-are-embeddings}\AgdaSpace{}%
\AgdaSymbol{:}%
\>[621I]\AgdaSymbol{\{}\AgdaBound{𝓧}\AgdaSpace{}%
\AgdaBound{𝓨}\AgdaSpace{}%
\AgdaSymbol{:}\AgdaSpace{}%
\AgdaPostulate{Universe}\AgdaSymbol{\}}\AgdaSpace{}%
\AgdaSymbol{\{}\AgdaBound{X}\AgdaSpace{}%
\AgdaSymbol{:}\AgdaSpace{}%
\AgdaBound{𝓧}\AgdaSpace{}%
\AgdaOperator{\AgdaFunction{̇}}\AgdaSymbol{\}}\AgdaSpace{}%
\AgdaSymbol{\{}\AgdaBound{Y}\AgdaSpace{}%
\AgdaSymbol{:}\AgdaSpace{}%
\AgdaBound{𝓨}\AgdaSpace{}%
\AgdaOperator{\AgdaFunction{̇}}\AgdaSymbol{\}}\AgdaSpace{}%
\AgdaSymbol{(}\AgdaBound{f}\AgdaSpace{}%
\AgdaSymbol{:}\AgdaSpace{}%
\AgdaBound{X}\AgdaSpace{}%
\AgdaSymbol{→}\AgdaSpace{}%
\AgdaBound{Y}\AgdaSymbol{)}\<%
\\
\>[0][@{}l@{\AgdaIndent{0}}]%
\>[1]\AgdaSymbol{→}%
\>[.][@{}l@{}]\<[621I]%
\>[29]\AgdaFunction{invertible}\AgdaSpace{}%
\AgdaBound{f}\AgdaSpace{}%
\AgdaSymbol{→}\AgdaSpace{}%
\AgdaFunction{is-embedding}\AgdaSpace{}%
\AgdaBound{f}\<%
\\
%
\\[\AgdaEmptyExtraSkip]%
\>[0]\AgdaFunction{invertibles-are-embeddings}\AgdaSpace{}%
\AgdaBound{f}\AgdaSpace{}%
\AgdaBound{fi}\AgdaSpace{}%
\AgdaSymbol{=}\AgdaSpace{}%
\AgdaFunction{equivs-are-embeddings}\AgdaSpace{}%
\AgdaBound{f}\AgdaSpace{}%
\AgdaSymbol{(}\AgdaFunction{invertibles-are-equivs}\AgdaSpace{}%
\AgdaBound{f}\AgdaSpace{}%
\AgdaBound{fi}\AgdaSymbol{)}\<%
\end{code}

%%% Local Variables:
%%% mode: latex
%%% TeX-master: "ualib-part1-20210304.tex"
%%% End:


\subsection{Lifts}\label{sec:lifts-altern-univ}
% : making peace with a noncumulative universe hierarchy
\firstsentence{Overture}{Lifts}
\subsubsection{The noncumulative universe hierarchy}\label{the-noncumulative-hierarchy}

The hierarchy of universe levels in Agda looks like this:\\[-4pt]

\ab{𝓤₀} \as : \ab{𝓤₁}\AgdaOperator{\AgdaFunction{̇}} , ~ ~ \ab{𝓤₁} \as : \ab{𝓤₂}\AgdaOperator{\AgdaFunction{̇}} , ~ ~ \ab{𝓤₂} \as : \ab{𝓤₃}\AgdaOperator{\AgdaFunction{̇}} , \ldots{}\\[4pt]
This means that the type level of \ab{𝓤₀} is \ab{𝓤₁}\AgdaOperator{\AgdaFunction{̇}}, and for each \ab n the type level of \ab{𝓤ₙ} is \ab{𝓤ₙ₊₁}\AgdaOperator{\AgdaFunction{̇}}.
It is important to note, however, this does \emph{not} imply that \ab{𝓤₀} \as : \ab{𝓤₂}\AgdaOperator{\AgdaFunction{̇}} and \ab{𝓤₀} \as : \ab{𝓤₃}\AgdaOperator{\AgdaFunction{̇}}, and so on. In other words, Agda's universe hierarchy is \defn{noncummulative}. This makes it possible to treat universe levels more generally and precisely, which is nice. On the other hand (in this author's experience) a noncummulative hierarchy can sometimes make for a nonfun proof assistant.

Luckily, there are ways to overcome this technical issue. We describe general lifting and lowering functions below, and then later, in \S\ref{lifts-of-algebras},
 % \href{https://ualib.gitlab.io/Algebras.Algebras.html\#lifts-of-algebras}{Lifts
we'll see the domain-specific analogs of these tools which turn out to have some nice properties that make them very effective for resolving universe level problems when working with algebra datatypes.

\subsubsection{Lifting and lowering}\label{lifting-and-lowering}

Let us be more concrete about what is at issue here by giving an example. Agda frequently encounters errors during the type-checking process and responds by printing an error message. Often the message has the following form.
{\color{red}{\small
\begin{verbatim}
  Algebras.lagda:498,20-23 𝓤 != 𝓞 ⊔ 𝓥 ⊔ 𝓤 ⁺ when checking that... has type...
\end{verbatim}}}
\noindent This error message means that Agda encountered the universe \ab{𝓤} on line 498 (columns 20--23) of the file \af{Algebras.lagda}, but was expecting to find the universe \ab{𝓞} \aop ⊔ \ab 𝓥 \aop ⊔ \ab 𝓤\af ⁺ instead.

To make these situations easier to deal with, we have developed some domain specific tools for the lifting and lowering of universe levels inhabited by algebra types. (Later we do the same for other domain specific types like homomorphisms, subalgebras, products, etc). This must be done carefully to avoid making the type theory inconsistent. In particular, we cannot lower the level of a type unless it was previously lifted to a (higher than necessary) universe level.

A general \af{Lift} record type, similar to the one found in the \am{Level} module of the \agdastdlib, is defined as follows.
\ccpad
\begin{code}%
\>[0]\AgdaKeyword{record}\AgdaSpace{}%
\AgdaRecord{Lift}\AgdaSpace{}%
\AgdaSymbol{\{}\AgdaBound{𝓦}\AgdaSpace{}%
\AgdaBound{𝓤}\AgdaSpace{}%
\AgdaSymbol{:}\AgdaSpace{}%
\AgdaPostulate{Universe}\AgdaSymbol{\}}\AgdaSpace{}%
\AgdaSymbol{(}\AgdaBound{X}\AgdaSpace{}%
\AgdaSymbol{:}\AgdaSpace{}%
\AgdaBound{𝓤}\AgdaSpace{}%
\AgdaOperator{\AgdaFunction{̇}}\AgdaSymbol{)}\AgdaSpace{}%
\AgdaSymbol{:}\AgdaSpace{}%
\AgdaBound{𝓤}\AgdaSpace{}%
\AgdaOperator{\AgdaPrimitive{⊔}}\AgdaSpace{}%
\AgdaBound{𝓦}\AgdaSpace{}%
\AgdaOperator{\AgdaFunction{̇}}%
\>[50]\AgdaKeyword{where}\<%
\\
\>[0][@{}l@{\AgdaIndent{0}}]%
\>[1]\AgdaKeyword{constructor}\AgdaSpace{}%
\AgdaInductiveConstructor{lift}\<%
\\
%
\>[1]\AgdaKeyword{field}\AgdaSpace{}%
\AgdaField{lower}\AgdaSpace{}%
\AgdaSymbol{:}\AgdaSpace{}%
\AgdaBound{X}\<%
\\
\>[0]\AgdaKeyword{open}\AgdaSpace{}%
\AgdaModule{Lift}\<%
\end{code}
\ccpad
Using the \af{Lift} type, we define two functions that lift the domain (respectively, codomain) type of a function to a higher universe.
\ccpad
\begin{code}%
\>[1]\AgdaFunction{lift-dom}\AgdaSpace{}%
\AgdaSymbol{:}\AgdaSpace{}%
\AgdaSymbol{\{}\AgdaBound{X}\AgdaSpace{}%
\AgdaSymbol{:}\AgdaSpace{}%
\AgdaBound{𝓧}\AgdaSpace{}%
\AgdaOperator{\AgdaFunction{̇}}\AgdaSymbol{\}\{}\AgdaBound{Y}\AgdaSpace{}%
\AgdaSymbol{:}\AgdaSpace{}%
\AgdaBound{𝓨}\AgdaSpace{}%
\AgdaOperator{\AgdaFunction{̇}}\AgdaSymbol{\}}\AgdaSpace{}%
\AgdaSymbol{→}\AgdaSpace{}%
\AgdaSymbol{(}\AgdaBound{X}\AgdaSpace{}%
\AgdaSymbol{→}\AgdaSpace{}%
\AgdaBound{Y}\AgdaSymbol{)}\AgdaSpace{}%
\AgdaSymbol{→}\AgdaSpace{}%
\AgdaSymbol{(}\AgdaRecord{Lift}\AgdaSymbol{\{}\AgdaBound{𝓦}\AgdaSymbol{\}\{}\AgdaBound{𝓧}\AgdaSymbol{\}}\AgdaSpace{}%
\AgdaBound{X}\AgdaSpace{}%
\AgdaSymbol{→}\AgdaSpace{}%
\AgdaBound{Y}\AgdaSymbol{)}\<%
\\
%
\>[1]\AgdaFunction{lift-dom}\AgdaSpace{}%
\AgdaBound{f}\AgdaSpace{}%
\AgdaSymbol{=}\AgdaSpace{}%
\AgdaSymbol{λ}\AgdaSpace{}%
\AgdaBound{x}\AgdaSpace{}%
\AgdaSymbol{→}\AgdaSpace{}%
\AgdaSymbol{(}\AgdaBound{f}\AgdaSpace{}%
\AgdaSymbol{(}\AgdaField{lower}\AgdaSpace{}%
\AgdaBound{x}\AgdaSymbol{))}\<%
\\
%
\\[\AgdaEmptyExtraSkip]%
%
\>[1]\AgdaFunction{lift-cod}\AgdaSpace{}%
\AgdaSymbol{:}\AgdaSpace{}%
\AgdaSymbol{\{}\AgdaBound{X}\AgdaSpace{}%
\AgdaSymbol{:}\AgdaSpace{}%
\AgdaBound{𝓧}\AgdaSpace{}%
\AgdaOperator{\AgdaFunction{̇}}\AgdaSymbol{\}\{}\AgdaBound{Y}\AgdaSpace{}%
\AgdaSymbol{:}\AgdaSpace{}%
\AgdaBound{𝓨}\AgdaSpace{}%
\AgdaOperator{\AgdaFunction{̇}}\AgdaSymbol{\}}\AgdaSpace{}%
\AgdaSymbol{→}\AgdaSpace{}%
\AgdaSymbol{(}\AgdaBound{X}\AgdaSpace{}%
\AgdaSymbol{→}\AgdaSpace{}%
\AgdaBound{Y}\AgdaSymbol{)}\AgdaSpace{}%
\AgdaSymbol{→}\AgdaSpace{}%
\AgdaSymbol{(}\AgdaBound{X}\AgdaSpace{}%
\AgdaSymbol{→}\AgdaSpace{}%
\AgdaRecord{Lift}\AgdaSymbol{\{}\AgdaBound{𝓦}\AgdaSymbol{\}\{}\AgdaBound{𝓨}\AgdaSymbol{\}}\AgdaSpace{}%
\AgdaBound{Y}\AgdaSymbol{)}\<%
\\
%
\>[1]\AgdaFunction{lift-cod}\AgdaSpace{}%
\AgdaBound{f}\AgdaSpace{}%
\AgdaSymbol{=}\AgdaSpace{}%
\AgdaSymbol{λ}\AgdaSpace{}%
\AgdaBound{x}\AgdaSpace{}%
\AgdaSymbol{→}\AgdaSpace{}%
\AgdaInductiveConstructor{lift}\AgdaSpace{}%
\AgdaSymbol{(}\AgdaBound{f}\AgdaSpace{}%
\AgdaBound{x}\AgdaSymbol{)}\<%
\end{code}
\ccpad
Naturally, we can combine these and define a function that lifts both the domain and codomain type.
\ccpad
\begin{code}
\>[1]\AgdaFunction{lift-fun}\AgdaSpace{}%
\AgdaSymbol{:}\AgdaSpace{}%
\AgdaSymbol{\{}\AgdaBound{X}\AgdaSpace{}%
\AgdaSymbol{:}\AgdaSpace{}%
\AgdaBound{𝓧}\AgdaSpace{}%
\AgdaOperator{\AgdaFunction{̇}}\AgdaSymbol{\}\{}\AgdaBound{Y}\AgdaSpace{}%
\AgdaSymbol{:}\AgdaSpace{}%
\AgdaBound{𝓨}\AgdaSpace{}%
\AgdaOperator{\AgdaFunction{̇}}\AgdaSymbol{\}}\AgdaSpace{}%
\AgdaSymbol{→}\AgdaSpace{}%
\AgdaSymbol{(}\AgdaBound{X}\AgdaSpace{}%
\AgdaSymbol{→}\AgdaSpace{}%
\AgdaBound{Y}\AgdaSymbol{)}\AgdaSpace{}%
\AgdaSymbol{→}\AgdaSpace{}%
\AgdaSymbol{(}\AgdaRecord{Lift}\AgdaSymbol{\{}\AgdaBound{𝓦}\AgdaSymbol{\}\{}\AgdaBound{𝓧}\AgdaSymbol{\}}\AgdaSpace{}%
\AgdaBound{X}\AgdaSpace{}%
\AgdaSymbol{→}\AgdaSpace{}%
\AgdaRecord{Lift}\AgdaSymbol{\{}\AgdaBound{𝓩}\AgdaSymbol{\}\{}\AgdaBound{𝓨}\AgdaSymbol{\}}\AgdaSpace{}%
\AgdaBound{Y}\AgdaSymbol{)}\<%
\\
%
\>[1]\AgdaFunction{lift-fun}\AgdaSpace{}%
\AgdaBound{f}\AgdaSpace{}%
\AgdaSymbol{=}\AgdaSpace{}%
\AgdaSymbol{λ}\AgdaSpace{}%
\AgdaBound{x}\AgdaSpace{}%
\AgdaSymbol{→}\AgdaSpace{}%
\AgdaInductiveConstructor{lift}\AgdaSpace{}%
\AgdaSymbol{(}\AgdaBound{f}\AgdaSpace{}%
\AgdaSymbol{(}\AgdaField{lower}\AgdaSpace{}%
\AgdaBound{x}\AgdaSymbol{))}\<%
\end{code}
\ccpad
For example, \af{lift-dom} takes a function \ab f defined on the domain \AgdaBound{X}\AgdaSpace{}\AgdaSymbol{:}\AgdaSpace{}\AgdaBound{𝓧}\AgdaSpace{}\AgdaOperator{\AgdaFunction{̇}}\AgdaSpace{} and returns a function defined on the domain \ar{Lift}\as{\{}\ab 𝓦\as{\}}\as{\{}\ab 𝓧\as{\}} \ab X \as : \ab 𝓧 \aop ⊔ \ab 𝓦\aof ̇.

The point of a ramified hierarchy of universes is to avoid Russell's paradox, and this would be subverted if we lower the universe of a type that hasn't already be lifted.  Fortunately, however, we can prove that \aic{lift} and \afld{lower} compose to the identity. Later, there will be some situations that require these facts, so we formalize them and their (trivial) proofs.
\ccpad
\begin{code}%
\>[0]\AgdaFunction{lower∼lift}\AgdaSpace{}%
\AgdaSymbol{:}\AgdaSpace{}%
\AgdaSymbol{\{}\AgdaBound{𝓦}\AgdaSpace{}%
\AgdaBound{𝓧}\AgdaSpace{}%
\AgdaSymbol{:}\AgdaSpace{}%
\AgdaPostulate{Universe}\AgdaSymbol{\}\{}\AgdaBound{X}\AgdaSpace{}%
\AgdaSymbol{:}\AgdaSpace{}%
\AgdaBound{𝓧}\AgdaSpace{}%
\AgdaOperator{\AgdaFunction{̇}}\AgdaSymbol{\}}\AgdaSpace{}%
\AgdaSymbol{→}\AgdaSpace{}%
\AgdaField{lower}\AgdaSymbol{\{}\AgdaBound{𝓦}\AgdaSymbol{\}\{}\AgdaBound{𝓧}\AgdaSymbol{\}}\AgdaSpace{}%
\AgdaOperator{\AgdaFunction{∘}}\AgdaSpace{}%
\AgdaInductiveConstructor{lift}\AgdaSpace{}%
\AgdaOperator{\AgdaDatatype{≡}}\AgdaSpace{}%
\AgdaFunction{𝑖𝑑}\AgdaSpace{}%
\AgdaBound{X}\<%
\\
\>[0]\AgdaFunction{lower∼lift}\AgdaSpace{}%
\AgdaSymbol{=}\AgdaSpace{}%
\AgdaInductiveConstructor{refl}\AgdaSpace{}%
\AgdaSymbol{\AgdaUnderscore{}}\<%
\\
%
\\[\AgdaEmptyExtraSkip]%
\>[0]\AgdaFunction{lift∼lower}\AgdaSpace{}%
\AgdaSymbol{:}\AgdaSpace{}%
\AgdaSymbol{\{}\AgdaBound{𝓦}\AgdaSpace{}%
\AgdaBound{𝓧}\AgdaSpace{}%
\AgdaSymbol{:}\AgdaSpace{}%
\AgdaPostulate{Universe}\AgdaSymbol{\}\{}\AgdaBound{X}\AgdaSpace{}%
\AgdaSymbol{:}\AgdaSpace{}%
\AgdaBound{𝓧}\AgdaSpace{}%
\AgdaOperator{\AgdaFunction{̇}}\AgdaSymbol{\}}\AgdaSpace{}%
\AgdaSymbol{→}\AgdaSpace{}%
\AgdaInductiveConstructor{lift}\AgdaSpace{}%
\AgdaOperator{\AgdaFunction{∘}}\AgdaSpace{}%
\AgdaField{lower}\AgdaSpace{}%
\AgdaOperator{\AgdaDatatype{≡}}\AgdaSpace{}%
\AgdaFunction{𝑖𝑑}\AgdaSpace{}%
\AgdaSymbol{(}\AgdaRecord{Lift}\AgdaSymbol{\{}\AgdaBound{𝓦}\AgdaSymbol{\}\{}\AgdaBound{𝓧}\AgdaSymbol{\}}\AgdaSpace{}%
\AgdaBound{X}\AgdaSymbol{)}\<%
\\
\>[0]\AgdaFunction{lift∼lower}\AgdaSpace{}%
\AgdaSymbol{=}\AgdaSpace{}%
\AgdaInductiveConstructor{refl}\AgdaSpace{}%
\AgdaSymbol{\AgdaUnderscore{}}\<%
\end{code}


%%% Local Variables:
%%% mode: latex
%%% TeX-master: "ualib-part1-20210304.tex"
%%% End:





% % -*- TeX-master: "ualib-part1.tex" -*-
%%% Local Variables: 
%%% mode: latex
%%% TeX-engine: 'xetex
%%% End:
\section{Relation Types}\label{sec:relat-quot-types}
We now present the \agdaualib's \ualibhtml{Relations} module and its submodules. In \S\ref{sec:discrete-relations} we define types that represent \emph{unary} and \emph{binary relations}, which we refer to as ``discrete relations'' to contrast them with the (``continuous'') \emph{general} and \emph{dependent relations} that we introduce in \S\ref{sec:continuous-relations}. We call the latter ``continuous relations'' because they can have arbitrary arity (general relations) and they can be defined over arbitrary families of types (dependent relations).

\subsection{Discrete relations}\label{sec:discrete-relations}
% : predicates, axiom of extensionality, compatibility
\firstsentence{Relations}{Discrete}
% \context{%
% \{\ab 𝓤 \ab 𝓦 \ab 𝓧 \ab 𝓨 \ab 𝓩 \as : \AgdaPostulate{Universe}\}%
% \{\ab A \as : \ab 𝓤\AgdaOperator{\AgdaFunction{̇}}\}%
% \{\ab B \as : \ab 𝓦\AgdaOperator{\AgdaFunction{̇}}\}%
% \{\ab C \as : \ab 𝓩\AgdaOperator{\AgdaFunction{̇}}\}%
% }
% -*- TeX-master: "ualib-part1.tex" -*-
%%% Local Variables: 
%%% mode: latex
%%% TeX-engine: 'xetex
%%% End:
\paragraph*{Unary relations}

In set theory, given two sets \ab A and \ab P, we say that \ab P is a \defn{subset} of \ab A, and we write \ab P \af ⊆ \ab A, just in case \as ∀~\ab x~(\ab x~\af ∈~\ab P~\as →~\ab x~\af ∈~\ab A). We need a mechanism for representing this notion in Agda. A typical approach is to use a \emph{unary predicate} type which we will denote by \af{Pred} and define as follows.  Given two universes \ab 𝓤 \ab 𝓦 and a type \ab A~\as :~\ab 𝓤\af ̇, the type \af{Pred}~\ab A~\ab 𝓦 represents \emph{properties} that inhabitants of \ab A may or may not satisfy.  We write \ab P~\as :~\af{Pred}~\ab A~\ab 𝓤 to represent the % semantic concept of the 
collection of inhabitants of \ab A that satisfy (or belong to) \ab P. Here is the definition.\footnote{\label{relunary}cf.~\texttt{Relation/Unary.agda} in the \agdastdlib.}
\ccpad
\begin{code}%
\>[0]\AgdaFunction{Pred}\AgdaSpace{}%
\AgdaSymbol{:}\AgdaSpace{}%
% \AgdaSymbol{\{}\AgdaBound{𝓤}\AgdaSpace{}%
% \AgdaSymbol{:}\AgdaSpace{}%
% \AgdaPostulate{Universe}\AgdaSymbol{\}}\AgdaSpace{}%
% \AgdaSymbol{→}\AgdaSpace{}%
\AgdaBound{𝓤}\AgdaSpace{}%
\AgdaOperator{\AgdaFunction{̇}}\AgdaSpace{}%
\AgdaSymbol{→}\AgdaSpace{}%
\AgdaSymbol{(}\AgdaBound{𝓦}\AgdaSpace{}%
\AgdaSymbol{:}\AgdaSpace{}%
\AgdaPostulate{Universe}\AgdaSymbol{)}\AgdaSpace{}%
\AgdaSymbol{→}\AgdaSpace{}%
\AgdaBound{𝓤}\AgdaSpace{}%
\AgdaOperator{\AgdaPrimitive{⊔}}\AgdaSpace{}%
\AgdaBound{𝓦}\AgdaSpace{}%
\AgdaOperator{\AgdaPrimitive{⁺}}\AgdaSpace{}%
\AgdaOperator{\AgdaFunction{̇}}\<%
\\
\>[0]\AgdaFunction{Pred}\AgdaSpace{}%
\AgdaBound{A}\AgdaSpace{}%
\AgdaBound{𝓦}\AgdaSpace{}%
\AgdaSymbol{=}\AgdaSpace{}%
\AgdaBound{A}\AgdaSpace{}%
\AgdaSymbol{→}\AgdaSpace{}%
\AgdaBound{𝓦}\AgdaSpace{}%
\AgdaOperator{\AgdaFunction{̇}}\<%
\end{code}
\ccpad
This is a general unary predicate type, but by taking the codomain to be \af{Bool} = \{0, 1\}, we obtain the usual interpretation of set membership; that is,
% \footnote{The type \af 𝟚 is defined in the \AgdaModule{MGS-MLTT} module of \typtop.}  
for \ab P~\as :~\af{Pred}~\ab A~\af{Bool} and for each \abt{x}{A}, we would interpret \ab P~\ab x~\ad ≡ 0 to mean \ab x ∉ \ab A, and \ab P~\ab x~\ad ≡ 1 to mean \ab x ∈ \ab A.


% To reiterate, given a type \ab A \as : \ab 𝓤\af ̇, we think of \af{Pred} \ab A \ab 𝓦 as the type of a property that inhabitants of \ab A may or may not satisfy. If \ab P \as : \af{Pred} \ab A \ab 𝓦, then we view \ab P as a collection of inhabitants of type \ab A that ``satisfy property \ab P,'' or that ``belong to the subset \ab P of \ab A.''
% Later we consider predicates over the class of algebras in a given signature. In the \ualibhtml{Algebras} module we will define the type \af{Algebra} \ab 𝓤 \ab 𝑆 of \ab 𝑆-algebras with domain type \ab 𝓤\af ̇, and the type \af{Pred} (\af{Algebra} \ab 𝓤 \ab 𝑆) \ab 𝓦 will represent classes of \ab 𝑆-algebras with certain properties.





%\subsubsection{Membership and inclusion relations}\label{membership-and-inclusion-relations}
% Like the \agdastdlib, t
The \ualib includes types that represent the \emph{element inclusion} and \emph{subset inclusion} relations from set theory. For example, given a predicate \af P, we may represent that  ``\ab x belongs to \af P'' or that ``\ab x has property \af{P},'' by writing either \ab x \af ∈ \af P or \af P \ab x.  The definition of \af ∈ is standard, as is the definition of \af{⊆} for the \defn{subset} relation; nonetheless, here they are.\cref{relunary}
\ccpad
\begin{code}%
\>[0]\AgdaOperator{\AgdaFunction{\AgdaUnderscore{}∈\AgdaUnderscore{}}}\AgdaSpace{}%
\AgdaSymbol{:}\AgdaSpace{}%
\AgdaBound{A}\AgdaSpace{}%
\AgdaSymbol{→}\AgdaSpace{}%
\AgdaFunction{Pred}\AgdaSpace{}%
\AgdaBound{A}\AgdaSpace{}%
\AgdaBound{𝓦}\AgdaSpace{}%
\AgdaSymbol{→}\AgdaSpace{}%
\AgdaBound{𝓦}\AgdaSpace{}%
\AgdaOperator{\AgdaFunction{̇}}\<%
\\
\>[0]\AgdaBound{x}\AgdaSpace{}%
\AgdaOperator{\AgdaFunction{∈}}\AgdaSpace{}%
\AgdaBound{P}\AgdaSpace{}%
\AgdaSymbol{=}\AgdaSpace{}%
\AgdaBound{P}\AgdaSpace{}%
\AgdaBound{x}\<%
\end{code}
\scpad
\begin{code}%
\>[1]\AgdaOperator{\AgdaFunction{\AgdaUnderscore{}⊆\AgdaUnderscore{}}}\AgdaSpace{}%
\AgdaSymbol{:}\AgdaSpace{}%
\AgdaFunction{Pred}\AgdaSpace{}%
\AgdaBound{A}\AgdaSpace{}%
\AgdaBound{𝓦}\AgdaSpace{}%
\AgdaSymbol{→}\AgdaSpace{}%
\AgdaFunction{Pred}\AgdaSpace{}%
\AgdaBound{A}\AgdaSpace{}%
\AgdaBound{𝓩}\AgdaSpace{}%
\AgdaSymbol{→}\AgdaSpace{}%
\AgdaBound{𝓤}\AgdaSpace{}%
\AgdaOperator{\AgdaPrimitive{⊔}}\AgdaSpace{}%
\AgdaBound{𝓦}\AgdaSpace{}%
\AgdaOperator{\AgdaPrimitive{⊔}}\AgdaSpace{}%
\AgdaBound{𝓩}\AgdaSpace{}%
\AgdaOperator{\AgdaFunction{̇}}\<%
\\
%
\>[1]\AgdaBound{P}\AgdaSpace{}%
\AgdaOperator{\AgdaFunction{⊆}}\AgdaSpace{}%
\AgdaBound{Q}\AgdaSpace{}%
\AgdaSymbol{=}\AgdaSpace{}%
\AgdaSymbol{∀}\AgdaSpace{}%
\AgdaSymbol{\{}\AgdaBound{x}\AgdaSymbol{\}}\AgdaSpace{}%
\AgdaSymbol{→}\AgdaSpace{}%
\AgdaBound{x}\AgdaSpace{}%
\AgdaOperator{\AgdaFunction{∈}}\AgdaSpace{}%
\AgdaBound{P}\AgdaSpace{}%
\AgdaSymbol{→}\AgdaSpace{}%
\AgdaBound{x}\AgdaSpace{}%
\AgdaOperator{\AgdaFunction{∈}}\AgdaSpace{}%
\AgdaBound{Q}\<%
\end{code}


\paragraph*{Predicates toolbox}
Here is a small collection of tools that will come in handy later. The first is an inductive type that represents \defn{disjoint union}.\footnote{\label{uhints}%
  \textbf{Unicode Hints}. In \agdamode, \texttt{\textbackslash{}u+} ↝ \af{⊎}, \texttt{\textbackslash{}b0} ↝ \af 𝟘, \texttt{\textbackslash{}B0} ↝ \af 𝟎.}
\ccpad % \texttt{\textbackslash{}.=} ↝ \af{≐}, , \texttt{\textbackslash{}b1} ↝ \af 𝟙, \texttt{\textbackslash{}B1} ↝ \af 𝟏
\begin{code}
\>[0]\AgdaKeyword{data}\AgdaSpace{}%
\AgdaOperator{\AgdaDatatype{\AgdaUnderscore{}⊎\AgdaUnderscore{}}}\AgdaSpace{}%
% \AgdaSymbol{\{}\AgdaBound{𝓤}\AgdaSpace{}%
% \AgdaBound{𝓦}\AgdaSpace{}%
% \AgdaSymbol{:}\AgdaSpace{}%
% \AgdaPostulate{Universe}\AgdaSymbol{\}}
\AgdaSymbol{(}\AgdaBound{A}\AgdaSpace{}%
\AgdaSymbol{:}\AgdaSpace{}%
\AgdaBound{𝓤}\AgdaSpace{}%
\AgdaOperator{\AgdaFunction{̇}}\AgdaSymbol{)}\AgdaSpace{}%
\AgdaSymbol{(}\AgdaBound{B}\AgdaSpace{}%
\AgdaSymbol{:}\AgdaSpace{}%
\AgdaBound{𝓦}\AgdaSpace{}%
\AgdaOperator{\AgdaFunction{̇}}\AgdaSymbol{)}\AgdaSpace{}%
\AgdaSymbol{:}\AgdaSpace{}%
\AgdaBound{𝓤}\AgdaSpace{}%
\AgdaOperator{\AgdaPrimitive{⊔}}\AgdaSpace{}%
\AgdaBound{𝓦}\AgdaSpace{}%
\AgdaOperator{\AgdaFunction{̇}}\AgdaSpace{}%
\AgdaKeyword{where}\<%
\\
\>[0][@{}l@{\AgdaIndent{0}}]%
\>[1]\AgdaInductiveConstructor{inj₁}\AgdaSpace{}%
\AgdaSymbol{:}\AgdaSpace{}%
\AgdaSymbol{(}\AgdaBound{x}\AgdaSpace{}%
\AgdaSymbol{:}\AgdaSpace{}%
\AgdaBound{A}\AgdaSymbol{)}\AgdaSpace{}%
\AgdaSymbol{→}\AgdaSpace{}%
\AgdaBound{A}\AgdaSpace{}%
\AgdaOperator{\AgdaDatatype{⊎}}\AgdaSpace{}%
\AgdaBound{B}\<%
\\
%
\>[1]\AgdaInductiveConstructor{inj₂}\AgdaSpace{}%
\AgdaSymbol{:}\AgdaSpace{}%
\AgdaSymbol{(}\AgdaBound{y}\AgdaSpace{}%
\AgdaSymbol{:}\AgdaSpace{}%
\AgdaBound{B}\AgdaSymbol{)}\AgdaSpace{}%
\AgdaSymbol{→}\AgdaSpace{}%
\AgdaBound{A}\AgdaSpace{}%
\AgdaOperator{\AgdaDatatype{⊎}}\AgdaSpace{}%
\AgdaBound{B}\<%
\end{code}
\ccpad
And this can be used to define a type representing \defn{union}, as follows.
\ccpad
\begin{code}
\>[0]\AgdaOperator{\AgdaFunction{\AgdaUnderscore{}∪\AgdaUnderscore{}}}\AgdaSpace{}%
\AgdaSymbol{:}\AgdaSpace{}%
% \AgdaSymbol{\{}\AgdaBound{𝓤}\AgdaSpace{}%
% \AgdaBound{𝓦}\AgdaSpace{}%
% \AgdaBound{𝓩}\AgdaSpace{}%
% \AgdaSymbol{:}\AgdaSpace{}%
% \AgdaPostulate{Universe}\AgdaSymbol{\}\{}\AgdaBound{A}\AgdaSpace{}%
% \AgdaSymbol{:}\AgdaSpace{}%
% \AgdaBound{𝓤}\AgdaSpace{}%
% \AgdaOperator{\AgdaFunction{̇}}\AgdaSymbol{\}}\AgdaSpace{}%
% \AgdaSymbol{→}\AgdaSpace{}%
\AgdaFunction{Pred}\AgdaSpace{}%
\AgdaBound{A}\AgdaSpace{}%
\AgdaBound{𝓦}\AgdaSpace{}%
\AgdaSymbol{→}\AgdaSpace{}%
\AgdaFunction{Pred}\AgdaSpace{}%
\AgdaBound{A}\AgdaSpace{}%
\AgdaBound{𝓩}\AgdaSpace{}%
\AgdaSymbol{→}\AgdaSpace{}%
\AgdaFunction{Pred}\AgdaSpace{}%
\AgdaBound{A}\AgdaSpace{}%
\AgdaSymbol{(}\AgdaBound{𝓦}\AgdaSpace{}%
\AgdaOperator{\AgdaPrimitive{⊔}}\AgdaSpace{}%
\AgdaBound{𝓩}\AgdaSymbol{)}\<%
\\
\>[0]\AgdaBound{P}\AgdaSpace{}%
\AgdaOperator{\AgdaFunction{∪}}\AgdaSpace{}%
\AgdaBound{Q}\AgdaSpace{}%
\AgdaSymbol{=}\AgdaSpace{}%
\AgdaSymbol{λ}\AgdaSpace{}%
\AgdaBound{x}\AgdaSpace{}%
\AgdaSymbol{→}\AgdaSpace{}%
\AgdaBound{x}\AgdaSpace{}%
\AgdaOperator{\AgdaFunction{∈}}\AgdaSpace{}%
\AgdaBound{P}\AgdaSpace{}%
\AgdaOperator{\AgdaDatatype{⊎}}\AgdaSpace{}%
\AgdaBound{x}\AgdaSpace{}%
\AgdaOperator{\AgdaFunction{∈}}\AgdaSpace{}%
\AgdaBound{Q}\<%
\end{code}
\ccpad
Next we define convenient notation for asserting that the image of a function (the first argument) is contained in a predicate (the second argument).
\ccpad
\begin{code}
\>[1]\AgdaOperator{\AgdaFunction{Im\AgdaUnderscore{}⊆\AgdaUnderscore{}}}\AgdaSpace{}%
\AgdaSymbol{:}\AgdaSpace{}%
\AgdaSymbol{(}\AgdaBound{A}\AgdaSpace{}%
\AgdaSymbol{→}\AgdaSpace{}%
\AgdaBound{B}\AgdaSymbol{)}\AgdaSpace{}%
\AgdaSymbol{→}\AgdaSpace{}%
\AgdaFunction{Pred}\AgdaSpace{}%
\AgdaBound{B}\AgdaSpace{}%
\AgdaBound{𝓩}\AgdaSpace{}%
\AgdaSymbol{→}\AgdaSpace{}%
\AgdaBound{𝓤}\AgdaSpace{}%
\AgdaOperator{\AgdaPrimitive{⊔}}\AgdaSpace{}%
\AgdaBound{𝓩}\AgdaSpace{}%
\AgdaOperator{\AgdaFunction{̇}}\<%
\\
%
\>[1]\AgdaOperator{\AgdaFunction{Im}}\AgdaSpace{}%
\AgdaBound{f}\AgdaSpace{}%
\AgdaOperator{\AgdaFunction{⊆}}\AgdaSpace{}%
\AgdaBound{S}\AgdaSpace{}%
\AgdaSymbol{=}\AgdaSpace{}%
\AgdaSymbol{∀}\AgdaSpace{}%
\AgdaBound{x}\AgdaSpace{}%
\AgdaSymbol{→}\AgdaSpace{}%
\AgdaBound{f}\AgdaSpace{}%
\AgdaBound{x}\AgdaSpace{}%
\AgdaOperator{\AgdaFunction{∈}}\AgdaSpace{}%
\AgdaBound{S}\<%
\end{code}
\ccpad
The \defn{empty set} is naturally represented by the \defn{empty type}, \af 𝟘, and the latter is defined in the \am{Empty-Type} module of \typtop.\cref{uhints}$^, $\footnote{%
The empty type is an inductive type with no constructors; that is, \AgdaKeyword{data}\AgdaSpace{}%
\AgdaDatatype{𝟘}\AgdaSpace{}%
\AgdaSymbol{\{}\AgdaBound{𝓤}\AgdaSymbol{\}}\AgdaSpace{}%
\AgdaSymbol{:}\AgdaSpace{}%
\AgdaBound{𝓤}\AgdaSpace{}%
\AgdaOperator{\AgdaFunction{̇}}\AgdaSpace{}%
\AgdaKeyword{where}\AgdaSpace{}%
\AgdaComment{-- (empty body)}.}
\ccpad
\begin{code}
% \>[0]\AgdaKeyword{open}\AgdaSpace{}%
% \AgdaKeyword{import}\AgdaSpace{}%
% \AgdaModule{Empty-Type}\AgdaSpace{}%
% \AgdaKeyword{using}\AgdaSpace{}%
% \AgdaSymbol{(}\AgdaDatatype{𝟘}\AgdaSymbol{)}\<%
% \\
% %
% \\[\AgdaEmptyExtraSkip]%
\>[0]\AgdaFunction{∅}\AgdaSpace{}%
\AgdaSymbol{:}\AgdaSpace{}%
% \AgdaSymbol{\{}\AgdaBound{𝓤}\AgdaSpace{}%
% \AgdaSymbol{:}\AgdaSpace{}%
% \AgdaPostulate{Universe}\AgdaSymbol{\}\{}\AgdaBound{A}\AgdaSpace{}%
% \AgdaSymbol{:}\AgdaSpace{}%
% \AgdaBound{𝓤}\AgdaSpace{}%
% \AgdaOperator{\AgdaFunction{̇}}\AgdaSymbol{\}}\AgdaSpace{}%
% \AgdaSymbol{→}\AgdaSpace{}%
\AgdaFunction{Pred}\AgdaSpace{}%
\AgdaBound{A}\AgdaSpace{}%
\AgdaPrimitive{𝓤₀}\<%
\\
\>[0]\AgdaFunction{∅}\AgdaSpace{}%
\AgdaSymbol{\AgdaUnderscore{}}\AgdaSpace{}%
\AgdaSymbol{=}\AgdaSpace{}%
\AgdaDatatype{𝟘}\<%
\end{code}
\ccpad
We close our little predicates toolbox with a natural way to encode \defn{singletons}.
\ccpad
\begin{code}
\>[0]\AgdaOperator{\AgdaFunction{{\AgdaUnderscore{}}}}\AgdaSpace{}%
\AgdaSymbol{:}\AgdaSpace{}%
% \AgdaSymbol{\{}\AgdaBound{𝓤}\AgdaSpace{}%
% \AgdaSymbol{:}\AgdaSpace{}%
% \AgdaPostulate{Universe}\AgdaSymbol{\}\{}\AgdaBound{A}\AgdaSpace{}%
% \AgdaSymbol{:}\AgdaSpace{}%
% \AgdaBound{𝓤}\AgdaSpace{}%
% \AgdaOperator{\AgdaFunction{̇}}\AgdaSymbol{\}}\AgdaSpace{}%
% \AgdaSymbol{→}\AgdaSpace{}%
\AgdaBound{A}\AgdaSpace{}%
\AgdaSymbol{→}\AgdaSpace{}%
\AgdaFunction{Pred}\AgdaSpace{}%
\AgdaBound{A}\AgdaSpace{}%
\AgdaSymbol{\AgdaUnderscore{}}\<%
\\
\>[0]\AgdaOperator{\AgdaFunction{{}}\AgdaSpace{}%
\AgdaBound{x}\AgdaSpace{}%
\AgdaOperator{\AgdaFunction{}}}\AgdaSpace{}%
\AgdaSymbol{=}\AgdaSpace{}%
\AgdaBound{x}\AgdaSpace{}%
\AgdaOperator{\AgdaDatatype{≡\AgdaUnderscore{}}}\<%
\end{code}





\paragraph*{Binary Relations} %\label{sec:binary-relations}
In set theory, a binary relation on a set \ab{A} is simply a subset of the Cartesian product \ab A × \ab A. As such, we could model such a relation as a (unary) predicate over the product type \ab A \af × \ab A, or as an inhabitant of the function type \ab A \as → \ab A \as → \ab 𝓦\af ̇ (for some universe \ab 𝓦).  Note, however, this is not the same as a unary predicate over the function type \ab A \as → \ab A since the latter has type  (\ab A~\as →~\ab A)~\as →~\ab 𝓦\af ̇, while a binary relation should have type \ab A~\as →~(\ab A~\as →~\ab 𝓦\af ̇). 
% \footnote{Note that a binary relation from \ab A to \ab B is not simply a unary predicate over the type \ab A~\as →~\ab B.  The binary relation has type \ab A~\as →~(\ab B~\as →~\ab 𝓩\af ̇) whereas a unary predicate over \ab A~\as →~\ab B has type (\ab A~\as →~\ab B)~\as →~\ab 𝓩\af ̇.}

A generalization of the notion of binary relation is a \emph{relation from} \ab{A} \emph{to} \ab{B}, which we define first and treat binary relations on a single \ab{A} as a special case.
\ccpad
\begin{code}
\>[1]\AgdaFunction{REL}\AgdaSpace{}%
\AgdaSymbol{:}\AgdaSpace{}%
\AgdaBound{𝓤}%
\AgdaOperator{\AgdaFunction{̇}}\AgdaSpace{}%
\AgdaSymbol{→}\AgdaSpace{}%
\AgdaBound{𝓦}%
\AgdaOperator{\AgdaFunction{̇}}\AgdaSpace{}%
\AgdaSymbol{→}\AgdaSpace{}%
\AgdaSymbol{(}\AgdaBound{𝓩}\AgdaSpace{}%
\AgdaSymbol{:}\AgdaSpace{}%
\AgdaPostulate{Universe}\AgdaSymbol{)}\AgdaSpace{}%
\AgdaSymbol{→}\AgdaSpace{}%
\AgdaBound{𝓤}\AgdaSpace{}%
\AgdaOperator{\AgdaPrimitive{⊔}}\AgdaSpace{}%
\AgdaBound{𝓦}\AgdaSpace{}%
\AgdaOperator{\AgdaPrimitive{⊔}}\AgdaSpace{}%
\AgdaBound{𝓩}\AgdaSpace{}%
\AgdaOperator{\AgdaPrimitive{⁺}}\AgdaSpace{}%
\AgdaOperator{\AgdaFunction{̇}}\<%
\\
%
\>[1]\AgdaFunction{REL}\AgdaSpace{}%
\AgdaBound{A}\AgdaSpace{}%
\AgdaBound{B}\AgdaSpace{}%
\AgdaBound{𝓩}\AgdaSpace{}%
\AgdaSymbol{=}\AgdaSpace{}%
\AgdaBound{A}\AgdaSpace{}%
\AgdaSymbol{→}\AgdaSpace{}%
\AgdaBound{B}\AgdaSpace{}%
\AgdaSymbol{→}\AgdaSpace{}%
\AgdaBound{𝓩}%
\AgdaOperator{\AgdaFunction{̇}}\<%
\\
%
\\[\AgdaEmptyExtraSkip]%
%
\>[1]\AgdaFunction{Rel}\AgdaSpace{}%
\AgdaSymbol{:}\AgdaSpace{}%
\AgdaBound{𝓤}%
\AgdaOperator{\AgdaFunction{̇}}\AgdaSpace{}%
\AgdaSymbol{→}\AgdaSpace{}%
\AgdaSymbol{(}\AgdaBound{𝓩}\AgdaSpace{}%
\AgdaSymbol{:}\AgdaSpace{}%
\AgdaPostulate{Universe}\AgdaSymbol{)}\AgdaSpace{}%
\AgdaSymbol{→}\AgdaSpace{}%
\AgdaBound{𝓤}\AgdaSpace{}%
\AgdaOperator{\AgdaPrimitive{⊔}}\AgdaSpace{}%
\AgdaBound{𝓩}\AgdaSpace{}%
\AgdaOperator{\AgdaPrimitive{⁺}}%
\AgdaOperator{\AgdaFunction{̇}}\<%
\\
%
\>[1]\AgdaFunction{Rel}\AgdaSpace{}%
\AgdaBound{A}\AgdaSpace{}%
\AgdaBound{𝓩}\AgdaSpace{}%
\AgdaSymbol{=}\AgdaSpace{}%
\AgdaFunction{REL}\AgdaSpace{}%
\AgdaBound{A}\AgdaSpace{}%
\AgdaBound{A}\AgdaSpace{}%
\AgdaBound{𝓩}\<%
\end{code}

\paragraph*{The kernel of a function}
The \defn{kernel} of a function \ab f \as : \ab A \AgdaSymbol{→} \ab B is defined informally by \{(\ab x , \ab y) ∈ \ab A × \ab A : \ab f \ab x = \ab f \ab y\}. This can be represented in type theory in a number of ways, each of which may be useful in a particular context. For example, we could define the kernel to be an inhabitant of a (binary) relation type, a (unary) predicate type, a (curried) Sigma type, or an (uncurried) Sigma type. Since the first two alternatives are the ones we use thoughout the \ualib, we present them here.
\ccpad
\begin{code}%
\>[1]\AgdaFunction{ker}\AgdaSpace{}%
\AgdaSymbol{:}\AgdaSpace{}%
\AgdaSymbol{(}\AgdaBound{A}\AgdaSpace{}%
\AgdaSymbol{→}\AgdaSpace{}%
\AgdaBound{B}\AgdaSymbol{)}\AgdaSpace{}%
\AgdaSymbol{→}\AgdaSpace{}%
\AgdaFunction{Rel}\AgdaSpace{}%
\AgdaBound{A}\AgdaSpace{}%
\AgdaBound{𝓦}\<%
\\
%
\>[1]\AgdaFunction{ker}\AgdaSpace{}%
\AgdaBound{g}\AgdaSpace{}%
\AgdaBound{x}\AgdaSpace{}%
\AgdaBound{y}\AgdaSpace{}%
\AgdaSymbol{=}\AgdaSpace{}%
\AgdaBound{g}\AgdaSpace{}%
\AgdaBound{x}\AgdaSpace{}%
\AgdaOperator{\AgdaDatatype{≡}}\AgdaSpace{}%
\AgdaBound{g}\AgdaSpace{}%
\AgdaBound{y}\<%
\\
%
\\[\AgdaEmptyExtraSkip]%
%
\>[1]\AgdaFunction{kernel}\AgdaSpace{}%
\AgdaSymbol{:}\AgdaSpace{}%
\AgdaSymbol{(}\AgdaBound{A}\AgdaSpace{}%
\AgdaSymbol{→}\AgdaSpace{}%
\AgdaBound{B}\AgdaSymbol{)}\AgdaSpace{}%
\AgdaSymbol{→}\AgdaSpace{}%
\AgdaFunction{Pred}\AgdaSpace{}%
\AgdaSymbol{(}\AgdaBound{A}\AgdaSpace{}%
\AgdaOperator{\AgdaFunction{×}}\AgdaSpace{}%
\AgdaBound{A}\AgdaSymbol{)}\AgdaSpace{}%
\AgdaBound{𝓦}\<%
\\
%
\>[1]\AgdaFunction{kernel}\AgdaSpace{}%
\AgdaBound{g}\AgdaSpace{}%
\AgdaSymbol{(}\AgdaBound{x}\AgdaSpace{}%
\AgdaOperator{\AgdaInductiveConstructor{,}}\AgdaSpace{}%
\AgdaBound{y}\AgdaSymbol{)}\AgdaSpace{}%
\AgdaSymbol{=}\AgdaSpace{}%
\AgdaBound{g}\AgdaSpace{}%
\AgdaBound{x}\AgdaSpace{}%
\AgdaOperator{\AgdaDatatype{≡}}\AgdaSpace{}%
\AgdaBound{g}\AgdaSpace{}%
\AgdaBound{y}\<%
% \\
% %
% \\[\AgdaEmptyExtraSkip]%
% %
% \>[1]\AgdaFunction{ker-sigma}\AgdaSpace{}%
% \AgdaSymbol{:}\AgdaSpace{}%
% \AgdaSymbol{(}\AgdaBound{A}\AgdaSpace{}%
% \AgdaSymbol{→}\AgdaSpace{}%
% \AgdaBound{B}\AgdaSymbol{)}\AgdaSpace{}%
% \AgdaSymbol{→}\AgdaSpace{}%
% \AgdaBound{𝓤}\AgdaSpace{}%
% \AgdaOperator{\AgdaPrimitive{⊔}}\AgdaSpace{}%
% \AgdaBound{𝓦}\AgdaSpace{}%
% \AgdaOperator{\AgdaFunction{̇}}\<%
% \\
% %
% \>[1]\AgdaFunction{ker-sigma}\AgdaSpace{}%
% \AgdaBound{g}\AgdaSpace{}%
% \AgdaSymbol{=}\AgdaSpace{}%
% \AgdaFunction{Σ}\AgdaSpace{}%
% \AgdaBound{x}\AgdaSpace{}%
% \AgdaFunction{꞉}\AgdaSpace{}%
% \AgdaBound{A}\AgdaSpace{}%
% \AgdaFunction{,}\AgdaSpace{}%
% \AgdaFunction{Σ}\AgdaSpace{}%
% \AgdaBound{y}\AgdaSpace{}%
% \AgdaFunction{꞉}\AgdaSpace{}%
% \AgdaBound{A}\AgdaSpace{}%
% \AgdaFunction{,}\AgdaSpace{}%
% \AgdaBound{g}\AgdaSpace{}%
% \AgdaBound{x}\AgdaSpace{}%
% \AgdaOperator{\AgdaDatatype{≡}}\AgdaSpace{}%
% \AgdaBound{g}\AgdaSpace{}%
% \AgdaBound{y}\<%
% \\
% %
% \\[\AgdaEmptyExtraSkip]%
% %
% \>[1]\AgdaFunction{ker-sigma'}\AgdaSpace{}%
% \AgdaSymbol{:}\AgdaSpace{}%
% \AgdaSymbol{(}\AgdaBound{A}\AgdaSpace{}%
% \AgdaSymbol{→}\AgdaSpace{}%
% \AgdaBound{B}\AgdaSymbol{)}\AgdaSpace{}%
% \AgdaSymbol{→}\AgdaSpace{}%
% \AgdaBound{𝓤}\AgdaSpace{}%
% \AgdaOperator{\AgdaPrimitive{⊔}}\AgdaSpace{}%
% \AgdaBound{𝓦}\AgdaSpace{}%
% \AgdaOperator{\AgdaFunction{̇}}\<%
% \\
% %
% \>[1]\AgdaFunction{ker-sigma'}\AgdaSpace{}%
% \AgdaBound{g}\AgdaSpace{}%
% \AgdaSymbol{=}\AgdaSpace{}%
% \AgdaFunction{Σ}\AgdaSpace{}%
% \AgdaBound{(x}\AgdaSpace{}%
% \AgdaBound{,}\AgdaSpace{}%
% \AgdaBound{y)}\AgdaSpace{}%
% \AgdaFunction{꞉}\AgdaSpace{}%
% \AgdaSymbol{(}\AgdaBound{A}\AgdaSpace{}%
% \AgdaOperator{\AgdaFunction{×}}\AgdaSpace{}%
% \AgdaBound{A}\AgdaSymbol{)}\AgdaSpace{}%
% \AgdaFunction{,}\AgdaSpace{}%
% \AgdaBound{g}\AgdaSpace{}%
% \AgdaBound{x}\AgdaSpace{}%
% \AgdaOperator{\AgdaDatatype{≡}}\AgdaSpace{}%
% \AgdaBound{g}\AgdaSpace{}%
% \AgdaBound{y}\<%
\end{code}
\ccpad
Similarly, the \defn{identity relation} (which is equivalent to the kernel of an injective function) can be represented by a number of different types. Here we only show the representation that we use later to construct the zero congruence. The notation we use here is close to that of conventional algebra notation, where $0_A$ is used to denote the identity relation $\{(x, y) \in A \times A : x = y\}$.\cref{uhints}
\ccpad
\begin{code}%
\>[1]\AgdaFunction{𝟎}\AgdaSpace{}%
\AgdaSymbol{:}\AgdaSpace{}%
\AgdaFunction{Rel}\AgdaSpace{}%
\AgdaBound{A}\AgdaSpace{}%
\AgdaBound{𝓤}\<%
\\
%
\>[1]\AgdaFunction{𝟎}\AgdaSpace{}%
\AgdaBound{x}\AgdaSpace{}%
\AgdaBound{y}\AgdaSpace{}%
\AgdaSymbol{=}\AgdaSpace{}%
\AgdaBound{x}\AgdaSpace{}%
\AgdaOperator{\AgdaDatatype{≡}}\AgdaSpace{}%
\AgdaBound{y}\<%
% \\
% %
% \\[\AgdaEmptyExtraSkip]%
% %
% \>[1]\AgdaFunction{𝟎-pred}\AgdaSpace{}%
% \AgdaSymbol{:}\AgdaSpace{}%
% \AgdaFunction{Pred}\AgdaSpace{}%
% \AgdaSymbol{(}\AgdaBound{A}\AgdaSpace{}%
% \AgdaOperator{\AgdaFunction{×}}\AgdaSpace{}%
% \AgdaBound{A}\AgdaSymbol{)}\AgdaSpace{}%
% \AgdaBound{𝓤}\<%
% \\
% %
% \>[1]\AgdaFunction{𝟎-pred}\AgdaSpace{}%
% \AgdaSymbol{(}\AgdaBound{x}\AgdaSpace{}%
% \AgdaOperator{\AgdaInductiveConstructor{,}}\AgdaSpace{}%
% \AgdaBound{y}\AgdaSymbol{)}\AgdaSpace{}%
% \AgdaSymbol{=}\AgdaSpace{}%
% \AgdaBound{x}\AgdaSpace{}%
% \AgdaOperator{\AgdaDatatype{≡}}\AgdaSpace{}%
% \AgdaBound{y}\<%
% \\
% %
% \\[\AgdaEmptyExtraSkip]%
% %
% \>[1]\AgdaFunction{𝟎-sigma}\AgdaSpace{}%
% \AgdaSymbol{:}\AgdaSpace{}%
% \AgdaBound{𝓤}\AgdaSpace{}%
% \AgdaOperator{\AgdaFunction{̇}}\<%
% \\
% %
% \>[1]\AgdaFunction{𝟎-sigma}\AgdaSpace{}%
% \AgdaSymbol{=}\AgdaSpace{}%
% \AgdaFunction{Σ}\AgdaSpace{}%
% \AgdaBound{x}\AgdaSpace{}%
% \AgdaFunction{꞉}\AgdaSpace{}%
% \AgdaBound{A}\AgdaSpace{}%
% \AgdaFunction{,}\AgdaSpace{}%
% \AgdaFunction{Σ}\AgdaSpace{}%
% \AgdaBound{y}\AgdaSpace{}%
% \AgdaFunction{꞉}\AgdaSpace{}%
% \AgdaBound{A}\AgdaSpace{}%
% \AgdaFunction{,}\AgdaSpace{}%
% \AgdaBound{x}\AgdaSpace{}%
% \AgdaOperator{\AgdaDatatype{≡}}\AgdaSpace{}%
% \AgdaBound{y}\<%
% \\
% %
% \\[\AgdaEmptyExtraSkip]%
% %
% \>[1]\AgdaFunction{𝟎-sigma'}\AgdaSpace{}%
% \AgdaSymbol{:}\AgdaSpace{}%
% \AgdaBound{𝓤}\AgdaSpace{}%
% \AgdaOperator{\AgdaFunction{̇}}\<%
% \\
% %
% \>[1]\AgdaFunction{𝟎-sigma'}\AgdaSpace{}%
% \AgdaSymbol{=}\AgdaSpace{}%
% \AgdaFunction{Σ}\AgdaSpace{}%
% \AgdaBound{(x}\AgdaSpace{}%
% \AgdaBound{,}\AgdaSpace{}%
% \AgdaBound{y)}\AgdaSpace{}%
% \AgdaFunction{꞉}\AgdaSpace{}%
% \AgdaSymbol{(}\AgdaBound{A}\AgdaSpace{}%
% \AgdaOperator{\AgdaFunction{×}}\AgdaSpace{}%
% \AgdaBound{A}\AgdaSymbol{)}\AgdaSpace{}%
% \AgdaFunction{,}\AgdaSpace{}%
% \AgdaBound{x}\AgdaSpace{}%
% \AgdaOperator{\AgdaDatatype{≡}}\AgdaSpace{}%
% \AgdaBound{y}\<%
\end{code}
% \ccpad
% Finally, the \defn{total relation} over \ab A, which in set theory is the full Cartesian product \ab A~\af ×~\ab A, can be represented using the one-element type from the \am{Unit-Type} module of \typtop as follows.\cref{uhints}$^, $\footnote{%
% The one-element type is an inductive type with a single constructor, denoted \AgdaInductiveConstructor{⋆}, as follows: \AgdaKeyword{data}\AgdaSpace{}%
% \AgdaDatatype{𝟙}\AgdaSpace{}%
% \AgdaSymbol{\{}\AgdaBound{𝓤}\AgdaSymbol{\}}\AgdaSpace{}%
% \AgdaSymbol{:}\AgdaSpace{}%
% \AgdaBound{𝓤}\AgdaSpace{}%
% \AgdaOperator{\AgdaFunction{̇}}\AgdaSpace{}%
% \AgdaKeyword{where}\AgdaSpace{}\AgdaInductiveConstructor{⋆}\AgdaSpace{}%
% \AgdaSymbol{:}\AgdaSpace{}%
% \AgdaDatatype{𝟙}.}
% \ccpad
% \begin{code}%
% \>[1]\AgdaFunction{𝟏}\AgdaSpace{}%
% \AgdaSymbol{:}\AgdaSpace{}%
% \AgdaFunction{Rel}\AgdaSpace{}%
% \AgdaBound{A}\AgdaSpace{}%
% \AgdaPrimitive{𝓤₀}\<%
% \\
% %
% \>[1]\AgdaFunction{𝟏}\AgdaSpace{}%
% \AgdaBound{a}\AgdaSpace{}%
% \AgdaBound{b}\AgdaSpace{}%
% \AgdaSymbol{=}\AgdaSpace{}%
% \AgdaFunction{𝟙}\<%
% \end{code}


\paragraph*{The implication relation\protect\footnotemark}
\footnotetext{The definitions here are from the \agdastdlib, translated into \typtop/\ualib notation.}

The following types represent \defn{implication} for binary relations.
\ccpad
\begin{code}%
\>[0]\AgdaOperator{\AgdaFunction{\AgdaUnderscore{}on\AgdaUnderscore{}}}\AgdaSpace{}%
\AgdaSymbol{:}\AgdaSpace{}%
\AgdaSymbol{(}\AgdaBound{B}\AgdaSpace{}%
\AgdaSymbol{→}\AgdaSpace{}%
\AgdaBound{B}\AgdaSpace{}%
\AgdaSymbol{→}\AgdaSpace{}%
\AgdaBound{C}\AgdaSymbol{)}\AgdaSpace{}%
\AgdaSymbol{→}\AgdaSpace{}%
\AgdaSymbol{(}\AgdaBound{A}\AgdaSpace{}%
\AgdaSymbol{→}\AgdaSpace{}%
\AgdaBound{B}\AgdaSymbol{)}\AgdaSpace{}%
\AgdaSymbol{→}\AgdaSpace{}%
\AgdaSymbol{(}\AgdaBound{A}\AgdaSpace{}%
\AgdaSymbol{→}\AgdaSpace{}%
\AgdaBound{A}\AgdaSpace{}%
\AgdaSymbol{→}\AgdaSpace{}%
\AgdaBound{C}\AgdaSymbol{)}\<%
\\
\>[0]\AgdaBound{R}\AgdaSpace{}%
\AgdaOperator{\AgdaFunction{on}}\AgdaSpace{}%
\AgdaBound{g}\AgdaSpace{}%
\AgdaSymbol{=}\AgdaSpace{}%
\AgdaSymbol{λ}\AgdaSpace{}%
\AgdaBound{x}\AgdaSpace{}%
\AgdaBound{y}\AgdaSpace{}%
\AgdaSymbol{→}\AgdaSpace{}%
\AgdaBound{R}\AgdaSpace{}%
\AgdaSymbol{(}\AgdaBound{g}\AgdaSpace{}%
\AgdaBound{x}\AgdaSymbol{)}\AgdaSpace{}%
\AgdaSymbol{(}\AgdaBound{g}\AgdaSpace{}%
\AgdaBound{y}\AgdaSymbol{)}\<%
\end{code}
\scpad
\begin{code}
\>[1]\AgdaOperator{\AgdaFunction{\AgdaUnderscore{}⇒\AgdaUnderscore{}}}\AgdaSpace{}%
\AgdaSymbol{:}\AgdaSpace{}%
\AgdaFunction{REL}\AgdaSpace{}%
\AgdaBound{A}\AgdaSpace{}%
\AgdaBound{B}\AgdaSpace{}%
\AgdaBound{𝓧}\AgdaSpace{}%
\AgdaSymbol{→}\AgdaSpace{}%
\AgdaFunction{REL}\AgdaSpace{}%
\AgdaBound{A}\AgdaSpace{}%
\AgdaBound{B}\AgdaSpace{}%
\AgdaBound{𝓨}\AgdaSpace{}%
\AgdaSymbol{→}\AgdaSpace{}%
\AgdaBound{𝓤}\AgdaSpace{}%
\AgdaOperator{\AgdaPrimitive{⊔}}\AgdaSpace{}%
\AgdaBound{𝓦}\AgdaSpace{}%
\AgdaOperator{\AgdaPrimitive{⊔}}\AgdaSpace{}%
\AgdaBound{𝓧}\AgdaSpace{}%
\AgdaOperator{\AgdaPrimitive{⊔}}\AgdaSpace{}%
\AgdaBound{𝓨}\AgdaSpace{}%
\AgdaOperator{\AgdaFunction{̇}}\<%
\\
%
\>[1]\AgdaBound{P}\AgdaSpace{}%
\AgdaOperator{\AgdaFunction{⇒}}\AgdaSpace{}%
\AgdaBound{Q}\AgdaSpace{}%
\AgdaSymbol{=}\AgdaSpace{}%
\AgdaSymbol{∀}\AgdaSpace{}%
\AgdaSymbol{\{}\AgdaBound{i}\AgdaSpace{}%
\AgdaBound{j}\AgdaSymbol{\}}\AgdaSpace{}%
\AgdaSymbol{→}\AgdaSpace{}%
\AgdaBound{P}\AgdaSpace{}%
\AgdaBound{i}\AgdaSpace{}%
\AgdaBound{j}\AgdaSpace{}%
\AgdaSymbol{→}\AgdaSpace{}%
\AgdaBound{Q}\AgdaSpace{}%
\AgdaBound{i}\AgdaSpace{}%
\AgdaBound{j}\<%
\end{code}
\ccpad
These combine to give a nice, general implication operation.
\ccpad
\begin{code}%
\>[1]\AgdaOperator{\AgdaFunction{\AgdaUnderscore{}=[\AgdaUnderscore{}]⇒\AgdaUnderscore{}}}\AgdaSpace{}%
\AgdaSymbol{:}\AgdaSpace{}%
\AgdaFunction{Rel}\AgdaSpace{}%
\AgdaBound{A}\AgdaSpace{}%
\AgdaBound{𝓧}\AgdaSpace{}%
\AgdaSymbol{→}\AgdaSpace{}%
\AgdaSymbol{(}\AgdaBound{A}\AgdaSpace{}%
\AgdaSymbol{→}\AgdaSpace{}%
\AgdaBound{B}\AgdaSymbol{)}\AgdaSpace{}%
\AgdaSymbol{→}\AgdaSpace{}%
\AgdaFunction{Rel}\AgdaSpace{}%
\AgdaBound{B}\AgdaSpace{}%
\AgdaBound{𝓨}\AgdaSpace{}%
\AgdaSymbol{→}\AgdaSpace{}%
\AgdaBound{𝓤}\AgdaSpace{}%
\AgdaOperator{\AgdaPrimitive{⊔}}\AgdaSpace{}%
\AgdaBound{𝓧}\AgdaSpace{}%
\AgdaOperator{\AgdaPrimitive{⊔}}\AgdaSpace{}%
\AgdaBound{𝓨}\AgdaSpace{}%
\AgdaOperator{\AgdaFunction{̇}}\<%
\\
%
\>[1]\AgdaBound{P}\AgdaSpace{}%
\AgdaOperator{\AgdaFunction{=[}}\AgdaSpace{}%
\AgdaBound{g}\AgdaSpace{}%
\AgdaOperator{\AgdaFunction{]⇒}}\AgdaSpace{}%
\AgdaBound{Q}\AgdaSpace{}%
\AgdaSymbol{=}\AgdaSpace{}%
\AgdaBound{P}\AgdaSpace{}%
\AgdaOperator{\AgdaFunction{⇒}}\AgdaSpace{}%
\AgdaSymbol{(}\AgdaBound{Q}\AgdaSpace{}%
\AgdaOperator{\AgdaFunction{on}}\AgdaSpace{}%
\AgdaBound{g}\AgdaSymbol{)}\<%
\end{code}

\paragraph*{Operation type and compatibility}%\label{compatibility-of-functions-and-binary-relations}

\textbf{Notation}. For consistency and readability, throughout the \ualib we reserve two universe variables for special purposes.  The first of these is \ab 𝓞 which shall be reserved for types that represent \defn{operation symbols} (see \ualibhtml{Algebras.Signatures}). The second is \ab 𝓥 which we reserve for types representing \defn{arities} of relations or operations.

Below we will define types that are useful for asserting and proving facts about \defn{compatibility} of operations and relations, but first we need a general type with which to represent operations.  Here is the definition, which we justify below.
\ccpad
\begin{code}
% \>[0]\AgdaComment{--The type of operations}\<%
% \\
\>[0]\AgdaFunction{Op}\AgdaSpace{}%
\AgdaSymbol{:}\AgdaSpace{}%
\AgdaGeneralizable{𝓥}\AgdaSpace{}%
\AgdaOperator{\AgdaFunction{̇}}\AgdaSpace{}%
\AgdaSymbol{→}\AgdaSpace{}%
\AgdaGeneralizable{𝓤}\AgdaSpace{}%
\AgdaOperator{\AgdaFunction{̇}}\AgdaSpace{}%
\AgdaSymbol{→}\AgdaSpace{}%
\AgdaGeneralizable{𝓤}\AgdaSpace{}%
\AgdaOperator{\AgdaPrimitive{⊔}}\AgdaSpace{}%
\AgdaGeneralizable{𝓥}\AgdaSpace{}%
\AgdaOperator{\AgdaFunction{̇}}\<%
\\
\>[0]\AgdaFunction{Op}\AgdaSpace{}%
\AgdaBound{I}\AgdaSpace{}%
\AgdaBound{A}\AgdaSpace{}%
\AgdaSymbol{=}\AgdaSpace{}%
\AgdaSymbol{(}\AgdaBound{I}\AgdaSpace{}%
\AgdaSymbol{→}\AgdaSpace{}%
\AgdaBound{A}\AgdaSymbol{)}\AgdaSpace{}%
\AgdaSymbol{→}\AgdaSpace{}%
\AgdaBound{A}\<%
\end{code}
\ccpad
The definition of \af{Op} codifies the arity of an operation as an arbitrary type \ab I~\as :~\ab 𝓥\af ̇, which gives us a very general way to represent an operation as a function type with domain \ab I~\as →~\ab A (the type of ``\ab I-tuples'') and codomain \ab A. For example, the \ab I-\defn{ary projection operations} on \ab A are represented as inhabitants of the type \af{Op}~\ab I~\ab A as follows.
\ccpad
\begin{code}
\>[0]\AgdaFunction{π}\AgdaSpace{}%
\AgdaSymbol{:}\AgdaSpace{}%
\AgdaSymbol{\{}\AgdaBound{I}\AgdaSpace{}%
\AgdaSymbol{:}\AgdaSpace{}%
\AgdaGeneralizable{𝓥}\AgdaSpace{}%
\AgdaOperator{\AgdaFunction{̇}}\AgdaSpace{}%
\AgdaSymbol{\}}\AgdaSpace{}%
\AgdaSymbol{\{}\AgdaBound{A}\AgdaSpace{}%
\AgdaSymbol{:}\AgdaSpace{}%
\AgdaGeneralizable{𝓤}\AgdaSpace{}%
\AgdaOperator{\AgdaFunction{̇}}\AgdaSpace{}%
\AgdaSymbol{\}}\AgdaSpace{}%
\AgdaSymbol{→}\AgdaSpace{}%
\AgdaBound{I}\AgdaSpace{}%
\AgdaSymbol{→}\AgdaSpace{}%
\AgdaFunction{Op}\AgdaSpace{}%
\AgdaBound{I}\AgdaSpace{}%
\AgdaBound{A}\<%
\\
\>[0]\AgdaFunction{π}\AgdaSpace{}%
\AgdaBound{i}\AgdaSpace{}%
\AgdaBound{x}\AgdaSpace{}%
\AgdaSymbol{=}\AgdaSpace{}%
\AgdaBound{x}\AgdaSpace{}%
\AgdaBound{i}\<%
\end{code}
\scpad
% Before discussing general and dependent relations, we pause to define some types that are useful for asserting and proving facts about \emph{compatibility} of functions with binary relations. 

Let us review the informal definition of compatibility. Suppose \ab{A} and \ab{I} are types and fix \ab{𝑓}~\as :~\AgdaFunction{Op}~\AgdaBound{I}~\AgdaBound{A} and \ab{R}~\as :~\af{Rel}~\ab A~\ab 𝓦 (an \ab{I}-ary operation and a binary relation on \ab{A}, respectively). We say that \ab{𝑓} and \ab{R} are \emph{compatible} and we write\footnote{The symbol \af{\textbar{}:} denoting compatibility comes from Cliff Bergman's universal algebra textbook~\cite{Bergman:2012}.} \ab{𝑓}~\af{\textbar{}:}~\ab R just in case \as{∀}~\ab u~\ab v~\as :~\ab I~\as →~\ab A,\\[-8pt]

\ad{Π}~\ab i~\as ꞉ \ab I \af , \ab R (\ab u \ab i) (\ab v \ab i) ~ \as{→} ~ \ab{R} (\ab f \ab u) (\ab f \ab v).\\[4pt]
To implement this in Agda, we first define a function \af{eval-rel} which ``lifts'' a binary relation to the corresponding \ab{I}-ary relation, and we use this to define the function \AgdaFunction{|:} representing \defn{compatibility of an \ab{I}-ary operation with a binary relation}.
\ccpad
\begin{code}%
\>[0]\AgdaFunction{eval-rel}\AgdaSpace{}%
\AgdaSymbol{:}\AgdaSpace{}%
\AgdaSymbol{\{}\AgdaBound{A}\AgdaSpace{}%
\AgdaSymbol{:}\AgdaSpace{}%
\AgdaGeneralizable{𝓤}\AgdaSpace{}%
\AgdaOperator{\AgdaFunction{̇}}\AgdaSymbol{\}\{}\AgdaBound{I}\AgdaSpace{}%
\AgdaSymbol{:}\AgdaSpace{}%
\AgdaGeneralizable{𝓥}\AgdaSpace{}%
\AgdaOperator{\AgdaFunction{̇}}\AgdaSymbol{\}}\AgdaSpace{}%
\AgdaSymbol{→}\AgdaSpace{}%
\AgdaFunction{Rel}\AgdaSpace{}%
\AgdaBound{A}\AgdaSpace{}%
\AgdaGeneralizable{𝓦}\AgdaSpace{}%
\AgdaSymbol{→}\AgdaSpace{}%
\AgdaFunction{Rel}\AgdaSpace{}%
\AgdaSymbol{(}\AgdaBound{I}\AgdaSpace{}%
\AgdaSymbol{→}\AgdaSpace{}%
\AgdaBound{A}\AgdaSymbol{)(}\AgdaGeneralizable{𝓥}\AgdaSpace{}%
\AgdaOperator{\AgdaPrimitive{⊔}}\AgdaSpace{}%
\AgdaGeneralizable{𝓦}\AgdaSymbol{)}\<%
\\
\>[0]\AgdaFunction{eval-rel}\AgdaSpace{}%
\AgdaBound{R}\AgdaSpace{}%
\AgdaBound{u}\AgdaSpace{}%
\AgdaBound{v}\AgdaSpace{}%
\AgdaSymbol{=}\AgdaSpace{}%
\AgdaFunction{Π}\AgdaSpace{}%
\AgdaBound{i}\AgdaSpace{}%
\AgdaFunction{꞉}\AgdaSpace{}%
\AgdaSymbol{\AgdaUnderscore{}}\AgdaSpace{}%
\AgdaFunction{,}\AgdaSpace{}%
\AgdaBound{R}\AgdaSpace{}%
\AgdaSymbol{(}\AgdaBound{u}\AgdaSpace{}%
\AgdaBound{i}\AgdaSymbol{)}\AgdaSpace{}%
\AgdaSymbol{(}\AgdaBound{v}\AgdaSpace{}%
\AgdaBound{i}\AgdaSymbol{)}\<%
\\
%
\\[\AgdaEmptyExtraSkip]%
\>[0]\AgdaOperator{\AgdaFunction{\AgdaUnderscore{}|:\AgdaUnderscore{}}}\AgdaSpace{}%
\AgdaSymbol{:}\AgdaSpace{}%
\AgdaSymbol{\{}\AgdaBound{A}\AgdaSpace{}%
\AgdaSymbol{:}\AgdaSpace{}%
\AgdaGeneralizable{𝓤}\AgdaSpace{}%
\AgdaOperator{\AgdaFunction{̇}}\AgdaSymbol{\}\{}\AgdaBound{I}\AgdaSpace{}%
\AgdaSymbol{:}\AgdaSpace{}%
\AgdaGeneralizable{𝓥}\AgdaSpace{}%
\AgdaOperator{\AgdaFunction{̇}}\AgdaSymbol{\}}\AgdaSpace{}%
\AgdaSymbol{→}\AgdaSpace{}%
\AgdaFunction{Op}\AgdaSpace{}%
\AgdaBound{I}\AgdaSpace{}%
\AgdaBound{A}\AgdaSpace{}%
\AgdaSymbol{→}\AgdaSpace{}%
\AgdaFunction{Rel}\AgdaSpace{}%
\AgdaBound{A}\AgdaSpace{}%
\AgdaGeneralizable{𝓦}\AgdaSpace{}%
\AgdaSymbol{→}\AgdaSpace{}%
\AgdaGeneralizable{𝓤}\AgdaSpace{}%
\AgdaOperator{\AgdaPrimitive{⊔}}\AgdaSpace{}%
\AgdaGeneralizable{𝓥}\AgdaSpace{}%
\AgdaOperator{\AgdaPrimitive{⊔}}\AgdaSpace{}%
\AgdaGeneralizable{𝓦}\AgdaSpace{}%
\AgdaOperator{\AgdaFunction{̇}}\<%
\\
\>[0]\AgdaBound{f}\AgdaSpace{}%
\AgdaOperator{\AgdaFunction{|:}}\AgdaSpace{}%
\AgdaBound{R}%
\>[8]\AgdaSymbol{=}\AgdaSpace{}%
\AgdaSymbol{(}\AgdaFunction{eval-rel}\AgdaSpace{}%
\AgdaBound{R}\AgdaSymbol{)}\AgdaSpace{}%
\AgdaOperator{\AgdaFunction{=[}}\AgdaSpace{}%
\AgdaBound{f}\AgdaSpace{}%
\AgdaOperator{\AgdaFunction{]⇒}}\AgdaSpace{}%
\AgdaBound{R}\<%
\end{code}
\ccpad
% \footnote{Initially we called the first function
% \af{lift-rel} because it ``lifts'' a binary relation on \ab A to a binary relation on tuples of type \ab I~\as →~\ab A. However, we renamed it \af{eval-rel} to avoid confusion with the universe level \AgdaRecord{Lift} type defined in the \ualibhtml{Overture.Lifts} module, or with \af{free-lift} (\ualibhtml{Terms.Basic}) which lifts a map defined on generators to a map on the thing being generated. Also, o
% (Here we silently added the type \ab I~\as :~\ab 𝓥\af ̇, representing \emph{relation arity}, to the context; we will have more to say about relation arities in the next section (\S\ref{sec:continuous-relations}).)

%%%%%%%%%%%%%%%%%%%%%%%%%%%%%%%%%%%%%%%%%%%%%%%%%%%%%%%%%%%%%%%%%%%%%%%%%%%%%%%%%%%%%%%%%%%%%%%%%%
\subsection{Continuous relations*\protect\footnotemark}\label{sec:continuous-relations}\footnotetext{\label{starred}%
Sections marked with an asterisk include new types that are more abstract and general than those presented elsewhere, but they are used only very rarely in other parts of the \ualib. Therefore, the sections marked * may be safely skimmed or skipped when first encountering them.}
% : arbitrary-sorted relations of arbitrary arity
\firstsentencefull{Relations}{Continuous}
% \context{%
% \{\ab 𝓤 \ab 𝓥 \ab 𝓦 \as : \AgdaPostulate{Universe}\}%
% \{\ab I \ab J \as : \ab 𝓥\AgdaOperator{\AgdaFunction{̇}}\}%
% \{\ab A \as : \ab 𝓤\AgdaOperator{\AgdaFunction{̇}}\}%
% }
% -*- TeX-master: "ualib-part1.tex" -*-
%%% Local Variables: 
%%% mode: latex
%%% TeX-engine: 'xetex
%%% End:
\subsubsection{Motivation}\label{sec:motivation-1}
In set theory, an \ab n-ary relation on a set \ab{A} is a subset of the \ab n-fold product \ab{A} \ad × \as ⋯ \ad × \ab A. We could try to model these as predicates over a product type representing \ab{A} \ad × \as ⋯ \ad × \ab A, or as relations of type \ab{A} \as → \ab A \as → \as ⋯ \as → \ab A \as → \ab 𝓦\af ̇ (for some universe \ab 𝓦). To implement such types requires knowing the arity in advance, and then form an \ab n-fold product or \ab n-fold arrow. It turns out to be both easier and more general if we define an arity type \ab{I}~\as :~\ab 𝓥\af ̇ , and define the type representing \ab{I}-ary relations on \ab{A} as the function type (\ab{I} \as → \ab A) \as → \ab 𝓦\af ̇. Then, if we happen to be interested in \ab n-ary relations for some natural number \ab{n}, we could take \ab{I} to be the \ab n-element type \ad{Fin}~\ab n,~\cite{agda-fin}.

Below we will define \af{ContRel} to be the type (\ab{I}~\as →~\ab A)~\as →~\ab 𝓦\af ̇ and we will call this the type of \defn{continuous relations}. This generalizes the discrete relations we defined in \ualibhtml{Relations.Discrete} (unary, binary, etc.) since continuous relations can be of arbitrary arity.  They are not completely general, however, since they are defined over a single type. Said another way, these are ``single-sorted'' relations. We will remove this limitation when we define the type of \emph{dependent continuous relations}. Just as \af{Rel} \ab A \ab 𝓦 was the single-sorted special case of the multisorted \af{REL} \ab A \ab B \ab 𝓦 type, so too will \af{ContRel} \ab I \ab A \ab 𝓦 be the single-sorted version of dependent continuous relations. The latter will represent relations that not only have arbitrary arities, but also are defined over arbitrary families of types.

To be more concrete, given an arbitrary family \ab{A}~\as :~\ab I~\as →~\ab 𝓤\af ̇ of types, we may have a relation from \ab{A}~\ab i to \ab{A}~\ab j to \ab{A}~\ab k to $\ldots$, where the collection represented by the indexing type \ab I might not even be enumerable.\footnote{\label{uncountable}Because the collection represented by the indexing type \ab I might not even be enumerable, technically speaking, instead of ``\ab{A} \ab i to \ab{A} \ab j to \ab{A} \ab k to $\ldots$,'' we should have written something like ``$\mathsf{\af{TO}}$ (\abt{i}{I}) , \ab{A} \ab i.''}
We will refer to such relations as \defn{dependent continuous relations} (or \defn{dependent relations}) because the definition of a type that represents them requires dependent types. The \af{DepRel} type that we define below manifests this completely general notion of relation.

\subsubsection{Continuous relation types}\label{continuous-relation-types}

We now define the type \af{ContRel} which represents predicates of arbitrary arity over a single type \ab{A}. We call this the type of \defn{continuous relations}.\footnote{%
For consistency and readability, throughout the \ualib we reserve two universe variables for special purposes. The first of these is 𝓞 which shall be reserved for types that represent \emph{operation symbols} (see \ualibhtml{Algebras.Signatures}). The second is 𝓥 which we reserve for types representing \emph{arities} of relations or operations.}
\ccpad
\begin{code}%
\>[0]\AgdaFunction{ContRel}\AgdaSpace{}%
\AgdaSymbol{:}\AgdaSpace{}%
\AgdaGeneralizable{𝓥}\AgdaSpace{}%
\AgdaOperator{\AgdaFunction{̇}}\AgdaSpace{}%
\AgdaSymbol{→}\AgdaSpace{}%
\AgdaGeneralizable{𝓤}\AgdaSpace{}%
\AgdaOperator{\AgdaFunction{̇}}\AgdaSpace{}%
\AgdaSymbol{→}\AgdaSpace{}%
\AgdaSymbol{(}\AgdaBound{𝓦}\AgdaSpace{}%
\AgdaSymbol{:}\AgdaSpace{}%
\AgdaPostulate{Universe}\AgdaSymbol{)}\AgdaSpace{}%
\AgdaSymbol{→}\AgdaSpace{}%
\AgdaGeneralizable{𝓥}\AgdaSpace{}%
\AgdaOperator{\AgdaPrimitive{⊔}}\AgdaSpace{}%
\AgdaGeneralizable{𝓤}\AgdaSpace{}%
\AgdaOperator{\AgdaPrimitive{⊔}}\AgdaSpace{}%
\AgdaBound{𝓦}\AgdaSpace{}%
\AgdaOperator{\AgdaPrimitive{⁺}}\AgdaSpace{}%
\AgdaOperator{\AgdaFunction{̇}}\<%
\\
\>[0]\AgdaFunction{ContRel}\AgdaSpace{}%
\AgdaBound{I}\AgdaSpace{}%
\AgdaBound{A}\AgdaSpace{}%
\AgdaBound{𝓦}\AgdaSpace{}%
\AgdaSymbol{=}\AgdaSpace{}%
\AgdaSymbol{(}\AgdaBound{I}\AgdaSpace{}%
\AgdaSymbol{→}\AgdaSpace{}%
\AgdaBound{A}\AgdaSymbol{)}\AgdaSpace{}%
\AgdaSymbol{→}\AgdaSpace{}%
\AgdaBound{𝓦}\AgdaSpace{}%
\AgdaOperator{\AgdaFunction{̇}}\<%
\end{code}
\ccpad
% \subsubsection{Compatibility with continuous relations}\label{compatibility-with-continuous-relations}
Next we present types that are useful for asserting and proving facts about the compatibility of functions with continuous relations. The first is an \defn{evaluation} function which ``lifts'' an \ab I-ary relation to an (\ab I~\as →~\ab J)-ary relation. The lifted relation will relate an \ab I-tuple of \ab J-tuples when the ``\ab I-slices'' (or ``rows'') of the \ab J-tuples belong to the original relation. The second definition is a type that represents compatibility of an operation with a continuous relation. (We'll dissect these definitions below.)
\ccpad
\begin{code}%
% \>[0]\AgdaKeyword{module}\AgdaSpace{}%
% \AgdaModule{\AgdaUnderscore{}}\AgdaSpace{}%
% \AgdaSymbol{\{}\AgdaBound{I}\AgdaSpace{}%
% \AgdaBound{J}\AgdaSpace{}%
% \AgdaSymbol{:}\AgdaSpace{}%
% \AgdaGeneralizable{𝓥}\AgdaSpace{}%
% \AgdaOperator{\AgdaFunction{̇}}\AgdaSymbol{\}}\AgdaSpace{}%
% \AgdaSymbol{\{}\AgdaBound{A}\AgdaSpace{}%
% \AgdaSymbol{:}\AgdaSpace{}%
% \AgdaGeneralizable{𝓤}\AgdaSpace{}%
% \AgdaOperator{\AgdaFunction{̇}}\AgdaSymbol{\}}\AgdaSpace{}%
% \AgdaKeyword{where}\<%
% \\
% %
% \\[\AgdaEmptyExtraSkip]%
\>[0][@{}l@{\AgdaIndent{0}}]%
\>[1]\AgdaFunction{eval-cont-rel}\AgdaSpace{}%
\AgdaSymbol{:}\AgdaSpace{}%
\AgdaFunction{ContRel}\AgdaSpace{}%
\AgdaBound{I}\AgdaSpace{}%
\AgdaBound{A}\AgdaSpace{}%
\AgdaGeneralizable{𝓦}\AgdaSpace{}%
\AgdaSymbol{→}\AgdaSpace{}%
\AgdaSymbol{(}\AgdaBound{I}\AgdaSpace{}%
\AgdaSymbol{→}\AgdaSpace{}%
\AgdaBound{J}\AgdaSpace{}%
\AgdaSymbol{→}\AgdaSpace{}%
\AgdaBound{A}\AgdaSymbol{)}\AgdaSpace{}%
\AgdaSymbol{→}\AgdaSpace{}%
\AgdaBound{𝓥}\AgdaSpace{}%
\AgdaOperator{\AgdaPrimitive{⊔}}\AgdaSpace{}%
\AgdaGeneralizable{𝓦}\AgdaSpace{}%
\AgdaOperator{\AgdaFunction{̇}}\<%
\\
%
\>[1]\AgdaFunction{eval-cont-rel}\AgdaSpace{}%
\AgdaBound{R}\AgdaSpace{}%
\AgdaBound{𝒶}\AgdaSpace{}%
\AgdaSymbol{=}\AgdaSpace{}%
\AgdaFunction{Π}\AgdaSpace{}%
\AgdaBound{j}\AgdaSpace{}%
\AgdaFunction{꞉}\AgdaSpace{}%
\AgdaBound{J}\AgdaSpace{}%
\AgdaFunction{,}\AgdaSpace{}%
\AgdaBound{R}\AgdaSpace{}%
\AgdaSymbol{λ}\AgdaSpace{}%
\AgdaBound{i}\AgdaSpace{}%
\AgdaSymbol{→}\AgdaSpace{}%
\AgdaBound{𝒶}\AgdaSpace{}%
\AgdaBound{i}\AgdaSpace{}%
\AgdaBound{j}\<%
\\
%
\\[\AgdaEmptyExtraSkip]%
%
\>[1]\AgdaFunction{cont-compatible-fun}\AgdaSpace{}%
\AgdaSymbol{:}\AgdaSpace{}%
\AgdaSymbol{((}\AgdaBound{J}\AgdaSpace{}%
\AgdaSymbol{→}\AgdaSpace{}%
\AgdaBound{A}\AgdaSymbol{)}\AgdaSpace{}%
\AgdaSymbol{→}\AgdaSpace{}%
\AgdaBound{A}\AgdaSymbol{)}\AgdaSpace{}%
\AgdaSymbol{→}\AgdaSpace{}%
\AgdaFunction{ContRel}\AgdaSpace{}%
\AgdaBound{I}\AgdaSpace{}%
\AgdaBound{A}\AgdaSpace{}%
\AgdaGeneralizable{𝓦}\AgdaSpace{}%
\AgdaSymbol{→}\AgdaSpace{}%
\AgdaBound{𝓥}\AgdaSpace{}%
\AgdaOperator{\AgdaPrimitive{⊔}}\AgdaSpace{}%
\AgdaBound{𝓤}\AgdaSpace{}%
\AgdaOperator{\AgdaPrimitive{⊔}}\AgdaSpace{}%
\AgdaGeneralizable{𝓦}\AgdaSpace{}%
\AgdaOperator{\AgdaFunction{̇}}\<%
\\
%
\>[1]\AgdaFunction{cont-compatible-fun}\AgdaSpace{}%
\AgdaBound{𝑓}\AgdaSpace{}%
\AgdaBound{R}%
\>[26]\AgdaSymbol{=}\AgdaSpace{}%
\AgdaFunction{Π}\AgdaSpace{}%
\AgdaBound{𝒶}\AgdaSpace{}%
\AgdaFunction{꞉}\AgdaSpace{}%
\AgdaSymbol{(}\AgdaBound{I}\AgdaSpace{}%
\AgdaSymbol{→}\AgdaSpace{}%
\AgdaBound{J}\AgdaSpace{}%
\AgdaSymbol{→}\AgdaSpace{}%
\AgdaBound{A}\AgdaSymbol{)}\AgdaSpace{}%
\AgdaFunction{,}\AgdaSpace{}%
\AgdaSymbol{(}\AgdaFunction{eval-cont-rel}\AgdaSpace{}%
\AgdaBound{R}\AgdaSpace{}%
\AgdaBound{𝒶}\AgdaSpace{}%
\AgdaSymbol{→}\AgdaSpace{}%
\AgdaBound{R}\AgdaSpace{}%
\AgdaSymbol{λ}\AgdaSpace{}%
\AgdaBound{i}\AgdaSpace{}%
\AgdaSymbol{→}\AgdaSpace{}%
\AgdaSymbol{(}\AgdaBound{𝑓}\AgdaSpace{}%
\AgdaSymbol{(}\AgdaBound{𝒶}\AgdaSpace{}%
\AgdaBound{i}\AgdaSymbol{)))}\<%
\end{code}
\ccpad
% In the definition of \af{cont-compatible-fun}, we let Agda infer the type \ab{I}
% \as → (\ab J \as → \ab A) of \ab{𝒶}.
To readers who find the syntax of the last two definitions nauseating, we recommend focusing on the semantics. First, internalize the fact that \ab 𝒶~\as :~\ab I~\as →~\ab J~\as →~\ab A denotes an \ab{I}-tuple of \ab{J}-tuples of inhabitants of \ab{A}. Next, recall that a continuous relation \ab{R} represents a certain collection of \ab{I}-tuples. Specifically, if \ab x~\as :~\ab I~\as →~\ab A is an \ab{I}-tuple, then \ab{R}~\ab x denotes the assertion that ``\ab{x} belongs to \ab{R}'' or ``\ab{x} satisfies \ab{R}.''  For each continuous relation \ab{R}, the type \af{eval-cont-rel}~\ab R represents a certain collection of \ab{I}-tuples of \ab{J}-tuples, namely, the tuples \ab{𝒶}~\as :~\ab I~\as →~\ab J~\as →~\ab A for which \af{eval-cont-rel}~\ab R~\ab 𝒶 holds. For simplicity, pretend that \ab{J} is a finite set, say, \{1, 2, ...,~\ab J\}, so that we can write down a couple of the \ab{J}-tuples as columns. For example, here are the \ab i-th and \ab k-th columns (for some \ab i~\ab k~\as :~\ab I).
% ~
% \vskip-10mm
% ~
\begin{center}
\begin{tabular}{rll}
\ab 𝒶 \ab i 1 & \ab 𝒶 \ab k 1 & \\
\ab 𝒶 \ab i 2 &    \ab  𝒶 \ab k 2 & \\
\ab 𝒶 \ab i 3 & \ab 𝒶 \ab k 3 &   ← (if there are \ab I columns, then each row forms an \ab I-tuple)  \\
  ⋮     &     ⋮  & \\
\ab 𝒶 \ab i \ab J  &   \ab 𝒶 \ab k \ab J  & \\
\end{tabular}
\end{center}
% ~
% \vskip-5mm
% ~
Now \af{eval-cont-rel}~\ab R~\ab 𝒶 is defined by \as{∀}~\ab j~\as →~\ab R~(\as λ~\ab i~\as →~(\ab 𝒶~\ab i)~\ab j) which represents the assertion that each row of the \ab{I} columns shown above %(which evidently is an \ab{I}-tuple)
 belongs to the original relation \ab{R}. Finally, \af{cont-compatible-fun} takes a \ab{J}-ary operation on \ab{A}, say, \ab 𝑓~\as :~(\ab J~\as →~\ab A)~\as →~\ab A, and an \ab I-tuple \ab{𝒶}~\ab i~\as :~\ab J~\as →~\ab A of \ab{J}-tuples, and determines whether the \ab I-tuple \as λ~\ab i~\as →~\ab 𝑓~(\ab 𝒶~\ab i) belongs to \ab R.
% Finally, digest all the types involved in these definitions and note how nicely they align (as they must if type-checking is to succeed!). For example, \ab 𝒶~\as :~\ab I~\as →~\ab J~\as →~\ab A is precisely the type on which the relation \af{eval-cont-rel}~\ab R is defined.

\subsubsection{Dependent relations}\label{dependent-relations}

In this section we exploit the power of dependent types to define a completely general relation type. Specifically, we let the tuples inhabit a dependent function type, where the codomain may depend upon the input coordinate \ab i \as : \ab I of the domain. 
Heuristically, think of the inhabitants of the following type as relations from \ab{A} \ab{i} to \ab{A} \ab{j} to \ab{A} \ab{k} to …. (This is only an heuristic since \ab I can represent an uncountable collection.\cref{uncountable})
\ccpad
\begin{code}%
\>[0]\AgdaFunction{DepRel}\AgdaSpace{}%
\AgdaSymbol{:}\AgdaSpace{}%
\AgdaSymbol{(}\AgdaBound{I}\AgdaSpace{}%
\AgdaSymbol{:}\AgdaSpace{}%
\AgdaGeneralizable{𝓥}\AgdaSpace{}%
\AgdaOperator{\AgdaFunction{̇}}\AgdaSymbol{)(}\AgdaBound{A}\AgdaSpace{}%
\AgdaSymbol{:}\AgdaSpace{}%
\AgdaBound{I}\AgdaSpace{}%
\AgdaSymbol{→}\AgdaSpace{}%
\AgdaGeneralizable{𝓤}\AgdaSpace{}%
\AgdaOperator{\AgdaFunction{̇}}\AgdaSymbol{)(}\AgdaBound{𝓦}\AgdaSpace{}%
\AgdaSymbol{:}\AgdaSpace{}%
\AgdaPostulate{Universe}\AgdaSymbol{)}\AgdaSpace{}%
\AgdaSymbol{→}\AgdaSpace{}%
\AgdaGeneralizable{𝓥}\AgdaSpace{}%
\AgdaOperator{\AgdaPrimitive{⊔}}\AgdaSpace{}%
\AgdaGeneralizable{𝓤}\AgdaSpace{}%
\AgdaOperator{\AgdaPrimitive{⊔}}\AgdaSpace{}%
\AgdaBound{𝓦}\AgdaSpace{}%
\AgdaOperator{\AgdaPrimitive{⁺}}\AgdaSpace{}%
\AgdaOperator{\AgdaFunction{̇}}\<%
\\
\>[0]\AgdaFunction{DepRel}\AgdaSpace{}%
\AgdaBound{I}\AgdaSpace{}%
\AgdaBound{A}\AgdaSpace{}%
\AgdaBound{𝓦}\AgdaSpace{}%
\AgdaSymbol{=}\AgdaSpace{}%
\AgdaFunction{Π}\AgdaSpace{}%
\AgdaBound{A}\AgdaSpace{}%
\AgdaSymbol{→}\AgdaSpace{}%
\AgdaBound{𝓦}\AgdaSpace{}%
\AgdaOperator{\AgdaFunction{̇}}\<%
\end{code}
\ccpad
We call \af{DepRel} the type of \defn{dependent relations}.

% \subsubsection{Compatibility with dependent relations}\label{compatibility-with-dependent-relations}

Above we saw lifts of continuous relations and what it means for such relations to be compatible with functions. We conclude this module by defining the (only slightly more complicated) lift of dependent relations, and the type that represents compatibility of a tuple of operations with a dependent relation.
\ccpad
\begin{code}%
\>[0]\AgdaKeyword{module}\AgdaSpace{}%
\AgdaModule{\AgdaUnderscore{}}\AgdaSpace{}%
\AgdaSymbol{\{}\AgdaBound{I}\AgdaSpace{}%
\AgdaBound{J}\AgdaSpace{}%
\AgdaSymbol{:}\AgdaSpace{}%
\AgdaGeneralizable{𝓥}\AgdaSpace{}%
\AgdaOperator{\AgdaFunction{̇}}\AgdaSymbol{\}}\AgdaSpace{}%
\AgdaSymbol{\{}\AgdaBound{𝒜}\AgdaSpace{}%
\AgdaSymbol{:}\AgdaSpace{}%
\AgdaBound{I}\AgdaSpace{}%
\AgdaSymbol{→}\AgdaSpace{}%
\AgdaGeneralizable{𝓤}\AgdaSpace{}%
\AgdaOperator{\AgdaFunction{̇}}\AgdaSymbol{\}}\AgdaSpace{}%
\AgdaKeyword{where}\<%
\\
%
\\[\AgdaEmptyExtraSkip]%
\>[0][@{}l@{\AgdaIndent{0}}]%
\>[1]\AgdaFunction{eval-dep-rel}\AgdaSpace{}%
\AgdaSymbol{:}\AgdaSpace{}%
\AgdaFunction{DepRel}\AgdaSpace{}%
\AgdaBound{I}\AgdaSpace{}%
\AgdaBound{𝒜}\AgdaSpace{}%
\AgdaGeneralizable{𝓦}\AgdaSpace{}%
\AgdaSymbol{→}\AgdaSpace{}%
\AgdaSymbol{(∀}\AgdaSpace{}%
\AgdaBound{i}\AgdaSpace{}%
\AgdaSymbol{→}\AgdaSpace{}%
\AgdaBound{J}\AgdaSpace{}%
\AgdaSymbol{→}\AgdaSpace{}%
\AgdaBound{𝒜}\AgdaSpace{}%
\AgdaBound{i}\AgdaSymbol{)}\AgdaSpace{}%
\AgdaSymbol{→}\AgdaSpace{}%
\AgdaBound{𝓥}\AgdaSpace{}%
\AgdaOperator{\AgdaPrimitive{⊔}}\AgdaSpace{}%
\AgdaGeneralizable{𝓦}\AgdaSpace{}%
\AgdaOperator{\AgdaFunction{̇}}\<%
\\
%
\>[1]\AgdaFunction{eval-dep-rel}\AgdaSpace{}%
\AgdaBound{R}\AgdaSpace{}%
\AgdaBound{𝒶}\AgdaSpace{}%
\AgdaSymbol{=}\AgdaSpace{}%
\AgdaSymbol{∀}\AgdaSpace{}%
\AgdaSymbol{(}\AgdaBound{j}\AgdaSpace{}%
\AgdaSymbol{:}\AgdaSpace{}%
\AgdaBound{J}\AgdaSymbol{)}\AgdaSpace{}%
\AgdaSymbol{→}\AgdaSpace{}%
\AgdaBound{R}\AgdaSpace{}%
\AgdaSymbol{(λ}\AgdaSpace{}%
\AgdaBound{i}\AgdaSpace{}%
\AgdaSymbol{→}\AgdaSpace{}%
\AgdaSymbol{(}\AgdaBound{𝒶}\AgdaSpace{}%
\AgdaBound{i}\AgdaSymbol{)}\AgdaSpace{}%
\AgdaBound{j}\AgdaSymbol{)}\<%
\\
%
\\[\AgdaEmptyExtraSkip]%
%
\>[1]\AgdaFunction{dep-compatible-fun}\AgdaSpace{}%
\AgdaSymbol{:}\AgdaSpace{}%
\AgdaSymbol{(∀}\AgdaSpace{}%
\AgdaBound{i}\AgdaSpace{}%
\AgdaSymbol{→}\AgdaSpace{}%
\AgdaSymbol{(}\AgdaBound{J}\AgdaSpace{}%
\AgdaSymbol{→}\AgdaSpace{}%
\AgdaBound{𝒜}\AgdaSpace{}%
\AgdaBound{i}\AgdaSymbol{)}\AgdaSpace{}%
\AgdaSymbol{→}\AgdaSpace{}%
\AgdaBound{𝒜}\AgdaSpace{}%
\AgdaBound{i}\AgdaSymbol{)}\AgdaSpace{}%
\AgdaSymbol{→}\AgdaSpace{}%
\AgdaFunction{DepRel}\AgdaSpace{}%
\AgdaBound{I}\AgdaSpace{}%
\AgdaBound{𝒜}\AgdaSpace{}%
\AgdaGeneralizable{𝓦}\AgdaSpace{}%
\AgdaSymbol{→}\AgdaSpace{}%
\AgdaBound{𝓥}\AgdaSpace{}%
\AgdaOperator{\AgdaPrimitive{⊔}}\AgdaSpace{}%
\AgdaBound{𝓤}\AgdaSpace{}%
\AgdaOperator{\AgdaPrimitive{⊔}}\AgdaSpace{}%
\AgdaGeneralizable{𝓦}\AgdaSpace{}%
\AgdaOperator{\AgdaFunction{̇}}\<%
\\
%
\>[1]\AgdaFunction{dep-compatible-fun}\AgdaSpace{}%
\AgdaBound{𝑓}\AgdaSpace{}%
\AgdaBound{R}%
\>[25]\AgdaSymbol{=}\AgdaSpace{}%
\AgdaSymbol{∀}\AgdaSpace{}%
\AgdaBound{𝒶}\AgdaSpace{}%
\AgdaSymbol{→}\AgdaSpace{}%
\AgdaSymbol{(}\AgdaFunction{eval-dep-rel}\AgdaSpace{}%
\AgdaBound{R}\AgdaSymbol{)}\AgdaSpace{}%
\AgdaBound{𝒶}\AgdaSpace{}%
\AgdaSymbol{→}\AgdaSpace{}%
\AgdaBound{R}\AgdaSpace{}%
\AgdaSymbol{λ}\AgdaSpace{}%
\AgdaBound{i}\AgdaSpace{}%
\AgdaSymbol{→}\AgdaSpace{}%
\AgdaSymbol{(}\AgdaBound{𝑓}\AgdaSpace{}%
\AgdaBound{i}\AgdaSymbol{)(}\AgdaBound{𝒶}\AgdaSpace{}%
\AgdaBound{i}\AgdaSymbol{)}\<%
\end{code}
\ccpad
In the definition of \af{dep-compatible-fun}, we let Agda infer the type of \ab{𝒶}, which is \af Π~\ab i~\af ꞉~\ab I~\af ,~(\ab J~\as →~\ab 𝒜~\ab i) in this case; equivalently, \AgdaSymbol{∀}\AgdaSpace{}%
\AgdaBound{i}\AgdaSpace{}%
\AgdaSymbol{→}\AgdaSpace{}%
\AgdaSymbol{(}\AgdaBound{J}\AgdaSpace{}%
\AgdaSymbol{→}\AgdaSpace{}%
\AgdaBound{𝒜}\AgdaSpace{}%
\AgdaBound{i}\AgdaSymbol{)}.


%%%%%%%%%%%%%%%%%%%%%%%%%%%%%%%%%%%%%%%%%%%%%%%%%%%%%%%%%%%%%%%%%%%%%%%%%%%%
\subsection{Quotients}\label{sec:equiv-relat-quot}\firstsentence{Relations}{Quotients}
% \context{\{\ab 𝓤 𝓦 \as : \AgdaPostulate{Universe}\}\{\ab A \as : \ab 𝓤\AgdaOperator{\AgdaFunction{̇}}\}}
%: equivalences, class representatives, quotient types
% -*- TeX-master: "ualib-part1.tex" -*-
%%% Local Variables: 
%%% mode: latex
%%% TeX-engine: 'xetex
%%% End:
\subsubsection{Properties of binary relations}
\label{props-of-bin-rels}
In the \ualibhtml{Relations.Discrete} module we defined types for representing and reasoning about binary relations on \ab A. In this module we will define types for binary relations that have special properties. The most important special properties of relations are the ones we now define.
\ccpad
\begin{code}%
% \>[0]\AgdaKeyword{module}\AgdaSpace{}%
% \AgdaModule{\AgdaUnderscore{}}\AgdaSpace{}%
% \AgdaSymbol{\{}\AgdaBound{𝓤}\AgdaSpace{}%
% \AgdaSymbol{:}\AgdaSpace{}%
% \AgdaPostulate{Universe}\AgdaSymbol{\}\{}\AgdaBound{A}\AgdaSpace{}%
% \AgdaSymbol{:}\AgdaSpace{}%
% \AgdaBound{𝓤}\AgdaSpace{}%
% \AgdaOperator{\AgdaFunction{̇}}\AgdaSpace{}%
% \AgdaSymbol{\}\{}\AgdaBound{𝓦}\AgdaSpace{}%
% \AgdaSymbol{:}\AgdaSpace{}%
% \AgdaPostulate{Universe}\AgdaSymbol{\}}\AgdaSpace{}%
% \AgdaKeyword{where}\<%
% \\
% %
% \\[\AgdaEmptyExtraSkip]%
% \>[0][@{}l@{\AgdaIndent{0}}]%
\>[1]\AgdaFunction{reflexive}\AgdaSpace{}%
\AgdaSymbol{:}\AgdaSpace{}%
\AgdaFunction{Rel}\AgdaSpace{}%
\AgdaBound{A}\AgdaSpace{}%
\AgdaBound{𝓦}\AgdaSpace{}%
\AgdaSymbol{→}\AgdaSpace{}%
\AgdaBound{𝓤}\AgdaSpace{}%
\AgdaOperator{\AgdaPrimitive{⊔}}\AgdaSpace{}%
\AgdaBound{𝓦}\AgdaSpace{}%
\AgdaOperator{\AgdaFunction{̇}}\<%
\\
%
\>[1]\AgdaFunction{reflexive}\AgdaSpace{}%
\AgdaOperator{\AgdaBound{\AgdaUnderscore{}≈\AgdaUnderscore{}}}\AgdaSpace{}%
\AgdaSymbol{=}\AgdaSpace{}%
\AgdaSymbol{∀}\AgdaSpace{}%
\AgdaBound{x}\AgdaSpace{}%
\AgdaSymbol{→}\AgdaSpace{}%
\AgdaBound{x}\AgdaSpace{}%
\AgdaOperator{\AgdaBound{≈}}\AgdaSpace{}%
\AgdaBound{x}\<%
\\
%
\\[\AgdaEmptyExtraSkip]%
%
\>[1]\AgdaFunction{symmetric}\AgdaSpace{}%
\AgdaSymbol{:}\AgdaSpace{}%
\AgdaFunction{Rel}\AgdaSpace{}%
\AgdaBound{A}\AgdaSpace{}%
\AgdaBound{𝓦}\AgdaSpace{}%
\AgdaSymbol{→}\AgdaSpace{}%
\AgdaBound{𝓤}\AgdaSpace{}%
\AgdaOperator{\AgdaPrimitive{⊔}}\AgdaSpace{}%
\AgdaBound{𝓦}\AgdaSpace{}%
\AgdaOperator{\AgdaFunction{̇}}\<%
\\
%
\>[1]\AgdaFunction{symmetric}\AgdaSpace{}%
\AgdaOperator{\AgdaBound{\AgdaUnderscore{}≈\AgdaUnderscore{}}}\AgdaSpace{}%
\AgdaSymbol{=}\AgdaSpace{}%
\AgdaSymbol{∀}\AgdaSpace{}%
\AgdaBound{x}\AgdaSpace{}%
\AgdaBound{y}\AgdaSpace{}%
\AgdaSymbol{→}\AgdaSpace{}%
\AgdaBound{x}\AgdaSpace{}%
\AgdaOperator{\AgdaBound{≈}}\AgdaSpace{}%
\AgdaBound{y}\AgdaSpace{}%
\AgdaSymbol{→}\AgdaSpace{}%
\AgdaBound{y}\AgdaSpace{}%
\AgdaOperator{\AgdaBound{≈}}\AgdaSpace{}%
\AgdaBound{x}\<%
\\
%
\\[\AgdaEmptyExtraSkip]%
%
\>[1]\AgdaFunction{antisymmetric}\AgdaSpace{}%
\AgdaSymbol{:}\AgdaSpace{}%
\AgdaFunction{Rel}\AgdaSpace{}%
\AgdaBound{A}\AgdaSpace{}%
\AgdaBound{𝓦}\AgdaSpace{}%
\AgdaSymbol{→}\AgdaSpace{}%
\AgdaBound{𝓤}\AgdaSpace{}%
\AgdaOperator{\AgdaPrimitive{⊔}}\AgdaSpace{}%
\AgdaBound{𝓦}\AgdaSpace{}%
\AgdaOperator{\AgdaFunction{̇}}\<%
\\
%
\>[1]\AgdaFunction{antisymmetric}\AgdaSpace{}%
\AgdaOperator{\AgdaBound{\AgdaUnderscore{}≈\AgdaUnderscore{}}}\AgdaSpace{}%
\AgdaSymbol{=}\AgdaSpace{}%
\AgdaSymbol{∀}\AgdaSpace{}%
\AgdaBound{x}\AgdaSpace{}%
\AgdaBound{y}\AgdaSpace{}%
\AgdaSymbol{→}\AgdaSpace{}%
\AgdaBound{x}\AgdaSpace{}%
\AgdaOperator{\AgdaBound{≈}}\AgdaSpace{}%
\AgdaBound{y}\AgdaSpace{}%
\AgdaSymbol{→}\AgdaSpace{}%
\AgdaBound{y}\AgdaSpace{}%
\AgdaOperator{\AgdaBound{≈}}\AgdaSpace{}%
\AgdaBound{x}\AgdaSpace{}%
\AgdaSymbol{→}\AgdaSpace{}%
\AgdaBound{x}\AgdaSpace{}%
\AgdaOperator{\AgdaDatatype{≡}}\AgdaSpace{}%
\AgdaBound{y}\<%
\\
%
\\[\AgdaEmptyExtraSkip]%
%
\>[1]\AgdaFunction{transitive}\AgdaSpace{}%
\AgdaSymbol{:}\AgdaSpace{}%
\AgdaFunction{Rel}\AgdaSpace{}%
\AgdaBound{A}\AgdaSpace{}%
\AgdaBound{𝓦}\AgdaSpace{}%
\AgdaSymbol{→}\AgdaSpace{}%
\AgdaBound{𝓤}\AgdaSpace{}%
\AgdaOperator{\AgdaPrimitive{⊔}}\AgdaSpace{}%
\AgdaBound{𝓦}\AgdaSpace{}%
\AgdaOperator{\AgdaFunction{̇}}\<%
\\
%
\>[1]\AgdaFunction{transitive}\AgdaSpace{}%
\AgdaOperator{\AgdaBound{\AgdaUnderscore{}≈\AgdaUnderscore{}}}\AgdaSpace{}%
\AgdaSymbol{=}\AgdaSpace{}%
\AgdaSymbol{∀}\AgdaSpace{}%
\AgdaBound{x}\AgdaSpace{}%
\AgdaBound{y}\AgdaSpace{}%
\AgdaBound{z}\AgdaSpace{}%
\AgdaSymbol{→}\AgdaSpace{}%
\AgdaBound{x}\AgdaSpace{}%
\AgdaOperator{\AgdaBound{≈}}\AgdaSpace{}%
\AgdaBound{y}\AgdaSpace{}%
\AgdaSymbol{→}\AgdaSpace{}%
\AgdaBound{y}\AgdaSpace{}%
\AgdaOperator{\AgdaBound{≈}}\AgdaSpace{}%
\AgdaBound{z}\AgdaSpace{}%
\AgdaSymbol{→}\AgdaSpace{}%
\AgdaBound{x}\AgdaSpace{}%
\AgdaOperator{\AgdaBound{≈}}\AgdaSpace{}%
\AgdaBound{z}\<%
\end{code}
\ccpad
The \typetopology library defines a \defn{uniqueness-of-proofs} principle for binary relations.
\ccpad
\begin{code}%
% \>[0]\AgdaKeyword{module}\AgdaSpace{}%
% \AgdaModule{hide-is-subsingleton-valued}\AgdaSpace{}%
% \AgdaSymbol{\{}\AgdaBound{𝓤}\AgdaSpace{}%
% \AgdaBound{𝓦}\AgdaSpace{}%
% \AgdaSymbol{:}\AgdaSpace{}%
% \AgdaPostulate{Universe}\AgdaSymbol{\}\{}\AgdaBound{A}\AgdaSpace{}%
% \AgdaSymbol{:}\AgdaSpace{}%
% \AgdaBound{𝓤}\AgdaSpace{}%
% \AgdaOperator{\AgdaFunction{̇}}\AgdaSpace{}%
% \AgdaSymbol{\}}\AgdaSpace{}%
% \AgdaKeyword{where}\<%
% \\
% %
% \\[\AgdaEmptyExtraSkip]%
% \>[0][@{}l@{\AgdaIndent{0}}]%
\>[1]\AgdaFunction{is-subsingleton-valued}\AgdaSpace{}%
\AgdaSymbol{:}\AgdaSpace{}%
\AgdaFunction{Rel}\AgdaSpace{}%
\AgdaBound{A}\AgdaSpace{}%
\AgdaBound{𝓦}\AgdaSpace{}%
\AgdaSymbol{→}\AgdaSpace{}%
\AgdaBound{𝓤}\AgdaSpace{}%
\AgdaOperator{\AgdaPrimitive{⊔}}\AgdaSpace{}%
\AgdaBound{𝓦}\AgdaSpace{}%
\AgdaOperator{\AgdaFunction{̇}}\<%
\\
%
\>[1]\AgdaFunction{is-subsingleton-valued}%
\>[25]\AgdaOperator{\AgdaBound{\AgdaUnderscore{}≈\AgdaUnderscore{}}}\AgdaSpace{}%
\AgdaSymbol{=}\AgdaSpace{}%
\AgdaSymbol{∀}\AgdaSpace{}%
\AgdaBound{x}\AgdaSpace{}%
\AgdaBound{y}\AgdaSpace{}%
\AgdaSymbol{→}\AgdaSpace{}%
\AgdaFunction{is-subsingleton}\AgdaSpace{}%
\AgdaSymbol{(}\AgdaBound{x}\AgdaSpace{}%
\AgdaOperator{\AgdaBound{≈}}\AgdaSpace{}%
\AgdaBound{y}\AgdaSymbol{)}\<%
\end{code}
\ccpad
Thus, if \ab R \as : \af{Rel} \ab A \ab 𝓦, then \AgdaFunction{is-subsingleton-valued} \ab R asserts that for each pair \ab x \ab y \as : \ab A there is \emph{at most one proof} of \ab R \ab x \ab y. In the section on \emph{Truncation} below (\S~\ref{sec:trunc-sets-prop}) we introduce a number of similar but more general types to represent uniqueness-of-proofs principles for relations of arbitrary arity, over arbitrary types.







\subsubsection{Equivalence classes}\label{equivalence-classes}

A binary relation is called a \defn{preorder} if it is reflexive and transitive. An \defn{equivalence relation} is a symmetric preorder. Here are the types we use to represent these concepts in the \ualib.
\ccpad
\begin{code}%
% \>[0]\AgdaKeyword{module}\AgdaSpace{}%
% \AgdaModule{\AgdaUnderscore{}}\AgdaSpace{}%
% \AgdaSymbol{\{}\AgdaBound{𝓤}\AgdaSpace{}%
% \AgdaBound{𝓦}\AgdaSpace{}%
% \AgdaSymbol{:}\AgdaSpace{}%
% \AgdaPostulate{Universe}\AgdaSymbol{\}\{}\AgdaBound{A}\AgdaSpace{}%
% \AgdaSymbol{:}\AgdaSpace{}%
% \AgdaBound{𝓤}\AgdaSpace{}%
% \AgdaOperator{\AgdaFunction{̇}}\AgdaSpace{}%
% \AgdaSymbol{\}}\AgdaSpace{}%
% \AgdaKeyword{where}\<%
% \\
% %
% \\[\AgdaEmptyExtraSkip]%
% \>[0][@{}l@{\AgdaIndent{0}}]%
\>[1]\AgdaFunction{is-preorder}\AgdaSpace{}%
\AgdaSymbol{:}\AgdaSpace{}%
\AgdaFunction{Rel}\AgdaSpace{}%
\AgdaBound{A}\AgdaSpace{}%
\AgdaBound{𝓦}\AgdaSpace{}%
\AgdaSymbol{→}\AgdaSpace{}%
\AgdaBound{𝓤}\AgdaSpace{}%
\AgdaOperator{\AgdaPrimitive{⊔}}\AgdaSpace{}%
\AgdaBound{𝓦}\AgdaSpace{}%
\AgdaOperator{\AgdaFunction{̇}}\<%
\\
%
\>[1]\AgdaFunction{is-preorder}\AgdaSpace{}%
\AgdaOperator{\AgdaBound{\AgdaUnderscore{}≈\AgdaUnderscore{}}}\AgdaSpace{}%
\AgdaSymbol{=}\AgdaSpace{}%
\AgdaFunction{is-subsingleton-valued}\AgdaSpace{}%
\AgdaOperator{\AgdaBound{\AgdaUnderscore{}≈\AgdaUnderscore{}}}\AgdaSpace{}%
\AgdaOperator{\AgdaFunction{×}}\AgdaSpace{}%
\AgdaFunction{reflexive}\AgdaSpace{}%
\AgdaOperator{\AgdaBound{\AgdaUnderscore{}≈\AgdaUnderscore{}}}\AgdaSpace{}%
\AgdaOperator{\AgdaFunction{×}}\AgdaSpace{}%
\AgdaFunction{transitive}\AgdaSpace{}%
\AgdaOperator{\AgdaBound{\AgdaUnderscore{}≈\AgdaUnderscore{}}}\<%
\\
%
\\[\AgdaEmptyExtraSkip]%
%
\>[1]\AgdaKeyword{record}\AgdaSpace{}%
\AgdaRecord{IsEquivalence}\AgdaSpace{}%
\AgdaSymbol{(}\AgdaOperator{\AgdaBound{\AgdaUnderscore{}≈\AgdaUnderscore{}}}\AgdaSpace{}%
\AgdaSymbol{:}\AgdaSpace{}%
\AgdaFunction{Rel}\AgdaSpace{}%
\AgdaBound{A}\AgdaSpace{}%
\AgdaBound{𝓦}\AgdaSymbol{)}\AgdaSpace{}%
\AgdaSymbol{:}\AgdaSpace{}%
\AgdaBound{𝓤}\AgdaSpace{}%
\AgdaOperator{\AgdaPrimitive{⊔}}\AgdaSpace{}%
\AgdaBound{𝓦}\AgdaSpace{}%
\AgdaOperator{\AgdaFunction{̇}}\AgdaSpace{}%
\AgdaKeyword{where}\<%
\\
\>[1][@{}l@{\AgdaIndent{0}}]%
\>[2]\AgdaKeyword{field}\<%
\\
\>[2][@{}l@{\AgdaIndent{0}}]%
\>[3]\AgdaField{rfl}%
\>[9]\AgdaSymbol{:}\AgdaSpace{}%
\AgdaFunction{reflexive}\AgdaSpace{}%
\AgdaOperator{\AgdaBound{\AgdaUnderscore{}≈\AgdaUnderscore{}}}\<%
\\
%
\>[3]\AgdaField{sym}%
\>[9]\AgdaSymbol{:}\AgdaSpace{}%
\AgdaFunction{symmetric}\AgdaSpace{}%
\AgdaOperator{\AgdaBound{\AgdaUnderscore{}≈\AgdaUnderscore{}}}\<%
\\
%
\>[3]\AgdaField{trans}\AgdaSpace{}%
\AgdaSymbol{:}\AgdaSpace{}%
\AgdaFunction{transitive}\AgdaSpace{}%
\AgdaOperator{\AgdaBound{\AgdaUnderscore{}≈\AgdaUnderscore{}}}\<%
\\
%
\\[\AgdaEmptyExtraSkip]%
%
\>[1]\AgdaFunction{is-equivalence}\AgdaSpace{}%
\AgdaSymbol{:}\AgdaSpace{}%
\AgdaFunction{Rel}\AgdaSpace{}%
\AgdaBound{A}\AgdaSpace{}%
\AgdaBound{𝓦}\AgdaSpace{}%
\AgdaSymbol{→}\AgdaSpace{}%
\AgdaBound{𝓤}\AgdaSpace{}%
\AgdaOperator{\AgdaPrimitive{⊔}}\AgdaSpace{}%
\AgdaBound{𝓦}\AgdaSpace{}%
\AgdaOperator{\AgdaFunction{̇}}\<%
\\
%
\>[1]\AgdaFunction{is-equivalence}\AgdaSpace{}%
\AgdaOperator{\AgdaBound{\AgdaUnderscore{}≈\AgdaUnderscore{}}}\AgdaSpace{}%
\AgdaSymbol{=}\AgdaSpace{}%
\AgdaFunction{is-preorder}\AgdaSpace{}%
\AgdaOperator{\AgdaBound{\AgdaUnderscore{}≈\AgdaUnderscore{}}}\AgdaSpace{}%
\AgdaOperator{\AgdaFunction{×}}\AgdaSpace{}%
\AgdaFunction{symmetric}\AgdaSpace{}%
\AgdaOperator{\AgdaBound{\AgdaUnderscore{}≈\AgdaUnderscore{}}}\<%
\end{code}
\scpad

An easy first example of an equivalence relation is the kernel of any function. Here is how we prove that the kernel of a function is, indeed, an equivalence relation on the domain of the function.
\ccpad
\begin{code}%
% \>[0]\AgdaKeyword{module}\AgdaSpace{}%
% \AgdaModule{\AgdaUnderscore{}}\AgdaSpace{}%
% \AgdaSymbol{\{}\AgdaBound{𝓤}\AgdaSpace{}%
% \AgdaBound{𝓦}\AgdaSpace{}%
% \AgdaSymbol{:}\AgdaSpace{}%
% \AgdaPostulate{Universe}\AgdaSymbol{\}\{}\AgdaBound{A}\AgdaSpace{}%
% \AgdaSymbol{:}\AgdaSpace{}%
% \AgdaBound{𝓤}\AgdaSpace{}%
% \AgdaOperator{\AgdaFunction{̇}}\AgdaSymbol{\}\{}\AgdaBound{B}\AgdaSpace{}%
% \AgdaSymbol{:}\AgdaSpace{}%
% \AgdaBound{𝓦}\AgdaSpace{}%
% \AgdaOperator{\AgdaFunction{̇}}\AgdaSymbol{\}}\AgdaSpace{}%
% \AgdaKeyword{where}\<%
% \\
% %
% \\[\AgdaEmptyExtraSkip]%
% \>[0][@{}l@{\AgdaIndent{0}}]%
\>[1]\AgdaFunction{map-kernel-IsEquivalence}\AgdaSpace{}%
\AgdaSymbol{:}\AgdaSpace{}%
\AgdaSymbol{(}\AgdaBound{f}\AgdaSpace{}%
\AgdaSymbol{:}\AgdaSpace{}%
\AgdaBound{A}\AgdaSpace{}%
\AgdaSymbol{→}\AgdaSpace{}%
\AgdaBound{B}\AgdaSymbol{)}\AgdaSpace{}%
\AgdaSymbol{→}\AgdaSpace{}%
\AgdaRecord{IsEquivalence}\AgdaSpace{}%
\AgdaSymbol{(}\AgdaFunction{ker}\AgdaSymbol{\{}\AgdaBound{𝓤}\AgdaSymbol{\}\{}\AgdaBound{𝓦}\AgdaSymbol{\}}\AgdaSpace{}%
\AgdaBound{f}\AgdaSymbol{)}\<%
\\
%
\>[1]\AgdaFunction{map-kernel-IsEquivalence}\AgdaSpace{}%
\AgdaBound{f}\AgdaSpace{}%
\AgdaSymbol{=}\AgdaSpace{}%
\AgdaKeyword{record}%
\>[235I]\AgdaSymbol{\{}\AgdaSpace{}%
\AgdaField{rfl}\AgdaSpace{}%
\AgdaSymbol{=}\AgdaSpace{}%
\AgdaSymbol{λ}\AgdaSpace{}%
\AgdaBound{x}\AgdaSpace{}%
\AgdaSymbol{→}\AgdaSpace{}%
\AgdaInductiveConstructor{refl}\<%
\\
\>[.][@{}l@{}]\<[235I]%
\>[37]\AgdaSymbol{;}\AgdaSpace{}%
\AgdaField{sym}\AgdaSpace{}%
\AgdaSymbol{=}\AgdaSpace{}%
\AgdaSymbol{λ}\AgdaSpace{}%
\AgdaBound{x}\AgdaSpace{}%
\AgdaBound{y}\AgdaSpace{}%
\AgdaBound{x₁}\AgdaSpace{}%
\AgdaSymbol{→}\AgdaSpace{}%
\AgdaFunction{≡-sym}\AgdaSymbol{\{}\AgdaBound{𝓦}\AgdaSymbol{\}}\AgdaSpace{}%
\AgdaBound{x₁}\<%
\\
%
\>[37]\AgdaSymbol{;}\AgdaSpace{}%
\AgdaField{trans}\AgdaSpace{}%
\AgdaSymbol{=}\AgdaSpace{}%
\AgdaSymbol{λ}\AgdaSpace{}%
\AgdaBound{x}\AgdaSpace{}%
\AgdaBound{y}\AgdaSpace{}%
\AgdaBound{z}\AgdaSpace{}%
\AgdaBound{x₁}\AgdaSpace{}%
\AgdaBound{x₂}\AgdaSpace{}%
\AgdaSymbol{→}\AgdaSpace{}%
\AgdaFunction{≡-trans}\AgdaSpace{}%
\AgdaBound{x₁}\AgdaSpace{}%
\AgdaBound{x₂}\AgdaSpace{}%
\AgdaSymbol{\}}\<%
\end{code}

\subsubsection{Equivalence classes}\label{equivalence-classes-1}

If \ab R is an equivalence relation on \ab A, then for each \abt{𝑎}{A}, there is an \defn{equivalence class} containing \ab 𝑎, which we denote and define by \af{[} \ab 𝑎 \af{]} \ab R \as{:=} all \abt{𝑏}{A} such that \ab R \ab 𝑎 \ab 𝑏.
\ccpad
\begin{code}%
% \>[0]\AgdaKeyword{module}\AgdaSpace{}%
% \AgdaModule{\AgdaUnderscore{}}\AgdaSpace{}%
% \AgdaSymbol{\{}\AgdaBound{𝓤}\AgdaSpace{}%
% \AgdaBound{𝓦}\AgdaSpace{}%
% \AgdaSymbol{:}\AgdaSpace{}%
% \AgdaPostulate{Universe}\AgdaSymbol{\}\{}\AgdaBound{A}\AgdaSpace{}%
% \AgdaSymbol{:}\AgdaSpace{}%
% \AgdaBound{𝓤}\AgdaSpace{}%
% \AgdaOperator{\AgdaFunction{̇}}\AgdaSpace{}%
% \AgdaSymbol{\}}\AgdaSpace{}%
% \AgdaKeyword{where}\<%
% \\
% %
% \\[\AgdaEmptyExtraSkip]%
% \>[0][@{}l@{\AgdaIndent{0}}]%
\>[1]\AgdaOperator{\AgdaFunction{[\AgdaUnderscore{}]\AgdaUnderscore{}}}\AgdaSpace{}%
\AgdaSymbol{:}\AgdaSpace{}%
\AgdaBound{A}\AgdaSpace{}%
\AgdaSymbol{→}\AgdaSpace{}%
\AgdaFunction{Rel}\AgdaSpace{}%
\AgdaBound{A}\AgdaSpace{}%
\AgdaBound{𝓦}\AgdaSpace{}%
\AgdaSymbol{→}\AgdaSpace{}%
\AgdaFunction{Pred}\AgdaSpace{}%
\AgdaBound{A}\AgdaSpace{}%
\AgdaBound{𝓦}\<%
\\
%
\>[1]\AgdaOperator{\AgdaFunction{[}}\AgdaSpace{}%
\AgdaBound{𝑎}\AgdaSpace{}%
\AgdaOperator{\AgdaFunction{]}}\AgdaSpace{}%
\AgdaBound{R}\AgdaSpace{}%
\AgdaSymbol{=}\AgdaSpace{}%
\AgdaSymbol{λ}\AgdaSpace{}%
\AgdaBound{x}\AgdaSpace{}%
\AgdaSymbol{→}\AgdaSpace{}%
\AgdaBound{R}\AgdaSpace{}%
\AgdaBound{𝑎}\AgdaSpace{}%
\AgdaBound{x}\<%
\end{code}
\ccpad
Thus, \ab{x} \af ∈ \as{[} \ab a \as{]} \ab R if and only if \ab{R} \ab a \ab x, as desired. We often refer to \af{[} \ab 𝑎 \af{]} \ab R as the \emph{R-class containing} \ab 𝑎, and we represent the collection of all such \ab R-classes by the following type.
\ccpad
\begin{code}%
\>[1]\AgdaFunction{𝒞}\AgdaSpace{}%
\AgdaSymbol{:}\AgdaSpace{}%
\AgdaSymbol{\{}\AgdaBound{R}\AgdaSpace{}%
\AgdaSymbol{:}\AgdaSpace{}%
\AgdaFunction{Rel}\AgdaSpace{}%
\AgdaBound{A}\AgdaSpace{}%
\AgdaBound{𝓦}\AgdaSymbol{\}}\AgdaSpace{}%
\AgdaSymbol{→}\AgdaSpace{}%
\AgdaFunction{Pred}\AgdaSpace{}%
\AgdaBound{A}\AgdaSpace{}%
\AgdaBound{𝓦}\AgdaSpace{}%
\AgdaSymbol{→}\AgdaSpace{}%
\AgdaSymbol{(}\AgdaBound{𝓤}\AgdaSpace{}%
\AgdaOperator{\AgdaPrimitive{⊔}}\AgdaSpace{}%
\AgdaBound{𝓦}\AgdaSpace{}%
\AgdaOperator{\AgdaPrimitive{⁺}}\AgdaSymbol{)}\AgdaSpace{}%
\AgdaOperator{\AgdaFunction{̇}}\<%
\\
%
\>[1]\AgdaFunction{𝒞}%
\>[4]\AgdaSymbol{\{}\AgdaBound{R}\AgdaSymbol{\}}\AgdaSpace{}%
\AgdaBound{C}\AgdaSpace{}%
\AgdaSymbol{=}\AgdaSpace{}%
\AgdaFunction{Σ}\AgdaSpace{}%
\AgdaBound{a}\AgdaSpace{}%
\AgdaFunction{꞉}\AgdaSpace{}%
\AgdaBound{A}\AgdaSpace{}%
\AgdaFunction{,}\AgdaSpace{}%
\AgdaBound{C}\AgdaSpace{}%
\AgdaOperator{\AgdaDatatype{≡}}\AgdaSpace{}%
\AgdaSymbol{(}\AgdaSpace{}%
\AgdaOperator{\AgdaFunction{[}}\AgdaSpace{}%
\AgdaBound{a}\AgdaSpace{}%
\AgdaOperator{\AgdaFunction{]}}\AgdaSpace{}%
\AgdaBound{R}\AgdaSymbol{)}\<%
\end{code}
\ccpad
If \ab{R} is an equivalence relation on \ab{A}, then the \defn{quotient} of \ab{A} modulo \ab{R}, denoted by
\ab{A} \aof / \ab R, is defined as the collection \as{\{}\af{[} \ab 𝑎 \af{]} \ab R \as{∣}  \abt{𝑎}{A}\as{\}} of equivalence classes of \ab{R}. There are a few ways we could represent the quotient with respect to a relation as a type, but we find the following to be the most useful.
\ccpad
\begin{code}%
% \>[0]\AgdaKeyword{module}\AgdaSpace{}%
% \AgdaModule{\AgdaUnderscore{}}\AgdaSpace{}%
% \AgdaSymbol{\{}\AgdaBound{𝓤}\AgdaSpace{}%
% \AgdaBound{𝓦}\AgdaSpace{}%
% \AgdaSymbol{:}\AgdaSpace{}%
% \AgdaPostulate{Universe}\AgdaSymbol{\}}\AgdaSpace{}%
% \AgdaKeyword{where}\<%
% \\
% %
% \\[\AgdaEmptyExtraSkip]%
% \>[0][@{}l@{\AgdaIndent{0}}]%
\>[1]\AgdaOperator{\AgdaFunction{\AgdaUnderscore{}/\AgdaUnderscore{}}}\AgdaSpace{}%
\AgdaSymbol{:}\AgdaSpace{}%
\AgdaSymbol{(}\AgdaBound{A}\AgdaSpace{}%
\AgdaSymbol{:}\AgdaSpace{}%
\AgdaBound{𝓤}\AgdaSpace{}%
\AgdaOperator{\AgdaFunction{̇}}\AgdaSpace{}%
\AgdaSymbol{)}\AgdaSpace{}%
\AgdaSymbol{→}\AgdaSpace{}%
\AgdaFunction{Rel}\AgdaSpace{}%
\AgdaBound{A}\AgdaSpace{}%
\AgdaBound{𝓦}\AgdaSpace{}%
\AgdaSymbol{→}\AgdaSpace{}%
\AgdaBound{𝓤}\AgdaSpace{}%
\AgdaOperator{\AgdaPrimitive{⊔}}\AgdaSpace{}%
\AgdaSymbol{(}\AgdaBound{𝓦}\AgdaSpace{}%
\AgdaOperator{\AgdaPrimitive{⁺}}\AgdaSymbol{)}\AgdaSpace{}%
\AgdaOperator{\AgdaFunction{̇}}\<%
\\
%
\>[1]\AgdaBound{A}\AgdaSpace{}%
\AgdaOperator{\AgdaFunction{/}}\AgdaSpace{}%
\AgdaBound{R}\AgdaSpace{}%
\AgdaSymbol{=}\AgdaSpace{}%
\AgdaFunction{Σ}\AgdaSpace{}%
\AgdaBound{C}\AgdaSpace{}%
\AgdaFunction{꞉}\AgdaSpace{}%
\AgdaFunction{Pred}\AgdaSpace{}%
\AgdaBound{A}\AgdaSpace{}%
\AgdaBound{𝓦}\AgdaSpace{}%
\AgdaFunction{,}%
\>[27]\AgdaFunction{𝒞}\AgdaSpace{}%
\AgdaSymbol{\{}\AgdaArgument{R}\AgdaSpace{}%
\AgdaSymbol{=}\AgdaSpace{}%
\AgdaBound{R}\AgdaSymbol{\}}\AgdaSpace{}%
\AgdaBound{C}\<%
\end{code}
\ccpad
The next type is used to represent an `R`-class with a designated representative.
\ccpad
\begin{code}%
\>[0]\AgdaKeyword{module}\AgdaSpace{}%
\AgdaModule{\AgdaUnderscore{}}\AgdaSpace{}%
\AgdaSymbol{\{}\AgdaBound{𝓤}\AgdaSpace{}%
\AgdaBound{𝓦}\AgdaSpace{}%
\AgdaSymbol{:}\AgdaSpace{}%
\AgdaPostulate{Universe}\AgdaSymbol{\}\{}\AgdaBound{A}\AgdaSpace{}%
\AgdaSymbol{:}\AgdaSpace{}%
\AgdaBound{𝓤}\AgdaSpace{}%
\AgdaOperator{\AgdaFunction{̇}}\AgdaSymbol{\}}\AgdaSpace{}%
\AgdaKeyword{where}\<%
\\
%
\\[\AgdaEmptyExtraSkip]%
\>[0][@{}l@{\AgdaIndent{0}}]%
\>[1]\AgdaOperator{\AgdaFunction{⟦\AgdaUnderscore{}⟧}}\AgdaSpace{}%
\AgdaSymbol{:}\AgdaSpace{}%
\AgdaBound{A}\AgdaSpace{}%
\AgdaSymbol{→}\AgdaSpace{}%
\AgdaSymbol{\{}\AgdaBound{R}\AgdaSpace{}%
\AgdaSymbol{:}\AgdaSpace{}%
\AgdaFunction{Rel}\AgdaSpace{}%
\AgdaBound{A}\AgdaSpace{}%
\AgdaBound{𝓦}\AgdaSymbol{\}}\AgdaSpace{}%
\AgdaSymbol{→}\AgdaSpace{}%
\AgdaBound{A}\AgdaSpace{}%
\AgdaOperator{\AgdaFunction{/}}\AgdaSpace{}%
\AgdaBound{R}\<%
\\
%
\>[1]\AgdaOperator{\AgdaFunction{⟦}}\AgdaSpace{}%
\AgdaBound{a}\AgdaSpace{}%
\AgdaOperator{\AgdaFunction{⟧}}\AgdaSpace{}%
\AgdaSymbol{\{}\AgdaBound{R}\AgdaSymbol{\}}\AgdaSpace{}%
\AgdaSymbol{=}\AgdaSpace{}%
\AgdaOperator{\AgdaFunction{[}}\AgdaSpace{}%
\AgdaBound{a}\AgdaSpace{}%
\AgdaOperator{\AgdaFunction{]}}\AgdaSpace{}%
\AgdaBound{R}\AgdaSpace{}%
\AgdaOperator{\AgdaInductiveConstructor{,}}\AgdaSpace{}%
\AgdaBound{a}\AgdaSpace{}%
\AgdaOperator{\AgdaInductiveConstructor{,}}\AgdaSpace{}%
\AgdaInductiveConstructor{refl}\<%
\end{code}
\ccpad
This serves as a kind of \emph{introduction rule}.  Dually, the next type provides an *elimination rule*.\footnote{%
\textbf{Unicode Hint}. Type \af ⌜ and \af ⌝ as \texttt{\textbackslash{}cul} and \texttt{\textbackslash{}cur} in \agdamode.}
\ccpad
\begin{code}%
\>[1]\AgdaOperator{\AgdaFunction{⌜\AgdaUnderscore{}⌝}}\AgdaSpace{}%
\AgdaSymbol{:}\AgdaSpace{}%
\AgdaSymbol{\{}\AgdaBound{R}\AgdaSpace{}%
\AgdaSymbol{:}\AgdaSpace{}%
\AgdaFunction{Rel}\AgdaSpace{}%
\AgdaBound{A}\AgdaSpace{}%
\AgdaBound{𝓦}\AgdaSymbol{\}}\AgdaSpace{}%
\AgdaSymbol{→}\AgdaSpace{}%
\AgdaBound{A}\AgdaSpace{}%
\AgdaOperator{\AgdaFunction{/}}\AgdaSpace{}%
\AgdaBound{R}%
\>[30]\AgdaSymbol{→}\AgdaSpace{}%
\AgdaBound{A}\<%
\\
%
\\[\AgdaEmptyExtraSkip]%
%
\>[1]\AgdaOperator{\AgdaFunction{⌜}}\AgdaSpace{}%
\AgdaBound{𝕔}\AgdaSpace{}%
\AgdaOperator{\AgdaFunction{⌝}}\AgdaSpace{}%
\AgdaSymbol{=}\AgdaSpace{}%
\AgdaFunction{fst}\AgdaSpace{}%
\AgdaOperator{\AgdaFunction{∥}}\AgdaSpace{}%
\AgdaBound{𝕔}\AgdaSpace{}%
\AgdaOperator{\AgdaFunction{∥}}\<%
\end{code}
\scpad

Later we will need the following tools for working with the quotient types defined above.
\ccpad
\begin{code}%
\>[1]\AgdaFunction{/-subset}\AgdaSpace{}%
\AgdaSymbol{:}\AgdaSpace{}%
\AgdaSymbol{\{}\AgdaBound{x}\AgdaSpace{}%
\AgdaBound{y}\AgdaSpace{}%
\AgdaSymbol{:}\AgdaSpace{}%
\AgdaBound{A}\AgdaSymbol{\}\{}\AgdaBound{R}\AgdaSpace{}%
\AgdaSymbol{:}\AgdaSpace{}%
\AgdaFunction{Rel}\AgdaSpace{}%
\AgdaBound{A}\AgdaSpace{}%
\AgdaBound{𝓦}\AgdaSymbol{\}}\AgdaSpace{}%
\AgdaSymbol{→}\AgdaSpace{}%
\AgdaRecord{IsEquivalence}\AgdaSpace{}%
\AgdaBound{R}\AgdaSpace{}%
\AgdaSymbol{→}\AgdaSpace{}%
\AgdaBound{R}\AgdaSpace{}%
\AgdaBound{x}\AgdaSpace{}%
\AgdaBound{y}\AgdaSpace{}%
\AgdaSymbol{→}%
\>[64]\AgdaOperator{\AgdaFunction{[}}\AgdaSpace{}%
\AgdaBound{x}\AgdaSpace{}%
\AgdaOperator{\AgdaFunction{]}}\AgdaSpace{}%
\AgdaBound{R}%
\>[73]\AgdaOperator{\AgdaFunction{⊆}}%
\>[76]\AgdaOperator{\AgdaFunction{[}}\AgdaSpace{}%
\AgdaBound{y}\AgdaSpace{}%
\AgdaOperator{\AgdaFunction{]}}\AgdaSpace{}%
\AgdaBound{R}\<%
\\
%
\>[1]\AgdaFunction{/-subset}\AgdaSpace{}%
\AgdaSymbol{\{}\AgdaBound{x}\AgdaSymbol{\}\{}\AgdaBound{y}\AgdaSymbol{\}}\AgdaSpace{}%
\AgdaBound{Req}\AgdaSpace{}%
\AgdaBound{Rxy}\AgdaSpace{}%
\AgdaSymbol{\{}\AgdaBound{z}\AgdaSymbol{\}}\AgdaSpace{}%
\AgdaBound{Rxz}\AgdaSpace{}%
\AgdaSymbol{=}\AgdaSpace{}%
\AgdaSymbol{(}\AgdaField{trans}\AgdaSpace{}%
\AgdaBound{Req}\AgdaSymbol{)}\AgdaSpace{}%
\AgdaBound{y}\AgdaSpace{}%
\AgdaBound{x}\AgdaSpace{}%
\AgdaBound{z}\AgdaSpace{}%
\AgdaSymbol{(}\AgdaField{sym}\AgdaSpace{}%
\AgdaBound{Req}\AgdaSpace{}%
\AgdaBound{x}\AgdaSpace{}%
\AgdaBound{y}\AgdaSpace{}%
\AgdaBound{Rxy}\AgdaSymbol{)}\AgdaSpace{}%
\AgdaBound{Rxz}\<%
\\
%
\\[\AgdaEmptyExtraSkip]%
%
\>[1]\AgdaFunction{/-supset}\AgdaSpace{}%
\AgdaSymbol{:}\AgdaSpace{}%
\AgdaSymbol{\{}\AgdaBound{x}\AgdaSpace{}%
\AgdaBound{y}\AgdaSpace{}%
\AgdaSymbol{:}\AgdaSpace{}%
\AgdaBound{A}\AgdaSymbol{\}\{}\AgdaBound{R}\AgdaSpace{}%
\AgdaSymbol{:}\AgdaSpace{}%
\AgdaFunction{Rel}\AgdaSpace{}%
\AgdaBound{A}\AgdaSpace{}%
\AgdaBound{𝓦}\AgdaSymbol{\}}\AgdaSpace{}%
\AgdaSymbol{→}\AgdaSpace{}%
\AgdaRecord{IsEquivalence}\AgdaSpace{}%
\AgdaBound{R}\AgdaSpace{}%
\AgdaSymbol{→}\AgdaSpace{}%
\AgdaBound{R}\AgdaSpace{}%
\AgdaBound{x}\AgdaSpace{}%
\AgdaBound{y}\AgdaSpace{}%
\AgdaSymbol{→}%
\>[64]\AgdaOperator{\AgdaFunction{[}}\AgdaSpace{}%
\AgdaBound{y}\AgdaSpace{}%
\AgdaOperator{\AgdaFunction{]}}\AgdaSpace{}%
\AgdaBound{R}\AgdaSpace{}%
\AgdaOperator{\AgdaFunction{⊆}}\AgdaSpace{}%
\AgdaOperator{\AgdaFunction{[}}\AgdaSpace{}%
\AgdaBound{x}\AgdaSpace{}%
\AgdaOperator{\AgdaFunction{]}}\AgdaSpace{}%
\AgdaBound{R}\<%
\\
%
\>[1]\AgdaFunction{/-supset}\AgdaSpace{}%
\AgdaSymbol{\{}\AgdaBound{x}\AgdaSymbol{\}\{}\AgdaBound{y}\AgdaSymbol{\}}\AgdaSpace{}%
\AgdaBound{Req}\AgdaSpace{}%
\AgdaBound{Rxy}\AgdaSpace{}%
\AgdaSymbol{\{}\AgdaBound{z}\AgdaSymbol{\}}\AgdaSpace{}%
\AgdaBound{Ryz}\AgdaSpace{}%
\AgdaSymbol{=}\AgdaSpace{}%
\AgdaSymbol{(}\AgdaField{trans}\AgdaSpace{}%
\AgdaBound{Req}\AgdaSymbol{)}\AgdaSpace{}%
\AgdaBound{x}\AgdaSpace{}%
\AgdaBound{y}\AgdaSpace{}%
\AgdaBound{z}\AgdaSpace{}%
\AgdaBound{Rxy}\AgdaSpace{}%
\AgdaBound{Ryz}\<%
\\
%
\\[\AgdaEmptyExtraSkip]%
%
\>[1]\AgdaFunction{/-≐}\AgdaSpace{}%
\AgdaSymbol{:}\AgdaSpace{}%
\AgdaSymbol{\{}\AgdaBound{x}\AgdaSpace{}%
\AgdaBound{y}\AgdaSpace{}%
\AgdaSymbol{:}\AgdaSpace{}%
\AgdaBound{A}\AgdaSymbol{\}\{}\AgdaBound{R}\AgdaSpace{}%
\AgdaSymbol{:}\AgdaSpace{}%
\AgdaFunction{Rel}\AgdaSpace{}%
\AgdaBound{A}\AgdaSpace{}%
\AgdaBound{𝓦}\AgdaSymbol{\}}\AgdaSpace{}%
\AgdaSymbol{→}\AgdaSpace{}%
\AgdaRecord{IsEquivalence}\AgdaSpace{}%
\AgdaBound{R}\AgdaSpace{}%
\AgdaSymbol{→}\AgdaSpace{}%
\AgdaBound{R}\AgdaSpace{}%
\AgdaBound{x}\AgdaSpace{}%
\AgdaBound{y}\AgdaSpace{}%
\AgdaSymbol{→}%
\>[60]\AgdaOperator{\AgdaFunction{[}}\AgdaSpace{}%
\AgdaBound{x}\AgdaSpace{}%
\AgdaOperator{\AgdaFunction{]}}\AgdaSpace{}%
\AgdaBound{R}%
\>[69]\AgdaOperator{\AgdaFunction{≐}}%
\>[72]\AgdaOperator{\AgdaFunction{[}}\AgdaSpace{}%
\AgdaBound{y}\AgdaSpace{}%
\AgdaOperator{\AgdaFunction{]}}\AgdaSpace{}%
\AgdaBound{R}\<%
\\
%
\>[1]\AgdaFunction{/-≐}\AgdaSpace{}%
\AgdaBound{Req}\AgdaSpace{}%
\AgdaBound{Rxy}\AgdaSpace{}%
\AgdaSymbol{=}\AgdaSpace{}%
\AgdaFunction{/-subset}\AgdaSpace{}%
\AgdaBound{Req}\AgdaSpace{}%
\AgdaBound{Rxy}\AgdaSpace{}%
\AgdaOperator{\AgdaInductiveConstructor{,}}\AgdaSpace{}%
\AgdaFunction{/-supset}\AgdaSpace{}%
\AgdaBound{Req}\AgdaSpace{}%
\AgdaBound{Rxy}\<%
\end{code}


\subsection{Truncation}\label{sec:trunc-sets-prop}
%: continuous propositions, quotient extensionality
\firstsentence{Relations}{Truncation}
% -*- TeX-master: "ualib-part1.tex" -*-
%%% Local Variables: 
%%% mode: latex
%%% TeX-engine: 'xetex
%%% End:
Here we discuss \defn{truncation} and \defn{h-sets} (which we just call \defn{sets}). We first give a brief discussion of standard notions of \emph{truncation} from a viewpoint that seems useful for formalizing mathematics in Agda.\footnote{Readers wishing to learn more about truncation may wish to consult~\cite[\S34]{MHE},~\cite{Brunerie:2012}, or~\cite[\S7.1]{HoTT}.}

\subsubsection*{Uniqueness of identity proofs} %\label{sec:uniq-ident-proofs}
This brief introduction is intended for novices; those already familiar with the concept of \emph{truncation} and \emph{uniqueness of identity proofs} may want to skip to the next subsection.

In general, we may have multiple inhabitants of a given type, hence (via Curry-Howard) multiple proofs of a given proposition. For instance, suppose we have a type \ab{X} and an identity relation \aod{\_≡₀\_} on \ab{X} so that, given two inhabitants of \ab{X}, say, \ab{a} \ab b \as : \ab X, we can form the type \ab{a} \aod{≡₀} \ab b. Suppose \ab{p} and \ab{q} inhabit the type \ab{a} \aod{≡₀} \ab b; that is, \ab{p} and \ab{q} are proofs of \ab{a} \aod{≡₀} \ab b, in which case we write \ab{p} \ab q \as : \ab a \aod{≡₀} \ab b. We might then wonder whether the two proofs \ab{p} and \ab{q} are equivalent.

We are asking about an identity type on the identity type \aod{≡₀}, and whether there is some inhabitant, say, \ab{r} of this type; i.e., whether there is a proof \ab{r} \as : \ab p \aod{≡₁} \ab q that the proofs of \ab{a} \aod{≡₀} \ab{b} are the same. If such a proof exists for all \ab{p} \ab q \as : \ab{a} \aod{≡₀} \ab b, then the proof of \ab a \aod{≡₀} \ab b is unique; as a property of the types \ab{X} and \ad{≡₀}, this is sometimes called \defn{uniqueness of identity proofs}.

Now, perhaps we have two proofs, say, \ab{r} \ab s \as : \ab p \aod{≡₁} \ab q that the proofs \ab{p} and \ab{q} are equivalent. Then of course we wonder whether \ab{r} \aod{≡₂} \ab s has a proof!  But at some level we may decide that the potential to distinguish two proofs of an identity in a meaningful way (so-called \emph{proof-relevance}) is not useful or desirable. At that point, say, at level \ab{k}, we would be naturally inclined to assume that there is at most one proof of any identity of the form \ab{p} \aod{≡ₖ} \ab q. This is called \href{https://www.cs.bham.ac.uk/~mhe/HoTT-UF-in-Agda-Lecture-Notes/HoTT-UF-Agda.html\#truncation}{truncation} (at level \ab{k}).

\paragraph*{Sets} %\label{sec:sets}
In \href{https://homotopytypetheory.org}{homotopy type theory}, a type \ab{X} with an identity relation \ad{≡₀} is called a \defn{set} (or \defn{0-groupoid}) if for every pair \ab{x} \abt{y}{X} there is at most one proof of \ab{x} \aod{≡₀} \ab y. In other words, the type \ab{X}, along with it's equality type \ad{≡ₓ}, form a \emph{set} if for all \ab{x} \abt{y}{X} there is at most one proof of \ab{x} \aod{≡₀} \ab y.

This notion is formalized in \typtop using the type \af{is-set} which is defined using the \af{is-subsingleton} type (\S\ref{sec:inverse-image-invers}) as follows.
\ccpad
\begin{code}%
\>[1]\AgdaFunction{is-set}\AgdaSpace{}%
\AgdaSymbol{:}\AgdaSpace{}%
\AgdaBound{𝓤}\AgdaSpace{}%
\AgdaOperator{\AgdaFunction{̇}}\AgdaSpace{}%
\AgdaSymbol{→}\AgdaSpace{}%
\AgdaBound{𝓤}\AgdaSpace{}%
\AgdaOperator{\AgdaFunction{̇}}\<%
\\
%
\>[1]\AgdaFunction{is-set}\AgdaSpace{}%
\AgdaBound{A}\AgdaSpace{}%
\AgdaSymbol{=}\AgdaSpace{}%
\AgdaSymbol{(}\AgdaBound{x}\AgdaSpace{}%
\AgdaBound{y}\AgdaSpace{}%
\AgdaSymbol{:}\AgdaSpace{}%
\AgdaBound{A}\AgdaSymbol{)}\AgdaSpace{}%
\AgdaSymbol{→}\AgdaSpace{}%
\AgdaFunction{is-subsingleton}\AgdaSpace{}%
\AgdaSymbol{(}\AgdaBound{x}\AgdaSpace{}%
\AgdaOperator{\AgdaDatatype{≡}}\AgdaSpace{}%
\AgdaBound{y}\AgdaSymbol{)}\<%
\end{code}
\ccpad
Thus, the pair (\ab{X} , \ad{≡₀}) forms a set iff it satisfies \as{∀} \ab x \abt{y}{X} \as → \af{is-subsingleton} (\ab x \aod{≡₀} \ab y).

We will also need the function
\href{https://www.cs.bham.ac.uk/~mhe/HoTT-UF-in-Agda-Lecture-Notes/HoTT-UF-Agda.html\#sigmaequality}{to-Σ-≡},
which is part of Escardó's characterization of \emph{equality in Sigma types} in~\cite{MHE} and is defined as follows.
\ccpad
\begin{code}%
\>[1]\AgdaFunction{to-Σ-≡}\AgdaSpace{}%
\AgdaSymbol{:}\AgdaSpace{}%
\>[111I]\AgdaSymbol{\{}\AgdaBound{A}\AgdaSpace{}%
\AgdaSymbol{:}\AgdaSpace{}%
\AgdaBound{𝓤}\AgdaSpace{}%
\AgdaOperator{\AgdaFunction{̇}}%
\AgdaSymbol{\}}\AgdaSpace{}%
\AgdaSymbol{\{}\AgdaBound{B}\AgdaSpace{}%
\AgdaSymbol{:}\AgdaSpace{}%
\AgdaBound{A}\AgdaSpace{}%
\AgdaSymbol{→}\AgdaSpace{}%
\AgdaBound{𝓦}\AgdaSpace{}%
\AgdaOperator{\AgdaFunction{̇}}%
\AgdaSymbol{\}}\AgdaSpace{}%
\AgdaSymbol{\{}\AgdaBound{σ}\AgdaSpace{}%
\AgdaBound{τ}\AgdaSpace{}%
\AgdaSymbol{:}\AgdaSpace{}%
\AgdaRecord{Σ}\AgdaSpace{}%
\AgdaBound{B}\AgdaSymbol{\}}\<%
\\
\>[1][@{}l@{\AgdaIndent{0}}]%
\>[2]\AgdaSymbol{→}%
\>[.][@{}l@{}]\<[111I]%
\>[10]\AgdaFunction{Σ}\AgdaSpace{}%
\AgdaBound{p}\AgdaSpace{}%
\AgdaFunction{꞉}\AgdaSpace{}%
\AgdaOperator{\AgdaFunction{∣}}\AgdaSpace{}%
\AgdaBound{σ}\AgdaSpace{}%
\AgdaOperator{\AgdaFunction{∣}}\AgdaSpace{}%
\AgdaOperator{\AgdaDatatype{≡}}\AgdaSpace{}%
\AgdaOperator{\AgdaFunction{∣}}\AgdaSpace{}%
\AgdaBound{τ}\AgdaSpace{}%
\AgdaOperator{\AgdaFunction{∣}}\AgdaSpace{}%
\AgdaFunction{,}\AgdaSpace{}%
\AgdaSymbol{(}\AgdaFunction{transport}\AgdaSpace{}%
\AgdaBound{B}\AgdaSpace{}%
\AgdaBound{p}\AgdaSpace{}%
\AgdaOperator{\AgdaFunction{∥}}\AgdaSpace{}%
\AgdaBound{σ}\AgdaSpace{}%
\AgdaOperator{\AgdaFunction{∥}}\AgdaSymbol{)}\AgdaSpace{}%
\AgdaOperator{\AgdaDatatype{≡}}\AgdaSpace{}%
\AgdaOperator{\AgdaFunction{∥}}\AgdaSpace{}%
\AgdaBound{τ}\AgdaSpace{}%
\AgdaOperator{\AgdaFunction{∥}}\<%
\\
% \>[.][@{}l@{}]\<[111I]%
% \>[10]\AgdaComment{--------------------------------------------------------------}\<%
% \\
\>[2]\AgdaSymbol{→}%
\>[10]\AgdaBound{σ}\AgdaSpace{}%
\AgdaOperator{\AgdaDatatype{≡}}\AgdaSpace{}%
\AgdaBound{τ}\<%
\\
%
\\[\AgdaEmptyExtraSkip]%
%
\>[1]\AgdaFunction{to-Σ-≡}\AgdaSpace{}%
\AgdaSymbol{(}\AgdaInductiveConstructor{refl}\AgdaSpace{}%
\AgdaSymbol{\{}\AgdaArgument{x}\AgdaSpace{}%
\AgdaSymbol{=}\AgdaSpace{}%
\AgdaBound{x}\AgdaSymbol{\}}\AgdaSpace{}%
\AgdaOperator{\AgdaInductiveConstructor{,}}\AgdaSpace{}%
\AgdaInductiveConstructor{refl}\AgdaSpace{}%
\AgdaSymbol{\{}\AgdaArgument{x}\AgdaSpace{}%
\AgdaSymbol{=}\AgdaSpace{}%
\AgdaBound{a}\AgdaSymbol{\})}\AgdaSpace{}%
\AgdaSymbol{=}\AgdaSpace{}%
\AgdaInductiveConstructor{refl}\AgdaSpace{}%
\AgdaSymbol{\{}\AgdaArgument{x}\AgdaSpace{}%
\AgdaSymbol{=}\AgdaSpace{}%
\AgdaSymbol{(}\AgdaBound{x}\AgdaSpace{}%
\AgdaOperator{\AgdaInductiveConstructor{,}}\AgdaSpace{}%
\AgdaBound{a}\AgdaSymbol{)\}}\<%
\end{code}
\ccpad
We will use \af{is-embedding}, \af{is-set}, and \af{to-Σ-≡} in the next subsection to prove that a monic function into a set is an embedding.

\paragraph*{Injective functions are set embeddings} %\label{injective-functions-are-set-embeddings}
Before moving on to define propositions, we discharge an obligation mentioned but left unfulfilled in the
\href{https://ualib.gitlab.io/Overture.Inverses.html\#embeddings}{embeddings} section of the \ualibhtml{Overture.Inverses} module. Recall, we described and imported the \af{is-embedding} type, and we remarked that an embedding is not simply a monic function. However, if we assume that the codomain is truncated so as to have unique identity proofs, then we can prove that every monic function into that codomain will be an embedding. On the other hand, embeddings are always monic, so we will end up with an equivalence.
% Assume the context contains the following typing judgments: \AgdaSymbol{\{}\AgdaBound{𝓤}\AgdaSpace{}\AgdaBound{𝓦}\AgdaSpace{}%
% \AgdaSymbol{:}\AgdaSpace{}%
% \AgdaPostulate{Universe}\AgdaSymbol{\}\{}\AgdaBound{A}\AgdaSpace{}%
% \AgdaSymbol{:}\AgdaSpace{}%
% \AgdaBound{𝓤}\AgdaSpace{}%
% \AgdaOperator{\AgdaFunction{̇}}\AgdaSymbol{\}\{}\AgdaBound{B}\AgdaSpace{}%
% \AgdaSymbol{:}\AgdaSpace{}%
% \AgdaBound{𝓦}\AgdaSpace{}%
% \AgdaOperator{\AgdaFunction{̇}}\AgdaSymbol{\}}.
\ccpad
\begin{code}%
\>[1]\AgdaFunction{monic-is-embedding|Set}\AgdaSpace{}%
\AgdaSymbol{:}%
\>[143I]\AgdaSymbol{(}\AgdaBound{f}\AgdaSpace{}%
\AgdaSymbol{:}\AgdaSpace{}%
\AgdaBound{A}\AgdaSpace{}%
\AgdaSymbol{→}\AgdaSpace{}%
\AgdaBound{B}\AgdaSymbol{)}\AgdaSpace{}%
\AgdaSymbol{→}\AgdaSpace{}%
\AgdaFunction{is-set}\AgdaSpace{}%
\AgdaBound{B}\AgdaSpace{}%
\AgdaSymbol{→}\AgdaSpace{}%
\AgdaFunction{Monic}\AgdaSpace{}%
\AgdaBound{f}\AgdaSpace{}%
\AgdaSymbol{→}\AgdaSpace{}%
\AgdaFunction{is-embedding}\AgdaSpace{}%
\AgdaBound{f}\<%
\\
%
\\[\AgdaEmptyExtraSkip]%
%
\>[1]\AgdaFunction{monic-is-embedding|Set}\AgdaSpace{}%
\AgdaBound{f}\AgdaSpace{}%
\AgdaBound{Bset}\AgdaSpace{}%
\AgdaBound{fmon}\AgdaSpace{}%
\AgdaBound{b}\AgdaSpace{}%
\AgdaSymbol{(}\AgdaBound{u}\AgdaSpace{}%
\AgdaOperator{\AgdaInductiveConstructor{,}}\AgdaSpace{}%
\AgdaBound{fu≡b}\AgdaSymbol{)}\AgdaSpace{}%
\AgdaSymbol{(}\AgdaBound{v}\AgdaSpace{}%
\AgdaOperator{\AgdaInductiveConstructor{,}}\AgdaSpace{}%
\AgdaBound{fv≡b}\AgdaSymbol{)}\AgdaSpace{}%
\AgdaSymbol{=}\AgdaSpace{}%
\AgdaFunction{γ}\<%
\\
\>[1][@{}l@{\AgdaIndent{0}}]%
\>[2]\AgdaKeyword{where}\<%
\\
%
\>[2]\AgdaFunction{fuv}\AgdaSpace{}%
\AgdaSymbol{:}\AgdaSpace{}%
\AgdaBound{f}\AgdaSpace{}%
\AgdaBound{u}\AgdaSpace{}%
\AgdaOperator{\AgdaDatatype{≡}}\AgdaSpace{}%
\AgdaBound{f}\AgdaSpace{}%
\AgdaBound{v}\<%
\\
%
\>[2]\AgdaFunction{fuv}\AgdaSpace{}%
\AgdaSymbol{=}\AgdaSpace{}%
\AgdaFunction{≡-trans}\AgdaSpace{}%
\AgdaBound{fu≡b}\AgdaSpace{}%
\AgdaSymbol{(}\AgdaBound{fv≡b}\AgdaSpace{}%
\AgdaOperator{\AgdaFunction{⁻¹}}\AgdaSymbol{)}\<%
\\
%
\\[\AgdaEmptyExtraSkip]%
%
\>[2]\AgdaFunction{uv}\AgdaSpace{}%
\AgdaSymbol{:}\AgdaSpace{}%
\AgdaBound{u}\AgdaSpace{}%
\AgdaOperator{\AgdaDatatype{≡}}\AgdaSpace{}%
\AgdaBound{v}\<%
\\
%
\>[2]\AgdaFunction{uv}\AgdaSpace{}%
\AgdaSymbol{=}\AgdaSpace{}%
\AgdaBound{fmon}\AgdaSpace{}%
\AgdaBound{u}\AgdaSpace{}%
\AgdaBound{v}\AgdaSpace{}%
\AgdaFunction{fuv}\<%
\\
%
\\[\AgdaEmptyExtraSkip]%
%
\>[2]\AgdaFunction{arg1}\AgdaSpace{}%
\AgdaSymbol{:}\AgdaSpace{}%
\AgdaFunction{Σ}\AgdaSpace{}%
\AgdaBound{p}\AgdaSpace{}%
\AgdaFunction{꞉}\AgdaSpace{}%
\AgdaSymbol{(}\AgdaBound{u}\AgdaSpace{}%
\AgdaOperator{\AgdaDatatype{≡}}\AgdaSpace{}%
\AgdaBound{v}\AgdaSymbol{)}\AgdaSpace{}%
\AgdaFunction{,}\AgdaSpace{}%
\AgdaSymbol{(}\AgdaFunction{transport}\AgdaSpace{}%
\AgdaSymbol{(λ}\AgdaSpace{}%
\AgdaBound{a}\AgdaSpace{}%
\AgdaSymbol{→}\AgdaSpace{}%
\AgdaBound{f}\AgdaSpace{}%
\AgdaBound{a}\AgdaSpace{}%
\AgdaOperator{\AgdaDatatype{≡}}\AgdaSpace{}%
\AgdaBound{b}\AgdaSymbol{)}\AgdaSpace{}%
\AgdaBound{p}\AgdaSpace{}%
\AgdaBound{fu≡b}\AgdaSymbol{)}\AgdaSpace{}%
\AgdaOperator{\AgdaDatatype{≡}}\AgdaSpace{}%
\AgdaBound{fv≡b}\<%
\\
%
\>[2]\AgdaFunction{arg1}\AgdaSpace{}%
\AgdaSymbol{=}\AgdaSpace{}%
\AgdaFunction{uv}\AgdaSpace{}%
\AgdaOperator{\AgdaInductiveConstructor{,}}\AgdaSpace{}%
\AgdaBound{Bset}\AgdaSpace{}%
\AgdaSymbol{(}\AgdaBound{f}\AgdaSpace{}%
\AgdaBound{v}\AgdaSymbol{)}\AgdaSpace{}%
\AgdaBound{b}\AgdaSpace{}%
\AgdaSymbol{(}\AgdaFunction{transport}\AgdaSpace{}%
\AgdaSymbol{(λ}\AgdaSpace{}%
\AgdaBound{a}\AgdaSpace{}%
\AgdaSymbol{→}\AgdaSpace{}%
\AgdaBound{f}\AgdaSpace{}%
\AgdaBound{a}\AgdaSpace{}%
\AgdaOperator{\AgdaDatatype{≡}}\AgdaSpace{}%
\AgdaBound{b}\AgdaSymbol{)}\AgdaSpace{}%
\AgdaFunction{uv}\AgdaSpace{}%
\AgdaBound{fu≡b}\AgdaSymbol{)}\AgdaSpace{}%
\AgdaBound{fv≡b}\<%
\\
%
\\[\AgdaEmptyExtraSkip]%
%
\>[2]\AgdaFunction{γ}\AgdaSpace{}%
\AgdaSymbol{:}\AgdaSpace{}%
\AgdaBound{u}\AgdaSpace{}%
\AgdaOperator{\AgdaInductiveConstructor{,}}\AgdaSpace{}%
\AgdaBound{fu≡b}\AgdaSpace{}%
\AgdaOperator{\AgdaDatatype{≡}}\AgdaSpace{}%
\AgdaBound{v}\AgdaSpace{}%
\AgdaOperator{\AgdaInductiveConstructor{,}}\AgdaSpace{}%
\AgdaBound{fv≡b}\<%
\\
%
\>[2]\AgdaFunction{γ}\AgdaSpace{}%
\AgdaSymbol{=}\AgdaSpace{}%
\AgdaFunction{to-Σ-≡}\AgdaSpace{}%
\AgdaFunction{arg1}\<%
\end{code}
\ccpad
In stating the previous result, we introduce a new convention to which we will try to adhere. If the antecedent of a theorem includes the assumption that one of the types involved is a set, then we add to the name of the theorem the suffix \af{\textbar{}sets}, which calls to mind the standard notation for the restriction of a function to a subset of its domain.

Embeddings are always monic, so we conclude that when a function's codomain is a set, then that function is an embedding if and only if it is monic.
\ccpad
\begin{code}%
\>[0][@{}l@{\AgdaIndent{1}}]%
\>[1]\AgdaFunction{embedding-iff-monic|Set}\AgdaSpace{}%
\AgdaSymbol{:}%
\>[234I]\AgdaSymbol{(}\AgdaBound{f}\AgdaSpace{}%
\AgdaSymbol{:}\AgdaSpace{}%
\AgdaBound{A}\AgdaSpace{}%
\AgdaSymbol{→}\AgdaSpace{}%
\AgdaBound{B}\AgdaSymbol{)}\AgdaSpace{}%
\AgdaSymbol{→}\AgdaSpace{}%
\AgdaFunction{is-set}\AgdaSpace{}%
\AgdaBound{B}\AgdaSpace{}%
\AgdaSymbol{→}\AgdaSpace{}%
\AgdaFunction{is-embedding}\AgdaSpace{}%
\AgdaBound{f}\AgdaSpace{}%
\AgdaOperator{\AgdaFunction{⇔}}\AgdaSpace{}%
\AgdaFunction{Monic}\AgdaSpace{}%
\AgdaBound{f}\<%
\\
\>[1]\AgdaFunction{embedding-iff-monic|Set}\AgdaSpace{}%
\AgdaBound{f}\AgdaSpace{}%
\AgdaBound{Bset}\AgdaSpace{}%
\AgdaSymbol{=}\AgdaSpace{}%
\AgdaSymbol{(}\AgdaFunction{embedding-is-monic}\AgdaSpace{}%
\AgdaBound{f}\AgdaSymbol{)}\AgdaOperator{\AgdaInductiveConstructor{,}}\AgdaSpace{}%
\AgdaSymbol{(}\AgdaFunction{monic-is-embedding|Set}\AgdaSpace{}%
\AgdaBound{f}\AgdaSpace{}%
\AgdaBound{Bset}\AgdaSymbol{)}\<%
\end{code}


\subsection{Relation extensionality}\label{sec:relation-extensionality}
%: continuous propositions, quotient extensionality
\firstsentence{Relations}{Extensionality}
\input{35Extensionality}

% % -*- TeX-master: "ualib-part1.tex" -*-
%%% Local Variables: 
%%% mode: latex
%%% TeX-engine: 'xetex
%%% End:
\section{Algebra Types}\label{sec:algebra-types}
Finally we are ready to present the \ualibhtml{Algebras} module of the \agdaualib. Here we use type theory and Agda to codify the most basic objects of universal algebra, such as \emph{operations} and \emph{signatures} (\S\ref{sec:oper-sign}), \emph{algebras} (\S\ref{sec:algebras}), \emph{product algebras} (\S\ref{sec:product-algebras}), \emph{congruence relations} and \emph{quotient algebras}~(\S\ref{congruences}).

A popular way to represent algebraic structures in type theory is with record types. The Sigma type provides an equivalent alternative that we happen to prefer and we use it throughout the library, both for consistency and because of its direct connection to the existential quantifier of logic. Recall that inhabitants of the type \ad Σ~\ab x~\af ꞉~\ab X~\af ,~\ab P are pairs (\ab x,~\ab p) such that \abt{x}{X} and \ab p~\as : \ab P~\ab x. In this sense, when such a Sigma type is inhabited we conclude, ``there exists \ab x in \ab X such that \ab P~\ab x holds;'' in symbols, \as ∃~\ab x~\af ∈~\ab X~\af ,~\ab P~\ab x.  % Indeed, an inhabitant of  \ad Σ~\ab x~\af ꞉~\ab X~\af ,~\ab P~\ab x is a pair (\ab x , \ab p) such that \ab x inhabits \ab X and \ab p is a proof of \ab P \ab x.
Moreover, the pair (\ab x,~\ab p) is not merely a proof of the logical sentence \as ∃~\ab x~\af ∈~\ab X~\af ,~\ab P~\ab x; it is also a \emph{witness} of the truth of this sentence. We sometimes say that a proof of an existentially quantified sentence has ``computational content'' if it provides such a witness, or a function that can extract a witness from the proof.

\subsection{Signatures: types for operations \& signatures}\label{sec:oper-sign}
\firstsentence{Algebras}{Signatures}
% -*- TeX-master: "ualib-part1.tex" -*-
%%% Local Variables: 
%%% mode: latex
%%% TeX-engine: 'xetex
%%% End:
% \subsubsection{Operation type}\label{operation-type}
% We begin by defining the type of \defn{operations} as follows.
% \ccpad
% \begin{code}%
% \>[0][@{}l@{\AgdaIndent{0}}]%
% \>[1]\AgdaFunction{Op}\AgdaSpace{}%
% \AgdaSymbol{:}\AgdaSpace{}%
% \AgdaGeneralizable{𝓥}\AgdaSpace{}%
% \AgdaOperator{\AgdaFunction{̇}}\AgdaSpace{}%
% \AgdaSymbol{→}\AgdaSpace{}%
% \AgdaBound{𝓤}\AgdaSpace{}%
% \AgdaOperator{\AgdaFunction{̇}}\AgdaSpace{}%
% \AgdaSymbol{→}\AgdaSpace{}%
% \AgdaBound{𝓤}\AgdaSpace{}%
% \AgdaOperator{\AgdaPrimitive{⊔}}\AgdaSpace{}%
% \AgdaGeneralizable{𝓥}\AgdaSpace{}%
% \AgdaOperator{\AgdaFunction{̇}}\<%
% \\
% %
% \>[1]\AgdaFunction{Op}\AgdaSpace{}%
% \AgdaBound{I}\AgdaSpace{}%
% \AgdaBound{A}\AgdaSpace{}%
% \AgdaSymbol{=}\AgdaSpace{}%
% \AgdaSymbol{(}\AgdaBound{I}\AgdaSpace{}%
% \AgdaSymbol{→}\AgdaSpace{}%
% \AgdaBound{A}\AgdaSymbol{)}\AgdaSpace{}%
% \AgdaSymbol{→}\AgdaSpace{}%
% \AgdaBound{A}\<%
% \end{code}
% \ccpad
% The type \af{Op} encodes the arity of an operation as an arbitrary type \abt{I}{𝓥}, which gives us a very general way to represent an operation as a function type with domain \ab I \as → \ab A (the type of ``tuples'') and codomain \ab{A}. For example, the \ab{I}-\defn{ary projection operations} on \ab{A} can be codified as inhabitants of the type \af{Op} \ab I \af A in the following way.
% \ccpad
% \begin{code}%
% \>[0]\AgdaFunction{π}\AgdaSpace{}%
% \AgdaSymbol{:}\AgdaSpace{}%
% \AgdaSymbol{\{}\AgdaBound{I}\AgdaSpace{}%
% \AgdaSymbol{:}\AgdaSpace{}%
% \AgdaGeneralizable{𝓥}\AgdaSpace{}%
% \AgdaOperator{\AgdaFunction{̇}}%
% \AgdaSymbol{\}}\AgdaSpace{}%
% \AgdaSymbol{\{}\AgdaBound{A}\AgdaSpace{}%
% \AgdaSymbol{:}\AgdaSpace{}%
% \AgdaGeneralizable{𝓤}\AgdaSpace{}%
% \AgdaOperator{\AgdaFunction{̇}}\AgdaSpace{}%
% \AgdaSymbol{\}}\AgdaSpace{}%
% \AgdaSymbol{→}\AgdaSpace{}%
% \AgdaBound{I}\AgdaSpace{}%
% \AgdaSymbol{→}\AgdaSpace{}%
% \AgdaFunction{Op}\AgdaSpace{}%
% \AgdaBound{I}\AgdaSpace{}%
% \AgdaBound{A}\<%
% \\
% \>[0]\AgdaFunction{π}\AgdaSpace{}%
% \AgdaBound{i}\AgdaSpace{}%
% \AgdaBound{x}\AgdaSpace{}%
% \AgdaSymbol{=}\AgdaSpace{}%
% \AgdaBound{x}\AgdaSpace{}%
% \AgdaBound{i}\<%
% \end{code}

% \subsubsection{Signature type}\label{signature-type}

We define the signature of an algebraic structure in Agda like this.
\ccpad
\begin{code}%
\>[0]\AgdaFunction{Signature}\AgdaSpace{}%
\AgdaSymbol{:}\AgdaSpace{}%
\AgdaSymbol{(}\AgdaBound{𝓞}\AgdaSpace{}%
\AgdaBound{𝓥}\AgdaSpace{}%
\AgdaSymbol{:}\AgdaSpace{}%
\AgdaPostulate{Universe}\AgdaSymbol{)}\AgdaSpace{}%
\AgdaSymbol{→}\AgdaSpace{}%
\AgdaSymbol{(}\AgdaBound{𝓞}\AgdaSpace{}%
\AgdaOperator{\AgdaPrimitive{⊔}}\AgdaSpace{}%
\AgdaBound{𝓥}\AgdaSymbol{)}\AgdaSpace{}%
\AgdaOperator{\AgdaPrimitive{⁺}}\AgdaSpace{}%
\AgdaOperator{\AgdaFunction{̇}}\<%
\\
\>[0]\AgdaFunction{Signature}\AgdaSpace{}%
\AgdaBound{𝓞}\AgdaSpace{}%
\AgdaBound{𝓥}\AgdaSpace{}%
\AgdaSymbol{=}\AgdaSpace{}%
\AgdaFunction{Σ}\AgdaSpace{}%
\AgdaBound{F}\AgdaSpace{}%
\AgdaFunction{꞉}\AgdaSpace{}%
\AgdaBound{𝓞}\AgdaSpace{}%
\AgdaOperator{\AgdaFunction{̇}}\AgdaSpace{}%
\AgdaFunction{,}\AgdaSpace{}%
\AgdaSymbol{(}\AgdaBound{F}\AgdaSpace{}%
\AgdaSymbol{→}\AgdaSpace{}%
\AgdaBound{𝓥}\AgdaSpace{}%
\AgdaOperator{\AgdaFunction{̇}}\AgdaSymbol{)}\<%
\end{code}
\ccpad
As mentioned in \S\ref{sec:discrete-relations}, the symbol 𝓞 always denotes the universe of \emph{operation symbol} types, while 𝓥 is always the universe of \emph{arity} types.

In \S\ref{preliminaries} we defined special syntax for the first and second projections---namely, \af ∣\au\af ∣ and \af{∥}\au\af{∥}, respectively. Consequently, if \ab 𝑆 \as : \af{Signature} \ab 𝓞 \ab 𝓥 is a signature, then \af ∣~\ab 𝑆~\af ∣ denotes the set of operation symbols, and \af{∥}~\ab 𝑆~\af{∥} denotes the arity function. If \ab 𝑓 \as : \af ∣~\ab 𝑆~\af ∣ is an operation symbol in the signature \ab 𝑆, then \af{∥}~\ab 𝑆~\af{∥} \ab 𝑓 is the arity of \ab 𝑓.

% For reference, we recall the definition of the Sigma type, \af{Σ}, which is

% \begin{Shaded}
% \begin{Highlighting}[]
% \NormalTok{-Σ }\OtherTok{:} \OtherTok{\{}\NormalTok{𝓤 𝓥 }\OtherTok{:}\NormalTok{ Universe}\OtherTok{\}(}\NormalTok{X }\OtherTok{:}\NormalTok{ 𝓤 ̇}\OtherTok{)(}\NormalTok{Y }\OtherTok{:}\NormalTok{ X }\OtherTok{→}\NormalTok{ 𝓥 ̇}\OtherTok{)} \OtherTok{→}\NormalTok{ 𝓤 ⊔ 𝓥 ̇}
% \NormalTok{-Σ X Y }\OtherTok{=}\NormalTok{ Σ Y}
% \end{Highlighting}
% \end{Shaded}

\paragraph*{Example of a signature} %\label{example}
Here is how we could define the signature for \defn{monoids} as an inhabitant of the type \af{Signature} \ab 𝓞 \ab 𝓥.
\ccpad
\begin{code}%
\>[1]\AgdaKeyword{data}\AgdaSpace{}%
\AgdaDatatype{monoid-op}\AgdaSpace{}%
\AgdaSymbol{:}\AgdaSpace{}%
\AgdaBound{𝓞}\AgdaSpace{}%
\AgdaOperator{\AgdaFunction{̇}}\AgdaSpace{}%
\AgdaKeyword{where}\<%
\\
\>[1][@{}l@{\AgdaIndent{0}}]%
\>[2]\AgdaInductiveConstructor{e}\AgdaSpace{}%
\AgdaSymbol{:}\AgdaSpace{}%
\AgdaDatatype{monoid-op}\<%
\\
%
\>[2]\AgdaInductiveConstructor{·}\AgdaSpace{}%
\AgdaSymbol{:}\AgdaSpace{}%
\AgdaDatatype{monoid-op}\<%
\\
%
\\[\AgdaEmptyExtraSkip]%
%
\>[1]\AgdaFunction{monoid-sig}\AgdaSpace{}%
\AgdaSymbol{:}\AgdaSpace{}%
\AgdaFunction{Signature}\AgdaSpace{}%
\AgdaBound{𝓞}\AgdaSpace{}%
\AgdaPrimitive{𝓤₀}\<%
\\
%
\>[1]\AgdaFunction{monoid-sig}\AgdaSpace{}%
\AgdaSymbol{=}\AgdaSpace{}%
\AgdaDatatype{monoid-op}\AgdaSpace{}%
\AgdaOperator{\AgdaInductiveConstructor{,}}\AgdaSpace{}%
\AgdaSymbol{λ}\AgdaSpace{}%
\AgdaSymbol{\{}\AgdaSpace{}%
\AgdaInductiveConstructor{e}\AgdaSpace{}%
\AgdaSymbol{→}\AgdaSpace{}%
\AgdaFunction{𝟘}\AgdaSymbol{;}\AgdaSpace{}%
\AgdaInductiveConstructor{·}\AgdaSpace{}%
\AgdaSymbol{→}\AgdaSpace{}%
\AgdaFunction{𝟚}\AgdaSpace{}%
\AgdaSymbol{\}}\<%
\end{code}
\ccpad
Thus, the signature for a monoid consists of two operation symbols, \aic{e} and \aic{·}, and a function
\as λ \as{\{} \aic e \as → \af 𝟘 \as ; \aic · \as → \af 𝟚 \as{\}} which maps \aic{e} to the empty type \af 𝟘 (since \aic{e} is the nullary identity) and maps \aic{·} to the two element type \af 𝟚 (since \aic{·} is binary).\footnote{The types \af 𝟘 and \af 𝟚 are defined in the \AgdaModule{MGS-MLTT} module of \typtop.}


\subsection{Algebras: types for algebras, operation interpretation \& compatibility}\label{sec:algebras}
\firstsentence{Algebras}{Algebras}
% -*- TeX-master: "ualib-part1.tex" -*-
%%% Local Variables: 
%%% mode: latex
%%% TeX-engine: 'xetex
%%% End:
% \subsubsection{The Algebra type}\label{algebra-types}
For a fixed signature \ab{𝑆} \as : \af{Signature} \ab 𝓞 \ab 𝓥 and universe \ab 𝓤, we define the type of \defn{algebras in the signature} \ab 𝑆 (or \ab 𝑆-\defn{algebras}) with \defn{domain} (or \defn{carrier} or \defn{universe}) \abt{𝐴}{𝓤} as follows.
\ccpad
\begin{code}%
\>[0]\AgdaFunction{Algebra}\AgdaSpace{}%
\AgdaSymbol{:}\AgdaSpace{}%
\AgdaSymbol{(}\AgdaBound{𝓤}\AgdaSpace{}%
\AgdaSymbol{:}\AgdaSpace{}%
\AgdaPostulate{Universe}\AgdaSymbol{)(}\AgdaBound{𝑆}\AgdaSpace{}%
\AgdaSymbol{:}\AgdaSpace{}%
\AgdaFunction{Signature}\AgdaSpace{}%
\AgdaGeneralizable{𝓞}\AgdaSpace{}%
\AgdaGeneralizable{𝓥}\AgdaSymbol{)}\AgdaSpace{}%
\AgdaSymbol{→}\AgdaSpace{}%
\AgdaGeneralizable{𝓞}\AgdaSpace{}%
\AgdaOperator{\AgdaPrimitive{⊔}}\AgdaSpace{}%
\AgdaGeneralizable{𝓥}\AgdaSpace{}%
\AgdaOperator{\AgdaPrimitive{⊔}}\AgdaSpace{}%
\AgdaBound{𝓤}\AgdaSpace{}%
\AgdaOperator{\AgdaPrimitive{⁺}}\AgdaSpace{}%
\AgdaOperator{\AgdaFunction{̇}}\<%
\\
%
\\[\AgdaEmptyExtraSkip]%
\>[0]\AgdaFunction{Algebra}\AgdaSpace{}%
\AgdaBound{𝓤}\AgdaSpace{}%
\AgdaBound{𝑆}\AgdaSpace{}%
\AgdaSymbol{=}%
\>[29I]\AgdaFunction{Σ}\AgdaSpace{}%
\AgdaBound{A}\AgdaSpace{}%
\AgdaFunction{꞉}\AgdaSpace{}%
\AgdaBound{𝓤}\AgdaSpace{}%
\AgdaOperator{\AgdaFunction{̇}}\AgdaSpace{}%
\AgdaFunction{,}%
\>[50]\AgdaComment{-- the domain}\<%
\\
\>[.][@{}l@{}]\<[29I]%
\>[14]\AgdaFunction{Π}\AgdaSpace{}%
\AgdaBound{f}\AgdaSpace{}%
\AgdaFunction{꞉}\AgdaSpace{}%
\AgdaOperator{\AgdaFunction{∣}}\AgdaSpace{}%
\AgdaBound{𝑆}\AgdaSpace{}%
\AgdaOperator{\AgdaFunction{∣}}\AgdaSpace{}%
\AgdaFunction{,}\AgdaSpace{}%
\AgdaFunction{Op}\AgdaSpace{}%
\AgdaSymbol{(}\AgdaOperator{\AgdaFunction{∥}}\AgdaSpace{}%
\AgdaBound{𝑆}\AgdaSpace{}%
\AgdaOperator{\AgdaFunction{∥}}\AgdaSpace{}%
\AgdaBound{f}\AgdaSymbol{)}\AgdaSpace{}%
\AgdaBound{A}%
\>[50]\AgdaComment{-- the basic operations}\<%
\end{code}
\ccpad
To be more precise, we might call an inhabitant of this type an ``∞-algebra'' because its domain can be an arbitrary type, say, \abt{A}{𝓤} and need not be truncated at some level. (In particular, \ab{A} need not be a \emph{set}; see the discussion of sets in~\S~\ref{sec:trunc-sets-prop}.) We could then proceed to define the type of ``0-algebras'' as algebras whose domains are sets (which may be closer to what most of us have in mind when doing informal universal algebra). However, we have found that the domains of our algebras need to be sets in just a few places in the \ualib, and it seems preferable to work with general (∞-)algebras throughout and then assume \emph{uniqueness of identity proofs} (uip) explicitly and only where needed.  This makes any dependence on uip more transparent (which is also the reason \AgdaPragma{--without-K} appears at the top of every module in the \ualib).

\paragraph*{Algebras as record types} %\label{algebras-as-record-types}

Some people prefer to represent algebraic structures in type theory using records, and for those folks we offer the following (equivalent) definition.
\ccpad
\begin{code}%
\>[0]\AgdaKeyword{record}\AgdaSpace{}%
\AgdaRecord{algebra}\AgdaSpace{}%
\AgdaSymbol{(}\AgdaBound{𝓤}\AgdaSpace{}%
\AgdaSymbol{:}\AgdaSpace{}%
\AgdaPostulate{Universe}\AgdaSymbol{)}\AgdaSpace{}%
\AgdaSymbol{(}\AgdaBound{𝑆}\AgdaSpace{}%
\AgdaSymbol{:}\AgdaSpace{}%
\AgdaFunction{Signature}\AgdaSpace{}%
\AgdaGeneralizable{𝓞}\AgdaSpace{}%
\AgdaGeneralizable{𝓥}\AgdaSymbol{)}\AgdaSpace{}%
\AgdaSymbol{:}\AgdaSpace{}%
\AgdaSymbol{(}\AgdaBound{𝓞}\AgdaSpace{}%
\AgdaOperator{\AgdaPrimitive{⊔}}\AgdaSpace{}%
\AgdaBound{𝓥}\AgdaSpace{}%
\AgdaOperator{\AgdaPrimitive{⊔}}\AgdaSpace{}%
\AgdaBound{𝓤}\AgdaSymbol{)}\AgdaSpace{}%
\AgdaOperator{\AgdaPrimitive{⁺}}\AgdaSpace{}%
\AgdaOperator{\AgdaFunction{̇}}\AgdaSpace{}%
\AgdaKeyword{where}\<%
\\
\>[0][@{}l@{\AgdaIndent{0}}]%
\>[1]\AgdaKeyword{constructor}\AgdaSpace{}%
\AgdaInductiveConstructor{mkalg}\<%
\\
%
\>[1]\AgdaKeyword{field}\<%
\\
\>[1][@{}l@{\AgdaIndent{0}}]%
\>[2]\AgdaField{univ}\AgdaSpace{}%
\AgdaSymbol{:}\AgdaSpace{}%
\AgdaBound{𝓤}\AgdaSpace{}%
\AgdaOperator{\AgdaFunction{̇}}\<%
\\
%
\>[2]\AgdaField{op}\AgdaSpace{}%
\AgdaSymbol{:}\AgdaSpace{}%
\AgdaSymbol{(}\AgdaBound{f}\AgdaSpace{}%
\AgdaSymbol{:}\AgdaSpace{}%
\AgdaOperator{\AgdaFunction{∣}}\AgdaSpace{}%
\AgdaBound{𝑆}\AgdaSpace{}%
\AgdaOperator{\AgdaFunction{∣}}\AgdaSymbol{)}\AgdaSpace{}%
\AgdaSymbol{→}\AgdaSpace{}%
\AgdaSymbol{((}\AgdaOperator{\AgdaFunction{∥}}\AgdaSpace{}%
\AgdaBound{𝑆}\AgdaSpace{}%
\AgdaOperator{\AgdaFunction{∥}}\AgdaSpace{}%
\AgdaBound{f}\AgdaSymbol{)}\AgdaSpace{}%
\AgdaSymbol{→}\AgdaSpace{}%
\AgdaField{univ}\AgdaSymbol{)}\AgdaSpace{}%
\AgdaSymbol{→}\AgdaSpace{}%
\AgdaField{univ}\<%
\end{code}
\ccpad
This representation of algebras as inhabitants of a record type is equivalent to the representation using Sigma types in the sense that there is a bi-implication between the two representations. The proofs are immediate, so we omit them (see the \ualibhtml{Algebras.Algebras} module for details).
% \ccpad
% \begin{code}%
% \>[1]\AgdaFunction{algebra→Algebra}\AgdaSpace{}%
% \AgdaSymbol{:}\AgdaSpace{}%
% \AgdaRecord{algebra}\AgdaSpace{}%
% \AgdaGeneralizable{𝓤}\AgdaSpace{}%
% \AgdaBound{𝑆}\AgdaSpace{}%
% \AgdaSymbol{→}\AgdaSpace{}%
% \AgdaFunction{Algebra}\AgdaSpace{}%
% \AgdaGeneralizable{𝓤}\AgdaSpace{}%
% \AgdaBound{𝑆}\<%
% \\
% %
% \>[1]\AgdaFunction{algebra→Algebra}\AgdaSpace{}%
% \AgdaBound{𝑨}\AgdaSpace{}%
% \AgdaSymbol{=}\AgdaSpace{}%
% \AgdaSymbol{(}\AgdaField{univ}\AgdaSpace{}%
% \AgdaBound{𝑨}\AgdaSpace{}%
% \AgdaOperator{\AgdaInductiveConstructor{,}}\AgdaSpace{}%
% \AgdaField{op}\AgdaSpace{}%
% \AgdaBound{𝑨}\AgdaSymbol{)}\<%
% \\
% %
% \\[\AgdaEmptyExtraSkip]%
% %
% \>[1]\AgdaFunction{Algebra→algebra}\AgdaSpace{}%
% \AgdaSymbol{:}\AgdaSpace{}%
% \AgdaFunction{Algebra}\AgdaSpace{}%
% \AgdaGeneralizable{𝓤}\AgdaSpace{}%
% \AgdaBound{𝑆}\AgdaSpace{}%
% \AgdaSymbol{→}\AgdaSpace{}%
% \AgdaRecord{algebra}\AgdaSpace{}%
% \AgdaGeneralizable{𝓤}\AgdaSpace{}%
% \AgdaBound{𝑆}\<%
% \\
% %
% \>[1]\AgdaFunction{Algebra→algebra}\AgdaSpace{}%
% \AgdaBound{𝑨}\AgdaSpace{}%
% \AgdaSymbol{=}\AgdaSpace{}%
% \AgdaInductiveConstructor{mkalg}\AgdaSpace{}%
% \AgdaOperator{\AgdaFunction{∣}}\AgdaSpace{}%
% \AgdaBound{𝑨}\AgdaSpace{}%
% \AgdaOperator{\AgdaFunction{∣}}\AgdaSpace{}%
% \AgdaOperator{\AgdaFunction{∥}}\AgdaSpace{}%
% \AgdaBound{𝑨}\AgdaSpace{}%
% \AgdaOperator{\AgdaFunction{∥}}\<%
% \end{code}
% \ccpad

In developing the more advanced modules of the \agdaualib, we have used the Sigma type representation of algebras exclusively, though we occasionally use record types to represent other basic objects (congruences, subalgebras, and others).



\paragraph*{Operation interpretation syntax}\label{operation-interpretation-syntax}

We now define a convenient shorthand for the interpretation of an operation symbol. This looks more similar to the standard notation one finds in the literature as compared to the double bar notation we started with, so we will use this new notation almost exclusively in the remaining modules of the \ualib.
\ccpad
\begin{code}%
\>[1]\AgdaOperator{\AgdaFunction{\AgdaUnderscore{}̂\AgdaUnderscore{}}}\AgdaSpace{}%
\AgdaSymbol{:}\AgdaSpace{}%
\AgdaSymbol{(}\AgdaBound{𝑓}\AgdaSpace{}%
\AgdaSymbol{:}\AgdaSpace{}%
\AgdaOperator{\AgdaFunction{∣}}\AgdaSpace{}%
\AgdaBound{𝑆}\AgdaSpace{}%
\AgdaOperator{\AgdaFunction{∣}}\AgdaSymbol{)(}\AgdaBound{𝑨}\AgdaSpace{}%
\AgdaSymbol{:}\AgdaSpace{}%
\AgdaFunction{Algebra}\AgdaSpace{}%
\AgdaBound{𝓤}\AgdaSpace{}%
\AgdaBound{𝑆}\AgdaSymbol{)}\AgdaSpace{}%
\AgdaSymbol{→}\AgdaSpace{}%
\AgdaSymbol{(}\AgdaOperator{\AgdaFunction{∥}}\AgdaSpace{}%
\AgdaBound{𝑆}\AgdaSpace{}%
\AgdaOperator{\AgdaFunction{∥}}\AgdaSpace{}%
\AgdaBound{𝑓}%
\>[48]\AgdaSymbol{→}%
\>[51]\AgdaOperator{\AgdaFunction{∣}}\AgdaSpace{}%
\AgdaBound{𝑨}\AgdaSpace{}%
\AgdaOperator{\AgdaFunction{∣}}\AgdaSymbol{)}\AgdaSpace{}%
\AgdaSymbol{→}\AgdaSpace{}%
\AgdaOperator{\AgdaFunction{∣}}\AgdaSpace{}%
\AgdaBound{𝑨}\AgdaSpace{}%
\AgdaOperator{\AgdaFunction{∣}}\<%
\\
%
\\[\AgdaEmptyExtraSkip]%
%
\>[1]\AgdaBound{𝑓}\AgdaSpace{}%
\AgdaOperator{\AgdaFunction{̂}}\AgdaSpace{}%
\AgdaBound{𝑨}\AgdaSpace{}%
\AgdaSymbol{=}\AgdaSpace{}%
\AgdaSymbol{λ}\AgdaSpace{}%
\AgdaBound{𝑎}\AgdaSpace{}%
\AgdaSymbol{→}\AgdaSpace{}%
\AgdaSymbol{(}\AgdaOperator{\AgdaFunction{∥}}\AgdaSpace{}%
\AgdaBound{𝑨}\AgdaSpace{}%
\AgdaOperator{\AgdaFunction{∥}}\AgdaSpace{}%
\AgdaBound{𝑓}\AgdaSymbol{)}\AgdaSpace{}%
\AgdaBound{𝑎}\<%
\end{code}
\ccpad
Thus, if \ab{𝑓} \as : \af ∣ \ab 𝑆 \af ∣ is an operation symbol in the signature \ab{𝑆} and if \ab{𝑎} \as : \af ∥ \ab 𝑆 \af ∥ \ab 𝑓 \as → \af ∣ \ab 𝑨 \af ∣ is a tuple of the same arity, then (\ab{𝑓} \af ̂ \ab 𝑨) \ab 𝑎 denotes the operation \ab{𝑓} interpreted in \ab{𝑨} and evaluated at \ab{𝑎}.

% \paragraph*{Arbitrarily many variable symbols}\label{arbitrarily-many-variable-symbols}

% We sometimes want to assume that we have at our disposal an arbitrary collection \ab X of variable symbols such that, for every algebra \ab 𝑨, no matter the type of its domain, we have a surjective map \ab{h} \as : \ab X \as → \af ∣ \ab 𝑨 \af ∣ from variables onto the domain of \ab 𝑨. We may use the following definition to express this assumption when we need it.
% \ccpad
% \begin{code}%
% \>[0][@{}l@{\AgdaIndent{1}}]%
% \>[1]\AgdaOperator{\AgdaFunction{\AgdaUnderscore{}↠\AgdaUnderscore{}}}\AgdaSpace{}%
% \AgdaSymbol{:}\AgdaSpace{}%
% \AgdaSymbol{\{}\AgdaBound{𝓧}\AgdaSpace{}%
% \AgdaSymbol{:}\AgdaSpace{}%
% \AgdaPostulate{Universe}\AgdaSymbol{\}}\AgdaSpace{}%
% \AgdaSymbol{→}\AgdaSpace{}%
% \AgdaBound{𝓧}\AgdaSpace{}%
% \AgdaOperator{\AgdaFunction{̇}}\AgdaSpace{}%
% \AgdaSymbol{→}\AgdaSpace{}%
% \AgdaFunction{Algebra}\AgdaSpace{}%
% \AgdaGeneralizable{𝓤}\AgdaSpace{}%
% \AgdaBound{𝑆}\AgdaSpace{}%
% \AgdaSymbol{→}\AgdaSpace{}%
% \AgdaBound{𝓧}\AgdaSpace{}%
% \AgdaOperator{\AgdaPrimitive{⊔}}\AgdaSpace{}%
% \AgdaGeneralizable{𝓤}\AgdaSpace{}%
% \AgdaOperator{\AgdaFunction{̇}}\<%
% \\
% %
% \>[1]\AgdaBound{X}\AgdaSpace{}%
% \AgdaOperator{\AgdaFunction{↠}}\AgdaSpace{}%
% \AgdaBound{𝑨}\AgdaSpace{}%
% \AgdaSymbol{=}\AgdaSpace{}%
% \AgdaFunction{Σ}\AgdaSpace{}%
% \AgdaBound{h}\AgdaSpace{}%
% \AgdaFunction{꞉}\AgdaSpace{}%
% \AgdaSymbol{(}\AgdaBound{X}\AgdaSpace{}%
% \AgdaSymbol{→}\AgdaSpace{}%
% \AgdaOperator{\AgdaFunction{∣}}\AgdaSpace{}%
% \AgdaBound{𝑨}\AgdaSpace{}%
% \AgdaOperator{\AgdaFunction{∣}}\AgdaSymbol{)}\AgdaSpace{}%
% \AgdaFunction{,}\AgdaSpace{}%
% \AgdaFunction{Epic}\AgdaSpace{}%
% \AgdaBound{h}\<%
% \end{code}
% \ccpad
% Now we can assert, in a specific module, the existence of the surjective map described above by including the following line in that module's declaration, like so.
% \ccpad
% \begin{code}%
% \>[0]\AgdaKeyword{module}\AgdaSpace{}%
% \AgdaModule{\AgdaUnderscore{}}%
% \>[187I]\AgdaSymbol{\{}\AgdaBound{𝓧}\AgdaSpace{}%
% \AgdaSymbol{:}\AgdaSpace{}%
% \AgdaPostulate{Universe}\AgdaSymbol{\}\{}\AgdaBound{X}\AgdaSpace{}%
% \AgdaSymbol{:}\AgdaSpace{}%
% \AgdaBound{𝓧}\AgdaSpace{}%
% \AgdaOperator{\AgdaFunction{̇}}\AgdaSymbol{\}\{}\AgdaBound{𝑆}\AgdaSpace{}%
% \AgdaSymbol{:}\AgdaSpace{}%
% \AgdaFunction{Signature}\AgdaSpace{}%
% \AgdaGeneralizable{𝓞}\AgdaSpace{}%
% \AgdaGeneralizable{𝓥}\AgdaSymbol{\}}\<%
% \\
% \>[.][@{}l@{}]\<[187I]%
% \>[9]\AgdaSymbol{\{}\AgdaBound{𝕏}\AgdaSpace{}%
% \AgdaSymbol{:}\AgdaSpace{}%
% \AgdaSymbol{(}\AgdaBound{𝑨}\AgdaSpace{}%
% \AgdaSymbol{:}\AgdaSpace{}%
% \AgdaFunction{Algebra}\AgdaSpace{}%
% \AgdaGeneralizable{𝓤}\AgdaSpace{}%
% \AgdaBound{𝑆}\AgdaSymbol{)}\AgdaSpace{}%
% \AgdaSymbol{→}\AgdaSpace{}%
% \AgdaBound{X}\AgdaSpace{}%
% \AgdaOperator{\AgdaFunction{↠}}\AgdaSpace{}%
% \AgdaBound{𝑨}\AgdaSymbol{\}}\AgdaSpace{}%
% \AgdaKeyword{where}\<%
% \end{code}
% \ccpad
% Then \af{fst}(\af 𝕏 \ab 𝑨) will denote the surjective map from \ab X onto ∣ \ab 𝑨 \af ∣, and \af{snd}(\af 𝕏 \ab 𝑨) will be a proof that the map is surjective.








\paragraph*{Lifts of algebras}\label{lifts-of-algebras}

Recall, in \S\ref{sec:lifts-altern-univ} we described a common difficulty one encounters when working with a noncumulative universe hierarchy. We made a promise to provide some domain-specific level lifting and lowering methods. Here we fulfill this promise by supplying a couple of bespoke tools designed specifically for our operation and algebra types.
\ccpad
\begin{code}%
% \>[0]\AgdaKeyword{module}\AgdaSpace{}%
% \AgdaModule{\AgdaUnderscore{}}\AgdaSpace{}%
% \AgdaSymbol{\{}\AgdaBound{I}\AgdaSpace{}%
% \AgdaSymbol{:}\AgdaSpace{}%
% \AgdaGeneralizable{𝓥}\AgdaSpace{}%
% \AgdaOperator{\AgdaFunction{̇}}\AgdaSymbol{\}\{}\AgdaBound{A}\AgdaSpace{}%
% \AgdaSymbol{:}\AgdaSpace{}%
% \AgdaGeneralizable{𝓤}\AgdaSpace{}%
% \AgdaOperator{\AgdaFunction{̇}}\AgdaSymbol{\}}\AgdaSpace{}%
% \AgdaKeyword{where}\<%
% \\
% %
% \\[\AgdaEmptyExtraSkip]%
\>[0][@{}l@{\AgdaIndent{0}}]%
% \>[1]\AgdaKeyword{open}\AgdaSpace{}%
% \AgdaModule{Lift}\<%
% \\
% %
% \\[\AgdaEmptyExtraSkip]%
% %
\>[1]\AgdaFunction{Lift-op}\AgdaSpace{}%
\AgdaSymbol{:}\AgdaSpace{}%
\AgdaSymbol{((}\AgdaBound{I}\AgdaSpace{}%
\AgdaSymbol{→}\AgdaSpace{}%
\AgdaBound{A}\AgdaSymbol{)}\AgdaSpace{}%
\AgdaSymbol{→}\AgdaSpace{}%
\AgdaBound{A}\AgdaSymbol{)}\AgdaSpace{}%
\AgdaSymbol{→}\AgdaSpace{}%
\AgdaSymbol{(}\AgdaBound{𝓦}\AgdaSpace{}%
\AgdaSymbol{:}\AgdaSpace{}%
\AgdaPostulate{Universe}\AgdaSymbol{)}\AgdaSpace{}%
\AgdaSymbol{→}\AgdaSpace{}%
\AgdaSymbol{((}\AgdaBound{I}\AgdaSpace{}%
\AgdaSymbol{→}\AgdaSpace{}%
\AgdaRecord{Lift}\AgdaSymbol{\{}\AgdaBound{𝓦}\AgdaSymbol{\}}\AgdaSpace{}%
\AgdaBound{A}\AgdaSymbol{)}\AgdaSpace{}%
\AgdaSymbol{→}\AgdaSpace{}%
\AgdaRecord{Lift}\AgdaSpace{}%
\AgdaSymbol{\{}\AgdaBound{𝓦}\AgdaSymbol{\}}\AgdaSpace{}%
\AgdaBound{A}\AgdaSymbol{)}\<%
\\
%
\>[1]\AgdaFunction{Lift-op}\AgdaSpace{}%
\AgdaBound{f}\AgdaSpace{}%
\AgdaBound{𝓦}\AgdaSpace{}%
\AgdaSymbol{=}\AgdaSpace{}%
\AgdaSymbol{λ}\AgdaSpace{}%
\AgdaBound{x}\AgdaSpace{}%
\AgdaSymbol{→}\AgdaSpace{}%
\AgdaInductiveConstructor{lift}\AgdaSpace{}%
\AgdaSymbol{(}\AgdaBound{f}\AgdaSpace{}%
\AgdaSymbol{(λ}\AgdaSpace{}%
\AgdaBound{i}\AgdaSpace{}%
\AgdaSymbol{→}\AgdaSpace{}%
\AgdaField{lower}\AgdaSpace{}%
\AgdaSymbol{(}\AgdaBound{x}\AgdaSpace{}%
\AgdaBound{i}\AgdaSymbol{)))}\<%
\\
%
\\[\AgdaEmptyExtraSkip]%
% \>[0]\AgdaKeyword{module}\AgdaSpace{}%
% \AgdaModule{\AgdaUnderscore{}}\AgdaSpace{}%
% \AgdaSymbol{\{}\AgdaBound{𝑆}\AgdaSpace{}%
% \AgdaSymbol{:}\AgdaSpace{}%
% \AgdaFunction{Signature}\AgdaSpace{}%
% \AgdaGeneralizable{𝓞}\AgdaSpace{}%
% \AgdaGeneralizable{𝓥}\AgdaSymbol{\}}%
% \>[30]\AgdaKeyword{where}\<%
% \\
% %
% \\[\AgdaEmptyExtraSkip]%
\>[0][@{}l@{\AgdaIndent{0}}]%
% \>[1]\AgdaKeyword{open}\AgdaSpace{}%
% \AgdaModule{algebra}\<%
% \\
% %
% \\[\AgdaEmptyExtraSkip]%
% %
\>[1]\AgdaFunction{Lift-alg}\AgdaSpace{}%
\AgdaSymbol{:}\AgdaSpace{}%
\AgdaFunction{Algebra}\AgdaSpace{}%
\AgdaGeneralizable{𝓤}\AgdaSpace{}%
\AgdaBound{𝑆}\AgdaSpace{}%
\AgdaSymbol{→}\AgdaSpace{}%
\AgdaSymbol{(}\AgdaBound{𝓦}\AgdaSpace{}%
\AgdaSymbol{:}\AgdaSpace{}%
\AgdaPostulate{Universe}\AgdaSymbol{)}\AgdaSpace{}%
\AgdaSymbol{→}\AgdaSpace{}%
\AgdaFunction{Algebra}\AgdaSpace{}%
\AgdaSymbol{(}\AgdaGeneralizable{𝓤}\AgdaSpace{}%
\AgdaOperator{\AgdaPrimitive{⊔}}\AgdaSpace{}%
\AgdaBound{𝓦}\AgdaSymbol{)}\AgdaSpace{}%
\AgdaBound{𝑆}\<%
\\
%
\>[1]\AgdaFunction{Lift-alg}\AgdaSpace{}%
\AgdaBound{𝑨}\AgdaSpace{}%
\AgdaBound{𝓦}\AgdaSpace{}%
\AgdaSymbol{=}\AgdaSpace{}%
\AgdaRecord{Lift}\AgdaSpace{}%
\AgdaOperator{\AgdaFunction{∣}}\AgdaSpace{}%
\AgdaBound{𝑨}\AgdaSpace{}%
\AgdaOperator{\AgdaFunction{∣}}\AgdaSpace{}%
\AgdaOperator{\AgdaInductiveConstructor{,}}\AgdaSpace{}%
\AgdaSymbol{(λ}\AgdaSpace{}%
\AgdaSymbol{(}\AgdaBound{𝑓}\AgdaSpace{}%
\AgdaSymbol{:}\AgdaSpace{}%
\AgdaOperator{\AgdaFunction{∣}}\AgdaSpace{}%
\AgdaBound{𝑆}\AgdaSpace{}%
\AgdaOperator{\AgdaFunction{∣}}\AgdaSymbol{)}\AgdaSpace{}%
\AgdaSymbol{→}\AgdaSpace{}%
\AgdaFunction{Lift-op}\AgdaSpace{}%
\AgdaSymbol{(}\AgdaBound{𝑓}\AgdaSpace{}%
\AgdaOperator{\AgdaFunction{̂}}\AgdaSpace{}%
\AgdaBound{𝑨}\AgdaSymbol{)}\AgdaSpace{}%
\AgdaBound{𝓦}\AgdaSymbol{)}\<%
\\
%
\\[\AgdaEmptyExtraSkip]%
%
\>[1]\AgdaFunction{Lift-alg-record-type}\AgdaSpace{}%
\AgdaSymbol{:}\AgdaSpace{}%
\AgdaRecord{algebra}\AgdaSpace{}%
\AgdaGeneralizable{𝓤}\AgdaSpace{}%
\AgdaBound{𝑆}\AgdaSpace{}%
\AgdaSymbol{→}\AgdaSpace{}%
\AgdaSymbol{(}\AgdaBound{𝓦}\AgdaSpace{}%
\AgdaSymbol{:}\AgdaSpace{}%
\AgdaPostulate{Universe}\AgdaSymbol{)}\AgdaSpace{}%
\AgdaSymbol{→}\AgdaSpace{}%
\AgdaRecord{algebra}\AgdaSpace{}%
\AgdaSymbol{(}\AgdaGeneralizable{𝓤}\AgdaSpace{}%
\AgdaOperator{\AgdaPrimitive{⊔}}\AgdaSpace{}%
\AgdaBound{𝓦}\AgdaSymbol{)}\AgdaSpace{}%
\AgdaBound{𝑆}\<%
\\
%
\>[1]\AgdaFunction{Lift-alg-record-type}\AgdaSpace{}%
\AgdaBound{𝑨}\AgdaSpace{}%
\AgdaBound{𝓦}\AgdaSpace{}%
\AgdaSymbol{=}\AgdaSpace{}%
\AgdaInductiveConstructor{mkalg}\AgdaSpace{}%
\AgdaSymbol{(}\AgdaRecord{Lift}\AgdaSpace{}%
\AgdaSymbol{(}\AgdaField{univ}\AgdaSpace{}%
\AgdaBound{𝑨}\AgdaSymbol{))}\AgdaSpace{}%
\AgdaSymbol{(λ}\AgdaSpace{}%
\AgdaSymbol{(}\AgdaBound{f}\AgdaSpace{}%
\AgdaSymbol{:}\AgdaSpace{}%
\AgdaOperator{\AgdaFunction{∣}}\AgdaSpace{}%
\AgdaBound{𝑆}\AgdaSpace{}%
\AgdaOperator{\AgdaFunction{∣}}\AgdaSymbol{)}\AgdaSpace{}%
\AgdaSymbol{→}\AgdaSpace{}%
\AgdaFunction{Lift-op}\AgdaSpace{}%
\AgdaSymbol{((}\AgdaField{op}\AgdaSpace{}%
\AgdaBound{𝑨}\AgdaSymbol{)}\AgdaSpace{}%
\AgdaBound{f}\AgdaSymbol{)}\AgdaSpace{}%
\AgdaBound{𝓦}\AgdaSymbol{)}\<%
\end{code}
\ccpad
What makes the \AgdaFunction{Lift-alg} type so useful for resolving type level errors is the nice properties it possesses. Indeed, in the \ualib we prove that \af{Lift-alg} preserves term identities and is a \emph{homomorphism}, an \emph{algebraic invariant}, and a \emph{sublagebraic invariant}.\footnote{%
See \href{https://ualib.gitlab.io/Varieties.EquationalLogic.html\#lift-invariance}{EquationalLogic.html}, \href{https://ualib.gitlab.io/Homomorphisms.Basic.html\#exmples-of-homomorphisms}{Homomorphisms.Basic.html}, \href{https://ualib.gitlab.io/Homomorphisms.Isomorphisms.html\#lift-is-an-algebraic-invariant}{Isomorphisms.html}, and \href{https://ualib.gitlab.io/Subalgebras.Subalgebras.html\#lifts-of-subalgebras}{Subalgebras.html}, resp.}
% Our experience has shown \af{lift-alg} to be a perfectly adequate tool for resolving any universe level unification errors that arise when working with the \af{Algebra} type in Agda. We will see some examples of its effectiveness later.
% 's noncumulative hierarchy of type universes.






\paragraph*{Compatibility}\label{compatibility-of-binary-relations}

We now define the function \af{compatible} so that, if \ab{𝑨} is an algebra and \ab{R} a binary relation, then \af{compatible} \ab 𝑨 \ab R will denote the assertion that \ab{R} is \emph{compatible} with all basic operations of \ab{𝑨}; that is, for every operation symbol \ab{𝑓} \as : \af ∣ \ab 𝑆 \af ∣ and all pairs \ab u \ab v~\as :~\af{∥} \ab 𝑆 \af{∥} \ab 𝑓 \as → \af ∣ \ab 𝑨 \af ∣ of tuples from the domain of \ab 𝑨, the following implication holds:\\[-8pt]

(\af Π(\ab i \af ꞉ \ab I)~\af ,~\ab{R} (\ab u \ab i) (\ab v \ab i)) \hskip2mm \as →  \hskip2mm \ab{R} ((\ab 𝑓 \af ̂ \ab 𝑨) \ab u) ((\ab 𝑓 \af ̂ \ab 𝑨) \ab v).\\[4pt]
% ; in symbols,\\[-8pt]
% (\af Π \ab i \af ꞉ \ab I \af , \ab R (\ab u \ab i) (\ab u \ab i)) \hskip2mm \as →  \hskip2mm \ab R ((\ab 𝑓 \af ̂ \ab 𝑨) \ab u) ((\ab 𝑓 \af ̂ \ab 𝑨) \ab v).\\[4pt]
Using the relation |: (defined in \S\ref{sec:discrete-relations}) this implication is expressed compactly as (\ab 𝑓~\af ̂~\ab 𝑨)~\af{|:}~\ab R, yielding a compact representation of compatibility of algebraic operations and binary relations.
\ccpad
\begin{code}%
\>[0][@{}l@{\AgdaIndent{1}}]%
\>[1]\AgdaFunction{compatible}\AgdaSpace{}%
\AgdaSymbol{:}\AgdaSpace{}%
\AgdaSymbol{(}\AgdaBound{𝑨}\AgdaSpace{}%
\AgdaSymbol{:}\AgdaSpace{}%
\AgdaFunction{Algebra}\AgdaSpace{}%
\AgdaGeneralizable{𝓤}\AgdaSpace{}%
\AgdaBound{𝑆}\AgdaSymbol{)}\AgdaSpace{}%
\AgdaSymbol{→}\AgdaSpace{}%
\AgdaFunction{Rel}\AgdaSpace{}%
\AgdaOperator{\AgdaFunction{∣}}\AgdaSpace{}%
\AgdaBound{𝑨}\AgdaSpace{}%
\AgdaOperator{\AgdaFunction{∣}}\AgdaSpace{}%
\AgdaGeneralizable{𝓦}\AgdaSpace{}%
\AgdaSymbol{→}\AgdaSpace{}%
\AgdaBound{𝓞}\AgdaSpace{}%
\AgdaOperator{\AgdaPrimitive{⊔}}\AgdaSpace{}%
\AgdaGeneralizable{𝓤}\AgdaSpace{}%
\AgdaOperator{\AgdaPrimitive{⊔}}\AgdaSpace{}%
\AgdaBound{𝓥}\AgdaSpace{}%
\AgdaOperator{\AgdaPrimitive{⊔}}\AgdaSpace{}%
\AgdaGeneralizable{𝓦}\AgdaSpace{}%
\AgdaOperator{\AgdaFunction{̇}}\<%
\\
%
\>[1]\AgdaFunction{compatible}%
\>[13]\AgdaBound{𝑨}\AgdaSpace{}%
\AgdaBound{R}\AgdaSpace{}%
\AgdaSymbol{=}\AgdaSpace{}%
\AgdaSymbol{∀}\AgdaSpace{}%
\AgdaBound{𝑓}\AgdaSpace{}%
\AgdaSymbol{→}\AgdaSpace{}%
\AgdaSymbol{(}\AgdaBound{𝑓}\AgdaSpace{}%
\AgdaOperator{\AgdaFunction{̂}}\AgdaSpace{}%
\AgdaBound{𝑨}\AgdaSymbol{)}\AgdaSpace{}%
\AgdaFunction{|:}\AgdaSpace{}%
\AgdaBound{R}\<%
\end{code}
\scpad
% \paragraph*{Compatibility of continuous relations★}

In the \ualibhtml{Relations.Continuous} module we defined a function called \af{cont-compatible-op} to represent the assertion that a given continuous relation is compatible with a given operation. With that, it is easy to define a function, which we call \af{cont-compatible}, representing compatibility of a continuous relation with all operations of an algebra. Similarly, we define the analogous \af{dep-compatible} function for the (even more general)
type of dependent relations.
\ccpad
\begin{code}%
% \>[0]\AgdaKeyword{module}\AgdaSpace{}%
% \AgdaModule{continuous-compatibility}\AgdaSpace{}%
% \AgdaSymbol{\{}\AgdaBound{𝓤}\AgdaSpace{}%
% \AgdaBound{𝓦}\AgdaSpace{}%
% \AgdaSymbol{:}\AgdaSpace{}%
% \AgdaPostulate{Universe}\AgdaSymbol{\}\{}\AgdaBound{𝑆}\AgdaSpace{}%
% \AgdaSymbol{:}\AgdaSpace{}%
% \AgdaFunction{Signature}\AgdaSpace{}%
% \AgdaGeneralizable{𝓞}\AgdaSpace{}%
% \AgdaGeneralizable{𝓥}\AgdaSymbol{\}}\AgdaSpace{}%
% \AgdaKeyword{where}\<%
% \\
% \>[0][@{}l@{\AgdaIndent{0}}]%
% \>[1]\AgdaKeyword{open}\AgdaSpace{}%
% \AgdaKeyword{import}\AgdaSpace{}%
% \AgdaModule{Relations.Continuous}\AgdaSpace{}%
% \AgdaKeyword{using}\AgdaSpace{}%
% \AgdaSymbol{(}\AgdaFunction{ContRel}\AgdaSymbol{;}\AgdaSpace{}%
% \AgdaFunction{DepRel}\AgdaSymbol{;}\AgdaSpace{}%
% \AgdaFunction{cont-compatible-op}\AgdaSymbol{;}\AgdaSpace{}%
% \AgdaFunction{dep-compatible-op}\AgdaSymbol{)}\<%
% \\
% %
% \\[\AgdaEmptyExtraSkip]%
% %
\>[1]\AgdaFunction{cont-compatible}\AgdaSpace{}%
\AgdaSymbol{:}\AgdaSpace{}%
\AgdaSymbol{\{}\AgdaBound{I}\AgdaSpace{}%
\AgdaSymbol{:}\AgdaSpace{}%
\AgdaBound{𝓥}\AgdaSpace{}%
\AgdaOperator{\AgdaFunction{̇}}\AgdaSymbol{\}(}\AgdaBound{𝑨}\AgdaSpace{}%
\AgdaSymbol{:}\AgdaSpace{}%
\AgdaFunction{Algebra}\AgdaSpace{}%
\AgdaBound{𝓤}\AgdaSpace{}%
\AgdaBound{𝑆}\AgdaSymbol{)}\AgdaSpace{}%
\AgdaSymbol{→}\AgdaSpace{}%
\AgdaFunction{ContRel}\AgdaSpace{}%
\AgdaBound{I}\AgdaSpace{}%
\AgdaOperator{\AgdaFunction{∣}}\AgdaSpace{}%
\AgdaBound{𝑨}\AgdaSpace{}%
\AgdaOperator{\AgdaFunction{∣}}\AgdaSpace{}%
\AgdaBound{𝓦}\AgdaSpace{}%
\AgdaSymbol{→}\AgdaSpace{}%
\AgdaBound{𝓞}\AgdaSpace{}%
\AgdaOperator{\AgdaPrimitive{⊔}}\AgdaSpace{}%
\AgdaBound{𝓤}\AgdaSpace{}%
\AgdaOperator{\AgdaPrimitive{⊔}}\AgdaSpace{}%
\AgdaBound{𝓥}\AgdaSpace{}%
\AgdaOperator{\AgdaPrimitive{⊔}}\AgdaSpace{}%
\AgdaBound{𝓦}\AgdaSpace{}%
\AgdaOperator{\AgdaFunction{̇}}\<%
\\
%
\>[1]\AgdaFunction{cont-compatible}\AgdaSpace{}%
\AgdaBound{𝑨}\AgdaSpace{}%
\AgdaBound{R}\AgdaSpace{}%
\AgdaSymbol{=}\AgdaSpace{}%
\AgdaFunction{Π}\AgdaSpace{}%
\AgdaBound{𝑓}\AgdaSpace{}%
\AgdaFunction{꞉}\AgdaSpace{}%
\AgdaOperator{\AgdaFunction{∣}}\AgdaSpace{}%
\AgdaBound{𝑆}\AgdaSpace{}%
\AgdaOperator{\AgdaFunction{∣}}\AgdaSpace{}%
\AgdaFunction{,}\AgdaSpace{}%
\AgdaFunction{cont-compatible-op}\AgdaSpace{}%
\AgdaSymbol{(}\AgdaBound{𝑓}\AgdaSpace{}%
\AgdaOperator{\AgdaFunction{̂}}\AgdaSpace{}%
\AgdaBound{𝑨}\AgdaSymbol{)}\AgdaSpace{}%
\AgdaBound{R}\<%
\\
%
\\[\AgdaEmptyExtraSkip]%
%
\>[1]\AgdaFunction{dep-compatible}\AgdaSpace{}%
\AgdaSymbol{:}\AgdaSpace{}%
\AgdaSymbol{\{}\AgdaBound{I}\AgdaSpace{}%
\AgdaSymbol{:}\AgdaSpace{}%
\AgdaBound{𝓥}\AgdaSpace{}%
\AgdaOperator{\AgdaFunction{̇}}\AgdaSymbol{\}(}\AgdaBound{𝒜}\AgdaSpace{}%
\AgdaSymbol{:}\AgdaSpace{}%
\AgdaBound{I}\AgdaSpace{}%
\AgdaSymbol{→}\AgdaSpace{}%
\AgdaFunction{Algebra}\AgdaSpace{}%
\AgdaBound{𝓤}\AgdaSpace{}%
\AgdaBound{𝑆}\AgdaSymbol{)}\AgdaSpace{}%
\AgdaSymbol{→}\AgdaSpace{}%
\AgdaFunction{DepRel}\AgdaSpace{}%
\AgdaBound{I}\AgdaSpace{}%
\AgdaSymbol{(λ}\AgdaSpace{}%
\AgdaBound{i}\AgdaSpace{}%
\AgdaSymbol{→}\AgdaSpace{}%
\AgdaOperator{\AgdaFunction{∣}}\AgdaSpace{}%
\AgdaBound{𝒜}%
\>[72]\AgdaBound{i}\AgdaSpace{}%
\AgdaOperator{\AgdaFunction{∣}}\AgdaSymbol{)}\AgdaSpace{}%
\AgdaBound{𝓦}\AgdaSpace{}%
\AgdaSymbol{→}\AgdaSpace{}%
\AgdaBound{𝓞}\AgdaSpace{}%
\AgdaOperator{\AgdaPrimitive{⊔}}\AgdaSpace{}%
\AgdaBound{𝓤}\AgdaSpace{}%
\AgdaOperator{\AgdaPrimitive{⊔}}\AgdaSpace{}%
\AgdaBound{𝓥}\AgdaSpace{}%
\AgdaOperator{\AgdaPrimitive{⊔}}\AgdaSpace{}%
\AgdaBound{𝓦}\AgdaSpace{}%
\AgdaOperator{\AgdaFunction{̇}}\<%
\\
%
\>[1]\AgdaFunction{dep-compatible}\AgdaSpace{}%
\AgdaBound{𝒜}\AgdaSpace{}%
\AgdaBound{R}\AgdaSpace{}%
\AgdaSymbol{=}\AgdaSpace{}%
\AgdaFunction{Π}\AgdaSpace{}%
\AgdaBound{𝑓}\AgdaSpace{}%
\AgdaFunction{꞉}\AgdaSpace{}%
\AgdaOperator{\AgdaFunction{∣}}\AgdaSpace{}%
\AgdaBound{𝑆}\AgdaSpace{}%
\AgdaOperator{\AgdaFunction{∣}}\AgdaSpace{}%
\AgdaFunction{,}\AgdaSpace{}%
\AgdaFunction{dep-compatible-op}\AgdaSpace{}%
\AgdaSymbol{(λ}\AgdaSpace{}%
\AgdaBound{i}\AgdaSpace{}%
\AgdaSymbol{→}\AgdaSpace{}%
\AgdaBound{𝑓}\AgdaSpace{}%
\AgdaOperator{\AgdaFunction{̂}}\AgdaSpace{}%
\AgdaSymbol{(}\AgdaBound{𝒜}\AgdaSpace{}%
\AgdaBound{i}\AgdaSymbol{))}\AgdaSpace{}%
\AgdaBound{R}\<%
\end{code}


% \paragraph*{Compatibility of continuous relations*\protect\cref{starred}}\label{compatibility-of-continuous-relations}

% Similarly, a type representing \emph{compatibility of a continuous relation} with a single operation of an algebra is defined using \AgdaFunction{cont-compatible-fun}~(\S\ref{continuous-relation-types}).
% \ccpad
% \begin{code}%
% % \>[0][@{}l@{\AgdaIndent{0}}]%
% \>[1]\AgdaFunction{cont-compatible-op}\AgdaSpace{}%
% \AgdaSymbol{:}\AgdaSpace{}%
% \AgdaOperator{\AgdaFunction{∣}}\AgdaSpace{}%
% \AgdaBound{𝑆}\AgdaSpace{}%
% \AgdaOperator{\AgdaFunction{∣}}\AgdaSpace{}%
% \AgdaSymbol{→}\AgdaSpace{}%
% \AgdaFunction{ContRel}\AgdaSpace{}%
% \AgdaBound{I}\AgdaSpace{}%
% \AgdaOperator{\AgdaFunction{∣}}\AgdaSpace{}%
% \AgdaBound{𝑨}\AgdaSpace{}%
% \AgdaOperator{\AgdaFunction{∣}}\AgdaSpace{}%
% \AgdaGeneralizable{𝓦}\AgdaSpace{}%
% \AgdaSymbol{→}\AgdaSpace{}%
% \AgdaBound{𝓤}\AgdaSpace{}%
% \AgdaOperator{\AgdaPrimitive{⊔}}\AgdaSpace{}%
% \AgdaBound{𝓥}\AgdaSpace{}%
% \AgdaOperator{\AgdaPrimitive{⊔}}\AgdaSpace{}%
% \AgdaGeneralizable{𝓦}\AgdaSpace{}%
% \AgdaOperator{\AgdaFunction{̇}}\<%
% \\
% %
% \>[1]\AgdaFunction{cont-compatible-op}\AgdaSpace{}%
% \AgdaBound{𝑓}\AgdaSpace{}%
% \AgdaBound{R}\AgdaSpace{}%
% \AgdaSymbol{=}\AgdaSpace{}%
% \AgdaFunction{cont-compatible-fun}\AgdaSpace{}%
% \AgdaSymbol{(}\AgdaBound{𝑓}\AgdaSpace{}%
% \AgdaOperator{\AgdaFunction{̂}}\AgdaSpace{}%
% \AgdaBound{𝑨}\AgdaSymbol{)}\AgdaSpace{}%
% \AgdaBound{R}\<%
% \end{code}
% \ccpad
% Whence, we obtain a type representing \emph{compatibility of a continuous relation with an algebra}.
% \ccpad
% \begin{code}%
% \>[0][@{}l@{\AgdaIndent{1}}]%
% \>[1]\AgdaFunction{cont-compatible}\AgdaSpace{}%
% \AgdaSymbol{:}\AgdaSpace{}%
% \AgdaFunction{ContRel}\AgdaSpace{}%
% \AgdaBound{I}\AgdaSpace{}%
% \AgdaOperator{\AgdaFunction{∣}}\AgdaSpace{}%
% \AgdaBound{𝑨}\AgdaSpace{}%
% \AgdaOperator{\AgdaFunction{∣}}\AgdaSpace{}%
% \AgdaGeneralizable{𝓦}\AgdaSpace{}%
% \AgdaSymbol{→}\AgdaSpace{}%
% \AgdaBound{𝓞}\AgdaSpace{}%
% \AgdaOperator{\AgdaPrimitive{⊔}}\AgdaSpace{}%
% \AgdaBound{𝓤}\AgdaSpace{}%
% \AgdaOperator{\AgdaPrimitive{⊔}}\AgdaSpace{}%
% \AgdaBound{𝓥}\AgdaSpace{}%
% \AgdaOperator{\AgdaPrimitive{⊔}}\AgdaSpace{}%
% \AgdaGeneralizable{𝓦}\AgdaSpace{}%
% \AgdaOperator{\AgdaFunction{̇}}\<%
% \\
% %
% \>[1]\AgdaFunction{cont-compatible}\AgdaSpace{}%
% \AgdaBound{R}\AgdaSpace{}%
% \AgdaSymbol{=}\AgdaSpace{}%
% \AgdaFunction{Π}\AgdaSpace{}%
% \AgdaBound{𝑓}\AgdaSpace{}%
% \AgdaFunction{꞉}\AgdaSpace{}%
% \AgdaOperator{\AgdaFunction{∣}}\AgdaSpace{}%
% \AgdaBound{𝑆}\AgdaSpace{}%
% \AgdaOperator{\AgdaFunction{∣}}\AgdaSpace{}%
% \AgdaFunction{,}\AgdaSpace{}%
% \AgdaFunction{cont-compatible-op}\AgdaSpace{}%
% \AgdaBound{𝑓}\AgdaSpace{}%
% \AgdaBound{R}\<%
% \end{code}
% \protect\hypertarget{fn2}{}{2 This voilates the ``don't repeat yourself'' (dry) principle of programming, but it might make it easier for readers to see what's going on. (In the {[}UALib{]}{[}{]} we try to put transparency before elegance.)}



\subsection{Products: types for products over arbitrary classes}\label{sec:product-algebras}
\firstsentence{Algebras}{Products}
% -*- TeX-master: "ualib-part1.tex" -*-
%%% Local Variables: 
%%% mode: latex
%%% TeX-engine: 'xetex
%%% End:
% We assume a fixed signature \ab{𝑆} \as : \af{Signature} \ab 𝓞 \ab 𝓥 throughout the module by starting with the line
% \AgdaKeyword{module}\AgdaSpace{}%
% \AgdaModule{Algebras.Products}\AgdaSpace{}%
% \AgdaSymbol{\{}\AgdaBound{𝑆}\AgdaSpace{}%
% \AgdaSymbol{:}\AgdaSpace{}%
% \AgdaFunction{Signature}\AgdaSpace{}%
% \AgdaGeneralizable{𝓞}\AgdaSpace{}%
% \AgdaGeneralizable{𝓥}\AgdaSymbol{\}}\AgdaSpace{}%
% \AgdaKeyword{where}.%
% We begin this module by assuming a signature \ab{𝑆} \as : \af{Signature} \ab 𝓞 \ab 𝓥 which is then present and available throughout the module.  Because of this, in contrast to descriptions of previous modules, we present the first few lines of the \ualibhtml{Algebras.Products} module in full.
% \ccpad
% \begin{code}%
% \>[0]\AgdaSymbol{\{-\#}\AgdaSpace{}%
% \AgdaKeyword{OPTIONS}\AgdaSpace{}%
% \AgdaPragma{--without-K}\AgdaSpace{}%
% \AgdaPragma{--exact-split}\AgdaSpace{}%
% \AgdaPragma{--safe}\AgdaSpace{}%
% \AgdaSymbol{\#-\}}\<%
% \\
% %
% \\[\AgdaEmptyExtraSkip]%
% \>[0]\AgdaKeyword{open}\AgdaSpace{}%
% \AgdaKeyword{import}\AgdaSpace{}%
% \AgdaModule{Algebras.Signatures}\AgdaSpace{}%
% \AgdaKeyword{using}\AgdaSpace{}%
% \AgdaSymbol{(}\AgdaFunction{Signature}\AgdaSymbol{;}\AgdaSpace{}%
% \AgdaGeneralizable{𝓞}\AgdaSymbol{;}\AgdaSpace{}%
% \AgdaGeneralizable{𝓥}\AgdaSymbol{)}\<%
% \\
% %
% % \\[\AgdaEmptyExtraSkip]%
% \>[0]\AgdaKeyword{module}\AgdaSpace{}%
% \AgdaModule{Algebras.Products}\AgdaSpace{}%
% \AgdaSymbol{\{}\AgdaBound{𝑆}\AgdaSpace{}%
% \AgdaSymbol{:}\AgdaSpace{}%
% \AgdaFunction{Signature}\AgdaSpace{}%
% \AgdaGeneralizable{𝓞}\AgdaSpace{}%
% \AgdaGeneralizable{𝓥}\AgdaSymbol{\}}\AgdaSpace{}%
% \AgdaKeyword{where}\<%
% \\
% %
% % \\[\AgdaEmptyExtraSkip]%
% \>[0]\AgdaKeyword{open}\AgdaSpace{}%
% \AgdaKeyword{import}\AgdaSpace{}%
% \AgdaModule{Algebras.Algebras}\AgdaSpace{}%
% \AgdaKeyword{hiding}\AgdaSpace{}%
% \AgdaSymbol{(}\AgdaGeneralizable{𝓞}\AgdaSymbol{;}\AgdaSpace{}%
% \AgdaGeneralizable{𝓥}\AgdaSymbol{)}\AgdaSpace{}%
% \AgdaKeyword{public}\<%
% \end{code}
% \ccpad
% Notice that we must import the \af{Signature} type from \ualibhtml{Algebras.Signatures} first so that we can use it to declare the signature \AgdaBound{𝑆} as a parameter of the \ualibhtml{Algebras.Products} module.
% \>[0]\AgdaKeyword{module}\AgdaSpace{}%
% \AgdaModule{Algebras.Products}\AgdaSpace{}%
% \AgdaSymbol{\{}\AgdaBound{𝑆}\AgdaSpace{}%
% \AgdaSymbol{:}\AgdaSpace{}%
% \AgdaFunction{Signature}\AgdaSpace{}%
% \AgdaGeneralizable{𝓞}\AgdaSpace{}%
% \AgdaGeneralizable{𝓥}\AgdaSymbol{\}}\AgdaSpace{}%
% \AgdaKeyword{where}\<%
% \end{code}

Recall the informal definition of a \defn{product} of a family of \ab 𝑆-algebras. Given a type \ab I~\as :~\ab 𝓘\af ̇ and a family \ab 𝒜~\as :~\ab I~\as →~\ab{Algebra}~\ab 𝓤~\ab 𝑆, the \defn{product} \af ⨅~\ab 𝒜 is the algebra whose domain is the Cartesian product \af Π~\ab 𝑖~\as ꞉~\ab I~\af ,~\af ∣~\ab 𝒜~\ab 𝑖~\af ∣ of the domains of the algebras in \ab 𝒜, and the operation symbols are interpreted point-wise in the following sense: if \ab 𝑓 is a \ab J-ary operation symbol and if \ab 𝑎~\as :~\af Π~\ab 𝑖~\as ꞉~\ab I~\af ,~\ab J~\as →~\ab 𝒜~\ab 𝑖 gives, for each \ab 𝑖~\as :~\ab I, a \ab J-tuple of elements of \ab 𝒜~\ab 𝑖, then we define (\ab 𝑓~\af ̂~\af ⨅~\ab 𝒜)~\ab 𝑎 := (\ab i~\as :~\ab I)~\as → (\ab 𝑓~\af ̂~\ab 𝒜~\ab i)(\ab 𝑎~\ab i). We now define a type that codifies this informal definition of product algebra.
% \footnote{Alternative equivalent notation for the domain of the product is \as ∀~\ab i~\as →~\af ∣~\ab 𝒜~\ab i~\af ∣.}
\ccpad
\begin{code}%
% \>[0]\AgdaKeyword{module}\AgdaSpace{}%
% \AgdaModule{\AgdaUnderscore{}}\AgdaSpace{}%
% \AgdaSymbol{\{}\AgdaBound{𝓤}\AgdaSpace{}%
% \AgdaBound{𝓘}\AgdaSpace{}%
% \AgdaSymbol{:}\AgdaSpace{}%
% \AgdaPostulate{Universe}\AgdaSymbol{\}\{}\AgdaBound{I}\AgdaSpace{}%
% \AgdaSymbol{:}\AgdaSpace{}%
% \AgdaBound{𝓘}\AgdaSpace{}%
% \AgdaOperator{\AgdaFunction{̇}}\AgdaSpace{}%
% \AgdaSymbol{\}}\AgdaSpace{}%
% \AgdaKeyword{where}\<%
% \\
% %
% \\[\AgdaEmptyExtraSkip]%
\>[0][@{}l@{\AgdaIndent{0}}]%
\>[1]\AgdaFunction{⨅}\AgdaSpace{}%
\AgdaSymbol{:}\AgdaSpace{}%
\AgdaSymbol{(}\AgdaBound{𝒜}\AgdaSpace{}%
\AgdaSymbol{:}\AgdaSpace{}%
\AgdaBound{I}\AgdaSpace{}%
\AgdaSymbol{→}\AgdaSpace{}%
\AgdaFunction{Algebra}\AgdaSpace{}%
\AgdaBound{𝓤}\AgdaSpace{}%
\AgdaBound{𝑆}\AgdaSpace{}%
\AgdaSymbol{)}\AgdaSpace{}%
\AgdaSymbol{→}\AgdaSpace{}%
\AgdaFunction{Algebra}\AgdaSpace{}%
\AgdaSymbol{(}\AgdaBound{𝓘}\AgdaSpace{}%
\AgdaOperator{\AgdaPrimitive{⊔}}\AgdaSpace{}%
\AgdaBound{𝓤}\AgdaSymbol{)}\AgdaSpace{}%
\AgdaBound{𝑆}\<%
\\
%
\\[\AgdaEmptyExtraSkip]%
%
\>[1]\AgdaFunction{⨅}\AgdaSpace{}%
\AgdaBound{𝒜}\AgdaSpace{}%
\AgdaSymbol{=}%
\>[51I]\AgdaSymbol{(}\AgdaFunction{Π}\AgdaSpace{}%
\AgdaBound{i}\AgdaSpace{}%
\AgdaFunction{꞉}\AgdaSpace{}%
\AgdaBound{I}\AgdaSpace{}%
\AgdaFunction{,}\AgdaSpace{}%
\AgdaOperator{\AgdaFunction{∣}}\AgdaSpace{}%
\AgdaBound{𝒜}\AgdaSpace{}%
\AgdaBound{i}\AgdaSpace{}%
\AgdaOperator{\AgdaFunction{∣}}\AgdaSymbol{)}\AgdaSpace{}%
\AgdaOperator{\AgdaInductiveConstructor{,}}%
\>[43]\AgdaComment{-- domain of the product algebra}\<%
\\
\>[.][@{}l@{}]\<[51I]%
\>[7]\AgdaSymbol{λ}\AgdaSpace{}%
\AgdaBound{𝑓}\AgdaSpace{}%
\AgdaBound{𝑎}\AgdaSpace{}%
\AgdaBound{i}\AgdaSpace{}%
\AgdaSymbol{→}\AgdaSpace{}%
\AgdaSymbol{(}\AgdaBound{𝑓}\AgdaSpace{}%
\AgdaOperator{\AgdaFunction{̂}}\AgdaSpace{}%
\AgdaBound{𝒜}\AgdaSpace{}%
\AgdaBound{i}\AgdaSymbol{)}\AgdaSpace{}%
\AgdaSymbol{λ}\AgdaSpace{}%
\AgdaBound{x}\AgdaSpace{}%
\AgdaSymbol{→}\AgdaSpace{}%
\AgdaBound{𝑎}\AgdaSpace{}%
\AgdaBound{x}\AgdaSpace{}%
\AgdaBound{i}%
\>[43]\AgdaComment{-- basic operations of the product algebra}\<%
\end{code}
\ccpad
% The type just defined is the one we use whenever the product of a collection of
% algebras (of type \af{Algebra}) is required.
%%%%%%%%%%%%%%%%%%%%%%%%%%%%%%%%%%%%%%%%%%%%%%%%%%%%%%%%%%%%%%%%%%%%%%%%%%%%%%%%%%%%%%%%%
%% COMMENTED OUT %%%%%%%%%%%%%%%%%%%%%%%%%%%%%%%%%%%%%%%%%%%%%%%%%%%%%%%%%%%%%%%%%%%%%%%%
\begin{comment}
However, for the sake of completeness, here is how one could define a type representing the product of algebras inhabiting the record type \AgdaRecord{algebra}.
\ccpad
\begin{code}%
\>[0][@{}l@{\AgdaIndent{0}}]%
\>[1]\AgdaFunction{⨅'}\AgdaSpace{}%
\AgdaSymbol{:}\AgdaSpace{}%
\AgdaSymbol{(}\AgdaBound{𝒜}\AgdaSpace{}%
\AgdaSymbol{:}\AgdaSpace{}%
\AgdaBound{I}\AgdaSpace{}%
\AgdaSymbol{→}\AgdaSpace{}%
\AgdaRecord{algebra}\AgdaSpace{}%
\AgdaBound{𝓤}\AgdaSpace{}%
\AgdaBound{𝑆}\AgdaSymbol{)}\AgdaSpace{}%
\AgdaSymbol{→}\AgdaSpace{}%
\AgdaRecord{algebra}\AgdaSpace{}%
\AgdaSymbol{(}\AgdaBound{𝓘}\AgdaSpace{}%
\AgdaOperator{\AgdaPrimitive{⊔}}\AgdaSpace{}%
\AgdaBound{𝓤}\AgdaSymbol{)}\AgdaSpace{}%
\AgdaBound{𝑆}\<%
\\
\>[1]\AgdaFunction{⨅'}\AgdaSpace{}%
\AgdaBound{𝒜}\AgdaSpace{}%
\AgdaSymbol{=}\AgdaSpace{}%
\AgdaKeyword{record}%
\>[100I]\AgdaSymbol{\{}\AgdaSpace{}%
\AgdaField{univ}\AgdaSpace{}%
\AgdaSymbol{=}\AgdaSpace{}%
\AgdaSymbol{∀}\AgdaSpace{}%
\AgdaBound{i}\AgdaSpace{}%
\AgdaSymbol{→}\AgdaSpace{}%
\AgdaField{univ}\AgdaSpace{}%
\AgdaSymbol{(}\AgdaBound{𝒜}\AgdaSpace{}%
\AgdaBound{i}\AgdaSymbol{)}\AgdaSpace{}\AgdaSymbol{;}%
\>[55]\AgdaComment{-- domain}\<%
\\
\>[100I][@{}l@{\AgdaIndent{0}}]%
\>[15]\AgdaField{op}\AgdaSpace{}%
\AgdaSymbol{=}\AgdaSpace{}%
\AgdaSymbol{λ}\AgdaSpace{}%
\AgdaBound{𝑓}\AgdaSpace{}%
\AgdaBound{𝑎}\AgdaSpace{}%
\AgdaBound{i}\AgdaSpace{}%
\AgdaSymbol{→}\AgdaSpace{}%
\AgdaSymbol{(}\AgdaField{op}\AgdaSpace{}%
\AgdaSymbol{(}\AgdaBound{𝒜}\AgdaSpace{}%
\AgdaBound{i}\AgdaSymbol{))}\AgdaSpace{}%
\AgdaBound{𝑓}\AgdaSpace{}%
\AgdaSymbol{λ}\AgdaSpace{}%
\AgdaBound{x}\AgdaSpace{}%
\AgdaSymbol{→}\AgdaSpace{}%
\AgdaBound{𝑎}\AgdaSpace{}%
\AgdaBound{x}\AgdaSpace{}%
\AgdaBound{i}\AgdaSpace{}\AgdaSymbol{\}}%
\>[55]\AgdaComment{-- basic operations}\<%
\end{code}
\ccpad
\end{comment}
%% END COMMENTED OUT %%%%%%%%%%%%%%%%%%%%%%%%%%%%%%%%%%%%%%%%%%%%%%%%%%%%%%%%%%%%%%%%%%%%
%%%%%%%%%%%%%%%%%%%%%%%%%%%%%%%%%%%%%%%%%%%%%%%%%%%%%%%%%%%%%%%%%%%%%%%%%%%%%%%%%%%%%%%%%
Before going further, let us agree on another convenient notational convention, which is used in many of the later modules of the \ualib. Given a signature \ab{𝑆}~\as :~\af{Signature}~\ab 𝓞~\ab 𝓥, the type \af{Algebra}~\ab 𝓤~\ab 𝑆 has type \ab{𝓞}~\ap ⊔~\ab 𝓥~\ap ⊔~\ab 𝓤~\af ⁺\af ̇, and \ab{𝓞} \ap ⊔ \ab 𝓥 remains fixed since \ab{𝓞} and \ab 𝓥 always denote the universes of operation and arity types, respectively. Such levels occur so often in the \ualib that we define the following shorthand for their universes: \AgdaFunction{ov}\AgdaSpace{}%
\AgdaBound{𝓤}\AgdaSpace{}%
\AgdaSymbol{:=}\AgdaSpace{}%
\AgdaBound{𝓞}\AgdaSpace{}%
\AgdaOperator{\AgdaPrimitive{⊔}}\AgdaSpace{}%
\AgdaBound{𝓥}\AgdaSpace{}%
\AgdaOperator{\AgdaPrimitive{⊔}}\AgdaSpace{}%
\AgdaBound{𝓤}\AgdaSpace{}%
\AgdaOperator{\AgdaPrimitive{⁺}}.


\paragraph*{Products of classes of algebras}\label{products-of-classes-of-algebras}
% \context{\AgdaSymbol{\{}\AgdaBound{𝓤}\AgdaSpace{}%
% \AgdaBound{𝓧}\AgdaSpace{}%
% \AgdaSymbol{:}\AgdaSpace{}%
% \AgdaPostulate{Universe}\AgdaSymbol{\}\{}\AgdaBound{X}\AgdaSpace{}%
% \AgdaSymbol{:}\AgdaSpace{}%
% \AgdaBound{𝓧}\AgdaSpace{}%
% \AgdaOperator{\AgdaFunction{̇}}\AgdaSymbol{\}}}

An arbitrary class \ab 𝒦 of algebras is represented as a predicate over the type \AgdaFunction{Algebra}\AgdaSpace{}\AgdaBound{𝓤}\AgdaSpace{}\AgdaBound{𝑆}, for some universe level \AgdaBound{𝓤} and signature \AgdaBound{𝑆}.
That is, \ab 𝒦~\as :~\af{Pred}(\af{Algebra}~\ab 𝓤~\ab 𝑆)~𝓦 for some 𝓦.
Later we will formally state and prove that the product of all subalgebras of algebras in \ab 𝒦  belongs to the class \af{SP}(\ab 𝒦) of subalgebras of products of algebras in \ab 𝒦. That is, \af ⨅ \af S(\ab 𝒦) \af ∈ \af{SP}(\ab 𝒦). This turns out to be a nontrivial exercise.

To begin, we need to define types that represent products over arbitrary (non-indexed) families such as \ab 𝒦 or \af S(\ab 𝒦). Observe that \ad Π~\ab 𝒦 is certainly not what we want.  For recall that \af{Pred}(\ab{Algebra}~\ab 𝓤~\ab 𝑆)~\ab 𝓦 is just an alias for the function type \af{Algebra}~\ab 𝓤~\ab 𝑆~\as →~\ab 𝓦\af ̇, and the semantics of the latter takes \ab 𝒦~\ab 𝑨 to mean that \ab 𝑨 belongs to the class \ab 𝒦. Therefore, by definition\\[-10pt]

% \vskip-5mm
% \begin{align*}
% \ad Π~\ab 𝒦 &= \ad Π~\ab 𝑨~\af ꞉~(\af{Algebra}~\ab 𝓤~\ab 𝑆)~,~\ab 𝒦 𝑨\\
%              &= \ad Π~\ab 𝑨~\af ꞉~(\af{Algebra}~\ab 𝓤~\ab 𝑆)~,~\ab 𝑨~\af ∈~\ab 𝒦.
% \end{align*}
\ad Π~\ab 𝒦 \hskip1mm = \hskip1mm \ad Π~\ab 𝑨~\af ꞉~(\af{Algebra}~\ab 𝓤~\ab 𝑆)~,~\ab 𝒦 𝑨
             \hskip1mm = \hskip1mm \as ∀ (\ab 𝑨~\as :~\af{Algebra}~\ab 𝓤~\ab 𝑆)~\as →~\ab 𝑨~\af ∈~\ab 𝒦,\\[4pt]
which denotes the assertion that every inhabitant of the type \af{Algebra} \ab 𝓤 \ab 𝑆 belongs to \ab 𝒦. .  Evidently this is not the product algebra that we seek.

What we need is a type that serves to index the class \ab 𝒦, and a function \af 𝔄 that maps an index to the inhabitant of \ab 𝒦 at that index. But \ab 𝒦 is a predicate (of type (\af{Algebra}~\ab 𝓤~\ab 𝑆)~\as →~\ab 𝓦\af ̇) and the type \af{Algebra}~\ab 𝓤~\ab 𝑆 seems rather nebulous in that there is no natural indexing class with which to ``enumerate'' all inhabitants of \af{Algebra}~\ab 𝓤~\ab 𝑆 that belong to \ab 𝒦.\footnote{%
If you haven't seen this before, give it some thought and see if the correct type comes to you organically.}


The solution is to essentially take \ab 𝒦 itself to be the index type; at least heuristically that is how one can think of the type \af ℑ that we now define.\footnote{\textbf{Unicode Hints}. Some of our types are denoted with Gothic (``mathfrak'') symbols. To produce them in \agdamode, type \texttt{\textbackslash{}Mf} followed by a letter. For example, \texttt{\textbackslash{}MfI} ↝ \af ℑ.}
\ccpad
\begin{code}%
% \>[0]\AgdaKeyword{module}\AgdaSpace{}%
% \AgdaModule{class-products}\AgdaSpace{}%
% \AgdaSymbol{\{}\AgdaBound{𝓤}\AgdaSpace{}%
% \AgdaSymbol{:}\AgdaSpace{}%
% \AgdaPostulate{Universe}\AgdaSymbol{\}}\AgdaSpace{}%
% \AgdaSymbol{(}\AgdaBound{𝒦}\AgdaSpace{}%
% \AgdaSymbol{:}\AgdaSpace{}%
% \AgdaFunction{Pred}\AgdaSpace{}%
% \AgdaSymbol{(}\AgdaFunction{Algebra}\AgdaSpace{}%
% \AgdaBound{𝓤}\AgdaSpace{}%
% \AgdaBound{𝑆}\AgdaSymbol{)(}\AgdaFunction{ov}\AgdaSpace{}%
% \AgdaBound{𝓤}\AgdaSymbol{))}\AgdaSpace{}%
% \AgdaKeyword{where}\<%
% \\
% %
% \\[\AgdaEmptyExtraSkip]%
\>[0][@{}l@{\AgdaIndent{0}}]%
\>[1]\AgdaFunction{ℑ}\AgdaSpace{}%
\AgdaSymbol{:}\AgdaSpace{}%
\AgdaFunction{ov}\AgdaSpace{}%
\AgdaBound{𝓤}\AgdaSpace{}%
\AgdaOperator{\AgdaFunction{̇}}\<%
\\
%
\>[1]\AgdaFunction{ℑ}\AgdaSpace{}%
\AgdaSymbol{=}\AgdaSpace{}%
\AgdaFunction{Σ}\AgdaSpace{}%
\AgdaBound{𝑨}\AgdaSpace{}%
\AgdaFunction{꞉}\AgdaSpace{}%
\AgdaSymbol{(}\AgdaFunction{Algebra}\AgdaSpace{}%
\AgdaBound{𝓤}\AgdaSpace{}%
\AgdaBound{𝑆}\AgdaSymbol{)}\AgdaSpace{}%
\AgdaFunction{,}\AgdaSpace{}%
\AgdaSymbol{(}\AgdaBound{𝑨}\AgdaSpace{}%
\AgdaOperator{\AgdaFunction{∈}}\AgdaSpace{}%
\AgdaBound{𝒦}\AgdaSymbol{)}\<%
\end{code}
\ccpad
Taking the product over the index type \af ℑ requires a function that maps an index \abt{i}{ℑ} to the corresponding algebra.  Each \abt{i}{ℑ} denotes a pair, (\ab 𝑨 , \ab p), where \ab 𝑨 is an algebra and \ab p is a proof that \ab 𝑨 belongs to \ab 𝒦, so the function mapping such an index to the corresponding algebra is simply the first projection.
\ccpad
\begin{code}
\>[0][@{}l@{\AgdaIndent{0}}]%
\>[1]\AgdaFunction{𝔄}\AgdaSpace{}%
\AgdaSymbol{:}\AgdaSpace{}%
\AgdaFunction{ℑ}\AgdaSpace{}%
\AgdaSymbol{→}\AgdaSpace{}%
\AgdaFunction{Algebra}\AgdaSpace{}%
\AgdaBound{𝓤}\AgdaSpace{}%
\AgdaBound{𝑆}\<%
\\
%
\>[1]\AgdaFunction{𝔄}\AgdaSpace{}%
\AgdaSymbol{=}\AgdaSpace{}%
\AgdaSymbol{λ}\AgdaSpace{}%
\AgdaSymbol{(}\AgdaBound{i}\AgdaSpace{}%
\AgdaSymbol{:}\AgdaSpace{}%
\AgdaFunction{ℑ}\AgdaSymbol{)}\AgdaSpace{}%
\AgdaSymbol{→}\AgdaSpace{}%
\AgdaOperator{\AgdaFunction{∣}}\AgdaSpace{}%
\AgdaBound{i}\AgdaSpace{}%
\AgdaOperator{\AgdaFunction{∣}}\<%
\end{code}
\ccpad
Finally, we represent the product of all members of the class \ab 𝒦 by the following type.
\ccpad
\begin{code}%
\>[0][@{}l@{\AgdaIndent{0}}]%
\>[1]\AgdaFunction{class-product}\AgdaSpace{}%
\AgdaSymbol{:}\AgdaSpace{}%
\AgdaFunction{Algebra}\AgdaSpace{}%
\AgdaSymbol{(}\AgdaFunction{ov}\AgdaSpace{}%
\AgdaBound{𝓤}\AgdaSymbol{)}\AgdaSpace{}%
\AgdaBound{𝑆}\<%
\\
%
\>[1]\AgdaFunction{class-product}\AgdaSpace{}%
\AgdaSymbol{=}\AgdaSpace{}%
\AgdaFunction{⨅}\AgdaSpace{}%
\AgdaFunction{𝔄}\<%
\end{code}
\ccpad
Observe that the application % \af 𝔄 (\ab 𝑨 , \ab p) 
of \af 𝔄 to the pair (\ab 𝑨 , \ab p) (the result of which is simply the algebra \ab{𝑨}) may be viewed as the projection out of the product \af ⨅~\af 𝔄 and onto the ``(\ab 𝑨,~\ab p)-th component'' of that product.



\subsection{Congruences: types for congruences \& quotient algebras}\label{congruences}
\firstsentence{Algebras}{Congruences}
% -*- TeX-master: "ualib-part1.tex" -*-
%%% Local Variables: 
%%% mode: latex
%%% TeX-engine: 'xetex
%%% End:
A \defn{congruence relation} of an algebra \ab{𝑨} is defined to be an equivalence relation that is compatible with the basic operations of \ab 𝑨. This concept can be represented in a number of alternative but equivalent ways.  Informally, a relation is a congruence if and only if it is both an equivalence relation on the domain of \ab 𝑨 and a subalgebra of the square of \ab 𝑨.  Formally, we represent a congruence as an inhabitant of either a the Sigma type \af{Con} or the record type \af{Congruence}, which we now define.
\ccpad
\begin{code}%
\>[0]\AgdaFunction{Con}\AgdaSpace{}%
\AgdaSymbol{:}\AgdaSpace{}%
\AgdaSymbol{\{}\AgdaBound{𝓤}\AgdaSpace{}%
\AgdaSymbol{:}\AgdaSpace{}%
\AgdaFunction{Universe}\AgdaSymbol{\}(}\AgdaBound{𝑨}\AgdaSpace{}%
\AgdaSymbol{:}\AgdaSpace{}%
\AgdaFunction{Algebra}\AgdaSpace{}%
\AgdaBound{𝓤}\AgdaSpace{}%
\AgdaBound{𝑆}\AgdaSymbol{)}\AgdaSpace{}%
\AgdaSymbol{→}\AgdaSpace{}%
\AgdaFunction{ov}\AgdaSpace{}%
\AgdaBound{𝓤}\AgdaSpace{}%
\AgdaOperator{\AgdaFunction{̇}}\<%
\\
\>[0]\AgdaFunction{Con}\AgdaSpace{}%
\AgdaSymbol{\{}\AgdaBound{𝓤}\AgdaSymbol{\}}\AgdaSpace{}%
\AgdaBound{𝑨}\AgdaSpace{}%
\AgdaSymbol{=}\AgdaSpace{}%
\AgdaFunction{Σ}\AgdaSpace{}%
\AgdaBound{θ}\AgdaSpace{}%
\AgdaFunction{꞉}\AgdaSpace{}%
\AgdaSymbol{(}\AgdaSpace{}%
\AgdaFunction{Rel}\AgdaSpace{}%
\AgdaOperator{\AgdaFunction{∣}}\AgdaSpace{}%
\AgdaBound{𝑨}\AgdaSpace{}%
\AgdaOperator{\AgdaFunction{∣}}\AgdaSpace{}%
\AgdaBound{𝓤}\AgdaSpace{}%
\AgdaSymbol{)}\AgdaSpace{}%
\AgdaFunction{,}\AgdaSpace{}%
\AgdaRecord{IsEquivalence}\AgdaSpace{}%
\AgdaBound{θ}\AgdaSpace{}%
\AgdaOperator{\AgdaFunction{×}}\AgdaSpace{}%
\AgdaFunction{compatible}\AgdaSpace{}%
\AgdaBound{𝑨}\AgdaSpace{}%
\AgdaBound{θ}\<%
\\
%
\\[\AgdaEmptyExtraSkip]%
\>[0]\AgdaKeyword{record}\AgdaSpace{}%
\AgdaRecord{Congruence}\AgdaSpace{}%
\AgdaSymbol{\{}\AgdaBound{𝓤}\AgdaSpace{}%
\AgdaBound{𝓦}\AgdaSpace{}%
\AgdaSymbol{:}\AgdaSpace{}%
\AgdaFunction{Universe}\AgdaSymbol{\}}\AgdaSpace{}%
\AgdaSymbol{(}\AgdaBound{𝑨}\AgdaSpace{}%
\AgdaSymbol{:}\AgdaSpace{}%
\AgdaFunction{Algebra}\AgdaSpace{}%
\AgdaBound{𝓤}\AgdaSpace{}%
\AgdaBound{𝑆}\AgdaSymbol{)}\AgdaSpace{}%
\AgdaSymbol{:}\AgdaSpace{}%
\AgdaFunction{ov}\AgdaSpace{}%
\AgdaBound{𝓦}\AgdaSpace{}%
\AgdaOperator{\AgdaFunction{⊔}}\AgdaSpace{}%
\AgdaBound{𝓤}\AgdaSpace{}%
\AgdaOperator{\AgdaFunction{̇}}%
\>[67]\AgdaKeyword{where}\<%
\\
\>[0][@{}l@{\AgdaIndent{0}}]%
\>[1]\AgdaKeyword{constructor}\AgdaSpace{}%
\AgdaInductiveConstructor{mkcon}\<%
\\
%
\>[1]\AgdaKeyword{field}\<%
\\
\>[1][@{}l@{\AgdaIndent{0}}]%
\>[2]\AgdaOperator{\AgdaField{⟨\AgdaUnderscore{}⟩}}\AgdaSpace{}%
\AgdaSymbol{:}\AgdaSpace{}%
\AgdaFunction{Rel}\AgdaSpace{}%
\AgdaOperator{\AgdaFunction{∣}}\AgdaSpace{}%
\AgdaBound{𝑨}\AgdaSpace{}%
\AgdaOperator{\AgdaFunction{∣}}\AgdaSpace{}%
\AgdaBound{𝓦}\<%
\\
%
\>[2]\AgdaField{Compatible}\AgdaSpace{}%
\AgdaSymbol{:}\AgdaSpace{}%
\AgdaFunction{compatible}\AgdaSpace{}%
\AgdaBound{𝑨}\AgdaSpace{}%
\AgdaOperator{\AgdaField{⟨\AgdaUnderscore{}⟩}}\<%
\\
%
\>[2]\AgdaField{IsEquiv}\AgdaSpace{}%
\AgdaSymbol{:}\AgdaSpace{}%
\AgdaRecord{IsEquivalence}\AgdaSpace{}%
\AgdaOperator{\AgdaField{⟨\AgdaUnderscore{}⟩}}\<%
\end{code}
\ccpad
Each of these types captures the informal notion of congruence, and the only real difference between them is that \af{Congruence} includes the extra universe parameter \ab 𝓦 to accommodate more general underlying relations.   Otherwise, the two definitions are equivalent in the sense that each implies the other, as we now prove.
\ccpad
\begin{code}
% \>[0]\AgdaKeyword{module}\AgdaSpace{}%
% \AgdaModule{\AgdaUnderscore{}}\AgdaSpace{}%
% \AgdaSymbol{\{}\AgdaBound{𝑨}\AgdaSpace{}%
% \AgdaSymbol{:}\AgdaSpace{}%
% \AgdaFunction{Algebra}\AgdaSpace{}%
% \AgdaGeneralizable{𝓤}\AgdaSpace{}%
% \AgdaBound{𝑆}\AgdaSymbol{\}}\AgdaSpace{}%
% \AgdaKeyword{where}\<%
% \\
% %
% \\[\AgdaEmptyExtraSkip]%
\>[0][@{}l@{\AgdaIndent{0}}]%
\>[1]\AgdaFunction{Con→Congruence}\AgdaSpace{}%
\AgdaSymbol{:}\AgdaSpace{}%
\AgdaFunction{Con}\AgdaSpace{}%
\AgdaBound{𝑨}\AgdaSpace{}%
\AgdaSymbol{→}\AgdaSpace{}%
\AgdaRecord{Congruence}\AgdaSymbol{\{}\AgdaBound{𝓤}\AgdaSymbol{\}}\AgdaSpace{}%
\AgdaBound{𝑨}\<%
\\
%
\>[1]\AgdaFunction{Con→Congruence}\AgdaSpace{}%
\AgdaBound{θ}\AgdaSpace{}%
\AgdaSymbol{=}\AgdaSpace{}%
\AgdaInductiveConstructor{mkcon}\AgdaSpace{}%
\AgdaOperator{\AgdaFunction{∣}}\AgdaSpace{}%
\AgdaBound{θ}\AgdaSpace{}%
\AgdaOperator{\AgdaFunction{∣}}\AgdaSpace{}%
\AgdaSymbol{(}\AgdaFunction{fst}\AgdaSpace{}%
\AgdaOperator{\AgdaFunction{∥}}\AgdaSpace{}%
\AgdaBound{θ}\AgdaSpace{}%
\AgdaOperator{\AgdaFunction{∥}}\AgdaSymbol{)}\AgdaSpace{}%
\AgdaSymbol{(}\AgdaFunction{snd}\AgdaSpace{}%
\AgdaOperator{\AgdaFunction{∥}}\AgdaSpace{}%
\AgdaBound{θ}\AgdaSpace{}%
\AgdaOperator{\AgdaFunction{∥}}\AgdaSymbol{)}\<%
\\
%
% \\[\AgdaEmptyExtraSkip]%
% %
% \>[1]\AgdaKeyword{open}\AgdaSpace{}%
% \AgdaModule{Congruence}\<%
% \\
% %
\\[\AgdaEmptyExtraSkip]%
% %
\>[1]\AgdaFunction{Congruence→Con}\AgdaSpace{}%
\AgdaSymbol{:}\AgdaSpace{}%
\AgdaRecord{Congruence}\AgdaSymbol{\{}\AgdaBound{𝓤}\AgdaSymbol{\}}\AgdaSpace{}%
\AgdaBound{𝑨}\AgdaSpace{}%
\AgdaSymbol{→}%
\>[37]\AgdaFunction{Con}\AgdaSpace{}%
\AgdaBound{𝑨}\<%
\\
%
\>[1]\AgdaFunction{Congruence→Con}\AgdaSpace{}%
\AgdaBound{θ}\AgdaSpace{}%
\AgdaSymbol{=}\AgdaSpace{}%
\AgdaOperator{\AgdaField{⟨}}\AgdaSpace{}%
\AgdaBound{θ}\AgdaSpace{}%
\AgdaOperator{\AgdaField{⟩}}\AgdaSpace{}%
\AgdaOperator{\AgdaInductiveConstructor{,}}\AgdaSpace{}%
\AgdaField{IsEquiv}\AgdaSpace{}%
\AgdaBound{θ}\AgdaSpace{}%
\AgdaOperator{\AgdaInductiveConstructor{,}}\AgdaSpace{}%
\AgdaField{Compatible}\AgdaSpace{}%
\AgdaBound{θ}\<%
\end{code}
\ccpad
We defined the \defn{zero relation} \af{𝟎-rel} in the \ualibhtml{Relations.Discrete} module. We defined the zero relation `𝟎-rel` in the [Relations.Discrete][] module.  We now build the \defn{trivial congruence}, which has \af{𝟎-rel} as its underlying relation. Observe that \af{𝟎-rel} is equivalent to the identity relation \ad{≡} and these are obviously both equivalences. In fact, we already proved this of \ad{≡} in the \ualibhtml{Overture.Equality} module, so we simply apply the corresponding proofs.
\ccpad
\begin{code}%
\>[0]\AgdaFunction{𝟎-IsEquivalence}\AgdaSpace{}%
\AgdaSymbol{:}\AgdaSpace{}%
\AgdaSymbol{\{}\AgdaBound{A}\AgdaSpace{}%
\AgdaSymbol{:}\AgdaSpace{}%
\AgdaGeneralizable{𝓤}\AgdaSpace{}%
\AgdaOperator{\AgdaFunction{̇}}\AgdaSymbol{\}}\AgdaSpace{}%
\AgdaSymbol{→}%
\>[31]\AgdaRecord{IsEquivalence}\AgdaSpace{}%
\AgdaSymbol{\{}\AgdaArgument{A}\AgdaSpace{}%
\AgdaSymbol{=}\AgdaSpace{}%
\AgdaBound{A}\AgdaSymbol{\}}\AgdaSpace{}%
\AgdaFunction{𝟎}\<%
\\
\>[0]\AgdaFunction{𝟎-IsEquivalence}\AgdaSpace{}%
\AgdaSymbol{=}\AgdaSpace{}%
\AgdaKeyword{record}\AgdaSpace{}%
\AgdaSymbol{\{}\AgdaField{rfl}\AgdaSpace{}%
\AgdaSymbol{=}\AgdaSpace{}%
\AgdaSymbol{λ}\AgdaSpace{}%
\AgdaBound{x}\AgdaSpace{}%
\AgdaSymbol{→}\AgdaSpace{}%
\AgdaInductiveConstructor{refl}\AgdaSymbol{\{}\AgdaArgument{x}\AgdaSpace{}%
\AgdaSymbol{=}\AgdaSpace{}%
\AgdaBound{x}\AgdaSymbol{\};}\AgdaSpace{}%
\AgdaField{sym}\AgdaSpace{}%
\AgdaSymbol{=}\AgdaSpace{}%
\AgdaFunction{≡-symmetric}\AgdaSymbol{;}\AgdaSpace{}%
\AgdaField{trans}\AgdaSpace{}%
\AgdaSymbol{=}\AgdaSpace{}%
\AgdaFunction{≡-transitive}\AgdaSymbol{\}}\<%
\end{code}
\ccpad
Next we formally record another obvious fact---namely, that \af{𝟎-rel} is compatible with all operations of all algebras.
\ccpad
\begin{code}%
\>[0]\AgdaFunction{𝟎-compatible-op}\AgdaSpace{}%
\AgdaSymbol{:}\AgdaSpace{}%
\AgdaFunction{funext}\AgdaSpace{}%
\AgdaBound{𝓥}\AgdaSpace{}%
\AgdaGeneralizable{𝓤}\AgdaSpace{}%
\AgdaSymbol{→}\AgdaSpace{}%
\AgdaSymbol{\{}\AgdaBound{𝑨}\AgdaSpace{}%
\AgdaSymbol{:}\AgdaSpace{}%
\AgdaFunction{Algebra}\AgdaSpace{}%
\AgdaGeneralizable{𝓤}\AgdaSpace{}%
\AgdaBound{𝑆}\AgdaSymbol{\}}\AgdaSpace{}%
\AgdaSymbol{(}\AgdaBound{𝑓}\AgdaSpace{}%
\AgdaSymbol{:}\AgdaSpace{}%
\AgdaOperator{\AgdaFunction{∣}}\AgdaSpace{}%
\AgdaBound{𝑆}\AgdaSpace{}%
\AgdaOperator{\AgdaFunction{∣}}\AgdaSymbol{)}\AgdaSpace{}%
\AgdaSymbol{→}\AgdaSpace{}%
\AgdaFunction{compatible-fun}\AgdaSpace{}%
\AgdaSymbol{(}\AgdaBound{𝑓}\AgdaSpace{}%
\AgdaOperator{\AgdaFunction{̂}}\AgdaSpace{}%
\AgdaBound{𝑨}\AgdaSymbol{)}\AgdaSpace{}%
\AgdaFunction{𝟎}\<%
\\
\>[0]\AgdaFunction{𝟎-compatible-op}\AgdaSpace{}%
\AgdaBound{fe}\AgdaSpace{}%
\AgdaSymbol{\{}\AgdaBound{𝑨}\AgdaSymbol{\}}\AgdaSpace{}%
\AgdaBound{𝑓}\AgdaSpace{}%
\AgdaBound{ptws0}%
\>[32]\AgdaSymbol{=}\AgdaSpace{}%
\AgdaFunction{ap}\AgdaSpace{}%
\AgdaSymbol{(}\AgdaBound{𝑓}\AgdaSpace{}%
\AgdaOperator{\AgdaFunction{̂}}\AgdaSpace{}%
\AgdaBound{𝑨}\AgdaSymbol{)}\AgdaSpace{}%
\AgdaSymbol{(}\AgdaBound{fe}\AgdaSpace{}%
\AgdaSymbol{(λ}\AgdaSpace{}%
\AgdaBound{x}\AgdaSpace{}%
\AgdaSymbol{→}\AgdaSpace{}%
\AgdaBound{ptws0}\AgdaSpace{}%
\AgdaBound{x}\AgdaSymbol{))}\<%
\\
%
\\[\AgdaEmptyExtraSkip]%
\>[0]\AgdaFunction{𝟎-compatible}\AgdaSpace{}%
\AgdaSymbol{:}\AgdaSpace{}%
\AgdaFunction{funext}\AgdaSpace{}%
\AgdaBound{𝓥}\AgdaSpace{}%
\AgdaGeneralizable{𝓤}\AgdaSpace{}%
\AgdaSymbol{→}\AgdaSpace{}%
\AgdaSymbol{\{}\AgdaBound{𝑨}\AgdaSpace{}%
\AgdaSymbol{:}\AgdaSpace{}%
\AgdaFunction{Algebra}\AgdaSpace{}%
\AgdaGeneralizable{𝓤}\AgdaSpace{}%
\AgdaBound{𝑆}\AgdaSymbol{\}}\AgdaSpace{}%
\AgdaSymbol{→}\AgdaSpace{}%
\AgdaFunction{compatible}\AgdaSpace{}%
\AgdaBound{𝑨}\AgdaSpace{}%
\AgdaFunction{𝟎}\<%
\\
\>[0]\AgdaFunction{𝟎-compatible}\AgdaSpace{}%
\AgdaBound{fe}\AgdaSpace{}%
\AgdaSymbol{\{}\AgdaBound{𝑨}\AgdaSymbol{\}}\AgdaSpace{}%
\AgdaSymbol{=}\AgdaSpace{}%
\AgdaSymbol{λ}\AgdaSpace{}%
\AgdaBound{𝑓}\AgdaSpace{}%
\AgdaBound{args}\AgdaSpace{}%
\AgdaSymbol{→}\AgdaSpace{}%
\AgdaFunction{𝟎-compatible-op}\AgdaSpace{}%
\AgdaBound{fe}\AgdaSpace{}%
\AgdaSymbol{\{}\AgdaBound{𝑨}\AgdaSymbol{\}}\AgdaSpace{}%
\AgdaBound{𝑓}\AgdaSpace{}%
\AgdaBound{args}\<%
\end{code}
\ccpad
Finally, we have the ingredients need to construct the zero congruence of any algebra we like.
\ccpad
\begin{code}%
\>[0]\AgdaFunction{Δ}\AgdaSpace{}%
\AgdaSymbol{:}\AgdaSpace{}%
\AgdaFunction{funext}\AgdaSpace{}%
\AgdaBound{𝓥}\AgdaSpace{}%
\AgdaGeneralizable{𝓤}\AgdaSpace{}%
\AgdaSymbol{→}\AgdaSpace{}%
\AgdaSymbol{\{}\AgdaBound{𝑨}\AgdaSpace{}%
\AgdaSymbol{:}\AgdaSpace{}%
\AgdaFunction{Algebra}\AgdaSpace{}%
\AgdaGeneralizable{𝓤}\AgdaSpace{}%
\AgdaBound{𝑆}\AgdaSymbol{\}}\AgdaSpace{}%
\AgdaSymbol{→}\AgdaSpace{}%
\AgdaRecord{Congruence}\AgdaSpace{}%
\AgdaBound{𝑨}\<%
\\
\>[0]\AgdaFunction{Δ}\AgdaSpace{}%
\AgdaBound{fe}\AgdaSpace{}%
\AgdaSymbol{=}\AgdaSpace{}%
\AgdaInductiveConstructor{mkcon}\AgdaSpace{}%
\AgdaFunction{𝟎}\AgdaSpace{}%
\AgdaSymbol{(}\AgdaFunction{𝟎-compatible}\AgdaSpace{}%
\AgdaBound{fe}\AgdaSymbol{)}\AgdaSpace{}%
\AgdaFunction{𝟎-IsEquivalence}\<%
\end{code}
\ccpad
A concrete example is \af{⟪𝟎⟫}~\ab 𝑨~\af ╱~\ab θ, presented in the next subsection.

\subsubsection{Quotient Algebras}\label{quotient-algebras}
In many areas of abstract mathematics the \defn{quotient} of an algebra \ab 𝑨 with respect to a congruence relation \ab θ of \ab 𝑨 plays an important role. This quotient is typically denoted by \ab 𝑨 \af / \ab θ and Agda allows us to define and express quotients using this standard notation.\footnote{%
\textbf{Unicode Hints}. Produce the ╱ symbol in \agdamode by typing \texttt{\textbackslash{}-\/-\/-} and then \texttt{C-f} a number of times.}
\ccpad
\begin{code}%
\>[0][@{}l@{\AgdaIndent{0}}]%
\>[1]\AgdaOperator{\AgdaFunction{\AgdaUnderscore{}╱\AgdaUnderscore{}}}\AgdaSpace{}%
\AgdaSymbol{:}\AgdaSpace{}%
\AgdaSymbol{(}\AgdaBound{𝑨}\AgdaSpace{}%
\AgdaSymbol{:}\AgdaSpace{}%
\AgdaFunction{Algebra}\AgdaSpace{}%
\AgdaBound{𝓤}\AgdaSpace{}%
\AgdaBound{𝑆}\AgdaSymbol{)}\AgdaSpace{}%
\AgdaSymbol{→}\AgdaSpace{}%
\AgdaRecord{Congruence}\AgdaSymbol{\{}\AgdaBound{𝓦}\AgdaSymbol{\}}\AgdaSpace{}%
\AgdaBound{𝑨}\AgdaSpace{}%
\AgdaSymbol{→}\AgdaSpace{}%
\AgdaFunction{Algebra}\AgdaSpace{}%
\AgdaSymbol{(}\AgdaBound{𝓤}\AgdaSpace{}%
\AgdaOperator{\AgdaFunction{⊔}}\AgdaSpace{}%
\AgdaBound{𝓦}\AgdaSpace{}%
\AgdaOperator{\AgdaFunction{⁺}}\AgdaSymbol{)}\AgdaSpace{}%
\AgdaBound{𝑆}\<%
\\
%
\\[\AgdaEmptyExtraSkip]%
%
\>[1]\AgdaBound{𝑨}\AgdaSpace{}%
\AgdaOperator{\AgdaFunction{╱}}\AgdaSpace{}%
\AgdaBound{θ}\AgdaSpace{}%
\AgdaSymbol{=}%
\>[258I]\AgdaSymbol{(}\AgdaSpace{}%
\AgdaOperator{\AgdaFunction{∣}}\AgdaSpace{}%
\AgdaBound{𝑨}\AgdaSpace{}%
\AgdaOperator{\AgdaFunction{∣}}\AgdaSpace{}%
\AgdaOperator{\AgdaFunction{/}}\AgdaSpace{}%
\AgdaOperator{\AgdaField{⟨}}\AgdaSpace{}%
\AgdaBound{θ}\AgdaSpace{}%
\AgdaOperator{\AgdaField{⟩}}\AgdaSpace{}%
\AgdaSymbol{)}\AgdaSpace{}%
\AgdaOperator{\AgdaInductiveConstructor{,}}%
\>[49]\AgdaComment{-- the domain of the quotient algebra}\<%
\\
%
\\[\AgdaEmptyExtraSkip]%
\>[.][@{}l@{}]\<[258I]%
\>[9]\AgdaSymbol{λ}\AgdaSpace{}%
\AgdaBound{𝑓}\AgdaSpace{}%
\AgdaBound{𝒂}\AgdaSpace{}%
\AgdaSymbol{→}\AgdaSpace{}%
\AgdaOperator{\AgdaFunction{⟪}}\AgdaSymbol{(}\AgdaBound{𝑓}\AgdaSpace{}%
\AgdaOperator{\AgdaFunction{̂}}\AgdaSpace{}%
\AgdaBound{𝑨}\AgdaSymbol{)}\AgdaSpace{}%
\AgdaSymbol{(λ}\AgdaSpace{}%
\AgdaBound{i}\AgdaSpace{}%
\AgdaSymbol{→}\AgdaSpace{}%
\AgdaOperator{\AgdaFunction{∣}}\AgdaSpace{}%
\AgdaOperator{\AgdaFunction{∥}}\AgdaSpace{}%
\AgdaBound{𝒂}\AgdaSpace{}%
\AgdaBound{i}\AgdaSpace{}%
\AgdaOperator{\AgdaFunction{∥}}\AgdaSpace{}%
\AgdaOperator{\AgdaFunction{∣}}\AgdaSymbol{)}\AgdaOperator{\AgdaFunction{⟫}}%
\>[49]\AgdaComment{-- the basic operations of the quotient algebra}\<%
\end{code}
\ccpad
\textbf{Example}.
Denote by \af{𝟎[} \ab 𝑨~\af ╱~\ab θ~\af ] the zero relation on the quotient algebra \ab 𝑨~\af ╱~\ab θ, which is defined as follows.
\ccpad
\begin{code}%
\>[0][@{}l@{\AgdaIndent{0}}]%
\>[1]\AgdaOperator{\AgdaFunction{𝟎[\AgdaUnderscore{}╱\AgdaUnderscore{}]}}\AgdaSpace{}%
\AgdaSymbol{:}\AgdaSpace{}%
\AgdaSymbol{(}\AgdaBound{𝑨}\AgdaSpace{}%
\AgdaSymbol{:}\AgdaSpace{}%
\AgdaFunction{Algebra}\AgdaSpace{}%
\AgdaBound{𝓤}\AgdaSpace{}%
\AgdaBound{𝑆}\AgdaSymbol{)(}\AgdaBound{θ}\AgdaSpace{}%
\AgdaSymbol{:}\AgdaSpace{}%
\AgdaRecord{Congruence}\AgdaSymbol{\{}\AgdaBound{𝓦}\AgdaSymbol{\}}\AgdaSpace{}%
\AgdaBound{𝑨}\AgdaSymbol{)}\AgdaSpace{}%
\AgdaSymbol{→}\AgdaSpace{}%
\AgdaFunction{Rel}\AgdaSpace{}%
\AgdaSymbol{(}\AgdaOperator{\AgdaFunction{∣}}\AgdaSpace{}%
\AgdaBound{𝑨}\AgdaSpace{}%
\AgdaOperator{\AgdaFunction{∣}}\AgdaSpace{}%
\AgdaOperator{\AgdaFunction{/}}\AgdaSpace{}%
\AgdaOperator{\AgdaField{⟨}}\AgdaSpace{}%
\AgdaBound{θ}\AgdaSpace{}%
\AgdaOperator{\AgdaField{⟩}}\AgdaSymbol{)(}\AgdaBound{𝓤}\AgdaSpace{}%
\AgdaOperator{\AgdaFunction{⊔}}\AgdaSpace{}%
\AgdaBound{𝓦}\AgdaSpace{}%
\AgdaOperator{\AgdaFunction{⁺}}\AgdaSymbol{)}\<%
\\
%
\>[1]\AgdaOperator{\AgdaFunction{𝟎[}}\AgdaSpace{}%
\AgdaBound{𝑨}\AgdaSpace{}%
\AgdaOperator{\AgdaFunction{╱}}\AgdaSpace{}%
\AgdaBound{θ}\AgdaSpace{}%
\AgdaOperator{\AgdaFunction{]}}\AgdaSpace{}%
\AgdaSymbol{=}\AgdaSpace{}%
\AgdaSymbol{λ}\AgdaSpace{}%
\AgdaBound{u}\AgdaSpace{}%
\AgdaBound{v}\AgdaSpace{}%
\AgdaSymbol{→}\AgdaSpace{}%
\AgdaBound{u}\AgdaSpace{}%
\AgdaOperator{\AgdaDatatype{≡}}\AgdaSpace{}%
\AgdaBound{v}\<%
\end{code}
\ccpad
From this we obtain the zero congruence of \ab 𝑨~\af ╱~\ab θ by applying the \af Δ function defined above.
\ccpad
\begin{code}%
\>[0][@{}l@{\AgdaIndent{0}}]%
\>[1]\AgdaOperator{\AgdaFunction{⟪𝟎⟫\AgdaUnderscore{}╱\AgdaUnderscore{}}}\AgdaSpace{}%
\AgdaSymbol{:}\AgdaSpace{}%
\>[259I]\AgdaSymbol{(}\AgdaBound{𝑨}\AgdaSpace{}%
\AgdaSymbol{:}\AgdaSpace{}%
\AgdaFunction{Algebra}\AgdaSpace{}%
\AgdaBound{𝓤}\AgdaSpace{}%
\AgdaBound{𝑆}\AgdaSymbol{)(}\AgdaBound{θ}\AgdaSpace{}%
\AgdaSymbol{:}\AgdaSpace{}%
\AgdaRecord{Congruence}\AgdaSymbol{\{}\AgdaBound{𝓦}\AgdaSymbol{\}}\AgdaSpace{}%
\AgdaBound{𝑨}\AgdaSymbol{)\{}\AgdaBound{fe}\AgdaSpace{}%
\AgdaSymbol{:}\AgdaSpace{}%
\AgdaFunction{funext}\AgdaSpace{}%
\AgdaBound{𝓥}\AgdaSpace{}%
\AgdaSymbol{(}\AgdaBound{𝓤}\AgdaSpace{}%
\AgdaOperator{\AgdaFunction{⊔}}\AgdaSpace{}%
\AgdaBound{𝓦}\AgdaSpace{}%
\AgdaOperator{\AgdaFunction{⁺}}\AgdaSymbol{)\}}\<%
\\
\>[1][@{}l@{\AgdaIndent{0}}]%
\>[2]\AgdaSymbol{→}%
\>[.][@{}l@{}]\<[259I]%
\>[14]\AgdaRecord{Congruence}\AgdaSpace{}%
\AgdaSymbol{(}\AgdaBound{𝑨}\AgdaSpace{}%
\AgdaOperator{\AgdaFunction{╱}}\AgdaSpace{}%
\AgdaBound{θ}\AgdaSymbol{)}\<%
\\
%
\>[1]\AgdaSymbol{(}\AgdaOperator{\AgdaFunction{⟪𝟎⟫}}\AgdaSpace{}%
\AgdaBound{𝑨}\AgdaSpace{}%
\AgdaOperator{\AgdaFunction{╱}}\AgdaSpace{}%
\AgdaBound{θ}\AgdaSymbol{)}\AgdaSpace{}%
\AgdaSymbol{\{}\AgdaBound{fe}\AgdaSymbol{\}}\AgdaSpace{}%
\AgdaSymbol{=}\AgdaSpace{}%
\AgdaFunction{Δ}\AgdaSpace{}%
\AgdaBound{fe}\<%
\end{code}
\ccpad
Finally, the following \defn{elimination rule} is sometimes useful.
\ccpad
\begin{code}%
% \>[0]\AgdaKeyword{module}\AgdaSpace{}%
% \AgdaModule{\AgdaUnderscore{}}\AgdaSpace{}%
% \AgdaSymbol{\{}\AgdaBound{𝓤}\AgdaSpace{}%
% \AgdaBound{𝓦}\AgdaSpace{}%
% \AgdaSymbol{:}\AgdaSpace{}%
% \AgdaFunction{Universe}\AgdaSymbol{\}\{}\AgdaBound{𝑨}\AgdaSpace{}%
% \AgdaSymbol{:}\AgdaSpace{}%
% \AgdaFunction{Algebra}\AgdaSpace{}%
% \AgdaBound{𝓤}\AgdaSpace{}%
% \AgdaBound{𝑆}\AgdaSymbol{\}}\AgdaSpace{}%
% \AgdaKeyword{where}\<%
% \\
% %
% \\[\AgdaEmptyExtraSkip]%
\>[0][@{}l@{\AgdaIndent{0}}]%
\>[1]\AgdaFunction{╱-≡}\AgdaSpace{}%
\AgdaSymbol{:}\AgdaSpace{}%
\AgdaSymbol{(}\AgdaBound{θ}\AgdaSpace{}%
\AgdaSymbol{:}\AgdaSpace{}%
\AgdaRecord{Congruence}\AgdaSymbol{\{}\AgdaBound{𝓦}\AgdaSymbol{\}}\AgdaSpace{}%
\AgdaBound{𝑨}\AgdaSymbol{)\{}\AgdaBound{u}\AgdaSpace{}%
\AgdaBound{v}\AgdaSpace{}%
\AgdaSymbol{:}\AgdaSpace{}%
\AgdaOperator{\AgdaFunction{∣}}\AgdaSpace{}%
\AgdaBound{𝑨}\AgdaSpace{}%
\AgdaOperator{\AgdaFunction{∣}}\AgdaSymbol{\}}\AgdaSpace{}%
\AgdaSymbol{→}\AgdaSpace{}%
\AgdaOperator{\AgdaFunction{⟪}}\AgdaSpace{}%
\AgdaBound{u}\AgdaSpace{}%
\AgdaOperator{\AgdaFunction{⟫}}\AgdaSymbol{\{}\AgdaOperator{\AgdaField{⟨}}\AgdaSpace{}%
\AgdaBound{θ}\AgdaSpace{}%
\AgdaOperator{\AgdaField{⟩}}\AgdaSymbol{\}}\AgdaSpace{}%
\AgdaOperator{\AgdaDatatype{≡}}\AgdaSpace{}%
\AgdaOperator{\AgdaFunction{⟪}}\AgdaSpace{}%
\AgdaBound{v}\AgdaSpace{}%
\AgdaOperator{\AgdaFunction{⟫}}\AgdaSpace{}%
\AgdaSymbol{→}\AgdaSpace{}%
\AgdaOperator{\AgdaField{⟨}}\AgdaSpace{}%
\AgdaBound{θ}\AgdaSpace{}%
\AgdaOperator{\AgdaField{⟩}}\AgdaSpace{}%
\AgdaBound{u}\AgdaSpace{}%
\AgdaBound{v}\<%
\\
%
\>[1]\AgdaFunction{╱-≡}\AgdaSpace{}%
\AgdaBound{θ}\AgdaSpace{}%
\AgdaInductiveConstructor{refl}\AgdaSpace{}%
\AgdaSymbol{=}\AgdaSpace{}%
\AgdaField{IsEquivalence.rfl}\AgdaSpace{}%
\AgdaSymbol{(}\AgdaField{IsEquiv}\AgdaSpace{}%
\AgdaBound{θ}\AgdaSymbol{)}\AgdaSpace{}%
\AgdaSymbol{\AgdaUnderscore{}}\<%
\end{code}


% % -*- TeX-master: "ualib-part1.tex" -*-
%%% Local Variables: 
%%% mode: latex
%%% TeX-engine: 'xetex
%%% End:
\section{Concluding Remarks}\label{sec:concluding-remarks}
We've reached the end of Part 1 of our three-part series describing the \agdaualib.  Part 2 will cover homomorphism, terms, and subalgebras, and Part 3 will cover free algebras, equational classes of algebras (i.e., varieties), and Birkhoff's HSP theorem.

% We completed the first major milestone of this project in January 2021, which was the first formal, machine-checked proof of Birkhoff's HSP theorem.  Our next goal is to support current mathematical research by formalizing some recently proved theorems. For example, we intend to formalize theorems about the computational complexity of decidable properties of algebraic structures. Part of this effort will naturally involve further work on the library itself, and may lead to new observations or discoveries in dependent type theory or homotopy type theory.

%% One natural question is whether there are any objects of our research that cannot be represented constructively in type theory. % and, if so, whether these have suitable constructive substitutes.
%% %Along these lines, %% we should revisit and perhaps refine our proof of Birkhoff's theorem. A
%% As mentioned in \S~\ref{sec:contributions}, a constructive version of Birkhoff's theorem was presented by Carlstr\"om in~\cite{Carlstrom:2008}. We would like to know, for example, how the two new hypotheses required by Carlstr\"om %% adds to the classical theorem in order to make it constructive 
%% compare with the assumptions we make in our Agda proof of this result.

We conclude by noting that one of our goals is to make computer formalization of mathematics more accessible to mathematicians working in universal algebra and model theory. We welcome feedback from the community and are happy to field questions about the \ualib, how it is installed, and how it can be used to prove theorems that are not yet part of the library.  Merge requests submitted to the UALib's main gitlab repository are especially welcomed.  Please visit the repository at \url{https://gitlab.com/ualib/ualib.gitlab.io/} and help us improve it.





\paragraph*{Acknowledgments}
This work would not have been possible without the wonderful \agda language and
the \agdastdlib, developed and maintained by The Agda Team~\cite{agdastdlib}.
We thank the three anonymous referees for carefully reading the manuscript and
offering many excellent suggestions which resulted in a vast improvement in the
overall presentation.  One referee went above and beyond and provided us with a
simpler formalization of free algebras, in particular of \af{𝔽[~\ab{X}~]}.
This is based on image factorisation (i.e. epi/mono factorisation). This in
turn lead to simplifications of the proofs of the main theorem. We are
extremely grateful for this.
The first author was supported by the CoCoSym Project
under the ERC Consolidator Grant (ERC CoG), No.~771005.







%%%%%%%%%%%%%%%%%%%%%%%%%%%%%%%%%%%%%%%%%%%%%%%%%%%%%%%%%%%%%%%%%%%%%%%%%%%%%%%%%%%%%%%%%%
%%%%%%%%%%%%%%%%%%%%%%%%%%%%%%%%%%%%%%%%%%%%%%%%%%%%%%%%%%%%%%%%%%%%%%%%%%%%%%%%%%%%%%%%%%

%% Bibliography
\bibliographystyle{plainurl}

\bibliography{ualib_refs}

% \appendix
% \newpage
% \section{Dependency Graph}

% ~\hskip-1.5cm\includegraphics[scale=0.65]{ualib-graph-top.png}

% \newpage

% ~\hskip-2.5cm\includegraphics[scale=0.55]{ualib-graph-bot.png}


%% \section{Some Components of the Type Topology Library}
%% Here we collect some of the components from the \typetopology library that we used above but did not have space to discuss.  They are collected here for the reader's convenience and to keep the paper somewhat self-contained.

%% \begin{code}
\>[0]\AgdaFunction{transport}%
\>[1275I]\AgdaSymbol{:}\AgdaSpace{}%
\AgdaSymbol{\{}\AgdaBound{X}\AgdaSpace{}%
\AgdaSymbol{:}\AgdaSpace{}%
\AgdaGeneralizable{𝓤}\AgdaSpace{}%
\AgdaOperator{\AgdaFunction{̇}}\AgdaSpace{}%
\AgdaSymbol{\}}\AgdaSpace{}%
\AgdaSymbol{(}\AgdaBound{A}\AgdaSpace{}%
\AgdaSymbol{:}\AgdaSpace{}%
\AgdaBound{X}\AgdaSpace{}%
\AgdaSymbol{→}\AgdaSpace{}%
\AgdaGeneralizable{𝓥}\AgdaSpace{}%
\AgdaOperator{\AgdaFunction{̇}}\AgdaSpace{}%
\AgdaSymbol{)}\AgdaSpace{}%
\AgdaSymbol{\{}\AgdaBound{x}\AgdaSpace{}%
\AgdaBound{y}\AgdaSpace{}%
\AgdaSymbol{:}\AgdaSpace{}%
\AgdaBound{X}\AgdaSymbol{\}}\<%
\\
\>[.][@{}l@{}]\<[1275I]%
\>[10]\AgdaSymbol{→}\AgdaSpace{}%
\AgdaBound{x}\AgdaSpace{}%
\AgdaOperator{\AgdaDatatype{≡}}\AgdaSpace{}%
\AgdaBound{y}\AgdaSpace{}%
\AgdaSymbol{→}\AgdaSpace{}%
\AgdaBound{A}\AgdaSpace{}%
\AgdaBound{x}\AgdaSpace{}%
\AgdaSymbol{→}\AgdaSpace{}%
\AgdaBound{A}\AgdaSpace{}%
\AgdaBound{y}\<%
\\
%
\\[\AgdaEmptyExtraSkip]%
\>[0]\AgdaFunction{transport}\AgdaSpace{}%
\AgdaBound{A}\AgdaSpace{}%
\AgdaSymbol{(}\AgdaInductiveConstructor{refl}\AgdaSpace{}%
\AgdaBound{x}\AgdaSymbol{)}\AgdaSpace{}%
\AgdaSymbol{=}\AgdaSpace{}%
\AgdaFunction{𝑖𝑑}\AgdaSpace{}%
\AgdaSymbol{(}\AgdaBound{A}\AgdaSpace{}%
\AgdaBound{x}\AgdaSymbol{)}\<%
\end{code}
\scpad
\begin{code}
  \>[0]\AgdaFunction{to-Σ-≡}%
\>[1596I]\AgdaSymbol{:}\AgdaSpace{}%
\AgdaSymbol{\{}\AgdaBound{X}\AgdaSpace{}%
\AgdaSymbol{:}\AgdaSpace{}%
\AgdaGeneralizable{𝓤}\AgdaSpace{}%
\AgdaOperator{\AgdaFunction{̇}}\AgdaSpace{}%
\AgdaSymbol{\}}\AgdaSpace{}%
\AgdaSymbol{\{}\AgdaBound{A}\AgdaSpace{}%
\AgdaSymbol{:}\AgdaSpace{}%
\AgdaBound{X}\AgdaSpace{}%
\AgdaSymbol{→}\AgdaSpace{}%
\AgdaGeneralizable{𝓥}\AgdaSpace{}%
\AgdaOperator{\AgdaFunction{̇}}\AgdaSpace{}%
\AgdaSymbol{\}}\AgdaSpace{}%
\AgdaSymbol{\{}\AgdaBound{σ}\AgdaSpace{}%
\AgdaBound{τ}\AgdaSpace{}%
\AgdaSymbol{:}\AgdaSpace{}%
\AgdaRecord{Σ}\AgdaSpace{}%
\AgdaBound{A}\AgdaSymbol{\}}\<%
\\
\>[.][@{}l@{}]\<[1596I]%
\>[7]\AgdaSymbol{→}\AgdaSpace{}%
\AgdaSymbol{(}\AgdaFunction{Σ}\AgdaSpace{}%
\AgdaBound{p}\AgdaSpace{}%
\AgdaFunction{꞉}\AgdaSpace{}%
\AgdaFunction{pr₁}\AgdaSpace{}%
\AgdaBound{σ}\AgdaSpace{}%
\AgdaOperator{\AgdaDatatype{≡}}\AgdaSpace{}%
\AgdaFunction{pr₁}\AgdaSpace{}%
\AgdaBound{τ}\AgdaSpace{}%
\AgdaFunction{,}\AgdaSpace{}%
\AgdaFunction{transport}\AgdaSpace{}%
\AgdaBound{A}\AgdaSpace{}%
\AgdaBound{p}\AgdaSpace{}%
\AgdaSymbol{(}\AgdaFunction{pr₂}\AgdaSpace{}%
\AgdaBound{σ}\AgdaSymbol{)}\AgdaSpace{}%
\AgdaOperator{\AgdaDatatype{≡}}\AgdaSpace{}%
\AgdaFunction{pr₂}\AgdaSpace{}%
\AgdaBound{τ}\AgdaSymbol{)}\<%
\\
%
\>[7]\AgdaSymbol{→}\AgdaSpace{}%
\AgdaBound{σ}\AgdaSpace{}%
\AgdaOperator{\AgdaDatatype{≡}}\AgdaSpace{}%
\AgdaBound{τ}\<%
\\
%
\\[\AgdaEmptyExtraSkip]%
\>[0]\AgdaFunction{to-Σ-≡}\AgdaSpace{}%
\AgdaSymbol{(}\AgdaInductiveConstructor{refl}\AgdaSpace{}%
\AgdaBound{x}\AgdaSpace{}%
\AgdaOperator{\AgdaInductiveConstructor{,}}\AgdaSpace{}%
\AgdaInductiveConstructor{refl}\AgdaSpace{}%
\AgdaBound{a}\AgdaSymbol{)}\AgdaSpace{}%
\AgdaSymbol{=}\AgdaSpace{}%
\AgdaInductiveConstructor{refl}\AgdaSpace{}%
\AgdaSymbol{(}\AgdaBound{x}\AgdaSpace{}%
\AgdaOperator{\AgdaInductiveConstructor{,}}\AgdaSpace{}%
\AgdaBound{a}\AgdaSymbol{)}\<%
\end{code}


\subsubsection{Dependent function extensionality}\label{dependent-function-extensionality}

dfunext : ∀ 𝓤 𝓥 → (𝓤 ⊔ 𝓥)⁺ ̇
dfunext 𝓤 𝓥 = {X : 𝓤 ̇ } {A : X → 𝓥 ̇ } {f g : Π A} → f ∼ g → f ≡ g

global-dfunext : 𝓤ω
global-dfunext = ∀ {𝓤 𝓥} → dfunext 𝓤 𝓥

happly : {X : 𝓤 ̇ } {A : X → 𝓥 ̇ } (f g : Π A) → f ≡ g → f ∼ g
happly f g p x = ap (λ - → - x) p

Extensionality for dependent function types is defined as follows.
\ccpad
\begin{code}%
\>[0]\AgdaFunction{dep-extensionality}\AgdaSpace{}%
\AgdaSymbol{:}\AgdaSpace{}%
\AgdaSymbol{∀}\AgdaSpace{}%
\AgdaBound{𝓤}\AgdaSpace{}%
\AgdaBound{𝓦}\AgdaSpace{}%
\AgdaSymbol{→}\AgdaSpace{}%
\AgdaBound{𝓤}\AgdaSpace{}%
\AgdaOperator{\AgdaPrimitive{⁺}}\AgdaSpace{}%
\AgdaOperator{\AgdaPrimitive{⊔}}\AgdaSpace{}%
\AgdaBound{𝓦}\AgdaSpace{}%
\AgdaOperator{\AgdaPrimitive{⁺}}\AgdaSpace{}%
\AgdaOperator{\AgdaFunction{̇}}\<%
\\
\>[0]\AgdaFunction{dep-extensionality}\AgdaSpace{}%
\AgdaBound{𝓤}\AgdaSpace{}%
\AgdaBound{𝓦}\AgdaSpace{}%
\AgdaSymbol{=}\AgdaSpace{}%
\AgdaSymbol{\{}\AgdaBound{A}\AgdaSpace{}%
\AgdaSymbol{:}\AgdaSpace{}%
\AgdaBound{𝓤}\AgdaSpace{}%
\AgdaOperator{\AgdaFunction{̇}}\AgdaSpace{}%
\AgdaSymbol{\}}\AgdaSpace{}%
\AgdaSymbol{\{}\AgdaBound{B}\AgdaSpace{}%
\AgdaSymbol{:}\AgdaSpace{}%
\AgdaBound{A}\AgdaSpace{}%
\AgdaSymbol{→}\AgdaSpace{}%
\AgdaBound{𝓦}\AgdaSpace{}%
\AgdaOperator{\AgdaFunction{̇}}\AgdaSpace{}%
\AgdaSymbol{\}}\<%
\\
\>[0][@{}l@{\AgdaIndent{0}}]%
\>[2]\AgdaSymbol{\{}\AgdaBound{f}\AgdaSpace{}%
\AgdaBound{g}\AgdaSpace{}%
\AgdaSymbol{:}\AgdaSpace{}%
\AgdaSymbol{∀(}\AgdaBound{x}\AgdaSpace{}%
\AgdaSymbol{:}\AgdaSpace{}%
\AgdaBound{A}\AgdaSymbol{)}\AgdaSpace{}%
\AgdaSymbol{→}\AgdaSpace{}%
\AgdaBound{B}\AgdaSpace{}%
\AgdaBound{x}\AgdaSymbol{\}}\AgdaSpace{}%
\AgdaSymbol{→}%
\>[28]\AgdaBound{f}\AgdaSpace{}%
\AgdaOperator{\AgdaFunction{∼}}\AgdaSpace{}%
\AgdaBound{g}%
\>[35]\AgdaSymbol{→}%
\>[38]\AgdaBound{f}\AgdaSpace{}%
\AgdaOperator{\AgdaDatatype{≡}}\AgdaSpace{}%
\AgdaBound{g}\<%
\end{code}
\ccpad
Sometimes we need extensionality principles that work at all universe levels, and Agda is capable of expressing such principles, which belong to the special 𝓤ω type, as follows:
\ccpad
\begin{code}%
\>[0]\AgdaFunction{∀-extensionality}\AgdaSpace{}%
\AgdaSymbol{:}\AgdaSpace{}%
\AgdaPrimitive{𝓤ω}\<%
\\
\>[0]\AgdaFunction{∀-extensionality}\AgdaSpace{}%
\AgdaSymbol{=}\AgdaSpace{}%
\AgdaSymbol{∀}%
\>[22]\AgdaSymbol{\{}\AgdaBound{𝓤}\AgdaSpace{}%
\AgdaBound{𝓥}\AgdaSymbol{\}}\AgdaSpace{}%
\AgdaSymbol{→}\AgdaSpace{}%
\AgdaFunction{extensionality}\AgdaSpace{}%
\AgdaBound{𝓤}\AgdaSpace{}%
\AgdaBound{𝓥}\<%
\\
%
\\[\AgdaEmptyExtraSkip]%
\>[0]\AgdaFunction{∀-dep-extensionality}\AgdaSpace{}%
\AgdaSymbol{:}\AgdaSpace{}%
\AgdaPrimitive{𝓤ω}\<%
\\
\>[0]\AgdaFunction{∀-dep-extensionality}\AgdaSpace{}%
\AgdaSymbol{=}\AgdaSpace{}%
\AgdaSymbol{∀}\AgdaSpace{}%
\AgdaSymbol{\{}\AgdaBound{𝓤}\AgdaSpace{}%
\AgdaBound{𝓥}\AgdaSymbol{\}}\AgdaSpace{}%
\AgdaSymbol{→}\AgdaSpace{}%
\AgdaFunction{dep-extensionality}\AgdaSpace{}%
\AgdaBound{𝓤}\AgdaSpace{}%
\AgdaBound{𝓥}\<%
\end{code}
\ccpad
More details about the 𝓤ω type are available at \href{https://agda.readthedocs.io/en/latest/language/universe-levels.html#expressions-of-kind-set}{agda.readthedocs.io}.
\ccpad
\begin{code}%
\>[0]\AgdaFunction{extensionality-lemma}\AgdaSpace{}%
\AgdaSymbol{:}%
\>[119I]\AgdaSymbol{∀}\AgdaSpace{}%
\AgdaSymbol{\{}\AgdaBound{𝓘}\AgdaSpace{}%
\AgdaBound{𝓤}\AgdaSpace{}%
\AgdaBound{𝓥}\AgdaSpace{}%
\AgdaBound{𝓣}\AgdaSymbol{\}}\AgdaSpace{}%
\AgdaSymbol{→}\<%
\\
\>[.][@{}l@{}]\<[119I]%
\>[23]\AgdaSymbol{\{}\AgdaBound{I}\AgdaSpace{}%
\AgdaSymbol{:}\AgdaSpace{}%
\AgdaBound{𝓘}\AgdaSpace{}%
\AgdaOperator{\AgdaFunction{̇}}\AgdaSpace{}%
\AgdaSymbol{\}\{}\AgdaBound{X}\AgdaSpace{}%
\AgdaSymbol{:}\AgdaSpace{}%
\AgdaBound{𝓤}\AgdaSpace{}%
\AgdaOperator{\AgdaFunction{̇}}\AgdaSpace{}%
\AgdaSymbol{\}\{}\AgdaBound{A}\AgdaSpace{}%
\AgdaSymbol{:}\AgdaSpace{}%
\AgdaBound{I}\AgdaSpace{}%
\AgdaSymbol{→}\AgdaSpace{}%
\AgdaBound{𝓥}\AgdaSpace{}%
\AgdaOperator{\AgdaFunction{̇}}\AgdaSpace{}%
\AgdaSymbol{\}}\<%
\\
%
\>[23]\AgdaSymbol{(}\AgdaBound{p}\AgdaSpace{}%
\AgdaBound{q}\AgdaSpace{}%
\AgdaSymbol{:}\AgdaSpace{}%
\AgdaSymbol{(}\AgdaBound{i}\AgdaSpace{}%
\AgdaSymbol{:}\AgdaSpace{}%
\AgdaBound{I}\AgdaSymbol{)}\AgdaSpace{}%
\AgdaSymbol{→}\AgdaSpace{}%
\AgdaSymbol{(}\AgdaBound{X}\AgdaSpace{}%
\AgdaSymbol{→}\AgdaSpace{}%
\AgdaBound{A}\AgdaSpace{}%
\AgdaBound{i}\AgdaSymbol{)}\AgdaSpace{}%
\AgdaSymbol{→}\AgdaSpace{}%
\AgdaBound{𝓣}\AgdaSpace{}%
\AgdaOperator{\AgdaFunction{̇}}\AgdaSpace{}%
\AgdaSymbol{)}\<%
\\
%
\>[23]\AgdaSymbol{(}\AgdaBound{args}\AgdaSpace{}%
\AgdaSymbol{:}\AgdaSpace{}%
\AgdaBound{X}\AgdaSpace{}%
\AgdaSymbol{→}\AgdaSpace{}%
\AgdaSymbol{(}\AgdaFunction{Π}\AgdaSpace{}%
\AgdaBound{A}\AgdaSymbol{))}\<%
\\
\>[0][@{}l@{\AgdaIndent{0}}]%
\>[1]\AgdaSymbol{→}%
\>[23]\AgdaBound{p}\AgdaSpace{}%
\AgdaOperator{\AgdaDatatype{≡}}\AgdaSpace{}%
\AgdaBound{q}\<%
\\
\>[1][@{}l@{\AgdaIndent{0}}]%
\>[23]\AgdaComment{-------------------------------------------------}\<%
\\
\>[0][@{}l@{\AgdaIndent{0}}]%
\>[1]\AgdaSymbol{→}%
\>[23]\AgdaSymbol{(λ}\AgdaSpace{}%
\AgdaBound{i}\AgdaSpace{}%
\AgdaSymbol{→}\AgdaSpace{}%
\AgdaSymbol{(}\AgdaBound{p}\AgdaSpace{}%
\AgdaBound{i}\AgdaSymbol{)(λ}\AgdaSpace{}%
\AgdaBound{x}\AgdaSpace{}%
\AgdaSymbol{→}\AgdaSpace{}%
\AgdaBound{args}\AgdaSpace{}%
\AgdaBound{x}\AgdaSpace{}%
\AgdaBound{i}\AgdaSymbol{))}\AgdaSpace{}%
\AgdaOperator{\AgdaDatatype{≡}}\AgdaSpace{}%
\AgdaSymbol{(λ}\AgdaSpace{}%
\AgdaBound{i}\AgdaSpace{}%
\AgdaSymbol{→}\AgdaSpace{}%
\AgdaSymbol{(}\AgdaBound{q}\AgdaSpace{}%
\AgdaBound{i}\AgdaSymbol{)(λ}\AgdaSpace{}%
\AgdaBound{x}\AgdaSpace{}%
\AgdaSymbol{→}\AgdaSpace{}%
\AgdaBound{args}\AgdaSpace{}%
\AgdaBound{x}\AgdaSpace{}%
\AgdaBound{i}\AgdaSymbol{))}\<%
\\
%
\\[\AgdaEmptyExtraSkip]%
\>[0]\AgdaFunction{extensionality-lemma}\AgdaSpace{}%
\AgdaBound{p}\AgdaSpace{}%
\AgdaBound{q}\AgdaSpace{}%
\AgdaBound{args}\AgdaSpace{}%
\AgdaBound{p≡q}\AgdaSpace{}%
\AgdaSymbol{=}\<%
\\
\>[0][@{}l@{\AgdaIndent{0}}]%
\>[1]\AgdaFunction{ap}\AgdaSpace{}%
\AgdaSymbol{(λ}\AgdaSpace{}%
\AgdaBound{-}\AgdaSpace{}%
\AgdaSymbol{→}\AgdaSpace{}%
\AgdaSymbol{λ}\AgdaSpace{}%
\AgdaBound{i}\AgdaSpace{}%
\AgdaSymbol{→}\AgdaSpace{}%
\AgdaSymbol{(}\AgdaBound{-}\AgdaSpace{}%
\AgdaBound{i}\AgdaSymbol{)}\AgdaSpace{}%
\AgdaSymbol{(λ}\AgdaSpace{}%
\AgdaBound{x}\AgdaSpace{}%
\AgdaSymbol{→}\AgdaSpace{}%
\AgdaBound{args}\AgdaSpace{}%
\AgdaBound{x}\AgdaSpace{}%
\AgdaBound{i}\AgdaSymbol{))}\AgdaSpace{}%
\AgdaBound{p≡q}\<%
\end{code}






%%%%%%%%%%%%%%%%%%%%%%%%%%%%%%%%%%%%%%%%%%%%%%%%%%%%%%%%%%%%%%%%%%%%%%%%%%%
\end{document} %%%%%%%%%%%%%%%%%%%%%%%%%%%%%%%%%%%%%%%%%%%%%%%%%%%%%%%%%%%%
%%%%%%%%%%%%%%%%%%%%%%%%%%%%%%%%%%%%%%%%%%%%%%%%%%%%%%%%%%%%%%%%%%%%%%%%%%%

