Suppose we are given a type \ab I \as : \ab 𝓘 (of ``indices'') and an indexed family \ab 𝒜 \as : \ab I \as → \af{Algebra} \ab 𝓤 \ab 𝑆 of \ab 𝑆-algebras. Then we define the \emph{product algebra}
\AgdaFunction{⨅} \AgdaBound{𝒜} in the following natural way.
%% %
%% %
%% \footnote{To distinguish the product algebra from the standard dependent function type available in the \agdastdlib, instead of \af ∏ (\texttt{\textbackslash prod}) or \af Π (\texttt{\textbackslash Pi}), we use the symbol \AgdaFunction{⨅}, which is typed in \agdamode as \texttt{\textbackslash Glb}.}%
%% %
\ccpad
\begin{code}%
\>[0]\AgdaFunction{⨅}\AgdaSpace{}%
\AgdaSymbol{:}\AgdaSpace{}%
%% \AgdaSymbol{\{}\AgdaBound{𝓤}\AgdaSpace{}%
%% \AgdaBound{𝓘}\AgdaSpace{}%
%% \AgdaSymbol{:}\AgdaSpace{}%
%% \AgdaPostulate{Universe}\AgdaSymbol{\}}
\AgdaSymbol{\{}\AgdaBound{I}\AgdaSpace{}%
\AgdaSymbol{:}\AgdaSpace{}%
\AgdaBound{𝓘}\AgdaSpace{}%
\AgdaOperator{\AgdaFunction{̇}}\AgdaSpace{}%
\AgdaSymbol{\}(}\AgdaBound{𝒜}\AgdaSpace{}%
\AgdaSymbol{:}\AgdaSpace{}%
\AgdaBound{I}\AgdaSpace{}%
\AgdaSymbol{→}\AgdaSpace{}%
\AgdaFunction{Algebra}\AgdaSpace{}%
\AgdaBound{𝓤}\AgdaSpace{}%
\AgdaBound{𝑆}\AgdaSpace{}%
\AgdaSymbol{)}\AgdaSpace{}%
\AgdaSymbol{→}\AgdaSpace{}%
\AgdaFunction{Algebra}\AgdaSpace{}%
\AgdaSymbol{(}\AgdaBound{𝓘}\AgdaSpace{}%
\AgdaOperator{\AgdaPrimitive{⊔}}\AgdaSpace{}%
\AgdaBound{𝓤}\AgdaSymbol{)}\AgdaSpace{}%
\AgdaBound{𝑆}\<%
\\
%
\\[\AgdaEmptyExtraSkip]%
\>[0]\AgdaFunction{⨅}\AgdaSpace{}%
\AgdaBound{𝒜}\AgdaSpace{}%
\AgdaSymbol{=}%
\>[48I]\AgdaSymbol{(∀}\AgdaSpace{}%
\AgdaBound{i}\AgdaSpace{}%
\AgdaSymbol{→}\AgdaSpace{}%
\AgdaOperator{\AgdaFunction{∣}}\AgdaSpace{}%
\AgdaBound{𝒜}\AgdaSpace{}%
\AgdaBound{i}\AgdaSpace{}%
\AgdaOperator{\AgdaFunction{∣}}\AgdaSymbol{)}\AgdaSpace{}%
\AgdaOperator{\AgdaInductiveConstructor{,}}%
\>[39]\AgdaComment{-- domain of the product algebra}\<%
\\
%
\>[.][@{}l@{}]\<[48I]%
\>[7]\AgdaSymbol{λ}\AgdaSpace{}%
\AgdaBound{𝑓}\AgdaSpace{}%
\AgdaBound{𝑎}\AgdaSpace{}%
\AgdaBound{i}\AgdaSpace{}%
\AgdaSymbol{→}\AgdaSpace{}%
\AgdaSymbol{(}\AgdaBound{𝑓}\AgdaSpace{}%
\AgdaOperator{\AgdaFunction{̂}}\AgdaSpace{}%
\AgdaBound{𝒜}\AgdaSpace{}%
\AgdaBound{i}\AgdaSymbol{)}\AgdaSpace{}%
\AgdaSymbol{λ}\AgdaSpace{}%
\AgdaBound{x}\AgdaSpace{}%
\AgdaSymbol{→}\AgdaSpace{}%
\AgdaBound{𝑎}\AgdaSpace{}%
\AgdaBound{x}\AgdaSpace{}%
\AgdaBound{i}%
\>[39]\AgdaComment{-- basic operations of the product algebra}\<%
\end{code}
\ccpad

Before proceeding, we define some convenient syntactic sugar. The type \af{Algebra} \ab 𝓤 \ab 𝑆 itself has a type; it is \ab 𝓞 \as ⊔ \ab 𝓥 \as ⊔ \ab 𝓤 \af ⁺ \af ̇ . This type appears quite often throughout the \ualib, so it is worthwhile to define the following shorthand for its universe level.
\ccpad
\begin{code}%
\>[0]\AgdaFunction{ov}\AgdaSpace{}%
\AgdaSymbol{:}\AgdaSpace{}%
\AgdaPostulate{Universe}\AgdaSpace{}%
\AgdaSymbol{→}\AgdaSpace{}%
\AgdaPostulate{Universe}\<%
\\
\>[0]\AgdaFunction{ov}\AgdaSpace{}%
\AgdaBound{𝓤}\AgdaSpace{}%
\AgdaSymbol{=}\AgdaSpace{}%
\AgdaBound{𝓞}\AgdaSpace{}%
\AgdaOperator{\AgdaPrimitive{⊔}}\AgdaSpace{}%
\AgdaBound{𝓥}\AgdaSpace{}%
\AgdaOperator{\AgdaPrimitive{⊔}}\AgdaSpace{}%
\AgdaBound{𝓤}\AgdaSpace{}%
\AgdaOperator{\AgdaPrimitive{⁺}}\<%
\end{code}

\subsection{Products of classes of algebras}\label{products-of-classes-of-algebras}
In the course of using Agda to formalize deep theorems in universal algebra (e.g., Birkhoff's HSP theorem), we were faced at a certain point with proving that the product of all subalgebras of algebras in a class 𝒦 belongs to the class SP(𝒦) of subalgebras of products of algebras in 𝒦.  That is, we had to prove \af ⨅ \ad S(\ab 𝒦) \af ∈ \ad{SP}(\ab 𝒦). This turned out to be surprisingly nontrivial. In fact, it wasn't even clear (at least to this author) how one should express the product of a whole class of algebras as a dependent type. After a number of failed attempts, the right type revealed itself in the form of the \af{class-product} function defined below. Now that we have the type in our hands, it seems quite natural, if not obvious.

First, we need a type that will serve to index a given class of algebras, as well as the product of all members of the class.
\ccpad
\begin{code}%
\>[1]\AgdaFunction{ℑ}\AgdaSpace{}%
\AgdaSymbol{:}\AgdaSpace{}%
\AgdaFunction{Pred}\AgdaSpace{}%
\AgdaSymbol{(}\AgdaFunction{Algebra}\AgdaSpace{}%
\AgdaBound{𝓤}\AgdaSpace{}%
\AgdaBound{𝑆}\AgdaSymbol{)(}\AgdaFunction{ov}\AgdaSpace{}%
\AgdaBound{𝓤}\AgdaSymbol{)}\AgdaSpace{}%
\AgdaSymbol{→}\AgdaSpace{}%
\AgdaSymbol{(}\AgdaBound{𝓧}\AgdaSpace{}%
\AgdaOperator{\AgdaPrimitive{⊔}}\AgdaSpace{}%
\AgdaFunction{ov}\AgdaSpace{}%
\AgdaBound{𝓤}\AgdaSymbol{)}\AgdaSpace{}%
\AgdaOperator{\AgdaFunction{̇}}\<%
\\
%
\\[\AgdaEmptyExtraSkip]%
%
\>[1]\AgdaFunction{ℑ}\AgdaSpace{}%
\AgdaBound{𝒦}\AgdaSpace{}%
\AgdaSymbol{=}\AgdaSpace{}%
\AgdaFunction{Σ}\AgdaSpace{}%
\AgdaBound{𝑨}\AgdaSpace{}%
\AgdaFunction{꞉}\AgdaSpace{}%
\AgdaSymbol{(}\AgdaFunction{Algebra}\AgdaSpace{}%
\AgdaBound{𝓤}\AgdaSpace{}%
\AgdaBound{𝑆}\AgdaSymbol{)}\AgdaSpace{}%
\AgdaFunction{,}\AgdaSpace{}%
\AgdaSymbol{(}\AgdaBound{𝑨}\AgdaSpace{}%
\AgdaOperator{\AgdaFunction{∈}}\AgdaSpace{}%
\AgdaBound{𝒦}\AgdaSymbol{)}\AgdaSpace{}%
\AgdaOperator{\AgdaFunction{×}}\AgdaSpace{}%
\AgdaSymbol{(}\AgdaBound{X}\AgdaSpace{}%
\AgdaSymbol{→}\AgdaSpace{}%
\AgdaOperator{\AgdaFunction{∣}}\AgdaSpace{}%
\AgdaBound{𝑨}\AgdaSpace{}%
\AgdaOperator{\AgdaFunction{∣}}\AgdaSymbol{)}\<%
\end{code}
\ccpad
Notice that the second component of this dependent pair type is \ab𝑨 \af ∈ \ab 𝒦 \ad × (\ab X \as → \af ∣ \ab 𝑨 \af ∣). In previous versions of the \ualib, the second component was simply \ab 𝑨 \af ∈ \ab 𝒦. However, we realized that adding a mapping of type \ab X \as → \af ∣ \ab 𝑨 \af ∣ is quite useful. The reason for this will become clear later; for now, suffice it to say that a map \ab X \as → \af ∣ \ab 𝑨 \af ∣ may be viewed as a context and we want to keep the context completely general. Including this context map in the index type \af ℑ accomplishes this.

Forming the product over the index type \af ℑ requires a function that takes an index \ab i \as : \af ℑ and returns the corresponding algebra. Each \ab i \as : \af ℑ is a triple of the form (\ab 𝑨 , \ab p , \ab h), where \ab 𝑨~\as :~\af{Algebra}~\ab 𝓤~\ab 𝑆 and \ab p \as : \ab 𝑨 \af ∈ \ab 𝒦 and \ab h \as : \ab X \as → \af ∣ \ab 𝑨 \as ∣, so the function mapping an index to the corresponding algebra is simply the first projection.
\ccpad
\begin{code}%
\>[1]\AgdaFunction{𝔄}\AgdaSpace{}%
\AgdaSymbol{:}\AgdaSpace{}%
\AgdaSymbol{(}\AgdaBound{𝒦}\AgdaSpace{}%
\AgdaSymbol{:}\AgdaSpace{}%
\AgdaFunction{Pred}\AgdaSpace{}%
\AgdaSymbol{(}\AgdaFunction{Algebra}\AgdaSpace{}%
\AgdaBound{𝓤}\AgdaSpace{}%
\AgdaBound{𝑆}\AgdaSymbol{)(}\AgdaFunction{ov}\AgdaSpace{}%
\AgdaBound{𝓤}\AgdaSymbol{))}\AgdaSpace{}%
\AgdaSymbol{→}\AgdaSpace{}%
\AgdaFunction{ℑ}\AgdaSpace{}%
\AgdaBound{𝒦}\AgdaSpace{}%
\AgdaSymbol{→}\AgdaSpace{}%
\AgdaFunction{Algebra}\AgdaSpace{}%
\AgdaBound{𝓤}\AgdaSpace{}%
\AgdaBound{𝑆}\<%
\\
%
\\[\AgdaEmptyExtraSkip]%
%
\>[1]\AgdaFunction{𝔄}\AgdaSpace{}%
\AgdaBound{𝒦}\AgdaSpace{}%
\AgdaSymbol{=}\AgdaSpace{}%
\AgdaSymbol{λ}\AgdaSpace{}%
\AgdaSymbol{(}\AgdaBound{i}\AgdaSpace{}%
\AgdaSymbol{:}\AgdaSpace{}%
\AgdaSymbol{(}\AgdaFunction{ℑ}\AgdaSpace{}%
\AgdaBound{𝒦}\AgdaSymbol{))}\AgdaSpace{}%
\AgdaSymbol{→}\AgdaSpace{}%
\AgdaOperator{\AgdaFunction{∣}}\AgdaSpace{}%
\AgdaBound{i}\AgdaSpace{}%
\AgdaOperator{\AgdaFunction{∣}}\<%
\end{code}
\ccpad
Finally, we define \af{class-product} which represents the product of all members of \ab 𝒦.
\ccpad
\begin{code}%
\>[1]\AgdaFunction{class-product}\AgdaSpace{}%
\AgdaSymbol{:}\AgdaSpace{}%
\AgdaFunction{Pred}\AgdaSpace{}%
\AgdaSymbol{(}\AgdaFunction{Algebra}\AgdaSpace{}%
\AgdaBound{𝓤}\AgdaSpace{}%
\AgdaBound{𝑆}\AgdaSymbol{)(}\AgdaFunction{ov}\AgdaSpace{}%
\AgdaBound{𝓤}\AgdaSymbol{)}\AgdaSpace{}%
\AgdaSymbol{→}\AgdaSpace{}%
\AgdaFunction{Algebra}\AgdaSpace{}%
\AgdaSymbol{(}\AgdaBound{𝓧}\AgdaSpace{}%
\AgdaOperator{\AgdaPrimitive{⊔}}\AgdaSpace{}%
\AgdaFunction{ov}\AgdaSpace{}%
\AgdaBound{𝓤}\AgdaSymbol{)}\AgdaSpace{}%
\AgdaBound{𝑆}\<%
\\
%
\\[\AgdaEmptyExtraSkip]%
%
\>[1]\AgdaFunction{class-product}\AgdaSpace{}%
\AgdaBound{𝒦}\AgdaSpace{}%
\AgdaSymbol{=}\AgdaSpace{}%
\AgdaFunction{⨅}\AgdaSpace{}%
\AgdaSymbol{(}\AgdaSpace{}%
\AgdaFunction{𝔄}\AgdaSpace{}%
\AgdaBound{𝒦}\AgdaSpace{}%
\AgdaSymbol{)}\<%
\end{code}
\ccpad
If \ab p \as : \ab 𝑨 \af ∈ \ab 𝒦 and \ab h \as : \ab X \as → \af ∣ \ab 𝑨 \af ∣, then the triple (\ab 𝑨 , \ab p , \ab h) \af ∈ \af ℑ \ab 𝒦 denotes an index of the class, and \af 𝔄 (\ab 𝑨 , \ab p , \ab h) (= \ab{𝑨}) is the projection of the product \af ⨅ ( \af 𝔄 \ab 𝒦 ) onto the (\ab 𝑨 , \ab p , \ab h)-th component.
