\documentclass[a4paper,UKenglish,cleveref,autoref,thm-restate]{lipics-v2021}
%% The next 3 lines are recommended at: https://agda.readthedocs.io/en/v2.6.1.1/tools/generating-latex.html#known-pitfalls-and-issues
\usepackage{fontspec}
\defaultfontfeatures[\rmfamily]{}
\setmainfont{Latin Modern Roman}%% \usepackage{mathabx}


\usepackage{proof-dashed}
%% \usepackage{unicode-math}
\usepackage{comment}
%% \setmainfont{XITS}
%% [    Extension = .otf,
%%    UprightFont = *-Regular,
%%       BoldFont = *-Bold,
%%     ItalicFont = *-Italic,
%% BoldItalicFont = *-BoldItalic,
%% ]
%% \setmathfont{XITSMath-Regular}
%% [    Extension = .otf,
%%       BoldFont = XITSMath-Bold,
%% ]

%% \usepackage{newunicodechar}
%% \newunicodechar{λ}{\ensuremath{\mathnormal\lambda}}
%% \newunicodechar{⋂}{\ensuremath{\mathnormal\bigcap}}
%% \newunicodechar{⨅}{\ensuremath{\mathnormal\bigsqcap}}

\usepackage[newwjd,links]{agda}
\usepackage{wjd}
\usepackage{unixode}

%This is a template for producing LIPIcs articles. 
%See lipics-v2021-authors-guidelines.pdf for further information.
%for A4 paper format use option "a4paper", for US-letter use option "letterpaper"
%for british hyphenation rules use option "UKenglish", for american hyphenation rules use option "USenglish"
%for section-numbered lemmas etc., use "numberwithinsect"
%for enabling cleveref support, use "cleveref"
%for enabling autoref support, use "autoref"
%for anonymousing the authors (e.g. for double-blind review), add "anonymous"
%for enabling thm-restate support, use "thm-restate"
%for enabling a two-column layout for the author/affilation part (only applicable for > 6 authors), use "authorcolumns"
%for producing a PDF according the PDF/A standard, add "pdfa"

%\graphicspath{{./graphics/}}%helpful if your graphic files are in another directory

%% \pdfmapfile{+fontawesome5.map}
\bibliographystyle{plainurl}% the mandatory bibstyle

\title{The Agda UALib and Birkhoff's Theorem\\ in Martin-L\"of Dependent Type Theory}

\titlerunning{} %The Agda Universal Algebra Library} %TODO optional, please use if title is longer than one line

\author{William DeMeo}
       {Department of Algebra, Charles University in Prague \and \url{https://williamdemeo.gitlab.io}}
       {williamdemeo@gmail.com}
       {https://orcid.org/0000-0003-1832-5690}
       {}

\authorrunning{}
%W.~J.~DeMeo}

\usepackage{fancyhdr}
\fancyfoot{}
\pagestyle{fancy}
\fancyhf{}
\fancyhead[ER]{\sffamily\bfseries \leftmark}
\fancyhead[OL]{\sffamily\bfseries The Agda Universal Algebra Library}
\fancyhead[EL,OR]{\sffamily\bfseries \thepage}
\renewcommand{\headrulewidth}{0pt}

\Copyright{William J. DeMeo}

\ccsdesc[500]{Theory of computation~Constructive mathematics}
\ccsdesc[500]{Theory of computation~Type theory}
\ccsdesc[500]{Theory of computation~Logic and verification}
\ccsdesc[300]{Computing methodologies~Representation of mathematical objects}
\ccsdesc[300]{Theory of computation~Type structures}
%% \ccsdesc[100]{\textcolor{red}{Replace ccsdesc macro with valid one}} %TODO mandatory: Please choose ACM 2012 classifications from https://dl.acm.org/ccs/ccs_flat.cfm 

\keywords{Universal algebra, Equational logic, Martin-Löf Type Theory, Birkhoff’s HSP Theorem, Formalization of mathematics, Agda, Proof assistant}

\category{} %optional, e.g. invited paper

\relatedversion{hosted on arXiv}
\relatedversiondetails[linktext={arxiv.org/pdf/2101.10166}]{Extended Version}{https://arxiv.org/pdf/2101.10166}

%% \supplement{Source code at \ualibrepo}%
%% optional, e.g. related research data, source code, ... hosted on a repository like zenodo, figshare, GitHub, ...
%% \supplementdetails[linktext={opt. text shown instead of the URL}, cite=DBLP:books/mk/GrayR93, subcategory={Description, Subcategory}, swhid={Software Heritage Identifier}]{General Classification (e.g. Software, Dataset, Model, ...)}{URL to related version} %linktext, cite, and subcategory are optional
%% \supplementdetails[swhid={Software Heritage Identifier}]{Software}{https://gitlab.com/ualib/ualib.gitlab.io.git}
\supplement{~\\ \textit{Documentation}: \ualibdotorg}%
\supplementdetails{Software}{https://gitlab.com/ualib/ualib.gitlab.io.git}

\acknowledgements{The author wishes to thank Hyeyoung Shin and Siva Somayyajula for their contributions to this project and \MartinEscardo for creating the \TypeTopology library and teaching a course on \href{https://www.cs.bham.ac.uk/~mhe/HoTT-UF-in-Agda-Lecture-Notes/index.html}{Univalent Foundations of Mathematics with Agda} at the \href{http://events.cs.bham.ac.uk/mgs2019/}{2019 Midlands Graduate School in Computing Science}. Of course, this work would not exist in its current form without the Agda language due to Catarina Coquand and Ulf Norell.}

%\nolinenumbers %uncomment to disable line numbering

%\hideLIPIcs  %uncomment to remove references to LIPIcs series (logo, DOI, ...), e.g. when preparing a pre-final version to be uploaded to arXiv or another public repository

%Editor-only macros:: begin (do not touch as author)%%%%%%%%%%%%%%%%%%%%%%%%%%%%%%%%%%
\EventEditors{}
\EventNoEds{2}
\EventLongTitle{}
\EventShortTitle{}
\EventAcronym{}
\EventYear{2021}
\EventDate{January 19, 2021}
\EventLocation{}
\EventLogo{}
\SeriesVolume{0}
\ArticleNo{0}
%%%%%%%%%%%%%%%%%%%%%%%%%%%%%%%%%%%%%%%%%%%%%%%%%%%%%%

\begin{document}

\maketitle
%TODO mandatory: add short abstract of the document

\begin{abstract}
The Agda Universal Algebra Library (\ualib) is a library of types and programs (theorems and proofs) we developed to formalize the foundations of universal algebra in Martin-Löf dependent type theory using the \agda programming language and proof assistant. This paper describes the UALib and demonstrates how it makes Agda more accessible to working mathematicians (like ourselves) as a tool for formally verifying ``known'' results in general algebra and related fields, as well as for discovering new theorems in these areas.  The library already includes a substantial collection of definitions, theorems, and proofs from universal algebra and as such provides many examples that exhibit the power of inductive and dependent types for representing and reasoning about algebraic and relational structures.

The first major milestone of the \ualib project is a complete proof of Birkhoff's HSP Theorem. To the best of our knowledge, this is the first time Birkhoff's Theorem has been formulated and proved in dependent type theory and verified with a proof assistant.

In this paper we describe the UALib and the formal proof of Birkhoff's theorem, discussing some of the challenges we faced and how these hurdles were overcome. In so doing, we illustrate the effectiveness of dependent type theory, Agda, and the UALib for proving and verifying theorems in universal algebra.
\end{abstract}

%% \setcounter{tocdepth}{2}
%% \tableofcontents

\section{Introduction}\label{sec:introduction}
To support formalization in type theory of research level mathematics in universal algebra and related fields, we present the Agda Universal Algebra Library (\agdaualib), a software library containing formal statements and proofs of the core definitions and results of universal algebra. 
The \agdaualib is written in \agda~\cite{Norell:2009}, a programming language and proof assistant based on Martin-L\"of Type Theory that not only supports dependent and inductive types, but also provides powerful \emph{proof tactics} for proving things about the objects that inhabit these types.

There have been a number of prior efforts to formalize parts of universal algebra in type theory, most notably
\begin{itemize}
  \item Capretta~\cite{Capretta:1999} (1999) formalized the basics of universal algebra in the Calculus of Inductive Constructions using the Coq proof assistant;
    \item Spitters and van der Weegen~\cite{Spitters:2011} (2011) formalized the basics of universal algebra and some classical algebraic structures, also in the Calculus of Inductive Constructions using the Coq proof assistant, promoting the use of type classes as a preferable alternative to setoids;
 \item Gunther, et al~\cite{Gunther:2018} developed what seems to be (prior to the \ualib) the most extensive library of formal universal algebra to date; in particular, this work includes a formalization of some basic equational logic; the project (like the \ualib) uses Martin-L\"of dependent type theory and the Agda proof assistant.
\end{itemize}
Some other projects aimed at formalizing mathematics generally, and algebra in particular, have developed into very extensive libraries that include definitions, theorems, and proofs about algebraic structures, such as groups, rings, fields, and modules.  However, the goals of this prior work seems to be a formal library of definitions, theorems, and proofs about specific classical algebraic structures, as opposed to the general theory of (universal) algebras.

Although the \agdaualib project was initiated relatively recently (in 2018), the part of universal algebra and equational logic that it formalizes already extends beyond the scope of prior efforts.  In particular, the \ualib now includes the only formal, constructive, machine-checked proof of Birkhoff's variety theorem that we know of. We remark that, with the exception of~\cite{Carlstrom:2008}, all other proofs of Birkhoff's theorem we have seen are informal and not known to be constructive.

%% \subsection{Vision and Goals}\label{vision-and-goals}
The seminal idea for the \agdaualib project is the observation that, on the one hand, a number of fundamental constructions in universal algebra can be defined recursively, and theorems about them proved by structural induction, while, on the other hand, inductive and dependent types make possible very precise formal representations of recursively defined objects, which often admit elegant constructive proofs of properties of such objects.  An important feature of such proofs in type theory is that they are total functional programs and, as such, they are computable and composable.  These observations suggest that there is much to be gained from implementing universal algebra in a language, such as Martin-L\"of type theory, that features dependent and inductive types.

\subsection{Objectives}\label{objectives}
One goal of this project is to express the foundations of universal algebra constructively, and to formally verify the foundations using the Agda proof assistant. Thus we aim to codify the edifice upon which our mathematical research stands, and demonstrate that advancements in our field can be expressed in type theory and formally verified in a way that we, and other working mathematicians, can easily understand and verify.  We hope our library inspires and encourages others to formalize and verify their own mathematical research so that we may more easily understand and verify their results.

As our field is both deep and broad, formalizing its foundations in type theory may seem like a daunting task.  Yet, it is our strong conviction that the subject is well served by a modern, constructive, machine-checked presentation of its foundations, and that the mere act of reinterpreting the foundations in the language of dependent type theory offers a fresh perspective and a deeper understanding, and sows the seeds of new discoveries.

We emphasize that our objective is not merely to translate existing results into an alternative formal language, but rather to provide a system that is useful for conducting research and proving new theorems.  That is how we intend to use the library now that we have achieved our initial goal of formalizing the core foundations of the subject, including the celebrated Birkhoff HSP Theorem.

Our own research experience has taught us that a proof assistant and programming language (like Agda), when equipped with specialized libraries and domain-specific tactics to automate proof idioms of a particular field, can be an extremely powerful and effective asset. We believe that such libraries, and the proof assistants they support, ought to become indispensable tools in the working mathematician's toolkit.

%% Finally, readers of this paper will benefit most from actively experimenting with Agda and the agda-ualib.

%% \subsection{Installing the library}\label{installing-the-library}
%% The main repository for the \agdaualib is \url{https://gitlab.com/ualib/ualib.gitlab.io}.

%% There are installation instructions in the main README.md file in that repository, but really all one needs is a working Agda installation and a clone of the \agdaualib repository.
\subsection{Contributions}
\label{sec:contributions}
Apart from the library itself, we describes the formal implementation and proof of a deep result in universal algebra, which was among the first major results of our subject---namely, Garrett Birkhoff's celebrated HSP Theorem~\cite{Birkhoff:1935}.  This theorem says that a \textit{variety} (a class of algebras closed under quotients, subalgebras, and products) is an equational class.  More precisely, a class 𝒦 of algebras is closed under the taking of quotients, subalgebras, and arbitrary products if and only if 𝒦 is the class of all algebras that satisfy a set of equations.

The fact that we now have a proof of Birkhoff's Theorem in Agda is noteworthy, not only because this is the first time the theorem has been proved in dependent type theory and verified with a proof assistant, but also because our proof is \emph{constructive}. Judging from the paper~\cite{Carlstrom:2008} of Carlstr\"om, it is clearly a nontrivial exercise to take a well-known informal proof of a theorem like Birkhoff's (as presented in, e.g.,~\cite{Bergman:2012}) and to show that it can be formalized using only constructive logical assumptions, without appealing to, say, the Law of the Excluded Middle, or the Axiom of Choice.

%% Prelude %%%%%%%%%%%%%%%%%%%%%%%%%%%%%%%%%%%%%%%%%%%%%%%%%
\section{Prelude}\label{sec:prelude}

\subsection{Preliminaries}\label{sec:preliminaries}
This section describes some aspects of the \ualibPreliminaries module of the \agdaualib, which imports everything we need from Martin Escardo's \typetopology library, and defines many other basic types and proves some of their properties.  We will not describe the entire Preliminaries module here. Instead, we focus on those aspects that might differ from standard Agda syntax or cause confusion in the sequel.

%%--BEGIN: 
%% \noindent \textbf{Notation}. Here are some acronyms that we use frequently.

%% \begin{itemize}
%% \tightlist
%% \item \mhe = \MHE
%% \item \mltt = \MLTT
%% \end{itemize}

\subsubsection{Options}\label{options}
All Agda programs begin by setting some options and by importing from existing libraries (in our case, the \TypeTopology library by Martin Escardo).

Lgical axioms and deduction rules can be specified according to what one wishes to assume. For example, each Agda source code file in the \ualib begins with the following line:
\ccpad
\begin{code}%
\>[0]\AgdaSymbol{\{-\#}\AgdaSpace{}%
\AgdaKeyword{OPTIONS}\AgdaSpace{}%
\AgdaPragma{--without-K}\AgdaSpace{}%
\AgdaPragma{--exact-split}\AgdaSpace{}%
\AgdaPragma{--safe}\AgdaSpace{}%
\AgdaSymbol{\#-\}}\<%
\end{code}
\ccpad
These options control certain foundational assumptions that Agda assumes when checking the program to verify its correctness.

\begin{itemize}
\item
  \texttt{without-K} disables \axiomk; see also the in the \agdarefk in the \agdaref.
\item
  \texttt{exact-split} makes Agda accept only those definitions that
  behave like so-called \emph{judgmental} or \emph{definitional}
  equalities. MHE explains this by saying it ``makes sure that pattern
  matching corresponds to Martin-Löf eliminators;'' see also the
  \patternmatching of the \agdatools documentation.
\item
  \texttt{safe} ensures that nothing is postulated outright---every non-MLTT axiom has to be an explicit assumption (e.g., an argument to a function or module); see also \href{https://agda.readthedocs.io/en/v2.6.1/tools/command-line-options.html\#cmdoption-safe}{this section} of the \agdatools documentation and the \href{https://agda.readthedocs.io/en/v2.6.1/language/safe-agda.html\#safe-agda}{Safe Agda section} of the \agdaref.
\end{itemize}


%% \subsubsection{Modules}\label{modules}

%% The \AgdaKeyword{OPTIONS} directive is followed by some imports or the start of a module.
%% For example, the \Preliminaries module begins with \AgdaKeyword{module}\AgdaSpace{}%
%% \AgdaModule{UALib.Prelude.Preliminaries}\AgdaSpace{}\AgdaKeyword{where}.

%% Sometimes we want to pass in parameters that will be assumed throughout the module. For instance, when working with algebras we often assume they come from a particular fixed signature \AgdaBound{𝑆}, which we could fix as a parameter at the start of a module. We'll see many examples later, but here's an example:
%% \AgdaKeyword{module}\AgdaSpace{}\AgdaModule{\AgdaUnderscore{}}\AgdaSpace{}%
%% \AgdaSymbol{\{}\AgdaBound{𝑆}\AgdaSpace{}%
%% \AgdaSymbol{:}\AgdaSpace{}%
%% \AgdaFunction{Signature}\AgdaSpace{}%
%% \AgdaGeneralizable{𝓞}\AgdaSpace{}%
%% \AgdaGeneralizable{𝓥}\AgdaSymbol{\}}\AgdaSpace{}%
%% \AgdaKeyword{where}.


\subsubsection{Imports from Type Topology}\label{imports-from-type-topology}
Following \MartinEscardo, we refer to universes using capitalized script letters from near the end of the alphabet, e.g., \ab 𝓤, \ab 𝓥, \ab 𝓦, \ab 𝓧, \ab 𝓨, \ab 𝓩, etc. Also, in the \universes module, Escardo defines the \af ̇ operator which maps a universe \ab 𝓤 (i.e., a level) to \Set \ab 𝓤, and the latter has type \Set(\lsuc \ab 𝓤). The level \lzero is renamed \AgdaBound{𝓤₀}, so \AgdaBound{𝓤₀}\af ̇ is an alias for \Set \lzero.\footnote{Incidentally, \Set \lzero corresponds to \texttt{Sort\sP{3}0} in the \href{https://leanprover.github.io/}{Lean} proof assistant language.} Thus, \ab 𝓤 ̇ is simply an alias for \Set \ab 𝓤, and we have \Set \ab 𝓤 : \Set (\lsuc \ab 𝓤). Finally, \Set(\lsuc \lzero) is denoted by \Set \AgdaBound{𝓤₀} ⁺ which we (and \mhe) denote by \AgdaBound{𝓤₀}~⁺~̇.
Table~\ref{tab:dictionary} translates between standard \agda syntax and \mhe/\ualib notation.

\begin{table}
\begin{tabular}{r|l}
Standard \agda                        &          \mhe/\ualib \\
\hline
\AgdaKeyword{Level}          &   \AgdaFunction{Universe}\\
\ab 𝓤 : \AgdaKeyword{Level}  & \ab 𝓤 : \AgdaFunction{Universe}\\
\Set \ab 𝓤                  &       \ab 𝓤 ̇ \\
\lsuc \ab 𝓤                   &    \ab 𝓤 ⁺\\
\Set (\lsuc \ab 𝓤) &    \ab 𝓤 ⁺ ̇\\
\lzero                       &         \AgdaBound{𝓤₀}\\
%% \Set \lzero              &    \AgdaBound{𝓤₀}~~̇\\
%% \lsuc \lzero                  &    \AgdaBound{𝓤₀}⁺\\
%% \Set (\lsuc \lzero) & \AgdaBound{𝓤₀} ⁺ ̇\\
\AgdaFunction{Setω}  &         \AgdaFunction{𝓤ω}
\end{tabular}
\caption{Special notation for universe levels}
\label{tab:dictionary}
\end{table}

To justify the introduction of this somewhat nonstandard notation for universe levels, \mhe points out that the Agda library uses \AgdaKeyword{Level} for universes (so what we write as \ab 𝓤 ̇ is written \Set\sP{3}\ab 𝓤 in standard Agda), but in univalent mathematics the types in \ab 𝓤 ̇ need not be sets, so the standard Agda notation can be misleading.

There will be many occasions calling for a type living in the universe that is the least upper bound of two universes, say, \ab 𝓤 ̇ and \ab 𝓥 ̇ . The universe (\ab 𝓤 \AgdaSymbol{⊔} \ab 𝓥) ̇ denotes this least upper bound.\footnote{Since \AgdaUnderscore{}\AgdaSymbol{⊔}\AgdaUnderscore{} has higher precedence than \AgdaUnderscore{}̇, we could omit parentheses here and simply write \ab 𝓤 \AgdaSymbol{⊔} \ab 𝓥 ̇.} Here \ab 𝓤 \AgdaSymbol{⊔} \ab 𝓥 is used to denote the universe level corresponding to the least upper bound of the levels \ab 𝓤 and \ab 𝓥, where the \AgdaUnderscore{}\AgdaSymbol{⊔}\AgdaUnderscore{} is an \agda primitive designed for precisely this purpose.

\subsubsection{Notation for dependent pairs}\label{Preliminaries.sssec:dependent-pair-type}
\textit{Dependent pair types} (or \textit{Sigma types}) are defined in the \typetopology library as a record, as follows:
\ccpad
%==========================================================
% FORMATTED AGDA CODE
\begin{code}
\>[0]\AgdaKeyword{record}\AgdaSpace{}%
\AgdaRecord{Σ}\AgdaSpace{}%
\AgdaSymbol{\{}\AgdaBound{𝓤}\AgdaSpace{}%
\AgdaBound{𝓥}\AgdaSymbol{\}}\AgdaSpace{}%
\AgdaSymbol{\{}\AgdaBound{X}\AgdaSpace{}%
\AgdaSymbol{:}\AgdaSpace{}%
\AgdaBound{𝓤}\AgdaSpace{}%
\AgdaOperator{\AgdaFunction{̇}}\AgdaSpace{}%
\AgdaSymbol{\}}\AgdaSpace{}%
\AgdaSymbol{(}\AgdaBound{Y}\AgdaSpace{}%
\AgdaSymbol{:}\AgdaSpace{}%
\AgdaBound{X}\AgdaSpace{}%
\AgdaSymbol{→}\AgdaSpace{}%
\AgdaBound{𝓥}\AgdaSpace{}%
\AgdaOperator{\AgdaFunction{̇}}\AgdaSpace{}%
\AgdaSymbol{)}\AgdaSpace{}%
\AgdaSymbol{:}\AgdaSpace{}%
\AgdaBound{𝓤}\AgdaSpace{}%
\AgdaOperator{\AgdaPrimitive{⊔}}\AgdaSpace{}%
\AgdaBound{𝓥}\AgdaSpace{}%
\AgdaOperator{\AgdaFunction{̇}}%
\>[52]\AgdaKeyword{where}\<%
\\
\>[0][@{}l@{\AgdaIndent{0}}]%
\>[2]\AgdaKeyword{constructor}\AgdaSpace{}%
%% \<%
%% \\
%% \>[2][@{}l@{\AgdaIndent{0}}]%
%% \>[3]
\AgdaOperator{\AgdaInductiveConstructor{\AgdaUnderscore{},\AgdaUnderscore{}}}\<%
\\
%
\>[2]\AgdaKeyword{field}\<%
\\
\>[2][@{}l@{\AgdaIndent{0}}]%
\>[3]\AgdaField{pr₁}\AgdaSpace{}%
\AgdaSymbol{:}\AgdaSpace{}%
\AgdaBound{X}\<%
\\
%
\>[3]\AgdaField{pr₂}\AgdaSpace{}%
\AgdaSymbol{:}\AgdaSpace{}%
\AgdaBound{Y}\AgdaSpace{}%
\AgdaField{pr₁}\<%
\\
%
\>[0]\AgdaKeyword{infixr}\AgdaSpace{}%
\AgdaNumber{50}\AgdaSpace{}%
\AgdaOperator{\AgdaInductiveConstructor{\AgdaUnderscore{},\AgdaUnderscore{}}}\<%
\\
\>[0]\<%
\end{code}
\ccpad
Agda's default syntax for the dependent pair type is (\AgdaRecord{Σ}\sP{3}\AgdaSymbol{λ}(\ab x\sP{3}꞉\sP{3}\ab X)\sP{3}\as →\sP{3}\ab y), but we prefer the notation \AgdaRecord{Σ}~\ab x~꞉~\ab X~,~\ab y, which is closer to the standard syntax appearing in the literature. In~\cite{MHE}, Escardo makes the preferred notation available by making the index type explicit, as follows.\footnote{The symbol ꞉ in the expression \AgdaRecord{Σ}\sP{3}\ab x\sP{3}\as ꞉\sP{3}\ab X\sP{3}\AgdaComma\sP{3}\ab y is not the ordinary colon (:); rather, it is the symbol obtained by typing \texttt{\textbackslash{}:4} in \agdatwomode.}
\ccpad
\begin{code}
\>[0]\AgdaFunction{-Σ}\AgdaSpace{}%
\AgdaSymbol{:}\AgdaSpace{}%
\AgdaSymbol{\{}\AgdaBound{𝓤}\AgdaSpace{}%
\AgdaBound{𝓥}\AgdaSpace{}%
\AgdaSymbol{:}\AgdaSpace{}%
\AgdaPostulate{Universe}\AgdaSymbol{\}}\AgdaSpace{}%
\AgdaSymbol{(}\AgdaBound{X}\AgdaSpace{}%
\AgdaSymbol{:}\AgdaSpace{}%
\AgdaBound{𝓤}\AgdaSpace{}%
\AgdaOperator{\AgdaFunction{̇}}\AgdaSpace{}%
\AgdaSymbol{)}\AgdaSpace{}%
\AgdaSymbol{(}\AgdaBound{Y}\AgdaSpace{}%
\AgdaSymbol{:}\AgdaSpace{}%
\AgdaBound{X}\AgdaSpace{}%
\AgdaSymbol{→}\AgdaSpace{}%
\AgdaBound{𝓥}\AgdaSpace{}%
\AgdaOperator{\AgdaFunction{̇}}\AgdaSpace{}%
\AgdaSymbol{)}\AgdaSpace{}%
\AgdaSymbol{→}\AgdaSpace{}%
\AgdaBound{𝓤}\AgdaSpace{}%
\AgdaOperator{\AgdaPrimitive{⊔}}\AgdaSpace{}%
\AgdaBound{𝓥}\AgdaSpace{}%
\AgdaOperator{\AgdaFunction{̇}}\<%
\\
\>[0]\AgdaFunction{-Σ}\AgdaSpace{}%
\AgdaBound{X}\AgdaSpace{}%
\AgdaBound{Y}\AgdaSpace{}%
\AgdaSymbol{=}\AgdaSpace{}%
\AgdaRecord{Σ}\AgdaSpace{}%
\AgdaBound{Y}\<%
\\
\>[0]\AgdaKeyword{syntax}\AgdaSpace{}%
\AgdaFunction{-Σ}\AgdaSpace{}%
\AgdaBound{X}\AgdaSpace{}%
\AgdaSymbol{(}\AgdaSymbol{λ}\AgdaSpace{}%
\AgdaBound{x}\AgdaSpace{}%
\AgdaSymbol{→}\AgdaSpace{}%
\AgdaBound{y}\AgdaSymbol{)}\AgdaSpace{}%
\AgdaSymbol{=}\AgdaSpace{}%
\AgdaFunction{Σ}\AgdaSpace{}%
\AgdaBound{x}\AgdaSpace{}%
\AgdaFunction{꞉}\AgdaSpace{}%
\AgdaBound{X}\AgdaSpace{}%
\AgdaFunction{,}\AgdaSpace{}%
\AgdaBound{y}\<%
\end{code}%
\ccpad

\newcommand\FstUnder{\AgdaOperator{\AgdaFunction{∣\AgdaUnderscore{}∣}}\xspace}
\newcommand\SndUnder{\AgdaOperator{\AgdaFunction{∥\AgdaUnderscore{}∥}}\xspace}
Convenient notations for the first and second projections out of a product are \FstUnder and \SndUnder, respectively. However, to improve readability or to avoid notation clashes with other modules, we sometimes use more standard alternatives, such as \AgdaFunction{pr₁} and \AgdaFunction{pr₂}, or \AgdaFunction{fst} and \AgdaFunction{snd}, or some combination of these.
\ccpad
\begin{code}%
\>[0]\AgdaKeyword{module}\AgdaSpace{}%
\AgdaModule{\AgdaUnderscore{}}\AgdaSpace{}%
\AgdaSymbol{\{}\AgdaBound{𝓤}\AgdaSpace{}%
\AgdaSymbol{:}\AgdaSpace{}%
\AgdaPostulate{Universe}\AgdaSymbol{\}}\AgdaSpace{}%
\AgdaKeyword{where}\<%
\\
\>[0][@{}l@{\AgdaIndent{0}}]%
\>[1]\AgdaOperator{\AgdaFunction{∣\AgdaUnderscore{}∣}}\AgdaSpace{}%
\AgdaFunction{fst}\AgdaSpace{}%
\AgdaSymbol{:}\AgdaSpace{}%
\AgdaSymbol{\{}\AgdaBound{X}\AgdaSpace{}%
\AgdaSymbol{:}\AgdaSpace{}%
\AgdaBound{𝓤}\AgdaSpace{}%
\AgdaOperator{\AgdaFunction{̇}}\AgdaSpace{}%
\AgdaSymbol{\}\{}\AgdaBound{Y}\AgdaSpace{}%
\AgdaSymbol{:}\AgdaSpace{}%
\AgdaBound{X}\AgdaSpace{}%
\AgdaSymbol{→}\AgdaSpace{}%
\AgdaGeneralizable{𝓥}\AgdaSpace{}%
\AgdaOperator{\AgdaFunction{̇}}\AgdaSymbol{\}}\AgdaSpace{}%
\AgdaSymbol{→}\AgdaSpace{}%
\AgdaRecord{Σ}\AgdaSpace{}%
\AgdaBound{Y}\AgdaSpace{}%
\AgdaSymbol{→}\AgdaSpace{}%
\AgdaBound{X}\<%
\\
%
\>[1]\AgdaOperator{\AgdaFunction{∣}}\AgdaSpace{}%
\AgdaBound{x}\AgdaSpace{}%
\AgdaOperator{\AgdaInductiveConstructor{,}}\AgdaSpace{}%
\AgdaBound{y}\AgdaSpace{}%
\AgdaOperator{\AgdaFunction{∣}}\AgdaSpace{}%
\AgdaSymbol{=}\AgdaSpace{}%
\AgdaBound{x}\<%
\\
%
\>[1]\AgdaFunction{fst}\AgdaSpace{}%
\AgdaSymbol{(}\AgdaBound{x}\AgdaSpace{}%
\AgdaOperator{\AgdaInductiveConstructor{,}}\AgdaSpace{}%
\AgdaBound{y}\AgdaSymbol{)}\AgdaSpace{}%
\AgdaSymbol{=}\AgdaSpace{}%
\AgdaBound{x}\<%
\\
%
\>[1]\AgdaOperator{\AgdaFunction{∥\AgdaUnderscore{}∥}}\AgdaSpace{}%
\AgdaFunction{snd}\AgdaSpace{}%
\AgdaSymbol{:}\AgdaSpace{}%
\AgdaSymbol{\{}\AgdaBound{X}\AgdaSpace{}%
\AgdaSymbol{:}\AgdaSpace{}%
\AgdaBound{𝓤}\AgdaSpace{}%
\AgdaOperator{\AgdaFunction{̇}}\AgdaSpace{}%
\AgdaSymbol{\}\{}\AgdaBound{Y}\AgdaSpace{}%
\AgdaSymbol{:}\AgdaSpace{}%
\AgdaBound{X}\AgdaSpace{}%
\AgdaSymbol{→}\AgdaSpace{}%
\AgdaGeneralizable{𝓥}\AgdaSpace{}%
\AgdaOperator{\AgdaFunction{̇}}\AgdaSpace{}%
\AgdaSymbol{\}}\AgdaSpace{}%
\AgdaSymbol{→}\AgdaSpace{}%
\AgdaSymbol{(}\AgdaBound{z}\AgdaSpace{}%
\AgdaSymbol{:}\AgdaSpace{}%
\AgdaRecord{Σ}\AgdaSpace{}%
\AgdaBound{Y}\AgdaSymbol{)}\AgdaSpace{}%
\AgdaSymbol{→}\AgdaSpace{}%
\AgdaBound{Y}\AgdaSpace{}%
\AgdaSymbol{(}\AgdaFunction{pr₁}\AgdaSpace{}%
\AgdaBound{z}\AgdaSymbol{)}\<%
\\
%
\>[1]\AgdaOperator{\AgdaFunction{∥}}\AgdaSpace{}%
\AgdaBound{x}\AgdaSpace{}%
\AgdaOperator{\AgdaInductiveConstructor{,}}\AgdaSpace{}%
\AgdaBound{y}\AgdaSpace{}%
\AgdaOperator{\AgdaFunction{∥}}\AgdaSpace{}%
\AgdaSymbol{=}\AgdaSpace{}%
\AgdaBound{y}\<%
\\
%
\>[1]\AgdaFunction{snd}\AgdaSpace{}%
\AgdaSymbol{(}\AgdaBound{x}\AgdaSpace{}%
\AgdaOperator{\AgdaInductiveConstructor{,}}\AgdaSpace{}%
\AgdaBound{y}\AgdaSymbol{)}\AgdaSpace{}%
\AgdaSymbol{=}\AgdaSpace{}%
\AgdaBound{y}\<%
\end{code}

%%%%%%%%%%%%%%%%%%%%%% UNUSED %%%%%%%%%%%%%%%%%%%%%%%%%%%%%%%%%%%%%%%%%%%%
\begin{comment}
\subsubsection{Dependent function type}\label{Preliminaries.sssec:dependent-function-type}
The so-called \textbf{dependent function type} (or ``Pi type'') is defined in the \typetopology library as follows.
\ccpad
\begin{code}
\>[0]\AgdaFunction{Π}\AgdaSpace{}%
\AgdaSymbol{:}\AgdaSpace{}%
\AgdaSymbol{\{}\AgdaBound{X}\AgdaSpace{}%
\AgdaSymbol{:}\AgdaSpace{}%
\AgdaGeneralizable{𝓤}\AgdaSpace{}%
\AgdaOperator{\AgdaFunction{̇}}\AgdaSpace{}%
\AgdaSymbol{\}}\AgdaSpace{}%
\AgdaSymbol{(}\AgdaBound{A}\AgdaSpace{}%
\AgdaSymbol{:}\AgdaSpace{}%
\AgdaBound{X}\AgdaSpace{}%
\AgdaSymbol{→}\AgdaSpace{}%
\AgdaGeneralizable{𝓥}\AgdaSpace{}%
\AgdaOperator{\AgdaFunction{̇}}\AgdaSpace{}%
\AgdaSymbol{)}\AgdaSpace{}%
\AgdaSymbol{→}\AgdaSpace{}%
\AgdaGeneralizable{𝓤}\AgdaSpace{}%
\AgdaOperator{\AgdaPrimitive{⊔}}\AgdaSpace{}%
\AgdaGeneralizable{𝓥}\AgdaSpace{}%
\AgdaOperator{\AgdaFunction{̇}}\<%
\\
\>[0]\AgdaFunction{Π}\AgdaSpace{}%
\AgdaSymbol{\{}\AgdaBound{𝓤}\AgdaSymbol{\}}\AgdaSpace{}%
\AgdaSymbol{\{}\AgdaBound{𝓥}\AgdaSymbol{\}}\AgdaSpace{}%
\AgdaSymbol{\{}\AgdaBound{X}\AgdaSymbol{\}}\AgdaSpace{}%
\AgdaBound{A}\AgdaSpace{}%
\AgdaSymbol{=}\AgdaSpace{}%
\AgdaSymbol{(}\AgdaBound{x}\AgdaSpace{}%
\AgdaSymbol{:}\AgdaSpace{}%
\AgdaBound{X}\AgdaSymbol{)}\AgdaSpace{}%
\AgdaSymbol{→}\AgdaSpace{}%
\AgdaBound{A}\AgdaSpace{}%
\AgdaBound{x}\<%
\end{code}
\ccpad
To make the syntax for \texttt{Π} conform to the standard notation for Pi types, \mhe uses the same trick as the one used above for Sigma types.
\ccpad
\begin{code}
\>[0]\AgdaFunction{-Π}\AgdaSpace{}%
\AgdaSymbol{:}\AgdaSpace{}%
\AgdaSymbol{\{}\AgdaBound{𝓤}\AgdaSpace{}%
\AgdaBound{𝓥}\AgdaSpace{}%
\AgdaSymbol{:}\AgdaSpace{}%
\AgdaPostulate{Universe}\AgdaSymbol{\}}\AgdaSpace{}%
\AgdaSymbol{(}\AgdaBound{X}\AgdaSpace{}%
\AgdaSymbol{:}\AgdaSpace{}%
\AgdaBound{𝓤}\AgdaSpace{}%
\AgdaOperator{\AgdaFunction{̇}}\AgdaSpace{}%
\AgdaSymbol{)}\AgdaSpace{}%
\AgdaSymbol{(}\AgdaBound{Y}\AgdaSpace{}%
\AgdaSymbol{:}\AgdaSpace{}%
\AgdaBound{X}\AgdaSpace{}%
\AgdaSymbol{→}\AgdaSpace{}%
\AgdaBound{𝓥}\AgdaSpace{}%
\AgdaOperator{\AgdaFunction{̇}}\AgdaSpace{}%
\AgdaSymbol{)}\AgdaSpace{}%
\AgdaSymbol{→}\AgdaSpace{}%
\AgdaBound{𝓤}\AgdaSpace{}%
\AgdaOperator{\AgdaPrimitive{⊔}}\AgdaSpace{}%
\AgdaBound{𝓥}\AgdaSpace{}%
\AgdaOperator{\AgdaFunction{̇}}\<%
\\
\>[0]\AgdaFunction{-Π}\AgdaSpace{}%
\AgdaBound{X}\AgdaSpace{}%
\AgdaBound{Y}\AgdaSpace{}%
\AgdaSymbol{=}\AgdaSpace{}%
\AgdaFunction{Π}\AgdaSpace{}%
\AgdaBound{Y}\<%
\\
%
\>[0]\AgdaKeyword{syntax}\AgdaSpace{}%
\AgdaFunction{-Π}\AgdaSpace{}%
\AgdaBound{A}\AgdaSpace{}%
\AgdaSymbol{(λ}\AgdaSpace{}%
\AgdaBound{x}\AgdaSpace{}%
\AgdaSymbol{→}\AgdaSpace{}%
\AgdaBound{b}\AgdaSymbol{)}\AgdaSpace{}%
\AgdaSymbol{=}\AgdaSpace{}%
\AgdaFunction{Π}\AgdaSpace{}%
\AgdaBound{x}\AgdaSpace{}%
\AgdaFunction{꞉}\AgdaSpace{}%
\AgdaBound{A}\AgdaSpace{}%
\AgdaFunction{,}\AgdaSpace{}%
\AgdaBound{b}\<%
\\
\>[0]\AgdaKeyword{infixr}\AgdaSpace{}%
\AgdaNumber{-1}\AgdaSpace{}%
\AgdaFunction{-Π}\<%
\end{code}
\end{comment}
%%%%%%%%%%%%%%%%%%%%%%%%%%%%%%%%%%%%%%%%%

\subsubsection{Truncation and sets}\label{Preliminaries.sssec:truncation}
In general, we may have many inhabitants of a given type and, via the Curry-Howard correspondence, many proofs of a given proposition. For instance, suppose we have a type \ab X and an identity relation \AgdaOperator{\AgdaDatatype{≡ₓ}} on \ab X. Then, given two inhabitants \ab a and \ab b of type \ab X, we may ask whether \ab a \AgdaOperator{\AgdaDatatype{≡ₓ}} \ab b.

Suppose \ab p and \ab q inhabit the identity type \ab a \AgdaOperator{\AgdaDatatype{≡ₓ}} b; that is, \ab p and \ab q are proofs of \ab a \AgdaOperator{\AgdaDatatype{≡ₓ}} \ab b, in which case we write \ab p\sP{3}\sP{3}\ab q : \ab a \AgdaOperator{\AgdaDatatype{≡ₓ}} \ab b. Then we might wonder whether and in what sense are the two proofs \ab p and \ab q the ``same.'' We are asking about an identity type on the identity type \AgdaOperator{\AgdaDatatype{≡ₓ}}, and whether there is some inhabitant \ab r of this type; i.e., whether there is a proof \ab r : \ab p \AgdaOperator{\AgdaDatatype{≡ₓ₁}} \ab q that the proof of \ab a \AgdaOperator{\AgdaDatatype{≡ₓ}} \ab b is unique. (This property is sometimes called \emph{uniqueness of identity proofs}.)

Perhaps we have two proofs, say, \ab r\sP{3}\sP{3} \ab s : \ab p \AgdaOperator{\AgdaDatatype{≡ₓ₁}} \ab q. Then of course our next question will be whether \ab r \AgdaOperator{\AgdaDatatype{≡ₓ₂}} \ab s has a proof!  But at some level we may decide that the potential to distinguish two proofs of an identity in a meaningful way (so-called \emph{proof relevance}) is not useful or desirable. At that point, say, at level \ab k, we might assume that there is at most one proof of any identity of the form \ab p  \AgdaOperator{\AgdaDatatype{≡ₓₖ}} \ab q. This is called \href{https://www.cs.bham.ac.uk/~mhe/HoTT-UF-in-Agda-Lecture-Notes/HoTT-UF-Agda.html#truncation}{truncation}.

We will see some examples of trunction later when we require it to complete some of the \ualib modules leading up to the proof of Birkhoff's HSP Theorem. Readers who want more details should refer to \href{https://www.cs.bham.ac.uk/~mhe/HoTT-UF-in-Agda-Lecture-Notes/HoTT-UF-Agda.html\#truncation}{Section 34} and \href{https://www.cs.bham.ac.uk/~mhe/HoTT-UF-in-Agda-Lecture-Notes/HoTT-UF-Agda.html\#resizing}{35} of MHE's notes, or \href{https://homotopytypetheory.org/2012/09/16/truncations-and-truncated-higher-inductive-types/}{Guillaume Brunerie, Truncations and truncated higher inductive types}, or Section 7.1 of the \hottbook.

We take this opportunity to say what it means in type theory to say that a type is a \emph{set}. A type \ab X \as : \ab 𝓤 \af ̇ with an identity relation \AgdaDatatype{≡ₓ} is called a \textbf{set} (or 0-\textbf{groupoid}) if for every pair \ab a \ab b \as : \ab X of elements of type \ab X there is at most one proof of \ab a \AgdaDatatype{≡ₓ} \ab b. This is formalized in the \TypeTopology library as follows:
%% \footnote{As \mhe explains, ``at this point, with the definition of these notions, we are entering the realm of univalent mathematics, but not yet needing the univalence axiom.''}
\ccpad
\begin{code}
\>[0]\AgdaFunction{is-set}\AgdaSpace{}%
\AgdaSymbol{:}\AgdaSpace{}%
\AgdaGeneralizable{𝓤}\AgdaSpace{}%
\AgdaOperator{\AgdaFunction{̇}}\AgdaSpace{}%
\AgdaSymbol{→}\AgdaSpace{}%
\AgdaGeneralizable{𝓤}\AgdaSpace{}%
\AgdaOperator{\AgdaFunction{̇}}\<%
\\
\>[0]\AgdaFunction{is-set}\AgdaSpace{}%
\AgdaBound{X}\AgdaSpace{}%
\AgdaSymbol{=}\AgdaSpace{}%
\AgdaSymbol{(}\AgdaBound{x}\AgdaSpace{}%
\AgdaBound{y}\AgdaSpace{}%
\AgdaSymbol{:}\AgdaSpace{}%
\AgdaBound{X}\AgdaSymbol{)}\AgdaSpace{}%
\AgdaSymbol{→}\AgdaSpace{}%
\AgdaFunction{is-subsingleton}\AgdaSpace{}%
\AgdaSymbol{(}\AgdaBound{x}\AgdaSpace{}%
\AgdaOperator{\AgdaDatatype{≡}}\AgdaSpace{}%
\AgdaBound{y}\AgdaSymbol{)}\<%
\end{code}
-----------------------------------------------

%% \noindent \textbf{Notation}. Here are some acronyms that we use frequently.
%% \begin{itemize}
%% \tightlist
%% \item \mhe = \MHE
%% \item \mltt = \MLTT
%% \end{itemize}

\subsubsection{Options}\label{options}
Agda programs typically begin by setting some options and by importing from existing libraries. For example, each Agda source code file in the \ualib begins with the following line:
\ccpad
\begin{code}[number=code:options]
\>[0]\AgdaSymbol{\{-\#}\AgdaSpace{}%
\AgdaKeyword{OPTIONS}\AgdaSpace{}%
\AgdaPragma{--without-K}\AgdaSpace{}%
\AgdaPragma{--exact-split}\AgdaSpace{}%
\AgdaPragma{--safe}\AgdaSpace{}%
\AgdaSymbol{\#-\}}\<%
\end{code}
\ccpad
These options control certain foundational assumptions that Agda assumes when checking the program to verify its correctness.  The options set with the line above have the following effects.
\begin{enumerate}
\item
  \texttt{without-K} disables \axiomk; see also the in the \agdarefk in the \agdaref.
\item
  \texttt{exact-split} makes Agda accept only those definitions that behave like so-called \emph{judgmental} or \emph{definitional} equalities. Escardo explains that this ``makes sure that pattern matching corresponds to Martin-Löf eliminators;'' see also the \patternmatching of the \agdatools documentation.
\item
  \texttt{safe} ensures that nothing is postulated outright---every non-MLTT axiom has to be an explicit assumption (e.g., an argument to a function or module); see also \href{https://agda.readthedocs.io/en/v2.6.1/tools/command-line-options.html\#cmdoption-safe}{this section} of the \agdatools documentation and the \href{https://agda.readthedocs.io/en/v2.6.1/language/safe-agda.html\#safe-agda}{Safe Agda section} of the \agdaref.
\end{enumerate}
An \AgdaKeyword{OPTIONS} directive identical to the one in~(\ref{code:options}) above appears at the top of every \texttt{lagda} file of the \agdaualib. In the remainder of this document, we take assumptions 1--3 for granted without mentioning them explicitly.

\subsubsection{Imports from Type Topology}\label{imports-from-type-topology}
We adopt to the notation of \MartinEscardo's \typetopology library~\cite{MHE} and refer to universes using capitalized script letters from near the end of the alphabet, e.g., \ab 𝓤, \ab 𝓥, \ab 𝓦, \ab 𝓧, \ab 𝓨, \ab 𝓩, etc. Also, in his \AgdaModule{Universes} module, Escardo defines the \af ̇ operator which maps a universe \ab 𝓤 (i.e., a level) to \Set \ab 𝓤, and the latter has type \Set(\lsuc \ab 𝓤). The level \lzero is renamed \AgdaBound{𝓤₀}, so \AgdaBound{𝓤₀}\af ̇ is an alias for \Set \lzero.\footnote{For those familiar the \href{https://leanprover.github.io/}{Lean} proof assistant, our \Set \lzero is analogous to Lean's \texttt{Sort 0}.} Thus, \ab 𝓤 ̇ is simply an alias for \Set \ab 𝓤, and we have \Set \ab 𝓤 : \Set (\lsuc \ab 𝓤). %% Finally, \Set(\lsuc \lzero) is denoted by \Set \AgdaBound{𝓤₀} ⁺ which we (and \mhe) denote by \AgdaBound{𝓤₀}~⁺~̇.
Table~\ref{tab:dictionary} translates between standard \agda syntax and \mhe/\ualib notation.

\begin{table}
\begin{tabular}{r|l}
Standard \agda                        &          \mhe/\ualib \\
\hline
\AgdaKeyword{Level}          &   \AgdaFunction{Universe}\\
\ab 𝓤 : \AgdaKeyword{Level}  & \ab 𝓤 : \AgdaFunction{Universe}\\
\Set \ab 𝓤                  &       \ab 𝓤 ̇ \\
\lsuc \ab 𝓤                   &    \ab 𝓤 ⁺\\
\Set (\lsuc \ab 𝓤) &    \ab 𝓤 ⁺ ̇\\
\lzero                       &         \AgdaBound{𝓤₀}\\
%% \Set \lzero              &    \AgdaBound{𝓤₀}~~̇\\
%% \lsuc \lzero                  &    \AgdaBound{𝓤₀}⁺\\
%% \Set (\lsuc \lzero) & \AgdaBound{𝓤₀} ⁺ ̇\\
\AgdaFunction{Setω}  &         \AgdaFunction{𝓤ω}
\end{tabular}
\caption{Special notation for universe levels}
\label{tab:dictionary}
\end{table}

To justify the introduction of this somewhat nonstandard notation for universe levels, \escardo points out that the Agda library uses \AgdaKeyword{Level} for universes (so what we write as \ab 𝓤 ̇ is written \Set\sP{3}\ab 𝓤 in standard Agda), but in univalent mathematics the types in \ab 𝓤 ̇ need not be sets, so the standard Agda notation can be misleading.

There will be many occasions calling for a type living in the universe that is the least upper bound of two universes, say, \ab 𝓤 ̇ and \ab 𝓥 ̇ . The universe (\ab 𝓤 \AgdaSymbol{⊔} \ab 𝓥) ̇ denotes this least upper bound.\footnote{Since \AgdaUnderscore{}\AgdaSymbol{⊔}\AgdaUnderscore{} has higher precedence than \AgdaUnderscore{}̇, we could omit parentheses here and simply write \ab 𝓤 \AgdaSymbol{⊔} \ab 𝓥 ̇.} Here \ab 𝓤 \AgdaSymbol{⊔} \ab 𝓥 is used to denote the universe level corresponding to the least upper bound of the levels \ab 𝓤 and \ab 𝓥, where the \AgdaUnderscore{}\AgdaSymbol{⊔}\AgdaUnderscore{} is an \agda primitive designed for precisely this purpose.

\subsubsection{Notation for dependent pairs}\label{sssec:dependent-pair-type}
\textit{Dependent pair types} (or \textit{Sigma types}) are defined in the \typetopology library as a record, as follows:
\ccpad
\begin{code}
\>[0]\AgdaKeyword{record}\AgdaSpace{}%
\AgdaRecord{Σ}\AgdaSpace{}%
\AgdaSymbol{\{}\AgdaBound{𝓤}\AgdaSpace{}%
\AgdaBound{𝓥}\AgdaSymbol{\}}\AgdaSpace{}%
\AgdaSymbol{\{}\AgdaBound{X}\AgdaSpace{}%
\AgdaSymbol{:}\AgdaSpace{}%
\AgdaBound{𝓤}\AgdaSpace{}%
\AgdaOperator{\AgdaFunction{̇}}\AgdaSpace{}%
\AgdaSymbol{\}}\AgdaSpace{}%
\AgdaSymbol{(}\AgdaBound{Y}\AgdaSpace{}%
\AgdaSymbol{:}\AgdaSpace{}%
\AgdaBound{X}\AgdaSpace{}%
\AgdaSymbol{→}\AgdaSpace{}%
\AgdaBound{𝓥}\AgdaSpace{}%
\AgdaOperator{\AgdaFunction{̇}}\AgdaSpace{}%
\AgdaSymbol{)}\AgdaSpace{}%
\AgdaSymbol{:}\AgdaSpace{}%
\AgdaBound{𝓤}\AgdaSpace{}%
\AgdaOperator{\AgdaPrimitive{⊔}}\AgdaSpace{}%
\AgdaBound{𝓥}\AgdaSpace{}%
\AgdaOperator{\AgdaFunction{̇}}%
\>[52]\AgdaKeyword{where}\<%
\\
\>[0][@{}l@{\AgdaIndent{0}}]%
\>[2]\AgdaKeyword{constructor}\AgdaSpace{}%
\AgdaOperator{\AgdaInductiveConstructor{\AgdaUnderscore{},\AgdaUnderscore{}}}\<%
\\
%
\>[2]\AgdaKeyword{field}\<%
\\
\>[2][@{}l@{\AgdaIndent{0}}]%
\>[3]\AgdaField{pr₁}\AgdaSpace{}%
\AgdaSymbol{:}\AgdaSpace{}%
\AgdaBound{X}\<%
\\
%
\>[3]\AgdaField{pr₂}\AgdaSpace{}%
\AgdaSymbol{:}\AgdaSpace{}%
\AgdaBound{Y}\AgdaSpace{}%
\AgdaField{pr₁}\<%
\\
%
\>[0]\AgdaKeyword{infixr}\AgdaSpace{}%
\AgdaNumber{50}\AgdaSpace{}%
\AgdaOperator{\AgdaInductiveConstructor{\AgdaUnderscore{},\AgdaUnderscore{}}}\<%
\\
\>[0]\<%
\end{code}
\ccpad
Agda's default syntax for the dependent pair type is \AgdaRecord{Σ}\sP{3}\AgdaSymbol{λ}(\ab x\sP{3}꞉\sP{3}\ab X)\sP{3}\as →\sP{3}\ab Y, but we prefer the notation \AgdaRecord{Σ}~\ab x~꞉~\ab X~,~\ab Y, which is closer to the standard syntax appearing in the literature. In~\cite{MHE}, Escardo makes the preferred notation available by making the index type explicit, as follows.\footnote{The symbol \as ꞉ in the expression \AgdaRecord{Σ}\sP{3}\ab x\sP{3}\as ꞉\sP{3}\ab X\sP{3}\AgdaComma\sP{3}\ab Y is not the ordinary colon (:); rather, it is the symbol obtained by typing \texttt{\textbackslash{}:4} in \agdatwomode.}
\ccpad
\begin{code}
\>[0]\AgdaFunction{-Σ}\AgdaSpace{}%
\AgdaSymbol{:}\AgdaSpace{}%
\AgdaSymbol{\{}\AgdaBound{𝓤}\AgdaSpace{}%
\AgdaBound{𝓥}\AgdaSpace{}%
\AgdaSymbol{:}\AgdaSpace{}%
\AgdaPostulate{Universe}\AgdaSymbol{\}}\AgdaSpace{}%
\AgdaSymbol{(}\AgdaBound{X}\AgdaSpace{}%
\AgdaSymbol{:}\AgdaSpace{}%
\AgdaBound{𝓤}\AgdaSpace{}%
\AgdaOperator{\AgdaFunction{̇}}\AgdaSpace{}%
\AgdaSymbol{)}\AgdaSpace{}%
\AgdaSymbol{(}\AgdaBound{Y}\AgdaSpace{}%
\AgdaSymbol{:}\AgdaSpace{}%
\AgdaBound{X}\AgdaSpace{}%
\AgdaSymbol{→}\AgdaSpace{}%
\AgdaBound{𝓥}\AgdaSpace{}%
\AgdaOperator{\AgdaFunction{̇}}\AgdaSpace{}%
\AgdaSymbol{)}\AgdaSpace{}%
\AgdaSymbol{→}\AgdaSpace{}%
\AgdaBound{𝓤}\AgdaSpace{}%
\AgdaOperator{\AgdaPrimitive{⊔}}\AgdaSpace{}%
\AgdaBound{𝓥}\AgdaSpace{}%
\AgdaOperator{\AgdaFunction{̇}}\<%
\\
\>[0]\AgdaFunction{-Σ}\AgdaSpace{}%
\AgdaBound{X}\AgdaSpace{}%
\AgdaBound{Y}\AgdaSpace{}%
\AgdaSymbol{=}\AgdaSpace{}%
\AgdaRecord{Σ}\AgdaSpace{}%
\AgdaBound{Y}\<%
\\
\>[0]\AgdaKeyword{syntax}\AgdaSpace{}%
\AgdaFunction{-Σ}\AgdaSpace{}%
\AgdaBound{X}\AgdaSpace{}%
\AgdaSymbol{(}\AgdaSymbol{λ}\AgdaSpace{}%
\AgdaBound{x}\AgdaSpace{}%
\AgdaSymbol{→}\AgdaSpace{}%
\AgdaBound{Y}\AgdaSymbol{)}\AgdaSpace{}%
\AgdaSymbol{=}\AgdaSpace{}%
\AgdaFunction{Σ}\AgdaSpace{}%
\AgdaBound{x}\AgdaSpace{}%
\AgdaFunction{꞉}\AgdaSpace{}%
\AgdaBound{X}\AgdaSpace{}%
\AgdaFunction{,}\AgdaSpace{}%
\AgdaBound{Y}\<%
\end{code}%
\ccpad

\newcommand\FstUnder{\AgdaOperator{\AgdaFunction{∣\AgdaUnderscore{}∣}}\xspace}
\newcommand\SndUnder{\AgdaOperator{\AgdaFunction{∥\AgdaUnderscore{}∥}}\xspace}
Convenient notations for the first and second projections out of a product are \FstUnder and \SndUnder, respectively. However, to improve readability or to avoid notation clashes with other modules, we sometimes use more standard alternatives, such as \AgdaFunction{pr₁} and \AgdaFunction{pr₂}, or \AgdaFunction{fst} and \AgdaFunction{snd}, or some combination of these.
\ccpad
\begin{code}%
\>[1]\AgdaOperator{\AgdaFunction{∣\AgdaUnderscore{}∣}}\AgdaSpace{}%
\AgdaFunction{fst}\AgdaSpace{}%
\AgdaSymbol{:}\AgdaSpace{}%
\AgdaSymbol{\{}\AgdaBound{X}\AgdaSpace{}%
\AgdaSymbol{:}\AgdaSpace{}%
\AgdaBound{𝓤}\AgdaSpace{}%
\AgdaOperator{\AgdaFunction{̇}}\AgdaSpace{}%
\AgdaSymbol{\}\{}\AgdaBound{Y}\AgdaSpace{}%
\AgdaSymbol{:}\AgdaSpace{}%
\AgdaBound{X}\AgdaSpace{}%
\AgdaSymbol{→}\AgdaSpace{}%
\AgdaGeneralizable{𝓥}\AgdaSpace{}%
\AgdaOperator{\AgdaFunction{̇}}\AgdaSymbol{\}}\AgdaSpace{}%
\AgdaSymbol{→}\AgdaSpace{}%
\AgdaRecord{Σ}\AgdaSpace{}%
\AgdaBound{Y}\AgdaSpace{}%
\AgdaSymbol{→}\AgdaSpace{}%
\AgdaBound{X}\<%
\\
%
\>[1]\AgdaOperator{\AgdaFunction{∣}}\AgdaSpace{}%
\AgdaBound{x}\AgdaSpace{}%
\AgdaOperator{\AgdaInductiveConstructor{,}}\AgdaSpace{}%
\AgdaBound{y}\AgdaSpace{}%
\AgdaOperator{\AgdaFunction{∣}}\AgdaSpace{}%
\AgdaSymbol{=}\AgdaSpace{}%
\AgdaBound{x}\<%
\\
%
\>[1]\AgdaFunction{fst}\AgdaSpace{}%
\AgdaSymbol{(}\AgdaBound{x}\AgdaSpace{}%
\AgdaOperator{\AgdaInductiveConstructor{,}}\AgdaSpace{}%
\AgdaBound{y}\AgdaSymbol{)}\AgdaSpace{}%
\AgdaSymbol{=}\AgdaSpace{}%
\AgdaBound{x}\<%
\\
%
\\
\>[1]\AgdaOperator{\AgdaFunction{∥\AgdaUnderscore{}∥}}\AgdaSpace{}%
\AgdaFunction{snd}\AgdaSpace{}%
\AgdaSymbol{:}\AgdaSpace{}%
\AgdaSymbol{\{}\AgdaBound{X}\AgdaSpace{}%
\AgdaSymbol{:}\AgdaSpace{}%
\AgdaBound{𝓤}\AgdaSpace{}%
\AgdaOperator{\AgdaFunction{̇}}\AgdaSpace{}%
\AgdaSymbol{\}\{}\AgdaBound{Y}\AgdaSpace{}%
\AgdaSymbol{:}\AgdaSpace{}%
\AgdaBound{X}\AgdaSpace{}%
\AgdaSymbol{→}\AgdaSpace{}%
\AgdaGeneralizable{𝓥}\AgdaSpace{}%
\AgdaOperator{\AgdaFunction{̇}}\AgdaSpace{}%
\AgdaSymbol{\}}\AgdaSpace{}%
\AgdaSymbol{→}\AgdaSpace{}%
\AgdaSymbol{(}\AgdaBound{z}\AgdaSpace{}%
\AgdaSymbol{:}\AgdaSpace{}%
\AgdaRecord{Σ}\AgdaSpace{}%
\AgdaBound{Y}\AgdaSymbol{)}\AgdaSpace{}%
\AgdaSymbol{→}\AgdaSpace{}%
\AgdaBound{Y}\AgdaSpace{}%
\AgdaSymbol{(}\AgdaFunction{pr₁}\AgdaSpace{}%
\AgdaBound{z}\AgdaSymbol{)}\<%
\\
%
\>[1]\AgdaOperator{\AgdaFunction{∥}}\AgdaSpace{}%
\AgdaBound{x}\AgdaSpace{}%
\AgdaOperator{\AgdaInductiveConstructor{,}}\AgdaSpace{}%
\AgdaBound{y}\AgdaSpace{}%
\AgdaOperator{\AgdaFunction{∥}}\AgdaSpace{}%
\AgdaSymbol{=}\AgdaSpace{}%
\AgdaBound{y}\<%
\\
%
\>[1]\AgdaFunction{snd}\AgdaSpace{}%
\AgdaSymbol{(}\AgdaBound{x}\AgdaSpace{}%
\AgdaOperator{\AgdaInductiveConstructor{,}}\AgdaSpace{}%
\AgdaBound{y}\AgdaSymbol{)}\AgdaSpace{}%
\AgdaSymbol{=}\AgdaSpace{}%
\AgdaBound{y}\<%
\end{code}

\subsubsection{Truncation and sets}\label{sssec:truncation}
In general, we may have many inhabitants of a given type and, via the Curry-Howard correspondence, many proofs of a given proposition. For instance, suppose we have a type \ab X and an identity relation \AgdaOperator{\AgdaDatatype{≡ₓ}} on \ab X. Then, given two inhabitants \ab a and \ab b of type \ab X, we may ask whether \ab a \AgdaOperator{\AgdaDatatype{≡ₓ}} \ab b.

Suppose \ab p and \ab q inhabit the identity type \ab a \AgdaOperator{\AgdaDatatype{≡ₓ}} b; that is, \ab p and \ab q are proofs of \ab a \AgdaOperator{\AgdaDatatype{≡ₓ}} \ab b, in which case we write \ab p\sP{3}\sP{3}\ab q : \ab a \AgdaOperator{\AgdaDatatype{≡ₓ}} \ab b. Then we might wonder whether and in what sense are the two proofs \ab p and \ab q the ``same.'' We are asking about an identity type on the identity type \AgdaOperator{\AgdaDatatype{≡ₓ}}, and whether there is some inhabitant \ab r of this type; i.e., whether there is a proof \ab r : \ab p \AgdaOperator{\AgdaDatatype{≡ₓ₁}} \ab q that the proof of \ab a \AgdaOperator{\AgdaDatatype{≡ₓ}} \ab b is unique. (This property is sometimes called \emph{uniqueness of identity proofs}.)

Perhaps we have two proofs, say, \ab r\sP{3}\sP{3} \ab s : \ab p \AgdaOperator{\AgdaDatatype{≡ₓ₁}} \ab q. Then of course our next question will be whether \ab r \AgdaOperator{\AgdaDatatype{≡ₓ₂}} \ab s has a proof!  But at some level we may decide that the potential to distinguish two proofs of an identity in a meaningful way (so-called \emph{proof relevance}) is not useful or desirable. At that point, say, at level \ab k, we might assume that there is at most one proof of any identity of the form \ab p  \AgdaOperator{\AgdaDatatype{≡ₓₖ}} \ab q. This is called \href{https://www.cs.bham.ac.uk/~mhe/HoTT-UF-in-Agda-Lecture-Notes/HoTT-UF-Agda.html#truncation}{truncation}.

We will see some examples of trunction later when we require it to complete some of the \ualib modules leading up to the proof of Birkhoff's HSP Theorem. Readers who want more details should refer to \href{https://www.cs.bham.ac.uk/~mhe/HoTT-UF-in-Agda-Lecture-Notes/HoTT-UF-Agda.html\#truncation}{Section 34} and \href{https://www.cs.bham.ac.uk/~mhe/HoTT-UF-in-Agda-Lecture-Notes/HoTT-UF-Agda.html\#resizing}{35} of \escardo's notes~\cite{MHE}, or \href{https://homotopytypetheory.org/2012/09/16/truncations-and-truncated-higher-inductive-types/}{Guillaume Brunerie, Truncations and truncated higher inductive types}, or Section 7.1 of the \hottbook.

We take this opportunity to say what it means in type theory to say that a type is a \emph{set}. A type \ab X~\as :~\ab 𝓤~\af ̇ with an identity relation \AgdaDatatype{≡ₓ} is called a \textbf{set} (or 0-\textbf{groupoid}) if for every pair \ab a \ab b \as : \ab X of elements of type \ab X there is at most one proof of \ab a \AgdaDatatype{≡ₓ} \ab b. This is formalized in the \TypeTopology library as follows:
%% \footnote{As \mhe explains, ``at this point, with the definition of these notions, we are entering the realm of univalent mathematics, but not yet needing the univalence axiom.''}
\ccpad
\begin{code}[number=code:is-set]
\>[0]\AgdaFunction{is-set}\AgdaSpace{}%
\AgdaSymbol{:}\AgdaSpace{}%
\AgdaGeneralizable{𝓤}\AgdaSpace{}%
\AgdaOperator{\AgdaFunction{̇}}\AgdaSpace{}%
\AgdaSymbol{→}\AgdaSpace{}%
\AgdaGeneralizable{𝓤}\AgdaSpace{}%
\AgdaOperator{\AgdaFunction{̇}}\<%
\\
\>[0]\AgdaFunction{is-set}\AgdaSpace{}%
\AgdaBound{X}\AgdaSpace{}%
\AgdaSymbol{=}\AgdaSpace{}%
\AgdaSymbol{(}\AgdaBound{x}\AgdaSpace{}%
\AgdaBound{y}\AgdaSpace{}%
\AgdaSymbol{:}\AgdaSpace{}%
\AgdaBound{X}\AgdaSymbol{)}\AgdaSpace{}%
\AgdaSymbol{→}\AgdaSpace{}%
\AgdaFunction{is-subsingleton}\AgdaSpace{}%
\AgdaSymbol{(}\AgdaBound{x}\AgdaSpace{}%
\AgdaOperator{\AgdaDatatype{≡}}\AgdaSpace{}%
\AgdaBound{y}\AgdaSymbol{)}\<%
\end{code}
%%--END: 
%% \noindent \textbf{Notation}. Here are some acronyms that we use frequently.

%% \begin{itemize}
%% \tightlist
%% \item \mhe = \MHE
%% \item \mltt = \MLTT
%% \end{itemize}

\subsubsection{Options}\label{options}
All Agda programs begin by setting some options and by importing from existing libraries (in our case, the \TypeTopology library by Martin Escardo).

Lgical axioms and deduction rules can be specified according to what one wishes to assume. For example, each Agda source code file in the \ualib begins with the following line:
\ccpad
\begin{code}%
\>[0]\AgdaSymbol{\{-\#}\AgdaSpace{}%
\AgdaKeyword{OPTIONS}\AgdaSpace{}%
\AgdaPragma{--without-K}\AgdaSpace{}%
\AgdaPragma{--exact-split}\AgdaSpace{}%
\AgdaPragma{--safe}\AgdaSpace{}%
\AgdaSymbol{\#-\}}\<%
\end{code}
\ccpad
These options control certain foundational assumptions that Agda assumes when checking the program to verify its correctness.

\begin{itemize}
\item
  \texttt{without-K} disables \axiomk; see also the in the \agdarefk in the \agdaref.
\item
  \texttt{exact-split} makes Agda accept only those definitions that
  behave like so-called \emph{judgmental} or \emph{definitional}
  equalities. MHE explains this by saying it ``makes sure that pattern
  matching corresponds to Martin-Löf eliminators;'' see also the
  \patternmatching of the \agdatools documentation.
\item
  \texttt{safe} ensures that nothing is postulated outright---every non-MLTT axiom has to be an explicit assumption (e.g., an argument to a function or module); see also \href{https://agda.readthedocs.io/en/v2.6.1/tools/command-line-options.html\#cmdoption-safe}{this section} of the \agdatools documentation and the \href{https://agda.readthedocs.io/en/v2.6.1/language/safe-agda.html\#safe-agda}{Safe Agda section} of the \agdaref.
\end{itemize}


%% \subsubsection{Modules}\label{modules}

%% The \AgdaKeyword{OPTIONS} directive is followed by some imports or the start of a module.
%% For example, the \Preliminaries module begins with \AgdaKeyword{module}\AgdaSpace{}%
%% \AgdaModule{UALib.Prelude.Preliminaries}\AgdaSpace{}\AgdaKeyword{where}.

%% Sometimes we want to pass in parameters that will be assumed throughout the module. For instance, when working with algebras we often assume they come from a particular fixed signature \AgdaBound{𝑆}, which we could fix as a parameter at the start of a module. We'll see many examples later, but here's an example:
%% \AgdaKeyword{module}\AgdaSpace{}\AgdaModule{\AgdaUnderscore{}}\AgdaSpace{}%
%% \AgdaSymbol{\{}\AgdaBound{𝑆}\AgdaSpace{}%
%% \AgdaSymbol{:}\AgdaSpace{}%
%% \AgdaFunction{Signature}\AgdaSpace{}%
%% \AgdaGeneralizable{𝓞}\AgdaSpace{}%
%% \AgdaGeneralizable{𝓥}\AgdaSymbol{\}}\AgdaSpace{}%
%% \AgdaKeyword{where}.


\subsubsection{Imports from Type Topology}\label{imports-from-type-topology}
Following \MartinEscardo, we refer to universes using capitalized script letters from near the end of the alphabet, e.g., \ab 𝓤, \ab 𝓥, \ab 𝓦, \ab 𝓧, \ab 𝓨, \ab 𝓩, etc. Also, in the \universes module, Escardo defines the \af ̇ operator which maps a universe \ab 𝓤 (i.e., a level) to \Set \ab 𝓤, and the latter has type \Set(\lsuc \ab 𝓤). The level \lzero is renamed \AgdaBound{𝓤₀}, so \AgdaBound{𝓤₀}\af ̇ is an alias for \Set \lzero.\footnote{Incidentally, \Set \lzero corresponds to \texttt{Sort\sP{3}0} in the \href{https://leanprover.github.io/}{Lean} proof assistant language.} Thus, \ab 𝓤 ̇ is simply an alias for \Set \ab 𝓤, and we have \Set \ab 𝓤 : \Set (\lsuc \ab 𝓤). Finally, \Set(\lsuc \lzero) is denoted by \Set \AgdaBound{𝓤₀} ⁺ which we (and \mhe) denote by \AgdaBound{𝓤₀}~⁺~̇.
Table~\ref{tab:dictionary} translates between standard \agda syntax and \mhe/\ualib notation.

\begin{table}
\begin{tabular}{r|l}
Standard \agda                        &          \mhe/\ualib \\
\hline
\AgdaKeyword{Level}          &   \AgdaFunction{Universe}\\
\ab 𝓤 : \AgdaKeyword{Level}  & \ab 𝓤 : \AgdaFunction{Universe}\\
\Set \ab 𝓤                  &       \ab 𝓤 ̇ \\
\lsuc \ab 𝓤                   &    \ab 𝓤 ⁺\\
\Set (\lsuc \ab 𝓤) &    \ab 𝓤 ⁺ ̇\\
\lzero                       &         \AgdaBound{𝓤₀}\\
%% \Set \lzero              &    \AgdaBound{𝓤₀}~~̇\\
%% \lsuc \lzero                  &    \AgdaBound{𝓤₀}⁺\\
%% \Set (\lsuc \lzero) & \AgdaBound{𝓤₀} ⁺ ̇\\
\AgdaFunction{Setω}  &         \AgdaFunction{𝓤ω}
\end{tabular}
\caption{Special notation for universe levels}
\label{tab:dictionary}
\end{table}

To justify the introduction of this somewhat nonstandard notation for universe levels, \mhe points out that the Agda library uses \AgdaKeyword{Level} for universes (so what we write as \ab 𝓤 ̇ is written \Set\sP{3}\ab 𝓤 in standard Agda), but in univalent mathematics the types in \ab 𝓤 ̇ need not be sets, so the standard Agda notation can be misleading.

There will be many occasions calling for a type living in the universe that is the least upper bound of two universes, say, \ab 𝓤 ̇ and \ab 𝓥 ̇ . The universe (\ab 𝓤 \AgdaSymbol{⊔} \ab 𝓥) ̇ denotes this least upper bound.\footnote{Since \AgdaUnderscore{}\AgdaSymbol{⊔}\AgdaUnderscore{} has higher precedence than \AgdaUnderscore{}̇, we could omit parentheses here and simply write \ab 𝓤 \AgdaSymbol{⊔} \ab 𝓥 ̇.} Here \ab 𝓤 \AgdaSymbol{⊔} \ab 𝓥 is used to denote the universe level corresponding to the least upper bound of the levels \ab 𝓤 and \ab 𝓥, where the \AgdaUnderscore{}\AgdaSymbol{⊔}\AgdaUnderscore{} is an \agda primitive designed for precisely this purpose.

\subsubsection{Notation for dependent pairs}\label{Preliminaries.sssec:dependent-pair-type}
\textit{Dependent pair types} (or \textit{Sigma types}) are defined in the \typetopology library as a record, as follows:
\ccpad
%==========================================================
% FORMATTED AGDA CODE
\begin{code}
\>[0]\AgdaKeyword{record}\AgdaSpace{}%
\AgdaRecord{Σ}\AgdaSpace{}%
\AgdaSymbol{\{}\AgdaBound{𝓤}\AgdaSpace{}%
\AgdaBound{𝓥}\AgdaSymbol{\}}\AgdaSpace{}%
\AgdaSymbol{\{}\AgdaBound{X}\AgdaSpace{}%
\AgdaSymbol{:}\AgdaSpace{}%
\AgdaBound{𝓤}\AgdaSpace{}%
\AgdaOperator{\AgdaFunction{̇}}\AgdaSpace{}%
\AgdaSymbol{\}}\AgdaSpace{}%
\AgdaSymbol{(}\AgdaBound{Y}\AgdaSpace{}%
\AgdaSymbol{:}\AgdaSpace{}%
\AgdaBound{X}\AgdaSpace{}%
\AgdaSymbol{→}\AgdaSpace{}%
\AgdaBound{𝓥}\AgdaSpace{}%
\AgdaOperator{\AgdaFunction{̇}}\AgdaSpace{}%
\AgdaSymbol{)}\AgdaSpace{}%
\AgdaSymbol{:}\AgdaSpace{}%
\AgdaBound{𝓤}\AgdaSpace{}%
\AgdaOperator{\AgdaPrimitive{⊔}}\AgdaSpace{}%
\AgdaBound{𝓥}\AgdaSpace{}%
\AgdaOperator{\AgdaFunction{̇}}%
\>[52]\AgdaKeyword{where}\<%
\\
\>[0][@{}l@{\AgdaIndent{0}}]%
\>[2]\AgdaKeyword{constructor}\AgdaSpace{}%
%% \<%
%% \\
%% \>[2][@{}l@{\AgdaIndent{0}}]%
%% \>[3]
\AgdaOperator{\AgdaInductiveConstructor{\AgdaUnderscore{},\AgdaUnderscore{}}}\<%
\\
%
\>[2]\AgdaKeyword{field}\<%
\\
\>[2][@{}l@{\AgdaIndent{0}}]%
\>[3]\AgdaField{pr₁}\AgdaSpace{}%
\AgdaSymbol{:}\AgdaSpace{}%
\AgdaBound{X}\<%
\\
%
\>[3]\AgdaField{pr₂}\AgdaSpace{}%
\AgdaSymbol{:}\AgdaSpace{}%
\AgdaBound{Y}\AgdaSpace{}%
\AgdaField{pr₁}\<%
\\
%
\>[0]\AgdaKeyword{infixr}\AgdaSpace{}%
\AgdaNumber{50}\AgdaSpace{}%
\AgdaOperator{\AgdaInductiveConstructor{\AgdaUnderscore{},\AgdaUnderscore{}}}\<%
\\
\>[0]\<%
\end{code}
\ccpad
Agda's default syntax for the dependent pair type is (\AgdaRecord{Σ}\sP{3}\AgdaSymbol{λ}(\ab x\sP{3}꞉\sP{3}\ab X)\sP{3}\as →\sP{3}\ab y), but we prefer the notation \AgdaRecord{Σ}~\ab x~꞉~\ab X~,~\ab y, which is closer to the standard syntax appearing in the literature. In~\cite{MHE}, Escardo makes the preferred notation available by making the index type explicit, as follows.\footnote{The symbol ꞉ in the expression \AgdaRecord{Σ}\sP{3}\ab x\sP{3}\as ꞉\sP{3}\ab X\sP{3}\AgdaComma\sP{3}\ab y is not the ordinary colon (:); rather, it is the symbol obtained by typing \texttt{\textbackslash{}:4} in \agdatwomode.}
\ccpad
\begin{code}
\>[0]\AgdaFunction{-Σ}\AgdaSpace{}%
\AgdaSymbol{:}\AgdaSpace{}%
\AgdaSymbol{\{}\AgdaBound{𝓤}\AgdaSpace{}%
\AgdaBound{𝓥}\AgdaSpace{}%
\AgdaSymbol{:}\AgdaSpace{}%
\AgdaPostulate{Universe}\AgdaSymbol{\}}\AgdaSpace{}%
\AgdaSymbol{(}\AgdaBound{X}\AgdaSpace{}%
\AgdaSymbol{:}\AgdaSpace{}%
\AgdaBound{𝓤}\AgdaSpace{}%
\AgdaOperator{\AgdaFunction{̇}}\AgdaSpace{}%
\AgdaSymbol{)}\AgdaSpace{}%
\AgdaSymbol{(}\AgdaBound{Y}\AgdaSpace{}%
\AgdaSymbol{:}\AgdaSpace{}%
\AgdaBound{X}\AgdaSpace{}%
\AgdaSymbol{→}\AgdaSpace{}%
\AgdaBound{𝓥}\AgdaSpace{}%
\AgdaOperator{\AgdaFunction{̇}}\AgdaSpace{}%
\AgdaSymbol{)}\AgdaSpace{}%
\AgdaSymbol{→}\AgdaSpace{}%
\AgdaBound{𝓤}\AgdaSpace{}%
\AgdaOperator{\AgdaPrimitive{⊔}}\AgdaSpace{}%
\AgdaBound{𝓥}\AgdaSpace{}%
\AgdaOperator{\AgdaFunction{̇}}\<%
\\
\>[0]\AgdaFunction{-Σ}\AgdaSpace{}%
\AgdaBound{X}\AgdaSpace{}%
\AgdaBound{Y}\AgdaSpace{}%
\AgdaSymbol{=}\AgdaSpace{}%
\AgdaRecord{Σ}\AgdaSpace{}%
\AgdaBound{Y}\<%
\\
\>[0]\AgdaKeyword{syntax}\AgdaSpace{}%
\AgdaFunction{-Σ}\AgdaSpace{}%
\AgdaBound{X}\AgdaSpace{}%
\AgdaSymbol{(}\AgdaSymbol{λ}\AgdaSpace{}%
\AgdaBound{x}\AgdaSpace{}%
\AgdaSymbol{→}\AgdaSpace{}%
\AgdaBound{y}\AgdaSymbol{)}\AgdaSpace{}%
\AgdaSymbol{=}\AgdaSpace{}%
\AgdaFunction{Σ}\AgdaSpace{}%
\AgdaBound{x}\AgdaSpace{}%
\AgdaFunction{꞉}\AgdaSpace{}%
\AgdaBound{X}\AgdaSpace{}%
\AgdaFunction{,}\AgdaSpace{}%
\AgdaBound{y}\<%
\end{code}%
\ccpad

\newcommand\FstUnder{\AgdaOperator{\AgdaFunction{∣\AgdaUnderscore{}∣}}\xspace}
\newcommand\SndUnder{\AgdaOperator{\AgdaFunction{∥\AgdaUnderscore{}∥}}\xspace}
Convenient notations for the first and second projections out of a product are \FstUnder and \SndUnder, respectively. However, to improve readability or to avoid notation clashes with other modules, we sometimes use more standard alternatives, such as \AgdaFunction{pr₁} and \AgdaFunction{pr₂}, or \AgdaFunction{fst} and \AgdaFunction{snd}, or some combination of these.
\ccpad
\begin{code}%
\>[0]\AgdaKeyword{module}\AgdaSpace{}%
\AgdaModule{\AgdaUnderscore{}}\AgdaSpace{}%
\AgdaSymbol{\{}\AgdaBound{𝓤}\AgdaSpace{}%
\AgdaSymbol{:}\AgdaSpace{}%
\AgdaPostulate{Universe}\AgdaSymbol{\}}\AgdaSpace{}%
\AgdaKeyword{where}\<%
\\
\>[0][@{}l@{\AgdaIndent{0}}]%
\>[1]\AgdaOperator{\AgdaFunction{∣\AgdaUnderscore{}∣}}\AgdaSpace{}%
\AgdaFunction{fst}\AgdaSpace{}%
\AgdaSymbol{:}\AgdaSpace{}%
\AgdaSymbol{\{}\AgdaBound{X}\AgdaSpace{}%
\AgdaSymbol{:}\AgdaSpace{}%
\AgdaBound{𝓤}\AgdaSpace{}%
\AgdaOperator{\AgdaFunction{̇}}\AgdaSpace{}%
\AgdaSymbol{\}\{}\AgdaBound{Y}\AgdaSpace{}%
\AgdaSymbol{:}\AgdaSpace{}%
\AgdaBound{X}\AgdaSpace{}%
\AgdaSymbol{→}\AgdaSpace{}%
\AgdaGeneralizable{𝓥}\AgdaSpace{}%
\AgdaOperator{\AgdaFunction{̇}}\AgdaSymbol{\}}\AgdaSpace{}%
\AgdaSymbol{→}\AgdaSpace{}%
\AgdaRecord{Σ}\AgdaSpace{}%
\AgdaBound{Y}\AgdaSpace{}%
\AgdaSymbol{→}\AgdaSpace{}%
\AgdaBound{X}\<%
\\
%
\>[1]\AgdaOperator{\AgdaFunction{∣}}\AgdaSpace{}%
\AgdaBound{x}\AgdaSpace{}%
\AgdaOperator{\AgdaInductiveConstructor{,}}\AgdaSpace{}%
\AgdaBound{y}\AgdaSpace{}%
\AgdaOperator{\AgdaFunction{∣}}\AgdaSpace{}%
\AgdaSymbol{=}\AgdaSpace{}%
\AgdaBound{x}\<%
\\
%
\>[1]\AgdaFunction{fst}\AgdaSpace{}%
\AgdaSymbol{(}\AgdaBound{x}\AgdaSpace{}%
\AgdaOperator{\AgdaInductiveConstructor{,}}\AgdaSpace{}%
\AgdaBound{y}\AgdaSymbol{)}\AgdaSpace{}%
\AgdaSymbol{=}\AgdaSpace{}%
\AgdaBound{x}\<%
\\
%
\>[1]\AgdaOperator{\AgdaFunction{∥\AgdaUnderscore{}∥}}\AgdaSpace{}%
\AgdaFunction{snd}\AgdaSpace{}%
\AgdaSymbol{:}\AgdaSpace{}%
\AgdaSymbol{\{}\AgdaBound{X}\AgdaSpace{}%
\AgdaSymbol{:}\AgdaSpace{}%
\AgdaBound{𝓤}\AgdaSpace{}%
\AgdaOperator{\AgdaFunction{̇}}\AgdaSpace{}%
\AgdaSymbol{\}\{}\AgdaBound{Y}\AgdaSpace{}%
\AgdaSymbol{:}\AgdaSpace{}%
\AgdaBound{X}\AgdaSpace{}%
\AgdaSymbol{→}\AgdaSpace{}%
\AgdaGeneralizable{𝓥}\AgdaSpace{}%
\AgdaOperator{\AgdaFunction{̇}}\AgdaSpace{}%
\AgdaSymbol{\}}\AgdaSpace{}%
\AgdaSymbol{→}\AgdaSpace{}%
\AgdaSymbol{(}\AgdaBound{z}\AgdaSpace{}%
\AgdaSymbol{:}\AgdaSpace{}%
\AgdaRecord{Σ}\AgdaSpace{}%
\AgdaBound{Y}\AgdaSymbol{)}\AgdaSpace{}%
\AgdaSymbol{→}\AgdaSpace{}%
\AgdaBound{Y}\AgdaSpace{}%
\AgdaSymbol{(}\AgdaFunction{pr₁}\AgdaSpace{}%
\AgdaBound{z}\AgdaSymbol{)}\<%
\\
%
\>[1]\AgdaOperator{\AgdaFunction{∥}}\AgdaSpace{}%
\AgdaBound{x}\AgdaSpace{}%
\AgdaOperator{\AgdaInductiveConstructor{,}}\AgdaSpace{}%
\AgdaBound{y}\AgdaSpace{}%
\AgdaOperator{\AgdaFunction{∥}}\AgdaSpace{}%
\AgdaSymbol{=}\AgdaSpace{}%
\AgdaBound{y}\<%
\\
%
\>[1]\AgdaFunction{snd}\AgdaSpace{}%
\AgdaSymbol{(}\AgdaBound{x}\AgdaSpace{}%
\AgdaOperator{\AgdaInductiveConstructor{,}}\AgdaSpace{}%
\AgdaBound{y}\AgdaSymbol{)}\AgdaSpace{}%
\AgdaSymbol{=}\AgdaSpace{}%
\AgdaBound{y}\<%
\end{code}

%%%%%%%%%%%%%%%%%%%%%% UNUSED %%%%%%%%%%%%%%%%%%%%%%%%%%%%%%%%%%%%%%%%%%%%
\begin{comment}
\subsubsection{Dependent function type}\label{Preliminaries.sssec:dependent-function-type}
The so-called \textbf{dependent function type} (or ``Pi type'') is defined in the \typetopology library as follows.
\ccpad
\begin{code}
\>[0]\AgdaFunction{Π}\AgdaSpace{}%
\AgdaSymbol{:}\AgdaSpace{}%
\AgdaSymbol{\{}\AgdaBound{X}\AgdaSpace{}%
\AgdaSymbol{:}\AgdaSpace{}%
\AgdaGeneralizable{𝓤}\AgdaSpace{}%
\AgdaOperator{\AgdaFunction{̇}}\AgdaSpace{}%
\AgdaSymbol{\}}\AgdaSpace{}%
\AgdaSymbol{(}\AgdaBound{A}\AgdaSpace{}%
\AgdaSymbol{:}\AgdaSpace{}%
\AgdaBound{X}\AgdaSpace{}%
\AgdaSymbol{→}\AgdaSpace{}%
\AgdaGeneralizable{𝓥}\AgdaSpace{}%
\AgdaOperator{\AgdaFunction{̇}}\AgdaSpace{}%
\AgdaSymbol{)}\AgdaSpace{}%
\AgdaSymbol{→}\AgdaSpace{}%
\AgdaGeneralizable{𝓤}\AgdaSpace{}%
\AgdaOperator{\AgdaPrimitive{⊔}}\AgdaSpace{}%
\AgdaGeneralizable{𝓥}\AgdaSpace{}%
\AgdaOperator{\AgdaFunction{̇}}\<%
\\
\>[0]\AgdaFunction{Π}\AgdaSpace{}%
\AgdaSymbol{\{}\AgdaBound{𝓤}\AgdaSymbol{\}}\AgdaSpace{}%
\AgdaSymbol{\{}\AgdaBound{𝓥}\AgdaSymbol{\}}\AgdaSpace{}%
\AgdaSymbol{\{}\AgdaBound{X}\AgdaSymbol{\}}\AgdaSpace{}%
\AgdaBound{A}\AgdaSpace{}%
\AgdaSymbol{=}\AgdaSpace{}%
\AgdaSymbol{(}\AgdaBound{x}\AgdaSpace{}%
\AgdaSymbol{:}\AgdaSpace{}%
\AgdaBound{X}\AgdaSymbol{)}\AgdaSpace{}%
\AgdaSymbol{→}\AgdaSpace{}%
\AgdaBound{A}\AgdaSpace{}%
\AgdaBound{x}\<%
\end{code}
\ccpad
To make the syntax for \texttt{Π} conform to the standard notation for Pi types, \mhe uses the same trick as the one used above for Sigma types.
\ccpad
\begin{code}
\>[0]\AgdaFunction{-Π}\AgdaSpace{}%
\AgdaSymbol{:}\AgdaSpace{}%
\AgdaSymbol{\{}\AgdaBound{𝓤}\AgdaSpace{}%
\AgdaBound{𝓥}\AgdaSpace{}%
\AgdaSymbol{:}\AgdaSpace{}%
\AgdaPostulate{Universe}\AgdaSymbol{\}}\AgdaSpace{}%
\AgdaSymbol{(}\AgdaBound{X}\AgdaSpace{}%
\AgdaSymbol{:}\AgdaSpace{}%
\AgdaBound{𝓤}\AgdaSpace{}%
\AgdaOperator{\AgdaFunction{̇}}\AgdaSpace{}%
\AgdaSymbol{)}\AgdaSpace{}%
\AgdaSymbol{(}\AgdaBound{Y}\AgdaSpace{}%
\AgdaSymbol{:}\AgdaSpace{}%
\AgdaBound{X}\AgdaSpace{}%
\AgdaSymbol{→}\AgdaSpace{}%
\AgdaBound{𝓥}\AgdaSpace{}%
\AgdaOperator{\AgdaFunction{̇}}\AgdaSpace{}%
\AgdaSymbol{)}\AgdaSpace{}%
\AgdaSymbol{→}\AgdaSpace{}%
\AgdaBound{𝓤}\AgdaSpace{}%
\AgdaOperator{\AgdaPrimitive{⊔}}\AgdaSpace{}%
\AgdaBound{𝓥}\AgdaSpace{}%
\AgdaOperator{\AgdaFunction{̇}}\<%
\\
\>[0]\AgdaFunction{-Π}\AgdaSpace{}%
\AgdaBound{X}\AgdaSpace{}%
\AgdaBound{Y}\AgdaSpace{}%
\AgdaSymbol{=}\AgdaSpace{}%
\AgdaFunction{Π}\AgdaSpace{}%
\AgdaBound{Y}\<%
\\
%
\>[0]\AgdaKeyword{syntax}\AgdaSpace{}%
\AgdaFunction{-Π}\AgdaSpace{}%
\AgdaBound{A}\AgdaSpace{}%
\AgdaSymbol{(λ}\AgdaSpace{}%
\AgdaBound{x}\AgdaSpace{}%
\AgdaSymbol{→}\AgdaSpace{}%
\AgdaBound{b}\AgdaSymbol{)}\AgdaSpace{}%
\AgdaSymbol{=}\AgdaSpace{}%
\AgdaFunction{Π}\AgdaSpace{}%
\AgdaBound{x}\AgdaSpace{}%
\AgdaFunction{꞉}\AgdaSpace{}%
\AgdaBound{A}\AgdaSpace{}%
\AgdaFunction{,}\AgdaSpace{}%
\AgdaBound{b}\<%
\\
\>[0]\AgdaKeyword{infixr}\AgdaSpace{}%
\AgdaNumber{-1}\AgdaSpace{}%
\AgdaFunction{-Π}\<%
\end{code}
\end{comment}
%%%%%%%%%%%%%%%%%%%%%%%%%%%%%%%%%%%%%%%%%

\subsubsection{Truncation and sets}\label{Preliminaries.sssec:truncation}
In general, we may have many inhabitants of a given type and, via the Curry-Howard correspondence, many proofs of a given proposition. For instance, suppose we have a type \ab X and an identity relation \AgdaOperator{\AgdaDatatype{≡ₓ}} on \ab X. Then, given two inhabitants \ab a and \ab b of type \ab X, we may ask whether \ab a \AgdaOperator{\AgdaDatatype{≡ₓ}} \ab b.

Suppose \ab p and \ab q inhabit the identity type \ab a \AgdaOperator{\AgdaDatatype{≡ₓ}} b; that is, \ab p and \ab q are proofs of \ab a \AgdaOperator{\AgdaDatatype{≡ₓ}} \ab b, in which case we write \ab p\sP{3}\sP{3}\ab q : \ab a \AgdaOperator{\AgdaDatatype{≡ₓ}} \ab b. Then we might wonder whether and in what sense are the two proofs \ab p and \ab q the ``same.'' We are asking about an identity type on the identity type \AgdaOperator{\AgdaDatatype{≡ₓ}}, and whether there is some inhabitant \ab r of this type; i.e., whether there is a proof \ab r : \ab p \AgdaOperator{\AgdaDatatype{≡ₓ₁}} \ab q that the proof of \ab a \AgdaOperator{\AgdaDatatype{≡ₓ}} \ab b is unique. (This property is sometimes called \emph{uniqueness of identity proofs}.)

Perhaps we have two proofs, say, \ab r\sP{3}\sP{3} \ab s : \ab p \AgdaOperator{\AgdaDatatype{≡ₓ₁}} \ab q. Then of course our next question will be whether \ab r \AgdaOperator{\AgdaDatatype{≡ₓ₂}} \ab s has a proof!  But at some level we may decide that the potential to distinguish two proofs of an identity in a meaningful way (so-called \emph{proof relevance}) is not useful or desirable. At that point, say, at level \ab k, we might assume that there is at most one proof of any identity of the form \ab p  \AgdaOperator{\AgdaDatatype{≡ₓₖ}} \ab q. This is called \href{https://www.cs.bham.ac.uk/~mhe/HoTT-UF-in-Agda-Lecture-Notes/HoTT-UF-Agda.html#truncation}{truncation}.

We will see some examples of trunction later when we require it to complete some of the \ualib modules leading up to the proof of Birkhoff's HSP Theorem. Readers who want more details should refer to \href{https://www.cs.bham.ac.uk/~mhe/HoTT-UF-in-Agda-Lecture-Notes/HoTT-UF-Agda.html\#truncation}{Section 34} and \href{https://www.cs.bham.ac.uk/~mhe/HoTT-UF-in-Agda-Lecture-Notes/HoTT-UF-Agda.html\#resizing}{35} of MHE's notes, or \href{https://homotopytypetheory.org/2012/09/16/truncations-and-truncated-higher-inductive-types/}{Guillaume Brunerie, Truncations and truncated higher inductive types}, or Section 7.1 of the \hottbook.

We take this opportunity to say what it means in type theory to say that a type is a \emph{set}. A type \ab X \as : \ab 𝓤 \af ̇ with an identity relation \AgdaDatatype{≡ₓ} is called a \textbf{set} (or 0-\textbf{groupoid}) if for every pair \ab a \ab b \as : \ab X of elements of type \ab X there is at most one proof of \ab a \AgdaDatatype{≡ₓ} \ab b. This is formalized in the \TypeTopology library as follows:
%% \footnote{As \mhe explains, ``at this point, with the definition of these notions, we are entering the realm of univalent mathematics, but not yet needing the univalence axiom.''}
\ccpad
\begin{code}
\>[0]\AgdaFunction{is-set}\AgdaSpace{}%
\AgdaSymbol{:}\AgdaSpace{}%
\AgdaGeneralizable{𝓤}\AgdaSpace{}%
\AgdaOperator{\AgdaFunction{̇}}\AgdaSpace{}%
\AgdaSymbol{→}\AgdaSpace{}%
\AgdaGeneralizable{𝓤}\AgdaSpace{}%
\AgdaOperator{\AgdaFunction{̇}}\<%
\\
\>[0]\AgdaFunction{is-set}\AgdaSpace{}%
\AgdaBound{X}\AgdaSpace{}%
\AgdaSymbol{=}\AgdaSpace{}%
\AgdaSymbol{(}\AgdaBound{x}\AgdaSpace{}%
\AgdaBound{y}\AgdaSpace{}%
\AgdaSymbol{:}\AgdaSpace{}%
\AgdaBound{X}\AgdaSymbol{)}\AgdaSpace{}%
\AgdaSymbol{→}\AgdaSpace{}%
\AgdaFunction{is-subsingleton}\AgdaSpace{}%
\AgdaSymbol{(}\AgdaBound{x}\AgdaSpace{}%
\AgdaOperator{\AgdaDatatype{≡}}\AgdaSpace{}%
\AgdaBound{y}\AgdaSymbol{)}\<%
\end{code}
-----------------------------------------------







\subsection{Equality}\label{ssec:equality}

%%--BEGIN: \input{Equality-short.tex}-----------------------------------------------

%% \subsubsection{refl}\label{Equality.sssec:refl}
Perhaps the most important types in type theory are the equality types. The \emph{definitional equality} we use is a standard one and is often referred to as ``reflexivity'' or ``refl''.  In our case, it is defined in the Identity-Type module of the \TypeTopology library, but apart from syntax it is equivalent to the identity type used in most other Agda libraries. Here is the definition.
%% full listing of the \texttt{Identity-Type} module.
\ccpad
\begin{code}%
\>[0]\AgdaKeyword{data}\AgdaSpace{}%
\AgdaOperator{\AgdaDatatype{\AgdaUnderscore{}≡\AgdaUnderscore{}}}\AgdaSpace{}%
\AgdaSymbol{\{}\AgdaBound{𝓤}\AgdaSymbol{\}}\AgdaSpace{}%
\AgdaSymbol{\{}\AgdaBound{X}\AgdaSpace{}%
\AgdaSymbol{:}\AgdaSpace{}%
\AgdaBound{𝓤}\AgdaSpace{}%
\AgdaOperator{\AgdaFunction{̇}}\AgdaSpace{}%
\AgdaSymbol{\}}\AgdaSpace{}%
\AgdaSymbol{:}\AgdaSpace{}%
\AgdaBound{X}\AgdaSpace{}%
\AgdaSymbol{→}\AgdaSpace{}%
\AgdaBound{X}\AgdaSpace{}%
\AgdaSymbol{→}\AgdaSpace{}%
\AgdaBound{𝓤}\AgdaSpace{}%
\AgdaOperator{\AgdaFunction{̇}}\AgdaSpace{}%
\AgdaKeyword{where}\<%
\\
\>[0][@{}l@{\AgdaIndent{0}}]%
\>[2]\AgdaInductiveConstructor{refl}\AgdaSpace{}%
\AgdaSymbol{:}\AgdaSpace{}%
\AgdaSymbol{\{}\AgdaBound{x}\AgdaSpace{}%
\AgdaSymbol{:}\AgdaSpace{}%
\AgdaBound{X}\AgdaSymbol{\}}\AgdaSpace{}%
\AgdaSymbol{→}\AgdaSpace{}%
\AgdaBound{x}\AgdaSpace{}%
\AgdaOperator{\AgdaDatatype{≡}}\AgdaSpace{}%
\AgdaBound{x}\<%
\end{code}
\ccpad

%% (MOST OF Equality-short.tex OMITTED)

%%--BEGIN: \input{Equality-short.tex}-----------------------------------------------







\subsection{Inverses, Epics and Monics}\label{sec:inverses}
This section describes some of the more important definitions in the \ualibInverses module of the \agdaualib.
%%
%%
%%-- BEGIN: In particular, we define the syntax we use to represent the semantic concept of the \textbf{inverse image} of a function.
\ccpad
\begin{code}%
%% \>[0]\AgdaSymbol{\{-\#}\AgdaSpace{}%
%% \AgdaKeyword{OPTIONS}\AgdaSpace{}%
%% \AgdaPragma{--without-K}\AgdaSpace{}%
%% \AgdaPragma{--exact-split}\AgdaSpace{}%
%% \AgdaPragma{--safe}\AgdaSpace{}%
%% \AgdaSymbol{\#-\}}\<%
%% \\
%% %
%% \ccpad
%% %
%% \>[0]\AgdaKeyword{module}\AgdaSpace{}%
%% \AgdaModule{UALib.Prelude.Inverses}\AgdaSpace{}%
%% \AgdaKeyword{where}\<%
%% \\
%% %
%% \ccpad
%% %
%% \>[0]\AgdaKeyword{open}\AgdaSpace{}%
%% \AgdaKeyword{import}\AgdaSpace{}%
%% \AgdaModule{UALib.Prelude.Equality}\AgdaSpace{}%
%% \AgdaKeyword{public}\<%
%% \\
%% %
%% \ccpad
%% %
%% \>[0]\AgdaKeyword{open}\AgdaSpace{}%
%% \AgdaKeyword{import}\AgdaSpace{}%
%% \AgdaModule{UALib.Prelude.Preliminaries}\AgdaSpace{}%
%% \AgdaKeyword{using}\AgdaSpace{}%
%% \AgdaSymbol{(}\AgdaOperator{\AgdaFunction{\AgdaUnderscore{}⁻¹}}\AgdaSymbol{;}\AgdaSpace{}%
%% \AgdaFunction{funext}\AgdaSymbol{;}\AgdaSpace{}%
%% \AgdaOperator{\AgdaFunction{\AgdaUnderscore{}∘\AgdaUnderscore{}}}\AgdaSymbol{;}\AgdaSpace{}%
%% \AgdaOperator{\AgdaFunction{\AgdaUnderscore{}∙\AgdaUnderscore{}}}\AgdaSymbol{;}\AgdaSpace{}%
%% \AgdaFunction{𝑖𝑑}\AgdaSymbol{;}\AgdaSpace{}%
%% \AgdaFunction{fst}\AgdaSymbol{;}\AgdaSpace{}%
%% \AgdaFunction{snd}\AgdaSymbol{;}\AgdaSpace{}%
%% \AgdaFunction{is-set}\AgdaSymbol{;}\AgdaSpace{}%
%% \AgdaFunction{is-embedding}\AgdaSymbol{;}\<%
%% \\
%% \>[0][@{}l@{\AgdaIndent{0}}]%
%% \>[1]\AgdaFunction{transport}\AgdaSymbol{;}\AgdaSpace{}%
%% \AgdaFunction{to-Σ-≡}\AgdaSymbol{;}\AgdaSpace{}%
%% \AgdaFunction{invertible}\AgdaSymbol{;}\AgdaSpace{}%
%% \AgdaFunction{equivs-are-embeddings}\AgdaSymbol{;}\AgdaSpace{}%
%% \AgdaFunction{invertibles-are-equivs}\AgdaSymbol{;}\AgdaSpace{}%
%% \AgdaFunction{fiber}\AgdaSymbol{;}\AgdaSpace{}%
%% \AgdaInductiveConstructor{𝓇ℯ𝒻𝓁}\AgdaSymbol{)}\AgdaSpace{}%
%% \AgdaKeyword{public}\<%
%% \\
%% %
%% \ccpad
%% %
\>[0]\AgdaKeyword{module}\AgdaSpace{}%
\AgdaModule{\AgdaUnderscore{}}\AgdaSpace{}%
\AgdaSymbol{\{}\AgdaBound{𝓤}\AgdaSpace{}%
\AgdaBound{𝓦}\AgdaSpace{}%
\AgdaSymbol{:}\AgdaSpace{}%
\AgdaPostulate{Universe}\AgdaSymbol{\}}\AgdaSpace{}%
\AgdaKeyword{where}\<%
\\
%
\\
\>[0][@{}l@{\AgdaIndent{0}}]%
\>[1]\AgdaKeyword{data}\AgdaSpace{}%
\AgdaOperator{\AgdaDatatype{Image\AgdaUnderscore{}∋\AgdaUnderscore{}}}\AgdaSpace{}%
\AgdaSymbol{\{}\AgdaBound{A}\AgdaSpace{}%
\AgdaSymbol{:}\AgdaSpace{}%
\AgdaBound{𝓤}\AgdaSpace{}%
\AgdaOperator{\AgdaFunction{̇}}\AgdaSpace{}%
\AgdaSymbol{\}\{}\AgdaBound{B}\AgdaSpace{}%
\AgdaSymbol{:}\AgdaSpace{}%
\AgdaBound{𝓦}\AgdaSpace{}%
\AgdaOperator{\AgdaFunction{̇}}\AgdaSpace{}%
\AgdaSymbol{\}(}\AgdaBound{f}\AgdaSpace{}%
\AgdaSymbol{:}\AgdaSpace{}%
\AgdaBound{A}\AgdaSpace{}%
\AgdaSymbol{→}\AgdaSpace{}%
\AgdaBound{B}\AgdaSymbol{)}\AgdaSpace{}%
\AgdaSymbol{:}\AgdaSpace{}%
\AgdaBound{B}\AgdaSpace{}%
\AgdaSymbol{→}\AgdaSpace{}%
\AgdaBound{𝓤}\AgdaSpace{}%
\AgdaOperator{\AgdaPrimitive{⊔}}\AgdaSpace{}%
\AgdaBound{𝓦}\AgdaSpace{}%
\AgdaOperator{\AgdaFunction{̇}}\<%
\\
\>[1][@{}l@{\AgdaIndent{0}}]%
\>[2]\AgdaKeyword{where}\<%
\\
%
\>[2]\AgdaInductiveConstructor{im}\AgdaSpace{}%
\AgdaSymbol{:}\AgdaSpace{}%
\AgdaSymbol{(}\AgdaBound{x}\AgdaSpace{}%
\AgdaSymbol{:}\AgdaSpace{}%
\AgdaBound{A}\AgdaSymbol{)}\AgdaSpace{}%
\AgdaSymbol{→}\AgdaSpace{}%
\AgdaOperator{\AgdaDatatype{Image}}\AgdaSpace{}%
\AgdaBound{f}\AgdaSpace{}%
\AgdaOperator{\AgdaDatatype{∋}}\AgdaSpace{}%
\AgdaBound{f}\AgdaSpace{}%
\AgdaBound{x}\<%
\\
%
\>[2]\AgdaInductiveConstructor{eq}\AgdaSpace{}%
\AgdaSymbol{:}\AgdaSpace{}%
\AgdaSymbol{(}\AgdaBound{b}\AgdaSpace{}%
\AgdaSymbol{:}\AgdaSpace{}%
\AgdaBound{B}\AgdaSymbol{)}\AgdaSpace{}%
\AgdaSymbol{→}\AgdaSpace{}%
\AgdaSymbol{(}\AgdaBound{a}\AgdaSpace{}%
\AgdaSymbol{:}\AgdaSpace{}%
\AgdaBound{A}\AgdaSymbol{)}\AgdaSpace{}%
\AgdaSymbol{→}\AgdaSpace{}%
\AgdaBound{b}\AgdaSpace{}%
\AgdaOperator{\AgdaDatatype{≡}}\AgdaSpace{}%
\AgdaBound{f}\AgdaSpace{}%
\AgdaBound{a}\AgdaSpace{}%
\AgdaSymbol{→}\AgdaSpace{}%
\AgdaOperator{\AgdaDatatype{Image}}\AgdaSpace{}%
\AgdaBound{f}\AgdaSpace{}%
\AgdaOperator{\AgdaDatatype{∋}}\AgdaSpace{}%
\AgdaBound{b}\<%
\end{code}
\ccpad
Note that an inhabitant of \af{Image} \ab f \ad ∋ \ab b is a dependent pair (\ab a \ac \ab p), where \ab a \as : \ab A and \ab p \as : \ab b  \ad ≡ \ab f \ab a is a proof that \ab f maps \ab a to \ab b. Thus, a proof that \ab b belongs to the image of \ab f (i.e., an inhabitant of \af{Image} \ab f \ad ∋ \ab b), is always accompanied by a witness \ab a \as : \ab A, and a proof that \ab b \ad{≡} \ab f \ab a, so the (pseudo-)inverse can actually be \emph{computed}.
%% \ccpad
%% \begin{code}
%% \>[1]\AgdaFunction{ImageIsImage}%
%% \>[84I]\AgdaSymbol{:}\AgdaSpace{}%
%% \AgdaSymbol{\{}\AgdaBound{A}\AgdaSpace{}%
%% \AgdaSymbol{:}\AgdaSpace{}%
%% \AgdaBound{𝓤}\AgdaSpace{}%
%% \AgdaOperator{\AgdaFunction{̇}}\AgdaSpace{}%
%% \AgdaSymbol{\}\{}\AgdaBound{B}\AgdaSpace{}%
%% \AgdaSymbol{:}\AgdaSpace{}%
%% \AgdaBound{𝓦}\AgdaSpace{}%
%% \AgdaOperator{\AgdaFunction{̇}}\AgdaSpace{}%
%% \AgdaSymbol{\}}\<%
%% \\
%% \>[84I][@{}l@{\AgdaIndent{0}}]%
%% \>[15]\AgdaSymbol{(}\AgdaBound{f}\AgdaSpace{}%
%% \AgdaSymbol{:}\AgdaSpace{}%
%% \AgdaBound{A}\AgdaSpace{}%
%% \AgdaSymbol{→}\AgdaSpace{}%
%% \AgdaBound{B}\AgdaSymbol{)}\AgdaSpace{}%
%% \AgdaSymbol{(}\AgdaBound{b}\AgdaSpace{}%
%% \AgdaSymbol{:}\AgdaSpace{}%
%% \AgdaBound{B}\AgdaSymbol{)}\AgdaSpace{}%
%% \AgdaSymbol{(}\AgdaBound{a}\AgdaSpace{}%
%% \AgdaSymbol{:}\AgdaSpace{}%
%% \AgdaBound{A}\AgdaSymbol{)}\<%
%% \\
%% \>[1][@{}l@{\AgdaIndent{0}}]%
%% \>[2]\AgdaSymbol{→}%
%% \>[15]\AgdaBound{b}\AgdaSpace{}%
%% \AgdaOperator{\AgdaDatatype{≡}}\AgdaSpace{}%
%% \AgdaBound{f}\AgdaSpace{}%
%% \AgdaBound{a}\<%
%% \\
%% \>[2][@{}l@{\AgdaIndent{0}}]%
%% \>[15]\AgdaComment{-------------------------}\<%
%% \\
%% %
%% \>[2]\AgdaSymbol{→}%
%% \>[15]\AgdaOperator{\AgdaDatatype{Image}}\AgdaSpace{}%
%% \AgdaBound{f}\AgdaSpace{}%
%% \AgdaOperator{\AgdaDatatype{∋}}\AgdaSpace{}%
%% \AgdaBound{b}\<%
%% \\
%% %
%% \>[1]\AgdaFunction{ImageIsImage}\AgdaSpace{}%
%% \AgdaSymbol{\{}\AgdaBound{A}\AgdaSymbol{\}\{}\AgdaBound{B}\AgdaSymbol{\}}\AgdaSpace{}%
%% \AgdaBound{f}\AgdaSpace{}%
%% \AgdaBound{b}\AgdaSpace{}%
%% \AgdaBound{a}\AgdaSpace{}%
%% \AgdaBound{b≡fa}\AgdaSpace{}%
%% \AgdaSymbol{=}\AgdaSpace{}%
%% \AgdaInductiveConstructor{eq}\AgdaSpace{}%
%% \AgdaBound{b}\AgdaSpace{}%
%% \AgdaBound{a}\AgdaSpace{}%
%% \AgdaBound{b≡fa}\<%
%% \end{code}
%% \ccpad

For convenience, we define a pseudo-inverse function, which we call \af{Inv}, that takes \ab b \as : \ab B and (\ab a \ac \ab p) \as : \af{Image} \ab f \ad ∋ \ab b and returns \ab a.
\ccpad
\begin{code}%
\>[0][@{}l@{\AgdaIndent{1}}]%
\>[1]\AgdaFunction{Inv}\AgdaSpace{}%
\AgdaSymbol{:}\AgdaSpace{}%
\AgdaSymbol{\{}\AgdaBound{A}\AgdaSpace{}%
\AgdaSymbol{:}\AgdaSpace{}%
\AgdaBound{𝓤}\AgdaSpace{}%
\AgdaOperator{\AgdaFunction{̇}}\AgdaSpace{}%
\AgdaSymbol{\}\{}\AgdaBound{B}\AgdaSpace{}%
\AgdaSymbol{:}\AgdaSpace{}%
\AgdaBound{𝓦}\AgdaSpace{}%
\AgdaOperator{\AgdaFunction{̇}}\AgdaSpace{}%
\AgdaSymbol{\}(}\AgdaBound{f}\AgdaSpace{}%
\AgdaSymbol{:}\AgdaSpace{}%
\AgdaBound{A}\AgdaSpace{}%
\AgdaSymbol{→}\AgdaSpace{}%
\AgdaBound{B}\AgdaSymbol{)(}\AgdaBound{b}\AgdaSpace{}%
\AgdaSymbol{:}\AgdaSpace{}%
\AgdaBound{B}\AgdaSymbol{)}\AgdaSpace{}%
\AgdaSymbol{→}\AgdaSpace{}%
\AgdaOperator{\AgdaDatatype{Image}}\AgdaSpace{}%
\AgdaBound{f}\AgdaSpace{}%
\AgdaOperator{\AgdaDatatype{∋}}\AgdaSpace{}%
\AgdaBound{b}%
\>[61]\AgdaSymbol{→}%
\>[64]\AgdaBound{A}\<%
\\
%
\>[1]\AgdaFunction{Inv}\AgdaSpace{}%
\AgdaBound{f}\AgdaSpace{}%
\AgdaDottedPattern{\AgdaSymbol{.(}}\AgdaDottedPattern{\AgdaBound{f}}\AgdaSpace{}%
\AgdaDottedPattern{\AgdaBound{a}}\AgdaDottedPattern{\AgdaSymbol{)}}\AgdaSpace{}%
\AgdaSymbol{(}\AgdaInductiveConstructor{im}\AgdaSpace{}%
\AgdaBound{a}\AgdaSymbol{)}\AgdaSpace{}%
\AgdaSymbol{=}\AgdaSpace{}%
\AgdaBound{a}\<%
\\
%
\>[1]\AgdaFunction{Inv}\AgdaSpace{}%
\AgdaBound{f}\AgdaSpace{}%
\AgdaBound{b}\AgdaSpace{}%
\AgdaSymbol{(}\AgdaInductiveConstructor{eq}\AgdaSpace{}%
\AgdaBound{b}\AgdaSpace{}%
\AgdaBound{a}\AgdaSpace{}%
\AgdaBound{b≡fa}\AgdaSymbol{)}\AgdaSpace{}%
\AgdaSymbol{=}\AgdaSpace{}%
\AgdaBound{a}\<%
\end{code}
\ccpad
Of course, we can prove that \af{Inv} \ab f is really the (right-)inverse of \ab f.
\ccpad
\begin{code}%
\>[0][@{}l@{\AgdaIndent{1}}]%
\>[1]\AgdaFunction{InvIsInv}%
\>[156I]\AgdaSymbol{:}%
\>[157I]\AgdaSymbol{\{}\AgdaBound{A}\AgdaSpace{}%
\AgdaSymbol{:}\AgdaSpace{}%
\AgdaBound{𝓤}\AgdaSpace{}%
\AgdaOperator{\AgdaFunction{̇}}\AgdaSpace{}%
\AgdaSymbol{\}}\AgdaSpace{}%
\AgdaSymbol{\{}\AgdaBound{B}\AgdaSpace{}%
\AgdaSymbol{:}\AgdaSpace{}%
\AgdaBound{𝓦}\AgdaSpace{}%
\AgdaOperator{\AgdaFunction{̇}}\AgdaSpace{}%
\AgdaSymbol{\}}\AgdaSpace{}%
\AgdaSymbol{(}\AgdaBound{f}\AgdaSpace{}%
\AgdaSymbol{:}\AgdaSpace{}%
\AgdaBound{A}\AgdaSpace{}%
\AgdaSymbol{→}\AgdaSpace{}%
\AgdaBound{B}\AgdaSymbol{)}\<%
\\
\>[.][@{}l@{}]\<[157I]%
\>[12]\AgdaSymbol{(}\AgdaBound{b}\AgdaSpace{}%
\AgdaSymbol{:}\AgdaSpace{}%
\AgdaBound{B}\AgdaSymbol{)}\AgdaSpace{}%
\AgdaSymbol{(}\AgdaBound{b∈Imgf}\AgdaSpace{}%
\AgdaSymbol{:}\AgdaSpace{}%
\AgdaOperator{\AgdaDatatype{Image}}\AgdaSpace{}%
\AgdaBound{f}\AgdaSpace{}%
\AgdaOperator{\AgdaDatatype{∋}}\AgdaSpace{}%
\AgdaBound{b}\AgdaSymbol{)}\<%
\\
\>[156I][@{}l@{\AgdaIndent{0}}]%
\>[11]\AgdaComment{---------------------------------}\<%
\\
\>[1][@{}l@{\AgdaIndent{0}}]%
\>[2]\AgdaSymbol{→}%
\>[12]\AgdaBound{f}\AgdaSpace{}%
\AgdaSymbol{(}\AgdaFunction{Inv}\AgdaSpace{}%
\AgdaBound{f}\AgdaSpace{}%
\AgdaBound{b}\AgdaSpace{}%
\AgdaBound{b∈Imgf}\AgdaSymbol{)}\AgdaSpace{}%
\AgdaOperator{\AgdaDatatype{≡}}\AgdaSpace{}%
\AgdaBound{b}\<%
\\
%
\>[1]\AgdaFunction{InvIsInv}\AgdaSpace{}%
\AgdaBound{f}\AgdaSpace{}%
\AgdaDottedPattern{\AgdaSymbol{.(}}\AgdaDottedPattern{\AgdaBound{f}}\AgdaSpace{}%
\AgdaDottedPattern{\AgdaBound{a}}\AgdaDottedPattern{\AgdaSymbol{)}}\AgdaSpace{}%
\AgdaSymbol{(}\AgdaInductiveConstructor{im}\AgdaSpace{}%
\AgdaBound{a}\AgdaSymbol{)}\AgdaSpace{}%
\AgdaSymbol{=}\AgdaSpace{}%
\AgdaInductiveConstructor{refl}\AgdaSpace{}%
\AgdaSymbol{\AgdaUnderscore{}}\<%
\\
%
\>[1]\AgdaFunction{InvIsInv}\AgdaSpace{}%
\AgdaBound{f}\AgdaSpace{}%
\AgdaBound{b}\AgdaSpace{}%
\AgdaSymbol{(}\AgdaInductiveConstructor{eq}\AgdaSpace{}%
\AgdaBound{b}\AgdaSpace{}%
\AgdaBound{a}\AgdaSpace{}%
\AgdaBound{b≡fa}\AgdaSymbol{)}\AgdaSpace{}%
\AgdaSymbol{=}\AgdaSpace{}%
\AgdaBound{b≡fa}\AgdaSpace{}%
\AgdaOperator{\AgdaFunction{⁻¹}}\<%
\end{code}

\subsubsection{Surjective functions}\label{surjective-functions}
An epic (or surjective) function from type \AgdaTyped{A}{𝓤 ̇} to type \AgdaTyped{B}{𝓦 ̇} is as an inhabitant of the \af{Epic} type, which we define as follows.
\ccpad
\begin{code}%
\>[0][@{}l@{\AgdaIndent{1}}]%
\>[1]\AgdaFunction{Epic}\AgdaSpace{}%
\AgdaSymbol{:}\AgdaSpace{}%
\AgdaSymbol{\{}\AgdaBound{A}\AgdaSpace{}%
\AgdaSymbol{:}\AgdaSpace{}%
\AgdaBound{𝓤}\AgdaSpace{}%
\AgdaOperator{\AgdaFunction{̇}}\AgdaSpace{}%
\AgdaSymbol{\}}\AgdaSpace{}%
\AgdaSymbol{\{}\AgdaBound{B}\AgdaSpace{}%
\AgdaSymbol{:}\AgdaSpace{}%
\AgdaBound{𝓦}\AgdaSpace{}%
\AgdaOperator{\AgdaFunction{̇}}\AgdaSpace{}%
\AgdaSymbol{\}}\AgdaSpace{}%
\AgdaSymbol{(}\AgdaBound{g}\AgdaSpace{}%
\AgdaSymbol{:}\AgdaSpace{}%
\AgdaBound{A}\AgdaSpace{}%
\AgdaSymbol{→}\AgdaSpace{}%
\AgdaBound{B}\AgdaSymbol{)}\AgdaSpace{}%
\AgdaSymbol{→}%
\>[45]\AgdaBound{𝓤}\AgdaSpace{}%
\AgdaOperator{\AgdaPrimitive{⊔}}\AgdaSpace{}%
\AgdaBound{𝓦}\AgdaSpace{}%
\AgdaOperator{\AgdaFunction{̇}}\<%
\\
%
\>[1]\AgdaFunction{Epic}\AgdaSpace{}%
\AgdaBound{g}\AgdaSpace{}%
\AgdaSymbol{=}\AgdaSpace{}%
\AgdaSymbol{∀}\AgdaSpace{}%
\AgdaBound{y}\AgdaSpace{}%
\AgdaSymbol{→}\AgdaSpace{}%
\AgdaOperator{\AgdaDatatype{Image}}\AgdaSpace{}%
\AgdaBound{g}\AgdaSpace{}%
\AgdaOperator{\AgdaDatatype{∋}}\AgdaSpace{}%
\AgdaBound{y}\<%
\\
\>[0]\<%
\end{code}
\ccpad
We obtain the right-inverse (or pseudo-inverse) of an epic function \af f by applying the function \af{EpicInv} (which we now define) to the function \af f along with a proof, \ab{fE} \as : \af{Epic} \ab f, that \af f is surjective.
\ccpad
\begin{code}%
\>[0][@{}l@{\AgdaIndent{1}}]%
\>[1]\AgdaFunction{EpicInv}\AgdaSpace{}%
\AgdaSymbol{:}%
\>[233I]\AgdaSymbol{\{}\AgdaBound{A}\AgdaSpace{}%
\AgdaSymbol{:}\AgdaSpace{}%
\AgdaBound{𝓤}\AgdaSpace{}%
\AgdaOperator{\AgdaFunction{̇}}\AgdaSpace{}%
\AgdaSymbol{\}}\AgdaSpace{}%
\AgdaSymbol{\{}\AgdaBound{B}\AgdaSpace{}%
\AgdaSymbol{:}\AgdaSpace{}%
\AgdaBound{𝓦}\AgdaSpace{}%
\AgdaOperator{\AgdaFunction{̇}}\AgdaSpace{}%
\AgdaSymbol{\}}\<%
\\
\>[.][@{}l@{}]\<[233I]%
\>[11]\AgdaSymbol{(}\AgdaBound{f}\AgdaSpace{}%
\AgdaSymbol{:}\AgdaSpace{}%
\AgdaBound{A}\AgdaSpace{}%
\AgdaSymbol{→}\AgdaSpace{}%
\AgdaBound{B}\AgdaSymbol{)}\AgdaSpace{}%
\AgdaSymbol{→}\AgdaSpace{}%
\AgdaFunction{Epic}\AgdaSpace{}%
\AgdaBound{f}\<%
\\
%
\>[11]\AgdaComment{-------------------------}\<%
\\
\>[1][@{}l@{\AgdaIndent{0}}]%
\>[2]\AgdaSymbol{→}%
\>[11]\AgdaBound{B}\AgdaSpace{}%
\AgdaSymbol{→}\AgdaSpace{}%
\AgdaBound{A}\<%
\\
%
\>[1]\AgdaFunction{EpicInv}\AgdaSpace{}%
\AgdaBound{f}\AgdaSpace{}%
\AgdaBound{fE}\AgdaSpace{}%
\AgdaBound{b}\AgdaSpace{}%
\AgdaSymbol{=}\AgdaSpace{}%
\AgdaFunction{Inv}\AgdaSpace{}%
\AgdaBound{f}\AgdaSpace{}%
\AgdaBound{b}\AgdaSpace{}%
\AgdaSymbol{(}\AgdaBound{fE}\AgdaSpace{}%
\AgdaBound{b}\AgdaSymbol{)}\<%
\end{code}
\ccpad
The function defined by \af{EpicInv} \ab f \ab{fE} is indeed the right-inverse of \ab f.
\ccpad
\begin{code}%
\>[0][@{}l@{\AgdaIndent{1}}]%
\>[1]\AgdaFunction{EpicInvIsRightInv}\AgdaSpace{}%
\AgdaSymbol{:}%
\>[262I]\AgdaFunction{funext}\AgdaSpace{}%
\AgdaBound{𝓦}\AgdaSpace{}%
\AgdaBound{𝓦}\AgdaSpace{}%
\AgdaSymbol{→}\AgdaSpace{}%
\AgdaSymbol{\{}\AgdaBound{A}\AgdaSpace{}%
\AgdaSymbol{:}\AgdaSpace{}%
\AgdaBound{𝓤}\AgdaSpace{}%
\AgdaOperator{\AgdaFunction{̇}}\AgdaSpace{}%
\AgdaSymbol{\}}\AgdaSpace{}%
\AgdaSymbol{\{}\AgdaBound{B}\AgdaSpace{}%
\AgdaSymbol{:}\AgdaSpace{}%
\AgdaBound{𝓦}\AgdaSpace{}%
\AgdaOperator{\AgdaFunction{̇}}\AgdaSpace{}%
\AgdaSymbol{\}}\<%
\\
\>[.][@{}l@{}]\<[262I]%
\>[21]\AgdaSymbol{(}\AgdaBound{f}\AgdaSpace{}%
\AgdaSymbol{:}\AgdaSpace{}%
\AgdaBound{A}\AgdaSpace{}%
\AgdaSymbol{→}\AgdaSpace{}%
\AgdaBound{B}\AgdaSymbol{)}%
\>[34]\AgdaSymbol{(}\AgdaBound{fE}\AgdaSpace{}%
\AgdaSymbol{:}\AgdaSpace{}%
\AgdaFunction{Epic}\AgdaSpace{}%
\AgdaBound{f}\AgdaSymbol{)}\<%
\\
%
\>[21]\AgdaComment{----------------------------}\<%
\\
\>[1][@{}l@{\AgdaIndent{0}}]%
\>[2]\AgdaSymbol{→}%
\>[21]\AgdaBound{f}\AgdaSpace{}%
\AgdaOperator{\AgdaFunction{∘}}\AgdaSpace{}%
\AgdaSymbol{(}\AgdaFunction{EpicInv}\AgdaSpace{}%
\AgdaBound{f}\AgdaSpace{}%
\AgdaBound{fE}\AgdaSymbol{)}\AgdaSpace{}%
\AgdaOperator{\AgdaDatatype{≡}}\AgdaSpace{}%
\AgdaFunction{𝑖𝑑}\AgdaSpace{}%
\AgdaBound{B}\<%
\\
%
\>[1]\AgdaFunction{EpicInvIsRightInv}\AgdaSpace{}%
\AgdaBound{fe}\AgdaSpace{}%
\AgdaBound{f}\AgdaSpace{}%
\AgdaBound{fE}\AgdaSpace{}%
\AgdaSymbol{=}\AgdaSpace{}%
\AgdaBound{fe}\AgdaSpace{}%
\AgdaSymbol{(λ}\AgdaSpace{}%
\AgdaBound{x}\AgdaSpace{}%
\AgdaSymbol{→}\AgdaSpace{}%
\AgdaFunction{InvIsInv}\AgdaSpace{}%
\AgdaBound{f}\AgdaSpace{}%
\AgdaBound{x}\AgdaSpace{}%
\AgdaSymbol{(}\AgdaBound{fE}\AgdaSpace{}%
\AgdaBound{x}\AgdaSymbol{))}\<%
\end{code}

\subsubsection{Injective functions}\label{injective-functions}
We say that a function \AgdaTyped{g}{A → B} is monic (or injective) if we have a proof of \af{Monic} \ab g, where
\ccpad
\begin{code}%
\>[0][@{}l@{\AgdaIndent{1}}]%
\>[1]\AgdaFunction{Monic}\AgdaSpace{}%
\AgdaSymbol{:}\AgdaSpace{}%
\AgdaSymbol{\{}\AgdaBound{A}\AgdaSpace{}%
\AgdaSymbol{:}\AgdaSpace{}%
\AgdaBound{𝓤}\AgdaSpace{}%
\AgdaOperator{\AgdaFunction{̇}}\AgdaSpace{}%
\AgdaSymbol{\}}\AgdaSpace{}%
\AgdaSymbol{\{}\AgdaBound{B}\AgdaSpace{}%
\AgdaSymbol{:}\AgdaSpace{}%
\AgdaBound{𝓦}\AgdaSpace{}%
\AgdaOperator{\AgdaFunction{̇}}\AgdaSpace{}%
\AgdaSymbol{\}(}\AgdaBound{g}\AgdaSpace{}%
\AgdaSymbol{:}\AgdaSpace{}%
\AgdaBound{A}\AgdaSpace{}%
\AgdaSymbol{→}\AgdaSpace{}%
\AgdaBound{B}\AgdaSymbol{)}\AgdaSpace{}%
\AgdaSymbol{→}\AgdaSpace{}%
\AgdaBound{𝓤}\AgdaSpace{}%
\AgdaOperator{\AgdaPrimitive{⊔}}\AgdaSpace{}%
\AgdaBound{𝓦}\AgdaSpace{}%
\AgdaOperator{\AgdaFunction{̇}}\<%
\\
%
\>[1]\AgdaFunction{Monic}\AgdaSpace{}%
\AgdaBound{g}\AgdaSpace{}%
\AgdaSymbol{=}\AgdaSpace{}%
\AgdaSymbol{∀}\AgdaSpace{}%
\AgdaBound{a₁}\AgdaSpace{}%
\AgdaBound{a₂}\AgdaSpace{}%
\AgdaSymbol{→}\AgdaSpace{}%
\AgdaBound{g}\AgdaSpace{}%
\AgdaBound{a₁}\AgdaSpace{}%
\AgdaOperator{\AgdaDatatype{≡}}\AgdaSpace{}%
\AgdaBound{g}\AgdaSpace{}%
\AgdaBound{a₂}\AgdaSpace{}%
\AgdaSymbol{→}\AgdaSpace{}%
\AgdaBound{a₁}\AgdaSpace{}%
\AgdaOperator{\AgdaDatatype{≡}}\AgdaSpace{}%
\AgdaBound{a₂}\<%
\end{code}
\ccpad
Again, we obtain a pseudo-inverse. Here it is obtained by applying the function \af{MonicInv} to \ab g and a proof that \ab g is monic.
\ccpad
\begin{code}%
\>[0][@{}l@{\AgdaIndent{1}}]%
\>[1]\AgdaComment{--The (pseudo-)inverse of a monic function}\<%
\\
%
\>[1]\AgdaFunction{MonicInv}%
\>[338I]\AgdaSymbol{:}%
\>[339I]\AgdaSymbol{\{}\AgdaBound{A}\AgdaSpace{}%
\AgdaSymbol{:}\AgdaSpace{}%
\AgdaBound{𝓤}\AgdaSpace{}%
\AgdaOperator{\AgdaFunction{̇}}\AgdaSpace{}%
\AgdaSymbol{\}}\AgdaSpace{}%
\AgdaSymbol{\{}\AgdaBound{B}\AgdaSpace{}%
\AgdaSymbol{:}\AgdaSpace{}%
\AgdaBound{𝓦}\AgdaSpace{}%
\AgdaOperator{\AgdaFunction{̇}}\AgdaSpace{}%
\AgdaSymbol{\}}\<%
\\
\>[.][@{}l@{}]\<[339I]%
\>[12]\AgdaSymbol{(}\AgdaBound{f}\AgdaSpace{}%
\AgdaSymbol{:}\AgdaSpace{}%
\AgdaBound{A}\AgdaSpace{}%
\AgdaSymbol{→}\AgdaSpace{}%
\AgdaBound{B}\AgdaSymbol{)}%
\>[25]\AgdaSymbol{→}%
\>[28]\AgdaFunction{Monic}\AgdaSpace{}%
\AgdaBound{f}\<%
\\
\>[338I][@{}l@{\AgdaIndent{0}}]%
\>[11]\AgdaComment{-----------------------------}\<%
\\
\>[1][@{}l@{\AgdaIndent{0}}]%
\>[2]\AgdaSymbol{→}%
\>[12]\AgdaSymbol{(}\AgdaBound{b}\AgdaSpace{}%
\AgdaSymbol{:}\AgdaSpace{}%
\AgdaBound{B}\AgdaSymbol{)}\AgdaSpace{}%
\AgdaSymbol{→}%
\>[23]\AgdaOperator{\AgdaDatatype{Image}}\AgdaSpace{}%
\AgdaBound{f}\AgdaSpace{}%
\AgdaOperator{\AgdaDatatype{∋}}\AgdaSpace{}%
\AgdaBound{b}%
\>[36]\AgdaSymbol{→}%
\>[39]\AgdaBound{A}\<%
\\
%
\\[\AgdaEmptyExtraSkip]%
%
\>[1]\AgdaFunction{MonicInv}\AgdaSpace{}%
\AgdaBound{f}\AgdaSpace{}%
\AgdaSymbol{\AgdaUnderscore{}}\AgdaSpace{}%
\AgdaSymbol{=}\AgdaSpace{}%
\AgdaSymbol{λ}\AgdaSpace{}%
\AgdaBound{b}\AgdaSpace{}%
\AgdaBound{Imf∋b}\AgdaSpace{}%
\AgdaSymbol{→}\AgdaSpace{}%
\AgdaFunction{Inv}\AgdaSpace{}%
\AgdaBound{f}\AgdaSpace{}%
\AgdaBound{b}\AgdaSpace{}%
\AgdaBound{Imf∋b}\<%
\end{code}
\ccpad
The function defined by \AgdaCatchallClause{MonicInv f fM} is the left-inverse of \ab f.
\ccpad
\begin{code}%
\>[0][@{}l@{\AgdaIndent{1}}]%
\>[1]\AgdaComment{--The (psudo-)inverse of a monic is the left inverse.}\<%
\\
%
\>[1]\AgdaFunction{MonicInvIsLeftInv}%
\>[371I]\AgdaSymbol{:}%
\>[372I]\AgdaSymbol{\{}\AgdaBound{A}\AgdaSpace{}%
\AgdaSymbol{:}\AgdaSpace{}%
\AgdaBound{𝓤}\AgdaSpace{}%
\AgdaOperator{\AgdaFunction{̇}}\AgdaSpace{}%
\AgdaSymbol{\}\{}\AgdaBound{B}\AgdaSpace{}%
\AgdaSymbol{:}\AgdaSpace{}%
\AgdaBound{𝓦}\AgdaSpace{}%
\AgdaOperator{\AgdaFunction{̇}}\AgdaSpace{}%
\AgdaSymbol{\}}\<%
\\
\>[.][@{}l@{}]\<[372I]%
\>[21]\AgdaSymbol{(}\AgdaBound{f}\AgdaSpace{}%
\AgdaSymbol{:}\AgdaSpace{}%
\AgdaBound{A}\AgdaSpace{}%
\AgdaSymbol{→}\AgdaSpace{}%
\AgdaBound{B}\AgdaSymbol{)}\AgdaSpace{}%
\AgdaSymbol{(}\AgdaBound{fmonic}\AgdaSpace{}%
\AgdaSymbol{:}\AgdaSpace{}%
\AgdaFunction{Monic}\AgdaSpace{}%
\AgdaBound{f}\AgdaSymbol{)(}\AgdaBound{x}\AgdaSpace{}%
\AgdaSymbol{:}\AgdaSpace{}%
\AgdaBound{A}\AgdaSymbol{)}\<%
\\
\>[371I][@{}l@{\AgdaIndent{0}}]%
\>[20]\AgdaComment{----------------------------------------}\<%
\\
\>[1][@{}l@{\AgdaIndent{0}}]%
\>[3]\AgdaSymbol{→}%
\>[21]\AgdaSymbol{(}\AgdaFunction{MonicInv}\AgdaSpace{}%
\AgdaBound{f}\AgdaSpace{}%
\AgdaBound{fmonic}\AgdaSymbol{)}\AgdaSpace{}%
\AgdaSymbol{(}\AgdaBound{f}\AgdaSpace{}%
\AgdaBound{x}\AgdaSymbol{)}\AgdaSpace{}%
\AgdaSymbol{(}\AgdaInductiveConstructor{im}\AgdaSpace{}%
\AgdaBound{x}\AgdaSymbol{)}\AgdaSpace{}%
\AgdaOperator{\AgdaDatatype{≡}}\AgdaSpace{}%
\AgdaBound{x}\<%
\\
%
\\[\AgdaEmptyExtraSkip]%
%
\>[1]\AgdaFunction{MonicInvIsLeftInv}\AgdaSpace{}%
\AgdaBound{f}\AgdaSpace{}%
\AgdaBound{fmonic}\AgdaSpace{}%
\AgdaBound{x}\AgdaSpace{}%
\AgdaSymbol{=}\AgdaSpace{}%
\AgdaInductiveConstructor{refl}\AgdaSpace{}%
\AgdaSymbol{\AgdaUnderscore{}}\<%
\end{code}

\subsubsection{Bijective functions}\label{bijective-functions}
Finally, bijective functions are defined.
\ccpad
\begin{code}%
\>[0][@{}l@{\AgdaIndent{1}}]%
\>[1]\AgdaFunction{Bijective}\AgdaSpace{}%
\AgdaSymbol{:}\AgdaSpace{}%
\AgdaSymbol{\{}\AgdaBound{A}\AgdaSpace{}%
\AgdaSymbol{:}\AgdaSpace{}%
\AgdaBound{𝓤}\AgdaSpace{}%
\AgdaOperator{\AgdaFunction{̇}}\AgdaSpace{}%
\AgdaSymbol{\}\{}\AgdaBound{B}\AgdaSpace{}%
\AgdaSymbol{:}\AgdaSpace{}%
\AgdaBound{𝓦}\AgdaSpace{}%
\AgdaOperator{\AgdaFunction{̇}}\AgdaSpace{}%
\AgdaSymbol{\}(}\AgdaBound{f}\AgdaSpace{}%
\AgdaSymbol{:}\AgdaSpace{}%
\AgdaBound{A}\AgdaSpace{}%
\AgdaSymbol{→}\AgdaSpace{}%
\AgdaBound{B}\AgdaSymbol{)}\AgdaSpace{}%
\AgdaSymbol{→}\AgdaSpace{}%
\AgdaBound{𝓤}\AgdaSpace{}%
\AgdaOperator{\AgdaPrimitive{⊔}}\AgdaSpace{}%
\AgdaBound{𝓦}\AgdaSpace{}%
\AgdaOperator{\AgdaFunction{̇}}\<%
\\
%
\>[1]\AgdaFunction{Bijective}\AgdaSpace{}%
\AgdaBound{f}\AgdaSpace{}%
\AgdaSymbol{=}\AgdaSpace{}%
\AgdaFunction{Epic}\AgdaSpace{}%
\AgdaBound{f}\AgdaSpace{}%
\AgdaOperator{\AgdaFunction{×}}\AgdaSpace{}%
\AgdaFunction{Monic}\AgdaSpace{}%
\AgdaBound{f}\<%
\\
%
\\[\AgdaEmptyExtraSkip]%
%
\>[1]\AgdaFunction{Inverse}\AgdaSpace{}%
\AgdaSymbol{:}%
\>[432I]\AgdaSymbol{\{}\AgdaBound{A}\AgdaSpace{}%
\AgdaSymbol{:}\AgdaSpace{}%
\AgdaBound{𝓤}\AgdaSpace{}%
\AgdaOperator{\AgdaFunction{̇}}\AgdaSpace{}%
\AgdaSymbol{\}}\AgdaSpace{}%
\AgdaSymbol{\{}\AgdaBound{B}\AgdaSpace{}%
\AgdaSymbol{:}\AgdaSpace{}%
\AgdaBound{𝓦}\AgdaSpace{}%
\AgdaOperator{\AgdaFunction{̇}}\AgdaSpace{}%
\AgdaSymbol{\}}\<%
\\
\>[432I][@{}l@{\AgdaIndent{0}}]%
\>[12]\AgdaSymbol{(}\AgdaBound{f}\AgdaSpace{}%
\AgdaSymbol{:}\AgdaSpace{}%
\AgdaBound{A}\AgdaSpace{}%
\AgdaSymbol{→}\AgdaSpace{}%
\AgdaBound{B}\AgdaSymbol{)}\AgdaSpace{}%
\AgdaSymbol{→}\AgdaSpace{}%
\AgdaFunction{Bijective}\AgdaSpace{}%
\AgdaBound{f}\<%
\\
%
\>[12]\AgdaComment{-------------------------}\<%
\\
\>[1][@{}l@{\AgdaIndent{0}}]%
\>[2]\AgdaSymbol{→}%
\>[12]\AgdaBound{B}\AgdaSpace{}%
\AgdaSymbol{→}\AgdaSpace{}%
\AgdaBound{A}\<%
\\
%
\>[1]\AgdaFunction{Inverse}\AgdaSpace{}%
\AgdaBound{f}\AgdaSpace{}%
\AgdaBound{fbi}\AgdaSpace{}%
\AgdaBound{b}\AgdaSpace{}%
\AgdaSymbol{=}\AgdaSpace{}%
\AgdaFunction{Inv}\AgdaSpace{}%
\AgdaBound{f}\AgdaSpace{}%
\AgdaBound{b}\AgdaSpace{}%
\AgdaSymbol{(}\AgdaFunction{fst}\AgdaSymbol{(}\AgdaSpace{}%
\AgdaBound{fbi}\AgdaSpace{}%
\AgdaSymbol{)}\AgdaSpace{}%
\AgdaBound{b}\AgdaSymbol{)}\<%
\end{code}
%% OMITTING NEUTRAL ELEMENTS
% \subsubsection{Neutral elements}\label{neutral-elements}
The next three lemmas appeared in the \texttt{UF-Base} and \texttt{UF-Equiv} modules which were (at one time) part of Matin Escsardo's UF Agda repository.
\ccpad
\begin{code}%
\>[0]\<%
\\
\>[0]\AgdaFunction{refl-left-neutral}\AgdaSpace{}%
\AgdaSymbol{:}\AgdaSpace{}%
\AgdaSymbol{\{}\AgdaBound{𝓤}\AgdaSpace{}%
\AgdaSymbol{:}\AgdaSpace{}%
\AgdaPostulate{Universe}\AgdaSymbol{\}}\AgdaSpace{}%
\AgdaSymbol{\{}\AgdaBound{X}\AgdaSpace{}%
\AgdaSymbol{:}\AgdaSpace{}%
\AgdaBound{𝓤}\AgdaSpace{}%
\AgdaOperator{\AgdaFunction{̇}}\AgdaSpace{}%
\AgdaSymbol{\}}\AgdaSpace{}%
\AgdaSymbol{\{}\AgdaBound{x}\AgdaSpace{}%
\AgdaBound{y}\AgdaSpace{}%
\AgdaSymbol{:}\AgdaSpace{}%
\AgdaBound{X}\AgdaSymbol{\}}\AgdaSpace{}%
\AgdaSymbol{(}\AgdaBound{p}\AgdaSpace{}%
\AgdaSymbol{:}\AgdaSpace{}%
\AgdaBound{x}\AgdaSpace{}%
\AgdaOperator{\AgdaDatatype{≡}}\AgdaSpace{}%
\AgdaBound{y}\AgdaSymbol{)}\AgdaSpace{}%
\AgdaSymbol{→}\AgdaSpace{}%
\AgdaSymbol{(}\AgdaInductiveConstructor{refl}\AgdaSpace{}%
\AgdaSymbol{\AgdaUnderscore{})}\AgdaSpace{}%
\AgdaOperator{\AgdaFunction{∙}}\AgdaSpace{}%
\AgdaBound{p}\AgdaSpace{}%
\AgdaOperator{\AgdaDatatype{≡}}\AgdaSpace{}%
\AgdaBound{p}\<%
\\
\>[0]\AgdaFunction{refl-left-neutral}\AgdaSpace{}%
\AgdaSymbol{(}\AgdaInductiveConstructor{refl}\AgdaSpace{}%
\AgdaSymbol{\AgdaUnderscore{})}\AgdaSpace{}%
\AgdaSymbol{=}\AgdaSpace{}%
\AgdaInductiveConstructor{refl}\AgdaSpace{}%
\AgdaSymbol{\AgdaUnderscore{}}\<%
\\
%
\\[\AgdaEmptyExtraSkip]%
\>[0]\AgdaFunction{refl-right-neutral}\AgdaSpace{}%
\AgdaSymbol{:}\AgdaSpace{}%
\AgdaSymbol{\{}\AgdaBound{𝓤}\AgdaSpace{}%
\AgdaSymbol{:}\AgdaSpace{}%
\AgdaPostulate{Universe}\AgdaSymbol{\}\{}\AgdaBound{X}\AgdaSpace{}%
\AgdaSymbol{:}\AgdaSpace{}%
\AgdaBound{𝓤}\AgdaSpace{}%
\AgdaOperator{\AgdaFunction{̇}}\AgdaSpace{}%
\AgdaSymbol{\}}\AgdaSpace{}%
\AgdaSymbol{\{}\AgdaBound{x}\AgdaSpace{}%
\AgdaBound{y}\AgdaSpace{}%
\AgdaSymbol{:}\AgdaSpace{}%
\AgdaBound{X}\AgdaSymbol{\}}\AgdaSpace{}%
\AgdaSymbol{(}\AgdaBound{p}\AgdaSpace{}%
\AgdaSymbol{:}\AgdaSpace{}%
\AgdaBound{x}\AgdaSpace{}%
\AgdaOperator{\AgdaDatatype{≡}}\AgdaSpace{}%
\AgdaBound{y}\AgdaSymbol{)}\AgdaSpace{}%
\AgdaSymbol{→}\AgdaSpace{}%
\AgdaBound{p}\AgdaSpace{}%
\AgdaOperator{\AgdaDatatype{≡}}\AgdaSpace{}%
\AgdaBound{p}\AgdaSpace{}%
\AgdaOperator{\AgdaFunction{∙}}\AgdaSpace{}%
\AgdaSymbol{(}\AgdaInductiveConstructor{refl}\AgdaSpace{}%
\AgdaSymbol{\AgdaUnderscore{})}\<%
\\
\>[0]\AgdaFunction{refl-right-neutral}\AgdaSpace{}%
\AgdaBound{p}\AgdaSpace{}%
\AgdaSymbol{=}\AgdaSpace{}%
\AgdaInductiveConstructor{refl}\AgdaSpace{}%
\AgdaSymbol{\AgdaUnderscore{}}\<%
\\
%
\\[\AgdaEmptyExtraSkip]%
\>[0]\AgdaFunction{identifications-in-fibers}\AgdaSpace{}%
\AgdaSymbol{:}%
\>[521I]\AgdaSymbol{\{}\AgdaBound{𝓤}\AgdaSpace{}%
\AgdaSymbol{:}\AgdaSpace{}%
\AgdaPostulate{Universe}\AgdaSymbol{\}}\AgdaSpace{}%
\AgdaSymbol{\{}\AgdaBound{X}\AgdaSpace{}%
\AgdaSymbol{:}\AgdaSpace{}%
\AgdaBound{𝓤}\AgdaSpace{}%
\AgdaOperator{\AgdaFunction{̇}}\AgdaSpace{}%
\AgdaSymbol{\}}\AgdaSpace{}%
\AgdaSymbol{\{}\AgdaBound{Y}\AgdaSpace{}%
\AgdaSymbol{:}\AgdaSpace{}%
\AgdaGeneralizable{𝓥}\AgdaSpace{}%
\AgdaOperator{\AgdaFunction{̇}}\AgdaSpace{}%
\AgdaSymbol{\}}\AgdaSpace{}%
\AgdaSymbol{(}\AgdaBound{f}\AgdaSpace{}%
\AgdaSymbol{:}\AgdaSpace{}%
\AgdaBound{X}\AgdaSpace{}%
\AgdaSymbol{→}\AgdaSpace{}%
\AgdaBound{Y}\AgdaSymbol{)}\<%
\\
\>[.][@{}l@{}]\<[521I]%
\>[28]\AgdaSymbol{(}\AgdaBound{y}\AgdaSpace{}%
\AgdaSymbol{:}\AgdaSpace{}%
\AgdaBound{Y}\AgdaSymbol{)}\AgdaSpace{}%
\AgdaSymbol{(}\AgdaBound{x}\AgdaSpace{}%
\AgdaBound{x'}\AgdaSpace{}%
\AgdaSymbol{:}\AgdaSpace{}%
\AgdaBound{X}\AgdaSymbol{)}\AgdaSpace{}%
\AgdaSymbol{(}\AgdaBound{p}\AgdaSpace{}%
\AgdaSymbol{:}\AgdaSpace{}%
\AgdaBound{f}\AgdaSpace{}%
\AgdaBound{x}\AgdaSpace{}%
\AgdaOperator{\AgdaDatatype{≡}}\AgdaSpace{}%
\AgdaBound{y}\AgdaSymbol{)}\AgdaSpace{}%
\AgdaSymbol{(}\AgdaBound{p'}\AgdaSpace{}%
\AgdaSymbol{:}\AgdaSpace{}%
\AgdaBound{f}\AgdaSpace{}%
\AgdaBound{x'}\AgdaSpace{}%
\AgdaOperator{\AgdaDatatype{≡}}\AgdaSpace{}%
\AgdaBound{y}\AgdaSymbol{)}\<%
\\
\>[0][@{}l@{\AgdaIndent{0}}]%
\>[1]\AgdaSymbol{→}%
\>[28]\AgdaSymbol{(}\AgdaFunction{Σ}\AgdaSpace{}%
\AgdaBound{γ}\AgdaSpace{}%
\AgdaFunction{꞉}\AgdaSpace{}%
\AgdaBound{x}\AgdaSpace{}%
\AgdaOperator{\AgdaDatatype{≡}}\AgdaSpace{}%
\AgdaBound{x'}\AgdaSpace{}%
\AgdaFunction{,}\AgdaSpace{}%
\AgdaFunction{ap}\AgdaSpace{}%
\AgdaBound{f}\AgdaSpace{}%
\AgdaBound{γ}\AgdaSpace{}%
\AgdaOperator{\AgdaFunction{∙}}\AgdaSpace{}%
\AgdaBound{p'}\AgdaSpace{}%
\AgdaOperator{\AgdaDatatype{≡}}\AgdaSpace{}%
\AgdaBound{p}\AgdaSymbol{)}\AgdaSpace{}%
\AgdaSymbol{→}\AgdaSpace{}%
\AgdaSymbol{(}\AgdaBound{x}\AgdaSpace{}%
\AgdaOperator{\AgdaInductiveConstructor{,}}\AgdaSpace{}%
\AgdaBound{p}\AgdaSymbol{)}\AgdaSpace{}%
\AgdaOperator{\AgdaDatatype{≡}}\AgdaSpace{}%
\AgdaSymbol{(}\AgdaBound{x'}\AgdaSpace{}%
\AgdaOperator{\AgdaInductiveConstructor{,}}\AgdaSpace{}%
\AgdaBound{p'}\AgdaSymbol{)}\<%
\\
\>[0]\AgdaFunction{identifications-in-fibers}\AgdaSpace{}%
\AgdaBound{f}\AgdaSpace{}%
\AgdaDottedPattern{\AgdaSymbol{.(}}\AgdaDottedPattern{\AgdaBound{f}}\AgdaSpace{}%
\AgdaDottedPattern{\AgdaBound{x}}\AgdaDottedPattern{\AgdaSymbol{)}}\AgdaSpace{}%
\AgdaBound{x}\AgdaSpace{}%
\AgdaDottedPattern{\AgdaSymbol{.}}\AgdaDottedPattern{\AgdaBound{x}}\AgdaSpace{}%
\AgdaInductiveConstructor{𝓇ℯ𝒻𝓁}\AgdaSpace{}%
\AgdaBound{p'}\AgdaSpace{}%
\AgdaSymbol{(}\AgdaInductiveConstructor{𝓇ℯ𝒻𝓁}\AgdaSpace{}%
\AgdaOperator{\AgdaInductiveConstructor{,}}\AgdaSpace{}%
\AgdaBound{r}\AgdaSymbol{)}\AgdaSpace{}%
\AgdaSymbol{=}\AgdaSpace{}%
\AgdaFunction{g}\<%
\\
\>[0][@{}l@{\AgdaIndent{0}}]%
\>[1]\AgdaKeyword{where}\<%
\\
\>[1][@{}l@{\AgdaIndent{0}}]%
\>[2]\AgdaFunction{g}\AgdaSpace{}%
\AgdaSymbol{:}\AgdaSpace{}%
\AgdaBound{x}\AgdaSpace{}%
\AgdaOperator{\AgdaInductiveConstructor{,}}\AgdaSpace{}%
\AgdaInductiveConstructor{𝓇ℯ𝒻𝓁}\AgdaSpace{}%
\AgdaOperator{\AgdaDatatype{≡}}\AgdaSpace{}%
\AgdaBound{x}\AgdaSpace{}%
\AgdaOperator{\AgdaInductiveConstructor{,}}\AgdaSpace{}%
\AgdaBound{p'}\<%
\\
%
\>[2]\AgdaFunction{g}\AgdaSpace{}%
\AgdaSymbol{=}\AgdaSpace{}%
\AgdaFunction{ap}\AgdaSpace{}%
\AgdaSymbol{(λ}\AgdaSpace{}%
\AgdaBound{-}\AgdaSpace{}%
\AgdaSymbol{→}\AgdaSpace{}%
\AgdaSymbol{(}\AgdaBound{x}\AgdaSpace{}%
\AgdaOperator{\AgdaInductiveConstructor{,}}\AgdaSpace{}%
\AgdaBound{-}\AgdaSymbol{))}\AgdaSpace{}%
\AgdaSymbol{(}\AgdaBound{r}\AgdaSpace{}%
\AgdaOperator{\AgdaFunction{⁻¹}}\AgdaSpace{}%
\AgdaOperator{\AgdaFunction{∙}}\AgdaSpace{}%
\AgdaFunction{refl-left-neutral}\AgdaSpace{}%
\AgdaSymbol{\AgdaUnderscore{})}\<%
\\
\>[0]\<%
\end{code}


\subsubsection{Injective functions are set embeddings}\label{injective-functions-are-set-embeddings}
This is the first point at which \href{UALib.Preface.html\#truncation}{truncation} comes into play. An
\href{https://www.cs.bham.ac.uk/~mhe/HoTT-UF-in-Agda-Lecture-Notes/HoTT-UF-Agda.html\#embeddings}{embedding} is defined in the \TypeTopology library as follows:
\ccpad
\begin{code}
\>[0]\AgdaFunction{is-embedding}\AgdaSpace{}%
\AgdaSymbol{:}\AgdaSpace{}%
\AgdaSymbol{\{}\AgdaBound{X}\AgdaSpace{}%
\AgdaSymbol{:}\AgdaSpace{}%
\AgdaGeneralizable{𝓤}\AgdaSpace{}%
\AgdaOperator{\AgdaFunction{̇}}\AgdaSpace{}%
\AgdaSymbol{\}}\AgdaSpace{}%
\AgdaSymbol{\{}\AgdaBound{Y}\AgdaSpace{}%
\AgdaSymbol{:}\AgdaSpace{}%
\AgdaGeneralizable{𝓥}\AgdaSpace{}%
\AgdaOperator{\AgdaFunction{̇}}\AgdaSpace{}%
\AgdaSymbol{\}}\AgdaSpace{}%
\AgdaSymbol{→}\AgdaSpace{}%
\AgdaSymbol{(}\AgdaBound{X}\AgdaSpace{}%
\AgdaSymbol{→}\AgdaSpace{}%
\AgdaBound{Y}\AgdaSymbol{)}\AgdaSpace{}%
\AgdaSymbol{→}\AgdaSpace{}%
\AgdaGeneralizable{𝓤}\AgdaSpace{}%
\AgdaOperator{\AgdaPrimitive{⊔}}\AgdaSpace{}%
\AgdaGeneralizable{𝓥}\AgdaSpace{}%
\AgdaOperator{\AgdaFunction{̇}}\<%
\\
\>[0]\AgdaFunction{is-embedding}\AgdaSpace{}%
\AgdaBound{f}\AgdaSpace{}%
\AgdaSymbol{=}\AgdaSpace{}%
\AgdaSymbol{(}\AgdaBound{y}\AgdaSpace{}%
\AgdaSymbol{:}\AgdaSpace{}%
\AgdaFunction{codomain}\AgdaSpace{}%
\AgdaBound{f}\AgdaSymbol{)}\AgdaSpace{}%
\AgdaSymbol{→}\AgdaSpace{}%
\AgdaFunction{is-subsingleton}\AgdaSpace{}%
\AgdaSymbol{(}\AgdaFunction{fiber}\AgdaSpace{}%
\AgdaBound{f}\AgdaSpace{}%
\AgdaBound{y}\AgdaSymbol{)}\<%
\end{code}
\ccpad
where
\ccpad
\begin{code}
\>[0]\AgdaFunction{is-subsingleton}\AgdaSpace{}%
\AgdaSymbol{:}\AgdaSpace{}%
\AgdaGeneralizable{𝓤}\AgdaSpace{}%
\AgdaOperator{\AgdaFunction{̇}}\AgdaSpace{}%
\AgdaSymbol{→}\AgdaSpace{}%
\AgdaGeneralizable{𝓤}\AgdaSpace{}%
\AgdaOperator{\AgdaFunction{̇}}\<%
\\
\>[0]\AgdaFunction{is-subsingleton}\AgdaSpace{}%
\AgdaBound{X}\AgdaSpace{}%
\AgdaSymbol{=}\AgdaSpace{}%
\AgdaSymbol{(}\AgdaBound{x}\AgdaSpace{}%
\AgdaBound{y}\AgdaSpace{}%
\AgdaSymbol{:}\AgdaSpace{}%
\AgdaBound{X}\AgdaSymbol{)}\AgdaSpace{}%
\AgdaSymbol{→}\AgdaSpace{}%
\AgdaBound{x}\AgdaSpace{}%
\AgdaOperator{\AgdaDatatype{≡}}\AgdaSpace{}%
\AgdaBound{y}\<%
\end{code}
\ccpad
and
\ccpad
\begin{code}
\>[0]\AgdaFunction{fiber}\AgdaSpace{}%
\AgdaSymbol{:}\AgdaSpace{}%
\AgdaSymbol{\{}\AgdaBound{X}\AgdaSpace{}%
\AgdaSymbol{:}\AgdaSpace{}%
\AgdaGeneralizable{𝓤}\AgdaSpace{}%
\AgdaOperator{\AgdaFunction{̇}}\AgdaSpace{}%
\AgdaSymbol{\}}\AgdaSpace{}%
\AgdaSymbol{\{}\AgdaBound{Y}\AgdaSpace{}%
\AgdaSymbol{:}\AgdaSpace{}%
\AgdaGeneralizable{𝓥}\AgdaSpace{}%
\AgdaOperator{\AgdaFunction{̇}}\AgdaSpace{}%
\AgdaSymbol{\}}\AgdaSpace{}%
\AgdaSymbol{(}\AgdaBound{f}\AgdaSpace{}%
\AgdaSymbol{:}\AgdaSpace{}%
\AgdaBound{X}\AgdaSpace{}%
\AgdaSymbol{→}\AgdaSpace{}%
\AgdaBound{Y}\AgdaSymbol{)}\AgdaSpace{}%
\AgdaSymbol{→}\AgdaSpace{}%
\AgdaBound{Y}\AgdaSpace{}%
\AgdaSymbol{→}\AgdaSpace{}%
\AgdaGeneralizable{𝓤}\AgdaSpace{}%
\AgdaOperator{\AgdaPrimitive{⊔}}\AgdaSpace{}%
\AgdaGeneralizable{𝓥}\AgdaSpace{}%
\AgdaOperator{\AgdaFunction{̇}}\<%
\\
\>[0]\AgdaFunction{fiber}\AgdaSpace{}%
\AgdaBound{f}\AgdaSpace{}%
\AgdaBound{y}\AgdaSpace{}%
\AgdaSymbol{=}\AgdaSpace{}%
\AgdaFunction{Σ}\AgdaSpace{}%
\AgdaBound{x}\AgdaSpace{}%
\AgdaFunction{꞉}\AgdaSpace{}%
\AgdaFunction{domain}\AgdaSpace{}%
\AgdaBound{f}\AgdaSpace{}%
\AgdaFunction{,}\AgdaSpace{}%
\AgdaBound{f}\AgdaSpace{}%
\AgdaBound{x}\AgdaSpace{}%
\AgdaOperator{\AgdaDatatype{≡}}\AgdaSpace{}%
\AgdaBound{y}\<%
\end{code}
\ccpad
This is a natural way to represent what we usually mean in mathematics by embedding. It does not correspond simply to an injective map. However, if we assume that the codomain type, \ab B, is a \emph{set} (i.e., has \emph{unique identity proofs}), then we can prove that a injective (i.e., \emph{monic}) function into \ab B is an embedding as follows:
\ccpad
\begin{code}%
\>[0]\AgdaKeyword{module}\AgdaSpace{}%
\AgdaModule{\AgdaUnderscore{}}\AgdaSpace{}%
\AgdaSymbol{\{}\AgdaBound{𝓤}\AgdaSpace{}%
\AgdaBound{𝓦}\AgdaSpace{}%
\AgdaSymbol{:}\AgdaSpace{}%
\AgdaPostulate{Universe}\AgdaSymbol{\}}\AgdaSpace{}%
\AgdaKeyword{where}\<%
\\
%
%% \>[1]\AgdaFunction{MonicInvIsLeftInv}%
%% \>[371I]\AgdaSymbol{:}%
%% \>[372I]\AgdaSymbol{\{}\AgdaBound{A}\AgdaSpace{}%
%% \AgdaSymbol{:}\AgdaSpace{}%
%% \AgdaBound{𝓤}\AgdaSpace{}%
%% \AgdaOperator{\AgdaFunction{̇}}\AgdaSpace{}%
%% \AgdaSymbol{\}\{}\AgdaBound{B}\AgdaSpace{}%
%% \AgdaSymbol{:}\AgdaSpace{}%
%% \AgdaBound{𝓦}\AgdaSpace{}%
%% \AgdaOperator{\AgdaFunction{̇}}\AgdaSpace{}%
%% \AgdaSymbol{\}}\<%
%% \\
%% \>[.][@{}l@{}]\<[372I]%
%% \>[21]\AgdaSymbol{(}\AgdaBound{f}\AgdaSpace{}%
%% \AgdaSymbol{:}\AgdaSpace{}%
%% \AgdaBound{A}\AgdaSpace{}%
%% \AgdaSymbol{→}\AgdaSpace{}%
%% \AgdaBound{B}\AgdaSymbol{)}\AgdaSpace{}%
%% \AgdaSymbol{(}\AgdaBound{fmonic}\AgdaSpace{}%
%% \AgdaSymbol{:}\AgdaSpace{}%
%% \AgdaFunction{Monic}\AgdaSpace{}%
%% \AgdaBound{f}\AgdaSymbol{)(}\AgdaBound{x}\AgdaSpace{}%
%% \AgdaSymbol{:}\AgdaSpace{}%
%% \AgdaBound{A}\AgdaSymbol{)}\<%
%% \\
%% \>[371I][@{}l@{\AgdaIndent{0}}]%
%% \>[20]\AgdaComment{----------------------------------------}\<%
%% \\
%% \>[1][@{}l@{\AgdaIndent{0}}]%
%% \>[3]\AgdaSymbol{→}%
%% \>[21]\AgdaSymbol{(}\AgdaFunction{MonicInv}\AgdaSpace{}%
%% \AgdaBound{f}\AgdaSpace{}%
%% \AgdaBound{fmonic}\AgdaSymbol{)}\AgdaSpace{}%
%% \AgdaSymbol{(}\AgdaBound{f}\AgdaSpace{}%
%% \AgdaBound{x}\AgdaSymbol{)}\AgdaSpace{}%
%% \AgdaSymbol{(}\AgdaInductiveConstructor{im}\AgdaSpace{}%
%% \AgdaBound{x}\AgdaSymbol{)}\AgdaSpace{}%
%% \AgdaOperator{\AgdaDatatype{≡}}\AgdaSpace{}%
%% \AgdaBound{x}\<%
%% \\
\\[\AgdaEmptyExtraSkip]%
\>[0][@{}l@{\AgdaIndent{0}}]%
\>[1]\AgdaFunction{monic-into-set-is-embedding}
\>[371I]\AgdaSymbol{:}%
\>[372I]\AgdaSymbol{\{}\AgdaBound{A}\AgdaSpace{}%
\AgdaSymbol{:}\AgdaSpace{}%
\AgdaSymbol{\{}\AgdaBound{A}\AgdaSpace{}%
\AgdaSymbol{:}\AgdaSpace{}%
\AgdaBound{𝓤}\AgdaSpace{}%
\AgdaOperator{\AgdaFunction{̇}}\AgdaSymbol{\}\{}\AgdaBound{B}\AgdaSpace{}%
\AgdaSymbol{:}\AgdaSpace{}%
\AgdaBound{𝓦}\AgdaSpace{}%
\AgdaOperator{\AgdaFunction{̇}}\AgdaSymbol{\}}\AgdaSpace{}%
\AgdaSymbol{→}\AgdaSpace{}%
\AgdaFunction{is-set}\AgdaSpace{}%
\AgdaBound{B}\<%
\\
\>[1][@{}l@{\AgdaIndent{0}}]%
\>[2]\AgdaSymbol{→}%
\>[.][@{}l@{}]\<[372I]%
\>[21]\AgdaSymbol{(}\AgdaBound{f}\AgdaSpace{}%
\AgdaSymbol{:}\AgdaSpace{}%
\AgdaBound{A}\AgdaSpace{}%
\AgdaSymbol{→}\AgdaSpace{}%
\AgdaBound{B}\AgdaSymbol{)}%
\>[44]\AgdaSymbol{→}%
\>[47]\AgdaFunction{Monic}\AgdaSpace{}%
\AgdaBound{f}\<%
\\
\>[2][@{}l@{\AgdaIndent{0}}]%
\>[21]\AgdaComment{---------------------------}\<%
\\
%
\>[2]\AgdaSymbol{→}%
\>[21]\AgdaFunction{is-embedding}\AgdaSpace{}%
\AgdaBound{f}\<%
\\
%
\\[\AgdaEmptyExtraSkip]%
%
\>[1]\AgdaFunction{monic-into-set-is-embedding}\AgdaSpace{}%
\AgdaSymbol{\{}\AgdaBound{A}\AgdaSymbol{\}}\AgdaSpace{}%
\AgdaBound{Bset}\AgdaSpace{}%
\AgdaBound{f}\AgdaSpace{}%
\AgdaBound{fmon}\AgdaSpace{}%
\AgdaBound{b}\AgdaSpace{}%
\AgdaSymbol{(}\AgdaBound{a}\AgdaSpace{}%
\AgdaOperator{\AgdaInductiveConstructor{,}}\AgdaSpace{}%
\AgdaBound{fa≡b}\AgdaSymbol{)}\AgdaSpace{}%
\AgdaSymbol{(}\AgdaBound{a'}\AgdaSpace{}%
\AgdaOperator{\AgdaInductiveConstructor{,}}\AgdaSpace{}%
\AgdaBound{fa'≡b}\AgdaSymbol{)}\AgdaSpace{}%
\AgdaSymbol{=}\AgdaSpace{}%
\AgdaFunction{γ}\<%
\\
\>[1][@{}l@{\AgdaIndent{0}}]%
\>[2]\AgdaKeyword{where}\<%
\\
\>[2][@{}l@{\AgdaIndent{0}}]%
\>[3]\AgdaFunction{faa'}\AgdaSpace{}%
\AgdaSymbol{:}\AgdaSpace{}%
\AgdaBound{f}\AgdaSpace{}%
\AgdaBound{a}\AgdaSpace{}%
\AgdaOperator{\AgdaDatatype{≡}}\AgdaSpace{}%
\AgdaBound{f}\AgdaSpace{}%
\AgdaBound{a'}\<%
\\
%
\>[3]\AgdaFunction{faa'}\AgdaSpace{}%
\AgdaSymbol{=}\AgdaSpace{}%
\AgdaFunction{≡-Trans}\AgdaSpace{}%
\AgdaSymbol{(}\AgdaBound{f}\AgdaSpace{}%
\AgdaBound{a}\AgdaSymbol{)}\AgdaSpace{}%
\AgdaSymbol{(}\AgdaBound{f}\AgdaSpace{}%
\AgdaBound{a'}\AgdaSymbol{)}\AgdaSpace{}%
\AgdaBound{fa≡b}\AgdaSpace{}%
\AgdaSymbol{(}\AgdaBound{fa'≡b}\AgdaSpace{}%
\AgdaOperator{\AgdaFunction{⁻¹}}\AgdaSymbol{)}\<%
\\
%
\\[\AgdaEmptyExtraSkip]%
%
\>[3]\AgdaFunction{aa'}\AgdaSpace{}%
\AgdaSymbol{:}\AgdaSpace{}%
\AgdaBound{a}\AgdaSpace{}%
\AgdaOperator{\AgdaDatatype{≡}}\AgdaSpace{}%
\AgdaBound{a'}\<%
\\
%
\>[3]\AgdaFunction{aa'}\AgdaSpace{}%
\AgdaSymbol{=}\AgdaSpace{}%
\AgdaBound{fmon}\AgdaSpace{}%
\AgdaBound{a}\AgdaSpace{}%
\AgdaBound{a'}\AgdaSpace{}%
\AgdaFunction{faa'}\<%
\\
%
\\[\AgdaEmptyExtraSkip]%
%
\>[3]\AgdaFunction{𝒜}\AgdaSpace{}%
\AgdaSymbol{:}\AgdaSpace{}%
\AgdaBound{A}\AgdaSpace{}%
\AgdaSymbol{→}\AgdaSpace{}%
\AgdaBound{𝓦}\AgdaSpace{}%
\AgdaOperator{\AgdaFunction{̇}}\<%
\\
%
\>[3]\AgdaFunction{𝒜}\AgdaSpace{}%
\AgdaBound{a}\AgdaSpace{}%
\AgdaSymbol{=}\AgdaSpace{}%
\AgdaBound{f}\AgdaSpace{}%
\AgdaBound{a}\AgdaSpace{}%
\AgdaOperator{\AgdaDatatype{≡}}\AgdaSpace{}%
\AgdaBound{b}\<%
\\
%
\\[\AgdaEmptyExtraSkip]%
%
\>[3]\AgdaFunction{arg1}\AgdaSpace{}%
\AgdaSymbol{:}\AgdaSpace{}%
\AgdaFunction{Σ}\AgdaSpace{}%
\AgdaBound{p}\AgdaSpace{}%
\AgdaFunction{꞉}\AgdaSpace{}%
\AgdaSymbol{(}\AgdaBound{a}\AgdaSpace{}%
\AgdaOperator{\AgdaDatatype{≡}}\AgdaSpace{}%
\AgdaBound{a'}\AgdaSymbol{)}\AgdaSpace{}%
\AgdaFunction{,}\AgdaSpace{}%
\AgdaSymbol{(}\AgdaFunction{transport}\AgdaSpace{}%
\AgdaFunction{𝒜}\AgdaSpace{}%
\AgdaBound{p}\AgdaSpace{}%
\AgdaBound{fa≡b}\AgdaSymbol{)}\AgdaSpace{}%
\AgdaOperator{\AgdaDatatype{≡}}\AgdaSpace{}%
\AgdaBound{fa'≡b}\<%
\\
%
\>[3]\AgdaFunction{arg1}\AgdaSpace{}%
\AgdaSymbol{=}\AgdaSpace{}%
\AgdaFunction{aa'}\AgdaSpace{}%
\AgdaOperator{\AgdaInductiveConstructor{,}}\AgdaSpace{}%
\AgdaBound{Bset}\AgdaSpace{}%
\AgdaSymbol{(}\AgdaBound{f}\AgdaSpace{}%
\AgdaBound{a'}\AgdaSymbol{)}\AgdaSpace{}%
\AgdaBound{b}\AgdaSpace{}%
\AgdaSymbol{(}\AgdaFunction{transport}\AgdaSpace{}%
\AgdaFunction{𝒜}\AgdaSpace{}%
\AgdaFunction{aa'}\AgdaSpace{}%
\AgdaBound{fa≡b}\AgdaSymbol{)}\AgdaSpace{}%
\AgdaBound{fa'≡b}\<%
\\
%
\\[\AgdaEmptyExtraSkip]%
%
\>[3]\AgdaFunction{γ}\AgdaSpace{}%
\AgdaSymbol{:}\AgdaSpace{}%
\AgdaBound{a}\AgdaSpace{}%
\AgdaOperator{\AgdaInductiveConstructor{,}}\AgdaSpace{}%
\AgdaBound{fa≡b}\AgdaSpace{}%
\AgdaOperator{\AgdaDatatype{≡}}\AgdaSpace{}%
\AgdaBound{a'}\AgdaSpace{}%
\AgdaOperator{\AgdaInductiveConstructor{,}}\AgdaSpace{}%
\AgdaBound{fa'≡b}\<%
\\
%
\>[3]\AgdaFunction{γ}\AgdaSpace{}%
\AgdaSymbol{=}\AgdaSpace{}%
\AgdaFunction{to-Σ-≡}\AgdaSpace{}%
\AgdaFunction{arg1}\<%
\\
\>[0]\<%
\end{code}

Of course, invertible maps are embeddings.

\begin{code}%
\>[0]\<%
\\
\>[0][@{}l@{\AgdaIndent{1}}]%
\>[1]\AgdaFunction{invertibles-are-embeddings}\AgdaSpace{}%
\>[371I]\AgdaSymbol{:}%
\>[372I]\AgdaSymbol{\{}\AgdaBound{X}\AgdaSpace{}%
\AgdaSymbol{:}\AgdaSpace{}%
\AgdaBound{𝓤}\AgdaSpace{}%
\AgdaOperator{\AgdaFunction{̇}}\AgdaSpace{}%
\AgdaSymbol{\}}\AgdaSpace{}%
\AgdaSymbol{\{}\AgdaBound{Y}\AgdaSpace{}%
\AgdaSymbol{:}\AgdaSpace{}%
\AgdaBound{𝓦}\AgdaSpace{}%
\AgdaOperator{\AgdaFunction{̇}}\AgdaSpace{}%
\AgdaSymbol{\}(}\AgdaBound{f}\AgdaSpace{}%
\AgdaSymbol{:}\AgdaSpace{}%
\AgdaBound{X}\AgdaSpace{}%
\AgdaSymbol{→}\AgdaSpace{}%
\AgdaBound{Y}\AgdaSymbol{)}\<%
\\
\>[1][@{}l@{\AgdaIndent{0}}]%
\>[2]\AgdaSymbol{→}%
\>[.][@{}l@{}]\<[372I]%
\>[21]\AgdaFunction{invertible}\AgdaSpace{}%
\AgdaBound{f}\AgdaSpace{}%
\AgdaSymbol{→}\AgdaSpace{}%
\AgdaFunction{is-embedding}\AgdaSpace{}%
\AgdaBound{f}\<%
\\
%
\>[1]\AgdaFunction{invertibles-are-embeddings}\AgdaSpace{}%
\AgdaBound{f}\AgdaSpace{}%
\AgdaBound{fi}\AgdaSpace{}%
\AgdaSymbol{=}\AgdaSpace{}%
\AgdaFunction{equivs-are-embeddings}\AgdaSpace{}%
\AgdaBound{f}\AgdaSpace{}%
\AgdaSymbol{(}\AgdaFunction{invertibles-are-equivs}\AgdaSpace{}%
\AgdaBound{f}\AgdaSpace{}%
\AgdaBound{fi}\AgdaSymbol{)}\<%
\\
\>[0]\<%
\end{code}

Finally, if we have a proof \texttt{p\ :\ is-embedding\ f} that the map \texttt{f} is an embedding, here's a tool that makes it easier to apply \texttt{p}.

\begin{code}%
\>[0]\<%
\\
\>[0][@{}l@{\AgdaIndent{1}}]%
\>[1]\AgdaComment{-- Embedding elimination (makes it easier to apply is-embedding)}\<%
\\
%
\>[1]\AgdaFunction{embedding-elim}\AgdaSpace{}%
\>[371I]\AgdaSymbol{:}%
\>[372I]\AgdaSymbol{\{}\AgdaBound{X}\AgdaSpace{}%
\AgdaSymbol{:}\AgdaSpace{}%
\AgdaBound{𝓤}\AgdaSpace{}%
\AgdaOperator{\AgdaFunction{̇}}\AgdaSpace{}%
\AgdaSymbol{\}}\AgdaSpace{}%
\AgdaSymbol{\{}\AgdaBound{Y}\AgdaSpace{}%
\AgdaSymbol{:}\AgdaSpace{}%
\AgdaBound{𝓦}\AgdaSpace{}%
\AgdaOperator{\AgdaFunction{̇}}\AgdaSpace{}%
\AgdaSymbol{\}\{}\AgdaBound{f}\AgdaSpace{}%
\AgdaSymbol{:}\AgdaSpace{}%
\AgdaBound{X}\AgdaSpace{}%
\AgdaSymbol{→}\AgdaSpace{}%
\AgdaBound{Y}\AgdaSymbol{\}}\<%
\\
\>[1][@{}l@{\AgdaIndent{0}}]%
\>[2]\AgdaSymbol{→}%
\>[.][@{}l@{}]\<[372I]%
\>[18]\AgdaFunction{is-embedding}\AgdaSpace{}%
\AgdaSymbol{→}\AgdaSpace{}%
\AgdaSymbol{(}\AgdaBound{x}\AgdaSpace{}%
\AgdaBound{x'}\AgdaSpace{}%
\AgdaSymbol{:}\AgdaSpace{}%
\AgdaBound{X}\AgdaSymbol{)}\<%
\\
%
\>[18]\AgdaComment{----------------------}\<%
\\
%
\>[2]\AgdaSymbol{→}%
\>[18]\AgdaBound{f}\AgdaSpace{}%
\AgdaBound{x}\AgdaSpace{}%
\AgdaOperator{\AgdaDatatype{≡}}\AgdaSpace{}%
\AgdaBound{f}\AgdaSpace{}%
\AgdaBound{x'}\AgdaSpace{}%
\AgdaSymbol{→}\AgdaSpace{}%
\AgdaBound{x}\AgdaSpace{}%
\AgdaOperator{\AgdaDatatype{≡}}\AgdaSpace{}%
\AgdaBound{x'}\<%
\\
%
\>[1]\AgdaFunction{embedding-elim}\AgdaSpace{}%
\AgdaSymbol{\{}\AgdaArgument{f}\AgdaSpace{}%
\AgdaSymbol{=}\AgdaSpace{}%
\AgdaBound{f}\AgdaSymbol{\}}\AgdaSpace{}%
\AgdaBound{femb}\AgdaSpace{}%
\AgdaBound{x}\AgdaSpace{}%
\AgdaBound{x'}\AgdaSpace{}%
\AgdaBound{fxfx'}\AgdaSpace{}%
\AgdaSymbol{=}\AgdaSpace{}%
\AgdaFunction{γ}\<%
\\
\>[1][@{}l@{\AgdaIndent{0}}]%
\>[2]\AgdaKeyword{where}\<%
\\
\>[2][@{}l@{\AgdaIndent{0}}]%
\>[3]\AgdaFunction{fibx}\AgdaSpace{}%
\AgdaSymbol{:}\AgdaSpace{}%
\AgdaFunction{fiber}\AgdaSpace{}%
\AgdaBound{f}\AgdaSpace{}%
\AgdaSymbol{(}\AgdaBound{f}\AgdaSpace{}%
\AgdaBound{x}\AgdaSymbol{)}\<%
\\
%
\>[3]\AgdaFunction{fibx}\AgdaSpace{}%
\AgdaSymbol{=}\AgdaSpace{}%
\AgdaBound{x}\AgdaSpace{}%
\AgdaOperator{\AgdaInductiveConstructor{,}}\AgdaSpace{}%
\AgdaInductiveConstructor{𝓇ℯ𝒻𝓁}\<%
\\
%
\>[3]\AgdaFunction{fibx'}\AgdaSpace{}%
\AgdaSymbol{:}\AgdaSpace{}%
\AgdaFunction{fiber}\AgdaSpace{}%
\AgdaBound{f}\AgdaSpace{}%
\AgdaSymbol{(}\AgdaBound{f}\AgdaSpace{}%
\AgdaBound{x}\AgdaSymbol{)}\<%
\\
%
\>[3]\AgdaFunction{fibx'}\AgdaSpace{}%
\AgdaSymbol{=}\AgdaSpace{}%
\AgdaBound{x'}\AgdaSpace{}%
\AgdaOperator{\AgdaInductiveConstructor{,}}\AgdaSpace{}%
\AgdaSymbol{((}\AgdaBound{fxfx'}\AgdaSymbol{)}\AgdaSpace{}%
\AgdaOperator{\AgdaFunction{⁻¹}}\AgdaSymbol{)}\<%
\\
%
\>[3]\AgdaFunction{iss-fibffx}\AgdaSpace{}%
\AgdaSymbol{:}\AgdaSpace{}%
\AgdaFunction{is-subsingleton}\AgdaSpace{}%
\AgdaSymbol{(}\AgdaFunction{fiber}\AgdaSpace{}%
\AgdaBound{f}\AgdaSpace{}%
\AgdaSymbol{(}\AgdaBound{f}\AgdaSpace{}%
\AgdaBound{x}\AgdaSymbol{))}\<%
\\
%
\>[3]\AgdaFunction{iss-fibffx}\AgdaSpace{}%
\AgdaSymbol{=}\AgdaSpace{}%
\AgdaBound{femb}\AgdaSpace{}%
\AgdaSymbol{(}\AgdaBound{f}\AgdaSpace{}%
\AgdaBound{x}\AgdaSymbol{)}\<%
\\
%
\>[3]\AgdaFunction{fibxfibx'}\AgdaSpace{}%
\AgdaSymbol{:}\AgdaSpace{}%
\AgdaFunction{fibx}\AgdaSpace{}%
\AgdaOperator{\AgdaDatatype{≡}}\AgdaSpace{}%
\AgdaFunction{fibx'}\<%
\\
%
\>[3]\AgdaFunction{fibxfibx'}\AgdaSpace{}%
\AgdaSymbol{=}\AgdaSpace{}%
\AgdaFunction{iss-fibffx}\AgdaSpace{}%
\AgdaFunction{fibx}\AgdaSpace{}%
\AgdaFunction{fibx'}\<%
\\
%
\>[3]\AgdaFunction{γ}\AgdaSpace{}%
\AgdaSymbol{:}\AgdaSpace{}%
\AgdaBound{x}\AgdaSpace{}%
\AgdaOperator{\AgdaDatatype{≡}}\AgdaSpace{}%
\AgdaBound{x'}\<%
\\
%
\>[3]\AgdaFunction{γ}\AgdaSpace{}%
\AgdaSymbol{=}\AgdaSpace{}%
\AgdaFunction{ap}\AgdaSpace{}%
\AgdaFunction{pr₁}\AgdaSpace{}%
\AgdaFunction{fibxfibx'}\<%
\end{code}
----------------------------------------------------
%%
In Subsection~\ref{sssec:inverse-image-type}, we define an inductive datatype that represents our semantic notion of the \textit{inverse image} of a function. In Subsections~\ref{sssec:epic-function-type} and~\ref{sssec:monic-function-type} we define types for \textit{epic} and \textit{monic} functions. Finally, in Subsections~\ref{sssec:monics-are-set-embeddings}, we examine the type of \textit{embeddings} (as defined by Escardo in~\cite{MHE}), and determine how this type relates to our definition of injective function type.

\subsubsection{Inverse image type}\label{sssec:inverse-image-type}

\begin{code}%
\>[0][@{}l@{\AgdaIndent{0}}]%
\>[1]\AgdaKeyword{data}\AgdaSpace{}%
\AgdaOperator{\AgdaDatatype{Image\AgdaUnderscore{}∋\AgdaUnderscore{}}}\AgdaSpace{}%
\AgdaSymbol{\{}\AgdaBound{A}\AgdaSpace{}%
\AgdaSymbol{:}\AgdaSpace{}%
\AgdaBound{𝓤}\AgdaSpace{}%
\AgdaOperator{\AgdaFunction{̇}}\AgdaSpace{}%
\AgdaSymbol{\}\{}\AgdaBound{B}\AgdaSpace{}%
\AgdaSymbol{:}\AgdaSpace{}%
\AgdaBound{𝓦}\AgdaSpace{}%
\AgdaOperator{\AgdaFunction{̇}}\AgdaSpace{}%
\AgdaSymbol{\}(}\AgdaBound{f}\AgdaSpace{}%
\AgdaSymbol{:}\AgdaSpace{}%
\AgdaBound{A}\AgdaSpace{}%
\AgdaSymbol{→}\AgdaSpace{}%
\AgdaBound{B}\AgdaSymbol{)}\AgdaSpace{}%
\AgdaSymbol{:}\AgdaSpace{}%
\AgdaBound{B}\AgdaSpace{}%
\AgdaSymbol{→}\AgdaSpace{}%
\AgdaBound{𝓤}\AgdaSpace{}%
\AgdaOperator{\AgdaPrimitive{⊔}}\AgdaSpace{}%
\AgdaBound{𝓦}\AgdaSpace{}%
\AgdaOperator{\AgdaFunction{̇}}\<%
\\
\>[1][@{}l@{\AgdaIndent{0}}]%
\>[2]\AgdaKeyword{where}\<%
\\
%
\>[2]\AgdaInductiveConstructor{im}\AgdaSpace{}%
\AgdaSymbol{:}\AgdaSpace{}%
\AgdaSymbol{(}\AgdaBound{x}\AgdaSpace{}%
\AgdaSymbol{:}\AgdaSpace{}%
\AgdaBound{A}\AgdaSymbol{)}\AgdaSpace{}%
\AgdaSymbol{→}\AgdaSpace{}%
\AgdaOperator{\AgdaDatatype{Image}}\AgdaSpace{}%
\AgdaBound{f}\AgdaSpace{}%
\AgdaOperator{\AgdaDatatype{∋}}\AgdaSpace{}%
\AgdaBound{f}\AgdaSpace{}%
\AgdaBound{x}\<%
\\
%
\>[2]\AgdaInductiveConstructor{eq}\AgdaSpace{}%
\AgdaSymbol{:}\AgdaSpace{}%
\AgdaSymbol{(}\AgdaBound{b}\AgdaSpace{}%
\AgdaSymbol{:}\AgdaSpace{}%
\AgdaBound{B}\AgdaSymbol{)}\AgdaSpace{}%
\AgdaSymbol{→}\AgdaSpace{}%
\AgdaSymbol{(}\AgdaBound{a}\AgdaSpace{}%
\AgdaSymbol{:}\AgdaSpace{}%
\AgdaBound{A}\AgdaSymbol{)}\AgdaSpace{}%
\AgdaSymbol{→}\AgdaSpace{}%
\AgdaBound{b}\AgdaSpace{}%
\AgdaOperator{\AgdaDatatype{≡}}\AgdaSpace{}%
\AgdaBound{f}\AgdaSpace{}%
\AgdaBound{a}\AgdaSpace{}%
\AgdaSymbol{→}\AgdaSpace{}%
\AgdaOperator{\AgdaDatatype{Image}}\AgdaSpace{}%
\AgdaBound{f}\AgdaSpace{}%
\AgdaOperator{\AgdaDatatype{∋}}\AgdaSpace{}%
\AgdaBound{b}\<%
\end{code}
\ccpad
Note that an inhabitant of \af{Image} \ab f \ad ∋ \ab b is a dependent pair (\ab a \ac \ab p), where \ab a \as : \ab A and \ab p \as : \ab b  \ad ≡ \ab f \ab a is a proof that \ab f maps \ab a to \ab b. Thus, a proof that \ab b belongs to the image of \ab f (i.e., an inhabitant of \af{Image} \ab f \ad ∋ \ab b), is always accompanied by a witness \ab a \as : \ab A, and a proof that \ab b \ad{≡} \ab f \ab a, so the inverse of a function \ab f can actually be \emph{computed} at every inhabitant of the image of \ab f.

We define an inverse function, which we call \af{Inv}, which, when given \ab b \as : \ab B and a proof (\ab a \ac \ab p) \as : \af{Image} \ab f \ad ∋ \ab b that \ab b belongs to the image of \ab f, produces \ab a (a preimage of \ab b under \ab f).
\ccpad
\begin{code}%
\>[0][@{}l@{\AgdaIndent{1}}]%
\>[1]\AgdaFunction{Inv}\AgdaSpace{}%
\AgdaSymbol{:}\AgdaSpace{}%
\AgdaSymbol{\{}\AgdaBound{A}\AgdaSpace{}%
\AgdaSymbol{:}\AgdaSpace{}%
\AgdaBound{𝓤}\AgdaSpace{}%
\AgdaOperator{\AgdaFunction{̇}}\AgdaSpace{}%
\AgdaSymbol{\}\{}\AgdaBound{B}\AgdaSpace{}%
\AgdaSymbol{:}\AgdaSpace{}%
\AgdaBound{𝓦}\AgdaSpace{}%
\AgdaOperator{\AgdaFunction{̇}}\AgdaSpace{}%
\AgdaSymbol{\}(}\AgdaBound{f}\AgdaSpace{}%
\AgdaSymbol{:}\AgdaSpace{}%
\AgdaBound{A}\AgdaSpace{}%
\AgdaSymbol{→}\AgdaSpace{}%
\AgdaBound{B}\AgdaSymbol{)(}\AgdaBound{b}\AgdaSpace{}%
\AgdaSymbol{:}\AgdaSpace{}%
\AgdaBound{B}\AgdaSymbol{)}\AgdaSpace{}%
\AgdaSymbol{→}\AgdaSpace{}%
\AgdaOperator{\AgdaDatatype{Image}}\AgdaSpace{}%
\AgdaBound{f}\AgdaSpace{}%
\AgdaOperator{\AgdaDatatype{∋}}\AgdaSpace{}%
\AgdaBound{b}%
\>[61]\AgdaSymbol{→}%
\>[64]\AgdaBound{A}\<%
\\
%
\>[1]\AgdaFunction{Inv}\AgdaSpace{}%
\AgdaBound{f}\AgdaSpace{}%
\AgdaDottedPattern{\AgdaSymbol{.(}}\AgdaDottedPattern{\AgdaBound{f}}\AgdaSpace{}%
\AgdaDottedPattern{\AgdaBound{a}}\AgdaDottedPattern{\AgdaSymbol{)}}\AgdaSpace{}%
\AgdaSymbol{(}\AgdaInductiveConstructor{im}\AgdaSpace{}%
\AgdaBound{a}\AgdaSymbol{)}\AgdaSpace{}%
\AgdaSymbol{=}\AgdaSpace{}%
\AgdaBound{a}\<%
\\
%
\>[1]\AgdaFunction{Inv}\AgdaSpace{}%
\AgdaBound{f}\AgdaSpace{}%
\AgdaUnderscore\AgdaSpace{}%
\AgdaSymbol{(}\AgdaInductiveConstructor{eq}\AgdaSpace{}%
\AgdaUnderscore\AgdaSpace{}%
\AgdaBound{a}\AgdaSpace{}%
\AgdaUnderscore\AgdaSymbol{)}\AgdaSpace{}%
\AgdaSymbol{=}\AgdaSpace{}%
\AgdaBound{a}\<%
\end{code}
\ccpad
Thus, the inverse is computed by pattern matching on the structure of the third explicit argument, which has (inductive) type \AgdaOperator{\AgdaDatatype{Image}}\AgdaSpace{}\AgdaBound{f}\AgdaSpace{}\AgdaOperator{\AgdaDatatype{∋}}\AgdaSpace{}\AgdaBound{b}.  Since there are two constructors, \AgdaInductiveConstructor{im} and \AgdaInductiveConstructor{eq}, that argument must take one of two forms. Either it has the form \AgdaInductiveConstructor{im}\AgdaSpace{}\AgdaBound{a} (in which case the second explicit argument is \as{.(}\ab f \ab a \as )),\footnote{The dotted pattern is used when the form of the argument is forced... todo: fix this sentence} or it has the form \AgdaInductiveConstructor{eq} \ab b \ab a \ab p, where \ab p is a proof of \ab b \aof ≡ \ab f \ab a.  (The underscore characters replace \ab b and \ab p in the definition since \af{Inv} doesn't care about them; it only needs to extract and return the preimage \ab a.)

We can formally prove that \af{Inv} \ab f is the right-inverse of \ab f, as follows.  Again, we use pattern matching and structural induction.
\ccpad
\begin{code}%
\>[0][@{}l@{\AgdaIndent{1}}]%
\>[1]\AgdaFunction{InvIsInv}%
\>[156I]\AgdaSymbol{:}%
\>[157I]\AgdaSymbol{\{}\AgdaBound{A}\AgdaSpace{}%
\AgdaSymbol{:}\AgdaSpace{}%
\AgdaBound{𝓤}\AgdaSpace{}%
\AgdaOperator{\AgdaFunction{̇}}\AgdaSpace{}%
\AgdaSymbol{\}}\AgdaSpace{}%
\AgdaSymbol{\{}\AgdaBound{B}\AgdaSpace{}%
\AgdaSymbol{:}\AgdaSpace{}%
\AgdaBound{𝓦}\AgdaSpace{}%
\AgdaOperator{\AgdaFunction{̇}}\AgdaSpace{}%
\AgdaSymbol{\}}\AgdaSpace{}%
\AgdaSymbol{(}\AgdaBound{f}\AgdaSpace{}%
\AgdaSymbol{:}\AgdaSpace{}%
\AgdaBound{A}\AgdaSpace{}%
\AgdaSymbol{→}\AgdaSpace{}%
\AgdaBound{B}\AgdaSymbol{)}\<%
\\
\>[.][@{}l@{}]\<[157I]%
\>[12]\AgdaSymbol{(}\AgdaBound{b}\AgdaSpace{}%
\AgdaSymbol{:}\AgdaSpace{}%
\AgdaBound{B}\AgdaSymbol{)}\AgdaSpace{}%
\AgdaSymbol{(}\AgdaBound{b∈Imgf}\AgdaSpace{}%
\AgdaSymbol{:}\AgdaSpace{}%
\AgdaOperator{\AgdaDatatype{Image}}\AgdaSpace{}%
\AgdaBound{f}\AgdaSpace{}%
\AgdaOperator{\AgdaDatatype{∋}}\AgdaSpace{}%
\AgdaBound{b}\AgdaSymbol{)}\<%
\\
\>[156I][@{}l@{\AgdaIndent{0}}]%
\>[11]\AgdaComment{----------------------}\<%
\\
\>[1][@{}l@{\AgdaIndent{0}}]%
\>[2]\AgdaSymbol{→}%
\>[12]\AgdaBound{f}\AgdaSpace{}%
\AgdaSymbol{(}\AgdaFunction{Inv}\AgdaSpace{}%
\AgdaBound{f}\AgdaSpace{}%
\AgdaBound{b}\AgdaSpace{}%
\AgdaBound{b∈Imgf}\AgdaSymbol{)}\AgdaSpace{}%
\AgdaOperator{\AgdaDatatype{≡}}\AgdaSpace{}%
\AgdaBound{b}\<%
\\
%
\>[1]\AgdaFunction{InvIsInv}\AgdaSpace{}%
\AgdaBound{f}\AgdaSpace{}%
\AgdaDottedPattern{\AgdaSymbol{.(}}\AgdaDottedPattern{\AgdaBound{f}}\AgdaSpace{}%
\AgdaDottedPattern{\AgdaBound{a}}\AgdaDottedPattern{\AgdaSymbol{)}}\AgdaSpace{}%
\AgdaSymbol{(}\AgdaInductiveConstructor{im}\AgdaSpace{}%
\AgdaBound{a}\AgdaSymbol{)}\AgdaSpace{}%
\AgdaSymbol{=}\AgdaSpace{}%
\AgdaInductiveConstructor{refl}\AgdaSpace{}%
\AgdaSymbol{\AgdaUnderscore{}}\<%
\\
%
\>[1]\AgdaFunction{InvIsInv}\AgdaSpace{}%
\AgdaBound{f}\AgdaSpace{}%
\AgdaBound{b}\AgdaSpace{}%
\AgdaSymbol{(}\AgdaInductiveConstructor{eq}\AgdaSpace{}%
\AgdaBound{b}\AgdaSpace{}%
\AgdaBound{a}\AgdaSpace{}%
\AgdaBound{b≡fa}\AgdaSymbol{)}\AgdaSpace{}%
\AgdaSymbol{=}\AgdaSpace{}%
\AgdaBound{b≡fa}\AgdaSpace{}%
\AgdaOperator{\AgdaFunction{⁻¹}}\<%
\end{code}
\ccpad
Here we give names to all the arguments for readability, but most of them could be replaced with underscores.

\subsubsection{Epic functions}\label{sssec:epic-function-type}
Given universes \ab 𝓤, \ab 𝓦, types \AgdaTyped{A}{𝓤 ̇} and \AgdaTyped{B}{𝓦 ̇}, and \ab f \as : \ab A \as → \ab B, we say that \ab f is an \emph{epic} (or \emph{surjective}) \emph{function from} \ab A \emph{to} \ab B provided we can produce an element (or proof or witness) of type \af{Epic} \ab f, where
\ccpad
\begin{code}%
\>[0][@{}l@{\AgdaIndent{1}}]%
\>[1]\AgdaFunction{Epic}\AgdaSpace{}%
\AgdaSymbol{:}\AgdaSpace{}%
\AgdaSymbol{\{}\AgdaBound{A}\AgdaSpace{}%
\AgdaSymbol{:}\AgdaSpace{}%
\AgdaBound{𝓤}\AgdaSpace{}%
\AgdaOperator{\AgdaFunction{̇}}\AgdaSpace{}%
\AgdaSymbol{\}}\AgdaSpace{}%
\AgdaSymbol{\{}\AgdaBound{B}\AgdaSpace{}%
\AgdaSymbol{:}\AgdaSpace{}%
\AgdaBound{𝓦}\AgdaSpace{}%
\AgdaOperator{\AgdaFunction{̇}}\AgdaSpace{}%
\AgdaSymbol{\}}\AgdaSpace{}%
\AgdaSymbol{(}\AgdaBound{f}\AgdaSpace{}%
\AgdaSymbol{:}\AgdaSpace{}%
\AgdaBound{A}\AgdaSpace{}%
\AgdaSymbol{→}\AgdaSpace{}%
\AgdaBound{B}\AgdaSymbol{)}\AgdaSpace{}%
\AgdaSymbol{→}%
\>[45]\AgdaBound{𝓤}\AgdaSpace{}%
\AgdaOperator{\AgdaPrimitive{⊔}}\AgdaSpace{}%
\AgdaBound{𝓦}\AgdaSpace{}%
\AgdaOperator{\AgdaFunction{̇}}\<%
\\
%
\>[1]\AgdaFunction{Epic}\AgdaSpace{}%
\AgdaBound{f}\AgdaSpace{}%
\AgdaSymbol{=}\AgdaSpace{}%
\AgdaSymbol{∀}\AgdaSpace{}%
\AgdaBound{y}\AgdaSpace{}%
\AgdaSymbol{→}\AgdaSpace{}%
\AgdaOperator{\AgdaDatatype{Image}}\AgdaSpace{}%
\AgdaBound{f}\AgdaSpace{}%
\AgdaOperator{\AgdaDatatype{∋}}\AgdaSpace{}%
\AgdaBound{y}\<%
\end{code}
\ccpad
We obtain the (right-) inverse of an epic function \ab f by applying the following function %% \af{EpicInv} defined below 
to \ab f and a proof %% , \ab{p} \as : \af{Epic} \ab f, 
that \ab f is epic.
\ccpad
\begin{code}%
\>[0][@{}l@{\AgdaIndent{1}}]%
\>[1]\AgdaFunction{EpicInv}\AgdaSpace{}%
\AgdaSymbol{:}%
\>[233I]\AgdaSymbol{\{}\AgdaBound{A}\AgdaSpace{}%
\AgdaSymbol{:}\AgdaSpace{}%
\AgdaBound{𝓤}\AgdaSpace{}%
\AgdaOperator{\AgdaFunction{̇}}\AgdaSpace{}%
\AgdaSymbol{\}}\AgdaSpace{}%
\AgdaSymbol{\{}\AgdaBound{B}\AgdaSpace{}%
\AgdaSymbol{:}\AgdaSpace{}%
\AgdaBound{𝓦}\AgdaSpace{}%
\AgdaOperator{\AgdaFunction{̇}}\AgdaSpace{}%
\AgdaSymbol{\}}\<%
\\
\>[.][@{}l@{}]\<[233I]%
\>[11]\AgdaSymbol{(}\AgdaBound{f}\AgdaSpace{}%
\AgdaSymbol{:}\AgdaSpace{}%
\AgdaBound{A}\AgdaSpace{}%
\AgdaSymbol{→}\AgdaSpace{}%
\AgdaBound{B}\AgdaSymbol{)}\AgdaSpace{}%
\AgdaSymbol{→}\AgdaSpace{}%
\AgdaFunction{Epic}\AgdaSpace{}%
\AgdaBound{f}\<%
\\
%
\>[11]\AgdaComment{--------------------}\<%
\\
\>[1][@{}l@{\AgdaIndent{0}}]%
\>[2]\AgdaSymbol{→}%
\>[11]\AgdaBound{B}\AgdaSpace{}%
\AgdaSymbol{→}\AgdaSpace{}%
\AgdaBound{A}\<%
\\
%
\>[1]\AgdaFunction{EpicInv}\AgdaSpace{}%
\AgdaBound{f}\AgdaSpace{}%
\AgdaBound{p}\AgdaSpace{}%
\AgdaBound{b}\AgdaSpace{}%
\AgdaSymbol{=}\AgdaSpace{}%
\AgdaFunction{Inv}\AgdaSpace{}%
\AgdaBound{f}\AgdaSpace{}%
\AgdaBound{b}\AgdaSpace{}%
\AgdaSymbol{(}\AgdaBound{p}\AgdaSpace{}%
\AgdaBound{b}\AgdaSymbol{)}\<%
\end{code}
\ccpad
The function defined by \af{EpicInv} \ab f \ab{p} is indeed the right-inverse of \ab f, as we now prove.
%% Here is how we formalize that assertion and its proof.
\ccpad
\begin{code}%
\>[0][@{}l@{\AgdaIndent{1}}]%
\>[1]\AgdaFunction{EpicInvIsRightInv}\AgdaSpace{}%
\AgdaSymbol{:}%
\>[262I]\AgdaFunction{funext}\AgdaSpace{}%
\AgdaBound{𝓦}\AgdaSpace{}%
\AgdaBound{𝓦}\AgdaSpace{}%
\AgdaSymbol{→}\AgdaSpace{}%
\AgdaSymbol{\{}\AgdaBound{A}\AgdaSpace{}%
\AgdaSymbol{:}\AgdaSpace{}%
\AgdaBound{𝓤}\AgdaSpace{}%
\AgdaOperator{\AgdaFunction{̇}}\AgdaSpace{}%
\AgdaSymbol{\}}\AgdaSpace{}%
\AgdaSymbol{\{}\AgdaBound{B}\AgdaSpace{}%
\AgdaSymbol{:}\AgdaSpace{}%
\AgdaBound{𝓦}\AgdaSpace{}%
\AgdaOperator{\AgdaFunction{̇}}\AgdaSpace{}%
\AgdaSymbol{\}}\<%
\\
\>[.][@{}l@{}]\<[262I]%
\>[21]\AgdaSymbol{(}\AgdaBound{f}\AgdaSpace{}%
\AgdaSymbol{:}\AgdaSpace{}%
\AgdaBound{A}\AgdaSpace{}%
\AgdaSymbol{→}\AgdaSpace{}%
\AgdaBound{B}\AgdaSymbol{)}%
\>[34]\AgdaSymbol{(}\AgdaBound{fE}\AgdaSpace{}%
\AgdaSymbol{:}\AgdaSpace{}%
\AgdaFunction{Epic}\AgdaSpace{}%
\AgdaBound{f}\AgdaSymbol{)}\<%
\\
%
\>[21]\AgdaComment{--------------------}\<%
\\
\>[1][@{}l@{\AgdaIndent{0}}]%
\>[2]\AgdaSymbol{→}%
\>[21]\AgdaBound{f}\AgdaSpace{}%
\AgdaOperator{\AgdaFunction{∘}}\AgdaSpace{}%
\AgdaSymbol{(}\AgdaFunction{EpicInv}\AgdaSpace{}%
\AgdaBound{f}\AgdaSpace{}%
\AgdaBound{fE}\AgdaSymbol{)}\AgdaSpace{}%
\AgdaOperator{\AgdaDatatype{≡}}\AgdaSpace{}%
\AgdaFunction{𝑖𝑑}\AgdaSpace{}%
\AgdaBound{B}\<%
\\
%
\>[1]\AgdaFunction{EpicInvIsRightInv}\AgdaSpace{}%
\AgdaBound{fe}\AgdaSpace{}%
\AgdaBound{f}\AgdaSpace{}%
\AgdaBound{fE}\AgdaSpace{}%
\AgdaSymbol{=}\AgdaSpace{}%
\AgdaBound{fe}\AgdaSpace{}%
\AgdaSymbol{(λ}\AgdaSpace{}%
\AgdaBound{x}\AgdaSpace{}%
\AgdaSymbol{→}\AgdaSpace{}%
\AgdaFunction{InvIsInv}\AgdaSpace{}%
\AgdaBound{f}\AgdaSpace{}%
\AgdaBound{x}\AgdaSpace{}%
\AgdaSymbol{(}\AgdaBound{fE}\AgdaSpace{}%
\AgdaBound{x}\AgdaSymbol{))}\<%
\end{code}

\subsubsection{Monic function type}\label{sssec:monic-function-type}
Given universes \ab 𝓤, \ab 𝓦, types \AgdaTyped{A}{𝓤 ̇} and \AgdaTyped{B}{𝓦 ̇}, and \ab f \as : \ab A \as → \ab B, we say that \ab f is a \textit{monic} (or \textit{injective}) \textit{function from} \ab A \textit{to} \ab B if we have a proof of \af{Monic} \ab f, where
\ccpad
\begin{code}%
\>[0][@{}l@{\AgdaIndent{1}}]%
\>[1]\AgdaFunction{Monic}\AgdaSpace{}%
\AgdaSymbol{:}\AgdaSpace{}%
\AgdaSymbol{\{}\AgdaBound{A}\AgdaSpace{}%
\AgdaSymbol{:}\AgdaSpace{}%
\AgdaBound{𝓤}\AgdaSpace{}%
\AgdaOperator{\AgdaFunction{̇}}\AgdaSpace{}%
\AgdaSymbol{\}}\AgdaSpace{}%
\AgdaSymbol{\{}\AgdaBound{B}\AgdaSpace{}%
\AgdaSymbol{:}\AgdaSpace{}%
\AgdaBound{𝓦}\AgdaSpace{}%
\AgdaOperator{\AgdaFunction{̇}}\AgdaSpace{}%
\AgdaSymbol{\}(}\AgdaBound{f}\AgdaSpace{}%
\AgdaSymbol{:}\AgdaSpace{}%
\AgdaBound{A}\AgdaSpace{}%
\AgdaSymbol{→}\AgdaSpace{}%
\AgdaBound{B}\AgdaSymbol{)}\AgdaSpace{}%
\AgdaSymbol{→}\AgdaSpace{}%
\AgdaBound{𝓤}\AgdaSpace{}%
\AgdaOperator{\AgdaPrimitive{⊔}}\AgdaSpace{}%
\AgdaBound{𝓦}\AgdaSpace{}%
\AgdaOperator{\AgdaFunction{̇}}\<%
\\
%
\>[1]\AgdaFunction{Monic}\AgdaSpace{}%
\AgdaBound{f}\AgdaSpace{}%
\AgdaSymbol{=}\AgdaSpace{}%
\AgdaSymbol{∀}\AgdaSpace{}%
\AgdaBound{a₁}\AgdaSpace{}%
\AgdaBound{a₂}\AgdaSpace{}%
\AgdaSymbol{→}\AgdaSpace{}%
\AgdaBound{f}\AgdaSpace{}%
\AgdaBound{a₁}\AgdaSpace{}%
\AgdaOperator{\AgdaDatatype{≡}}\AgdaSpace{}%
\AgdaBound{f}\AgdaSpace{}%
\AgdaBound{a₂}\AgdaSpace{}%
\AgdaSymbol{→}\AgdaSpace{}%
\AgdaBound{a₁}\AgdaSpace{}%
\AgdaOperator{\AgdaDatatype{≡}}\AgdaSpace{}%
\AgdaBound{a₂}\<%
\end{code}
\ccpad
We obtain the left-inverse by applying the function \af{MonicInv} to \ab f and a proof %% ,  \ab p \as : \af{Monic} \ab f,
that \ab f is monic.
\ccpad
\begin{code}%
\>[1]\AgdaFunction{MonicInv}%
\>[338I]\AgdaSymbol{:}%
\>[339I]\AgdaSymbol{\{}\AgdaBound{A}\AgdaSpace{}%
\AgdaSymbol{:}\AgdaSpace{}%
\AgdaBound{𝓤}\AgdaSpace{}%
\AgdaOperator{\AgdaFunction{̇}}\AgdaSpace{}%
\AgdaSymbol{\}}\AgdaSpace{}%
\AgdaSymbol{\{}\AgdaBound{B}\AgdaSpace{}%
\AgdaSymbol{:}\AgdaSpace{}%
\AgdaBound{𝓦}\AgdaSpace{}%
\AgdaOperator{\AgdaFunction{̇}}\AgdaSpace{}%
\AgdaSymbol{\}}\<%
\\
\>[.][@{}l@{}]\<[339I]%
\>[12]\AgdaSymbol{(}\AgdaBound{f}\AgdaSpace{}%
\AgdaSymbol{:}\AgdaSpace{}%
\AgdaBound{A}\AgdaSpace{}%
\AgdaSymbol{→}\AgdaSpace{}%
\AgdaBound{B}\AgdaSymbol{)}%
\>[25]\AgdaSymbol{→}%
\>[28]\AgdaFunction{Monic}\AgdaSpace{}%
\AgdaBound{f}\<%
\\
\>[338I][@{}l@{\AgdaIndent{0}}]%
\>[11]\AgdaComment{---------------------}\<%
\\
\>[1][@{}l@{\AgdaIndent{0}}]%
\>[2]\AgdaSymbol{→}%
\>[12]\AgdaSymbol{(}\AgdaBound{b}\AgdaSpace{}%
\AgdaSymbol{:}\AgdaSpace{}%
\AgdaBound{B}\AgdaSymbol{)}\AgdaSpace{}%
\AgdaSymbol{→}%
\>[23]\AgdaOperator{\AgdaDatatype{Image}}\AgdaSpace{}%
\AgdaBound{f}\AgdaSpace{}%
\AgdaOperator{\AgdaDatatype{∋}}\AgdaSpace{}%
\AgdaBound{b}%
\>[36]\AgdaSymbol{→}%
\>[39]\AgdaBound{A}\<%
\\
%
\\[\AgdaEmptyExtraSkip]%
%
\>[1]\AgdaFunction{MonicInv}\AgdaSpace{}%
\AgdaBound{f}\AgdaSpace{}%
\AgdaSymbol{\AgdaUnderscore{}}\AgdaSpace{}%
\AgdaSymbol{=}\AgdaSpace{}%
\AgdaSymbol{λ}\AgdaSpace{}%
\AgdaBound{b}\AgdaSpace{}%
\AgdaBound{Imf∋b}\AgdaSpace{}%
\AgdaSymbol{→}\AgdaSpace{}%
\AgdaFunction{Inv}\AgdaSpace{}%
\AgdaBound{f}\AgdaSpace{}%
\AgdaBound{b}\AgdaSpace{}%
\AgdaBound{Imf∋b}\<%
\end{code}
\ccpad
The function defined by \af{MonicInv} \ab f \ab p is the \emph{left-inverse} of \ab f.
\ccpad
\begin{code}%
\>[1]\AgdaFunction{MonicInvIsLeftInv}%
\>[371I]\AgdaSymbol{:}%
\>[372I]\AgdaSymbol{\{}\AgdaBound{A}\AgdaSpace{}%
\AgdaSymbol{:}\AgdaSpace{}%
\AgdaBound{𝓤}\AgdaSpace{}%
\AgdaOperator{\AgdaFunction{̇}}\AgdaSpace{}%
\AgdaSymbol{\}\{}\AgdaBound{B}\AgdaSpace{}%
\AgdaSymbol{:}\AgdaSpace{}%
\AgdaBound{𝓦}\AgdaSpace{}%
\AgdaOperator{\AgdaFunction{̇}}\AgdaSpace{}%
\AgdaSymbol{\}}\<%
\\
\>[.][@{}l@{}]\<[372I]%
\>[21]\AgdaSymbol{(}\AgdaBound{f}\AgdaSpace{}%
\AgdaSymbol{:}\AgdaSpace{}%
\AgdaBound{A}\AgdaSpace{}%
\AgdaSymbol{→}\AgdaSpace{}%
\AgdaBound{B}\AgdaSymbol{)}\AgdaSpace{}%
\AgdaSymbol{(}\AgdaBound{fmonic}\AgdaSpace{}%
\AgdaSymbol{:}\AgdaSpace{}%
\AgdaFunction{Monic}\AgdaSpace{}%
\AgdaBound{f}\AgdaSymbol{)(}\AgdaBound{x}\AgdaSpace{}%
\AgdaSymbol{:}\AgdaSpace{}%
\AgdaBound{A}\AgdaSymbol{)}\<%
\\
\>[371I][@{}l@{\AgdaIndent{0}}]%
\>[20]\AgdaComment{----------------------------}\<%
\\
\>[1][@{}l@{\AgdaIndent{0}}]%
\>[3]\AgdaSymbol{→}%
\>[21]\AgdaSymbol{(}\AgdaFunction{MonicInv}\AgdaSpace{}%
\AgdaBound{f}\AgdaSpace{}%
\AgdaBound{fmonic}\AgdaSymbol{)}\AgdaSpace{}%
\AgdaSymbol{(}\AgdaBound{f}\AgdaSpace{}%
\AgdaBound{x}\AgdaSymbol{)}\AgdaSpace{}%
\AgdaSymbol{(}\AgdaInductiveConstructor{im}\AgdaSpace{}%
\AgdaBound{x}\AgdaSymbol{)}\AgdaSpace{}%
\AgdaOperator{\AgdaDatatype{≡}}\AgdaSpace{}%
\AgdaBound{x}\<%
\\
%
\\[\AgdaEmptyExtraSkip]%
%
\>[1]\AgdaFunction{MonicInvIsLeftInv}\AgdaSpace{}%
\AgdaBound{f}\AgdaSpace{}%
\AgdaBound{fmonic}\AgdaSpace{}%
\AgdaBound{x}\AgdaSpace{}%
\AgdaSymbol{=}\AgdaSpace{}%
\AgdaInductiveConstructor{refl}\AgdaSpace{}%
\AgdaSymbol{\AgdaUnderscore{}}\<%
\end{code}








\subsubsection{Monics are set embeddings}\label{sssec:monics-are-set-embeddings}
This is the first point at which \href{https://ualib.gitlab.io/UALib.Prelude.Preliminaries.html#truncation}{truncation} (\ref{sssec:truncation}) comes into play.  An
\href{https://www.cs.bham.ac.uk/~mhe/HoTT-UF-in-Agda-Lecture-Notes/HoTT-UF-Agda.html\#embeddings}{\textit{embedding}} is defined in the \TypeTopology library via the three functions \af{is-embedding}, \af{is-subsingleton}, and \af{fiber}.
\ccpad
\begin{code}
\>[0]\AgdaFunction{is-embedding}\AgdaSpace{}%
\AgdaSymbol{:}\AgdaSpace{}%
\AgdaSymbol{\{}\AgdaBound{X}\AgdaSpace{}%
\AgdaSymbol{:}\AgdaSpace{}%
\AgdaGeneralizable{𝓤}\AgdaSpace{}%
\AgdaOperator{\AgdaFunction{̇}}\AgdaSpace{}%
\AgdaSymbol{\}}\AgdaSpace{}%
\AgdaSymbol{\{}\AgdaBound{Y}\AgdaSpace{}%
\AgdaSymbol{:}\AgdaSpace{}%
\AgdaGeneralizable{𝓥}\AgdaSpace{}%
\AgdaOperator{\AgdaFunction{̇}}\AgdaSpace{}%
\AgdaSymbol{\}}\AgdaSpace{}%
\AgdaSymbol{→}\AgdaSpace{}%
\AgdaSymbol{(}\AgdaBound{X}\AgdaSpace{}%
\AgdaSymbol{→}\AgdaSpace{}%
\AgdaBound{Y}\AgdaSymbol{)}\AgdaSpace{}%
\AgdaSymbol{→}\AgdaSpace{}%
\AgdaGeneralizable{𝓤}\AgdaSpace{}%
\AgdaOperator{\AgdaPrimitive{⊔}}\AgdaSpace{}%
\AgdaGeneralizable{𝓥}\AgdaSpace{}%
\AgdaOperator{\AgdaFunction{̇}}\<%
\\
\>[0]\AgdaFunction{is-embedding}\AgdaSpace{}%
\AgdaBound{f}\AgdaSpace{}%
\AgdaSymbol{=}\AgdaSpace{}%
\AgdaSymbol{(}\AgdaBound{y}\AgdaSpace{}%
\AgdaSymbol{:}\AgdaSpace{}%
\AgdaFunction{codomain}\AgdaSpace{}%
\AgdaBound{f}\AgdaSymbol{)}\AgdaSpace{}%
\AgdaSymbol{→}\AgdaSpace{}%
\AgdaFunction{is-subsingleton}\AgdaSpace{}%
\AgdaSymbol{(}\AgdaFunction{fiber}\AgdaSpace{}%
\AgdaBound{f}\AgdaSpace{}%
\AgdaBound{y}\AgdaSymbol{)}\<%
\end{code}
\scpad
\begin{code}
\>[0]\AgdaFunction{is-subsingleton}\AgdaSpace{}%
\AgdaSymbol{:}\AgdaSpace{}%
\AgdaGeneralizable{𝓤}\AgdaSpace{}%
\AgdaOperator{\AgdaFunction{̇}}\AgdaSpace{}%
\AgdaSymbol{→}\AgdaSpace{}%
\AgdaGeneralizable{𝓤}\AgdaSpace{}%
\AgdaOperator{\AgdaFunction{̇}}\<%
\\
\>[0]\AgdaFunction{is-subsingleton}\AgdaSpace{}%
\AgdaBound{X}\AgdaSpace{}%
\AgdaSymbol{=}\AgdaSpace{}%
\AgdaSymbol{(}\AgdaBound{x}\AgdaSpace{}%
\AgdaBound{y}\AgdaSpace{}%
\AgdaSymbol{:}\AgdaSpace{}%
\AgdaBound{X}\AgdaSymbol{)}\AgdaSpace{}%
\AgdaSymbol{→}\AgdaSpace{}%
\AgdaBound{x}\AgdaSpace{}%
\AgdaOperator{\AgdaDatatype{≡}}\AgdaSpace{}%
\AgdaBound{y}\<%
\end{code}
\scpad
\begin{code}
\>[0]\AgdaFunction{fiber}\AgdaSpace{}%
\AgdaSymbol{:}\AgdaSpace{}%
\AgdaSymbol{\{}\AgdaBound{X}\AgdaSpace{}%
\AgdaSymbol{:}\AgdaSpace{}%
\AgdaGeneralizable{𝓤}\AgdaSpace{}%
\AgdaOperator{\AgdaFunction{̇}}\AgdaSpace{}%
\AgdaSymbol{\}}\AgdaSpace{}%
\AgdaSymbol{\{}\AgdaBound{Y}\AgdaSpace{}%
\AgdaSymbol{:}\AgdaSpace{}%
\AgdaGeneralizable{𝓥}\AgdaSpace{}%
\AgdaOperator{\AgdaFunction{̇}}\AgdaSpace{}%
\AgdaSymbol{\}}\AgdaSpace{}%
\AgdaSymbol{(}\AgdaBound{f}\AgdaSpace{}%
\AgdaSymbol{:}\AgdaSpace{}%
\AgdaBound{X}\AgdaSpace{}%
\AgdaSymbol{→}\AgdaSpace{}%
\AgdaBound{Y}\AgdaSymbol{)}\AgdaSpace{}%
\AgdaSymbol{→}\AgdaSpace{}%
\AgdaBound{Y}\AgdaSpace{}%
\AgdaSymbol{→}\AgdaSpace{}%
\AgdaGeneralizable{𝓤}\AgdaSpace{}%
\AgdaOperator{\AgdaPrimitive{⊔}}\AgdaSpace{}%
\AgdaGeneralizable{𝓥}\AgdaSpace{}%
\AgdaOperator{\AgdaFunction{̇}}\<%
\\
\>[0]\AgdaFunction{fiber}\AgdaSpace{}%
\AgdaBound{f}\AgdaSpace{}%
\AgdaBound{y}\AgdaSpace{}%
\AgdaSymbol{=}\AgdaSpace{}%
\AgdaFunction{Σ}\AgdaSpace{}%
\AgdaBound{x}\AgdaSpace{}%
\AgdaFunction{꞉}\AgdaSpace{}%
\AgdaFunction{domain}\AgdaSpace{}%
\AgdaBound{f}\AgdaSpace{}%
\AgdaFunction{,}\AgdaSpace{}%
\AgdaBound{f}\AgdaSpace{}%
\AgdaBound{x}\AgdaSpace{}%
\AgdaOperator{\AgdaDatatype{≡}}\AgdaSpace{}%
\AgdaBound{y}\<%
\end{code}
\ccpad
This is a natural way to represent what we usually mean in mathematics by embedding. It does not correspond simply to a monic function, and it is important to understand why not. Suppose \ab f~\as :~\ab X~\as →~\ab Y is a monic function from \ab X to \ab Y, so we have a proof \ab p \as : \af{Monic} \ab f.  To prove \ab f is an embedding we must show that for every \ab y~\as :~\ab Y we have
\AgdaFunction{is-subsingleton}\AgdaSpace{}\AgdaSymbol{(}\AgdaFunction{fiber}\AgdaSpace{}%
\AgdaBound{f}\AgdaSpace{}%
\AgdaBound{y}\AgdaSymbol{)}.  That is, for all \ab y \as : \ab Y, we must prove the following implication:
%% show that if (\ab x \ac \ab p) and (\ab{x'} \ac \ab q) satisfy\\
\begin{equation*}
  \infer{(\ab x \Mac \ab p) \Maof ≡ (\ab{x'} \Mac \ab q)}
    {(\ab x \ \ab{x'} \Mas : \ab X) & (\ab p \Mas : \ab f \ \ab x \Maof ≡ \ab y) & (\ab q \Mas : \ab f\  \ab{x'} \Maof ≡ \ab y) & (\ab m \Mas : \af{Monic}\ \ab f)}
\end{equation*}
By \ab m, \ab p, and \ab q, we have \ab r \as : \ab x \aof ≡ \ab{x'}.
%% ; applying this to our initial antecedent gives (\af{ap} \ab q \ab r) \as : \ab f \ \ab x \aof ≡ \ab y.
%% \ab p \as : \ab f \ab x \aof ≡ \ab y and \ab q \as : \ab f \ab x \aof ≡ \ab y.
%% \\
%% \\
%% (\ab x \ab{x'} \as : \ab X) \hskip3mm (\ab a : \ab x \aof ≡ \ab{x'}) \hskip3mm (\ab p \as : \ab f \ab x \aof ≡ \ab y) \hskip3mm (\ab q \as : \ab f \ab x \aof ≡ \ab y)\\
%% \\
Thus, in order to prove \ab f~is an embedding, we must somehow show that the proofs \ab p and \ab q (each of which entails \ab f \ab x \aof ≡ \ab y) are the same.  However, there is no axiom or deduction rule in \mltt to indicate that \ab p \aof ≡ \ab q must hold; indeed, the two proofs may differ.

One way we could resolve this is to assume that the codomain type, \ab B, is a \emph{set}, i.e., has \emph{unique identity proofs}.
%% , then we can prove that a injective function into \ab B is an embedding.
Recall the definition~(\ref{code:is-set}) of \af{is-set} from the \typetopology library.
%% It is clear that this is precisely what we need.  That is, i
If the codomain \ab B of the function \ab f \as : \ab A → \ab B is a set, and if \ab p and \ab q are two proofs of an equality of elements of type \ab B, then \ab p \aof ≡ \ab q, and we can use this to prove that a injective function into the set \ab B is an embedding. The following function proves this claim.
\ccpad
\begin{code}%
\>[0][@{}l@{\AgdaIndent{0}}]%
\>[1]\AgdaFunction{monic-into-set-is-embedding}
\>[371I]\AgdaSymbol{:}%
\>[372I]\AgdaSymbol{\{}\AgdaBound{A}\AgdaSpace{}%
\AgdaSymbol{:}\AgdaSpace{}%
\AgdaBound{𝓤}\AgdaSpace{}%
\AgdaOperator{\AgdaFunction{̇}}\AgdaSymbol{\}\{}\AgdaBound{B}\AgdaSpace{}%
\AgdaSymbol{:}\AgdaSpace{}%
\AgdaBound{𝓦}\AgdaSpace{}%
\AgdaOperator{\AgdaFunction{̇}}\AgdaSymbol{\}}\AgdaSpace{}%
\AgdaSymbol{→}\AgdaSpace{}%
\AgdaFunction{is-set}\AgdaSpace{}%
\AgdaBound{B}\<%
\\
\>[1][@{}l@{\AgdaIndent{0}}]%
\>[2]\AgdaSymbol{→}%
\>[.][@{}l@{}]\<[372I]%
\>[21]\AgdaSymbol{(}\AgdaBound{f}\AgdaSpace{}%
\AgdaSymbol{:}\AgdaSpace{}%
\AgdaBound{A}\AgdaSpace{}%
\AgdaSymbol{→}\AgdaSpace{}%
\AgdaBound{B}\AgdaSymbol{)}%
\>[44]\AgdaSymbol{→}%
\>[47]\AgdaFunction{Monic}\AgdaSpace{}%
\AgdaBound{f}\<%
\\
\>[2][@{}l@{\AgdaIndent{0}}]%
\>[21]\AgdaComment{--------------------}\<%
\\
%
\>[2]\AgdaSymbol{→}%
\>[21]\AgdaFunction{is-embedding}\AgdaSpace{}%
\AgdaBound{f}\<%
\\
%
\\[\AgdaEmptyExtraSkip]%
%
\>[1]\AgdaFunction{monic-into-set-is-embedding}\AgdaSpace{}%
\AgdaSymbol{\{}\AgdaBound{A}\AgdaSymbol{\}}\AgdaSpace{}%
\AgdaBound{Bset}\AgdaSpace{}%
\AgdaBound{f}\AgdaSpace{}%
\AgdaBound{fmon}\AgdaSpace{}%
\AgdaBound{b}\AgdaSpace{}%
\AgdaSymbol{(}\AgdaBound{a}\AgdaSpace{}%
\AgdaOperator{\AgdaInductiveConstructor{,}}\AgdaSpace{}%
\AgdaBound{fa≡b}\AgdaSymbol{)}\AgdaSpace{}%
\AgdaSymbol{(}\AgdaBound{a'}\AgdaSpace{}%
\AgdaOperator{\AgdaInductiveConstructor{,}}\AgdaSpace{}%
\AgdaBound{fa'≡b}\AgdaSymbol{)}\AgdaSpace{}%
\AgdaSymbol{=}\AgdaSpace{}%
\AgdaFunction{γ}\<%
\\
\>[1][@{}l@{\AgdaIndent{0}}]%
\>[2]\AgdaKeyword{where}\<%
\\
\>[2][@{}l@{\AgdaIndent{0}}]%
\>[3]\AgdaFunction{faa'}\AgdaSpace{}%
\AgdaSymbol{:}\AgdaSpace{}%
\AgdaBound{f}\AgdaSpace{}%
\AgdaBound{a}\AgdaSpace{}%
\AgdaOperator{\AgdaDatatype{≡}}\AgdaSpace{}%
\AgdaBound{f}\AgdaSpace{}%
\AgdaBound{a'}\<%
\\
%
\>[3]\AgdaFunction{faa'}\AgdaSpace{}%
\AgdaSymbol{=}\AgdaSpace{}%
\AgdaFunction{≡-Trans}\AgdaSpace{}%
\AgdaSymbol{(}\AgdaBound{f}\AgdaSpace{}%
\AgdaBound{a}\AgdaSymbol{)}\AgdaSpace{}%
\AgdaSymbol{(}\AgdaBound{f}\AgdaSpace{}%
\AgdaBound{a'}\AgdaSymbol{)}\AgdaSpace{}%
\AgdaBound{fa≡b}\AgdaSpace{}%
\AgdaSymbol{(}\AgdaBound{fa'≡b}\AgdaSpace{}%
\AgdaOperator{\AgdaFunction{⁻¹}}\AgdaSymbol{)}\<%
\\
%
\\[\AgdaEmptyExtraSkip]%
%
\>[3]\AgdaFunction{aa'}\AgdaSpace{}%
\AgdaSymbol{:}\AgdaSpace{}%
\AgdaBound{a}\AgdaSpace{}%
\AgdaOperator{\AgdaDatatype{≡}}\AgdaSpace{}%
\AgdaBound{a'}\<%
\\
%
\>[3]\AgdaFunction{aa'}\AgdaSpace{}%
\AgdaSymbol{=}\AgdaSpace{}%
\AgdaBound{fmon}\AgdaSpace{}%
\AgdaBound{a}\AgdaSpace{}%
\AgdaBound{a'}\AgdaSpace{}%
\AgdaFunction{faa'}\<%
\\
%
\\[\AgdaEmptyExtraSkip]%
%
\>[3]\AgdaFunction{𝒜}\AgdaSpace{}%
\AgdaSymbol{:}\AgdaSpace{}%
\AgdaBound{A}\AgdaSpace{}%
\AgdaSymbol{→}\AgdaSpace{}%
\AgdaBound{𝓦}\AgdaSpace{}%
\AgdaOperator{\AgdaFunction{̇}}\<%
\\
%
\>[3]\AgdaFunction{𝒜}\AgdaSpace{}%
\AgdaBound{a}\AgdaSpace{}%
\AgdaSymbol{=}\AgdaSpace{}%
\AgdaBound{f}\AgdaSpace{}%
\AgdaBound{a}\AgdaSpace{}%
\AgdaOperator{\AgdaDatatype{≡}}\AgdaSpace{}%
\AgdaBound{b}\<%
\\
%
\\[\AgdaEmptyExtraSkip]%
%
\>[3]\AgdaFunction{arg1}\AgdaSpace{}%
\AgdaSymbol{:}\AgdaSpace{}%
\AgdaFunction{Σ}\AgdaSpace{}%
\AgdaBound{p}\AgdaSpace{}%
\AgdaFunction{꞉}\AgdaSpace{}%
\AgdaSymbol{(}\AgdaBound{a}\AgdaSpace{}%
\AgdaOperator{\AgdaDatatype{≡}}\AgdaSpace{}%
\AgdaBound{a'}\AgdaSymbol{)}\AgdaSpace{}%
\AgdaFunction{,}\AgdaSpace{}%
\AgdaSymbol{(}\AgdaFunction{transport}\AgdaSpace{}%
\AgdaFunction{𝒜}\AgdaSpace{}%
\AgdaBound{p}\AgdaSpace{}%
\AgdaBound{fa≡b}\AgdaSymbol{)}\AgdaSpace{}%
\AgdaOperator{\AgdaDatatype{≡}}\AgdaSpace{}%
\AgdaBound{fa'≡b}\<%
\\
%
\>[3]\AgdaFunction{arg1}\AgdaSpace{}%
\AgdaSymbol{=}\AgdaSpace{}%
\AgdaFunction{aa'}\AgdaSpace{}%
\AgdaOperator{\AgdaInductiveConstructor{,}}\AgdaSpace{}%
\AgdaBound{Bset}\AgdaSpace{}%
\AgdaSymbol{(}\AgdaBound{f}\AgdaSpace{}%
\AgdaBound{a'}\AgdaSymbol{)}\AgdaSpace{}%
\AgdaBound{b}\AgdaSpace{}%
\AgdaSymbol{(}\AgdaFunction{transport}\AgdaSpace{}%
\AgdaFunction{𝒜}\AgdaSpace{}%
\AgdaFunction{aa'}\AgdaSpace{}%
\AgdaBound{fa≡b}\AgdaSymbol{)}\AgdaSpace{}%
\AgdaBound{fa'≡b}\<%
\\
%
\\[\AgdaEmptyExtraSkip]%
%
\>[3]\AgdaFunction{γ}\AgdaSpace{}%
\AgdaSymbol{:}\AgdaSpace{}%
\AgdaBound{a}\AgdaSpace{}%
\AgdaOperator{\AgdaInductiveConstructor{,}}\AgdaSpace{}%
\AgdaBound{fa≡b}\AgdaSpace{}%
\AgdaOperator{\AgdaDatatype{≡}}\AgdaSpace{}%
\AgdaBound{a'}\AgdaSpace{}%
\AgdaOperator{\AgdaInductiveConstructor{,}}\AgdaSpace{}%
\AgdaBound{fa'≡b}\<%
\\
%
\>[3]\AgdaFunction{γ}\AgdaSpace{}%
\AgdaSymbol{=}\AgdaSpace{}%
\AgdaFunction{to-Σ-≡}\AgdaSpace{}%
\AgdaFunction{arg1}\<%
\\
\>[0]\<%
\end{code}

%%-- END: In particular, we define the syntax we use to represent the semantic concept of the \textbf{inverse image} of a function.
\ccpad
\begin{code}%
%% \>[0]\AgdaSymbol{\{-\#}\AgdaSpace{}%
%% \AgdaKeyword{OPTIONS}\AgdaSpace{}%
%% \AgdaPragma{--without-K}\AgdaSpace{}%
%% \AgdaPragma{--exact-split}\AgdaSpace{}%
%% \AgdaPragma{--safe}\AgdaSpace{}%
%% \AgdaSymbol{\#-\}}\<%
%% \\
%% %
%% \ccpad
%% %
%% \>[0]\AgdaKeyword{module}\AgdaSpace{}%
%% \AgdaModule{UALib.Prelude.Inverses}\AgdaSpace{}%
%% \AgdaKeyword{where}\<%
%% \\
%% %
%% \ccpad
%% %
%% \>[0]\AgdaKeyword{open}\AgdaSpace{}%
%% \AgdaKeyword{import}\AgdaSpace{}%
%% \AgdaModule{UALib.Prelude.Equality}\AgdaSpace{}%
%% \AgdaKeyword{public}\<%
%% \\
%% %
%% \ccpad
%% %
%% \>[0]\AgdaKeyword{open}\AgdaSpace{}%
%% \AgdaKeyword{import}\AgdaSpace{}%
%% \AgdaModule{UALib.Prelude.Preliminaries}\AgdaSpace{}%
%% \AgdaKeyword{using}\AgdaSpace{}%
%% \AgdaSymbol{(}\AgdaOperator{\AgdaFunction{\AgdaUnderscore{}⁻¹}}\AgdaSymbol{;}\AgdaSpace{}%
%% \AgdaFunction{funext}\AgdaSymbol{;}\AgdaSpace{}%
%% \AgdaOperator{\AgdaFunction{\AgdaUnderscore{}∘\AgdaUnderscore{}}}\AgdaSymbol{;}\AgdaSpace{}%
%% \AgdaOperator{\AgdaFunction{\AgdaUnderscore{}∙\AgdaUnderscore{}}}\AgdaSymbol{;}\AgdaSpace{}%
%% \AgdaFunction{𝑖𝑑}\AgdaSymbol{;}\AgdaSpace{}%
%% \AgdaFunction{fst}\AgdaSymbol{;}\AgdaSpace{}%
%% \AgdaFunction{snd}\AgdaSymbol{;}\AgdaSpace{}%
%% \AgdaFunction{is-set}\AgdaSymbol{;}\AgdaSpace{}%
%% \AgdaFunction{is-embedding}\AgdaSymbol{;}\<%
%% \\
%% \>[0][@{}l@{\AgdaIndent{0}}]%
%% \>[1]\AgdaFunction{transport}\AgdaSymbol{;}\AgdaSpace{}%
%% \AgdaFunction{to-Σ-≡}\AgdaSymbol{;}\AgdaSpace{}%
%% \AgdaFunction{invertible}\AgdaSymbol{;}\AgdaSpace{}%
%% \AgdaFunction{equivs-are-embeddings}\AgdaSymbol{;}\AgdaSpace{}%
%% \AgdaFunction{invertibles-are-equivs}\AgdaSymbol{;}\AgdaSpace{}%
%% \AgdaFunction{fiber}\AgdaSymbol{;}\AgdaSpace{}%
%% \AgdaInductiveConstructor{𝓇ℯ𝒻𝓁}\AgdaSymbol{)}\AgdaSpace{}%
%% \AgdaKeyword{public}\<%
%% \\
%% %
%% \ccpad
%% %
\>[0]\AgdaKeyword{module}\AgdaSpace{}%
\AgdaModule{\AgdaUnderscore{}}\AgdaSpace{}%
\AgdaSymbol{\{}\AgdaBound{𝓤}\AgdaSpace{}%
\AgdaBound{𝓦}\AgdaSpace{}%
\AgdaSymbol{:}\AgdaSpace{}%
\AgdaPostulate{Universe}\AgdaSymbol{\}}\AgdaSpace{}%
\AgdaKeyword{where}\<%
\\
%
\\
\>[0][@{}l@{\AgdaIndent{0}}]%
\>[1]\AgdaKeyword{data}\AgdaSpace{}%
\AgdaOperator{\AgdaDatatype{Image\AgdaUnderscore{}∋\AgdaUnderscore{}}}\AgdaSpace{}%
\AgdaSymbol{\{}\AgdaBound{A}\AgdaSpace{}%
\AgdaSymbol{:}\AgdaSpace{}%
\AgdaBound{𝓤}\AgdaSpace{}%
\AgdaOperator{\AgdaFunction{̇}}\AgdaSpace{}%
\AgdaSymbol{\}\{}\AgdaBound{B}\AgdaSpace{}%
\AgdaSymbol{:}\AgdaSpace{}%
\AgdaBound{𝓦}\AgdaSpace{}%
\AgdaOperator{\AgdaFunction{̇}}\AgdaSpace{}%
\AgdaSymbol{\}(}\AgdaBound{f}\AgdaSpace{}%
\AgdaSymbol{:}\AgdaSpace{}%
\AgdaBound{A}\AgdaSpace{}%
\AgdaSymbol{→}\AgdaSpace{}%
\AgdaBound{B}\AgdaSymbol{)}\AgdaSpace{}%
\AgdaSymbol{:}\AgdaSpace{}%
\AgdaBound{B}\AgdaSpace{}%
\AgdaSymbol{→}\AgdaSpace{}%
\AgdaBound{𝓤}\AgdaSpace{}%
\AgdaOperator{\AgdaPrimitive{⊔}}\AgdaSpace{}%
\AgdaBound{𝓦}\AgdaSpace{}%
\AgdaOperator{\AgdaFunction{̇}}\<%
\\
\>[1][@{}l@{\AgdaIndent{0}}]%
\>[2]\AgdaKeyword{where}\<%
\\
%
\>[2]\AgdaInductiveConstructor{im}\AgdaSpace{}%
\AgdaSymbol{:}\AgdaSpace{}%
\AgdaSymbol{(}\AgdaBound{x}\AgdaSpace{}%
\AgdaSymbol{:}\AgdaSpace{}%
\AgdaBound{A}\AgdaSymbol{)}\AgdaSpace{}%
\AgdaSymbol{→}\AgdaSpace{}%
\AgdaOperator{\AgdaDatatype{Image}}\AgdaSpace{}%
\AgdaBound{f}\AgdaSpace{}%
\AgdaOperator{\AgdaDatatype{∋}}\AgdaSpace{}%
\AgdaBound{f}\AgdaSpace{}%
\AgdaBound{x}\<%
\\
%
\>[2]\AgdaInductiveConstructor{eq}\AgdaSpace{}%
\AgdaSymbol{:}\AgdaSpace{}%
\AgdaSymbol{(}\AgdaBound{b}\AgdaSpace{}%
\AgdaSymbol{:}\AgdaSpace{}%
\AgdaBound{B}\AgdaSymbol{)}\AgdaSpace{}%
\AgdaSymbol{→}\AgdaSpace{}%
\AgdaSymbol{(}\AgdaBound{a}\AgdaSpace{}%
\AgdaSymbol{:}\AgdaSpace{}%
\AgdaBound{A}\AgdaSymbol{)}\AgdaSpace{}%
\AgdaSymbol{→}\AgdaSpace{}%
\AgdaBound{b}\AgdaSpace{}%
\AgdaOperator{\AgdaDatatype{≡}}\AgdaSpace{}%
\AgdaBound{f}\AgdaSpace{}%
\AgdaBound{a}\AgdaSpace{}%
\AgdaSymbol{→}\AgdaSpace{}%
\AgdaOperator{\AgdaDatatype{Image}}\AgdaSpace{}%
\AgdaBound{f}\AgdaSpace{}%
\AgdaOperator{\AgdaDatatype{∋}}\AgdaSpace{}%
\AgdaBound{b}\<%
\end{code}
\ccpad
Note that an inhabitant of \af{Image} \ab f \ad ∋ \ab b is a dependent pair (\ab a \ac \ab p), where \ab a \as : \ab A and \ab p \as : \ab b  \ad ≡ \ab f \ab a is a proof that \ab f maps \ab a to \ab b. Thus, a proof that \ab b belongs to the image of \ab f (i.e., an inhabitant of \af{Image} \ab f \ad ∋ \ab b), is always accompanied by a witness \ab a \as : \ab A, and a proof that \ab b \ad{≡} \ab f \ab a, so the (pseudo-)inverse can actually be \emph{computed}.
%% \ccpad
%% \begin{code}
%% \>[1]\AgdaFunction{ImageIsImage}%
%% \>[84I]\AgdaSymbol{:}\AgdaSpace{}%
%% \AgdaSymbol{\{}\AgdaBound{A}\AgdaSpace{}%
%% \AgdaSymbol{:}\AgdaSpace{}%
%% \AgdaBound{𝓤}\AgdaSpace{}%
%% \AgdaOperator{\AgdaFunction{̇}}\AgdaSpace{}%
%% \AgdaSymbol{\}\{}\AgdaBound{B}\AgdaSpace{}%
%% \AgdaSymbol{:}\AgdaSpace{}%
%% \AgdaBound{𝓦}\AgdaSpace{}%
%% \AgdaOperator{\AgdaFunction{̇}}\AgdaSpace{}%
%% \AgdaSymbol{\}}\<%
%% \\
%% \>[84I][@{}l@{\AgdaIndent{0}}]%
%% \>[15]\AgdaSymbol{(}\AgdaBound{f}\AgdaSpace{}%
%% \AgdaSymbol{:}\AgdaSpace{}%
%% \AgdaBound{A}\AgdaSpace{}%
%% \AgdaSymbol{→}\AgdaSpace{}%
%% \AgdaBound{B}\AgdaSymbol{)}\AgdaSpace{}%
%% \AgdaSymbol{(}\AgdaBound{b}\AgdaSpace{}%
%% \AgdaSymbol{:}\AgdaSpace{}%
%% \AgdaBound{B}\AgdaSymbol{)}\AgdaSpace{}%
%% \AgdaSymbol{(}\AgdaBound{a}\AgdaSpace{}%
%% \AgdaSymbol{:}\AgdaSpace{}%
%% \AgdaBound{A}\AgdaSymbol{)}\<%
%% \\
%% \>[1][@{}l@{\AgdaIndent{0}}]%
%% \>[2]\AgdaSymbol{→}%
%% \>[15]\AgdaBound{b}\AgdaSpace{}%
%% \AgdaOperator{\AgdaDatatype{≡}}\AgdaSpace{}%
%% \AgdaBound{f}\AgdaSpace{}%
%% \AgdaBound{a}\<%
%% \\
%% \>[2][@{}l@{\AgdaIndent{0}}]%
%% \>[15]\AgdaComment{-------------------------}\<%
%% \\
%% %
%% \>[2]\AgdaSymbol{→}%
%% \>[15]\AgdaOperator{\AgdaDatatype{Image}}\AgdaSpace{}%
%% \AgdaBound{f}\AgdaSpace{}%
%% \AgdaOperator{\AgdaDatatype{∋}}\AgdaSpace{}%
%% \AgdaBound{b}\<%
%% \\
%% %
%% \>[1]\AgdaFunction{ImageIsImage}\AgdaSpace{}%
%% \AgdaSymbol{\{}\AgdaBound{A}\AgdaSymbol{\}\{}\AgdaBound{B}\AgdaSymbol{\}}\AgdaSpace{}%
%% \AgdaBound{f}\AgdaSpace{}%
%% \AgdaBound{b}\AgdaSpace{}%
%% \AgdaBound{a}\AgdaSpace{}%
%% \AgdaBound{b≡fa}\AgdaSpace{}%
%% \AgdaSymbol{=}\AgdaSpace{}%
%% \AgdaInductiveConstructor{eq}\AgdaSpace{}%
%% \AgdaBound{b}\AgdaSpace{}%
%% \AgdaBound{a}\AgdaSpace{}%
%% \AgdaBound{b≡fa}\<%
%% \end{code}
%% \ccpad

For convenience, we define a pseudo-inverse function, which we call \af{Inv}, that takes \ab b \as : \ab B and (\ab a \ac \ab p) \as : \af{Image} \ab f \ad ∋ \ab b and returns \ab a.
\ccpad
\begin{code}%
\>[0][@{}l@{\AgdaIndent{1}}]%
\>[1]\AgdaFunction{Inv}\AgdaSpace{}%
\AgdaSymbol{:}\AgdaSpace{}%
\AgdaSymbol{\{}\AgdaBound{A}\AgdaSpace{}%
\AgdaSymbol{:}\AgdaSpace{}%
\AgdaBound{𝓤}\AgdaSpace{}%
\AgdaOperator{\AgdaFunction{̇}}\AgdaSpace{}%
\AgdaSymbol{\}\{}\AgdaBound{B}\AgdaSpace{}%
\AgdaSymbol{:}\AgdaSpace{}%
\AgdaBound{𝓦}\AgdaSpace{}%
\AgdaOperator{\AgdaFunction{̇}}\AgdaSpace{}%
\AgdaSymbol{\}(}\AgdaBound{f}\AgdaSpace{}%
\AgdaSymbol{:}\AgdaSpace{}%
\AgdaBound{A}\AgdaSpace{}%
\AgdaSymbol{→}\AgdaSpace{}%
\AgdaBound{B}\AgdaSymbol{)(}\AgdaBound{b}\AgdaSpace{}%
\AgdaSymbol{:}\AgdaSpace{}%
\AgdaBound{B}\AgdaSymbol{)}\AgdaSpace{}%
\AgdaSymbol{→}\AgdaSpace{}%
\AgdaOperator{\AgdaDatatype{Image}}\AgdaSpace{}%
\AgdaBound{f}\AgdaSpace{}%
\AgdaOperator{\AgdaDatatype{∋}}\AgdaSpace{}%
\AgdaBound{b}%
\>[61]\AgdaSymbol{→}%
\>[64]\AgdaBound{A}\<%
\\
%
\>[1]\AgdaFunction{Inv}\AgdaSpace{}%
\AgdaBound{f}\AgdaSpace{}%
\AgdaDottedPattern{\AgdaSymbol{.(}}\AgdaDottedPattern{\AgdaBound{f}}\AgdaSpace{}%
\AgdaDottedPattern{\AgdaBound{a}}\AgdaDottedPattern{\AgdaSymbol{)}}\AgdaSpace{}%
\AgdaSymbol{(}\AgdaInductiveConstructor{im}\AgdaSpace{}%
\AgdaBound{a}\AgdaSymbol{)}\AgdaSpace{}%
\AgdaSymbol{=}\AgdaSpace{}%
\AgdaBound{a}\<%
\\
%
\>[1]\AgdaFunction{Inv}\AgdaSpace{}%
\AgdaBound{f}\AgdaSpace{}%
\AgdaBound{b}\AgdaSpace{}%
\AgdaSymbol{(}\AgdaInductiveConstructor{eq}\AgdaSpace{}%
\AgdaBound{b}\AgdaSpace{}%
\AgdaBound{a}\AgdaSpace{}%
\AgdaBound{b≡fa}\AgdaSymbol{)}\AgdaSpace{}%
\AgdaSymbol{=}\AgdaSpace{}%
\AgdaBound{a}\<%
\end{code}
\ccpad
Of course, we can prove that \af{Inv} \ab f is really the (right-)inverse of \ab f.
\ccpad
\begin{code}%
\>[0][@{}l@{\AgdaIndent{1}}]%
\>[1]\AgdaFunction{InvIsInv}%
\>[156I]\AgdaSymbol{:}%
\>[157I]\AgdaSymbol{\{}\AgdaBound{A}\AgdaSpace{}%
\AgdaSymbol{:}\AgdaSpace{}%
\AgdaBound{𝓤}\AgdaSpace{}%
\AgdaOperator{\AgdaFunction{̇}}\AgdaSpace{}%
\AgdaSymbol{\}}\AgdaSpace{}%
\AgdaSymbol{\{}\AgdaBound{B}\AgdaSpace{}%
\AgdaSymbol{:}\AgdaSpace{}%
\AgdaBound{𝓦}\AgdaSpace{}%
\AgdaOperator{\AgdaFunction{̇}}\AgdaSpace{}%
\AgdaSymbol{\}}\AgdaSpace{}%
\AgdaSymbol{(}\AgdaBound{f}\AgdaSpace{}%
\AgdaSymbol{:}\AgdaSpace{}%
\AgdaBound{A}\AgdaSpace{}%
\AgdaSymbol{→}\AgdaSpace{}%
\AgdaBound{B}\AgdaSymbol{)}\<%
\\
\>[.][@{}l@{}]\<[157I]%
\>[12]\AgdaSymbol{(}\AgdaBound{b}\AgdaSpace{}%
\AgdaSymbol{:}\AgdaSpace{}%
\AgdaBound{B}\AgdaSymbol{)}\AgdaSpace{}%
\AgdaSymbol{(}\AgdaBound{b∈Imgf}\AgdaSpace{}%
\AgdaSymbol{:}\AgdaSpace{}%
\AgdaOperator{\AgdaDatatype{Image}}\AgdaSpace{}%
\AgdaBound{f}\AgdaSpace{}%
\AgdaOperator{\AgdaDatatype{∋}}\AgdaSpace{}%
\AgdaBound{b}\AgdaSymbol{)}\<%
\\
\>[156I][@{}l@{\AgdaIndent{0}}]%
\>[11]\AgdaComment{---------------------------------}\<%
\\
\>[1][@{}l@{\AgdaIndent{0}}]%
\>[2]\AgdaSymbol{→}%
\>[12]\AgdaBound{f}\AgdaSpace{}%
\AgdaSymbol{(}\AgdaFunction{Inv}\AgdaSpace{}%
\AgdaBound{f}\AgdaSpace{}%
\AgdaBound{b}\AgdaSpace{}%
\AgdaBound{b∈Imgf}\AgdaSymbol{)}\AgdaSpace{}%
\AgdaOperator{\AgdaDatatype{≡}}\AgdaSpace{}%
\AgdaBound{b}\<%
\\
%
\>[1]\AgdaFunction{InvIsInv}\AgdaSpace{}%
\AgdaBound{f}\AgdaSpace{}%
\AgdaDottedPattern{\AgdaSymbol{.(}}\AgdaDottedPattern{\AgdaBound{f}}\AgdaSpace{}%
\AgdaDottedPattern{\AgdaBound{a}}\AgdaDottedPattern{\AgdaSymbol{)}}\AgdaSpace{}%
\AgdaSymbol{(}\AgdaInductiveConstructor{im}\AgdaSpace{}%
\AgdaBound{a}\AgdaSymbol{)}\AgdaSpace{}%
\AgdaSymbol{=}\AgdaSpace{}%
\AgdaInductiveConstructor{refl}\AgdaSpace{}%
\AgdaSymbol{\AgdaUnderscore{}}\<%
\\
%
\>[1]\AgdaFunction{InvIsInv}\AgdaSpace{}%
\AgdaBound{f}\AgdaSpace{}%
\AgdaBound{b}\AgdaSpace{}%
\AgdaSymbol{(}\AgdaInductiveConstructor{eq}\AgdaSpace{}%
\AgdaBound{b}\AgdaSpace{}%
\AgdaBound{a}\AgdaSpace{}%
\AgdaBound{b≡fa}\AgdaSymbol{)}\AgdaSpace{}%
\AgdaSymbol{=}\AgdaSpace{}%
\AgdaBound{b≡fa}\AgdaSpace{}%
\AgdaOperator{\AgdaFunction{⁻¹}}\<%
\end{code}

\subsubsection{Surjective functions}\label{surjective-functions}
An epic (or surjective) function from type \AgdaTyped{A}{𝓤 ̇} to type \AgdaTyped{B}{𝓦 ̇} is as an inhabitant of the \af{Epic} type, which we define as follows.
\ccpad
\begin{code}%
\>[0][@{}l@{\AgdaIndent{1}}]%
\>[1]\AgdaFunction{Epic}\AgdaSpace{}%
\AgdaSymbol{:}\AgdaSpace{}%
\AgdaSymbol{\{}\AgdaBound{A}\AgdaSpace{}%
\AgdaSymbol{:}\AgdaSpace{}%
\AgdaBound{𝓤}\AgdaSpace{}%
\AgdaOperator{\AgdaFunction{̇}}\AgdaSpace{}%
\AgdaSymbol{\}}\AgdaSpace{}%
\AgdaSymbol{\{}\AgdaBound{B}\AgdaSpace{}%
\AgdaSymbol{:}\AgdaSpace{}%
\AgdaBound{𝓦}\AgdaSpace{}%
\AgdaOperator{\AgdaFunction{̇}}\AgdaSpace{}%
\AgdaSymbol{\}}\AgdaSpace{}%
\AgdaSymbol{(}\AgdaBound{g}\AgdaSpace{}%
\AgdaSymbol{:}\AgdaSpace{}%
\AgdaBound{A}\AgdaSpace{}%
\AgdaSymbol{→}\AgdaSpace{}%
\AgdaBound{B}\AgdaSymbol{)}\AgdaSpace{}%
\AgdaSymbol{→}%
\>[45]\AgdaBound{𝓤}\AgdaSpace{}%
\AgdaOperator{\AgdaPrimitive{⊔}}\AgdaSpace{}%
\AgdaBound{𝓦}\AgdaSpace{}%
\AgdaOperator{\AgdaFunction{̇}}\<%
\\
%
\>[1]\AgdaFunction{Epic}\AgdaSpace{}%
\AgdaBound{g}\AgdaSpace{}%
\AgdaSymbol{=}\AgdaSpace{}%
\AgdaSymbol{∀}\AgdaSpace{}%
\AgdaBound{y}\AgdaSpace{}%
\AgdaSymbol{→}\AgdaSpace{}%
\AgdaOperator{\AgdaDatatype{Image}}\AgdaSpace{}%
\AgdaBound{g}\AgdaSpace{}%
\AgdaOperator{\AgdaDatatype{∋}}\AgdaSpace{}%
\AgdaBound{y}\<%
\\
\>[0]\<%
\end{code}
\ccpad
We obtain the right-inverse (or pseudo-inverse) of an epic function \af f by applying the function \af{EpicInv} (which we now define) to the function \af f along with a proof, \ab{fE} \as : \af{Epic} \ab f, that \af f is surjective.
\ccpad
\begin{code}%
\>[0][@{}l@{\AgdaIndent{1}}]%
\>[1]\AgdaFunction{EpicInv}\AgdaSpace{}%
\AgdaSymbol{:}%
\>[233I]\AgdaSymbol{\{}\AgdaBound{A}\AgdaSpace{}%
\AgdaSymbol{:}\AgdaSpace{}%
\AgdaBound{𝓤}\AgdaSpace{}%
\AgdaOperator{\AgdaFunction{̇}}\AgdaSpace{}%
\AgdaSymbol{\}}\AgdaSpace{}%
\AgdaSymbol{\{}\AgdaBound{B}\AgdaSpace{}%
\AgdaSymbol{:}\AgdaSpace{}%
\AgdaBound{𝓦}\AgdaSpace{}%
\AgdaOperator{\AgdaFunction{̇}}\AgdaSpace{}%
\AgdaSymbol{\}}\<%
\\
\>[.][@{}l@{}]\<[233I]%
\>[11]\AgdaSymbol{(}\AgdaBound{f}\AgdaSpace{}%
\AgdaSymbol{:}\AgdaSpace{}%
\AgdaBound{A}\AgdaSpace{}%
\AgdaSymbol{→}\AgdaSpace{}%
\AgdaBound{B}\AgdaSymbol{)}\AgdaSpace{}%
\AgdaSymbol{→}\AgdaSpace{}%
\AgdaFunction{Epic}\AgdaSpace{}%
\AgdaBound{f}\<%
\\
%
\>[11]\AgdaComment{-------------------------}\<%
\\
\>[1][@{}l@{\AgdaIndent{0}}]%
\>[2]\AgdaSymbol{→}%
\>[11]\AgdaBound{B}\AgdaSpace{}%
\AgdaSymbol{→}\AgdaSpace{}%
\AgdaBound{A}\<%
\\
%
\>[1]\AgdaFunction{EpicInv}\AgdaSpace{}%
\AgdaBound{f}\AgdaSpace{}%
\AgdaBound{fE}\AgdaSpace{}%
\AgdaBound{b}\AgdaSpace{}%
\AgdaSymbol{=}\AgdaSpace{}%
\AgdaFunction{Inv}\AgdaSpace{}%
\AgdaBound{f}\AgdaSpace{}%
\AgdaBound{b}\AgdaSpace{}%
\AgdaSymbol{(}\AgdaBound{fE}\AgdaSpace{}%
\AgdaBound{b}\AgdaSymbol{)}\<%
\end{code}
\ccpad
The function defined by \af{EpicInv} \ab f \ab{fE} is indeed the right-inverse of \ab f.
\ccpad
\begin{code}%
\>[0][@{}l@{\AgdaIndent{1}}]%
\>[1]\AgdaFunction{EpicInvIsRightInv}\AgdaSpace{}%
\AgdaSymbol{:}%
\>[262I]\AgdaFunction{funext}\AgdaSpace{}%
\AgdaBound{𝓦}\AgdaSpace{}%
\AgdaBound{𝓦}\AgdaSpace{}%
\AgdaSymbol{→}\AgdaSpace{}%
\AgdaSymbol{\{}\AgdaBound{A}\AgdaSpace{}%
\AgdaSymbol{:}\AgdaSpace{}%
\AgdaBound{𝓤}\AgdaSpace{}%
\AgdaOperator{\AgdaFunction{̇}}\AgdaSpace{}%
\AgdaSymbol{\}}\AgdaSpace{}%
\AgdaSymbol{\{}\AgdaBound{B}\AgdaSpace{}%
\AgdaSymbol{:}\AgdaSpace{}%
\AgdaBound{𝓦}\AgdaSpace{}%
\AgdaOperator{\AgdaFunction{̇}}\AgdaSpace{}%
\AgdaSymbol{\}}\<%
\\
\>[.][@{}l@{}]\<[262I]%
\>[21]\AgdaSymbol{(}\AgdaBound{f}\AgdaSpace{}%
\AgdaSymbol{:}\AgdaSpace{}%
\AgdaBound{A}\AgdaSpace{}%
\AgdaSymbol{→}\AgdaSpace{}%
\AgdaBound{B}\AgdaSymbol{)}%
\>[34]\AgdaSymbol{(}\AgdaBound{fE}\AgdaSpace{}%
\AgdaSymbol{:}\AgdaSpace{}%
\AgdaFunction{Epic}\AgdaSpace{}%
\AgdaBound{f}\AgdaSymbol{)}\<%
\\
%
\>[21]\AgdaComment{----------------------------}\<%
\\
\>[1][@{}l@{\AgdaIndent{0}}]%
\>[2]\AgdaSymbol{→}%
\>[21]\AgdaBound{f}\AgdaSpace{}%
\AgdaOperator{\AgdaFunction{∘}}\AgdaSpace{}%
\AgdaSymbol{(}\AgdaFunction{EpicInv}\AgdaSpace{}%
\AgdaBound{f}\AgdaSpace{}%
\AgdaBound{fE}\AgdaSymbol{)}\AgdaSpace{}%
\AgdaOperator{\AgdaDatatype{≡}}\AgdaSpace{}%
\AgdaFunction{𝑖𝑑}\AgdaSpace{}%
\AgdaBound{B}\<%
\\
%
\>[1]\AgdaFunction{EpicInvIsRightInv}\AgdaSpace{}%
\AgdaBound{fe}\AgdaSpace{}%
\AgdaBound{f}\AgdaSpace{}%
\AgdaBound{fE}\AgdaSpace{}%
\AgdaSymbol{=}\AgdaSpace{}%
\AgdaBound{fe}\AgdaSpace{}%
\AgdaSymbol{(λ}\AgdaSpace{}%
\AgdaBound{x}\AgdaSpace{}%
\AgdaSymbol{→}\AgdaSpace{}%
\AgdaFunction{InvIsInv}\AgdaSpace{}%
\AgdaBound{f}\AgdaSpace{}%
\AgdaBound{x}\AgdaSpace{}%
\AgdaSymbol{(}\AgdaBound{fE}\AgdaSpace{}%
\AgdaBound{x}\AgdaSymbol{))}\<%
\end{code}

\subsubsection{Injective functions}\label{injective-functions}
We say that a function \AgdaTyped{g}{A → B} is monic (or injective) if we have a proof of \af{Monic} \ab g, where
\ccpad
\begin{code}%
\>[0][@{}l@{\AgdaIndent{1}}]%
\>[1]\AgdaFunction{Monic}\AgdaSpace{}%
\AgdaSymbol{:}\AgdaSpace{}%
\AgdaSymbol{\{}\AgdaBound{A}\AgdaSpace{}%
\AgdaSymbol{:}\AgdaSpace{}%
\AgdaBound{𝓤}\AgdaSpace{}%
\AgdaOperator{\AgdaFunction{̇}}\AgdaSpace{}%
\AgdaSymbol{\}}\AgdaSpace{}%
\AgdaSymbol{\{}\AgdaBound{B}\AgdaSpace{}%
\AgdaSymbol{:}\AgdaSpace{}%
\AgdaBound{𝓦}\AgdaSpace{}%
\AgdaOperator{\AgdaFunction{̇}}\AgdaSpace{}%
\AgdaSymbol{\}(}\AgdaBound{g}\AgdaSpace{}%
\AgdaSymbol{:}\AgdaSpace{}%
\AgdaBound{A}\AgdaSpace{}%
\AgdaSymbol{→}\AgdaSpace{}%
\AgdaBound{B}\AgdaSymbol{)}\AgdaSpace{}%
\AgdaSymbol{→}\AgdaSpace{}%
\AgdaBound{𝓤}\AgdaSpace{}%
\AgdaOperator{\AgdaPrimitive{⊔}}\AgdaSpace{}%
\AgdaBound{𝓦}\AgdaSpace{}%
\AgdaOperator{\AgdaFunction{̇}}\<%
\\
%
\>[1]\AgdaFunction{Monic}\AgdaSpace{}%
\AgdaBound{g}\AgdaSpace{}%
\AgdaSymbol{=}\AgdaSpace{}%
\AgdaSymbol{∀}\AgdaSpace{}%
\AgdaBound{a₁}\AgdaSpace{}%
\AgdaBound{a₂}\AgdaSpace{}%
\AgdaSymbol{→}\AgdaSpace{}%
\AgdaBound{g}\AgdaSpace{}%
\AgdaBound{a₁}\AgdaSpace{}%
\AgdaOperator{\AgdaDatatype{≡}}\AgdaSpace{}%
\AgdaBound{g}\AgdaSpace{}%
\AgdaBound{a₂}\AgdaSpace{}%
\AgdaSymbol{→}\AgdaSpace{}%
\AgdaBound{a₁}\AgdaSpace{}%
\AgdaOperator{\AgdaDatatype{≡}}\AgdaSpace{}%
\AgdaBound{a₂}\<%
\end{code}
\ccpad
Again, we obtain a pseudo-inverse. Here it is obtained by applying the function \af{MonicInv} to \ab g and a proof that \ab g is monic.
\ccpad
\begin{code}%
\>[0][@{}l@{\AgdaIndent{1}}]%
\>[1]\AgdaComment{--The (pseudo-)inverse of a monic function}\<%
\\
%
\>[1]\AgdaFunction{MonicInv}%
\>[338I]\AgdaSymbol{:}%
\>[339I]\AgdaSymbol{\{}\AgdaBound{A}\AgdaSpace{}%
\AgdaSymbol{:}\AgdaSpace{}%
\AgdaBound{𝓤}\AgdaSpace{}%
\AgdaOperator{\AgdaFunction{̇}}\AgdaSpace{}%
\AgdaSymbol{\}}\AgdaSpace{}%
\AgdaSymbol{\{}\AgdaBound{B}\AgdaSpace{}%
\AgdaSymbol{:}\AgdaSpace{}%
\AgdaBound{𝓦}\AgdaSpace{}%
\AgdaOperator{\AgdaFunction{̇}}\AgdaSpace{}%
\AgdaSymbol{\}}\<%
\\
\>[.][@{}l@{}]\<[339I]%
\>[12]\AgdaSymbol{(}\AgdaBound{f}\AgdaSpace{}%
\AgdaSymbol{:}\AgdaSpace{}%
\AgdaBound{A}\AgdaSpace{}%
\AgdaSymbol{→}\AgdaSpace{}%
\AgdaBound{B}\AgdaSymbol{)}%
\>[25]\AgdaSymbol{→}%
\>[28]\AgdaFunction{Monic}\AgdaSpace{}%
\AgdaBound{f}\<%
\\
\>[338I][@{}l@{\AgdaIndent{0}}]%
\>[11]\AgdaComment{-----------------------------}\<%
\\
\>[1][@{}l@{\AgdaIndent{0}}]%
\>[2]\AgdaSymbol{→}%
\>[12]\AgdaSymbol{(}\AgdaBound{b}\AgdaSpace{}%
\AgdaSymbol{:}\AgdaSpace{}%
\AgdaBound{B}\AgdaSymbol{)}\AgdaSpace{}%
\AgdaSymbol{→}%
\>[23]\AgdaOperator{\AgdaDatatype{Image}}\AgdaSpace{}%
\AgdaBound{f}\AgdaSpace{}%
\AgdaOperator{\AgdaDatatype{∋}}\AgdaSpace{}%
\AgdaBound{b}%
\>[36]\AgdaSymbol{→}%
\>[39]\AgdaBound{A}\<%
\\
%
\\[\AgdaEmptyExtraSkip]%
%
\>[1]\AgdaFunction{MonicInv}\AgdaSpace{}%
\AgdaBound{f}\AgdaSpace{}%
\AgdaSymbol{\AgdaUnderscore{}}\AgdaSpace{}%
\AgdaSymbol{=}\AgdaSpace{}%
\AgdaSymbol{λ}\AgdaSpace{}%
\AgdaBound{b}\AgdaSpace{}%
\AgdaBound{Imf∋b}\AgdaSpace{}%
\AgdaSymbol{→}\AgdaSpace{}%
\AgdaFunction{Inv}\AgdaSpace{}%
\AgdaBound{f}\AgdaSpace{}%
\AgdaBound{b}\AgdaSpace{}%
\AgdaBound{Imf∋b}\<%
\end{code}
\ccpad
The function defined by \AgdaCatchallClause{MonicInv f fM} is the left-inverse of \ab f.
\ccpad
\begin{code}%
\>[0][@{}l@{\AgdaIndent{1}}]%
\>[1]\AgdaComment{--The (psudo-)inverse of a monic is the left inverse.}\<%
\\
%
\>[1]\AgdaFunction{MonicInvIsLeftInv}%
\>[371I]\AgdaSymbol{:}%
\>[372I]\AgdaSymbol{\{}\AgdaBound{A}\AgdaSpace{}%
\AgdaSymbol{:}\AgdaSpace{}%
\AgdaBound{𝓤}\AgdaSpace{}%
\AgdaOperator{\AgdaFunction{̇}}\AgdaSpace{}%
\AgdaSymbol{\}\{}\AgdaBound{B}\AgdaSpace{}%
\AgdaSymbol{:}\AgdaSpace{}%
\AgdaBound{𝓦}\AgdaSpace{}%
\AgdaOperator{\AgdaFunction{̇}}\AgdaSpace{}%
\AgdaSymbol{\}}\<%
\\
\>[.][@{}l@{}]\<[372I]%
\>[21]\AgdaSymbol{(}\AgdaBound{f}\AgdaSpace{}%
\AgdaSymbol{:}\AgdaSpace{}%
\AgdaBound{A}\AgdaSpace{}%
\AgdaSymbol{→}\AgdaSpace{}%
\AgdaBound{B}\AgdaSymbol{)}\AgdaSpace{}%
\AgdaSymbol{(}\AgdaBound{fmonic}\AgdaSpace{}%
\AgdaSymbol{:}\AgdaSpace{}%
\AgdaFunction{Monic}\AgdaSpace{}%
\AgdaBound{f}\AgdaSymbol{)(}\AgdaBound{x}\AgdaSpace{}%
\AgdaSymbol{:}\AgdaSpace{}%
\AgdaBound{A}\AgdaSymbol{)}\<%
\\
\>[371I][@{}l@{\AgdaIndent{0}}]%
\>[20]\AgdaComment{----------------------------------------}\<%
\\
\>[1][@{}l@{\AgdaIndent{0}}]%
\>[3]\AgdaSymbol{→}%
\>[21]\AgdaSymbol{(}\AgdaFunction{MonicInv}\AgdaSpace{}%
\AgdaBound{f}\AgdaSpace{}%
\AgdaBound{fmonic}\AgdaSymbol{)}\AgdaSpace{}%
\AgdaSymbol{(}\AgdaBound{f}\AgdaSpace{}%
\AgdaBound{x}\AgdaSymbol{)}\AgdaSpace{}%
\AgdaSymbol{(}\AgdaInductiveConstructor{im}\AgdaSpace{}%
\AgdaBound{x}\AgdaSymbol{)}\AgdaSpace{}%
\AgdaOperator{\AgdaDatatype{≡}}\AgdaSpace{}%
\AgdaBound{x}\<%
\\
%
\\[\AgdaEmptyExtraSkip]%
%
\>[1]\AgdaFunction{MonicInvIsLeftInv}\AgdaSpace{}%
\AgdaBound{f}\AgdaSpace{}%
\AgdaBound{fmonic}\AgdaSpace{}%
\AgdaBound{x}\AgdaSpace{}%
\AgdaSymbol{=}\AgdaSpace{}%
\AgdaInductiveConstructor{refl}\AgdaSpace{}%
\AgdaSymbol{\AgdaUnderscore{}}\<%
\end{code}

\subsubsection{Bijective functions}\label{bijective-functions}
Finally, bijective functions are defined.
\ccpad
\begin{code}%
\>[0][@{}l@{\AgdaIndent{1}}]%
\>[1]\AgdaFunction{Bijective}\AgdaSpace{}%
\AgdaSymbol{:}\AgdaSpace{}%
\AgdaSymbol{\{}\AgdaBound{A}\AgdaSpace{}%
\AgdaSymbol{:}\AgdaSpace{}%
\AgdaBound{𝓤}\AgdaSpace{}%
\AgdaOperator{\AgdaFunction{̇}}\AgdaSpace{}%
\AgdaSymbol{\}\{}\AgdaBound{B}\AgdaSpace{}%
\AgdaSymbol{:}\AgdaSpace{}%
\AgdaBound{𝓦}\AgdaSpace{}%
\AgdaOperator{\AgdaFunction{̇}}\AgdaSpace{}%
\AgdaSymbol{\}(}\AgdaBound{f}\AgdaSpace{}%
\AgdaSymbol{:}\AgdaSpace{}%
\AgdaBound{A}\AgdaSpace{}%
\AgdaSymbol{→}\AgdaSpace{}%
\AgdaBound{B}\AgdaSymbol{)}\AgdaSpace{}%
\AgdaSymbol{→}\AgdaSpace{}%
\AgdaBound{𝓤}\AgdaSpace{}%
\AgdaOperator{\AgdaPrimitive{⊔}}\AgdaSpace{}%
\AgdaBound{𝓦}\AgdaSpace{}%
\AgdaOperator{\AgdaFunction{̇}}\<%
\\
%
\>[1]\AgdaFunction{Bijective}\AgdaSpace{}%
\AgdaBound{f}\AgdaSpace{}%
\AgdaSymbol{=}\AgdaSpace{}%
\AgdaFunction{Epic}\AgdaSpace{}%
\AgdaBound{f}\AgdaSpace{}%
\AgdaOperator{\AgdaFunction{×}}\AgdaSpace{}%
\AgdaFunction{Monic}\AgdaSpace{}%
\AgdaBound{f}\<%
\\
%
\\[\AgdaEmptyExtraSkip]%
%
\>[1]\AgdaFunction{Inverse}\AgdaSpace{}%
\AgdaSymbol{:}%
\>[432I]\AgdaSymbol{\{}\AgdaBound{A}\AgdaSpace{}%
\AgdaSymbol{:}\AgdaSpace{}%
\AgdaBound{𝓤}\AgdaSpace{}%
\AgdaOperator{\AgdaFunction{̇}}\AgdaSpace{}%
\AgdaSymbol{\}}\AgdaSpace{}%
\AgdaSymbol{\{}\AgdaBound{B}\AgdaSpace{}%
\AgdaSymbol{:}\AgdaSpace{}%
\AgdaBound{𝓦}\AgdaSpace{}%
\AgdaOperator{\AgdaFunction{̇}}\AgdaSpace{}%
\AgdaSymbol{\}}\<%
\\
\>[432I][@{}l@{\AgdaIndent{0}}]%
\>[12]\AgdaSymbol{(}\AgdaBound{f}\AgdaSpace{}%
\AgdaSymbol{:}\AgdaSpace{}%
\AgdaBound{A}\AgdaSpace{}%
\AgdaSymbol{→}\AgdaSpace{}%
\AgdaBound{B}\AgdaSymbol{)}\AgdaSpace{}%
\AgdaSymbol{→}\AgdaSpace{}%
\AgdaFunction{Bijective}\AgdaSpace{}%
\AgdaBound{f}\<%
\\
%
\>[12]\AgdaComment{-------------------------}\<%
\\
\>[1][@{}l@{\AgdaIndent{0}}]%
\>[2]\AgdaSymbol{→}%
\>[12]\AgdaBound{B}\AgdaSpace{}%
\AgdaSymbol{→}\AgdaSpace{}%
\AgdaBound{A}\<%
\\
%
\>[1]\AgdaFunction{Inverse}\AgdaSpace{}%
\AgdaBound{f}\AgdaSpace{}%
\AgdaBound{fbi}\AgdaSpace{}%
\AgdaBound{b}\AgdaSpace{}%
\AgdaSymbol{=}\AgdaSpace{}%
\AgdaFunction{Inv}\AgdaSpace{}%
\AgdaBound{f}\AgdaSpace{}%
\AgdaBound{b}\AgdaSpace{}%
\AgdaSymbol{(}\AgdaFunction{fst}\AgdaSymbol{(}\AgdaSpace{}%
\AgdaBound{fbi}\AgdaSpace{}%
\AgdaSymbol{)}\AgdaSpace{}%
\AgdaBound{b}\AgdaSymbol{)}\<%
\end{code}
%% OMITTING NEUTRAL ELEMENTS
% \subsubsection{Neutral elements}\label{neutral-elements}
The next three lemmas appeared in the \texttt{UF-Base} and \texttt{UF-Equiv} modules which were (at one time) part of Matin Escsardo's UF Agda repository.
\ccpad
\begin{code}%
\>[0]\<%
\\
\>[0]\AgdaFunction{refl-left-neutral}\AgdaSpace{}%
\AgdaSymbol{:}\AgdaSpace{}%
\AgdaSymbol{\{}\AgdaBound{𝓤}\AgdaSpace{}%
\AgdaSymbol{:}\AgdaSpace{}%
\AgdaPostulate{Universe}\AgdaSymbol{\}}\AgdaSpace{}%
\AgdaSymbol{\{}\AgdaBound{X}\AgdaSpace{}%
\AgdaSymbol{:}\AgdaSpace{}%
\AgdaBound{𝓤}\AgdaSpace{}%
\AgdaOperator{\AgdaFunction{̇}}\AgdaSpace{}%
\AgdaSymbol{\}}\AgdaSpace{}%
\AgdaSymbol{\{}\AgdaBound{x}\AgdaSpace{}%
\AgdaBound{y}\AgdaSpace{}%
\AgdaSymbol{:}\AgdaSpace{}%
\AgdaBound{X}\AgdaSymbol{\}}\AgdaSpace{}%
\AgdaSymbol{(}\AgdaBound{p}\AgdaSpace{}%
\AgdaSymbol{:}\AgdaSpace{}%
\AgdaBound{x}\AgdaSpace{}%
\AgdaOperator{\AgdaDatatype{≡}}\AgdaSpace{}%
\AgdaBound{y}\AgdaSymbol{)}\AgdaSpace{}%
\AgdaSymbol{→}\AgdaSpace{}%
\AgdaSymbol{(}\AgdaInductiveConstructor{refl}\AgdaSpace{}%
\AgdaSymbol{\AgdaUnderscore{})}\AgdaSpace{}%
\AgdaOperator{\AgdaFunction{∙}}\AgdaSpace{}%
\AgdaBound{p}\AgdaSpace{}%
\AgdaOperator{\AgdaDatatype{≡}}\AgdaSpace{}%
\AgdaBound{p}\<%
\\
\>[0]\AgdaFunction{refl-left-neutral}\AgdaSpace{}%
\AgdaSymbol{(}\AgdaInductiveConstructor{refl}\AgdaSpace{}%
\AgdaSymbol{\AgdaUnderscore{})}\AgdaSpace{}%
\AgdaSymbol{=}\AgdaSpace{}%
\AgdaInductiveConstructor{refl}\AgdaSpace{}%
\AgdaSymbol{\AgdaUnderscore{}}\<%
\\
%
\\[\AgdaEmptyExtraSkip]%
\>[0]\AgdaFunction{refl-right-neutral}\AgdaSpace{}%
\AgdaSymbol{:}\AgdaSpace{}%
\AgdaSymbol{\{}\AgdaBound{𝓤}\AgdaSpace{}%
\AgdaSymbol{:}\AgdaSpace{}%
\AgdaPostulate{Universe}\AgdaSymbol{\}\{}\AgdaBound{X}\AgdaSpace{}%
\AgdaSymbol{:}\AgdaSpace{}%
\AgdaBound{𝓤}\AgdaSpace{}%
\AgdaOperator{\AgdaFunction{̇}}\AgdaSpace{}%
\AgdaSymbol{\}}\AgdaSpace{}%
\AgdaSymbol{\{}\AgdaBound{x}\AgdaSpace{}%
\AgdaBound{y}\AgdaSpace{}%
\AgdaSymbol{:}\AgdaSpace{}%
\AgdaBound{X}\AgdaSymbol{\}}\AgdaSpace{}%
\AgdaSymbol{(}\AgdaBound{p}\AgdaSpace{}%
\AgdaSymbol{:}\AgdaSpace{}%
\AgdaBound{x}\AgdaSpace{}%
\AgdaOperator{\AgdaDatatype{≡}}\AgdaSpace{}%
\AgdaBound{y}\AgdaSymbol{)}\AgdaSpace{}%
\AgdaSymbol{→}\AgdaSpace{}%
\AgdaBound{p}\AgdaSpace{}%
\AgdaOperator{\AgdaDatatype{≡}}\AgdaSpace{}%
\AgdaBound{p}\AgdaSpace{}%
\AgdaOperator{\AgdaFunction{∙}}\AgdaSpace{}%
\AgdaSymbol{(}\AgdaInductiveConstructor{refl}\AgdaSpace{}%
\AgdaSymbol{\AgdaUnderscore{})}\<%
\\
\>[0]\AgdaFunction{refl-right-neutral}\AgdaSpace{}%
\AgdaBound{p}\AgdaSpace{}%
\AgdaSymbol{=}\AgdaSpace{}%
\AgdaInductiveConstructor{refl}\AgdaSpace{}%
\AgdaSymbol{\AgdaUnderscore{}}\<%
\\
%
\\[\AgdaEmptyExtraSkip]%
\>[0]\AgdaFunction{identifications-in-fibers}\AgdaSpace{}%
\AgdaSymbol{:}%
\>[521I]\AgdaSymbol{\{}\AgdaBound{𝓤}\AgdaSpace{}%
\AgdaSymbol{:}\AgdaSpace{}%
\AgdaPostulate{Universe}\AgdaSymbol{\}}\AgdaSpace{}%
\AgdaSymbol{\{}\AgdaBound{X}\AgdaSpace{}%
\AgdaSymbol{:}\AgdaSpace{}%
\AgdaBound{𝓤}\AgdaSpace{}%
\AgdaOperator{\AgdaFunction{̇}}\AgdaSpace{}%
\AgdaSymbol{\}}\AgdaSpace{}%
\AgdaSymbol{\{}\AgdaBound{Y}\AgdaSpace{}%
\AgdaSymbol{:}\AgdaSpace{}%
\AgdaGeneralizable{𝓥}\AgdaSpace{}%
\AgdaOperator{\AgdaFunction{̇}}\AgdaSpace{}%
\AgdaSymbol{\}}\AgdaSpace{}%
\AgdaSymbol{(}\AgdaBound{f}\AgdaSpace{}%
\AgdaSymbol{:}\AgdaSpace{}%
\AgdaBound{X}\AgdaSpace{}%
\AgdaSymbol{→}\AgdaSpace{}%
\AgdaBound{Y}\AgdaSymbol{)}\<%
\\
\>[.][@{}l@{}]\<[521I]%
\>[28]\AgdaSymbol{(}\AgdaBound{y}\AgdaSpace{}%
\AgdaSymbol{:}\AgdaSpace{}%
\AgdaBound{Y}\AgdaSymbol{)}\AgdaSpace{}%
\AgdaSymbol{(}\AgdaBound{x}\AgdaSpace{}%
\AgdaBound{x'}\AgdaSpace{}%
\AgdaSymbol{:}\AgdaSpace{}%
\AgdaBound{X}\AgdaSymbol{)}\AgdaSpace{}%
\AgdaSymbol{(}\AgdaBound{p}\AgdaSpace{}%
\AgdaSymbol{:}\AgdaSpace{}%
\AgdaBound{f}\AgdaSpace{}%
\AgdaBound{x}\AgdaSpace{}%
\AgdaOperator{\AgdaDatatype{≡}}\AgdaSpace{}%
\AgdaBound{y}\AgdaSymbol{)}\AgdaSpace{}%
\AgdaSymbol{(}\AgdaBound{p'}\AgdaSpace{}%
\AgdaSymbol{:}\AgdaSpace{}%
\AgdaBound{f}\AgdaSpace{}%
\AgdaBound{x'}\AgdaSpace{}%
\AgdaOperator{\AgdaDatatype{≡}}\AgdaSpace{}%
\AgdaBound{y}\AgdaSymbol{)}\<%
\\
\>[0][@{}l@{\AgdaIndent{0}}]%
\>[1]\AgdaSymbol{→}%
\>[28]\AgdaSymbol{(}\AgdaFunction{Σ}\AgdaSpace{}%
\AgdaBound{γ}\AgdaSpace{}%
\AgdaFunction{꞉}\AgdaSpace{}%
\AgdaBound{x}\AgdaSpace{}%
\AgdaOperator{\AgdaDatatype{≡}}\AgdaSpace{}%
\AgdaBound{x'}\AgdaSpace{}%
\AgdaFunction{,}\AgdaSpace{}%
\AgdaFunction{ap}\AgdaSpace{}%
\AgdaBound{f}\AgdaSpace{}%
\AgdaBound{γ}\AgdaSpace{}%
\AgdaOperator{\AgdaFunction{∙}}\AgdaSpace{}%
\AgdaBound{p'}\AgdaSpace{}%
\AgdaOperator{\AgdaDatatype{≡}}\AgdaSpace{}%
\AgdaBound{p}\AgdaSymbol{)}\AgdaSpace{}%
\AgdaSymbol{→}\AgdaSpace{}%
\AgdaSymbol{(}\AgdaBound{x}\AgdaSpace{}%
\AgdaOperator{\AgdaInductiveConstructor{,}}\AgdaSpace{}%
\AgdaBound{p}\AgdaSymbol{)}\AgdaSpace{}%
\AgdaOperator{\AgdaDatatype{≡}}\AgdaSpace{}%
\AgdaSymbol{(}\AgdaBound{x'}\AgdaSpace{}%
\AgdaOperator{\AgdaInductiveConstructor{,}}\AgdaSpace{}%
\AgdaBound{p'}\AgdaSymbol{)}\<%
\\
\>[0]\AgdaFunction{identifications-in-fibers}\AgdaSpace{}%
\AgdaBound{f}\AgdaSpace{}%
\AgdaDottedPattern{\AgdaSymbol{.(}}\AgdaDottedPattern{\AgdaBound{f}}\AgdaSpace{}%
\AgdaDottedPattern{\AgdaBound{x}}\AgdaDottedPattern{\AgdaSymbol{)}}\AgdaSpace{}%
\AgdaBound{x}\AgdaSpace{}%
\AgdaDottedPattern{\AgdaSymbol{.}}\AgdaDottedPattern{\AgdaBound{x}}\AgdaSpace{}%
\AgdaInductiveConstructor{𝓇ℯ𝒻𝓁}\AgdaSpace{}%
\AgdaBound{p'}\AgdaSpace{}%
\AgdaSymbol{(}\AgdaInductiveConstructor{𝓇ℯ𝒻𝓁}\AgdaSpace{}%
\AgdaOperator{\AgdaInductiveConstructor{,}}\AgdaSpace{}%
\AgdaBound{r}\AgdaSymbol{)}\AgdaSpace{}%
\AgdaSymbol{=}\AgdaSpace{}%
\AgdaFunction{g}\<%
\\
\>[0][@{}l@{\AgdaIndent{0}}]%
\>[1]\AgdaKeyword{where}\<%
\\
\>[1][@{}l@{\AgdaIndent{0}}]%
\>[2]\AgdaFunction{g}\AgdaSpace{}%
\AgdaSymbol{:}\AgdaSpace{}%
\AgdaBound{x}\AgdaSpace{}%
\AgdaOperator{\AgdaInductiveConstructor{,}}\AgdaSpace{}%
\AgdaInductiveConstructor{𝓇ℯ𝒻𝓁}\AgdaSpace{}%
\AgdaOperator{\AgdaDatatype{≡}}\AgdaSpace{}%
\AgdaBound{x}\AgdaSpace{}%
\AgdaOperator{\AgdaInductiveConstructor{,}}\AgdaSpace{}%
\AgdaBound{p'}\<%
\\
%
\>[2]\AgdaFunction{g}\AgdaSpace{}%
\AgdaSymbol{=}\AgdaSpace{}%
\AgdaFunction{ap}\AgdaSpace{}%
\AgdaSymbol{(λ}\AgdaSpace{}%
\AgdaBound{-}\AgdaSpace{}%
\AgdaSymbol{→}\AgdaSpace{}%
\AgdaSymbol{(}\AgdaBound{x}\AgdaSpace{}%
\AgdaOperator{\AgdaInductiveConstructor{,}}\AgdaSpace{}%
\AgdaBound{-}\AgdaSymbol{))}\AgdaSpace{}%
\AgdaSymbol{(}\AgdaBound{r}\AgdaSpace{}%
\AgdaOperator{\AgdaFunction{⁻¹}}\AgdaSpace{}%
\AgdaOperator{\AgdaFunction{∙}}\AgdaSpace{}%
\AgdaFunction{refl-left-neutral}\AgdaSpace{}%
\AgdaSymbol{\AgdaUnderscore{})}\<%
\\
\>[0]\<%
\end{code}


\subsubsection{Injective functions are set embeddings}\label{injective-functions-are-set-embeddings}
This is the first point at which \href{UALib.Preface.html\#truncation}{truncation} comes into play. An
\href{https://www.cs.bham.ac.uk/~mhe/HoTT-UF-in-Agda-Lecture-Notes/HoTT-UF-Agda.html\#embeddings}{embedding} is defined in the \TypeTopology library as follows:
\ccpad
\begin{code}
\>[0]\AgdaFunction{is-embedding}\AgdaSpace{}%
\AgdaSymbol{:}\AgdaSpace{}%
\AgdaSymbol{\{}\AgdaBound{X}\AgdaSpace{}%
\AgdaSymbol{:}\AgdaSpace{}%
\AgdaGeneralizable{𝓤}\AgdaSpace{}%
\AgdaOperator{\AgdaFunction{̇}}\AgdaSpace{}%
\AgdaSymbol{\}}\AgdaSpace{}%
\AgdaSymbol{\{}\AgdaBound{Y}\AgdaSpace{}%
\AgdaSymbol{:}\AgdaSpace{}%
\AgdaGeneralizable{𝓥}\AgdaSpace{}%
\AgdaOperator{\AgdaFunction{̇}}\AgdaSpace{}%
\AgdaSymbol{\}}\AgdaSpace{}%
\AgdaSymbol{→}\AgdaSpace{}%
\AgdaSymbol{(}\AgdaBound{X}\AgdaSpace{}%
\AgdaSymbol{→}\AgdaSpace{}%
\AgdaBound{Y}\AgdaSymbol{)}\AgdaSpace{}%
\AgdaSymbol{→}\AgdaSpace{}%
\AgdaGeneralizable{𝓤}\AgdaSpace{}%
\AgdaOperator{\AgdaPrimitive{⊔}}\AgdaSpace{}%
\AgdaGeneralizable{𝓥}\AgdaSpace{}%
\AgdaOperator{\AgdaFunction{̇}}\<%
\\
\>[0]\AgdaFunction{is-embedding}\AgdaSpace{}%
\AgdaBound{f}\AgdaSpace{}%
\AgdaSymbol{=}\AgdaSpace{}%
\AgdaSymbol{(}\AgdaBound{y}\AgdaSpace{}%
\AgdaSymbol{:}\AgdaSpace{}%
\AgdaFunction{codomain}\AgdaSpace{}%
\AgdaBound{f}\AgdaSymbol{)}\AgdaSpace{}%
\AgdaSymbol{→}\AgdaSpace{}%
\AgdaFunction{is-subsingleton}\AgdaSpace{}%
\AgdaSymbol{(}\AgdaFunction{fiber}\AgdaSpace{}%
\AgdaBound{f}\AgdaSpace{}%
\AgdaBound{y}\AgdaSymbol{)}\<%
\end{code}
\ccpad
where
\ccpad
\begin{code}
\>[0]\AgdaFunction{is-subsingleton}\AgdaSpace{}%
\AgdaSymbol{:}\AgdaSpace{}%
\AgdaGeneralizable{𝓤}\AgdaSpace{}%
\AgdaOperator{\AgdaFunction{̇}}\AgdaSpace{}%
\AgdaSymbol{→}\AgdaSpace{}%
\AgdaGeneralizable{𝓤}\AgdaSpace{}%
\AgdaOperator{\AgdaFunction{̇}}\<%
\\
\>[0]\AgdaFunction{is-subsingleton}\AgdaSpace{}%
\AgdaBound{X}\AgdaSpace{}%
\AgdaSymbol{=}\AgdaSpace{}%
\AgdaSymbol{(}\AgdaBound{x}\AgdaSpace{}%
\AgdaBound{y}\AgdaSpace{}%
\AgdaSymbol{:}\AgdaSpace{}%
\AgdaBound{X}\AgdaSymbol{)}\AgdaSpace{}%
\AgdaSymbol{→}\AgdaSpace{}%
\AgdaBound{x}\AgdaSpace{}%
\AgdaOperator{\AgdaDatatype{≡}}\AgdaSpace{}%
\AgdaBound{y}\<%
\end{code}
\ccpad
and
\ccpad
\begin{code}
\>[0]\AgdaFunction{fiber}\AgdaSpace{}%
\AgdaSymbol{:}\AgdaSpace{}%
\AgdaSymbol{\{}\AgdaBound{X}\AgdaSpace{}%
\AgdaSymbol{:}\AgdaSpace{}%
\AgdaGeneralizable{𝓤}\AgdaSpace{}%
\AgdaOperator{\AgdaFunction{̇}}\AgdaSpace{}%
\AgdaSymbol{\}}\AgdaSpace{}%
\AgdaSymbol{\{}\AgdaBound{Y}\AgdaSpace{}%
\AgdaSymbol{:}\AgdaSpace{}%
\AgdaGeneralizable{𝓥}\AgdaSpace{}%
\AgdaOperator{\AgdaFunction{̇}}\AgdaSpace{}%
\AgdaSymbol{\}}\AgdaSpace{}%
\AgdaSymbol{(}\AgdaBound{f}\AgdaSpace{}%
\AgdaSymbol{:}\AgdaSpace{}%
\AgdaBound{X}\AgdaSpace{}%
\AgdaSymbol{→}\AgdaSpace{}%
\AgdaBound{Y}\AgdaSymbol{)}\AgdaSpace{}%
\AgdaSymbol{→}\AgdaSpace{}%
\AgdaBound{Y}\AgdaSpace{}%
\AgdaSymbol{→}\AgdaSpace{}%
\AgdaGeneralizable{𝓤}\AgdaSpace{}%
\AgdaOperator{\AgdaPrimitive{⊔}}\AgdaSpace{}%
\AgdaGeneralizable{𝓥}\AgdaSpace{}%
\AgdaOperator{\AgdaFunction{̇}}\<%
\\
\>[0]\AgdaFunction{fiber}\AgdaSpace{}%
\AgdaBound{f}\AgdaSpace{}%
\AgdaBound{y}\AgdaSpace{}%
\AgdaSymbol{=}\AgdaSpace{}%
\AgdaFunction{Σ}\AgdaSpace{}%
\AgdaBound{x}\AgdaSpace{}%
\AgdaFunction{꞉}\AgdaSpace{}%
\AgdaFunction{domain}\AgdaSpace{}%
\AgdaBound{f}\AgdaSpace{}%
\AgdaFunction{,}\AgdaSpace{}%
\AgdaBound{f}\AgdaSpace{}%
\AgdaBound{x}\AgdaSpace{}%
\AgdaOperator{\AgdaDatatype{≡}}\AgdaSpace{}%
\AgdaBound{y}\<%
\end{code}
\ccpad
This is a natural way to represent what we usually mean in mathematics by embedding. It does not correspond simply to an injective map. However, if we assume that the codomain type, \ab B, is a \emph{set} (i.e., has \emph{unique identity proofs}), then we can prove that a injective (i.e., \emph{monic}) function into \ab B is an embedding as follows:
\ccpad
\begin{code}%
\>[0]\AgdaKeyword{module}\AgdaSpace{}%
\AgdaModule{\AgdaUnderscore{}}\AgdaSpace{}%
\AgdaSymbol{\{}\AgdaBound{𝓤}\AgdaSpace{}%
\AgdaBound{𝓦}\AgdaSpace{}%
\AgdaSymbol{:}\AgdaSpace{}%
\AgdaPostulate{Universe}\AgdaSymbol{\}}\AgdaSpace{}%
\AgdaKeyword{where}\<%
\\
%
%% \>[1]\AgdaFunction{MonicInvIsLeftInv}%
%% \>[371I]\AgdaSymbol{:}%
%% \>[372I]\AgdaSymbol{\{}\AgdaBound{A}\AgdaSpace{}%
%% \AgdaSymbol{:}\AgdaSpace{}%
%% \AgdaBound{𝓤}\AgdaSpace{}%
%% \AgdaOperator{\AgdaFunction{̇}}\AgdaSpace{}%
%% \AgdaSymbol{\}\{}\AgdaBound{B}\AgdaSpace{}%
%% \AgdaSymbol{:}\AgdaSpace{}%
%% \AgdaBound{𝓦}\AgdaSpace{}%
%% \AgdaOperator{\AgdaFunction{̇}}\AgdaSpace{}%
%% \AgdaSymbol{\}}\<%
%% \\
%% \>[.][@{}l@{}]\<[372I]%
%% \>[21]\AgdaSymbol{(}\AgdaBound{f}\AgdaSpace{}%
%% \AgdaSymbol{:}\AgdaSpace{}%
%% \AgdaBound{A}\AgdaSpace{}%
%% \AgdaSymbol{→}\AgdaSpace{}%
%% \AgdaBound{B}\AgdaSymbol{)}\AgdaSpace{}%
%% \AgdaSymbol{(}\AgdaBound{fmonic}\AgdaSpace{}%
%% \AgdaSymbol{:}\AgdaSpace{}%
%% \AgdaFunction{Monic}\AgdaSpace{}%
%% \AgdaBound{f}\AgdaSymbol{)(}\AgdaBound{x}\AgdaSpace{}%
%% \AgdaSymbol{:}\AgdaSpace{}%
%% \AgdaBound{A}\AgdaSymbol{)}\<%
%% \\
%% \>[371I][@{}l@{\AgdaIndent{0}}]%
%% \>[20]\AgdaComment{----------------------------------------}\<%
%% \\
%% \>[1][@{}l@{\AgdaIndent{0}}]%
%% \>[3]\AgdaSymbol{→}%
%% \>[21]\AgdaSymbol{(}\AgdaFunction{MonicInv}\AgdaSpace{}%
%% \AgdaBound{f}\AgdaSpace{}%
%% \AgdaBound{fmonic}\AgdaSymbol{)}\AgdaSpace{}%
%% \AgdaSymbol{(}\AgdaBound{f}\AgdaSpace{}%
%% \AgdaBound{x}\AgdaSymbol{)}\AgdaSpace{}%
%% \AgdaSymbol{(}\AgdaInductiveConstructor{im}\AgdaSpace{}%
%% \AgdaBound{x}\AgdaSymbol{)}\AgdaSpace{}%
%% \AgdaOperator{\AgdaDatatype{≡}}\AgdaSpace{}%
%% \AgdaBound{x}\<%
%% \\
\\[\AgdaEmptyExtraSkip]%
\>[0][@{}l@{\AgdaIndent{0}}]%
\>[1]\AgdaFunction{monic-into-set-is-embedding}
\>[371I]\AgdaSymbol{:}%
\>[372I]\AgdaSymbol{\{}\AgdaBound{A}\AgdaSpace{}%
\AgdaSymbol{:}\AgdaSpace{}%
\AgdaSymbol{\{}\AgdaBound{A}\AgdaSpace{}%
\AgdaSymbol{:}\AgdaSpace{}%
\AgdaBound{𝓤}\AgdaSpace{}%
\AgdaOperator{\AgdaFunction{̇}}\AgdaSymbol{\}\{}\AgdaBound{B}\AgdaSpace{}%
\AgdaSymbol{:}\AgdaSpace{}%
\AgdaBound{𝓦}\AgdaSpace{}%
\AgdaOperator{\AgdaFunction{̇}}\AgdaSymbol{\}}\AgdaSpace{}%
\AgdaSymbol{→}\AgdaSpace{}%
\AgdaFunction{is-set}\AgdaSpace{}%
\AgdaBound{B}\<%
\\
\>[1][@{}l@{\AgdaIndent{0}}]%
\>[2]\AgdaSymbol{→}%
\>[.][@{}l@{}]\<[372I]%
\>[21]\AgdaSymbol{(}\AgdaBound{f}\AgdaSpace{}%
\AgdaSymbol{:}\AgdaSpace{}%
\AgdaBound{A}\AgdaSpace{}%
\AgdaSymbol{→}\AgdaSpace{}%
\AgdaBound{B}\AgdaSymbol{)}%
\>[44]\AgdaSymbol{→}%
\>[47]\AgdaFunction{Monic}\AgdaSpace{}%
\AgdaBound{f}\<%
\\
\>[2][@{}l@{\AgdaIndent{0}}]%
\>[21]\AgdaComment{---------------------------}\<%
\\
%
\>[2]\AgdaSymbol{→}%
\>[21]\AgdaFunction{is-embedding}\AgdaSpace{}%
\AgdaBound{f}\<%
\\
%
\\[\AgdaEmptyExtraSkip]%
%
\>[1]\AgdaFunction{monic-into-set-is-embedding}\AgdaSpace{}%
\AgdaSymbol{\{}\AgdaBound{A}\AgdaSymbol{\}}\AgdaSpace{}%
\AgdaBound{Bset}\AgdaSpace{}%
\AgdaBound{f}\AgdaSpace{}%
\AgdaBound{fmon}\AgdaSpace{}%
\AgdaBound{b}\AgdaSpace{}%
\AgdaSymbol{(}\AgdaBound{a}\AgdaSpace{}%
\AgdaOperator{\AgdaInductiveConstructor{,}}\AgdaSpace{}%
\AgdaBound{fa≡b}\AgdaSymbol{)}\AgdaSpace{}%
\AgdaSymbol{(}\AgdaBound{a'}\AgdaSpace{}%
\AgdaOperator{\AgdaInductiveConstructor{,}}\AgdaSpace{}%
\AgdaBound{fa'≡b}\AgdaSymbol{)}\AgdaSpace{}%
\AgdaSymbol{=}\AgdaSpace{}%
\AgdaFunction{γ}\<%
\\
\>[1][@{}l@{\AgdaIndent{0}}]%
\>[2]\AgdaKeyword{where}\<%
\\
\>[2][@{}l@{\AgdaIndent{0}}]%
\>[3]\AgdaFunction{faa'}\AgdaSpace{}%
\AgdaSymbol{:}\AgdaSpace{}%
\AgdaBound{f}\AgdaSpace{}%
\AgdaBound{a}\AgdaSpace{}%
\AgdaOperator{\AgdaDatatype{≡}}\AgdaSpace{}%
\AgdaBound{f}\AgdaSpace{}%
\AgdaBound{a'}\<%
\\
%
\>[3]\AgdaFunction{faa'}\AgdaSpace{}%
\AgdaSymbol{=}\AgdaSpace{}%
\AgdaFunction{≡-Trans}\AgdaSpace{}%
\AgdaSymbol{(}\AgdaBound{f}\AgdaSpace{}%
\AgdaBound{a}\AgdaSymbol{)}\AgdaSpace{}%
\AgdaSymbol{(}\AgdaBound{f}\AgdaSpace{}%
\AgdaBound{a'}\AgdaSymbol{)}\AgdaSpace{}%
\AgdaBound{fa≡b}\AgdaSpace{}%
\AgdaSymbol{(}\AgdaBound{fa'≡b}\AgdaSpace{}%
\AgdaOperator{\AgdaFunction{⁻¹}}\AgdaSymbol{)}\<%
\\
%
\\[\AgdaEmptyExtraSkip]%
%
\>[3]\AgdaFunction{aa'}\AgdaSpace{}%
\AgdaSymbol{:}\AgdaSpace{}%
\AgdaBound{a}\AgdaSpace{}%
\AgdaOperator{\AgdaDatatype{≡}}\AgdaSpace{}%
\AgdaBound{a'}\<%
\\
%
\>[3]\AgdaFunction{aa'}\AgdaSpace{}%
\AgdaSymbol{=}\AgdaSpace{}%
\AgdaBound{fmon}\AgdaSpace{}%
\AgdaBound{a}\AgdaSpace{}%
\AgdaBound{a'}\AgdaSpace{}%
\AgdaFunction{faa'}\<%
\\
%
\\[\AgdaEmptyExtraSkip]%
%
\>[3]\AgdaFunction{𝒜}\AgdaSpace{}%
\AgdaSymbol{:}\AgdaSpace{}%
\AgdaBound{A}\AgdaSpace{}%
\AgdaSymbol{→}\AgdaSpace{}%
\AgdaBound{𝓦}\AgdaSpace{}%
\AgdaOperator{\AgdaFunction{̇}}\<%
\\
%
\>[3]\AgdaFunction{𝒜}\AgdaSpace{}%
\AgdaBound{a}\AgdaSpace{}%
\AgdaSymbol{=}\AgdaSpace{}%
\AgdaBound{f}\AgdaSpace{}%
\AgdaBound{a}\AgdaSpace{}%
\AgdaOperator{\AgdaDatatype{≡}}\AgdaSpace{}%
\AgdaBound{b}\<%
\\
%
\\[\AgdaEmptyExtraSkip]%
%
\>[3]\AgdaFunction{arg1}\AgdaSpace{}%
\AgdaSymbol{:}\AgdaSpace{}%
\AgdaFunction{Σ}\AgdaSpace{}%
\AgdaBound{p}\AgdaSpace{}%
\AgdaFunction{꞉}\AgdaSpace{}%
\AgdaSymbol{(}\AgdaBound{a}\AgdaSpace{}%
\AgdaOperator{\AgdaDatatype{≡}}\AgdaSpace{}%
\AgdaBound{a'}\AgdaSymbol{)}\AgdaSpace{}%
\AgdaFunction{,}\AgdaSpace{}%
\AgdaSymbol{(}\AgdaFunction{transport}\AgdaSpace{}%
\AgdaFunction{𝒜}\AgdaSpace{}%
\AgdaBound{p}\AgdaSpace{}%
\AgdaBound{fa≡b}\AgdaSymbol{)}\AgdaSpace{}%
\AgdaOperator{\AgdaDatatype{≡}}\AgdaSpace{}%
\AgdaBound{fa'≡b}\<%
\\
%
\>[3]\AgdaFunction{arg1}\AgdaSpace{}%
\AgdaSymbol{=}\AgdaSpace{}%
\AgdaFunction{aa'}\AgdaSpace{}%
\AgdaOperator{\AgdaInductiveConstructor{,}}\AgdaSpace{}%
\AgdaBound{Bset}\AgdaSpace{}%
\AgdaSymbol{(}\AgdaBound{f}\AgdaSpace{}%
\AgdaBound{a'}\AgdaSymbol{)}\AgdaSpace{}%
\AgdaBound{b}\AgdaSpace{}%
\AgdaSymbol{(}\AgdaFunction{transport}\AgdaSpace{}%
\AgdaFunction{𝒜}\AgdaSpace{}%
\AgdaFunction{aa'}\AgdaSpace{}%
\AgdaBound{fa≡b}\AgdaSymbol{)}\AgdaSpace{}%
\AgdaBound{fa'≡b}\<%
\\
%
\\[\AgdaEmptyExtraSkip]%
%
\>[3]\AgdaFunction{γ}\AgdaSpace{}%
\AgdaSymbol{:}\AgdaSpace{}%
\AgdaBound{a}\AgdaSpace{}%
\AgdaOperator{\AgdaInductiveConstructor{,}}\AgdaSpace{}%
\AgdaBound{fa≡b}\AgdaSpace{}%
\AgdaOperator{\AgdaDatatype{≡}}\AgdaSpace{}%
\AgdaBound{a'}\AgdaSpace{}%
\AgdaOperator{\AgdaInductiveConstructor{,}}\AgdaSpace{}%
\AgdaBound{fa'≡b}\<%
\\
%
\>[3]\AgdaFunction{γ}\AgdaSpace{}%
\AgdaSymbol{=}\AgdaSpace{}%
\AgdaFunction{to-Σ-≡}\AgdaSpace{}%
\AgdaFunction{arg1}\<%
\\
\>[0]\<%
\end{code}

Of course, invertible maps are embeddings.

\begin{code}%
\>[0]\<%
\\
\>[0][@{}l@{\AgdaIndent{1}}]%
\>[1]\AgdaFunction{invertibles-are-embeddings}\AgdaSpace{}%
\>[371I]\AgdaSymbol{:}%
\>[372I]\AgdaSymbol{\{}\AgdaBound{X}\AgdaSpace{}%
\AgdaSymbol{:}\AgdaSpace{}%
\AgdaBound{𝓤}\AgdaSpace{}%
\AgdaOperator{\AgdaFunction{̇}}\AgdaSpace{}%
\AgdaSymbol{\}}\AgdaSpace{}%
\AgdaSymbol{\{}\AgdaBound{Y}\AgdaSpace{}%
\AgdaSymbol{:}\AgdaSpace{}%
\AgdaBound{𝓦}\AgdaSpace{}%
\AgdaOperator{\AgdaFunction{̇}}\AgdaSpace{}%
\AgdaSymbol{\}(}\AgdaBound{f}\AgdaSpace{}%
\AgdaSymbol{:}\AgdaSpace{}%
\AgdaBound{X}\AgdaSpace{}%
\AgdaSymbol{→}\AgdaSpace{}%
\AgdaBound{Y}\AgdaSymbol{)}\<%
\\
\>[1][@{}l@{\AgdaIndent{0}}]%
\>[2]\AgdaSymbol{→}%
\>[.][@{}l@{}]\<[372I]%
\>[21]\AgdaFunction{invertible}\AgdaSpace{}%
\AgdaBound{f}\AgdaSpace{}%
\AgdaSymbol{→}\AgdaSpace{}%
\AgdaFunction{is-embedding}\AgdaSpace{}%
\AgdaBound{f}\<%
\\
%
\>[1]\AgdaFunction{invertibles-are-embeddings}\AgdaSpace{}%
\AgdaBound{f}\AgdaSpace{}%
\AgdaBound{fi}\AgdaSpace{}%
\AgdaSymbol{=}\AgdaSpace{}%
\AgdaFunction{equivs-are-embeddings}\AgdaSpace{}%
\AgdaBound{f}\AgdaSpace{}%
\AgdaSymbol{(}\AgdaFunction{invertibles-are-equivs}\AgdaSpace{}%
\AgdaBound{f}\AgdaSpace{}%
\AgdaBound{fi}\AgdaSymbol{)}\<%
\\
\>[0]\<%
\end{code}

Finally, if we have a proof \texttt{p\ :\ is-embedding\ f} that the map \texttt{f} is an embedding, here's a tool that makes it easier to apply \texttt{p}.

\begin{code}%
\>[0]\<%
\\
\>[0][@{}l@{\AgdaIndent{1}}]%
\>[1]\AgdaComment{-- Embedding elimination (makes it easier to apply is-embedding)}\<%
\\
%
\>[1]\AgdaFunction{embedding-elim}\AgdaSpace{}%
\>[371I]\AgdaSymbol{:}%
\>[372I]\AgdaSymbol{\{}\AgdaBound{X}\AgdaSpace{}%
\AgdaSymbol{:}\AgdaSpace{}%
\AgdaBound{𝓤}\AgdaSpace{}%
\AgdaOperator{\AgdaFunction{̇}}\AgdaSpace{}%
\AgdaSymbol{\}}\AgdaSpace{}%
\AgdaSymbol{\{}\AgdaBound{Y}\AgdaSpace{}%
\AgdaSymbol{:}\AgdaSpace{}%
\AgdaBound{𝓦}\AgdaSpace{}%
\AgdaOperator{\AgdaFunction{̇}}\AgdaSpace{}%
\AgdaSymbol{\}\{}\AgdaBound{f}\AgdaSpace{}%
\AgdaSymbol{:}\AgdaSpace{}%
\AgdaBound{X}\AgdaSpace{}%
\AgdaSymbol{→}\AgdaSpace{}%
\AgdaBound{Y}\AgdaSymbol{\}}\<%
\\
\>[1][@{}l@{\AgdaIndent{0}}]%
\>[2]\AgdaSymbol{→}%
\>[.][@{}l@{}]\<[372I]%
\>[18]\AgdaFunction{is-embedding}\AgdaSpace{}%
\AgdaSymbol{→}\AgdaSpace{}%
\AgdaSymbol{(}\AgdaBound{x}\AgdaSpace{}%
\AgdaBound{x'}\AgdaSpace{}%
\AgdaSymbol{:}\AgdaSpace{}%
\AgdaBound{X}\AgdaSymbol{)}\<%
\\
%
\>[18]\AgdaComment{----------------------}\<%
\\
%
\>[2]\AgdaSymbol{→}%
\>[18]\AgdaBound{f}\AgdaSpace{}%
\AgdaBound{x}\AgdaSpace{}%
\AgdaOperator{\AgdaDatatype{≡}}\AgdaSpace{}%
\AgdaBound{f}\AgdaSpace{}%
\AgdaBound{x'}\AgdaSpace{}%
\AgdaSymbol{→}\AgdaSpace{}%
\AgdaBound{x}\AgdaSpace{}%
\AgdaOperator{\AgdaDatatype{≡}}\AgdaSpace{}%
\AgdaBound{x'}\<%
\\
%
\>[1]\AgdaFunction{embedding-elim}\AgdaSpace{}%
\AgdaSymbol{\{}\AgdaArgument{f}\AgdaSpace{}%
\AgdaSymbol{=}\AgdaSpace{}%
\AgdaBound{f}\AgdaSymbol{\}}\AgdaSpace{}%
\AgdaBound{femb}\AgdaSpace{}%
\AgdaBound{x}\AgdaSpace{}%
\AgdaBound{x'}\AgdaSpace{}%
\AgdaBound{fxfx'}\AgdaSpace{}%
\AgdaSymbol{=}\AgdaSpace{}%
\AgdaFunction{γ}\<%
\\
\>[1][@{}l@{\AgdaIndent{0}}]%
\>[2]\AgdaKeyword{where}\<%
\\
\>[2][@{}l@{\AgdaIndent{0}}]%
\>[3]\AgdaFunction{fibx}\AgdaSpace{}%
\AgdaSymbol{:}\AgdaSpace{}%
\AgdaFunction{fiber}\AgdaSpace{}%
\AgdaBound{f}\AgdaSpace{}%
\AgdaSymbol{(}\AgdaBound{f}\AgdaSpace{}%
\AgdaBound{x}\AgdaSymbol{)}\<%
\\
%
\>[3]\AgdaFunction{fibx}\AgdaSpace{}%
\AgdaSymbol{=}\AgdaSpace{}%
\AgdaBound{x}\AgdaSpace{}%
\AgdaOperator{\AgdaInductiveConstructor{,}}\AgdaSpace{}%
\AgdaInductiveConstructor{𝓇ℯ𝒻𝓁}\<%
\\
%
\>[3]\AgdaFunction{fibx'}\AgdaSpace{}%
\AgdaSymbol{:}\AgdaSpace{}%
\AgdaFunction{fiber}\AgdaSpace{}%
\AgdaBound{f}\AgdaSpace{}%
\AgdaSymbol{(}\AgdaBound{f}\AgdaSpace{}%
\AgdaBound{x}\AgdaSymbol{)}\<%
\\
%
\>[3]\AgdaFunction{fibx'}\AgdaSpace{}%
\AgdaSymbol{=}\AgdaSpace{}%
\AgdaBound{x'}\AgdaSpace{}%
\AgdaOperator{\AgdaInductiveConstructor{,}}\AgdaSpace{}%
\AgdaSymbol{((}\AgdaBound{fxfx'}\AgdaSymbol{)}\AgdaSpace{}%
\AgdaOperator{\AgdaFunction{⁻¹}}\AgdaSymbol{)}\<%
\\
%
\>[3]\AgdaFunction{iss-fibffx}\AgdaSpace{}%
\AgdaSymbol{:}\AgdaSpace{}%
\AgdaFunction{is-subsingleton}\AgdaSpace{}%
\AgdaSymbol{(}\AgdaFunction{fiber}\AgdaSpace{}%
\AgdaBound{f}\AgdaSpace{}%
\AgdaSymbol{(}\AgdaBound{f}\AgdaSpace{}%
\AgdaBound{x}\AgdaSymbol{))}\<%
\\
%
\>[3]\AgdaFunction{iss-fibffx}\AgdaSpace{}%
\AgdaSymbol{=}\AgdaSpace{}%
\AgdaBound{femb}\AgdaSpace{}%
\AgdaSymbol{(}\AgdaBound{f}\AgdaSpace{}%
\AgdaBound{x}\AgdaSymbol{)}\<%
\\
%
\>[3]\AgdaFunction{fibxfibx'}\AgdaSpace{}%
\AgdaSymbol{:}\AgdaSpace{}%
\AgdaFunction{fibx}\AgdaSpace{}%
\AgdaOperator{\AgdaDatatype{≡}}\AgdaSpace{}%
\AgdaFunction{fibx'}\<%
\\
%
\>[3]\AgdaFunction{fibxfibx'}\AgdaSpace{}%
\AgdaSymbol{=}\AgdaSpace{}%
\AgdaFunction{iss-fibffx}\AgdaSpace{}%
\AgdaFunction{fibx}\AgdaSpace{}%
\AgdaFunction{fibx'}\<%
\\
%
\>[3]\AgdaFunction{γ}\AgdaSpace{}%
\AgdaSymbol{:}\AgdaSpace{}%
\AgdaBound{x}\AgdaSpace{}%
\AgdaOperator{\AgdaDatatype{≡}}\AgdaSpace{}%
\AgdaBound{x'}\<%
\\
%
\>[3]\AgdaFunction{γ}\AgdaSpace{}%
\AgdaSymbol{=}\AgdaSpace{}%
\AgdaFunction{ap}\AgdaSpace{}%
\AgdaFunction{pr₁}\AgdaSpace{}%
\AgdaFunction{fibxfibx'}\<%
\end{code}
----------------------------------------------------








\subsection{Extensionality}\label{sec:extensionality}
This section describes some of the extensionality principles defined in the \ualibExtensionality module of the \agdaualib that we will need later.

\subsubsection{Function extensionality}\label{function-extensionality}
Extensional equality of functions, or \emph{function extensionality}, means that any two point-wise equal functions are equal. This is known to be not provable or disprovable in Martin-Löf type theory. It is an independent statement.
In~\cite{MHE}, \escardo defines pointwise equality of (dependent) function in the following natural way.
\ccpad
\begin{code}
  \>[0]\AgdaOperator{\AgdaFunction{\AgdaUnderscore{}∼\AgdaUnderscore{}}}\AgdaSpace{}%
\AgdaSymbol{:}\AgdaSpace{}%
\AgdaSymbol{\{}\AgdaBound{X}\AgdaSpace{}%
\AgdaSymbol{:}\AgdaSpace{}%
\AgdaGeneralizable{𝓤}\AgdaSpace{}%
\AgdaOperator{\AgdaFunction{̇}}\AgdaSpace{}%
\AgdaSymbol{\}}\AgdaSpace{}%
\AgdaSymbol{\{}\AgdaBound{A}\AgdaSpace{}%
\AgdaSymbol{:}\AgdaSpace{}%
\AgdaBound{X}\AgdaSpace{}%
\AgdaSymbol{→}\AgdaSpace{}%
\AgdaGeneralizable{𝓥}\AgdaSpace{}%
\AgdaOperator{\AgdaFunction{̇}}\AgdaSpace{}%
\AgdaSymbol{\}}\AgdaSpace{}%
\AgdaSymbol{→}\AgdaSpace{}%
\AgdaFunction{Π}\AgdaSpace{}%
\AgdaBound{A}\AgdaSpace{}%
\AgdaSymbol{→}\AgdaSpace{}%
\AgdaFunction{Π}\AgdaSpace{}%
\AgdaBound{A}\AgdaSpace{}%
\AgdaSymbol{→}\AgdaSpace{}%
\AgdaGeneralizable{𝓤}\AgdaSpace{}%
\AgdaOperator{\AgdaPrimitive{⊔}}\AgdaSpace{}%
\AgdaGeneralizable{𝓥}\AgdaSpace{}%
\AgdaOperator{\AgdaFunction{̇}}\<%
\\
\>[0]\AgdaBound{f}\AgdaSpace{}%
\AgdaOperator{\AgdaFunction{∼}}\AgdaSpace{}%
\AgdaBound{g}\AgdaSpace{}%
\AgdaSymbol{=}\AgdaSpace{}%
\AgdaSymbol{∀}\AgdaSpace{}%
\AgdaBound{x}\AgdaSpace{}%
\AgdaSymbol{→}\AgdaSpace{}%
\AgdaBound{f}\AgdaSpace{}%
\AgdaBound{x}\AgdaSpace{}%
\AgdaOperator{\AgdaDatatype{≡}}\AgdaSpace{}%
\AgdaBound{g}\AgdaSpace{}%
\AgdaBound{x}\<%
\end{code}
\ccpad
Function extensionality is then defined as follows.
\ccpad
\begin{code}
  \>[0]\AgdaFunction{funext}\AgdaSpace{}%
\AgdaSymbol{:}\AgdaSpace{}%
\AgdaSymbol{∀}\AgdaSpace{}%
\AgdaBound{𝓤}\AgdaSpace{}%
\AgdaBound{𝓥}\AgdaSpace{}%
\AgdaSymbol{→}\AgdaSpace{}%
\AgdaSymbol{(}\AgdaBound{𝓤}\AgdaSpace{}%
\AgdaOperator{\AgdaPrimitive{⊔}}\AgdaSpace{}%
\AgdaBound{𝓥}\AgdaSymbol{)}\AgdaOperator{\AgdaPrimitive{⁺}}\AgdaSpace{}%
\AgdaOperator{\AgdaFunction{̇}}\<%
\\
\>[0]\AgdaFunction{funext}\AgdaSpace{}%
\AgdaBound{𝓤}\AgdaSpace{}%
\AgdaBound{𝓥}\AgdaSpace{}%
\AgdaSymbol{=}\AgdaSpace{}%
\AgdaSymbol{\{}\AgdaBound{X}\AgdaSpace{}%
\AgdaSymbol{:}\AgdaSpace{}%
\AgdaBound{𝓤}\AgdaSpace{}%
\AgdaOperator{\AgdaFunction{̇}}\AgdaSpace{}%
\AgdaSymbol{\}}\AgdaSpace{}%
\AgdaSymbol{\{}\AgdaBound{Y}\AgdaSpace{}%
\AgdaSymbol{:}\AgdaSpace{}%
\AgdaBound{𝓥}\AgdaSpace{}%
\AgdaOperator{\AgdaFunction{̇}}\AgdaSpace{}%
\AgdaSymbol{\}}\AgdaSpace{}%
\AgdaSymbol{\{}\AgdaBound{f}\AgdaSpace{}%
\AgdaBound{g}\AgdaSpace{}%
\AgdaSymbol{:}\AgdaSpace{}%
\AgdaBound{X}\AgdaSpace{}%
\AgdaSymbol{→}\AgdaSpace{}%
\AgdaBound{Y}\AgdaSymbol{\}}\AgdaSpace{}%
\AgdaSymbol{→}\AgdaSpace{}%
\AgdaBound{f}\AgdaSpace{}%
\AgdaOperator{\AgdaFunction{∼}}\AgdaSpace{}%
\AgdaBound{g}\AgdaSpace{}%
\AgdaSymbol{→}\AgdaSpace{}%
\AgdaBound{f}\AgdaSpace{}%
\AgdaOperator{\AgdaDatatype{≡}}\AgdaSpace{}%
\AgdaBound{g}\<%
\end{code}
\ccpad
Pointwise equality of functions is typically what one means in informal settings when one says that two functions are equal. 

Later we will need dependent and polymorphic notions of function extensionality, and definitions of these are given in the appendix.







\subsubsection{Function intensionality}\label{function-intensionality}
This is the opposite of function extensionality and it comes in handy whenever we have a definitional equality and need a point-wise equality.
\ccpad
\begin{code}%
\>[0]\AgdaFunction{intensionality}\AgdaSpace{}%
\AgdaSymbol{:}%
\>[200I]\AgdaSymbol{\{}\AgdaBound{𝓤}\AgdaSpace{}%
\AgdaBound{𝓦}\AgdaSpace{}%
\AgdaSymbol{:}\AgdaSpace{}%
\AgdaPostulate{Universe}\AgdaSymbol{\}}\AgdaSpace{}%
\AgdaSymbol{\{}\AgdaBound{A}\AgdaSpace{}%
\AgdaSymbol{:}\AgdaSpace{}%
\AgdaBound{𝓤}\AgdaSpace{}%
\AgdaOperator{\AgdaFunction{̇}}\AgdaSpace{}%
\AgdaSymbol{\}}\AgdaSpace{}%
\AgdaSymbol{\{}\AgdaBound{B}\AgdaSpace{}%
\AgdaSymbol{:}\AgdaSpace{}%
\AgdaBound{𝓦}\AgdaSpace{}%
\AgdaOperator{\AgdaFunction{̇}}\AgdaSpace{}%
\AgdaSymbol{\}}\AgdaSpace{}%
\AgdaSymbol{\{}\AgdaBound{f}\AgdaSpace{}%
\AgdaBound{g}\AgdaSpace{}%
\AgdaSymbol{:}\AgdaSpace{}%
\AgdaBound{A}\AgdaSpace{}%
\AgdaSymbol{→}\AgdaSpace{}%
\AgdaBound{B}\AgdaSymbol{\}}\<%
\\
\>[1][@{}l@{\AgdaIndent{0}}]%
\>[2]\AgdaSymbol{→}%
\>[.][@{}l@{}]\<[200I]%
\>[14]
\AgdaBound{f}\AgdaSpace{}%
\AgdaOperator{\AgdaDatatype{≡}}\AgdaSpace{}%
\AgdaBound{g}%
\>[25]\AgdaSymbol{→}%
\>[28]\AgdaSymbol{(}\AgdaBound{x}\AgdaSpace{}%
\AgdaSymbol{:}\AgdaSpace{}%
\AgdaBound{A}\AgdaSymbol{)}\<%
\\
%
\>[14]\AgdaComment{----------------}\<%
\\
%
\>[2]\AgdaSymbol{→}%
\>[14]\AgdaBound{f}\AgdaSpace{}%
\AgdaBound{x}\AgdaSpace{}%
\AgdaOperator{\AgdaDatatype{≡}}\AgdaSpace{}%
\AgdaBound{g}\AgdaSpace{}%
\AgdaBound{x}\<%
\\
%
\\[\AgdaEmptyExtraSkip]%
\>[0]\AgdaFunction{intensionality}%
\>[16]\AgdaSymbol{(}\AgdaInductiveConstructor{refl}\AgdaSpace{}%
\AgdaSymbol{\AgdaUnderscore{}}\AgdaSpace{}%
\AgdaSymbol{)}\AgdaSpace{}%
\AgdaSymbol{\AgdaUnderscore{}}%
\>[29]\AgdaSymbol{=}\AgdaSpace{}%
\AgdaInductiveConstructor{refl}\AgdaSpace{}%
\AgdaSymbol{\AgdaUnderscore{}}\<%
\end{code}
\ccpad
Of course, the intensionality principle has dependent and polymorphic analogues, but we omit the definitions, which can be found in the \href{https://ualib.gitlab.io/UALib.Prelude.Extensionality.html}{documentation for the \am{UALib.Prelude.Extensionality} module}.








%% Algebras %%%%%%%%%%%%%%%%%%%%%%%%%%%%%%%%%%%%%%%%%%%%%%
\section{Algebras}\label{sec:types-for-algebras}

\subsection{Operation and Signature Types}\label{sec:operation-and-signature-types}
This section describes parts of the \ualibSignatures module of the \agdaualib.
%% --BEGIN: \input{Signatures-short.tex}-----------------------------
\newcommand\signatOV{\AgdaFunction{Signature}\AgdaSpace{}\AgdaBound{𝓞}\AgdaSpace{}\AgdaBound{𝓥}\xspace}

\subsubsection{Operation type}\label{operation-type}
We define the type of \textbf{operations} and, as an example, the type of \textbf{projections}.
\ccpad
\begin{code}%
\>[1]\AgdaFunction{Op}\AgdaSpace{}%
\AgdaSymbol{:}\AgdaSpace{}%
\AgdaBound{𝓥}\AgdaSpace{}%
\AgdaOperator{\AgdaFunction{̇}}\AgdaSpace{}%
\AgdaSymbol{→}\AgdaSpace{}%
\AgdaBound{𝓤}\AgdaSpace{}%
\AgdaOperator{\AgdaFunction{̇}}\AgdaSpace{}%
\AgdaSymbol{→}\AgdaSpace{}%
\AgdaBound{𝓤}\AgdaSpace{}%
\AgdaOperator{\AgdaPrimitive{⊔}}\AgdaSpace{}%
\AgdaBound{𝓥}\AgdaSpace{}%
\AgdaOperator{\AgdaFunction{̇}}\<%
\\
%
\>[1]\AgdaFunction{Op}\AgdaSpace{}%
\AgdaBound{I}\AgdaSpace{}%
\AgdaBound{A}\AgdaSpace{}%
\AgdaSymbol{=}\AgdaSpace{}%
\AgdaSymbol{(}\AgdaBound{I}\AgdaSpace{}%
\AgdaSymbol{→}\AgdaSpace{}%
\AgdaBound{A}\AgdaSymbol{)}\AgdaSpace{}%
\AgdaSymbol{→}\AgdaSpace{}%
\AgdaBound{A}\<%
\\

\\[\AgdaEmptyExtraSkip]%
\>[1]\AgdaFunction{π}\AgdaSpace{}%
\AgdaSymbol{:}\AgdaSpace{}%
\AgdaSymbol{\{}\AgdaBound{I}\AgdaSpace{}%
\AgdaSymbol{:}\AgdaSpace{}%
\AgdaBound{𝓥}\AgdaSpace{}%
\AgdaOperator{\AgdaFunction{̇}}\AgdaSpace{}%
\AgdaSymbol{\}}\AgdaSpace{}%
\AgdaSymbol{\{}\AgdaBound{A}\AgdaSpace{}%
\AgdaSymbol{:}\AgdaSpace{}%
\AgdaBound{𝓤}\AgdaSpace{}%
\AgdaOperator{\AgdaFunction{̇}}\AgdaSpace{}%
\AgdaSymbol{\}}\AgdaSpace{}%
\AgdaSymbol{→}\AgdaSpace{}%
\AgdaBound{I}\AgdaSpace{}%
\AgdaSymbol{→}\AgdaSpace{}%
\AgdaFunction{Op}\AgdaSpace{}%
\AgdaBound{I}\AgdaSpace{}%
\AgdaBound{A}\<%
\\
%
\>[1]\AgdaFunction{π}\AgdaSpace{}%
\AgdaBound{i}\AgdaSpace{}%
\AgdaBound{x}\AgdaSpace{}%
\AgdaSymbol{=}\AgdaSpace{}%
\AgdaBound{x}\AgdaSpace{}%
\AgdaBound{i}\<%
\end{code}
\ccpad
The type \AgdaFunction{Op} encodes the arity of an operation as an arbitrary type \ab I : \ab 𝓥 ̇, which gives us a very general way to represent an operation as a function type with domain \ab I → \ab A (the type of ``tuples'') and codomain \ab A. The last two lines of the code block above codify the \ab i-th \ab I-ary projection operation on \ab A.

\subsubsection{Signature type}\label{signature-type}
We define the signature of an algebraic structure in Agda like this.
\ccpad
\begin{code}%
\>[0]\AgdaFunction{Signature}\AgdaSpace{}%
\AgdaSymbol{:}\AgdaSpace{}%
\AgdaSymbol{(}\AgdaBound{𝓞}\AgdaSpace{}%
\AgdaBound{𝓥}\AgdaSpace{}%
\AgdaSymbol{:}\AgdaSpace{}%
\AgdaPostulate{Universe}\AgdaSymbol{)}\AgdaSpace{}%
\AgdaSymbol{→}\AgdaSpace{}%
\AgdaSymbol{(}\AgdaBound{𝓞}\AgdaSpace{}%
\AgdaOperator{\AgdaPrimitive{⊔}}\AgdaSpace{}%
\AgdaBound{𝓥}\AgdaSymbol{)}\AgdaSpace{}%
\AgdaOperator{\AgdaPrimitive{⁺}}\AgdaSpace{}%
\AgdaOperator{\AgdaFunction{̇}}\<%
\\
\>[0]\AgdaFunction{Signature}\AgdaSpace{}%
\AgdaBound{𝓞}\AgdaSpace{}%
\AgdaBound{𝓥}\AgdaSpace{}%
\AgdaSymbol{=}\AgdaSpace{}%
\AgdaFunction{Σ}\AgdaSpace{}%
\AgdaBound{F}\AgdaSpace{}%
\AgdaFunction{꞉}\AgdaSpace{}%
\AgdaBound{𝓞}\AgdaSpace{}%
\AgdaOperator{\AgdaFunction{̇}}\AgdaSpace{}%
\AgdaFunction{,}\AgdaSpace{}%
\AgdaSymbol{(}\AgdaBound{F}\AgdaSpace{}%
\AgdaSymbol{→}\AgdaSpace{}%
\AgdaBound{𝓥}\AgdaSpace{}%
\AgdaOperator{\AgdaFunction{̇}}\AgdaSymbol{)}\<%
\end{code}
\ccpad
Here \AgdaBound{𝓞} is the universe level of operation symbol types, while \AgdaBound{𝓥} is the universe level of arity types.

Recall (\S\ref{sssec:dependent-pair-type}), the first and second projections are denoted by \ai ∣ and \ai ∥, so if \ab 𝑆 \as : \signatOV is a signature, then \aiab{∣}{𝑆} denotes the type of \defn{operation symbols}, and \aiab{∥}{𝑆} denotes the \defn{arity} function. If \ab 𝑓 \as : \aiab{∣}{𝑆} is an operation symbol in the signature \ab 𝑆, then \aiab{∥}{𝑆} \ab 𝑓 is the arity of \ab 𝑓.

For example,
%% \subsubsection{Example}\label{Signatures.sssec:example}
here is the signature of \emph{monoids}, as a member of the type \signatOV.
\ccpad
\begin{code}%
\>[0][@{}l@{\AgdaIndent{0}}]%
\>[1]\AgdaKeyword{data}\AgdaSpace{}%
\AgdaDatatype{monoid-op}\AgdaSpace{}%
\AgdaSymbol{:}\AgdaSpace{}%
\AgdaBound{𝓞}\AgdaSpace{}%
\AgdaOperator{\AgdaFunction{̇}}\AgdaSpace{}%
\AgdaKeyword{where}\<%
\\
\>[1][@{}l@{\AgdaIndent{0}}]%
\>[2]\AgdaInductiveConstructor{e}\AgdaSpace{}%
\AgdaSymbol{:}\AgdaSpace{}%
\AgdaDatatype{monoid-op}\<%
\\
%
\>[2]\AgdaInductiveConstructor{·}\AgdaSpace{}%
\AgdaSymbol{:}\AgdaSpace{}%
\AgdaDatatype{monoid-op}\<%
\\
%
\\[\AgdaEmptyExtraSkip]%
%
\>[1]\AgdaFunction{monoid-sig}\AgdaSpace{}%
\AgdaSymbol{:}\AgdaSpace{}%
\AgdaFunction{Signature}\AgdaSpace{}%
\AgdaBound{𝓞}\AgdaSpace{}%
\AgdaBound{𝓤₀}\<%
\\
%
\>[1]\AgdaFunction{monoid-sig}\AgdaSpace{}%
\AgdaSymbol{=}\AgdaSpace{}%
\AgdaDatatype{monoid-op}\AgdaSpace{}%
\AgdaOperator{\AgdaInductiveConstructor{,}}\AgdaSpace{}%
\AgdaSymbol{λ}\AgdaSpace{}%
\AgdaSymbol{\{}\AgdaSpace{}%
\AgdaInductiveConstructor{e}\AgdaSpace{}%
\AgdaSymbol{→}\AgdaSpace{}%
\AgdaFunction{𝟘}\AgdaSymbol{;}\AgdaSpace{}%
\AgdaInductiveConstructor{·}\AgdaSpace{}%
\AgdaSymbol{→}\AgdaSpace{}%
\AgdaFunction{𝟚}\AgdaSpace{}%
\AgdaSymbol{\}}\<%
\end{code}
\ccpad
The signature of monoids has two operation symbols, \aic e and \aic ·, and a function \as λ \as{\{} \AgdaInductiveConstructor{e} \as → \af 𝟘 \as ; \AgdaInductiveConstructor{·} \as → \af 𝟚 \as{\}} which maps \aic e to the empty type \af 𝟘 (since \aic e is nullary), and \aic · to the 2-element type \af 𝟚 (since \aic · is binary).
%% -- END: \input{Signatures-short.tex}-----------------------------






\newcommand\sigOV{\AgdaFunction{Signature}\AgdaSpace{}\AgdaBound{𝓞}\AgdaSpace{}\AgdaBound{𝓥}\xspace}
\subsubsection{Algebra type}\label{algebra-types-1}
For a fixed signature \ab S \as : \sigOV and universe \ab 𝓤, we define the type of \textbf{algebras in the signature} \ab 𝑆 (or \ab 𝑆-\textbf{algebras}) and with \textbf{domain} (or \textbf{carrier}) \ab 𝐴 \as : \ab 𝓤 \as ̇ as follows
\ccpad
\begin{code}%
\>[0]\AgdaFunction{Algebra}\AgdaSpace{}%
\AgdaSymbol{:}\AgdaSpace{}%
\AgdaSymbol{(}\AgdaBound{𝓤}\AgdaSpace{}%
\AgdaSymbol{:}\AgdaSpace{}%
\AgdaPostulate{Universe}\AgdaSymbol{)(}\AgdaBound{𝑆}\AgdaSpace{}%
\AgdaSymbol{:}\AgdaSpace{}%
\AgdaFunction{Signature}\AgdaSpace{}%
\AgdaGeneralizable{𝓞}\AgdaSpace{}%
\AgdaGeneralizable{𝓥}\AgdaSymbol{)}\AgdaSpace{}%
\AgdaSymbol{→}%
\>[50]\AgdaGeneralizable{𝓞}\AgdaSpace{}%
\AgdaOperator{\AgdaPrimitive{⊔}}\AgdaSpace{}%
\AgdaGeneralizable{𝓥}\AgdaSpace{}%
\AgdaOperator{\AgdaPrimitive{⊔}}\AgdaSpace{}%
\AgdaBound{𝓤}\AgdaSpace{}%
\AgdaOperator{\AgdaPrimitive{⁺}}\AgdaSpace{}%
\AgdaOperator{\AgdaFunction{̇}}\<%
\\
%
\\[\AgdaEmptyExtraSkip]%
\>[0]\AgdaFunction{Algebra}\AgdaSpace{}%
%% \>[0]\AgdaFunction{∞-algebra}\AgdaSpace{}%
\AgdaBound{𝓤}%
\>[13]\AgdaBound{𝑆}\AgdaSpace{}%
\AgdaSymbol{=}\AgdaSpace{}%
\AgdaFunction{Σ}\AgdaSpace{}%
\AgdaBound{A}\AgdaSpace{}%
\AgdaFunction{꞉}\AgdaSpace{}%
\AgdaBound{𝓤}\AgdaSpace{}%
\AgdaOperator{\AgdaFunction{̇}}\AgdaSpace{}%
\AgdaFunction{,}\AgdaSpace{}%
\AgdaSymbol{((}\AgdaBound{f}\AgdaSpace{}%
\AgdaSymbol{:}\AgdaSpace{}%
\AgdaOperator{\AgdaFunction{∣}}\AgdaSpace{}%
\AgdaBound{𝑆}\AgdaSpace{}%
\AgdaOperator{\AgdaFunction{∣}}\AgdaSymbol{)}\AgdaSpace{}%
\AgdaSymbol{→}\AgdaSpace{}%
\AgdaFunction{Op}\AgdaSpace{}%
\AgdaSymbol{(}\AgdaOperator{\AgdaFunction{∥}}\AgdaSpace{}%
\AgdaBound{𝑆}\AgdaSpace{}%
\AgdaOperator{\AgdaFunction{∥}}\AgdaSpace{}%
\AgdaBound{f}\AgdaSymbol{)}\AgdaSpace{}%
\AgdaBound{A}\AgdaSymbol{)}\<%
\end{code}
\ccpad
We may refer to an inhabitant of \af{Algebra} \ab 𝑆 \AgdaBound{𝓤} as an ``∞-algebra'' because its domain can be an arbitrary type, say, \ab A \as : \ab 𝓤 \af ̇ and need not be truncated at some level; in particular, \ab A need not be a \emph{set} (as defined in \S\ref{sssec:truncation}).

We could pause here to define the type of ``0-algebras,'' which are algebras whose domains are sets. This type is probably closer to what most of us think of when doing informal universal algebra. However, in the \ualib we have so far only needed to know that the domain of an algebra is a set in a handful of specific instances, so it seems preferable to work with general (∞-)algebras throughout the library and then assume \emph{uniquness of identity proofs} explicitly when and only when a proof relies on this assumption.

The type \ad{Algebra} \ab 𝓤 \ab 𝑆 itself has a type; it is \ab 𝓞 \af ⊔ \ab 𝓥 \af ⊔ \ab 𝓤 \af ⁺ \af ̇. This type appears so often in the \ualib that we will define the following shorthand for its universe level: \AgdaFunction{OV}\AgdaSpace{}\ab 𝓤 = \ab 𝓞 \af ⊔ \ab 𝓥 \af ⊔ \ab 𝓤 \af ⁺ \af ̇.\footnote{The \agdaualib includes an alternative definition of the type of algebras using records, but we don't discuss these here since they are not needed in the sequel. We refer the interested reader to the \agdaualib web site where these alternatives are presented in the \href{https://ualib.gitlab.io/UALib.Algebras.Algebras.html}{UALib.Algebras.Algebras} module.}










\subsubsection{Operation interpretation syntax}\label{operation-interpretation-syntax}
Next we define a convenient shorthand for the interpretation of an operation symbol. We use this often in the sequel.
\ccpad
\begin{code}%
\>[0][@{}l@{\AgdaIndent{1}}]%
\>[1]\AgdaOperator{\AgdaFunction{\AgdaUnderscore{}̂\AgdaUnderscore{}}}\AgdaSpace{}%
\AgdaSymbol{:}\AgdaSpace{}%
\AgdaSymbol{(}\AgdaBound{f}\AgdaSpace{}%
\AgdaSymbol{:}\AgdaSpace{}%
\AgdaOperator{\AgdaFunction{∣}}\AgdaSpace{}%
\AgdaBound{𝑆}\AgdaSpace{}%
\AgdaOperator{\AgdaFunction{∣}}\AgdaSymbol{)(}\AgdaBound{𝑨}\AgdaSpace{}%
\AgdaSymbol{:}\AgdaSpace{}%
\AgdaFunction{Algebra}\AgdaSpace{}%
\AgdaBound{𝓤}\AgdaSpace{}%
\AgdaBound{𝑆}\AgdaSymbol{)}\AgdaSpace{}%
\AgdaSymbol{→}\AgdaSpace{}%
\AgdaSymbol{(}\AgdaOperator{\AgdaFunction{∥}}\AgdaSpace{}%
\AgdaBound{𝑆}\AgdaSpace{}%
\AgdaOperator{\AgdaFunction{∥}}\AgdaSpace{}%
\AgdaBound{f}%
\>[48]\AgdaSymbol{→}%
\>[51]\AgdaOperator{\AgdaFunction{∣}}\AgdaSpace{}%
\AgdaBound{𝑨}\AgdaSpace{}%
\AgdaOperator{\AgdaFunction{∣}}\AgdaSymbol{)}\AgdaSpace{}%
\AgdaSymbol{→}\AgdaSpace{}%
\AgdaOperator{\AgdaFunction{∣}}\AgdaSpace{}%
\AgdaBound{𝑨}\AgdaSpace{}%
\AgdaOperator{\AgdaFunction{∣}}\<%
\\
%
\\[\AgdaEmptyExtraSkip]%
%
\>[1]\AgdaBound{f}\AgdaSpace{}%
\AgdaOperator{\AgdaFunction{̂}}\AgdaSpace{}%
\AgdaBound{𝑨}\AgdaSpace{}%
\AgdaSymbol{=}\AgdaSpace{}%
\AgdaSymbol{λ}\AgdaSpace{}%
\AgdaBound{x}\AgdaSpace{}%
\AgdaSymbol{→}\AgdaSpace{}%
\AgdaSymbol{(}\AgdaOperator{\AgdaFunction{∥}}\AgdaSpace{}%
\AgdaBound{𝑨}\AgdaSpace{}%
\AgdaOperator{\AgdaFunction{∥}}\AgdaSpace{}%
\AgdaBound{f}\AgdaSymbol{)}\AgdaSpace{}%
\AgdaBound{x}\<%
\end{code}
\ccpad
This is similar to the standard notation that one finds in the literature and seems much more natural to us than the double bar notation that we started with.

\subsubsection{Arbitrarily many variable symbols}
Finally, we will want to assume that we always have at our disposal an arbitrary collection \ab X of variable symbols such that, for every algebra \ab 𝑨, no matter the type of its domain, we have a surjective map \ab{h₀} \as : \ab X \as → \aiab{∣}{𝑨} from variables onto the domain of \ab 𝑨.
\ccpad
\begin{code}%
\>[0][@{}l@{\AgdaIndent{1}}]%
\>[1]\AgdaOperator{\AgdaFunction{\AgdaUnderscore{}↠\AgdaUnderscore{}}}\AgdaSpace{}%
\AgdaSymbol{:}\AgdaSpace{}%
\AgdaSymbol{\{}\AgdaBound{𝓤}\AgdaSpace{}%
\AgdaBound{𝓧}\AgdaSpace{}%
\AgdaSymbol{:}\AgdaSpace{}%
\AgdaPostulate{Universe}\AgdaSymbol{\}}\AgdaSpace{}%
\AgdaSymbol{→}\AgdaSpace{}%
\AgdaBound{𝓧}\AgdaSpace{}%
\AgdaOperator{\AgdaFunction{̇}}\AgdaSpace{}%
\AgdaSymbol{→}\AgdaSpace{}%
\AgdaFunction{Algebra}\AgdaSpace{}%
\AgdaBound{𝓤}\AgdaSpace{}%
\AgdaBound{𝑆}\AgdaSpace{}%
\AgdaSymbol{→}\AgdaSpace{}%
\AgdaBound{𝓧}\AgdaSpace{}%
\AgdaOperator{\AgdaPrimitive{⊔}}\AgdaSpace{}%
\AgdaBound{𝓤}\AgdaSpace{}%
\AgdaOperator{\AgdaFunction{̇}}\<%
\\
%
\>[1]\AgdaBound{X}\AgdaSpace{}%
\AgdaOperator{\AgdaFunction{↠}}\AgdaSpace{}%
\AgdaBound{𝑨}\AgdaSpace{}%
\AgdaSymbol{=}\AgdaSpace{}%
\AgdaFunction{Σ}\AgdaSpace{}%
\AgdaBound{h}\AgdaSpace{}%
\AgdaFunction{꞉}\AgdaSpace{}%
\AgdaSymbol{(}\AgdaBound{X}\AgdaSpace{}%
\AgdaSymbol{→}\AgdaSpace{}%
\AgdaOperator{\AgdaFunction{∣}}\AgdaSpace{}%
\AgdaBound{𝑨}\AgdaSpace{}%
\AgdaOperator{\AgdaFunction{∣}}\AgdaSymbol{)}\AgdaSpace{}%
\AgdaFunction{,}\AgdaSpace{}%
\AgdaFunction{Epic}\AgdaSpace{}%
\AgdaBound{h}\<%
\end{code}

%%--  END: \input{Algebras-short.tex}-----------------------------




















\subsubsection{Product algebra types}\label{sec:product-algebra-types}
This section describes some parts of the \ualibAlgProducts module of the \agdaualib.
%% -- BEGIN: \input{Products-short.tex}
We define products of algebras for the Sigma type representation defined above.
\ccpad
\begin{code}%
\>[1]\AgdaFunction{⨅}\AgdaSpace{}%
\AgdaSymbol{:}\AgdaSpace{}%
\AgdaSymbol{\{}\AgdaBound{𝓘}\AgdaSpace{}%
\AgdaSymbol{:}\AgdaSpace{}%
\AgdaPostulate{Universe}\AgdaSymbol{\}\{}\AgdaBound{I}\AgdaSpace{}%
\AgdaSymbol{:}\AgdaSpace{}%
\AgdaBound{𝓘}\AgdaSpace{}%
\AgdaOperator{\AgdaFunction{̇}}\AgdaSpace{}%
\AgdaSymbol{\}(}\AgdaBound{𝒜}\AgdaSpace{}%
\AgdaSymbol{:}\AgdaSpace{}%
\AgdaBound{I}\AgdaSpace{}%
\AgdaSymbol{→}\AgdaSpace{}%
\AgdaFunction{Algebra}\AgdaSpace{}%
\AgdaBound{𝓤}\AgdaSpace{}%
\AgdaBound{𝑆}\AgdaSpace{}%
\AgdaSymbol{)}\AgdaSpace{}%
\AgdaSymbol{→}\AgdaSpace{}%
\AgdaFunction{Algebra}\AgdaSpace{}%
\AgdaSymbol{(}\AgdaBound{𝓘}\AgdaSpace{}%
\AgdaOperator{\AgdaPrimitive{⊔}}\AgdaSpace{}%
\AgdaBound{𝓤}\AgdaSymbol{)}\AgdaSpace{}%
\AgdaBound{𝑆}\<%
\\
%
\>[1]\AgdaFunction{⨅}\AgdaSpace{}%
\AgdaSymbol{\{}\AgdaBound{𝓘}\AgdaSymbol{\}\{}\AgdaBound{I}\AgdaSymbol{\}}\AgdaSpace{}%
\AgdaBound{𝒜}\AgdaSpace{}%
\AgdaSymbol{=}\<%
\\
\>[1][@{}l@{\AgdaIndent{0}}]%
\>[2]\AgdaSymbol{((}\AgdaBound{i}\AgdaSpace{}%
\AgdaSymbol{:}\AgdaSpace{}%
\AgdaBound{I}\AgdaSymbol{)}\AgdaSpace{}%
\AgdaSymbol{→}\AgdaSpace{}%
\AgdaOperator{\AgdaFunction{∣}}\AgdaSpace{}%
\AgdaBound{𝒜}\AgdaSpace{}%
\AgdaBound{i}\AgdaSpace{}%
\AgdaOperator{\AgdaFunction{∣}}\AgdaSymbol{)}\AgdaSpace{}%
\AgdaOperator{\AgdaInductiveConstructor{,}}\AgdaSpace{}%
\AgdaSymbol{λ(}\AgdaBound{f}\AgdaSpace{}%
\AgdaSymbol{:}\AgdaSpace{}%
\AgdaOperator{\AgdaFunction{∣}}\AgdaSpace{}%
\AgdaBound{𝑆}\AgdaSpace{}%
\AgdaOperator{\AgdaFunction{∣}}\AgdaSymbol{)(}\AgdaBound{𝒂}\AgdaSpace{}%
\AgdaSymbol{:}\AgdaSpace{}%
\AgdaOperator{\AgdaFunction{∥}}\AgdaSpace{}%
\AgdaBound{𝑆}\AgdaSpace{}%
\AgdaOperator{\AgdaFunction{∥}}\AgdaSpace{}%
\AgdaBound{f}\AgdaSpace{}%
\AgdaSymbol{→}\AgdaSpace{}%
\AgdaSymbol{(}\AgdaBound{j}\AgdaSpace{}%
\AgdaSymbol{:}\AgdaSpace{}%
\AgdaBound{I}\AgdaSymbol{)}\AgdaSpace{}%
\AgdaSymbol{→}\AgdaSpace{}%
\AgdaOperator{\AgdaFunction{∣}}\AgdaSpace{}%
\AgdaBound{𝒜}\AgdaSpace{}%
\AgdaBound{j}\AgdaSpace{}%
\AgdaOperator{\AgdaFunction{∣}}\AgdaSymbol{)(}\AgdaBound{i}\AgdaSpace{}%
\AgdaSymbol{:}\AgdaSpace{}%
\AgdaBound{I}\AgdaSymbol{)}\AgdaSpace{}%
\AgdaSymbol{→}\AgdaSpace{}%
\AgdaSymbol{(}\AgdaBound{f}\AgdaSpace{}%
\AgdaOperator{\AgdaFunction{̂}}\AgdaSpace{}%
\AgdaBound{𝒜}\AgdaSpace{}%
\AgdaBound{i}\AgdaSymbol{)}\AgdaSpace{}%
\AgdaSymbol{λ\{}\AgdaBound{x}\AgdaSpace{}%
\AgdaSymbol{→}\AgdaSpace{}%
\AgdaBound{𝒂}\AgdaSpace{}%
\AgdaBound{x}\AgdaSpace{}%
\AgdaBound{i}\AgdaSymbol{\}}\<%
\end{code}

%% -- END: \input{Products-short.tex}




















\subsection{The Universe Hierarchy and Lifts}\label{sec:universe-hierarchy-and-lifts}
This section describes parts of the \ualibLifts module of the \agdaualib.
%% BEGIN: \input{Lifts-short.tex}




\subsubsection{The noncumulative hierarchy}\label{the-noncumulative-hierarchy}
The hierarchy of universe levels in Agda is structured as \ab{𝓤₀} \as : \ab{𝓤₁}, \hskip3mm \ab{𝓤₁} \as : \ab{𝓤₂}, \hskip3mm \ab{𝓤₂} \as : \ab{𝓤₃}, \ldots{}. This means that \ab{𝓤₀} has type \ab{𝓤₁} \AgdaFunction{̇} and \ab{𝓤ₙ} has type \ab{𝓤ₙ₊₁} \AgdaFunction{̇} for each \ab n.

It is important to note, however, this does \emph{not} imply that \ab{𝓤₀} \as : \ab{𝓤₂} and \ab{𝓤₀} \as : \ab{𝓤₃}, and so on. In other words, Agda's universe hierarchy is \emph{noncummulative}. This makes it possible to treat universe levels more generally and precisely, which is nice. On the other hand, it is this author's experience that a noncummulative hierarchy can sometimes make for a nonfun proof assistant.

Luckily, there are ways to subvert noncummulativity which, when used with care, do not introduce logically consistencies into the type theory. We describe some techniques we developed for this purpose that are specifically tailored for our domain of applications.

\subsubsection{Lifting and lowering}\label{lifting-and-lowering}
Let us be more concrete about what is at issue here by giving an example. Unless we are pedantic enough to worry about which universe level each of our types inhabits, then eventually we will encounter an error like the following:\\
\\
{\small
\AgdaError{Birkhoff.lagda:498,20-23}\\
\AgdaError{(𝓤 ⁺)\ !=\ (𝓞 ⁺)\ ⊔\ (𝓥 ⁺)\ ⊔\ ((𝓤 ⁺) ⁺)}\\
\AgdaError{when checking that the expression SP𝒦 has type}\\
\AgdaError{Pred\ (Σ\ (λ\ A\ →\ (f₁\ :\ ∣ 𝑆 ∣)\ →\ Op (∥ 𝑆 ∥ f₁) A)) \au 𝓦 \au 2346}}
\\
Let's translate the error above, just to confirm that we know what it means. Here, \agda has encountered a type at universe level \ab 𝓤 \af ⁺, on line 498 (in columns 20--23) of the file Birkhoff.lagda, but was expecting a type at level \ab 𝓞 \af ⁺ \af ⊔ \ab 𝓥 \af ⁺ \af ⊔ \ab 𝓤 \af ⁺ \af ⁺ instead.

To make these situations easier to deal with, the \ualib offers some domain specific tools for the \emph{lifting} and \emph{lowering} of universe levels of the main types in the library. In particular, we have functions that will lift or lower the universes of algebra types, homomorphisms, subalgebras, and products.

Of course, messing with the universe level of a type must be done carefully to avoid making the type theory inconsistent.  In particular, a necessary condition is that a type of a given universe level may not be converted to a type of lower universe level \emph{unless the given type was obtained from lifting another type to a higher-than-necessary universe level}.  If this is not clear, don't worry; just press on and soon enough there will be examples that make it clear.

A general \af{Lift} record type, similar to the one found in the \agdastdlib (in the \af{Level} module), is
defined as follows.
\ccpad
\begin{code}%
\>[0]\AgdaKeyword{record}\AgdaSpace{}%
\AgdaRecord{Lift}\AgdaSpace{}%
\AgdaSymbol{\{}\AgdaBound{𝓤}\AgdaSpace{}%
\AgdaBound{𝓦}\AgdaSpace{}%
\AgdaSymbol{:}\AgdaSpace{}%
\AgdaPostulate{Universe}\AgdaSymbol{\}}\AgdaSpace{}%
\AgdaSymbol{(}\AgdaBound{X}\AgdaSpace{}%
\AgdaSymbol{:}\AgdaSpace{}%
\AgdaBound{𝓤}\AgdaSpace{}%
\AgdaOperator{\AgdaFunction{̇}}\AgdaSymbol{)}\AgdaSpace{}%
\AgdaSymbol{:}\AgdaSpace{}%
\AgdaBound{𝓤}\AgdaSpace{}%
\AgdaOperator{\AgdaPrimitive{⊔}}\AgdaSpace{}%
\AgdaBound{𝓦}\AgdaSpace{}%
\AgdaOperator{\AgdaFunction{̇}}%
\>[50]\AgdaKeyword{where}\<%
\\
\>[0][@{}l@{\AgdaIndent{0}}]%
\>[1]\AgdaKeyword{constructor}\AgdaSpace{}%
\AgdaInductiveConstructor{lift}\<%
\\
%
\>[1]\AgdaKeyword{field}\AgdaSpace{}%
\AgdaField{lower}\AgdaSpace{}%
\AgdaSymbol{:}\AgdaSpace{}%
\AgdaBound{X}\<%
\\
\>[0]\AgdaKeyword{open}\AgdaSpace{}%
\AgdaModule{Lift}\<%
\end{code}
\ccpad
Next, we give various ways to lift function types.
\ccpad
\begin{code}%
\>[0]\AgdaFunction{lift-dom}\AgdaSpace{}%
\AgdaSymbol{:}\AgdaSpace{}%
\AgdaSymbol{\{}\AgdaBound{𝓧}\AgdaSpace{}%
\AgdaBound{𝓨}\AgdaSpace{}%
\AgdaBound{𝓦}\AgdaSpace{}%
\AgdaSymbol{:}\AgdaSpace{}%
\AgdaPostulate{Universe}\AgdaSymbol{\}\{}\AgdaBound{X}\AgdaSpace{}%
\AgdaSymbol{:}\AgdaSpace{}%
\AgdaBound{𝓧}\AgdaSpace{}%
\AgdaOperator{\AgdaFunction{̇}}\AgdaSymbol{\}\{}\AgdaBound{Y}\AgdaSpace{}%
\AgdaSymbol{:}\AgdaSpace{}%
\AgdaBound{𝓨}\AgdaSpace{}%
\AgdaOperator{\AgdaFunction{̇}}\AgdaSymbol{\}}\AgdaSpace{}%
\AgdaSymbol{→}\AgdaSpace{}%
\AgdaSymbol{(}\AgdaBound{X}\AgdaSpace{}%
\AgdaSymbol{→}\AgdaSpace{}%
\AgdaBound{Y}\AgdaSymbol{)}\AgdaSpace{}%
\AgdaSymbol{→}\AgdaSpace{}%
\AgdaSymbol{(}\AgdaRecord{Lift}\AgdaSymbol{\{}\AgdaBound{𝓧}\AgdaSymbol{\}\{}\AgdaBound{𝓦}\AgdaSymbol{\}}\AgdaSpace{}%
\AgdaBound{X}\AgdaSpace{}%
\AgdaSymbol{→}\AgdaSpace{}%
\AgdaBound{Y}\AgdaSymbol{)}\<%
\\
\>[0]\AgdaFunction{lift-dom}\AgdaSpace{}%
\AgdaBound{f}\AgdaSpace{}%
\AgdaSymbol{=}\AgdaSpace{}%
\AgdaSymbol{λ}\AgdaSpace{}%
\AgdaBound{x}\AgdaSpace{}%
\AgdaSymbol{→}\AgdaSpace{}%
\AgdaSymbol{(}\AgdaBound{f}\AgdaSpace{}%
\AgdaSymbol{(}\AgdaField{lower}\AgdaSpace{}%
\AgdaBound{x}\AgdaSymbol{))}\<%
\\
%
\\[\AgdaEmptyExtraSkip]%
\>[0]\AgdaFunction{lift-cod}\AgdaSpace{}%
\AgdaSymbol{:}\AgdaSpace{}%
\AgdaSymbol{\{}\AgdaBound{𝓧}\AgdaSpace{}%
\AgdaBound{𝓨}\AgdaSpace{}%
\AgdaBound{𝓦}\AgdaSpace{}%
\AgdaSymbol{:}\AgdaSpace{}%
\AgdaPostulate{Universe}\AgdaSymbol{\}\{}\AgdaBound{X}\AgdaSpace{}%
\AgdaSymbol{:}\AgdaSpace{}%
\AgdaBound{𝓧}\AgdaSpace{}%
\AgdaOperator{\AgdaFunction{̇}}\AgdaSymbol{\}\{}\AgdaBound{Y}\AgdaSpace{}%
\AgdaSymbol{:}\AgdaSpace{}%
\AgdaBound{𝓨}\AgdaSpace{}%
\AgdaOperator{\AgdaFunction{̇}}\AgdaSymbol{\}}\AgdaSpace{}%
\AgdaSymbol{→}\AgdaSpace{}%
\AgdaSymbol{(}\AgdaBound{X}\AgdaSpace{}%
\AgdaSymbol{→}\AgdaSpace{}%
\AgdaBound{Y}\AgdaSymbol{)}\AgdaSpace{}%
\AgdaSymbol{→}\AgdaSpace{}%
\AgdaSymbol{(}\AgdaBound{X}\AgdaSpace{}%
\AgdaSymbol{→}\AgdaSpace{}%
\AgdaRecord{Lift}\AgdaSymbol{\{}\AgdaBound{𝓨}\AgdaSymbol{\}\{}\AgdaBound{𝓦}\AgdaSymbol{\}}\AgdaSpace{}%
\AgdaBound{Y}\AgdaSymbol{)}\<%
\\
\>[0]\AgdaFunction{lift-cod}\AgdaSpace{}%
\AgdaBound{f}\AgdaSpace{}%
\AgdaSymbol{=}\AgdaSpace{}%
\AgdaSymbol{λ}\AgdaSpace{}%
\AgdaBound{x}\AgdaSpace{}%
\AgdaSymbol{→}\AgdaSpace{}%
\AgdaInductiveConstructor{lift}\AgdaSpace{}%
\AgdaSymbol{(}\AgdaBound{f}\AgdaSpace{}%
\AgdaBound{x}\AgdaSymbol{)}\<%
\\
%
\\[\AgdaEmptyExtraSkip]%
\>[0]\AgdaFunction{lift-fun}\AgdaSpace{}%
\AgdaSymbol{:}\AgdaSpace{}%
\AgdaSymbol{\{}\AgdaBound{𝓧}\AgdaSpace{}%
\AgdaBound{𝓨}\AgdaSpace{}%
\AgdaBound{𝓦}\AgdaSpace{}%
\AgdaBound{𝓩}\AgdaSpace{}%
\AgdaSymbol{:}\AgdaSpace{}%
\AgdaPostulate{Universe}\AgdaSymbol{\}\{}\AgdaBound{X}\AgdaSpace{}%
\AgdaSymbol{:}\AgdaSpace{}%
\AgdaBound{𝓧}\AgdaSpace{}%
\AgdaOperator{\AgdaFunction{̇}}\AgdaSymbol{\}\{}\AgdaBound{Y}\AgdaSpace{}%
\AgdaSymbol{:}\AgdaSpace{}%
\AgdaBound{𝓨}\AgdaSpace{}%
\AgdaOperator{\AgdaFunction{̇}}\AgdaSymbol{\}}\AgdaSpace{}%
\AgdaSymbol{→}\AgdaSpace{}%
\AgdaSymbol{(}\AgdaBound{X}\AgdaSpace{}%
\AgdaSymbol{→}\AgdaSpace{}%
\AgdaBound{Y}\AgdaSymbol{)}\AgdaSpace{}%
\AgdaSymbol{→}\AgdaSpace{}%
\AgdaSymbol{(}\AgdaRecord{Lift}\AgdaSymbol{\{}\AgdaBound{𝓧}\AgdaSymbol{\}\{}\AgdaBound{𝓦}\AgdaSymbol{\}}\AgdaSpace{}%
\AgdaBound{X}\AgdaSpace{}%
\AgdaSymbol{→}\AgdaSpace{}%
\AgdaRecord{Lift}\AgdaSymbol{\{}\AgdaBound{𝓨}\AgdaSymbol{\}\{}\AgdaBound{𝓩}\AgdaSymbol{\}}\AgdaSpace{}%
\AgdaBound{Y}\AgdaSymbol{)}\<%
\\
\>[0]\AgdaFunction{lift-fun}\AgdaSpace{}%
\AgdaBound{f}\AgdaSpace{}%
\AgdaSymbol{=}\AgdaSpace{}%
\AgdaSymbol{λ}\AgdaSpace{}%
\AgdaBound{x}\AgdaSpace{}%
\AgdaSymbol{→}\AgdaSpace{}%
\AgdaInductiveConstructor{lift}\AgdaSpace{}%
\AgdaSymbol{(}\AgdaBound{f}\AgdaSpace{}%
\AgdaSymbol{(}\AgdaField{lower}\AgdaSpace{}%
\AgdaBound{x}\AgdaSymbol{))}\<%
\end{code}
\ccpad
We will also need to know that lift and lower compose to the identity.
\ccpad
\begin{code}%
\>[0]\AgdaFunction{lower∼lift}\AgdaSpace{}%
\AgdaSymbol{:}\AgdaSpace{}%
\AgdaSymbol{\{}\AgdaBound{𝓧}\AgdaSpace{}%
\AgdaBound{𝓦}\AgdaSpace{}%
\AgdaSymbol{:}\AgdaSpace{}%
\AgdaPostulate{Universe}\AgdaSymbol{\}\{}\AgdaBound{X}\AgdaSpace{}%
\AgdaSymbol{:}\AgdaSpace{}%
\AgdaBound{𝓧}\AgdaSpace{}%
\AgdaOperator{\AgdaFunction{̇}}\AgdaSymbol{\}}\AgdaSpace{}%
\AgdaSymbol{→}\AgdaSpace{}%
\AgdaField{lower}\AgdaSymbol{\{}\AgdaBound{𝓧}\AgdaSymbol{\}\{}\AgdaBound{𝓦}\AgdaSymbol{\}}\AgdaSpace{}%
\AgdaOperator{\AgdaFunction{∘}}\AgdaSpace{}%
\AgdaInductiveConstructor{lift}\AgdaSpace{}%
\AgdaOperator{\AgdaDatatype{≡}}\AgdaSpace{}%
\AgdaFunction{𝑖𝑑}\AgdaSpace{}%
\AgdaBound{X}\<%
\\
\>[0]\AgdaFunction{lower∼lift}\AgdaSpace{}%
\AgdaSymbol{=}\AgdaSpace{}%
\AgdaInductiveConstructor{refl}\AgdaSpace{}%
\AgdaSymbol{\AgdaUnderscore{}}\<%
\\
%
\\[\AgdaEmptyExtraSkip]%
\>[0]\AgdaFunction{lift∼lower}\AgdaSpace{}%
\AgdaSymbol{:}\AgdaSpace{}%
\AgdaSymbol{\{}\AgdaBound{𝓧}\AgdaSpace{}%
\AgdaBound{𝓦}\AgdaSpace{}%
\AgdaSymbol{:}\AgdaSpace{}%
\AgdaPostulate{Universe}\AgdaSymbol{\}\{}\AgdaBound{X}\AgdaSpace{}%
\AgdaSymbol{:}\AgdaSpace{}%
\AgdaBound{𝓧}\AgdaSpace{}%
\AgdaOperator{\AgdaFunction{̇}}\AgdaSymbol{\}}\AgdaSpace{}%
\AgdaSymbol{→}\AgdaSpace{}%
\AgdaInductiveConstructor{lift}\AgdaSpace{}%
\AgdaOperator{\AgdaFunction{∘}}\AgdaSpace{}%
\AgdaField{lower}\AgdaSpace{}%
\AgdaOperator{\AgdaDatatype{≡}}\AgdaSpace{}%
\AgdaFunction{𝑖𝑑}\AgdaSpace{}%
\AgdaSymbol{(}\AgdaRecord{Lift}\AgdaSymbol{\{}\AgdaBound{𝓧}\AgdaSymbol{\}\{}\AgdaBound{𝓦}\AgdaSymbol{\}}\AgdaSpace{}%
\AgdaBound{X}\AgdaSymbol{)}\<%
\\
\>[0]\AgdaFunction{lift∼lower}\AgdaSpace{}%
\AgdaSymbol{=}\AgdaSpace{}%
\AgdaInductiveConstructor{refl}\AgdaSpace{}%
\AgdaSymbol{\AgdaUnderscore{}}\<%
\end{code}
\ccpad
Now, to be more domain-specific, we show how to lift algebraic operation types and then, finally, how to lift algebra types.
\ccpad
\begin{code}%
\>[0]\AgdaKeyword{module}\AgdaSpace{}%
\AgdaModule{\AgdaUnderscore{}}\AgdaSpace{}%
\AgdaSymbol{\{}\AgdaBound{𝑆}\AgdaSpace{}%
\AgdaSymbol{:}\AgdaSpace{}%
\AgdaFunction{Σ}\AgdaSpace{}%
\AgdaBound{F}\AgdaSpace{}%
\AgdaFunction{꞉}\AgdaSpace{}%
\AgdaGeneralizable{𝓞}\AgdaSpace{}%
\AgdaOperator{\AgdaFunction{̇}}\AgdaSpace{}%
\AgdaFunction{,}\AgdaSpace{}%
\AgdaSymbol{(}\AgdaSpace{}%
\AgdaBound{F}\AgdaSpace{}%
\AgdaSymbol{→}\AgdaSpace{}%
\AgdaGeneralizable{𝓥}\AgdaSpace{}%
\AgdaOperator{\AgdaFunction{̇}}\AgdaSymbol{)\}}\AgdaSpace{}%
\AgdaKeyword{where}\<%
\\
%
\\[\AgdaEmptyExtraSkip]%
\>[0][@{}l@{\AgdaIndent{0}}]%
\>[1]\AgdaFunction{lift-op}\AgdaSpace{}%
\AgdaSymbol{:}\AgdaSpace{}%
\AgdaSymbol{\{}\AgdaBound{𝓤}\AgdaSpace{}%
\AgdaSymbol{:}\AgdaSpace{}%
\AgdaPostulate{Universe}\AgdaSymbol{\}\{}\AgdaBound{I}\AgdaSpace{}%
\AgdaSymbol{:}\AgdaSpace{}%
\AgdaBound{𝓥}\AgdaSpace{}%
\AgdaOperator{\AgdaFunction{̇}}\AgdaSymbol{\}\{}\AgdaBound{A}\AgdaSpace{}%
\AgdaSymbol{:}\AgdaSpace{}%
\AgdaBound{𝓤}\AgdaSpace{}%
\AgdaOperator{\AgdaFunction{̇}}\AgdaSymbol{\}}\<%
\\
\>[1][@{}l@{\AgdaIndent{0}}]%
\>[2]\AgdaSymbol{→}%
\>[11]\AgdaSymbol{((}\AgdaBound{I}\AgdaSpace{}%
\AgdaSymbol{→}\AgdaSpace{}%
\AgdaBound{A}\AgdaSymbol{)}\AgdaSpace{}%
\AgdaSymbol{→}\AgdaSpace{}%
\AgdaBound{A}\AgdaSymbol{)}\AgdaSpace{}%
\AgdaSymbol{→}\AgdaSpace{}%
\AgdaSymbol{(}\AgdaBound{𝓦}\AgdaSpace{}%
\AgdaSymbol{:}\AgdaSpace{}%
\AgdaPostulate{Universe}\AgdaSymbol{)}\<%
\\
%
\>[2]\AgdaSymbol{→}%
\>[11]\AgdaSymbol{((}\AgdaBound{I}\AgdaSpace{}%
\AgdaSymbol{→}\AgdaSpace{}%
\AgdaRecord{Lift}\AgdaSymbol{\{}\AgdaBound{𝓤}\AgdaSymbol{\}\{}\AgdaBound{𝓦}\AgdaSymbol{\}}\AgdaBound{A}\AgdaSymbol{)}\AgdaSpace{}%
\AgdaSymbol{→}\AgdaSpace{}%
\AgdaRecord{Lift}\AgdaSymbol{\{}\AgdaBound{𝓤}\AgdaSymbol{\}\{}\AgdaBound{𝓦}\AgdaSymbol{\}}\AgdaBound{A}\AgdaSymbol{)}\<%
\\
%
\>[1]\AgdaFunction{lift-op}\AgdaSpace{}%
\AgdaBound{f}\AgdaSpace{}%
\AgdaBound{𝓦}\AgdaSpace{}%
\AgdaSymbol{=}\AgdaSpace{}%
\AgdaSymbol{λ}\AgdaSpace{}%
\AgdaBound{x}\AgdaSpace{}%
\AgdaSymbol{→}\AgdaSpace{}%
\AgdaInductiveConstructor{lift}\AgdaSpace{}%
\AgdaSymbol{(}\AgdaBound{f}\AgdaSpace{}%
\AgdaSymbol{(λ}\AgdaSpace{}%
\AgdaBound{i}\AgdaSpace{}%
\AgdaSymbol{→}\AgdaSpace{}%
\AgdaField{lower}\AgdaSpace{}%
\AgdaSymbol{(}\AgdaBound{x}\AgdaSpace{}%
\AgdaBound{i}\AgdaSymbol{)))}\<%
\\
%
\\[\AgdaEmptyExtraSkip]%
%
\>[1]\AgdaKeyword{open}\AgdaSpace{}%
\AgdaModule{algebra}\<%
\\
%
\\[\AgdaEmptyExtraSkip]%
%
\>[1]\AgdaFunction{lift-alg-record-type}\AgdaSpace{}%
\AgdaSymbol{:}\AgdaSpace{}%
\AgdaSymbol{\{}\AgdaBound{𝓤}\AgdaSpace{}%
\AgdaSymbol{:}\AgdaSpace{}%
\AgdaPostulate{Universe}\AgdaSymbol{\}}\AgdaSpace{}%
\AgdaSymbol{→}\AgdaSpace{}%
\AgdaRecord{algebra}\AgdaSpace{}%
\AgdaBound{𝓤}\AgdaSpace{}%
\AgdaBound{𝑆}\AgdaSpace{}%
\AgdaSymbol{→}\AgdaSpace{}%
\AgdaSymbol{(}\AgdaBound{𝓦}\AgdaSpace{}%
\AgdaSymbol{:}\AgdaSpace{}%
\AgdaPostulate{Universe}\AgdaSymbol{)}\AgdaSpace{}%
\AgdaSymbol{→}\AgdaSpace{}%
\AgdaRecord{algebra}\AgdaSpace{}%
\AgdaSymbol{(}\AgdaBound{𝓤}\AgdaSpace{}%
\AgdaOperator{\AgdaPrimitive{⊔}}\AgdaSpace{}%
\AgdaBound{𝓦}\AgdaSymbol{)}\AgdaSpace{}%
\AgdaBound{𝑆}\<%
\\
%
\>[1]\AgdaFunction{lift-alg-record-type}\AgdaSpace{}%
\AgdaBound{𝑨}\AgdaSpace{}%
\AgdaBound{𝓦}\AgdaSpace{}%
\AgdaSymbol{=}\AgdaSpace{}%
\AgdaInductiveConstructor{mkalg}\AgdaSpace{}%
\AgdaSymbol{(}\AgdaRecord{Lift}\AgdaSpace{}%
\AgdaSymbol{(}\AgdaField{univ}\AgdaSpace{}%
\AgdaBound{𝑨}\AgdaSymbol{))}\AgdaSpace{}%
\AgdaSymbol{(λ}\AgdaSpace{}%
\AgdaSymbol{(}\AgdaBound{f}\AgdaSpace{}%
\AgdaSymbol{:}\AgdaSpace{}%
\AgdaOperator{\AgdaFunction{∣}}\AgdaSpace{}%
\AgdaBound{𝑆}\AgdaSpace{}%
\AgdaOperator{\AgdaFunction{∣}}\AgdaSymbol{)}\AgdaSpace{}%
\AgdaSymbol{→}\AgdaSpace{}%
\AgdaFunction{lift-op}\AgdaSpace{}%
\AgdaSymbol{((}\AgdaField{op}\AgdaSpace{}%
\AgdaBound{𝑨}\AgdaSymbol{)}\AgdaSpace{}%
\AgdaBound{f}\AgdaSymbol{)}\AgdaSpace{}%
\AgdaBound{𝓦}\AgdaSymbol{)}\<%
\\
%
\\[\AgdaEmptyExtraSkip]%
%
\>[1]\AgdaFunction{lift-∞-algebra}\AgdaSpace{}%
\AgdaFunction{lift-alg}\AgdaSpace{}%
\AgdaSymbol{:}\AgdaSpace{}%
\AgdaSymbol{\{}\AgdaBound{𝓤}\AgdaSpace{}%
\AgdaSymbol{:}\AgdaSpace{}%
\AgdaPostulate{Universe}\AgdaSymbol{\}}\AgdaSpace{}%
\AgdaSymbol{→}\AgdaSpace{}%
\AgdaFunction{Algebra}\AgdaSpace{}%
\AgdaBound{𝓤}\AgdaSpace{}%
\AgdaBound{𝑆}\AgdaSpace{}%
\AgdaSymbol{→}\AgdaSpace{}%
\AgdaSymbol{(}\AgdaBound{𝓦}\AgdaSpace{}%
\AgdaSymbol{:}\AgdaSpace{}%
\AgdaPostulate{Universe}\AgdaSymbol{)}\AgdaSpace{}%
\AgdaSymbol{→}\AgdaSpace{}%
\AgdaFunction{Algebra}\AgdaSpace{}%
\AgdaSymbol{(}\AgdaBound{𝓤}\AgdaSpace{}%
\AgdaOperator{\AgdaPrimitive{⊔}}\AgdaSpace{}%
\AgdaBound{𝓦}\AgdaSymbol{)}\AgdaSpace{}%
\AgdaBound{𝑆}\<%
\\
%
\>[1]\AgdaFunction{lift-∞-algebra}\AgdaSpace{}%
\AgdaBound{𝑨}\AgdaSpace{}%
\AgdaBound{𝓦}\AgdaSpace{}%
\AgdaSymbol{=}\AgdaSpace{}%
\AgdaRecord{Lift}\AgdaSpace{}%
\AgdaOperator{\AgdaFunction{∣}}\AgdaSpace{}%
\AgdaBound{𝑨}\AgdaSpace{}%
\AgdaOperator{\AgdaFunction{∣}}\AgdaSpace{}%
\AgdaOperator{\AgdaInductiveConstructor{,}}\AgdaSpace{}%
\AgdaSymbol{(λ}\AgdaSpace{}%
\AgdaSymbol{(}\AgdaBound{f}\AgdaSpace{}%
\AgdaSymbol{:}\AgdaSpace{}%
\AgdaOperator{\AgdaFunction{∣}}\AgdaSpace{}%
\AgdaBound{𝑆}\AgdaSpace{}%
\AgdaOperator{\AgdaFunction{∣}}\AgdaSymbol{)}\AgdaSpace{}%
\AgdaSymbol{→}\AgdaSpace{}%
\AgdaFunction{lift-op}\AgdaSpace{}%
\AgdaSymbol{(}\AgdaOperator{\AgdaFunction{∥}}\AgdaSpace{}%
\AgdaBound{𝑨}\AgdaSpace{}%
\AgdaOperator{\AgdaFunction{∥}}\AgdaSpace{}%
\AgdaBound{f}\AgdaSymbol{)}\AgdaSpace{}%
\AgdaBound{𝓦}\AgdaSymbol{)}\<%
\\
%
\>[1]\AgdaFunction{lift-alg}\AgdaSpace{}%
\AgdaSymbol{=}\AgdaSpace{}%
\AgdaFunction{lift-∞-algebra}\<%
\end{code}

%% END: \input{Lifts-short.tex}








%% Relations %%%%%%%%%%%%%%%%%%%%%%%%%%%%%%%%%%%%%%%%%%%%%%%
\section{Relations}\label{sec:relations}

\subsection{Unary Relations (predicates)}\label{sec:predicates}
This section describes some parts of the \ualibUnary module of the \agdaualib.
%% BEGIN: \subsubsection{Unary Relation Types}\label{unary-relation-types}

This section presents the {[}UALib.Relations.Unary{]}{[}{]} module of
the {[}Agda Universal Algebra Library{]}{[}{]}.

We need a mechanism for implementing the notion of subsets in Agda. A
typical one is called \texttt{Pred} (for predicate). More generally,
\texttt{Pred\ A\ 𝓤} can be viewed as the type of a property that
elements of type \texttt{A} might satisfy. We write
\texttt{P\ :\ Pred\ A\ 𝓤} to represent the semantic concept of a
collection of elements of type \texttt{A} that satisfy the property
\texttt{P}.

\begin{code}%
\>[0]\<%
\\
\>[0]\AgdaSymbol{\{-\#}\AgdaSpace{}%
\AgdaKeyword{OPTIONS}\AgdaSpace{}%
\AgdaPragma{--without-K}\AgdaSpace{}%
\AgdaPragma{--exact-split}\AgdaSpace{}%
\AgdaPragma{--safe}\AgdaSpace{}%
\AgdaSymbol{\#-\}}\<%
\\
%
\\[\AgdaEmptyExtraSkip]%
\>[0]\AgdaKeyword{module}\AgdaSpace{}%
\AgdaModule{UALib.Relations.Unary}\AgdaSpace{}%
\AgdaKeyword{where}\<%
\\
%
\\[\AgdaEmptyExtraSkip]%
\>[0]\AgdaKeyword{open}\AgdaSpace{}%
\AgdaKeyword{import}\AgdaSpace{}%
\AgdaModule{UALib.Algebras.Lifts}\AgdaSpace{}%
\AgdaKeyword{public}\<%
\\
%
\\[\AgdaEmptyExtraSkip]%
\>[0]\AgdaKeyword{open}\AgdaSpace{}%
\AgdaKeyword{import}\AgdaSpace{}%
\AgdaModule{UALib.Prelude.Preliminaries}\AgdaSpace{}%
\AgdaKeyword{using}\AgdaSpace{}%
\AgdaSymbol{(}\AgdaFunction{¬}\AgdaSymbol{;}\AgdaSpace{}%
\AgdaFunction{propext}\AgdaSymbol{;}\AgdaSpace{}%
\AgdaFunction{global-dfunext}\AgdaSpace{}%
\AgdaSymbol{)}\AgdaSpace{}%
\AgdaKeyword{public}\<%
\\
\>[0]\<%
\end{code}

Here is the definition, which is similar to the one found in
the\texttt{Relation/Unary.agda} file of the {[}Agda Standard
Library{]}{[}{]}.

\begin{code}%
\>[0]\<%
\\
\>[0]\AgdaKeyword{module}\AgdaSpace{}%
\AgdaModule{\AgdaUnderscore{}}\AgdaSpace{}%
\AgdaSymbol{\{}\AgdaBound{𝓤}\AgdaSpace{}%
\AgdaSymbol{:}\AgdaSpace{}%
\AgdaPostulate{Universe}\AgdaSymbol{\}}\AgdaSpace{}%
\AgdaKeyword{where}\<%
\\
%
\\[\AgdaEmptyExtraSkip]%
\>[0][@{}l@{\AgdaIndent{0}}]%
\>[1]\AgdaFunction{Pred}\AgdaSpace{}%
\AgdaSymbol{:}\AgdaSpace{}%
\AgdaBound{𝓤}\AgdaSpace{}%
\AgdaOperator{\AgdaFunction{̇}}\AgdaSpace{}%
\AgdaSymbol{→}\AgdaSpace{}%
\AgdaSymbol{(}\AgdaBound{𝓥}\AgdaSpace{}%
\AgdaSymbol{:}\AgdaSpace{}%
\AgdaPostulate{Universe}\AgdaSymbol{)}\AgdaSpace{}%
\AgdaSymbol{→}\AgdaSpace{}%
\AgdaBound{𝓤}\AgdaSpace{}%
\AgdaOperator{\AgdaPrimitive{⊔}}\AgdaSpace{}%
\AgdaBound{𝓥}\AgdaSpace{}%
\AgdaOperator{\AgdaPrimitive{⁺}}\AgdaSpace{}%
\AgdaOperator{\AgdaFunction{̇}}\<%
\\
%
\>[1]\AgdaFunction{Pred}\AgdaSpace{}%
\AgdaBound{A}\AgdaSpace{}%
\AgdaBound{𝓥}\AgdaSpace{}%
\AgdaSymbol{=}\AgdaSpace{}%
\AgdaBound{A}\AgdaSpace{}%
\AgdaSymbol{→}\AgdaSpace{}%
\AgdaBound{𝓥}\AgdaSpace{}%
\AgdaOperator{\AgdaFunction{̇}}\<%
\\
\>[0]\<%
\end{code}

\paragraph{Unary relation truncation}\label{unary-relation-truncation}

The section on
\href{UALib.Prelude.Preliminaries.html\#truncation}{truncation} in the
module {[}UALib.Prelude.Preliminaries{]}{[}{]} describes the concepts of
\emph{truncation} and \emph{set} for ``proof-relevant'' mathematics.
Sometimes we will want to assume that a type is a \emph{set}. Recall,
this mean there is at most one proof that two elements are the same.
Analogously for predicates, we may wish to assume that there is at most
one proof that a given element satisfies the predicate.

\begin{code}%
\>[0]\<%
\\
\>[0][@{}l@{\AgdaIndent{1}}]%
\>[1]\AgdaFunction{Pred₀}\AgdaSpace{}%
\AgdaSymbol{:}\AgdaSpace{}%
\AgdaBound{𝓤}\AgdaSpace{}%
\AgdaOperator{\AgdaFunction{̇}}\AgdaSpace{}%
\AgdaSymbol{→}\AgdaSpace{}%
\AgdaSymbol{(}\AgdaBound{𝓥}\AgdaSpace{}%
\AgdaSymbol{:}\AgdaSpace{}%
\AgdaPostulate{Universe}\AgdaSymbol{)}\AgdaSpace{}%
\AgdaSymbol{→}\AgdaSpace{}%
\AgdaBound{𝓤}\AgdaSpace{}%
\AgdaOperator{\AgdaPrimitive{⊔}}\AgdaSpace{}%
\AgdaBound{𝓥}\AgdaSpace{}%
\AgdaOperator{\AgdaPrimitive{⁺}}\AgdaSpace{}%
\AgdaOperator{\AgdaFunction{̇}}\<%
\\
%
\>[1]\AgdaFunction{Pred₀}\AgdaSpace{}%
\AgdaBound{A}\AgdaSpace{}%
\AgdaBound{𝓥}\AgdaSpace{}%
\AgdaSymbol{=}\AgdaSpace{}%
\AgdaFunction{Σ}\AgdaSpace{}%
\AgdaBound{P}\AgdaSpace{}%
\AgdaFunction{꞉}\AgdaSpace{}%
\AgdaSymbol{(}\AgdaBound{A}\AgdaSpace{}%
\AgdaSymbol{→}\AgdaSpace{}%
\AgdaBound{𝓥}\AgdaSpace{}%
\AgdaOperator{\AgdaFunction{̇}}\AgdaSymbol{)}\AgdaSpace{}%
\AgdaFunction{,}\AgdaSpace{}%
\AgdaSymbol{∀}\AgdaSpace{}%
\AgdaBound{x}\AgdaSpace{}%
\AgdaSymbol{→}\AgdaSpace{}%
\AgdaFunction{is-subsingleton}\AgdaSpace{}%
\AgdaSymbol{(}\AgdaBound{P}\AgdaSpace{}%
\AgdaBound{x}\AgdaSymbol{)}\<%
\\
\>[0]\<%
\end{code}

Below we will often consider predicates over the class of all algebras
of a particular type. We will define the type of algebras
\texttt{Algebra\ 𝓤\ 𝑆} (for some universe level 𝓤). Like all types,
\texttt{Algebra\ 𝓤\ 𝑆} itself has a type which happens to be 𝓞 ⊔ 𝓥 ⊔ 𝓤 ⁺
̇ (as we will see in the module
\href{UALib.Algebras.Algebras.html}{UALib.Algebras}. Therefore, the type
of \texttt{Pred\ (Algebra\ 𝓤\ 𝑆)\ 𝓤} will be 𝓞 ⊔ 𝓥 ⊔ 𝓤 ⁺ ̇ as well.

The inhabitants of the type \texttt{Pred\ (Algebra\ 𝓤\ 𝑆)} 𝓤 are maps of
the form \texttt{𝑨\ →\ 𝓤\ ̇}; given an algebra
\texttt{𝑨\ :\ Algebra\ 𝓤\ 𝑆}, we have \texttt{Pred\ 𝑨\ 𝓤\ =\ 𝑨\ →\ 𝓤\ ̇}.

\paragraph{The membership relation}\label{the-membership-relation}

We introduce notation so that we may indicate that \texttt{x} ``belongs
to'' or ``inhabits'' at type \texttt{P}, or that \texttt{x} ``has
property'' \texttt{P}, by writing either \texttt{x\ ∈\ P} or
\texttt{P\ x} (cf. \texttt{Relation/Unary.agda} in the {[}Agda Standard
Library{]}{[}{]}).

\begin{code}%
\>[0]\<%
\\
\>[0]\AgdaKeyword{module}\AgdaSpace{}%
\AgdaModule{\AgdaUnderscore{}}\AgdaSpace{}%
\AgdaSymbol{\{}\AgdaBound{𝓤}\AgdaSpace{}%
\AgdaBound{𝓦}\AgdaSpace{}%
\AgdaSymbol{:}\AgdaSpace{}%
\AgdaPostulate{Universe}\AgdaSymbol{\}}\AgdaSpace{}%
\AgdaKeyword{where}\<%
\\
%
\\[\AgdaEmptyExtraSkip]%
\>[0][@{}l@{\AgdaIndent{0}}]%
\>[1]\AgdaOperator{\AgdaFunction{\AgdaUnderscore{}∈\AgdaUnderscore{}}}\AgdaSpace{}%
\AgdaSymbol{:}\AgdaSpace{}%
\AgdaSymbol{\{}\AgdaBound{A}\AgdaSpace{}%
\AgdaSymbol{:}\AgdaSpace{}%
\AgdaBound{𝓤}\AgdaSpace{}%
\AgdaOperator{\AgdaFunction{̇}}\AgdaSpace{}%
\AgdaSymbol{\}}\AgdaSpace{}%
\AgdaSymbol{→}\AgdaSpace{}%
\AgdaBound{A}\AgdaSpace{}%
\AgdaSymbol{→}\AgdaSpace{}%
\AgdaFunction{Pred}\AgdaSpace{}%
\AgdaBound{A}\AgdaSpace{}%
\AgdaBound{𝓦}\AgdaSpace{}%
\AgdaSymbol{→}\AgdaSpace{}%
\AgdaBound{𝓦}\AgdaSpace{}%
\AgdaOperator{\AgdaFunction{̇}}\<%
\\
%
\>[1]\AgdaBound{x}\AgdaSpace{}%
\AgdaOperator{\AgdaFunction{∈}}\AgdaSpace{}%
\AgdaBound{P}\AgdaSpace{}%
\AgdaSymbol{=}\AgdaSpace{}%
\AgdaBound{P}\AgdaSpace{}%
\AgdaBound{x}\<%
\\
%
\\[\AgdaEmptyExtraSkip]%
%
\>[1]\AgdaOperator{\AgdaFunction{\AgdaUnderscore{}∉\AgdaUnderscore{}}}\AgdaSpace{}%
\AgdaSymbol{:}\AgdaSpace{}%
\AgdaSymbol{\{}\AgdaBound{A}\AgdaSpace{}%
\AgdaSymbol{:}\AgdaSpace{}%
\AgdaBound{𝓤}\AgdaSpace{}%
\AgdaOperator{\AgdaFunction{̇}}\AgdaSpace{}%
\AgdaSymbol{\}}\AgdaSpace{}%
\AgdaSymbol{→}\AgdaSpace{}%
\AgdaBound{A}\AgdaSpace{}%
\AgdaSymbol{→}\AgdaSpace{}%
\AgdaFunction{Pred}\AgdaSpace{}%
\AgdaBound{A}\AgdaSpace{}%
\AgdaBound{𝓦}\AgdaSpace{}%
\AgdaSymbol{→}\AgdaSpace{}%
\AgdaBound{𝓦}\AgdaSpace{}%
\AgdaOperator{\AgdaFunction{̇}}\<%
\\
%
\>[1]\AgdaBound{x}\AgdaSpace{}%
\AgdaOperator{\AgdaFunction{∉}}\AgdaSpace{}%
\AgdaBound{P}\AgdaSpace{}%
\AgdaSymbol{=}\AgdaSpace{}%
\AgdaFunction{¬}\AgdaSpace{}%
\AgdaSymbol{(}\AgdaBound{x}\AgdaSpace{}%
\AgdaOperator{\AgdaFunction{∈}}\AgdaSpace{}%
\AgdaBound{P}\AgdaSymbol{)}\<%
\\
%
\\[\AgdaEmptyExtraSkip]%
%
\>[1]\AgdaKeyword{infix}\AgdaSpace{}%
\AgdaNumber{4}\AgdaSpace{}%
\AgdaOperator{\AgdaFunction{\AgdaUnderscore{}∈\AgdaUnderscore{}}}\AgdaSpace{}%
\AgdaOperator{\AgdaFunction{\AgdaUnderscore{}∉\AgdaUnderscore{}}}\<%
\\
\>[0]\<%
\end{code}

The ``subset'' relation is denoted, as usual, with the \texttt{⊆} symbol
(cf. \texttt{Relation/Unary.agda} in the {[}Agda Standard
Library{]}{[}{]}).

\begin{code}%
\>[0]\<%
\\
\>[0]\AgdaOperator{\AgdaFunction{\AgdaUnderscore{}⊆\AgdaUnderscore{}}}\AgdaSpace{}%
\AgdaSymbol{:}\AgdaSpace{}%
\AgdaSymbol{\{}\AgdaBound{𝓤}\AgdaSpace{}%
\AgdaBound{𝓦}\AgdaSpace{}%
\AgdaBound{𝓣}\AgdaSpace{}%
\AgdaSymbol{:}\AgdaSpace{}%
\AgdaPostulate{Universe}\AgdaSymbol{\}\{}\AgdaBound{A}\AgdaSpace{}%
\AgdaSymbol{:}\AgdaSpace{}%
\AgdaBound{𝓤}\AgdaSpace{}%
\AgdaOperator{\AgdaFunction{̇}}\AgdaSpace{}%
\AgdaSymbol{\}}\AgdaSpace{}%
\AgdaSymbol{→}\AgdaSpace{}%
\AgdaFunction{Pred}\AgdaSpace{}%
\AgdaBound{A}\AgdaSpace{}%
\AgdaBound{𝓦}\AgdaSpace{}%
\AgdaSymbol{→}\AgdaSpace{}%
\AgdaFunction{Pred}\AgdaSpace{}%
\AgdaBound{A}\AgdaSpace{}%
\AgdaBound{𝓣}\AgdaSpace{}%
\AgdaSymbol{→}\AgdaSpace{}%
\AgdaBound{𝓤}\AgdaSpace{}%
\AgdaOperator{\AgdaPrimitive{⊔}}\AgdaSpace{}%
\AgdaBound{𝓦}\AgdaSpace{}%
\AgdaOperator{\AgdaPrimitive{⊔}}\AgdaSpace{}%
\AgdaBound{𝓣}\AgdaSpace{}%
\AgdaOperator{\AgdaFunction{̇}}\<%
\\
\>[0]\AgdaBound{P}\AgdaSpace{}%
\AgdaOperator{\AgdaFunction{⊆}}\AgdaSpace{}%
\AgdaBound{Q}\AgdaSpace{}%
\AgdaSymbol{=}\AgdaSpace{}%
\AgdaSymbol{∀}\AgdaSpace{}%
\AgdaSymbol{\{}\AgdaBound{x}\AgdaSymbol{\}}\AgdaSpace{}%
\AgdaSymbol{→}\AgdaSpace{}%
\AgdaBound{x}\AgdaSpace{}%
\AgdaOperator{\AgdaFunction{∈}}\AgdaSpace{}%
\AgdaBound{P}\AgdaSpace{}%
\AgdaSymbol{→}\AgdaSpace{}%
\AgdaBound{x}\AgdaSpace{}%
\AgdaOperator{\AgdaFunction{∈}}\AgdaSpace{}%
\AgdaBound{Q}\<%
\\
%
\\[\AgdaEmptyExtraSkip]%
\>[0]\AgdaOperator{\AgdaFunction{\AgdaUnderscore{}⊇\AgdaUnderscore{}}}\AgdaSpace{}%
\AgdaSymbol{:}\AgdaSpace{}%
\AgdaSymbol{\{}\AgdaBound{𝓤}\AgdaSpace{}%
\AgdaBound{𝓦}\AgdaSpace{}%
\AgdaBound{𝓣}\AgdaSpace{}%
\AgdaSymbol{:}\AgdaSpace{}%
\AgdaPostulate{Universe}\AgdaSymbol{\}\{}\AgdaBound{A}\AgdaSpace{}%
\AgdaSymbol{:}\AgdaSpace{}%
\AgdaBound{𝓤}\AgdaSpace{}%
\AgdaOperator{\AgdaFunction{̇}}\AgdaSpace{}%
\AgdaSymbol{\}}\AgdaSpace{}%
\AgdaSymbol{→}\AgdaSpace{}%
\AgdaFunction{Pred}\AgdaSpace{}%
\AgdaBound{A}\AgdaSpace{}%
\AgdaBound{𝓦}\AgdaSpace{}%
\AgdaSymbol{→}\AgdaSpace{}%
\AgdaFunction{Pred}\AgdaSpace{}%
\AgdaBound{A}\AgdaSpace{}%
\AgdaBound{𝓣}\AgdaSpace{}%
\AgdaSymbol{→}\AgdaSpace{}%
\AgdaBound{𝓤}\AgdaSpace{}%
\AgdaOperator{\AgdaPrimitive{⊔}}\AgdaSpace{}%
\AgdaBound{𝓦}\AgdaSpace{}%
\AgdaOperator{\AgdaPrimitive{⊔}}\AgdaSpace{}%
\AgdaBound{𝓣}\AgdaSpace{}%
\AgdaOperator{\AgdaFunction{̇}}\<%
\\
\>[0]\AgdaBound{P}\AgdaSpace{}%
\AgdaOperator{\AgdaFunction{⊇}}\AgdaSpace{}%
\AgdaBound{Q}\AgdaSpace{}%
\AgdaSymbol{=}\AgdaSpace{}%
\AgdaBound{Q}\AgdaSpace{}%
\AgdaOperator{\AgdaFunction{⊆}}\AgdaSpace{}%
\AgdaBound{P}\<%
\\
%
\\[\AgdaEmptyExtraSkip]%
\>[0]\AgdaKeyword{infix}\AgdaSpace{}%
\AgdaNumber{4}\AgdaSpace{}%
\AgdaOperator{\AgdaFunction{\AgdaUnderscore{}⊆\AgdaUnderscore{}}}\AgdaSpace{}%
\AgdaOperator{\AgdaFunction{\AgdaUnderscore{}⊇\AgdaUnderscore{}}}\<%
\\
\>[0]\<%
\end{code}

In type theory everything is a type. As we have just seen, this includes
subsets. Since the notion of equality for types is usually a nontrivial
matter, it may be nontrivial to represent equality of subsets.
Fortunately, it is straightforward to write down a type that represents
what it means for two subsets to be the in informal (pencil-paper)
mathematics. In the {[}Agda UALib{]}{[}{]} we denote this \textbf{subset
equality} by =̇ and define it as follows.

\begin{code}%
\>[0]\<%
\\
\>[0]\AgdaOperator{\AgdaFunction{\AgdaUnderscore{}=̇\AgdaUnderscore{}}}\AgdaSpace{}%
\AgdaSymbol{:}\AgdaSpace{}%
\AgdaSymbol{\{}\AgdaBound{𝓤}\AgdaSpace{}%
\AgdaBound{𝓦}\AgdaSpace{}%
\AgdaBound{𝓣}\AgdaSpace{}%
\AgdaSymbol{:}\AgdaSpace{}%
\AgdaPostulate{Universe}\AgdaSymbol{\}\{}\AgdaBound{A}\AgdaSpace{}%
\AgdaSymbol{:}\AgdaSpace{}%
\AgdaBound{𝓤}\AgdaSpace{}%
\AgdaOperator{\AgdaFunction{̇}}\AgdaSpace{}%
\AgdaSymbol{\}}\AgdaSpace{}%
\AgdaSymbol{→}\AgdaSpace{}%
\AgdaFunction{Pred}\AgdaSpace{}%
\AgdaBound{A}\AgdaSpace{}%
\AgdaBound{𝓦}\AgdaSpace{}%
\AgdaSymbol{→}\AgdaSpace{}%
\AgdaFunction{Pred}\AgdaSpace{}%
\AgdaBound{A}\AgdaSpace{}%
\AgdaBound{𝓣}\AgdaSpace{}%
\AgdaSymbol{→}\AgdaSpace{}%
\AgdaBound{𝓤}\AgdaSpace{}%
\AgdaOperator{\AgdaPrimitive{⊔}}\AgdaSpace{}%
\AgdaBound{𝓦}\AgdaSpace{}%
\AgdaOperator{\AgdaPrimitive{⊔}}\AgdaSpace{}%
\AgdaBound{𝓣}\AgdaSpace{}%
\AgdaOperator{\AgdaFunction{̇}}\<%
\\
\>[0]\AgdaBound{P}\AgdaSpace{}%
\AgdaOperator{\AgdaFunction{=̇}}\AgdaSpace{}%
\AgdaBound{Q}\AgdaSpace{}%
\AgdaSymbol{=}\AgdaSpace{}%
\AgdaSymbol{(}\AgdaBound{P}\AgdaSpace{}%
\AgdaOperator{\AgdaFunction{⊆}}\AgdaSpace{}%
\AgdaBound{Q}\AgdaSymbol{)}\AgdaSpace{}%
\AgdaOperator{\AgdaFunction{×}}\AgdaSpace{}%
\AgdaSymbol{(}\AgdaBound{Q}\AgdaSpace{}%
\AgdaOperator{\AgdaFunction{⊆}}\AgdaSpace{}%
\AgdaBound{P}\AgdaSymbol{)}\<%
\\
\>[0]\<%
\end{code}

\paragraph{Predicates toolbox}\label{predicates-toolbox}

Here is a small collection of tools that will come in handy later.
Hopefully the meaning of each is self-explanatory.

\begin{code}%
\>[0]\<%
\\
\>[0]\AgdaOperator{\AgdaFunction{\AgdaUnderscore{}∈∈\AgdaUnderscore{}}}\AgdaSpace{}%
\AgdaSymbol{:}\AgdaSpace{}%
\AgdaSymbol{\{}\AgdaBound{𝓤}\AgdaSpace{}%
\AgdaBound{𝓦}\AgdaSpace{}%
\AgdaBound{𝓣}\AgdaSpace{}%
\AgdaSymbol{:}\AgdaSpace{}%
\AgdaPostulate{Universe}\AgdaSymbol{\}\{}\AgdaBound{A}\AgdaSpace{}%
\AgdaSymbol{:}\AgdaSpace{}%
\AgdaBound{𝓤}\AgdaSpace{}%
\AgdaOperator{\AgdaFunction{̇}}\AgdaSpace{}%
\AgdaSymbol{\}}\AgdaSpace{}%
\AgdaSymbol{\{}\AgdaBound{B}\AgdaSpace{}%
\AgdaSymbol{:}\AgdaSpace{}%
\AgdaBound{𝓦}\AgdaSpace{}%
\AgdaOperator{\AgdaFunction{̇}}\AgdaSpace{}%
\AgdaSymbol{\}}\AgdaSpace{}%
\AgdaSymbol{→}\AgdaSpace{}%
\AgdaSymbol{(}\AgdaBound{A}%
\>[53]\AgdaSymbol{→}%
\>[56]\AgdaBound{B}\AgdaSymbol{)}\AgdaSpace{}%
\AgdaSymbol{→}\AgdaSpace{}%
\AgdaFunction{Pred}\AgdaSpace{}%
\AgdaBound{B}\AgdaSpace{}%
\AgdaBound{𝓣}\AgdaSpace{}%
\AgdaSymbol{→}\AgdaSpace{}%
\AgdaBound{𝓤}\AgdaSpace{}%
\AgdaOperator{\AgdaPrimitive{⊔}}\AgdaSpace{}%
\AgdaBound{𝓣}\AgdaSpace{}%
\AgdaOperator{\AgdaFunction{̇}}\<%
\\
\>[0]\AgdaOperator{\AgdaFunction{\AgdaUnderscore{}∈∈\AgdaUnderscore{}}}\AgdaSpace{}%
\AgdaBound{f}\AgdaSpace{}%
\AgdaBound{S}\AgdaSpace{}%
\AgdaSymbol{=}\AgdaSpace{}%
\AgdaSymbol{(}\AgdaBound{x}\AgdaSpace{}%
\AgdaSymbol{:}\AgdaSpace{}%
\AgdaSymbol{\AgdaUnderscore{})}\AgdaSpace{}%
\AgdaSymbol{→}\AgdaSpace{}%
\AgdaBound{f}\AgdaSpace{}%
\AgdaBound{x}\AgdaSpace{}%
\AgdaOperator{\AgdaFunction{∈}}\AgdaSpace{}%
\AgdaBound{S}\<%
\\
%
\\[\AgdaEmptyExtraSkip]%
\>[0]\AgdaFunction{Pred-refl}\AgdaSpace{}%
\AgdaSymbol{:}\AgdaSpace{}%
\AgdaSymbol{\{}\AgdaBound{𝓤}\AgdaSpace{}%
\AgdaBound{𝓦}\AgdaSpace{}%
\AgdaSymbol{:}\AgdaSpace{}%
\AgdaPostulate{Universe}\AgdaSymbol{\}\{}\AgdaBound{A}\AgdaSpace{}%
\AgdaSymbol{:}\AgdaSpace{}%
\AgdaBound{𝓤}\AgdaSpace{}%
\AgdaOperator{\AgdaFunction{̇}}\AgdaSymbol{\}\{}\AgdaBound{P}\AgdaSpace{}%
\AgdaBound{Q}\AgdaSpace{}%
\AgdaSymbol{:}\AgdaSpace{}%
\AgdaFunction{Pred}\AgdaSpace{}%
\AgdaBound{A}\AgdaSpace{}%
\AgdaBound{𝓦}\AgdaSymbol{\}}\<%
\\
\>[0][@{}l@{\AgdaIndent{0}}]%
\>[1]\AgdaSymbol{→}%
\>[12]\AgdaBound{P}\AgdaSpace{}%
\AgdaOperator{\AgdaDatatype{≡}}\AgdaSpace{}%
\AgdaBound{Q}\AgdaSpace{}%
\AgdaSymbol{→}\AgdaSpace{}%
\AgdaSymbol{(}\AgdaBound{a}\AgdaSpace{}%
\AgdaSymbol{:}\AgdaSpace{}%
\AgdaBound{A}\AgdaSymbol{)}\AgdaSpace{}%
\AgdaSymbol{→}\AgdaSpace{}%
\AgdaBound{a}\AgdaSpace{}%
\AgdaOperator{\AgdaFunction{∈}}\AgdaSpace{}%
\AgdaBound{P}\AgdaSpace{}%
\AgdaSymbol{→}\AgdaSpace{}%
\AgdaBound{a}\AgdaSpace{}%
\AgdaOperator{\AgdaFunction{∈}}\AgdaSpace{}%
\AgdaBound{Q}\<%
\\
\>[0]\AgdaFunction{Pred-refl}\AgdaSpace{}%
\AgdaSymbol{(}\AgdaInductiveConstructor{refl}\AgdaSpace{}%
\AgdaSymbol{\AgdaUnderscore{})}\AgdaSpace{}%
\AgdaSymbol{\AgdaUnderscore{}}\AgdaSpace{}%
\AgdaSymbol{=}\AgdaSpace{}%
\AgdaSymbol{λ}\AgdaSpace{}%
\AgdaBound{z}\AgdaSpace{}%
\AgdaSymbol{→}\AgdaSpace{}%
\AgdaBound{z}\<%
\\
%
\\[\AgdaEmptyExtraSkip]%
\>[0]\AgdaFunction{Pred-≡}\AgdaSpace{}%
\AgdaSymbol{:}\AgdaSpace{}%
\AgdaSymbol{\{}\AgdaBound{𝓤}\AgdaSpace{}%
\AgdaBound{𝓦}\AgdaSpace{}%
\AgdaSymbol{:}\AgdaSpace{}%
\AgdaPostulate{Universe}\AgdaSymbol{\}\{}\AgdaBound{A}\AgdaSpace{}%
\AgdaSymbol{:}\AgdaSpace{}%
\AgdaBound{𝓤}\AgdaSpace{}%
\AgdaOperator{\AgdaFunction{̇}}\AgdaSymbol{\}\{}\AgdaBound{P}\AgdaSpace{}%
\AgdaBound{Q}\AgdaSpace{}%
\AgdaSymbol{:}\AgdaSpace{}%
\AgdaFunction{Pred}\AgdaSpace{}%
\AgdaBound{A}\AgdaSpace{}%
\AgdaBound{𝓦}\AgdaSymbol{\}}\<%
\\
\>[0][@{}l@{\AgdaIndent{0}}]%
\>[1]\AgdaSymbol{→}%
\>[12]\AgdaBound{P}\AgdaSpace{}%
\AgdaOperator{\AgdaDatatype{≡}}\AgdaSpace{}%
\AgdaBound{Q}\AgdaSpace{}%
\AgdaSymbol{→}\AgdaSpace{}%
\AgdaBound{P}\AgdaSpace{}%
\AgdaOperator{\AgdaFunction{=̇}}\AgdaSpace{}%
\AgdaBound{Q}\<%
\\
\>[0]\AgdaFunction{Pred-≡}\AgdaSpace{}%
\AgdaSymbol{(}\AgdaInductiveConstructor{refl}\AgdaSpace{}%
\AgdaSymbol{\AgdaUnderscore{})}\AgdaSpace{}%
\AgdaSymbol{=}\AgdaSpace{}%
\AgdaSymbol{(λ}\AgdaSpace{}%
\AgdaBound{z}\AgdaSpace{}%
\AgdaSymbol{→}\AgdaSpace{}%
\AgdaBound{z}\AgdaSymbol{)}\AgdaSpace{}%
\AgdaOperator{\AgdaInductiveConstructor{,}}\AgdaSpace{}%
\AgdaSymbol{λ}\AgdaSpace{}%
\AgdaBound{z}\AgdaSpace{}%
\AgdaSymbol{→}\AgdaSpace{}%
\AgdaBound{z}\<%
\\
%
\\[\AgdaEmptyExtraSkip]%
\>[0]\AgdaFunction{Pred-≡→⊆}\AgdaSpace{}%
\AgdaSymbol{:}\AgdaSpace{}%
\AgdaSymbol{\{}\AgdaBound{𝓤}\AgdaSpace{}%
\AgdaBound{𝓦}\AgdaSpace{}%
\AgdaSymbol{:}\AgdaSpace{}%
\AgdaPostulate{Universe}\AgdaSymbol{\}\{}\AgdaBound{A}\AgdaSpace{}%
\AgdaSymbol{:}\AgdaSpace{}%
\AgdaBound{𝓤}\AgdaSpace{}%
\AgdaOperator{\AgdaFunction{̇}}\AgdaSymbol{\}\{}\AgdaBound{P}\AgdaSpace{}%
\AgdaBound{Q}\AgdaSpace{}%
\AgdaSymbol{:}\AgdaSpace{}%
\AgdaFunction{Pred}\AgdaSpace{}%
\AgdaBound{A}\AgdaSpace{}%
\AgdaBound{𝓦}\AgdaSymbol{\}}\<%
\\
\>[0][@{}l@{\AgdaIndent{0}}]%
\>[1]\AgdaSymbol{→}%
\>[12]\AgdaBound{P}\AgdaSpace{}%
\AgdaOperator{\AgdaDatatype{≡}}\AgdaSpace{}%
\AgdaBound{Q}\AgdaSpace{}%
\AgdaSymbol{→}\AgdaSpace{}%
\AgdaSymbol{(}\AgdaBound{P}\AgdaSpace{}%
\AgdaOperator{\AgdaFunction{⊆}}\AgdaSpace{}%
\AgdaBound{Q}\AgdaSymbol{)}\<%
\\
\>[0]\AgdaFunction{Pred-≡→⊆}\AgdaSpace{}%
\AgdaSymbol{(}\AgdaInductiveConstructor{refl}\AgdaSpace{}%
\AgdaSymbol{\AgdaUnderscore{})}\AgdaSpace{}%
\AgdaSymbol{=}\AgdaSpace{}%
\AgdaSymbol{(λ}\AgdaSpace{}%
\AgdaBound{z}\AgdaSpace{}%
\AgdaSymbol{→}\AgdaSpace{}%
\AgdaBound{z}\AgdaSymbol{)}\<%
\\
%
\\[\AgdaEmptyExtraSkip]%
\>[0]\AgdaFunction{Pred-≡→⊇}\AgdaSpace{}%
\AgdaSymbol{:}\AgdaSpace{}%
\AgdaSymbol{\{}\AgdaBound{𝓤}\AgdaSpace{}%
\AgdaBound{𝓦}\AgdaSpace{}%
\AgdaSymbol{:}\AgdaSpace{}%
\AgdaPostulate{Universe}\AgdaSymbol{\}\{}\AgdaBound{A}\AgdaSpace{}%
\AgdaSymbol{:}\AgdaSpace{}%
\AgdaBound{𝓤}\AgdaSpace{}%
\AgdaOperator{\AgdaFunction{̇}}\AgdaSymbol{\}\{}\AgdaBound{P}\AgdaSpace{}%
\AgdaBound{Q}\AgdaSpace{}%
\AgdaSymbol{:}\AgdaSpace{}%
\AgdaFunction{Pred}\AgdaSpace{}%
\AgdaBound{A}\AgdaSpace{}%
\AgdaBound{𝓦}\AgdaSymbol{\}}\<%
\\
\>[0][@{}l@{\AgdaIndent{0}}]%
\>[1]\AgdaSymbol{→}%
\>[12]\AgdaBound{P}\AgdaSpace{}%
\AgdaOperator{\AgdaDatatype{≡}}\AgdaSpace{}%
\AgdaBound{Q}\AgdaSpace{}%
\AgdaSymbol{→}\AgdaSpace{}%
\AgdaSymbol{(}\AgdaBound{P}\AgdaSpace{}%
\AgdaOperator{\AgdaFunction{⊇}}\AgdaSpace{}%
\AgdaBound{Q}\AgdaSymbol{)}\<%
\\
\>[0]\AgdaFunction{Pred-≡→⊇}\AgdaSpace{}%
\AgdaSymbol{(}\AgdaInductiveConstructor{refl}\AgdaSpace{}%
\AgdaSymbol{\AgdaUnderscore{})}\AgdaSpace{}%
\AgdaSymbol{=}\AgdaSpace{}%
\AgdaSymbol{(λ}\AgdaSpace{}%
\AgdaBound{z}\AgdaSpace{}%
\AgdaSymbol{→}\AgdaSpace{}%
\AgdaBound{z}\AgdaSymbol{)}\<%
\\
%
\\[\AgdaEmptyExtraSkip]%
\>[0]\AgdaFunction{Pred-=̇-≡}\AgdaSpace{}%
\AgdaSymbol{:}\AgdaSpace{}%
\AgdaSymbol{\{}\AgdaBound{𝓤}\AgdaSpace{}%
\AgdaBound{𝓦}\AgdaSpace{}%
\AgdaSymbol{:}\AgdaSpace{}%
\AgdaPostulate{Universe}\AgdaSymbol{\}}\<%
\\
\>[0][@{}l@{\AgdaIndent{0}}]%
\>[1]\AgdaSymbol{→}%
\>[12]\AgdaFunction{propext}\AgdaSpace{}%
\AgdaBound{𝓦}\AgdaSpace{}%
\AgdaSymbol{→}\AgdaSpace{}%
\AgdaFunction{global-dfunext}\<%
\\
%
\>[1]\AgdaSymbol{→}%
\>[12]\AgdaSymbol{\{}\AgdaBound{A}\AgdaSpace{}%
\AgdaSymbol{:}\AgdaSpace{}%
\AgdaBound{𝓤}\AgdaSpace{}%
\AgdaOperator{\AgdaFunction{̇}}\AgdaSymbol{\}\{}\AgdaBound{P}\AgdaSpace{}%
\AgdaBound{Q}\AgdaSpace{}%
\AgdaSymbol{:}\AgdaSpace{}%
\AgdaFunction{Pred}\AgdaSpace{}%
\AgdaBound{A}\AgdaSpace{}%
\AgdaBound{𝓦}\AgdaSymbol{\}}\<%
\\
%
\>[1]\AgdaSymbol{→}%
\>[12]\AgdaSymbol{((}\AgdaBound{x}\AgdaSpace{}%
\AgdaSymbol{:}\AgdaSpace{}%
\AgdaBound{A}\AgdaSymbol{)}\AgdaSpace{}%
\AgdaSymbol{→}\AgdaSpace{}%
\AgdaFunction{is-subsingleton}\AgdaSpace{}%
\AgdaSymbol{(}\AgdaBound{P}\AgdaSpace{}%
\AgdaBound{x}\AgdaSymbol{))}\<%
\\
%
\>[1]\AgdaSymbol{→}%
\>[12]\AgdaSymbol{((}\AgdaBound{x}\AgdaSpace{}%
\AgdaSymbol{:}\AgdaSpace{}%
\AgdaBound{A}\AgdaSymbol{)}\AgdaSpace{}%
\AgdaSymbol{→}\AgdaSpace{}%
\AgdaFunction{is-subsingleton}\AgdaSpace{}%
\AgdaSymbol{(}\AgdaBound{Q}\AgdaSpace{}%
\AgdaBound{x}\AgdaSymbol{))}\<%
\\
%
\>[1]\AgdaSymbol{→}%
\>[12]\AgdaBound{P}\AgdaSpace{}%
\AgdaOperator{\AgdaFunction{=̇}}\AgdaSpace{}%
\AgdaBound{Q}\AgdaSpace{}%
\AgdaSymbol{→}\AgdaSpace{}%
\AgdaBound{P}\AgdaSpace{}%
\AgdaOperator{\AgdaDatatype{≡}}\AgdaSpace{}%
\AgdaBound{Q}\<%
\\
\>[0]\AgdaFunction{Pred-=̇-≡}\AgdaSpace{}%
\AgdaBound{pe}\AgdaSpace{}%
\AgdaBound{gfe}\AgdaSpace{}%
\AgdaSymbol{\{}\AgdaBound{A}\AgdaSymbol{\}\{}\AgdaBound{P}\AgdaSymbol{\}\{}\AgdaBound{Q}\AgdaSymbol{\}}\AgdaSpace{}%
\AgdaBound{ssP}\AgdaSpace{}%
\AgdaBound{ssQ}\AgdaSpace{}%
\AgdaSymbol{(}\AgdaBound{pq}\AgdaSpace{}%
\AgdaOperator{\AgdaInductiveConstructor{,}}\AgdaSpace{}%
\AgdaBound{qp}\AgdaSymbol{)}\AgdaSpace{}%
\AgdaSymbol{=}\AgdaSpace{}%
\AgdaBound{gfe}\AgdaSpace{}%
\AgdaFunction{γ}\<%
\\
\>[0][@{}l@{\AgdaIndent{0}}]%
\>[1]\AgdaKeyword{where}\<%
\\
\>[1][@{}l@{\AgdaIndent{0}}]%
\>[2]\AgdaFunction{γ}\AgdaSpace{}%
\AgdaSymbol{:}\AgdaSpace{}%
\AgdaSymbol{(}\AgdaBound{x}\AgdaSpace{}%
\AgdaSymbol{:}\AgdaSpace{}%
\AgdaBound{A}\AgdaSymbol{)}\AgdaSpace{}%
\AgdaSymbol{→}\AgdaSpace{}%
\AgdaBound{P}\AgdaSpace{}%
\AgdaBound{x}\AgdaSpace{}%
\AgdaOperator{\AgdaDatatype{≡}}\AgdaSpace{}%
\AgdaBound{Q}\AgdaSpace{}%
\AgdaBound{x}\<%
\\
%
\>[2]\AgdaFunction{γ}\AgdaSpace{}%
\AgdaBound{x}\AgdaSpace{}%
\AgdaSymbol{=}\AgdaSpace{}%
\AgdaBound{pe}\AgdaSpace{}%
\AgdaSymbol{(}\AgdaBound{ssP}\AgdaSpace{}%
\AgdaBound{x}\AgdaSymbol{)}\AgdaSpace{}%
\AgdaSymbol{(}\AgdaBound{ssQ}\AgdaSpace{}%
\AgdaBound{x}\AgdaSymbol{)}\AgdaSpace{}%
\AgdaBound{pq}\AgdaSpace{}%
\AgdaBound{qp}\<%
\\
%
\\[\AgdaEmptyExtraSkip]%
\>[0]\AgdaComment{-- Disjoint Union.}\<%
\\
\>[0]\AgdaKeyword{data}\AgdaSpace{}%
\AgdaOperator{\AgdaDatatype{\AgdaUnderscore{}⊎\AgdaUnderscore{}}}\AgdaSpace{}%
\AgdaSymbol{\{}\AgdaBound{𝓤}\AgdaSpace{}%
\AgdaBound{𝓦}\AgdaSpace{}%
\AgdaSymbol{:}\AgdaSpace{}%
\AgdaPostulate{Universe}\AgdaSymbol{\}(}\AgdaBound{A}\AgdaSpace{}%
\AgdaSymbol{:}\AgdaSpace{}%
\AgdaBound{𝓤}\AgdaSpace{}%
\AgdaOperator{\AgdaFunction{̇}}\AgdaSymbol{)}\AgdaSpace{}%
\AgdaSymbol{(}\AgdaBound{B}\AgdaSpace{}%
\AgdaSymbol{:}\AgdaSpace{}%
\AgdaBound{𝓦}\AgdaSpace{}%
\AgdaOperator{\AgdaFunction{̇}}\AgdaSymbol{)}\AgdaSpace{}%
\AgdaSymbol{:}\AgdaSpace{}%
\AgdaBound{𝓤}\AgdaSpace{}%
\AgdaOperator{\AgdaPrimitive{⊔}}\AgdaSpace{}%
\AgdaBound{𝓦}\AgdaSpace{}%
\AgdaOperator{\AgdaFunction{̇}}\AgdaSpace{}%
\AgdaKeyword{where}\<%
\\
\>[0][@{}l@{\AgdaIndent{0}}]%
\>[1]\AgdaInductiveConstructor{inj₁}\AgdaSpace{}%
\AgdaSymbol{:}\AgdaSpace{}%
\AgdaSymbol{(}\AgdaBound{x}\AgdaSpace{}%
\AgdaSymbol{:}\AgdaSpace{}%
\AgdaBound{A}\AgdaSymbol{)}\AgdaSpace{}%
\AgdaSymbol{→}\AgdaSpace{}%
\AgdaBound{A}\AgdaSpace{}%
\AgdaOperator{\AgdaDatatype{⊎}}\AgdaSpace{}%
\AgdaBound{B}\<%
\\
%
\>[1]\AgdaInductiveConstructor{inj₂}\AgdaSpace{}%
\AgdaSymbol{:}\AgdaSpace{}%
\AgdaSymbol{(}\AgdaBound{y}\AgdaSpace{}%
\AgdaSymbol{:}\AgdaSpace{}%
\AgdaBound{B}\AgdaSymbol{)}\AgdaSpace{}%
\AgdaSymbol{→}\AgdaSpace{}%
\AgdaBound{A}\AgdaSpace{}%
\AgdaOperator{\AgdaDatatype{⊎}}\AgdaSpace{}%
\AgdaBound{B}\<%
\\
\>[0]\AgdaKeyword{infixr}\AgdaSpace{}%
\AgdaNumber{1}\AgdaSpace{}%
\AgdaOperator{\AgdaDatatype{\AgdaUnderscore{}⊎\AgdaUnderscore{}}}\<%
\\
%
\\[\AgdaEmptyExtraSkip]%
\>[0]\AgdaComment{-- Union.}\<%
\\
\>[0]\AgdaOperator{\AgdaFunction{\AgdaUnderscore{}∪\AgdaUnderscore{}}}\AgdaSpace{}%
\AgdaSymbol{:}\AgdaSpace{}%
\AgdaSymbol{\{}\AgdaBound{𝓤}\AgdaSpace{}%
\AgdaBound{𝓦}\AgdaSpace{}%
\AgdaBound{𝓣}\AgdaSpace{}%
\AgdaSymbol{:}\AgdaSpace{}%
\AgdaPostulate{Universe}\AgdaSymbol{\}\{}\AgdaBound{A}\AgdaSpace{}%
\AgdaSymbol{:}\AgdaSpace{}%
\AgdaBound{𝓤}\AgdaSpace{}%
\AgdaOperator{\AgdaFunction{̇}}\AgdaSymbol{\}}\AgdaSpace{}%
\AgdaSymbol{→}\AgdaSpace{}%
\AgdaFunction{Pred}\AgdaSpace{}%
\AgdaBound{A}\AgdaSpace{}%
\AgdaBound{𝓦}\AgdaSpace{}%
\AgdaSymbol{→}\AgdaSpace{}%
\AgdaFunction{Pred}\AgdaSpace{}%
\AgdaBound{A}\AgdaSpace{}%
\AgdaBound{𝓣}\AgdaSpace{}%
\AgdaSymbol{→}\AgdaSpace{}%
\AgdaFunction{Pred}\AgdaSpace{}%
\AgdaBound{A}\AgdaSpace{}%
\AgdaSymbol{\AgdaUnderscore{}}\<%
\\
\>[0]\AgdaBound{P}\AgdaSpace{}%
\AgdaOperator{\AgdaFunction{∪}}\AgdaSpace{}%
\AgdaBound{Q}\AgdaSpace{}%
\AgdaSymbol{=}\AgdaSpace{}%
\AgdaSymbol{λ}\AgdaSpace{}%
\AgdaBound{x}\AgdaSpace{}%
\AgdaSymbol{→}\AgdaSpace{}%
\AgdaBound{x}\AgdaSpace{}%
\AgdaOperator{\AgdaFunction{∈}}\AgdaSpace{}%
\AgdaBound{P}\AgdaSpace{}%
\AgdaOperator{\AgdaDatatype{⊎}}\AgdaSpace{}%
\AgdaBound{x}\AgdaSpace{}%
\AgdaOperator{\AgdaFunction{∈}}\AgdaSpace{}%
\AgdaBound{Q}\<%
\\
\>[0]\AgdaKeyword{infixr}\AgdaSpace{}%
\AgdaNumber{1}\AgdaSpace{}%
\AgdaOperator{\AgdaFunction{\AgdaUnderscore{}∪\AgdaUnderscore{}}}\<%
\\
%
\\[\AgdaEmptyExtraSkip]%
\>[0]\AgdaComment{-- The empty set.}\<%
\\
\>[0]\AgdaFunction{∅}\AgdaSpace{}%
\AgdaSymbol{:}\AgdaSpace{}%
\AgdaSymbol{\{}\AgdaBound{𝓤}\AgdaSpace{}%
\AgdaSymbol{:}\AgdaSpace{}%
\AgdaPostulate{Universe}\AgdaSymbol{\}\{}\AgdaBound{A}\AgdaSpace{}%
\AgdaSymbol{:}\AgdaSpace{}%
\AgdaBound{𝓤}\AgdaSpace{}%
\AgdaOperator{\AgdaFunction{̇}}\AgdaSymbol{\}}\AgdaSpace{}%
\AgdaSymbol{→}\AgdaSpace{}%
\AgdaFunction{Pred}\AgdaSpace{}%
\AgdaBound{A}\AgdaSpace{}%
\AgdaPrimitive{𝓤₀}\<%
\\
\>[0]\AgdaFunction{∅}\AgdaSpace{}%
\AgdaSymbol{=}\AgdaSpace{}%
\AgdaSymbol{λ}\AgdaSpace{}%
\AgdaBound{\AgdaUnderscore{}}\AgdaSpace{}%
\AgdaSymbol{→}\AgdaSpace{}%
\AgdaFunction{𝟘}\<%
\\
%
\\[\AgdaEmptyExtraSkip]%
\>[0]\AgdaComment{-- Singletons.}\<%
\\
\>[0]\AgdaOperator{\AgdaFunction{{\AgdaUnderscore{}}}}\AgdaSpace{}%
\AgdaSymbol{:}\AgdaSpace{}%
\AgdaSymbol{\{}\AgdaBound{𝓤}\AgdaSpace{}%
\AgdaSymbol{:}\AgdaSpace{}%
\AgdaPostulate{Universe}\AgdaSymbol{\}\{}\AgdaBound{A}\AgdaSpace{}%
\AgdaSymbol{:}\AgdaSpace{}%
\AgdaBound{𝓤}\AgdaSpace{}%
\AgdaOperator{\AgdaFunction{̇}}\AgdaSymbol{\}}\AgdaSpace{}%
\AgdaSymbol{→}\AgdaSpace{}%
\AgdaBound{A}\AgdaSpace{}%
\AgdaSymbol{→}\AgdaSpace{}%
\AgdaFunction{Pred}\AgdaSpace{}%
\AgdaBound{A}\AgdaSpace{}%
\AgdaSymbol{\AgdaUnderscore{}}\<%
\\
\>[0]\AgdaOperator{\AgdaFunction{{}}\AgdaSpace{}%
\AgdaBound{x}\AgdaSpace{}%
\AgdaOperator{\AgdaFunction{}}}\AgdaSpace{}%
\AgdaSymbol{=}\AgdaSpace{}%
\AgdaBound{x}\AgdaSpace{}%
\AgdaOperator{\AgdaDatatype{≡\AgdaUnderscore{}}}\<%
\\
%
\\[\AgdaEmptyExtraSkip]%
\>[0]\AgdaOperator{\AgdaFunction{Im\AgdaUnderscore{}⊆\AgdaUnderscore{}}}\AgdaSpace{}%
\AgdaSymbol{:}\AgdaSpace{}%
\AgdaSymbol{\{}\AgdaBound{𝓤}\AgdaSpace{}%
\AgdaBound{𝓦}\AgdaSpace{}%
\AgdaBound{𝓣}\AgdaSpace{}%
\AgdaSymbol{:}\AgdaSpace{}%
\AgdaPostulate{Universe}\AgdaSymbol{\}}\AgdaSpace{}%
\AgdaSymbol{\{}\AgdaBound{A}\AgdaSpace{}%
\AgdaSymbol{:}\AgdaSpace{}%
\AgdaBound{𝓤}\AgdaSpace{}%
\AgdaOperator{\AgdaFunction{̇}}\AgdaSpace{}%
\AgdaSymbol{\}}\AgdaSpace{}%
\AgdaSymbol{\{}\AgdaBound{B}\AgdaSpace{}%
\AgdaSymbol{:}\AgdaSpace{}%
\AgdaBound{𝓦}\AgdaSpace{}%
\AgdaOperator{\AgdaFunction{̇}}\AgdaSpace{}%
\AgdaSymbol{\}}\AgdaSpace{}%
\AgdaSymbol{→}\AgdaSpace{}%
\AgdaSymbol{(}\AgdaBound{A}\AgdaSpace{}%
\AgdaSymbol{→}\AgdaSpace{}%
\AgdaBound{B}\AgdaSymbol{)}\AgdaSpace{}%
\AgdaSymbol{→}\AgdaSpace{}%
\AgdaFunction{Pred}\AgdaSpace{}%
\AgdaBound{B}\AgdaSpace{}%
\AgdaBound{𝓣}\AgdaSpace{}%
\AgdaSymbol{→}\AgdaSpace{}%
\AgdaBound{𝓤}\AgdaSpace{}%
\AgdaOperator{\AgdaPrimitive{⊔}}\AgdaSpace{}%
\AgdaBound{𝓣}\AgdaSpace{}%
\AgdaOperator{\AgdaFunction{̇}}\<%
\\
\>[0]\AgdaOperator{\AgdaFunction{Im\AgdaUnderscore{}⊆\AgdaUnderscore{}}}\AgdaSpace{}%
\AgdaSymbol{\{}\AgdaArgument{A}\AgdaSpace{}%
\AgdaSymbol{=}\AgdaSpace{}%
\AgdaBound{A}\AgdaSymbol{\}}\AgdaSpace{}%
\AgdaBound{f}\AgdaSpace{}%
\AgdaBound{S}\AgdaSpace{}%
\AgdaSymbol{=}\AgdaSpace{}%
\AgdaSymbol{(}\AgdaBound{x}\AgdaSpace{}%
\AgdaSymbol{:}\AgdaSpace{}%
\AgdaBound{A}\AgdaSymbol{)}\AgdaSpace{}%
\AgdaSymbol{→}\AgdaSpace{}%
\AgdaBound{f}\AgdaSpace{}%
\AgdaBound{x}\AgdaSpace{}%
\AgdaOperator{\AgdaFunction{∈}}\AgdaSpace{}%
\AgdaBound{S}\<%
\\
%
\\[\AgdaEmptyExtraSkip]%
\>[0]\AgdaFunction{img}\AgdaSpace{}%
\AgdaSymbol{:}%
\>[600I]\AgdaSymbol{\{}\AgdaBound{𝓤}\AgdaSpace{}%
\AgdaSymbol{:}\AgdaSpace{}%
\AgdaPostulate{Universe}\AgdaSymbol{\}\{}\AgdaBound{X}\AgdaSpace{}%
\AgdaSymbol{:}\AgdaSpace{}%
\AgdaBound{𝓤}\AgdaSpace{}%
\AgdaOperator{\AgdaFunction{̇}}\AgdaSpace{}%
\AgdaSymbol{\}}\AgdaSpace{}%
\AgdaSymbol{\{}\AgdaBound{Y}\AgdaSpace{}%
\AgdaSymbol{:}\AgdaSpace{}%
\AgdaBound{𝓤}\AgdaSpace{}%
\AgdaOperator{\AgdaFunction{̇}}\AgdaSpace{}%
\AgdaSymbol{\}}\<%
\\
\>[.][@{}l@{}]\<[600I]%
\>[6]\AgdaSymbol{(}\AgdaBound{f}\AgdaSpace{}%
\AgdaSymbol{:}\AgdaSpace{}%
\AgdaBound{X}\AgdaSpace{}%
\AgdaSymbol{→}\AgdaSpace{}%
\AgdaBound{Y}\AgdaSymbol{)}\AgdaSpace{}%
\AgdaSymbol{(}\AgdaBound{P}\AgdaSpace{}%
\AgdaSymbol{:}\AgdaSpace{}%
\AgdaFunction{Pred}\AgdaSpace{}%
\AgdaBound{Y}\AgdaSpace{}%
\AgdaBound{𝓤}\AgdaSymbol{)}\<%
\\
\>[0][@{}l@{\AgdaIndent{0}}]%
\>[1]\AgdaSymbol{→}%
\>[6]\AgdaOperator{\AgdaFunction{Im}}\AgdaSpace{}%
\AgdaBound{f}\AgdaSpace{}%
\AgdaOperator{\AgdaFunction{⊆}}\AgdaSpace{}%
\AgdaBound{P}\AgdaSpace{}%
\AgdaSymbol{→}%
\>[18]\AgdaBound{X}\AgdaSpace{}%
\AgdaSymbol{→}\AgdaSpace{}%
\AgdaRecord{Σ}\AgdaSpace{}%
\AgdaBound{P}\<%
\\
\>[0]\AgdaFunction{img}\AgdaSpace{}%
\AgdaSymbol{\{}\AgdaArgument{Y}\AgdaSpace{}%
\AgdaSymbol{=}\AgdaSpace{}%
\AgdaBound{Y}\AgdaSymbol{\}}\AgdaSpace{}%
\AgdaBound{f}\AgdaSpace{}%
\AgdaBound{P}\AgdaSpace{}%
\AgdaBound{Imf⊆P}\AgdaSpace{}%
\AgdaSymbol{=}\AgdaSpace{}%
\AgdaSymbol{λ}\AgdaSpace{}%
\AgdaBound{x₁}\AgdaSpace{}%
\AgdaSymbol{→}\AgdaSpace{}%
\AgdaBound{f}\AgdaSpace{}%
\AgdaBound{x₁}\AgdaSpace{}%
\AgdaOperator{\AgdaInductiveConstructor{,}}\AgdaSpace{}%
\AgdaBound{Imf⊆P}\AgdaSpace{}%
\AgdaBound{x₁}\<%
\\
\>[0]\<%
\end{code}

\paragraph{Predicate product and
transport}\label{predicate-product-and-transport}

The product \texttt{Π\ P} of a predicate \texttt{P\ :\ Pred\ X\ 𝓤} is
inhabited iff P x holds for all x : X.

\begin{code}%
\>[0]\<%
\\
\>[0]\AgdaFunction{ΠP-meaning}\AgdaSpace{}%
\AgdaSymbol{:}\AgdaSpace{}%
\AgdaSymbol{\{}\AgdaBound{𝓧}\AgdaSpace{}%
\AgdaBound{𝓤}\AgdaSpace{}%
\AgdaSymbol{:}\AgdaSpace{}%
\AgdaPostulate{Universe}\AgdaSymbol{\}\{}\AgdaBound{X}\AgdaSpace{}%
\AgdaSymbol{:}\AgdaSpace{}%
\AgdaBound{𝓧}\AgdaSpace{}%
\AgdaOperator{\AgdaFunction{̇}}\AgdaSymbol{\}\{}\AgdaBound{P}\AgdaSpace{}%
\AgdaSymbol{:}\AgdaSpace{}%
\AgdaFunction{Pred}\AgdaSpace{}%
\AgdaBound{X}\AgdaSpace{}%
\AgdaBound{𝓤}\AgdaSymbol{\}}\<%
\\
\>[0][@{}l@{\AgdaIndent{0}}]%
\>[1]\AgdaSymbol{→}%
\>[14]\AgdaFunction{Π}\AgdaSpace{}%
\AgdaBound{P}%
\>[19]\AgdaSymbol{→}%
\>[22]\AgdaSymbol{(}\AgdaBound{x}\AgdaSpace{}%
\AgdaSymbol{:}\AgdaSpace{}%
\AgdaBound{X}\AgdaSymbol{)}\AgdaSpace{}%
\AgdaSymbol{→}\AgdaSpace{}%
\AgdaBound{P}\AgdaSpace{}%
\AgdaBound{x}\<%
\\
\>[0]\AgdaFunction{ΠP-meaning}\AgdaSpace{}%
\AgdaBound{f}\AgdaSpace{}%
\AgdaBound{x}\AgdaSpace{}%
\AgdaSymbol{=}\AgdaSpace{}%
\AgdaBound{f}\AgdaSpace{}%
\AgdaBound{x}\<%
\\
\>[0]\<%
\end{code}

The following is a pair of useful ``transport'' lemmas for predicates.

\begin{code}%
\>[0]\AgdaKeyword{module}\AgdaSpace{}%
\AgdaModule{\AgdaUnderscore{}}\AgdaSpace{}%
\AgdaSymbol{\{}\AgdaBound{𝓤}\AgdaSpace{}%
\AgdaBound{𝓦}\AgdaSpace{}%
\AgdaSymbol{:}\AgdaSpace{}%
\AgdaPostulate{Universe}\AgdaSymbol{\}}\AgdaSpace{}%
\AgdaKeyword{where}\<%
\\
%
\\[\AgdaEmptyExtraSkip]%
\>[0][@{}l@{\AgdaIndent{0}}]%
\>[1]\AgdaFunction{cong-app-pred}%
\>[672I]\AgdaSymbol{:}%
\>[673I]\AgdaSymbol{\{}\AgdaBound{A}\AgdaSpace{}%
\AgdaSymbol{:}\AgdaSpace{}%
\AgdaBound{𝓤}\AgdaSpace{}%
\AgdaOperator{\AgdaFunction{̇}}\AgdaSpace{}%
\AgdaSymbol{\}\{}\AgdaBound{B₁}\AgdaSpace{}%
\AgdaBound{B₂}\AgdaSpace{}%
\AgdaSymbol{:}\AgdaSpace{}%
\AgdaFunction{Pred}\AgdaSpace{}%
\AgdaBound{A}\AgdaSpace{}%
\AgdaBound{𝓦}\AgdaSymbol{\}}\<%
\\
\>[.][@{}l@{}]\<[673I]%
\>[17]\AgdaSymbol{(}\AgdaBound{x}\AgdaSpace{}%
\AgdaSymbol{:}\AgdaSpace{}%
\AgdaBound{A}\AgdaSymbol{)}\AgdaSpace{}%
\AgdaSymbol{→}%
\>[28]\AgdaBound{x}\AgdaSpace{}%
\AgdaOperator{\AgdaFunction{∈}}\AgdaSpace{}%
\AgdaBound{B₁}%
\>[36]\AgdaSymbol{→}%
\>[39]\AgdaBound{B₁}\AgdaSpace{}%
\AgdaOperator{\AgdaDatatype{≡}}\AgdaSpace{}%
\AgdaBound{B₂}\<%
\\
\>[672I][@{}l@{\AgdaIndent{0}}]%
\>[16]\AgdaComment{------------------------------}\<%
\\
\>[1][@{}l@{\AgdaIndent{0}}]%
\>[2]\AgdaSymbol{→}%
\>[28]\AgdaBound{x}\AgdaSpace{}%
\AgdaOperator{\AgdaFunction{∈}}\AgdaSpace{}%
\AgdaBound{B₂}\<%
\\
%
\>[1]\AgdaFunction{cong-app-pred}\AgdaSpace{}%
\AgdaBound{x}\AgdaSpace{}%
\AgdaBound{x∈B₁}\AgdaSpace{}%
\AgdaSymbol{(}\AgdaInductiveConstructor{refl}\AgdaSpace{}%
\AgdaSymbol{\AgdaUnderscore{}}\AgdaSpace{}%
\AgdaSymbol{)}\AgdaSpace{}%
\AgdaSymbol{=}\AgdaSpace{}%
\AgdaBound{x∈B₁}\<%
\\
%
\\[\AgdaEmptyExtraSkip]%
%
\>[1]\AgdaFunction{cong-pred}\AgdaSpace{}%
\AgdaSymbol{:}%
\>[700I]\AgdaSymbol{\{}\AgdaBound{A}\AgdaSpace{}%
\AgdaSymbol{:}\AgdaSpace{}%
\AgdaBound{𝓤}\AgdaSpace{}%
\AgdaOperator{\AgdaFunction{̇}}\AgdaSpace{}%
\AgdaSymbol{\}\{}\AgdaBound{B}\AgdaSpace{}%
\AgdaSymbol{:}\AgdaSpace{}%
\AgdaFunction{Pred}\AgdaSpace{}%
\AgdaBound{A}\AgdaSpace{}%
\AgdaBound{𝓦}\AgdaSymbol{\}}\<%
\\
\>[.][@{}l@{}]\<[700I]%
\>[13]\AgdaSymbol{(}\AgdaBound{x}\AgdaSpace{}%
\AgdaBound{y}\AgdaSpace{}%
\AgdaSymbol{:}\AgdaSpace{}%
\AgdaBound{A}\AgdaSymbol{)}\AgdaSpace{}%
\AgdaSymbol{→}%
\>[26]\AgdaBound{x}\AgdaSpace{}%
\AgdaOperator{\AgdaFunction{∈}}\AgdaSpace{}%
\AgdaBound{B}%
\>[33]\AgdaSymbol{→}%
\>[36]\AgdaBound{x}\AgdaSpace{}%
\AgdaOperator{\AgdaDatatype{≡}}\AgdaSpace{}%
\AgdaBound{y}\<%
\\
%
\>[13]\AgdaComment{----------------------------}\<%
\\
\>[1][@{}l@{\AgdaIndent{0}}]%
\>[2]\AgdaSymbol{→}%
\>[26]\AgdaBound{y}\AgdaSpace{}%
\AgdaOperator{\AgdaFunction{∈}}\AgdaSpace{}%
\AgdaBound{B}\<%
\\
%
\>[1]\AgdaFunction{cong-pred}\AgdaSpace{}%
\AgdaBound{x}\AgdaSpace{}%
\AgdaDottedPattern{\AgdaSymbol{.}}\AgdaDottedPattern{\AgdaBound{x}}\AgdaSpace{}%
\AgdaBound{x∈B}\AgdaSpace{}%
\AgdaSymbol{(}\AgdaInductiveConstructor{refl}\AgdaSpace{}%
\AgdaSymbol{\AgdaUnderscore{}}\AgdaSpace{}%
\AgdaSymbol{)}\AgdaSpace{}%
\AgdaSymbol{=}\AgdaSpace{}%
\AgdaBound{x∈B}\<%
\end{code}

\begin{center}\rule{0.5\linewidth}{\linethickness}\end{center}

\href{UALib.Relations.html}{↑ UALib.Relations}
{\href{UALib.Relations.Binary.html}{UALib.Relations.Binary →}}

\{\% include UALib.Links.md \%\}


\newcommand{\ovu}{\ab 𝓞 \af ⊔ \ab 𝓥 \af ⊔ \ab 𝓤 \af ⁺ \af ̇}
We need a mechanism for implementing the notion of subsets in Agda. A typical one is called \af{Pred} (for predicate). More generally, \af{Pred} \ab A \ab 𝓤 can be viewed as the type of a property that elements of type \ab{A} might satisfy. We write \ab P \as : \af{Pred} \ab A \ab 𝓤 to represent the semantic concept of a collection of elements of type \ab{A} that satisfy the property \ab{P}.
Here is the definition, which is similar to the one found in the \texttt{Relation/Unary.agda} file of the \agdastdlib.
\ccpad
\begin{code}%
\>[1]\AgdaFunction{Pred}\AgdaSpace{}%
\AgdaSymbol{:}\AgdaSpace{}%
\AgdaBound{𝓤}\AgdaSpace{}%
\AgdaOperator{\AgdaFunction{̇}}\AgdaSpace{}%
\AgdaSymbol{→}\AgdaSpace{}%
\AgdaSymbol{(}\AgdaBound{𝓥}\AgdaSpace{}%
\AgdaSymbol{:}\AgdaSpace{}%
\AgdaPostulate{Universe}\AgdaSymbol{)}\AgdaSpace{}%
\AgdaSymbol{→}\AgdaSpace{}%
\AgdaBound{𝓤}\AgdaSpace{}%
\AgdaOperator{\AgdaPrimitive{⊔}}\AgdaSpace{}%
\AgdaBound{𝓥}\AgdaSpace{}%
\AgdaOperator{\AgdaPrimitive{⁺}}\AgdaSpace{}%
\AgdaOperator{\AgdaFunction{̇}}\<%
\\
%
\>[1]\AgdaFunction{Pred}\AgdaSpace{}%
\AgdaBound{A}\AgdaSpace{}%
\AgdaBound{𝓥}\AgdaSpace{}%
\AgdaSymbol{=}\AgdaSpace{}%
\AgdaBound{A}\AgdaSpace{}%
\AgdaSymbol{→}\AgdaSpace{}%
\AgdaBound{𝓥}\AgdaSpace{}%
\AgdaOperator{\AgdaFunction{̇}}\<%
\end{code}
\ccpad
Below we will often consider predicates over the class of all algebras of a particular type. We will define the type of algebras \af{Algebra} \ab 𝓤 \ab 𝑆 (for some universe level \ab 𝓤). Like all types, \af{Algebra} \ab 𝓤 \ab 𝑆 itself has a type which happens to be \ovu (see Section~\ref{algebra-types-1}). Therefore, the type of \af{Pred} (\af{Algebra} \ab 𝓤 \ab 𝑆)\ab 𝓤 is \ovu as well.  By definition, the inhabitants of the type \af{Pred} (\af{Algebra} \ab 𝓤 \ab 𝑆) \ab 𝓤 are maps of the form \ab 𝑨 \as → \ab 𝓤 \af ̇.
%% ; given an algebra \ab 𝑨 \as : \af{Algebra} \ab 𝓤 \ab 𝑆, we have \af{Pred} \ab 𝑨 \ab 𝓤 \as = \ab 𝑨 \as → \ab 𝓤 \af ̇.

\subsubsection{The membership relation}\label{the-membership-relation}
Notation for denoting that \ab x ``belongs to'' or ``inhabits'' type \ab{P}, or that \ab x ``has property'' \ab{P},is standard. We write either \ab x \af ∈ \ab P or \ab P \ab x (cf. \texttt{Relation/Unary.agda} in the \agdastdlib).
Similarly, the ``subset'' relation is denoted, as usual, with the \af{⊆} symbol (cf. \texttt{Relation/Unary.agda} in the \agdastdlib). The relations \af ∈ and \af{⊆} are defined in the \agdaualib as follows.
\ccpad
\begin{code}%
\>[1]\AgdaOperator{\AgdaFunction{\AgdaUnderscore{}∈\AgdaUnderscore{}}}\AgdaSpace{}%
\AgdaSymbol{:}\AgdaSpace{}%
\AgdaSymbol{\{}\AgdaBound{A}\AgdaSpace{}%
\AgdaSymbol{:}\AgdaSpace{}%
\AgdaBound{𝓤}\AgdaSpace{}%
\AgdaOperator{\AgdaFunction{̇}}\AgdaSpace{}%
\AgdaSymbol{\}}\AgdaSpace{}%
\AgdaSymbol{→}\AgdaSpace{}%
\AgdaBound{A}\AgdaSpace{}%
\AgdaSymbol{→}\AgdaSpace{}%
\AgdaFunction{Pred}\AgdaSpace{}%
\AgdaBound{A}\AgdaSpace{}%
\AgdaBound{𝓦}\AgdaSpace{}%
\AgdaSymbol{→}\AgdaSpace{}%
\AgdaBound{𝓦}\AgdaSpace{}%
\AgdaOperator{\AgdaFunction{̇}}\<%
\\
%
\>[1]\AgdaBound{x}\AgdaSpace{}%
\AgdaOperator{\AgdaFunction{∈}}\AgdaSpace{}%
\AgdaBound{P}\AgdaSpace{}%
\AgdaSymbol{=}\AgdaSpace{}%
\AgdaBound{P}\AgdaSpace{}%
\AgdaBound{x}\<%
\\
\\
\>[0]\AgdaOperator{\AgdaFunction{\AgdaUnderscore{}⊆\AgdaUnderscore{}}}\AgdaSpace{}%
\AgdaSymbol{:}\AgdaSpace{}%
\AgdaSymbol{\{}\AgdaBound{𝓤}\AgdaSpace{}%
\AgdaBound{𝓦}\AgdaSpace{}%
\AgdaBound{𝓣}\AgdaSpace{}%
\AgdaSymbol{:}\AgdaSpace{}%
\AgdaPostulate{Universe}\AgdaSymbol{\}\{}\AgdaBound{A}\AgdaSpace{}%
\AgdaSymbol{:}\AgdaSpace{}%
\AgdaBound{𝓤}\AgdaSpace{}%
\AgdaOperator{\AgdaFunction{̇}}\AgdaSpace{}%
\AgdaSymbol{\}}\AgdaSpace{}%
\AgdaSymbol{→}\AgdaSpace{}%
\AgdaFunction{Pred}\AgdaSpace{}%
\AgdaBound{A}\AgdaSpace{}%
\AgdaBound{𝓦}\AgdaSpace{}%
\AgdaSymbol{→}\AgdaSpace{}%
\AgdaFunction{Pred}\AgdaSpace{}%
\AgdaBound{A}\AgdaSpace{}%
\AgdaBound{𝓣}\AgdaSpace{}%
\AgdaSymbol{→}\AgdaSpace{}%
\AgdaBound{𝓤}\AgdaSpace{}%
\AgdaOperator{\AgdaPrimitive{⊔}}\AgdaSpace{}%
\AgdaBound{𝓦}\AgdaSpace{}%
\AgdaOperator{\AgdaPrimitive{⊔}}\AgdaSpace{}%
\AgdaBound{𝓣}\AgdaSpace{}%
\AgdaOperator{\AgdaFunction{̇}}\<%
\\
\>[0]\AgdaBound{P}\AgdaSpace{}%
\AgdaOperator{\AgdaFunction{⊆}}\AgdaSpace{}%
\AgdaBound{Q}\AgdaSpace{}%
\AgdaSymbol{=}\AgdaSpace{}%
\AgdaSymbol{∀}\AgdaSpace{}%
\AgdaSymbol{\{}\AgdaBound{x}\AgdaSymbol{\}}\AgdaSpace{}%
\AgdaSymbol{→}\AgdaSpace{}%
\AgdaBound{x}\AgdaSpace{}%
\AgdaOperator{\AgdaFunction{∈}}\AgdaSpace{}%
\AgdaBound{P}\AgdaSpace{}%
\AgdaSymbol{→}\AgdaSpace{}%
\AgdaBound{x}\AgdaSpace{}%
\AgdaOperator{\AgdaFunction{∈}}\AgdaSpace{}%
\AgdaBound{Q}\<%
\end{code}
\ccpad
In type theory everything is a type. As we have just seen, this includes subsets. Since the notion of equality for types is usually a nontrivial matter, it may be nontrivial to represent equality of subsets. Fortunately, it is straightforward to write down a type that represents what it means for two subsets to be equal in informal (pencil-paper) mathematics. In the \ualib we denote this \textit{subset equality} by =̇ and define as follows.
\ccpad
\begin{code}%
\>[0]\AgdaOperator{\AgdaFunction{\AgdaUnderscore{}=̇\AgdaUnderscore{}}}\AgdaSpace{}%
\AgdaSymbol{:}\AgdaSpace{}%
\AgdaSymbol{\{}\AgdaBound{𝓤}\AgdaSpace{}%
\AgdaBound{𝓦}\AgdaSpace{}%
\AgdaBound{𝓣}\AgdaSpace{}%
\AgdaSymbol{:}\AgdaSpace{}%
\AgdaPostulate{Universe}\AgdaSymbol{\}\{}\AgdaBound{A}\AgdaSpace{}%
\AgdaSymbol{:}\AgdaSpace{}%
\AgdaBound{𝓤}\AgdaSpace{}%
\AgdaOperator{\AgdaFunction{̇}}\AgdaSpace{}%
\AgdaSymbol{\}}\AgdaSpace{}%
\AgdaSymbol{→}\AgdaSpace{}%
\AgdaFunction{Pred}\AgdaSpace{}%
\AgdaBound{A}\AgdaSpace{}%
\AgdaBound{𝓦}\AgdaSpace{}%
\AgdaSymbol{→}\AgdaSpace{}%
\AgdaFunction{Pred}\AgdaSpace{}%
\AgdaBound{A}\AgdaSpace{}%
\AgdaBound{𝓣}\AgdaSpace{}%
\AgdaSymbol{→}\AgdaSpace{}%
\AgdaBound{𝓤}\AgdaSpace{}%
\AgdaOperator{\AgdaPrimitive{⊔}}\AgdaSpace{}%
\AgdaBound{𝓦}\AgdaSpace{}%
\AgdaOperator{\AgdaPrimitive{⊔}}\AgdaSpace{}%
\AgdaBound{𝓣}\AgdaSpace{}%
\AgdaOperator{\AgdaFunction{̇}}\<%
\\
\>[0]\AgdaBound{P}\AgdaSpace{}%
\AgdaOperator{\AgdaFunction{=̇}}\AgdaSpace{}%
\AgdaBound{Q}\AgdaSpace{}%
\AgdaSymbol{=}\AgdaSpace{}%
\AgdaSymbol{(}\AgdaBound{P}\AgdaSpace{}%
\AgdaOperator{\AgdaFunction{⊆}}\AgdaSpace{}%
\AgdaBound{Q}\AgdaSymbol{)}\AgdaSpace{}%
\AgdaOperator{\AgdaFunction{×}}\AgdaSpace{}%
\AgdaSymbol{(}\AgdaBound{Q}\AgdaSpace{}%
\AgdaOperator{\AgdaFunction{⊆}}\AgdaSpace{}%
\AgdaBound{P}\AgdaSymbol{)}\<%
\end{code}

%% -- END: \subsubsection{Unary Relation Types}\label{unary-relation-types}

This section presents the {[}UALib.Relations.Unary{]}{[}{]} module of
the {[}Agda Universal Algebra Library{]}{[}{]}.

We need a mechanism for implementing the notion of subsets in Agda. A
typical one is called \texttt{Pred} (for predicate). More generally,
\texttt{Pred\ A\ 𝓤} can be viewed as the type of a property that
elements of type \texttt{A} might satisfy. We write
\texttt{P\ :\ Pred\ A\ 𝓤} to represent the semantic concept of a
collection of elements of type \texttt{A} that satisfy the property
\texttt{P}.

\begin{code}%
\>[0]\<%
\\
\>[0]\AgdaSymbol{\{-\#}\AgdaSpace{}%
\AgdaKeyword{OPTIONS}\AgdaSpace{}%
\AgdaPragma{--without-K}\AgdaSpace{}%
\AgdaPragma{--exact-split}\AgdaSpace{}%
\AgdaPragma{--safe}\AgdaSpace{}%
\AgdaSymbol{\#-\}}\<%
\\
%
\\[\AgdaEmptyExtraSkip]%
\>[0]\AgdaKeyword{module}\AgdaSpace{}%
\AgdaModule{UALib.Relations.Unary}\AgdaSpace{}%
\AgdaKeyword{where}\<%
\\
%
\\[\AgdaEmptyExtraSkip]%
\>[0]\AgdaKeyword{open}\AgdaSpace{}%
\AgdaKeyword{import}\AgdaSpace{}%
\AgdaModule{UALib.Algebras.Lifts}\AgdaSpace{}%
\AgdaKeyword{public}\<%
\\
%
\\[\AgdaEmptyExtraSkip]%
\>[0]\AgdaKeyword{open}\AgdaSpace{}%
\AgdaKeyword{import}\AgdaSpace{}%
\AgdaModule{UALib.Prelude.Preliminaries}\AgdaSpace{}%
\AgdaKeyword{using}\AgdaSpace{}%
\AgdaSymbol{(}\AgdaFunction{¬}\AgdaSymbol{;}\AgdaSpace{}%
\AgdaFunction{propext}\AgdaSymbol{;}\AgdaSpace{}%
\AgdaFunction{global-dfunext}\AgdaSpace{}%
\AgdaSymbol{)}\AgdaSpace{}%
\AgdaKeyword{public}\<%
\\
\>[0]\<%
\end{code}

Here is the definition, which is similar to the one found in
the\texttt{Relation/Unary.agda} file of the {[}Agda Standard
Library{]}{[}{]}.

\begin{code}%
\>[0]\<%
\\
\>[0]\AgdaKeyword{module}\AgdaSpace{}%
\AgdaModule{\AgdaUnderscore{}}\AgdaSpace{}%
\AgdaSymbol{\{}\AgdaBound{𝓤}\AgdaSpace{}%
\AgdaSymbol{:}\AgdaSpace{}%
\AgdaPostulate{Universe}\AgdaSymbol{\}}\AgdaSpace{}%
\AgdaKeyword{where}\<%
\\
%
\\[\AgdaEmptyExtraSkip]%
\>[0][@{}l@{\AgdaIndent{0}}]%
\>[1]\AgdaFunction{Pred}\AgdaSpace{}%
\AgdaSymbol{:}\AgdaSpace{}%
\AgdaBound{𝓤}\AgdaSpace{}%
\AgdaOperator{\AgdaFunction{̇}}\AgdaSpace{}%
\AgdaSymbol{→}\AgdaSpace{}%
\AgdaSymbol{(}\AgdaBound{𝓥}\AgdaSpace{}%
\AgdaSymbol{:}\AgdaSpace{}%
\AgdaPostulate{Universe}\AgdaSymbol{)}\AgdaSpace{}%
\AgdaSymbol{→}\AgdaSpace{}%
\AgdaBound{𝓤}\AgdaSpace{}%
\AgdaOperator{\AgdaPrimitive{⊔}}\AgdaSpace{}%
\AgdaBound{𝓥}\AgdaSpace{}%
\AgdaOperator{\AgdaPrimitive{⁺}}\AgdaSpace{}%
\AgdaOperator{\AgdaFunction{̇}}\<%
\\
%
\>[1]\AgdaFunction{Pred}\AgdaSpace{}%
\AgdaBound{A}\AgdaSpace{}%
\AgdaBound{𝓥}\AgdaSpace{}%
\AgdaSymbol{=}\AgdaSpace{}%
\AgdaBound{A}\AgdaSpace{}%
\AgdaSymbol{→}\AgdaSpace{}%
\AgdaBound{𝓥}\AgdaSpace{}%
\AgdaOperator{\AgdaFunction{̇}}\<%
\\
\>[0]\<%
\end{code}

\paragraph{Unary relation truncation}\label{unary-relation-truncation}

The section on
\href{UALib.Prelude.Preliminaries.html\#truncation}{truncation} in the
module {[}UALib.Prelude.Preliminaries{]}{[}{]} describes the concepts of
\emph{truncation} and \emph{set} for ``proof-relevant'' mathematics.
Sometimes we will want to assume that a type is a \emph{set}. Recall,
this mean there is at most one proof that two elements are the same.
Analogously for predicates, we may wish to assume that there is at most
one proof that a given element satisfies the predicate.

\begin{code}%
\>[0]\<%
\\
\>[0][@{}l@{\AgdaIndent{1}}]%
\>[1]\AgdaFunction{Pred₀}\AgdaSpace{}%
\AgdaSymbol{:}\AgdaSpace{}%
\AgdaBound{𝓤}\AgdaSpace{}%
\AgdaOperator{\AgdaFunction{̇}}\AgdaSpace{}%
\AgdaSymbol{→}\AgdaSpace{}%
\AgdaSymbol{(}\AgdaBound{𝓥}\AgdaSpace{}%
\AgdaSymbol{:}\AgdaSpace{}%
\AgdaPostulate{Universe}\AgdaSymbol{)}\AgdaSpace{}%
\AgdaSymbol{→}\AgdaSpace{}%
\AgdaBound{𝓤}\AgdaSpace{}%
\AgdaOperator{\AgdaPrimitive{⊔}}\AgdaSpace{}%
\AgdaBound{𝓥}\AgdaSpace{}%
\AgdaOperator{\AgdaPrimitive{⁺}}\AgdaSpace{}%
\AgdaOperator{\AgdaFunction{̇}}\<%
\\
%
\>[1]\AgdaFunction{Pred₀}\AgdaSpace{}%
\AgdaBound{A}\AgdaSpace{}%
\AgdaBound{𝓥}\AgdaSpace{}%
\AgdaSymbol{=}\AgdaSpace{}%
\AgdaFunction{Σ}\AgdaSpace{}%
\AgdaBound{P}\AgdaSpace{}%
\AgdaFunction{꞉}\AgdaSpace{}%
\AgdaSymbol{(}\AgdaBound{A}\AgdaSpace{}%
\AgdaSymbol{→}\AgdaSpace{}%
\AgdaBound{𝓥}\AgdaSpace{}%
\AgdaOperator{\AgdaFunction{̇}}\AgdaSymbol{)}\AgdaSpace{}%
\AgdaFunction{,}\AgdaSpace{}%
\AgdaSymbol{∀}\AgdaSpace{}%
\AgdaBound{x}\AgdaSpace{}%
\AgdaSymbol{→}\AgdaSpace{}%
\AgdaFunction{is-subsingleton}\AgdaSpace{}%
\AgdaSymbol{(}\AgdaBound{P}\AgdaSpace{}%
\AgdaBound{x}\AgdaSymbol{)}\<%
\\
\>[0]\<%
\end{code}

Below we will often consider predicates over the class of all algebras
of a particular type. We will define the type of algebras
\texttt{Algebra\ 𝓤\ 𝑆} (for some universe level 𝓤). Like all types,
\texttt{Algebra\ 𝓤\ 𝑆} itself has a type which happens to be 𝓞 ⊔ 𝓥 ⊔ 𝓤 ⁺
̇ (as we will see in the module
\href{UALib.Algebras.Algebras.html}{UALib.Algebras}. Therefore, the type
of \texttt{Pred\ (Algebra\ 𝓤\ 𝑆)\ 𝓤} will be 𝓞 ⊔ 𝓥 ⊔ 𝓤 ⁺ ̇ as well.

The inhabitants of the type \texttt{Pred\ (Algebra\ 𝓤\ 𝑆)} 𝓤 are maps of
the form \texttt{𝑨\ →\ 𝓤\ ̇}; given an algebra
\texttt{𝑨\ :\ Algebra\ 𝓤\ 𝑆}, we have \texttt{Pred\ 𝑨\ 𝓤\ =\ 𝑨\ →\ 𝓤\ ̇}.

\paragraph{The membership relation}\label{the-membership-relation}

We introduce notation so that we may indicate that \texttt{x} ``belongs
to'' or ``inhabits'' at type \texttt{P}, or that \texttt{x} ``has
property'' \texttt{P}, by writing either \texttt{x\ ∈\ P} or
\texttt{P\ x} (cf. \texttt{Relation/Unary.agda} in the {[}Agda Standard
Library{]}{[}{]}).

\begin{code}%
\>[0]\<%
\\
\>[0]\AgdaKeyword{module}\AgdaSpace{}%
\AgdaModule{\AgdaUnderscore{}}\AgdaSpace{}%
\AgdaSymbol{\{}\AgdaBound{𝓤}\AgdaSpace{}%
\AgdaBound{𝓦}\AgdaSpace{}%
\AgdaSymbol{:}\AgdaSpace{}%
\AgdaPostulate{Universe}\AgdaSymbol{\}}\AgdaSpace{}%
\AgdaKeyword{where}\<%
\\
%
\\[\AgdaEmptyExtraSkip]%
\>[0][@{}l@{\AgdaIndent{0}}]%
\>[1]\AgdaOperator{\AgdaFunction{\AgdaUnderscore{}∈\AgdaUnderscore{}}}\AgdaSpace{}%
\AgdaSymbol{:}\AgdaSpace{}%
\AgdaSymbol{\{}\AgdaBound{A}\AgdaSpace{}%
\AgdaSymbol{:}\AgdaSpace{}%
\AgdaBound{𝓤}\AgdaSpace{}%
\AgdaOperator{\AgdaFunction{̇}}\AgdaSpace{}%
\AgdaSymbol{\}}\AgdaSpace{}%
\AgdaSymbol{→}\AgdaSpace{}%
\AgdaBound{A}\AgdaSpace{}%
\AgdaSymbol{→}\AgdaSpace{}%
\AgdaFunction{Pred}\AgdaSpace{}%
\AgdaBound{A}\AgdaSpace{}%
\AgdaBound{𝓦}\AgdaSpace{}%
\AgdaSymbol{→}\AgdaSpace{}%
\AgdaBound{𝓦}\AgdaSpace{}%
\AgdaOperator{\AgdaFunction{̇}}\<%
\\
%
\>[1]\AgdaBound{x}\AgdaSpace{}%
\AgdaOperator{\AgdaFunction{∈}}\AgdaSpace{}%
\AgdaBound{P}\AgdaSpace{}%
\AgdaSymbol{=}\AgdaSpace{}%
\AgdaBound{P}\AgdaSpace{}%
\AgdaBound{x}\<%
\\
%
\\[\AgdaEmptyExtraSkip]%
%
\>[1]\AgdaOperator{\AgdaFunction{\AgdaUnderscore{}∉\AgdaUnderscore{}}}\AgdaSpace{}%
\AgdaSymbol{:}\AgdaSpace{}%
\AgdaSymbol{\{}\AgdaBound{A}\AgdaSpace{}%
\AgdaSymbol{:}\AgdaSpace{}%
\AgdaBound{𝓤}\AgdaSpace{}%
\AgdaOperator{\AgdaFunction{̇}}\AgdaSpace{}%
\AgdaSymbol{\}}\AgdaSpace{}%
\AgdaSymbol{→}\AgdaSpace{}%
\AgdaBound{A}\AgdaSpace{}%
\AgdaSymbol{→}\AgdaSpace{}%
\AgdaFunction{Pred}\AgdaSpace{}%
\AgdaBound{A}\AgdaSpace{}%
\AgdaBound{𝓦}\AgdaSpace{}%
\AgdaSymbol{→}\AgdaSpace{}%
\AgdaBound{𝓦}\AgdaSpace{}%
\AgdaOperator{\AgdaFunction{̇}}\<%
\\
%
\>[1]\AgdaBound{x}\AgdaSpace{}%
\AgdaOperator{\AgdaFunction{∉}}\AgdaSpace{}%
\AgdaBound{P}\AgdaSpace{}%
\AgdaSymbol{=}\AgdaSpace{}%
\AgdaFunction{¬}\AgdaSpace{}%
\AgdaSymbol{(}\AgdaBound{x}\AgdaSpace{}%
\AgdaOperator{\AgdaFunction{∈}}\AgdaSpace{}%
\AgdaBound{P}\AgdaSymbol{)}\<%
\\
%
\\[\AgdaEmptyExtraSkip]%
%
\>[1]\AgdaKeyword{infix}\AgdaSpace{}%
\AgdaNumber{4}\AgdaSpace{}%
\AgdaOperator{\AgdaFunction{\AgdaUnderscore{}∈\AgdaUnderscore{}}}\AgdaSpace{}%
\AgdaOperator{\AgdaFunction{\AgdaUnderscore{}∉\AgdaUnderscore{}}}\<%
\\
\>[0]\<%
\end{code}

The ``subset'' relation is denoted, as usual, with the \texttt{⊆} symbol
(cf. \texttt{Relation/Unary.agda} in the {[}Agda Standard
Library{]}{[}{]}).

\begin{code}%
\>[0]\<%
\\
\>[0]\AgdaOperator{\AgdaFunction{\AgdaUnderscore{}⊆\AgdaUnderscore{}}}\AgdaSpace{}%
\AgdaSymbol{:}\AgdaSpace{}%
\AgdaSymbol{\{}\AgdaBound{𝓤}\AgdaSpace{}%
\AgdaBound{𝓦}\AgdaSpace{}%
\AgdaBound{𝓣}\AgdaSpace{}%
\AgdaSymbol{:}\AgdaSpace{}%
\AgdaPostulate{Universe}\AgdaSymbol{\}\{}\AgdaBound{A}\AgdaSpace{}%
\AgdaSymbol{:}\AgdaSpace{}%
\AgdaBound{𝓤}\AgdaSpace{}%
\AgdaOperator{\AgdaFunction{̇}}\AgdaSpace{}%
\AgdaSymbol{\}}\AgdaSpace{}%
\AgdaSymbol{→}\AgdaSpace{}%
\AgdaFunction{Pred}\AgdaSpace{}%
\AgdaBound{A}\AgdaSpace{}%
\AgdaBound{𝓦}\AgdaSpace{}%
\AgdaSymbol{→}\AgdaSpace{}%
\AgdaFunction{Pred}\AgdaSpace{}%
\AgdaBound{A}\AgdaSpace{}%
\AgdaBound{𝓣}\AgdaSpace{}%
\AgdaSymbol{→}\AgdaSpace{}%
\AgdaBound{𝓤}\AgdaSpace{}%
\AgdaOperator{\AgdaPrimitive{⊔}}\AgdaSpace{}%
\AgdaBound{𝓦}\AgdaSpace{}%
\AgdaOperator{\AgdaPrimitive{⊔}}\AgdaSpace{}%
\AgdaBound{𝓣}\AgdaSpace{}%
\AgdaOperator{\AgdaFunction{̇}}\<%
\\
\>[0]\AgdaBound{P}\AgdaSpace{}%
\AgdaOperator{\AgdaFunction{⊆}}\AgdaSpace{}%
\AgdaBound{Q}\AgdaSpace{}%
\AgdaSymbol{=}\AgdaSpace{}%
\AgdaSymbol{∀}\AgdaSpace{}%
\AgdaSymbol{\{}\AgdaBound{x}\AgdaSymbol{\}}\AgdaSpace{}%
\AgdaSymbol{→}\AgdaSpace{}%
\AgdaBound{x}\AgdaSpace{}%
\AgdaOperator{\AgdaFunction{∈}}\AgdaSpace{}%
\AgdaBound{P}\AgdaSpace{}%
\AgdaSymbol{→}\AgdaSpace{}%
\AgdaBound{x}\AgdaSpace{}%
\AgdaOperator{\AgdaFunction{∈}}\AgdaSpace{}%
\AgdaBound{Q}\<%
\\
%
\\[\AgdaEmptyExtraSkip]%
\>[0]\AgdaOperator{\AgdaFunction{\AgdaUnderscore{}⊇\AgdaUnderscore{}}}\AgdaSpace{}%
\AgdaSymbol{:}\AgdaSpace{}%
\AgdaSymbol{\{}\AgdaBound{𝓤}\AgdaSpace{}%
\AgdaBound{𝓦}\AgdaSpace{}%
\AgdaBound{𝓣}\AgdaSpace{}%
\AgdaSymbol{:}\AgdaSpace{}%
\AgdaPostulate{Universe}\AgdaSymbol{\}\{}\AgdaBound{A}\AgdaSpace{}%
\AgdaSymbol{:}\AgdaSpace{}%
\AgdaBound{𝓤}\AgdaSpace{}%
\AgdaOperator{\AgdaFunction{̇}}\AgdaSpace{}%
\AgdaSymbol{\}}\AgdaSpace{}%
\AgdaSymbol{→}\AgdaSpace{}%
\AgdaFunction{Pred}\AgdaSpace{}%
\AgdaBound{A}\AgdaSpace{}%
\AgdaBound{𝓦}\AgdaSpace{}%
\AgdaSymbol{→}\AgdaSpace{}%
\AgdaFunction{Pred}\AgdaSpace{}%
\AgdaBound{A}\AgdaSpace{}%
\AgdaBound{𝓣}\AgdaSpace{}%
\AgdaSymbol{→}\AgdaSpace{}%
\AgdaBound{𝓤}\AgdaSpace{}%
\AgdaOperator{\AgdaPrimitive{⊔}}\AgdaSpace{}%
\AgdaBound{𝓦}\AgdaSpace{}%
\AgdaOperator{\AgdaPrimitive{⊔}}\AgdaSpace{}%
\AgdaBound{𝓣}\AgdaSpace{}%
\AgdaOperator{\AgdaFunction{̇}}\<%
\\
\>[0]\AgdaBound{P}\AgdaSpace{}%
\AgdaOperator{\AgdaFunction{⊇}}\AgdaSpace{}%
\AgdaBound{Q}\AgdaSpace{}%
\AgdaSymbol{=}\AgdaSpace{}%
\AgdaBound{Q}\AgdaSpace{}%
\AgdaOperator{\AgdaFunction{⊆}}\AgdaSpace{}%
\AgdaBound{P}\<%
\\
%
\\[\AgdaEmptyExtraSkip]%
\>[0]\AgdaKeyword{infix}\AgdaSpace{}%
\AgdaNumber{4}\AgdaSpace{}%
\AgdaOperator{\AgdaFunction{\AgdaUnderscore{}⊆\AgdaUnderscore{}}}\AgdaSpace{}%
\AgdaOperator{\AgdaFunction{\AgdaUnderscore{}⊇\AgdaUnderscore{}}}\<%
\\
\>[0]\<%
\end{code}

In type theory everything is a type. As we have just seen, this includes
subsets. Since the notion of equality for types is usually a nontrivial
matter, it may be nontrivial to represent equality of subsets.
Fortunately, it is straightforward to write down a type that represents
what it means for two subsets to be the in informal (pencil-paper)
mathematics. In the {[}Agda UALib{]}{[}{]} we denote this \textbf{subset
equality} by =̇ and define it as follows.

\begin{code}%
\>[0]\<%
\\
\>[0]\AgdaOperator{\AgdaFunction{\AgdaUnderscore{}=̇\AgdaUnderscore{}}}\AgdaSpace{}%
\AgdaSymbol{:}\AgdaSpace{}%
\AgdaSymbol{\{}\AgdaBound{𝓤}\AgdaSpace{}%
\AgdaBound{𝓦}\AgdaSpace{}%
\AgdaBound{𝓣}\AgdaSpace{}%
\AgdaSymbol{:}\AgdaSpace{}%
\AgdaPostulate{Universe}\AgdaSymbol{\}\{}\AgdaBound{A}\AgdaSpace{}%
\AgdaSymbol{:}\AgdaSpace{}%
\AgdaBound{𝓤}\AgdaSpace{}%
\AgdaOperator{\AgdaFunction{̇}}\AgdaSpace{}%
\AgdaSymbol{\}}\AgdaSpace{}%
\AgdaSymbol{→}\AgdaSpace{}%
\AgdaFunction{Pred}\AgdaSpace{}%
\AgdaBound{A}\AgdaSpace{}%
\AgdaBound{𝓦}\AgdaSpace{}%
\AgdaSymbol{→}\AgdaSpace{}%
\AgdaFunction{Pred}\AgdaSpace{}%
\AgdaBound{A}\AgdaSpace{}%
\AgdaBound{𝓣}\AgdaSpace{}%
\AgdaSymbol{→}\AgdaSpace{}%
\AgdaBound{𝓤}\AgdaSpace{}%
\AgdaOperator{\AgdaPrimitive{⊔}}\AgdaSpace{}%
\AgdaBound{𝓦}\AgdaSpace{}%
\AgdaOperator{\AgdaPrimitive{⊔}}\AgdaSpace{}%
\AgdaBound{𝓣}\AgdaSpace{}%
\AgdaOperator{\AgdaFunction{̇}}\<%
\\
\>[0]\AgdaBound{P}\AgdaSpace{}%
\AgdaOperator{\AgdaFunction{=̇}}\AgdaSpace{}%
\AgdaBound{Q}\AgdaSpace{}%
\AgdaSymbol{=}\AgdaSpace{}%
\AgdaSymbol{(}\AgdaBound{P}\AgdaSpace{}%
\AgdaOperator{\AgdaFunction{⊆}}\AgdaSpace{}%
\AgdaBound{Q}\AgdaSymbol{)}\AgdaSpace{}%
\AgdaOperator{\AgdaFunction{×}}\AgdaSpace{}%
\AgdaSymbol{(}\AgdaBound{Q}\AgdaSpace{}%
\AgdaOperator{\AgdaFunction{⊆}}\AgdaSpace{}%
\AgdaBound{P}\AgdaSymbol{)}\<%
\\
\>[0]\<%
\end{code}

\paragraph{Predicates toolbox}\label{predicates-toolbox}

Here is a small collection of tools that will come in handy later.
Hopefully the meaning of each is self-explanatory.

\begin{code}%
\>[0]\<%
\\
\>[0]\AgdaOperator{\AgdaFunction{\AgdaUnderscore{}∈∈\AgdaUnderscore{}}}\AgdaSpace{}%
\AgdaSymbol{:}\AgdaSpace{}%
\AgdaSymbol{\{}\AgdaBound{𝓤}\AgdaSpace{}%
\AgdaBound{𝓦}\AgdaSpace{}%
\AgdaBound{𝓣}\AgdaSpace{}%
\AgdaSymbol{:}\AgdaSpace{}%
\AgdaPostulate{Universe}\AgdaSymbol{\}\{}\AgdaBound{A}\AgdaSpace{}%
\AgdaSymbol{:}\AgdaSpace{}%
\AgdaBound{𝓤}\AgdaSpace{}%
\AgdaOperator{\AgdaFunction{̇}}\AgdaSpace{}%
\AgdaSymbol{\}}\AgdaSpace{}%
\AgdaSymbol{\{}\AgdaBound{B}\AgdaSpace{}%
\AgdaSymbol{:}\AgdaSpace{}%
\AgdaBound{𝓦}\AgdaSpace{}%
\AgdaOperator{\AgdaFunction{̇}}\AgdaSpace{}%
\AgdaSymbol{\}}\AgdaSpace{}%
\AgdaSymbol{→}\AgdaSpace{}%
\AgdaSymbol{(}\AgdaBound{A}%
\>[53]\AgdaSymbol{→}%
\>[56]\AgdaBound{B}\AgdaSymbol{)}\AgdaSpace{}%
\AgdaSymbol{→}\AgdaSpace{}%
\AgdaFunction{Pred}\AgdaSpace{}%
\AgdaBound{B}\AgdaSpace{}%
\AgdaBound{𝓣}\AgdaSpace{}%
\AgdaSymbol{→}\AgdaSpace{}%
\AgdaBound{𝓤}\AgdaSpace{}%
\AgdaOperator{\AgdaPrimitive{⊔}}\AgdaSpace{}%
\AgdaBound{𝓣}\AgdaSpace{}%
\AgdaOperator{\AgdaFunction{̇}}\<%
\\
\>[0]\AgdaOperator{\AgdaFunction{\AgdaUnderscore{}∈∈\AgdaUnderscore{}}}\AgdaSpace{}%
\AgdaBound{f}\AgdaSpace{}%
\AgdaBound{S}\AgdaSpace{}%
\AgdaSymbol{=}\AgdaSpace{}%
\AgdaSymbol{(}\AgdaBound{x}\AgdaSpace{}%
\AgdaSymbol{:}\AgdaSpace{}%
\AgdaSymbol{\AgdaUnderscore{})}\AgdaSpace{}%
\AgdaSymbol{→}\AgdaSpace{}%
\AgdaBound{f}\AgdaSpace{}%
\AgdaBound{x}\AgdaSpace{}%
\AgdaOperator{\AgdaFunction{∈}}\AgdaSpace{}%
\AgdaBound{S}\<%
\\
%
\\[\AgdaEmptyExtraSkip]%
\>[0]\AgdaFunction{Pred-refl}\AgdaSpace{}%
\AgdaSymbol{:}\AgdaSpace{}%
\AgdaSymbol{\{}\AgdaBound{𝓤}\AgdaSpace{}%
\AgdaBound{𝓦}\AgdaSpace{}%
\AgdaSymbol{:}\AgdaSpace{}%
\AgdaPostulate{Universe}\AgdaSymbol{\}\{}\AgdaBound{A}\AgdaSpace{}%
\AgdaSymbol{:}\AgdaSpace{}%
\AgdaBound{𝓤}\AgdaSpace{}%
\AgdaOperator{\AgdaFunction{̇}}\AgdaSymbol{\}\{}\AgdaBound{P}\AgdaSpace{}%
\AgdaBound{Q}\AgdaSpace{}%
\AgdaSymbol{:}\AgdaSpace{}%
\AgdaFunction{Pred}\AgdaSpace{}%
\AgdaBound{A}\AgdaSpace{}%
\AgdaBound{𝓦}\AgdaSymbol{\}}\<%
\\
\>[0][@{}l@{\AgdaIndent{0}}]%
\>[1]\AgdaSymbol{→}%
\>[12]\AgdaBound{P}\AgdaSpace{}%
\AgdaOperator{\AgdaDatatype{≡}}\AgdaSpace{}%
\AgdaBound{Q}\AgdaSpace{}%
\AgdaSymbol{→}\AgdaSpace{}%
\AgdaSymbol{(}\AgdaBound{a}\AgdaSpace{}%
\AgdaSymbol{:}\AgdaSpace{}%
\AgdaBound{A}\AgdaSymbol{)}\AgdaSpace{}%
\AgdaSymbol{→}\AgdaSpace{}%
\AgdaBound{a}\AgdaSpace{}%
\AgdaOperator{\AgdaFunction{∈}}\AgdaSpace{}%
\AgdaBound{P}\AgdaSpace{}%
\AgdaSymbol{→}\AgdaSpace{}%
\AgdaBound{a}\AgdaSpace{}%
\AgdaOperator{\AgdaFunction{∈}}\AgdaSpace{}%
\AgdaBound{Q}\<%
\\
\>[0]\AgdaFunction{Pred-refl}\AgdaSpace{}%
\AgdaSymbol{(}\AgdaInductiveConstructor{refl}\AgdaSpace{}%
\AgdaSymbol{\AgdaUnderscore{})}\AgdaSpace{}%
\AgdaSymbol{\AgdaUnderscore{}}\AgdaSpace{}%
\AgdaSymbol{=}\AgdaSpace{}%
\AgdaSymbol{λ}\AgdaSpace{}%
\AgdaBound{z}\AgdaSpace{}%
\AgdaSymbol{→}\AgdaSpace{}%
\AgdaBound{z}\<%
\\
%
\\[\AgdaEmptyExtraSkip]%
\>[0]\AgdaFunction{Pred-≡}\AgdaSpace{}%
\AgdaSymbol{:}\AgdaSpace{}%
\AgdaSymbol{\{}\AgdaBound{𝓤}\AgdaSpace{}%
\AgdaBound{𝓦}\AgdaSpace{}%
\AgdaSymbol{:}\AgdaSpace{}%
\AgdaPostulate{Universe}\AgdaSymbol{\}\{}\AgdaBound{A}\AgdaSpace{}%
\AgdaSymbol{:}\AgdaSpace{}%
\AgdaBound{𝓤}\AgdaSpace{}%
\AgdaOperator{\AgdaFunction{̇}}\AgdaSymbol{\}\{}\AgdaBound{P}\AgdaSpace{}%
\AgdaBound{Q}\AgdaSpace{}%
\AgdaSymbol{:}\AgdaSpace{}%
\AgdaFunction{Pred}\AgdaSpace{}%
\AgdaBound{A}\AgdaSpace{}%
\AgdaBound{𝓦}\AgdaSymbol{\}}\<%
\\
\>[0][@{}l@{\AgdaIndent{0}}]%
\>[1]\AgdaSymbol{→}%
\>[12]\AgdaBound{P}\AgdaSpace{}%
\AgdaOperator{\AgdaDatatype{≡}}\AgdaSpace{}%
\AgdaBound{Q}\AgdaSpace{}%
\AgdaSymbol{→}\AgdaSpace{}%
\AgdaBound{P}\AgdaSpace{}%
\AgdaOperator{\AgdaFunction{=̇}}\AgdaSpace{}%
\AgdaBound{Q}\<%
\\
\>[0]\AgdaFunction{Pred-≡}\AgdaSpace{}%
\AgdaSymbol{(}\AgdaInductiveConstructor{refl}\AgdaSpace{}%
\AgdaSymbol{\AgdaUnderscore{})}\AgdaSpace{}%
\AgdaSymbol{=}\AgdaSpace{}%
\AgdaSymbol{(λ}\AgdaSpace{}%
\AgdaBound{z}\AgdaSpace{}%
\AgdaSymbol{→}\AgdaSpace{}%
\AgdaBound{z}\AgdaSymbol{)}\AgdaSpace{}%
\AgdaOperator{\AgdaInductiveConstructor{,}}\AgdaSpace{}%
\AgdaSymbol{λ}\AgdaSpace{}%
\AgdaBound{z}\AgdaSpace{}%
\AgdaSymbol{→}\AgdaSpace{}%
\AgdaBound{z}\<%
\\
%
\\[\AgdaEmptyExtraSkip]%
\>[0]\AgdaFunction{Pred-≡→⊆}\AgdaSpace{}%
\AgdaSymbol{:}\AgdaSpace{}%
\AgdaSymbol{\{}\AgdaBound{𝓤}\AgdaSpace{}%
\AgdaBound{𝓦}\AgdaSpace{}%
\AgdaSymbol{:}\AgdaSpace{}%
\AgdaPostulate{Universe}\AgdaSymbol{\}\{}\AgdaBound{A}\AgdaSpace{}%
\AgdaSymbol{:}\AgdaSpace{}%
\AgdaBound{𝓤}\AgdaSpace{}%
\AgdaOperator{\AgdaFunction{̇}}\AgdaSymbol{\}\{}\AgdaBound{P}\AgdaSpace{}%
\AgdaBound{Q}\AgdaSpace{}%
\AgdaSymbol{:}\AgdaSpace{}%
\AgdaFunction{Pred}\AgdaSpace{}%
\AgdaBound{A}\AgdaSpace{}%
\AgdaBound{𝓦}\AgdaSymbol{\}}\<%
\\
\>[0][@{}l@{\AgdaIndent{0}}]%
\>[1]\AgdaSymbol{→}%
\>[12]\AgdaBound{P}\AgdaSpace{}%
\AgdaOperator{\AgdaDatatype{≡}}\AgdaSpace{}%
\AgdaBound{Q}\AgdaSpace{}%
\AgdaSymbol{→}\AgdaSpace{}%
\AgdaSymbol{(}\AgdaBound{P}\AgdaSpace{}%
\AgdaOperator{\AgdaFunction{⊆}}\AgdaSpace{}%
\AgdaBound{Q}\AgdaSymbol{)}\<%
\\
\>[0]\AgdaFunction{Pred-≡→⊆}\AgdaSpace{}%
\AgdaSymbol{(}\AgdaInductiveConstructor{refl}\AgdaSpace{}%
\AgdaSymbol{\AgdaUnderscore{})}\AgdaSpace{}%
\AgdaSymbol{=}\AgdaSpace{}%
\AgdaSymbol{(λ}\AgdaSpace{}%
\AgdaBound{z}\AgdaSpace{}%
\AgdaSymbol{→}\AgdaSpace{}%
\AgdaBound{z}\AgdaSymbol{)}\<%
\\
%
\\[\AgdaEmptyExtraSkip]%
\>[0]\AgdaFunction{Pred-≡→⊇}\AgdaSpace{}%
\AgdaSymbol{:}\AgdaSpace{}%
\AgdaSymbol{\{}\AgdaBound{𝓤}\AgdaSpace{}%
\AgdaBound{𝓦}\AgdaSpace{}%
\AgdaSymbol{:}\AgdaSpace{}%
\AgdaPostulate{Universe}\AgdaSymbol{\}\{}\AgdaBound{A}\AgdaSpace{}%
\AgdaSymbol{:}\AgdaSpace{}%
\AgdaBound{𝓤}\AgdaSpace{}%
\AgdaOperator{\AgdaFunction{̇}}\AgdaSymbol{\}\{}\AgdaBound{P}\AgdaSpace{}%
\AgdaBound{Q}\AgdaSpace{}%
\AgdaSymbol{:}\AgdaSpace{}%
\AgdaFunction{Pred}\AgdaSpace{}%
\AgdaBound{A}\AgdaSpace{}%
\AgdaBound{𝓦}\AgdaSymbol{\}}\<%
\\
\>[0][@{}l@{\AgdaIndent{0}}]%
\>[1]\AgdaSymbol{→}%
\>[12]\AgdaBound{P}\AgdaSpace{}%
\AgdaOperator{\AgdaDatatype{≡}}\AgdaSpace{}%
\AgdaBound{Q}\AgdaSpace{}%
\AgdaSymbol{→}\AgdaSpace{}%
\AgdaSymbol{(}\AgdaBound{P}\AgdaSpace{}%
\AgdaOperator{\AgdaFunction{⊇}}\AgdaSpace{}%
\AgdaBound{Q}\AgdaSymbol{)}\<%
\\
\>[0]\AgdaFunction{Pred-≡→⊇}\AgdaSpace{}%
\AgdaSymbol{(}\AgdaInductiveConstructor{refl}\AgdaSpace{}%
\AgdaSymbol{\AgdaUnderscore{})}\AgdaSpace{}%
\AgdaSymbol{=}\AgdaSpace{}%
\AgdaSymbol{(λ}\AgdaSpace{}%
\AgdaBound{z}\AgdaSpace{}%
\AgdaSymbol{→}\AgdaSpace{}%
\AgdaBound{z}\AgdaSymbol{)}\<%
\\
%
\\[\AgdaEmptyExtraSkip]%
\>[0]\AgdaFunction{Pred-=̇-≡}\AgdaSpace{}%
\AgdaSymbol{:}\AgdaSpace{}%
\AgdaSymbol{\{}\AgdaBound{𝓤}\AgdaSpace{}%
\AgdaBound{𝓦}\AgdaSpace{}%
\AgdaSymbol{:}\AgdaSpace{}%
\AgdaPostulate{Universe}\AgdaSymbol{\}}\<%
\\
\>[0][@{}l@{\AgdaIndent{0}}]%
\>[1]\AgdaSymbol{→}%
\>[12]\AgdaFunction{propext}\AgdaSpace{}%
\AgdaBound{𝓦}\AgdaSpace{}%
\AgdaSymbol{→}\AgdaSpace{}%
\AgdaFunction{global-dfunext}\<%
\\
%
\>[1]\AgdaSymbol{→}%
\>[12]\AgdaSymbol{\{}\AgdaBound{A}\AgdaSpace{}%
\AgdaSymbol{:}\AgdaSpace{}%
\AgdaBound{𝓤}\AgdaSpace{}%
\AgdaOperator{\AgdaFunction{̇}}\AgdaSymbol{\}\{}\AgdaBound{P}\AgdaSpace{}%
\AgdaBound{Q}\AgdaSpace{}%
\AgdaSymbol{:}\AgdaSpace{}%
\AgdaFunction{Pred}\AgdaSpace{}%
\AgdaBound{A}\AgdaSpace{}%
\AgdaBound{𝓦}\AgdaSymbol{\}}\<%
\\
%
\>[1]\AgdaSymbol{→}%
\>[12]\AgdaSymbol{((}\AgdaBound{x}\AgdaSpace{}%
\AgdaSymbol{:}\AgdaSpace{}%
\AgdaBound{A}\AgdaSymbol{)}\AgdaSpace{}%
\AgdaSymbol{→}\AgdaSpace{}%
\AgdaFunction{is-subsingleton}\AgdaSpace{}%
\AgdaSymbol{(}\AgdaBound{P}\AgdaSpace{}%
\AgdaBound{x}\AgdaSymbol{))}\<%
\\
%
\>[1]\AgdaSymbol{→}%
\>[12]\AgdaSymbol{((}\AgdaBound{x}\AgdaSpace{}%
\AgdaSymbol{:}\AgdaSpace{}%
\AgdaBound{A}\AgdaSymbol{)}\AgdaSpace{}%
\AgdaSymbol{→}\AgdaSpace{}%
\AgdaFunction{is-subsingleton}\AgdaSpace{}%
\AgdaSymbol{(}\AgdaBound{Q}\AgdaSpace{}%
\AgdaBound{x}\AgdaSymbol{))}\<%
\\
%
\>[1]\AgdaSymbol{→}%
\>[12]\AgdaBound{P}\AgdaSpace{}%
\AgdaOperator{\AgdaFunction{=̇}}\AgdaSpace{}%
\AgdaBound{Q}\AgdaSpace{}%
\AgdaSymbol{→}\AgdaSpace{}%
\AgdaBound{P}\AgdaSpace{}%
\AgdaOperator{\AgdaDatatype{≡}}\AgdaSpace{}%
\AgdaBound{Q}\<%
\\
\>[0]\AgdaFunction{Pred-=̇-≡}\AgdaSpace{}%
\AgdaBound{pe}\AgdaSpace{}%
\AgdaBound{gfe}\AgdaSpace{}%
\AgdaSymbol{\{}\AgdaBound{A}\AgdaSymbol{\}\{}\AgdaBound{P}\AgdaSymbol{\}\{}\AgdaBound{Q}\AgdaSymbol{\}}\AgdaSpace{}%
\AgdaBound{ssP}\AgdaSpace{}%
\AgdaBound{ssQ}\AgdaSpace{}%
\AgdaSymbol{(}\AgdaBound{pq}\AgdaSpace{}%
\AgdaOperator{\AgdaInductiveConstructor{,}}\AgdaSpace{}%
\AgdaBound{qp}\AgdaSymbol{)}\AgdaSpace{}%
\AgdaSymbol{=}\AgdaSpace{}%
\AgdaBound{gfe}\AgdaSpace{}%
\AgdaFunction{γ}\<%
\\
\>[0][@{}l@{\AgdaIndent{0}}]%
\>[1]\AgdaKeyword{where}\<%
\\
\>[1][@{}l@{\AgdaIndent{0}}]%
\>[2]\AgdaFunction{γ}\AgdaSpace{}%
\AgdaSymbol{:}\AgdaSpace{}%
\AgdaSymbol{(}\AgdaBound{x}\AgdaSpace{}%
\AgdaSymbol{:}\AgdaSpace{}%
\AgdaBound{A}\AgdaSymbol{)}\AgdaSpace{}%
\AgdaSymbol{→}\AgdaSpace{}%
\AgdaBound{P}\AgdaSpace{}%
\AgdaBound{x}\AgdaSpace{}%
\AgdaOperator{\AgdaDatatype{≡}}\AgdaSpace{}%
\AgdaBound{Q}\AgdaSpace{}%
\AgdaBound{x}\<%
\\
%
\>[2]\AgdaFunction{γ}\AgdaSpace{}%
\AgdaBound{x}\AgdaSpace{}%
\AgdaSymbol{=}\AgdaSpace{}%
\AgdaBound{pe}\AgdaSpace{}%
\AgdaSymbol{(}\AgdaBound{ssP}\AgdaSpace{}%
\AgdaBound{x}\AgdaSymbol{)}\AgdaSpace{}%
\AgdaSymbol{(}\AgdaBound{ssQ}\AgdaSpace{}%
\AgdaBound{x}\AgdaSymbol{)}\AgdaSpace{}%
\AgdaBound{pq}\AgdaSpace{}%
\AgdaBound{qp}\<%
\\
%
\\[\AgdaEmptyExtraSkip]%
\>[0]\AgdaComment{-- Disjoint Union.}\<%
\\
\>[0]\AgdaKeyword{data}\AgdaSpace{}%
\AgdaOperator{\AgdaDatatype{\AgdaUnderscore{}⊎\AgdaUnderscore{}}}\AgdaSpace{}%
\AgdaSymbol{\{}\AgdaBound{𝓤}\AgdaSpace{}%
\AgdaBound{𝓦}\AgdaSpace{}%
\AgdaSymbol{:}\AgdaSpace{}%
\AgdaPostulate{Universe}\AgdaSymbol{\}(}\AgdaBound{A}\AgdaSpace{}%
\AgdaSymbol{:}\AgdaSpace{}%
\AgdaBound{𝓤}\AgdaSpace{}%
\AgdaOperator{\AgdaFunction{̇}}\AgdaSymbol{)}\AgdaSpace{}%
\AgdaSymbol{(}\AgdaBound{B}\AgdaSpace{}%
\AgdaSymbol{:}\AgdaSpace{}%
\AgdaBound{𝓦}\AgdaSpace{}%
\AgdaOperator{\AgdaFunction{̇}}\AgdaSymbol{)}\AgdaSpace{}%
\AgdaSymbol{:}\AgdaSpace{}%
\AgdaBound{𝓤}\AgdaSpace{}%
\AgdaOperator{\AgdaPrimitive{⊔}}\AgdaSpace{}%
\AgdaBound{𝓦}\AgdaSpace{}%
\AgdaOperator{\AgdaFunction{̇}}\AgdaSpace{}%
\AgdaKeyword{where}\<%
\\
\>[0][@{}l@{\AgdaIndent{0}}]%
\>[1]\AgdaInductiveConstructor{inj₁}\AgdaSpace{}%
\AgdaSymbol{:}\AgdaSpace{}%
\AgdaSymbol{(}\AgdaBound{x}\AgdaSpace{}%
\AgdaSymbol{:}\AgdaSpace{}%
\AgdaBound{A}\AgdaSymbol{)}\AgdaSpace{}%
\AgdaSymbol{→}\AgdaSpace{}%
\AgdaBound{A}\AgdaSpace{}%
\AgdaOperator{\AgdaDatatype{⊎}}\AgdaSpace{}%
\AgdaBound{B}\<%
\\
%
\>[1]\AgdaInductiveConstructor{inj₂}\AgdaSpace{}%
\AgdaSymbol{:}\AgdaSpace{}%
\AgdaSymbol{(}\AgdaBound{y}\AgdaSpace{}%
\AgdaSymbol{:}\AgdaSpace{}%
\AgdaBound{B}\AgdaSymbol{)}\AgdaSpace{}%
\AgdaSymbol{→}\AgdaSpace{}%
\AgdaBound{A}\AgdaSpace{}%
\AgdaOperator{\AgdaDatatype{⊎}}\AgdaSpace{}%
\AgdaBound{B}\<%
\\
\>[0]\AgdaKeyword{infixr}\AgdaSpace{}%
\AgdaNumber{1}\AgdaSpace{}%
\AgdaOperator{\AgdaDatatype{\AgdaUnderscore{}⊎\AgdaUnderscore{}}}\<%
\\
%
\\[\AgdaEmptyExtraSkip]%
\>[0]\AgdaComment{-- Union.}\<%
\\
\>[0]\AgdaOperator{\AgdaFunction{\AgdaUnderscore{}∪\AgdaUnderscore{}}}\AgdaSpace{}%
\AgdaSymbol{:}\AgdaSpace{}%
\AgdaSymbol{\{}\AgdaBound{𝓤}\AgdaSpace{}%
\AgdaBound{𝓦}\AgdaSpace{}%
\AgdaBound{𝓣}\AgdaSpace{}%
\AgdaSymbol{:}\AgdaSpace{}%
\AgdaPostulate{Universe}\AgdaSymbol{\}\{}\AgdaBound{A}\AgdaSpace{}%
\AgdaSymbol{:}\AgdaSpace{}%
\AgdaBound{𝓤}\AgdaSpace{}%
\AgdaOperator{\AgdaFunction{̇}}\AgdaSymbol{\}}\AgdaSpace{}%
\AgdaSymbol{→}\AgdaSpace{}%
\AgdaFunction{Pred}\AgdaSpace{}%
\AgdaBound{A}\AgdaSpace{}%
\AgdaBound{𝓦}\AgdaSpace{}%
\AgdaSymbol{→}\AgdaSpace{}%
\AgdaFunction{Pred}\AgdaSpace{}%
\AgdaBound{A}\AgdaSpace{}%
\AgdaBound{𝓣}\AgdaSpace{}%
\AgdaSymbol{→}\AgdaSpace{}%
\AgdaFunction{Pred}\AgdaSpace{}%
\AgdaBound{A}\AgdaSpace{}%
\AgdaSymbol{\AgdaUnderscore{}}\<%
\\
\>[0]\AgdaBound{P}\AgdaSpace{}%
\AgdaOperator{\AgdaFunction{∪}}\AgdaSpace{}%
\AgdaBound{Q}\AgdaSpace{}%
\AgdaSymbol{=}\AgdaSpace{}%
\AgdaSymbol{λ}\AgdaSpace{}%
\AgdaBound{x}\AgdaSpace{}%
\AgdaSymbol{→}\AgdaSpace{}%
\AgdaBound{x}\AgdaSpace{}%
\AgdaOperator{\AgdaFunction{∈}}\AgdaSpace{}%
\AgdaBound{P}\AgdaSpace{}%
\AgdaOperator{\AgdaDatatype{⊎}}\AgdaSpace{}%
\AgdaBound{x}\AgdaSpace{}%
\AgdaOperator{\AgdaFunction{∈}}\AgdaSpace{}%
\AgdaBound{Q}\<%
\\
\>[0]\AgdaKeyword{infixr}\AgdaSpace{}%
\AgdaNumber{1}\AgdaSpace{}%
\AgdaOperator{\AgdaFunction{\AgdaUnderscore{}∪\AgdaUnderscore{}}}\<%
\\
%
\\[\AgdaEmptyExtraSkip]%
\>[0]\AgdaComment{-- The empty set.}\<%
\\
\>[0]\AgdaFunction{∅}\AgdaSpace{}%
\AgdaSymbol{:}\AgdaSpace{}%
\AgdaSymbol{\{}\AgdaBound{𝓤}\AgdaSpace{}%
\AgdaSymbol{:}\AgdaSpace{}%
\AgdaPostulate{Universe}\AgdaSymbol{\}\{}\AgdaBound{A}\AgdaSpace{}%
\AgdaSymbol{:}\AgdaSpace{}%
\AgdaBound{𝓤}\AgdaSpace{}%
\AgdaOperator{\AgdaFunction{̇}}\AgdaSymbol{\}}\AgdaSpace{}%
\AgdaSymbol{→}\AgdaSpace{}%
\AgdaFunction{Pred}\AgdaSpace{}%
\AgdaBound{A}\AgdaSpace{}%
\AgdaPrimitive{𝓤₀}\<%
\\
\>[0]\AgdaFunction{∅}\AgdaSpace{}%
\AgdaSymbol{=}\AgdaSpace{}%
\AgdaSymbol{λ}\AgdaSpace{}%
\AgdaBound{\AgdaUnderscore{}}\AgdaSpace{}%
\AgdaSymbol{→}\AgdaSpace{}%
\AgdaFunction{𝟘}\<%
\\
%
\\[\AgdaEmptyExtraSkip]%
\>[0]\AgdaComment{-- Singletons.}\<%
\\
\>[0]\AgdaOperator{\AgdaFunction{{\AgdaUnderscore{}}}}\AgdaSpace{}%
\AgdaSymbol{:}\AgdaSpace{}%
\AgdaSymbol{\{}\AgdaBound{𝓤}\AgdaSpace{}%
\AgdaSymbol{:}\AgdaSpace{}%
\AgdaPostulate{Universe}\AgdaSymbol{\}\{}\AgdaBound{A}\AgdaSpace{}%
\AgdaSymbol{:}\AgdaSpace{}%
\AgdaBound{𝓤}\AgdaSpace{}%
\AgdaOperator{\AgdaFunction{̇}}\AgdaSymbol{\}}\AgdaSpace{}%
\AgdaSymbol{→}\AgdaSpace{}%
\AgdaBound{A}\AgdaSpace{}%
\AgdaSymbol{→}\AgdaSpace{}%
\AgdaFunction{Pred}\AgdaSpace{}%
\AgdaBound{A}\AgdaSpace{}%
\AgdaSymbol{\AgdaUnderscore{}}\<%
\\
\>[0]\AgdaOperator{\AgdaFunction{{}}\AgdaSpace{}%
\AgdaBound{x}\AgdaSpace{}%
\AgdaOperator{\AgdaFunction{}}}\AgdaSpace{}%
\AgdaSymbol{=}\AgdaSpace{}%
\AgdaBound{x}\AgdaSpace{}%
\AgdaOperator{\AgdaDatatype{≡\AgdaUnderscore{}}}\<%
\\
%
\\[\AgdaEmptyExtraSkip]%
\>[0]\AgdaOperator{\AgdaFunction{Im\AgdaUnderscore{}⊆\AgdaUnderscore{}}}\AgdaSpace{}%
\AgdaSymbol{:}\AgdaSpace{}%
\AgdaSymbol{\{}\AgdaBound{𝓤}\AgdaSpace{}%
\AgdaBound{𝓦}\AgdaSpace{}%
\AgdaBound{𝓣}\AgdaSpace{}%
\AgdaSymbol{:}\AgdaSpace{}%
\AgdaPostulate{Universe}\AgdaSymbol{\}}\AgdaSpace{}%
\AgdaSymbol{\{}\AgdaBound{A}\AgdaSpace{}%
\AgdaSymbol{:}\AgdaSpace{}%
\AgdaBound{𝓤}\AgdaSpace{}%
\AgdaOperator{\AgdaFunction{̇}}\AgdaSpace{}%
\AgdaSymbol{\}}\AgdaSpace{}%
\AgdaSymbol{\{}\AgdaBound{B}\AgdaSpace{}%
\AgdaSymbol{:}\AgdaSpace{}%
\AgdaBound{𝓦}\AgdaSpace{}%
\AgdaOperator{\AgdaFunction{̇}}\AgdaSpace{}%
\AgdaSymbol{\}}\AgdaSpace{}%
\AgdaSymbol{→}\AgdaSpace{}%
\AgdaSymbol{(}\AgdaBound{A}\AgdaSpace{}%
\AgdaSymbol{→}\AgdaSpace{}%
\AgdaBound{B}\AgdaSymbol{)}\AgdaSpace{}%
\AgdaSymbol{→}\AgdaSpace{}%
\AgdaFunction{Pred}\AgdaSpace{}%
\AgdaBound{B}\AgdaSpace{}%
\AgdaBound{𝓣}\AgdaSpace{}%
\AgdaSymbol{→}\AgdaSpace{}%
\AgdaBound{𝓤}\AgdaSpace{}%
\AgdaOperator{\AgdaPrimitive{⊔}}\AgdaSpace{}%
\AgdaBound{𝓣}\AgdaSpace{}%
\AgdaOperator{\AgdaFunction{̇}}\<%
\\
\>[0]\AgdaOperator{\AgdaFunction{Im\AgdaUnderscore{}⊆\AgdaUnderscore{}}}\AgdaSpace{}%
\AgdaSymbol{\{}\AgdaArgument{A}\AgdaSpace{}%
\AgdaSymbol{=}\AgdaSpace{}%
\AgdaBound{A}\AgdaSymbol{\}}\AgdaSpace{}%
\AgdaBound{f}\AgdaSpace{}%
\AgdaBound{S}\AgdaSpace{}%
\AgdaSymbol{=}\AgdaSpace{}%
\AgdaSymbol{(}\AgdaBound{x}\AgdaSpace{}%
\AgdaSymbol{:}\AgdaSpace{}%
\AgdaBound{A}\AgdaSymbol{)}\AgdaSpace{}%
\AgdaSymbol{→}\AgdaSpace{}%
\AgdaBound{f}\AgdaSpace{}%
\AgdaBound{x}\AgdaSpace{}%
\AgdaOperator{\AgdaFunction{∈}}\AgdaSpace{}%
\AgdaBound{S}\<%
\\
%
\\[\AgdaEmptyExtraSkip]%
\>[0]\AgdaFunction{img}\AgdaSpace{}%
\AgdaSymbol{:}%
\>[600I]\AgdaSymbol{\{}\AgdaBound{𝓤}\AgdaSpace{}%
\AgdaSymbol{:}\AgdaSpace{}%
\AgdaPostulate{Universe}\AgdaSymbol{\}\{}\AgdaBound{X}\AgdaSpace{}%
\AgdaSymbol{:}\AgdaSpace{}%
\AgdaBound{𝓤}\AgdaSpace{}%
\AgdaOperator{\AgdaFunction{̇}}\AgdaSpace{}%
\AgdaSymbol{\}}\AgdaSpace{}%
\AgdaSymbol{\{}\AgdaBound{Y}\AgdaSpace{}%
\AgdaSymbol{:}\AgdaSpace{}%
\AgdaBound{𝓤}\AgdaSpace{}%
\AgdaOperator{\AgdaFunction{̇}}\AgdaSpace{}%
\AgdaSymbol{\}}\<%
\\
\>[.][@{}l@{}]\<[600I]%
\>[6]\AgdaSymbol{(}\AgdaBound{f}\AgdaSpace{}%
\AgdaSymbol{:}\AgdaSpace{}%
\AgdaBound{X}\AgdaSpace{}%
\AgdaSymbol{→}\AgdaSpace{}%
\AgdaBound{Y}\AgdaSymbol{)}\AgdaSpace{}%
\AgdaSymbol{(}\AgdaBound{P}\AgdaSpace{}%
\AgdaSymbol{:}\AgdaSpace{}%
\AgdaFunction{Pred}\AgdaSpace{}%
\AgdaBound{Y}\AgdaSpace{}%
\AgdaBound{𝓤}\AgdaSymbol{)}\<%
\\
\>[0][@{}l@{\AgdaIndent{0}}]%
\>[1]\AgdaSymbol{→}%
\>[6]\AgdaOperator{\AgdaFunction{Im}}\AgdaSpace{}%
\AgdaBound{f}\AgdaSpace{}%
\AgdaOperator{\AgdaFunction{⊆}}\AgdaSpace{}%
\AgdaBound{P}\AgdaSpace{}%
\AgdaSymbol{→}%
\>[18]\AgdaBound{X}\AgdaSpace{}%
\AgdaSymbol{→}\AgdaSpace{}%
\AgdaRecord{Σ}\AgdaSpace{}%
\AgdaBound{P}\<%
\\
\>[0]\AgdaFunction{img}\AgdaSpace{}%
\AgdaSymbol{\{}\AgdaArgument{Y}\AgdaSpace{}%
\AgdaSymbol{=}\AgdaSpace{}%
\AgdaBound{Y}\AgdaSymbol{\}}\AgdaSpace{}%
\AgdaBound{f}\AgdaSpace{}%
\AgdaBound{P}\AgdaSpace{}%
\AgdaBound{Imf⊆P}\AgdaSpace{}%
\AgdaSymbol{=}\AgdaSpace{}%
\AgdaSymbol{λ}\AgdaSpace{}%
\AgdaBound{x₁}\AgdaSpace{}%
\AgdaSymbol{→}\AgdaSpace{}%
\AgdaBound{f}\AgdaSpace{}%
\AgdaBound{x₁}\AgdaSpace{}%
\AgdaOperator{\AgdaInductiveConstructor{,}}\AgdaSpace{}%
\AgdaBound{Imf⊆P}\AgdaSpace{}%
\AgdaBound{x₁}\<%
\\
\>[0]\<%
\end{code}

\paragraph{Predicate product and
transport}\label{predicate-product-and-transport}

The product \texttt{Π\ P} of a predicate \texttt{P\ :\ Pred\ X\ 𝓤} is
inhabited iff P x holds for all x : X.

\begin{code}%
\>[0]\<%
\\
\>[0]\AgdaFunction{ΠP-meaning}\AgdaSpace{}%
\AgdaSymbol{:}\AgdaSpace{}%
\AgdaSymbol{\{}\AgdaBound{𝓧}\AgdaSpace{}%
\AgdaBound{𝓤}\AgdaSpace{}%
\AgdaSymbol{:}\AgdaSpace{}%
\AgdaPostulate{Universe}\AgdaSymbol{\}\{}\AgdaBound{X}\AgdaSpace{}%
\AgdaSymbol{:}\AgdaSpace{}%
\AgdaBound{𝓧}\AgdaSpace{}%
\AgdaOperator{\AgdaFunction{̇}}\AgdaSymbol{\}\{}\AgdaBound{P}\AgdaSpace{}%
\AgdaSymbol{:}\AgdaSpace{}%
\AgdaFunction{Pred}\AgdaSpace{}%
\AgdaBound{X}\AgdaSpace{}%
\AgdaBound{𝓤}\AgdaSymbol{\}}\<%
\\
\>[0][@{}l@{\AgdaIndent{0}}]%
\>[1]\AgdaSymbol{→}%
\>[14]\AgdaFunction{Π}\AgdaSpace{}%
\AgdaBound{P}%
\>[19]\AgdaSymbol{→}%
\>[22]\AgdaSymbol{(}\AgdaBound{x}\AgdaSpace{}%
\AgdaSymbol{:}\AgdaSpace{}%
\AgdaBound{X}\AgdaSymbol{)}\AgdaSpace{}%
\AgdaSymbol{→}\AgdaSpace{}%
\AgdaBound{P}\AgdaSpace{}%
\AgdaBound{x}\<%
\\
\>[0]\AgdaFunction{ΠP-meaning}\AgdaSpace{}%
\AgdaBound{f}\AgdaSpace{}%
\AgdaBound{x}\AgdaSpace{}%
\AgdaSymbol{=}\AgdaSpace{}%
\AgdaBound{f}\AgdaSpace{}%
\AgdaBound{x}\<%
\\
\>[0]\<%
\end{code}

The following is a pair of useful ``transport'' lemmas for predicates.

\begin{code}%
\>[0]\AgdaKeyword{module}\AgdaSpace{}%
\AgdaModule{\AgdaUnderscore{}}\AgdaSpace{}%
\AgdaSymbol{\{}\AgdaBound{𝓤}\AgdaSpace{}%
\AgdaBound{𝓦}\AgdaSpace{}%
\AgdaSymbol{:}\AgdaSpace{}%
\AgdaPostulate{Universe}\AgdaSymbol{\}}\AgdaSpace{}%
\AgdaKeyword{where}\<%
\\
%
\\[\AgdaEmptyExtraSkip]%
\>[0][@{}l@{\AgdaIndent{0}}]%
\>[1]\AgdaFunction{cong-app-pred}%
\>[672I]\AgdaSymbol{:}%
\>[673I]\AgdaSymbol{\{}\AgdaBound{A}\AgdaSpace{}%
\AgdaSymbol{:}\AgdaSpace{}%
\AgdaBound{𝓤}\AgdaSpace{}%
\AgdaOperator{\AgdaFunction{̇}}\AgdaSpace{}%
\AgdaSymbol{\}\{}\AgdaBound{B₁}\AgdaSpace{}%
\AgdaBound{B₂}\AgdaSpace{}%
\AgdaSymbol{:}\AgdaSpace{}%
\AgdaFunction{Pred}\AgdaSpace{}%
\AgdaBound{A}\AgdaSpace{}%
\AgdaBound{𝓦}\AgdaSymbol{\}}\<%
\\
\>[.][@{}l@{}]\<[673I]%
\>[17]\AgdaSymbol{(}\AgdaBound{x}\AgdaSpace{}%
\AgdaSymbol{:}\AgdaSpace{}%
\AgdaBound{A}\AgdaSymbol{)}\AgdaSpace{}%
\AgdaSymbol{→}%
\>[28]\AgdaBound{x}\AgdaSpace{}%
\AgdaOperator{\AgdaFunction{∈}}\AgdaSpace{}%
\AgdaBound{B₁}%
\>[36]\AgdaSymbol{→}%
\>[39]\AgdaBound{B₁}\AgdaSpace{}%
\AgdaOperator{\AgdaDatatype{≡}}\AgdaSpace{}%
\AgdaBound{B₂}\<%
\\
\>[672I][@{}l@{\AgdaIndent{0}}]%
\>[16]\AgdaComment{------------------------------}\<%
\\
\>[1][@{}l@{\AgdaIndent{0}}]%
\>[2]\AgdaSymbol{→}%
\>[28]\AgdaBound{x}\AgdaSpace{}%
\AgdaOperator{\AgdaFunction{∈}}\AgdaSpace{}%
\AgdaBound{B₂}\<%
\\
%
\>[1]\AgdaFunction{cong-app-pred}\AgdaSpace{}%
\AgdaBound{x}\AgdaSpace{}%
\AgdaBound{x∈B₁}\AgdaSpace{}%
\AgdaSymbol{(}\AgdaInductiveConstructor{refl}\AgdaSpace{}%
\AgdaSymbol{\AgdaUnderscore{}}\AgdaSpace{}%
\AgdaSymbol{)}\AgdaSpace{}%
\AgdaSymbol{=}\AgdaSpace{}%
\AgdaBound{x∈B₁}\<%
\\
%
\\[\AgdaEmptyExtraSkip]%
%
\>[1]\AgdaFunction{cong-pred}\AgdaSpace{}%
\AgdaSymbol{:}%
\>[700I]\AgdaSymbol{\{}\AgdaBound{A}\AgdaSpace{}%
\AgdaSymbol{:}\AgdaSpace{}%
\AgdaBound{𝓤}\AgdaSpace{}%
\AgdaOperator{\AgdaFunction{̇}}\AgdaSpace{}%
\AgdaSymbol{\}\{}\AgdaBound{B}\AgdaSpace{}%
\AgdaSymbol{:}\AgdaSpace{}%
\AgdaFunction{Pred}\AgdaSpace{}%
\AgdaBound{A}\AgdaSpace{}%
\AgdaBound{𝓦}\AgdaSymbol{\}}\<%
\\
\>[.][@{}l@{}]\<[700I]%
\>[13]\AgdaSymbol{(}\AgdaBound{x}\AgdaSpace{}%
\AgdaBound{y}\AgdaSpace{}%
\AgdaSymbol{:}\AgdaSpace{}%
\AgdaBound{A}\AgdaSymbol{)}\AgdaSpace{}%
\AgdaSymbol{→}%
\>[26]\AgdaBound{x}\AgdaSpace{}%
\AgdaOperator{\AgdaFunction{∈}}\AgdaSpace{}%
\AgdaBound{B}%
\>[33]\AgdaSymbol{→}%
\>[36]\AgdaBound{x}\AgdaSpace{}%
\AgdaOperator{\AgdaDatatype{≡}}\AgdaSpace{}%
\AgdaBound{y}\<%
\\
%
\>[13]\AgdaComment{----------------------------}\<%
\\
\>[1][@{}l@{\AgdaIndent{0}}]%
\>[2]\AgdaSymbol{→}%
\>[26]\AgdaBound{y}\AgdaSpace{}%
\AgdaOperator{\AgdaFunction{∈}}\AgdaSpace{}%
\AgdaBound{B}\<%
\\
%
\>[1]\AgdaFunction{cong-pred}\AgdaSpace{}%
\AgdaBound{x}\AgdaSpace{}%
\AgdaDottedPattern{\AgdaSymbol{.}}\AgdaDottedPattern{\AgdaBound{x}}\AgdaSpace{}%
\AgdaBound{x∈B}\AgdaSpace{}%
\AgdaSymbol{(}\AgdaInductiveConstructor{refl}\AgdaSpace{}%
\AgdaSymbol{\AgdaUnderscore{}}\AgdaSpace{}%
\AgdaSymbol{)}\AgdaSpace{}%
\AgdaSymbol{=}\AgdaSpace{}%
\AgdaBound{x∈B}\<%
\end{code}

\begin{center}\rule{0.5\linewidth}{\linethickness}\end{center}

\href{UALib.Relations.html}{↑ UALib.Relations}
{\href{UALib.Relations.Binary.html}{UALib.Relations.Binary →}}

\{\% include UALib.Links.md \%\}
 -------------------------------------------









\subsection{Binary Relations and Kernel Types}\label{binary-relation-and-kernel-types}
This section highlights a few relevant parts of the \ualibBinary module of the \agdaualib.
%%-- BEGIN: In set theory, a binary relation on a set \ab{A} is simply a subset of the product \ab A \af × \ab A. As such, we could model these as predicates over the type \ab A \af × \ab A, or as relations of type \ab A \as → \ab A \as → \ab 𝓡 \af ̇ (for some universe \ab 𝓡). We define these below.

A generalization of the notion of binary relation is a \emph{relation from} \ab{A} \emph{to} \ab{B}, which we define first and treat binary relations on a single \ab{A} as a special case.
\ccpad
\begin{code}%
\>[0]\AgdaKeyword{module}\AgdaSpace{}%
\AgdaModule{UALib.Relations.Binary}\AgdaSpace{}%
\AgdaKeyword{where}\<%
\\
%
\\[\AgdaEmptyExtraSkip]%
\>[0]\AgdaKeyword{open}\AgdaSpace{}%
\AgdaKeyword{import}\AgdaSpace{}%
\AgdaModule{UALib.Relations.Unary}\AgdaSpace{}%
\AgdaKeyword{public}\<%
\\
%
\\[\AgdaEmptyExtraSkip]%
\>[0]\AgdaKeyword{module}\AgdaSpace{}%
\AgdaModule{\AgdaUnderscore{}}\AgdaSpace{}%
\AgdaSymbol{\{}\AgdaBound{𝓤}\AgdaSpace{}%
\AgdaSymbol{:}\AgdaSpace{}%
\AgdaPostulate{Universe}\AgdaSymbol{\}}\AgdaSpace{}%
\AgdaKeyword{where}\<%
\\
%
\\[\AgdaEmptyExtraSkip]%
\>[0][@{}l@{\AgdaIndent{0}}]%
\>[1]\AgdaFunction{REL}\AgdaSpace{}%
\AgdaSymbol{:}\AgdaSpace{}%
\AgdaSymbol{\{}\AgdaBound{𝓡}\AgdaSpace{}%
\AgdaSymbol{:}\AgdaSpace{}%
\AgdaPostulate{Universe}\AgdaSymbol{\}}\AgdaSpace{}%
\AgdaSymbol{→}\AgdaSpace{}%
\AgdaBound{𝓤}\AgdaSpace{}%
\AgdaOperator{\AgdaFunction{̇}}\AgdaSpace{}%
\AgdaSymbol{→}\AgdaSpace{}%
\AgdaBound{𝓡}\AgdaSpace{}%
\AgdaOperator{\AgdaFunction{̇}}\AgdaSpace{}%
\AgdaSymbol{→}\AgdaSpace{}%
\AgdaSymbol{(}\AgdaBound{𝓝}\AgdaSpace{}%
\AgdaSymbol{:}\AgdaSpace{}%
\AgdaPostulate{Universe}\AgdaSymbol{)}\AgdaSpace{}%
\AgdaSymbol{→}\AgdaSpace{}%
\AgdaSymbol{(}\AgdaBound{𝓤}\AgdaSpace{}%
\AgdaOperator{\AgdaPrimitive{⊔}}\AgdaSpace{}%
\AgdaBound{𝓡}\AgdaSpace{}%
\AgdaOperator{\AgdaPrimitive{⊔}}\AgdaSpace{}%
\AgdaBound{𝓝}\AgdaSpace{}%
\AgdaOperator{\AgdaPrimitive{⁺}}\AgdaSymbol{)}\AgdaSpace{}%
\AgdaOperator{\AgdaFunction{̇}}\<%
\\
%
\>[1]\AgdaFunction{REL}\AgdaSpace{}%
\AgdaBound{A}\AgdaSpace{}%
\AgdaBound{B}\AgdaSpace{}%
\AgdaBound{𝓝}\AgdaSpace{}%
\AgdaSymbol{=}\AgdaSpace{}%
\AgdaBound{A}\AgdaSpace{}%
\AgdaSymbol{→}\AgdaSpace{}%
\AgdaBound{B}\AgdaSpace{}%
\AgdaSymbol{→}\AgdaSpace{}%
\AgdaBound{𝓝}\AgdaSpace{}%
\AgdaOperator{\AgdaFunction{̇}}\<%
\end{code}

\subsubsection{Kernels}\label{Binary.sssec:kernels}
The kernel of a function can be defined in many ways. For example,
\ccpad
\begin{code}%
\>[0][@{}l@{\AgdaIndent{1}}]%
\>[1]\AgdaFunction{KER}\AgdaSpace{}%
\AgdaSymbol{:}\AgdaSpace{}%
\AgdaSymbol{\{}\AgdaBound{𝓡}\AgdaSpace{}%
\AgdaSymbol{:}\AgdaSpace{}%
\AgdaPostulate{Universe}\AgdaSymbol{\}}\AgdaSpace{}%
\AgdaSymbol{\{}\AgdaBound{A}\AgdaSpace{}%
\AgdaSymbol{:}\AgdaSpace{}%
\AgdaBound{𝓤}\AgdaSpace{}%
\AgdaOperator{\AgdaFunction{̇}}\AgdaSpace{}%
\AgdaSymbol{\}}\AgdaSpace{}%
\AgdaSymbol{\{}\AgdaBound{B}\AgdaSpace{}%
\AgdaSymbol{:}\AgdaSpace{}%
\AgdaBound{𝓡}\AgdaSpace{}%
\AgdaOperator{\AgdaFunction{̇}}\AgdaSpace{}%
\AgdaSymbol{\}}\AgdaSpace{}%
\AgdaSymbol{→}\AgdaSpace{}%
\AgdaSymbol{(}\AgdaBound{A}\AgdaSpace{}%
\AgdaSymbol{→}\AgdaSpace{}%
\AgdaBound{B}\AgdaSymbol{)}\AgdaSpace{}%
\AgdaSymbol{→}\AgdaSpace{}%
\AgdaBound{𝓤}\AgdaSpace{}%
\AgdaOperator{\AgdaPrimitive{⊔}}\AgdaSpace{}%
\AgdaBound{𝓡}\AgdaSpace{}%
\AgdaOperator{\AgdaFunction{̇}}\<%
\\
%
\>[1]\AgdaFunction{KER}\AgdaSpace{}%
\AgdaSymbol{\{}\AgdaBound{𝓡}\AgdaSymbol{\}}\AgdaSpace{}%
\AgdaSymbol{\{}\AgdaBound{A}\AgdaSymbol{\}}\AgdaSpace{}%
\AgdaBound{g}\AgdaSpace{}%
\AgdaSymbol{=}\AgdaSpace{}%
\AgdaFunction{Σ}\AgdaSpace{}%
\AgdaBound{x}\AgdaSpace{}%
\AgdaFunction{꞉}\AgdaSpace{}%
\AgdaBound{A}\AgdaSpace{}%
\AgdaFunction{,}\AgdaSpace{}%
\AgdaFunction{Σ}\AgdaSpace{}%
\AgdaBound{y}\AgdaSpace{}%
\AgdaFunction{꞉}\AgdaSpace{}%
\AgdaBound{A}\AgdaSpace{}%
\AgdaFunction{,}\AgdaSpace{}%
\AgdaBound{g}\AgdaSpace{}%
\AgdaBound{x}\AgdaSpace{}%
\AgdaOperator{\AgdaDatatype{≡}}\AgdaSpace{}%
\AgdaBound{g}\AgdaSpace{}%
\AgdaBound{y}\<%
\end{code}
\ccpad
or as a unary relation (predicate) over the Cartesian product,
\ccpad
\begin{code}%
\>[0][@{}l@{\AgdaIndent{1}}]%
\>[1]\AgdaFunction{KER-pred}\AgdaSpace{}%
\AgdaSymbol{:}\AgdaSpace{}%
\AgdaSymbol{\{}\AgdaBound{𝓡}\AgdaSpace{}%
\AgdaSymbol{:}\AgdaSpace{}%
\AgdaPostulate{Universe}\AgdaSymbol{\}}\AgdaSpace{}%
\AgdaSymbol{\{}\AgdaBound{A}\AgdaSpace{}%
\AgdaSymbol{:}\AgdaSpace{}%
\AgdaBound{𝓤}\AgdaSpace{}%
\AgdaOperator{\AgdaFunction{̇}}\AgdaSymbol{\}\{}\AgdaBound{B}\AgdaSpace{}%
\AgdaSymbol{:}\AgdaSpace{}%
\AgdaBound{𝓡}\AgdaSpace{}%
\AgdaOperator{\AgdaFunction{̇}}\AgdaSymbol{\}}\AgdaSpace{}%
\AgdaSymbol{→}\AgdaSpace{}%
\AgdaSymbol{(}\AgdaBound{A}\AgdaSpace{}%
\AgdaSymbol{→}\AgdaSpace{}%
\AgdaBound{B}\AgdaSymbol{)}\AgdaSpace{}%
\AgdaSymbol{→}\AgdaSpace{}%
\AgdaFunction{Pred}\AgdaSpace{}%
\AgdaSymbol{(}\AgdaBound{A}\AgdaSpace{}%
\AgdaOperator{\AgdaFunction{×}}\AgdaSpace{}%
\AgdaBound{A}\AgdaSymbol{)}\AgdaSpace{}%
\AgdaBound{𝓡}\<%
\\
%
\>[1]\AgdaFunction{KER-pred}\AgdaSpace{}%
\AgdaBound{g}\AgdaSpace{}%
\AgdaSymbol{(}\AgdaBound{x}\AgdaSpace{}%
\AgdaOperator{\AgdaInductiveConstructor{,}}\AgdaSpace{}%
\AgdaBound{y}\AgdaSymbol{)}\AgdaSpace{}%
\AgdaSymbol{=}\AgdaSpace{}%
\AgdaBound{g}\AgdaSpace{}%
\AgdaBound{x}\AgdaSpace{}%
\AgdaOperator{\AgdaDatatype{≡}}\AgdaSpace{}%
\AgdaBound{g}\AgdaSpace{}%
\AgdaBound{y}\<%
\end{code}
\ccpad
or as a relation from \ab{A} to \ab{B},
\ccpad
\begin{code}%
\>[0][@{}l@{\AgdaIndent{1}}]%
\>[1]\AgdaFunction{Rel}\AgdaSpace{}%
\AgdaSymbol{:}\AgdaSpace{}%
\AgdaBound{𝓤}\AgdaSpace{}%
\AgdaOperator{\AgdaFunction{̇}}\AgdaSpace{}%
\AgdaSymbol{→}\AgdaSpace{}%
\AgdaSymbol{(}\AgdaBound{𝓝}\AgdaSpace{}%
\AgdaSymbol{:}\AgdaSpace{}%
\AgdaPostulate{Universe}\AgdaSymbol{)}\AgdaSpace{}%
\AgdaSymbol{→}\AgdaSpace{}%
\AgdaBound{𝓤}\AgdaSpace{}%
\AgdaOperator{\AgdaPrimitive{⊔}}\AgdaSpace{}%
\AgdaBound{𝓝}\AgdaSpace{}%
\AgdaOperator{\AgdaPrimitive{⁺}}\AgdaSpace{}%
\AgdaOperator{\AgdaFunction{̇}}\<%
\\
%
\>[1]\AgdaFunction{Rel}\AgdaSpace{}%
\AgdaBound{A}\AgdaSpace{}%
\AgdaBound{𝓝}\AgdaSpace{}%
\AgdaSymbol{=}\AgdaSpace{}%
\AgdaFunction{REL}\AgdaSpace{}%
\AgdaBound{A}\AgdaSpace{}%
\AgdaBound{A}\AgdaSpace{}%
\AgdaBound{𝓝}\<%
\\
%
\\[\AgdaEmptyExtraSkip]%
%
\>[1]\AgdaFunction{KER-rel}\AgdaSpace{}%
\AgdaSymbol{:}\AgdaSpace{}%
\AgdaSymbol{\{}\AgdaBound{𝓡}\AgdaSpace{}%
\AgdaSymbol{:}\AgdaSpace{}%
\AgdaPostulate{Universe}\AgdaSymbol{\}\{}\AgdaBound{A}\AgdaSpace{}%
\AgdaSymbol{:}\AgdaSpace{}%
\AgdaBound{𝓤}\AgdaSpace{}%
\AgdaOperator{\AgdaFunction{̇}}\AgdaSpace{}%
\AgdaSymbol{\}}\AgdaSpace{}%
\AgdaSymbol{\{}\AgdaBound{B}\AgdaSpace{}%
\AgdaSymbol{:}\AgdaSpace{}%
\AgdaBound{𝓡}\AgdaSpace{}%
\AgdaOperator{\AgdaFunction{̇}}\AgdaSpace{}%
\AgdaSymbol{\}}\AgdaSpace{}%
\AgdaSymbol{→}\AgdaSpace{}%
\AgdaSymbol{(}\AgdaBound{A}\AgdaSpace{}%
\AgdaSymbol{→}\AgdaSpace{}%
\AgdaBound{B}\AgdaSymbol{)}\AgdaSpace{}%
\AgdaSymbol{→}\AgdaSpace{}%
\AgdaFunction{Rel}\AgdaSpace{}%
\AgdaBound{A}\AgdaSpace{}%
\AgdaBound{𝓡}\<%
\\
%
\>[1]\AgdaFunction{KER-rel}\AgdaSpace{}%
\AgdaBound{g}\AgdaSpace{}%
\AgdaBound{x}\AgdaSpace{}%
\AgdaBound{y}\AgdaSpace{}%
\AgdaSymbol{=}\AgdaSpace{}%
\AgdaBound{g}\AgdaSpace{}%
\AgdaBound{x}\AgdaSpace{}%
\AgdaOperator{\AgdaDatatype{≡}}\AgdaSpace{}%
\AgdaBound{g}\AgdaSpace{}%
\AgdaBound{y}\<%
\end{code}

\subsubsection{Examples}\label{Binary.sssec:examples}

\begin{code}%
\>[0][@{}l@{\AgdaIndent{1}}]%
\>[1]\AgdaFunction{ker}\AgdaSpace{}%
\AgdaSymbol{:}\AgdaSpace{}%
\AgdaSymbol{\{}\AgdaBound{A}\AgdaSpace{}%
\AgdaBound{B}\AgdaSpace{}%
\AgdaSymbol{:}\AgdaSpace{}%
\AgdaBound{𝓤}\AgdaSpace{}%
\AgdaOperator{\AgdaFunction{̇}}\AgdaSpace{}%
\AgdaSymbol{\}}\AgdaSpace{}%
\AgdaSymbol{→}\AgdaSpace{}%
\AgdaSymbol{(}\AgdaBound{A}\AgdaSpace{}%
\AgdaSymbol{→}\AgdaSpace{}%
\AgdaBound{B}\AgdaSymbol{)}\AgdaSpace{}%
\AgdaSymbol{→}\AgdaSpace{}%
\AgdaBound{𝓤}\AgdaSpace{}%
\AgdaOperator{\AgdaFunction{̇}}\<%
\\
%
\>[1]\AgdaFunction{ker}\AgdaSpace{}%
\AgdaSymbol{=}\AgdaSpace{}%
\AgdaFunction{KER}\AgdaSymbol{\{}\AgdaBound{𝓤}\AgdaSymbol{\}}\<%
\\
%
\\[\AgdaEmptyExtraSkip]%
%
\>[1]\AgdaFunction{ker-rel}\AgdaSpace{}%
\AgdaSymbol{:}\AgdaSpace{}%
\AgdaSymbol{\{}\AgdaBound{A}\AgdaSpace{}%
\AgdaBound{B}\AgdaSpace{}%
\AgdaSymbol{:}\AgdaSpace{}%
\AgdaBound{𝓤}\AgdaSpace{}%
\AgdaOperator{\AgdaFunction{̇}}\AgdaSpace{}%
\AgdaSymbol{\}}\AgdaSpace{}%
\AgdaSymbol{→}\AgdaSpace{}%
\AgdaSymbol{(}\AgdaBound{A}\AgdaSpace{}%
\AgdaSymbol{→}\AgdaSpace{}%
\AgdaBound{B}\AgdaSymbol{)}\AgdaSpace{}%
\AgdaSymbol{→}\AgdaSpace{}%
\AgdaFunction{Rel}\AgdaSpace{}%
\AgdaBound{A}\AgdaSpace{}%
\AgdaBound{𝓤}\<%
\\
%
\>[1]\AgdaFunction{ker-rel}\AgdaSpace{}%
\AgdaSymbol{=}\AgdaSpace{}%
\AgdaFunction{KER-rel}\AgdaSpace{}%
\AgdaSymbol{\{}\AgdaBound{𝓤}\AgdaSymbol{\}}\<%
\\
%
\\[\AgdaEmptyExtraSkip]%
%
\>[1]\AgdaFunction{ker-pred}\AgdaSpace{}%
\AgdaSymbol{:}\AgdaSpace{}%
\AgdaSymbol{\{}\AgdaBound{A}\AgdaSpace{}%
\AgdaBound{B}\AgdaSpace{}%
\AgdaSymbol{:}\AgdaSpace{}%
\AgdaBound{𝓤}\AgdaSpace{}%
\AgdaOperator{\AgdaFunction{̇}}\AgdaSpace{}%
\AgdaSymbol{\}}\AgdaSpace{}%
\AgdaSymbol{→}\AgdaSpace{}%
\AgdaSymbol{(}\AgdaBound{A}\AgdaSpace{}%
\AgdaSymbol{→}\AgdaSpace{}%
\AgdaBound{B}\AgdaSymbol{)}\AgdaSpace{}%
\AgdaSymbol{→}\AgdaSpace{}%
\AgdaFunction{Pred}\AgdaSpace{}%
\AgdaSymbol{(}\AgdaBound{A}\AgdaSpace{}%
\AgdaOperator{\AgdaFunction{×}}\AgdaSpace{}%
\AgdaBound{A}\AgdaSymbol{)}\AgdaSpace{}%
\AgdaBound{𝓤}\<%
\\
%
\>[1]\AgdaFunction{ker-pred}\AgdaSpace{}%
\AgdaSymbol{=}\AgdaSpace{}%
\AgdaFunction{KER-pred}\AgdaSpace{}%
\AgdaSymbol{\{}\AgdaBound{𝓤}\AgdaSymbol{\}}\<%
\\
%
\\[\AgdaEmptyExtraSkip]%
%
\>[1]\AgdaComment{--The identity relation.}\<%
\\
%
\>[1]\AgdaFunction{𝟎}\AgdaSpace{}%
\AgdaSymbol{:}\AgdaSpace{}%
\AgdaSymbol{\{}\AgdaBound{A}\AgdaSpace{}%
\AgdaSymbol{:}\AgdaSpace{}%
\AgdaBound{𝓤}\AgdaSpace{}%
\AgdaOperator{\AgdaFunction{̇}}\AgdaSpace{}%
\AgdaSymbol{\}}\AgdaSpace{}%
\AgdaSymbol{→}\AgdaSpace{}%
\AgdaBound{𝓤}\AgdaSpace{}%
\AgdaOperator{\AgdaFunction{̇}}\<%
\\
%
\>[1]\AgdaFunction{𝟎}\AgdaSpace{}%
\AgdaSymbol{\{}\AgdaBound{A}\AgdaSymbol{\}}\AgdaSpace{}%
\AgdaSymbol{=}\AgdaSpace{}%
\AgdaFunction{Σ}\AgdaSpace{}%
\AgdaBound{a}\AgdaSpace{}%
\AgdaFunction{꞉}\AgdaSpace{}%
\AgdaBound{A}\AgdaSpace{}%
\AgdaFunction{,}\AgdaSpace{}%
\AgdaFunction{Σ}\AgdaSpace{}%
\AgdaBound{b}\AgdaSpace{}%
\AgdaFunction{꞉}\AgdaSpace{}%
\AgdaBound{A}\AgdaSpace{}%
\AgdaFunction{,}\AgdaSpace{}%
\AgdaBound{a}\AgdaSpace{}%
\AgdaOperator{\AgdaDatatype{≡}}\AgdaSpace{}%
\AgdaBound{b}\<%
\\
%
\\[\AgdaEmptyExtraSkip]%
%
\>[1]\AgdaComment{--...as a binary relation...}\<%
\\
%
\>[1]\AgdaFunction{𝟎-rel}\AgdaSpace{}%
\AgdaSymbol{:}\AgdaSpace{}%
\AgdaSymbol{\{}\AgdaBound{A}\AgdaSpace{}%
\AgdaSymbol{:}\AgdaSpace{}%
\AgdaBound{𝓤}\AgdaSpace{}%
\AgdaOperator{\AgdaFunction{̇}}\AgdaSpace{}%
\AgdaSymbol{\}}\AgdaSpace{}%
\AgdaSymbol{→}\AgdaSpace{}%
\AgdaFunction{Rel}\AgdaSpace{}%
\AgdaBound{A}\AgdaSpace{}%
\AgdaBound{𝓤}\<%
\\
%
\>[1]\AgdaFunction{𝟎-rel}\AgdaSpace{}%
\AgdaBound{a}\AgdaSpace{}%
\AgdaBound{b}\AgdaSpace{}%
\AgdaSymbol{=}\AgdaSpace{}%
\AgdaBound{a}\AgdaSpace{}%
\AgdaOperator{\AgdaDatatype{≡}}\AgdaSpace{}%
\AgdaBound{b}\<%
\\
%
\\[\AgdaEmptyExtraSkip]%
%
\>[1]\AgdaComment{--...as a binary predicate...}\<%
\\
%
\>[1]\AgdaFunction{𝟎-pred}\AgdaSpace{}%
\AgdaSymbol{:}\AgdaSpace{}%
\AgdaSymbol{\{}\AgdaBound{A}\AgdaSpace{}%
\AgdaSymbol{:}\AgdaSpace{}%
\AgdaBound{𝓤}\AgdaSpace{}%
\AgdaOperator{\AgdaFunction{̇}}\AgdaSpace{}%
\AgdaSymbol{\}}\AgdaSpace{}%
\AgdaSymbol{→}\AgdaSpace{}%
\AgdaFunction{Pred}\AgdaSpace{}%
\AgdaSymbol{(}\AgdaBound{A}\AgdaSpace{}%
\AgdaOperator{\AgdaFunction{×}}\AgdaSpace{}%
\AgdaBound{A}\AgdaSymbol{)}\AgdaSpace{}%
\AgdaBound{𝓤}\<%
\\
%
\>[1]\AgdaFunction{𝟎-pred}\AgdaSpace{}%
\AgdaSymbol{(}\AgdaBound{a}\AgdaSpace{}%
\AgdaOperator{\AgdaInductiveConstructor{,}}\AgdaSpace{}%
\AgdaBound{a'}\AgdaSymbol{)}\AgdaSpace{}%
\AgdaSymbol{=}\AgdaSpace{}%
\AgdaBound{a}\AgdaSpace{}%
\AgdaOperator{\AgdaDatatype{≡}}\AgdaSpace{}%
\AgdaBound{a'}\<%
\\
%
\\[\AgdaEmptyExtraSkip]%
%
\>[1]\AgdaFunction{𝟎-pred'}\AgdaSpace{}%
\AgdaSymbol{:}\AgdaSpace{}%
\AgdaSymbol{\{}\AgdaBound{A}\AgdaSpace{}%
\AgdaSymbol{:}\AgdaSpace{}%
\AgdaBound{𝓤}\AgdaSpace{}%
\AgdaOperator{\AgdaFunction{̇}}\AgdaSpace{}%
\AgdaSymbol{\}}\AgdaSpace{}%
\AgdaSymbol{→}\AgdaSpace{}%
\AgdaBound{𝓤}\AgdaSpace{}%
\AgdaOperator{\AgdaFunction{̇}}\<%
\\
%
\>[1]\AgdaFunction{𝟎-pred'}\AgdaSpace{}%
\AgdaSymbol{\{}\AgdaBound{A}\AgdaSymbol{\}}\AgdaSpace{}%
\AgdaSymbol{=}\AgdaSpace{}%
\AgdaFunction{Σ}\AgdaSpace{}%
\AgdaBound{p}\AgdaSpace{}%
\AgdaFunction{꞉}\AgdaSpace{}%
\AgdaSymbol{(}\AgdaBound{A}\AgdaSpace{}%
\AgdaOperator{\AgdaFunction{×}}\AgdaSpace{}%
\AgdaBound{A}\AgdaSymbol{)}\AgdaSpace{}%
\AgdaFunction{,}\AgdaSpace{}%
\AgdaOperator{\AgdaFunction{∣}}\AgdaSpace{}%
\AgdaBound{p}\AgdaSpace{}%
\AgdaOperator{\AgdaFunction{∣}}\AgdaSpace{}%
\AgdaOperator{\AgdaDatatype{≡}}\AgdaSpace{}%
\AgdaOperator{\AgdaFunction{∥}}\AgdaSpace{}%
\AgdaBound{p}\AgdaSpace{}%
\AgdaOperator{\AgdaFunction{∥}}\<%
\\
%
\\[\AgdaEmptyExtraSkip]%
%
\>[1]\AgdaComment{--...on the domain of an algebra...}\<%
\\
%
\\[\AgdaEmptyExtraSkip]%
%
\>[1]\AgdaFunction{𝟎-alg-rel}\AgdaSpace{}%
\AgdaSymbol{:}\AgdaSpace{}%
\AgdaSymbol{\{}\AgdaBound{𝑆}\AgdaSpace{}%
\AgdaSymbol{:}\AgdaSpace{}%
\AgdaFunction{Signature}\AgdaSpace{}%
\AgdaGeneralizable{𝓞}\AgdaSpace{}%
\AgdaGeneralizable{𝓥}\AgdaSymbol{\}\{}\AgdaBound{𝑨}\AgdaSpace{}%
\AgdaSymbol{:}\AgdaSpace{}%
\AgdaFunction{Algebra}\AgdaSpace{}%
\AgdaBound{𝓤}\AgdaSpace{}%
\AgdaBound{𝑆}\AgdaSymbol{\}}\AgdaSpace{}%
\AgdaSymbol{→}\AgdaSpace{}%
\AgdaBound{𝓤}\AgdaSpace{}%
\AgdaOperator{\AgdaFunction{̇}}\<%
\\
%
\>[1]\AgdaFunction{𝟎-alg-rel}\AgdaSpace{}%
\AgdaSymbol{\{}\AgdaArgument{𝑨}\AgdaSpace{}%
\AgdaSymbol{=}\AgdaSpace{}%
\AgdaBound{𝑨}\AgdaSymbol{\}}\AgdaSpace{}%
\AgdaSymbol{=}\AgdaSpace{}%
\AgdaFunction{Σ}\AgdaSpace{}%
\AgdaBound{a}\AgdaSpace{}%
\AgdaFunction{꞉}\AgdaSpace{}%
\AgdaOperator{\AgdaFunction{∣}}\AgdaSpace{}%
\AgdaBound{𝑨}\AgdaSpace{}%
\AgdaOperator{\AgdaFunction{∣}}\AgdaSpace{}%
\AgdaFunction{,}\AgdaSpace{}%
\AgdaFunction{Σ}\AgdaSpace{}%
\AgdaBound{b}\AgdaSpace{}%
\AgdaFunction{꞉}\AgdaSpace{}%
\AgdaOperator{\AgdaFunction{∣}}\AgdaSpace{}%
\AgdaBound{𝑨}\AgdaSpace{}%
\AgdaOperator{\AgdaFunction{∣}}\AgdaSpace{}%
\AgdaFunction{,}\AgdaSpace{}%
\AgdaBound{a}\AgdaSpace{}%
\AgdaOperator{\AgdaDatatype{≡}}\AgdaSpace{}%
\AgdaBound{b}\<%
\\
%
\\[\AgdaEmptyExtraSkip]%
%
\>[1]\AgdaComment{-- The total relation A × A}\<%
\\
%
\>[1]\AgdaFunction{𝟏}\AgdaSpace{}%
\AgdaSymbol{:}\AgdaSpace{}%
\AgdaSymbol{\{}\AgdaBound{A}\AgdaSpace{}%
\AgdaSymbol{:}\AgdaSpace{}%
\AgdaBound{𝓤}\AgdaSpace{}%
\AgdaOperator{\AgdaFunction{̇}}\AgdaSpace{}%
\AgdaSymbol{\}}\AgdaSpace{}%
\AgdaSymbol{→}\AgdaSpace{}%
\AgdaFunction{Rel}\AgdaSpace{}%
\AgdaBound{A}\AgdaSpace{}%
\AgdaPrimitive{𝓤₀}\<%
\\
%
\>[1]\AgdaFunction{𝟏}\AgdaSpace{}%
\AgdaBound{a}\AgdaSpace{}%
\AgdaBound{b}\AgdaSpace{}%
\AgdaSymbol{=}\AgdaSpace{}%
\AgdaFunction{𝟙}\<%
\end{code}

\subsubsection{Properties of binary relations}\label{Binary.sssec:properties-of-binary-relations}
\begin{code}%
\>[1]\AgdaFunction{reflexive}\AgdaSpace{}%
\AgdaSymbol{:}\AgdaSpace{}%
\AgdaSymbol{\{}\AgdaBound{𝓡}\AgdaSpace{}%
\AgdaSymbol{:}\AgdaSpace{}%
\AgdaPostulate{Universe}\AgdaSymbol{\}\{}\AgdaBound{X}\AgdaSpace{}%
\AgdaSymbol{:}\AgdaSpace{}%
\AgdaBound{𝓤}\AgdaSpace{}%
\AgdaOperator{\AgdaFunction{̇}}\AgdaSpace{}%
\AgdaSymbol{\}}\AgdaSpace{}%
\AgdaSymbol{→}\AgdaSpace{}%
\AgdaFunction{Rel}\AgdaSpace{}%
\AgdaBound{X}\AgdaSpace{}%
\AgdaBound{𝓡}\AgdaSpace{}%
\AgdaSymbol{→}\AgdaSpace{}%
\AgdaBound{𝓤}\AgdaSpace{}%
\AgdaOperator{\AgdaPrimitive{⊔}}\AgdaSpace{}%
\AgdaBound{𝓡}\AgdaSpace{}%
\AgdaOperator{\AgdaFunction{̇}}\<%
\\
%
\>[1]\AgdaFunction{reflexive}\AgdaSpace{}%
\AgdaOperator{\AgdaBound{\AgdaUnderscore{}≈\AgdaUnderscore{}}}\AgdaSpace{}%
\AgdaSymbol{=}\AgdaSpace{}%
\AgdaSymbol{∀}\AgdaSpace{}%
\AgdaBound{x}\AgdaSpace{}%
\AgdaSymbol{→}\AgdaSpace{}%
\AgdaBound{x}\AgdaSpace{}%
\AgdaOperator{\AgdaBound{≈}}\AgdaSpace{}%
\AgdaBound{x}\<%
\\
%
\\[\AgdaEmptyExtraSkip]%
%
\>[1]\AgdaFunction{symmetric}\AgdaSpace{}%
\AgdaSymbol{:}\AgdaSpace{}%
\AgdaSymbol{\{}\AgdaBound{𝓡}\AgdaSpace{}%
\AgdaSymbol{:}\AgdaSpace{}%
\AgdaPostulate{Universe}\AgdaSymbol{\}\{}\AgdaBound{X}\AgdaSpace{}%
\AgdaSymbol{:}\AgdaSpace{}%
\AgdaBound{𝓤}\AgdaSpace{}%
\AgdaOperator{\AgdaFunction{̇}}\AgdaSpace{}%
\AgdaSymbol{\}}\AgdaSpace{}%
\AgdaSymbol{→}\AgdaSpace{}%
\AgdaFunction{Rel}\AgdaSpace{}%
\AgdaBound{X}\AgdaSpace{}%
\AgdaBound{𝓡}\AgdaSpace{}%
\AgdaSymbol{→}\AgdaSpace{}%
\AgdaBound{𝓤}\AgdaSpace{}%
\AgdaOperator{\AgdaPrimitive{⊔}}\AgdaSpace{}%
\AgdaBound{𝓡}\AgdaSpace{}%
\AgdaOperator{\AgdaFunction{̇}}\<%
\\
%
\>[1]\AgdaFunction{symmetric}\AgdaSpace{}%
\AgdaOperator{\AgdaBound{\AgdaUnderscore{}≈\AgdaUnderscore{}}}\AgdaSpace{}%
\AgdaSymbol{=}\AgdaSpace{}%
\AgdaSymbol{∀}\AgdaSpace{}%
\AgdaBound{x}\AgdaSpace{}%
\AgdaBound{y}\AgdaSpace{}%
\AgdaSymbol{→}\AgdaSpace{}%
\AgdaBound{x}\AgdaSpace{}%
\AgdaOperator{\AgdaBound{≈}}\AgdaSpace{}%
\AgdaBound{y}\AgdaSpace{}%
\AgdaSymbol{→}\AgdaSpace{}%
\AgdaBound{y}\AgdaSpace{}%
\AgdaOperator{\AgdaBound{≈}}\AgdaSpace{}%
\AgdaBound{x}\<%
\\
%
\\[\AgdaEmptyExtraSkip]%
%
\>[1]\AgdaFunction{transitive}\AgdaSpace{}%
\AgdaSymbol{:}\AgdaSpace{}%
\AgdaSymbol{\{}\AgdaBound{𝓡}\AgdaSpace{}%
\AgdaSymbol{:}\AgdaSpace{}%
\AgdaPostulate{Universe}\AgdaSymbol{\}\{}\AgdaBound{X}\AgdaSpace{}%
\AgdaSymbol{:}\AgdaSpace{}%
\AgdaBound{𝓤}\AgdaSpace{}%
\AgdaOperator{\AgdaFunction{̇}}\AgdaSpace{}%
\AgdaSymbol{\}}\AgdaSpace{}%
\AgdaSymbol{→}\AgdaSpace{}%
\AgdaFunction{Rel}\AgdaSpace{}%
\AgdaBound{X}\AgdaSpace{}%
\AgdaBound{𝓡}\AgdaSpace{}%
\AgdaSymbol{→}\AgdaSpace{}%
\AgdaBound{𝓤}\AgdaSpace{}%
\AgdaOperator{\AgdaPrimitive{⊔}}\AgdaSpace{}%
\AgdaBound{𝓡}\AgdaSpace{}%
\AgdaOperator{\AgdaFunction{̇}}\<%
\\
%
\>[1]\AgdaFunction{transitive}\AgdaSpace{}%
\AgdaOperator{\AgdaBound{\AgdaUnderscore{}≈\AgdaUnderscore{}}}\AgdaSpace{}%
\AgdaSymbol{=}\AgdaSpace{}%
\AgdaSymbol{∀}\AgdaSpace{}%
\AgdaBound{x}\AgdaSpace{}%
\AgdaBound{y}\AgdaSpace{}%
\AgdaBound{z}\AgdaSpace{}%
\AgdaSymbol{→}\AgdaSpace{}%
\AgdaBound{x}\AgdaSpace{}%
\AgdaOperator{\AgdaBound{≈}}\AgdaSpace{}%
\AgdaBound{y}\AgdaSpace{}%
\AgdaSymbol{→}\AgdaSpace{}%
\AgdaBound{y}\AgdaSpace{}%
\AgdaOperator{\AgdaBound{≈}}\AgdaSpace{}%
\AgdaBound{z}\AgdaSpace{}%
\AgdaSymbol{→}\AgdaSpace{}%
\AgdaBound{x}\AgdaSpace{}%
\AgdaOperator{\AgdaBound{≈}}\AgdaSpace{}%
\AgdaBound{z}\<%
\\
%
\\[\AgdaEmptyExtraSkip]%
%
\>[1]\AgdaFunction{is-subsingleton-valued}\AgdaSpace{}%
\AgdaSymbol{:}\AgdaSpace{}%
\AgdaSymbol{\{}\AgdaBound{𝓡}\AgdaSpace{}%
\AgdaSymbol{:}\AgdaSpace{}%
\AgdaPostulate{Universe}\AgdaSymbol{\}\{}\AgdaBound{A}\AgdaSpace{}%
\AgdaSymbol{:}\AgdaSpace{}%
\AgdaBound{𝓤}\AgdaSpace{}%
\AgdaOperator{\AgdaFunction{̇}}\AgdaSpace{}%
\AgdaSymbol{\}}\AgdaSpace{}%
\AgdaSymbol{→}\AgdaSpace{}%
\AgdaFunction{Rel}\AgdaSpace{}%
\AgdaBound{A}\AgdaSpace{}%
\AgdaBound{𝓡}\AgdaSpace{}%
\AgdaSymbol{→}\AgdaSpace{}%
\AgdaBound{𝓤}\AgdaSpace{}%
\AgdaOperator{\AgdaPrimitive{⊔}}\AgdaSpace{}%
\AgdaBound{𝓡}\AgdaSpace{}%
\AgdaOperator{\AgdaFunction{̇}}\<%
\\
%
\>[1]\AgdaFunction{is-subsingleton-valued}%
\>[25]\AgdaOperator{\AgdaBound{\AgdaUnderscore{}≈\AgdaUnderscore{}}}\AgdaSpace{}%
\AgdaSymbol{=}\AgdaSpace{}%
\AgdaSymbol{∀}\AgdaSpace{}%
\AgdaBound{x}\AgdaSpace{}%
\AgdaBound{y}\AgdaSpace{}%
\AgdaSymbol{→}\AgdaSpace{}%
\AgdaFunction{is-prop}\AgdaSpace{}%
\AgdaSymbol{(}\AgdaBound{x}\AgdaSpace{}%
\AgdaOperator{\AgdaBound{≈}}\AgdaSpace{}%
\AgdaBound{y}\AgdaSymbol{)}\<%
\end{code}

\subsubsection{Binary relation truncation}\label{Binary.sssec:binary-relation-truncation}
Recall, in Section~\ref{Preliminaries.sssec:truncation} we described the concept of truncation as it relates to ``proof-relevant'' mathematics.  Given a binary relation \af P, it may be necessary or desirable to assume that there is at most one way to prove that a given pair of elements is \af P-related\footnote{This is another example of ``proof-irrelevance''; indeed, proofs of \af P \ab x \ab y are indistinguishable, or rather any distinctions are irrelevant in the context of interest.} We use Escardo's \af{is-subsingleton} type to express this strong (truncation at level 1) assumption in the following definition: We say that (\ab x  \ac \ab y) belongs to \af P, or that \ab x and \ab y are \af P-related if and only if both \af P \ab x \ab y \emph{and} \af{is-subsingleton} (\af P \ab x \ab y) holds.
\ccpad
\begin{code}%
\>[1]\AgdaFunction{Rel₀}\AgdaSpace{}%
\AgdaSymbol{:}\AgdaSpace{}%
\AgdaBound{𝓤}\AgdaSpace{}%
\AgdaOperator{\AgdaFunction{̇}}\AgdaSpace{}%
\AgdaSymbol{→}\AgdaSpace{}%
\AgdaSymbol{(}\AgdaBound{𝓝}\AgdaSpace{}%
\AgdaSymbol{:}\AgdaSpace{}%
\AgdaPostulate{Universe}\AgdaSymbol{)}\AgdaSpace{}%
\AgdaSymbol{→}\AgdaSpace{}%
\AgdaBound{𝓤}\AgdaSpace{}%
\AgdaOperator{\AgdaPrimitive{⊔}}\AgdaSpace{}%
\AgdaBound{𝓝}\AgdaSpace{}%
\AgdaOperator{\AgdaPrimitive{⁺}}\AgdaSpace{}%
\AgdaOperator{\AgdaFunction{̇}}\<%
\\
%
\>[1]\AgdaFunction{Rel₀}\AgdaSpace{}%
\AgdaBound{A}\AgdaSpace{}%
\AgdaBound{𝓝}\AgdaSpace{}%
\AgdaSymbol{=}\AgdaSpace{}%
\AgdaFunction{Σ}\AgdaSpace{}%
\AgdaBound{P}\AgdaSpace{}%
\AgdaFunction{꞉}\AgdaSpace{}%
\AgdaSymbol{(}\AgdaBound{A}\AgdaSpace{}%
\AgdaSymbol{→}\AgdaSpace{}%
\AgdaBound{A}\AgdaSpace{}%
\AgdaSymbol{→}\AgdaSpace{}%
\AgdaBound{𝓝}\AgdaSpace{}%
\AgdaOperator{\AgdaFunction{̇}}\AgdaSymbol{)}\AgdaSpace{}%
\AgdaFunction{,}\AgdaSpace{}%
\AgdaSymbol{∀}\AgdaSpace{}%
\AgdaBound{x}\AgdaSpace{}%
\AgdaBound{y}\AgdaSpace{}%
\AgdaSymbol{→}\AgdaSpace{}%
\AgdaFunction{is-subsingleton}\AgdaSpace{}%
\AgdaSymbol{(}\AgdaBound{P}\AgdaSpace{}%
\AgdaBound{x}\AgdaSpace{}%
\AgdaBound{y}\AgdaSymbol{)}\<%
\end{code}
\ccpad
As above we define a \textbf{set} to be a type \ab{X} with the following property: for all \ab x \ab y \as : \ab X there is at most one proof that \ab x \ad ≡ \ab y. In other words, \ab{X} is a set if and only if it satisfies the following:
\ccpad

\as ∀ \ab x \ab y \as : \ab X  \as → \af{is-subsingleton} ( \ab x \af ≡ \ab y )

\subsubsection{Implication}\label{implication}

We denote and define implication as follows.

\begin{code}%
\>[0]\<%
\\
\>[0]\AgdaComment{-- (syntactic sugar)}\<%
\\
\>[0]\AgdaOperator{\AgdaFunction{\AgdaUnderscore{}on\AgdaUnderscore{}}}\AgdaSpace{}%
\AgdaSymbol{:}\AgdaSpace{}%
\AgdaSymbol{\{}\AgdaBound{𝓤}\AgdaSpace{}%
\AgdaBound{𝓥}\AgdaSpace{}%
\AgdaBound{𝓦}\AgdaSpace{}%
\AgdaSymbol{:}\AgdaSpace{}%
\AgdaPostulate{Universe}\AgdaSymbol{\}\{}\AgdaBound{A}\AgdaSpace{}%
\AgdaSymbol{:}\AgdaSpace{}%
\AgdaBound{𝓤}\AgdaSpace{}%
\AgdaOperator{\AgdaFunction{̇}}\AgdaSymbol{\}\{}\AgdaBound{B}\AgdaSpace{}%
\AgdaSymbol{:}\AgdaSpace{}%
\AgdaBound{𝓥}\AgdaSpace{}%
\AgdaOperator{\AgdaFunction{̇}}\AgdaSymbol{\}\{}\AgdaBound{C}\AgdaSpace{}%
\AgdaSymbol{:}\AgdaSpace{}%
\AgdaBound{𝓦}\AgdaSpace{}%
\AgdaOperator{\AgdaFunction{̇}}\AgdaSymbol{\}}\<%
\\
\>[0][@{}l@{\AgdaIndent{0}}]%
\>[1]\AgdaSymbol{→}%
\>[7]\AgdaSymbol{(}\AgdaBound{B}\AgdaSpace{}%
\AgdaSymbol{→}\AgdaSpace{}%
\AgdaBound{B}\AgdaSpace{}%
\AgdaSymbol{→}\AgdaSpace{}%
\AgdaBound{C}\AgdaSymbol{)}\AgdaSpace{}%
\AgdaSymbol{→}\AgdaSpace{}%
\AgdaSymbol{(}\AgdaBound{A}\AgdaSpace{}%
\AgdaSymbol{→}\AgdaSpace{}%
\AgdaBound{B}\AgdaSymbol{)}\AgdaSpace{}%
\AgdaSymbol{→}\AgdaSpace{}%
\AgdaSymbol{(}\AgdaBound{A}\AgdaSpace{}%
\AgdaSymbol{→}\AgdaSpace{}%
\AgdaBound{A}\AgdaSpace{}%
\AgdaSymbol{→}\AgdaSpace{}%
\AgdaBound{C}\AgdaSymbol{)}\<%
\\
%
\\[\AgdaEmptyExtraSkip]%
\>[0]\AgdaOperator{\AgdaBound{\AgdaUnderscore{}*\AgdaUnderscore{}}}\AgdaSpace{}%
\AgdaOperator{\AgdaFunction{on}}\AgdaSpace{}%
\AgdaBound{g}\AgdaSpace{}%
\AgdaSymbol{=}\AgdaSpace{}%
\AgdaSymbol{λ}\AgdaSpace{}%
\AgdaBound{x}\AgdaSpace{}%
\AgdaBound{y}\AgdaSpace{}%
\AgdaSymbol{→}\AgdaSpace{}%
\AgdaBound{g}\AgdaSpace{}%
\AgdaBound{x}\AgdaSpace{}%
\AgdaOperator{\AgdaBound{*}}\AgdaSpace{}%
\AgdaBound{g}\AgdaSpace{}%
\AgdaBound{y}\<%
\\
%
\\[\AgdaEmptyExtraSkip]%
%
\\[\AgdaEmptyExtraSkip]%
\>[0]\AgdaOperator{\AgdaFunction{\AgdaUnderscore{}⇒\AgdaUnderscore{}}}\AgdaSpace{}%
\AgdaSymbol{:}\AgdaSpace{}%
\AgdaSymbol{\{}\AgdaBound{𝓤}\AgdaSpace{}%
\AgdaBound{𝓥}\AgdaSpace{}%
\AgdaBound{𝓦}\AgdaSpace{}%
\AgdaBound{𝓧}\AgdaSpace{}%
\AgdaSymbol{:}\AgdaSpace{}%
\AgdaPostulate{Universe}\AgdaSymbol{\}\{}\AgdaBound{A}\AgdaSpace{}%
\AgdaSymbol{:}\AgdaSpace{}%
\AgdaBound{𝓤}\AgdaSpace{}%
\AgdaOperator{\AgdaFunction{̇}}\AgdaSpace{}%
\AgdaSymbol{\}}\AgdaSpace{}%
\AgdaSymbol{\{}\AgdaBound{B}\AgdaSpace{}%
\AgdaSymbol{:}\AgdaSpace{}%
\AgdaBound{𝓥}\AgdaSpace{}%
\AgdaOperator{\AgdaFunction{̇}}\AgdaSpace{}%
\AgdaSymbol{\}}\<%
\\
\>[0][@{}l@{\AgdaIndent{0}}]%
\>[1]\AgdaSymbol{→}%
\>[6]\AgdaFunction{REL}\AgdaSpace{}%
\AgdaBound{A}\AgdaSpace{}%
\AgdaBound{B}\AgdaSpace{}%
\AgdaBound{𝓦}\AgdaSpace{}%
\AgdaSymbol{→}\AgdaSpace{}%
\AgdaFunction{REL}\AgdaSpace{}%
\AgdaBound{A}\AgdaSpace{}%
\AgdaBound{B}\AgdaSpace{}%
\AgdaBound{𝓧}\AgdaSpace{}%
\AgdaSymbol{→}\AgdaSpace{}%
\AgdaBound{𝓤}\AgdaSpace{}%
\AgdaOperator{\AgdaPrimitive{⊔}}\AgdaSpace{}%
\AgdaBound{𝓥}\AgdaSpace{}%
\AgdaOperator{\AgdaPrimitive{⊔}}\AgdaSpace{}%
\AgdaBound{𝓦}\AgdaSpace{}%
\AgdaOperator{\AgdaPrimitive{⊔}}\AgdaSpace{}%
\AgdaBound{𝓧}\AgdaSpace{}%
\AgdaOperator{\AgdaFunction{̇}}\<%
\\
%
\\[\AgdaEmptyExtraSkip]%
\>[0]\AgdaBound{P}\AgdaSpace{}%
\AgdaOperator{\AgdaFunction{⇒}}\AgdaSpace{}%
\AgdaBound{Q}\AgdaSpace{}%
\AgdaSymbol{=}\AgdaSpace{}%
\AgdaSymbol{∀}\AgdaSpace{}%
\AgdaSymbol{\{}\AgdaBound{i}\AgdaSpace{}%
\AgdaBound{j}\AgdaSymbol{\}}\AgdaSpace{}%
\AgdaSymbol{→}\AgdaSpace{}%
\AgdaBound{P}\AgdaSpace{}%
\AgdaBound{i}\AgdaSpace{}%
\AgdaBound{j}\AgdaSpace{}%
\AgdaSymbol{→}\AgdaSpace{}%
\AgdaBound{Q}\AgdaSpace{}%
\AgdaBound{i}\AgdaSpace{}%
\AgdaBound{j}\<%
\\
%
\\[\AgdaEmptyExtraSkip]%
\>[0]\AgdaKeyword{infixr}\AgdaSpace{}%
\AgdaNumber{4}\AgdaSpace{}%
\AgdaOperator{\AgdaFunction{\AgdaUnderscore{}⇒\AgdaUnderscore{}}}\<%
\end{code}
\ccpad
Here is a more general version that we borrow from the standard library and translate into MHE/UALib notation.
\ccpad
\begin{code}%
\>[0]\AgdaOperator{\AgdaFunction{\AgdaUnderscore{}=[\AgdaUnderscore{}]⇒\AgdaUnderscore{}}}\AgdaSpace{}%
\AgdaSymbol{:}\AgdaSpace{}%
\AgdaSymbol{\{}\AgdaBound{𝓤}\AgdaSpace{}%
\AgdaBound{𝓥}\AgdaSpace{}%
\AgdaBound{𝓡}\AgdaSpace{}%
\AgdaBound{𝓢}\AgdaSpace{}%
\AgdaSymbol{:}\AgdaSpace{}%
\AgdaPostulate{Universe}\AgdaSymbol{\}\{}\AgdaBound{A}\AgdaSpace{}%
\AgdaSymbol{:}\AgdaSpace{}%
\AgdaBound{𝓤}\AgdaSpace{}%
\AgdaOperator{\AgdaFunction{̇}}\AgdaSpace{}%
\AgdaSymbol{\}}\AgdaSpace{}%
\AgdaSymbol{\{}\AgdaBound{B}\AgdaSpace{}%
\AgdaSymbol{:}\AgdaSpace{}%
\AgdaBound{𝓥}\AgdaSpace{}%
\AgdaOperator{\AgdaFunction{̇}}\AgdaSpace{}%
\AgdaSymbol{\}}\<%
\\
\>[0][@{}l@{\AgdaIndent{0}}]%
\>[1]\AgdaSymbol{→}%
\>[10]\AgdaFunction{Rel}\AgdaSpace{}%
\AgdaBound{A}\AgdaSpace{}%
\AgdaBound{𝓡}\AgdaSpace{}%
\AgdaSymbol{→}\AgdaSpace{}%
\AgdaSymbol{(}\AgdaBound{A}\AgdaSpace{}%
\AgdaSymbol{→}\AgdaSpace{}%
\AgdaBound{B}\AgdaSymbol{)}\AgdaSpace{}%
\AgdaSymbol{→}\AgdaSpace{}%
\AgdaFunction{Rel}\AgdaSpace{}%
\AgdaBound{B}\AgdaSpace{}%
\AgdaBound{𝓢}\AgdaSpace{}%
\AgdaSymbol{→}\AgdaSpace{}%
\AgdaBound{𝓤}\AgdaSpace{}%
\AgdaOperator{\AgdaPrimitive{⊔}}\AgdaSpace{}%
\AgdaBound{𝓡}\AgdaSpace{}%
\AgdaOperator{\AgdaPrimitive{⊔}}\AgdaSpace{}%
\AgdaBound{𝓢}\AgdaSpace{}%
\AgdaOperator{\AgdaFunction{̇}}\<%
\\
%
\\[\AgdaEmptyExtraSkip]%
\>[0]\AgdaBound{P}\AgdaSpace{}%
\AgdaOperator{\AgdaFunction{=[}}\AgdaSpace{}%
\AgdaBound{g}\AgdaSpace{}%
\AgdaOperator{\AgdaFunction{]⇒}}\AgdaSpace{}%
\AgdaBound{Q}\AgdaSpace{}%
\AgdaSymbol{=}\AgdaSpace{}%
\AgdaBound{P}\AgdaSpace{}%
\AgdaOperator{\AgdaFunction{⇒}}\AgdaSpace{}%
\AgdaSymbol{(}\AgdaBound{Q}\AgdaSpace{}%
\AgdaOperator{\AgdaFunction{on}}\AgdaSpace{}%
\AgdaBound{g}\AgdaSymbol{)}\<%
\\
%
\\[\AgdaEmptyExtraSkip]%
\>[0]\AgdaKeyword{infixr}\AgdaSpace{}%
\AgdaNumber{4}\AgdaSpace{}%
\AgdaOperator{\AgdaFunction{\AgdaUnderscore{}=[\AgdaUnderscore{}]⇒\AgdaUnderscore{}}}\<%
\\
\>[0]\<%
\end{code}

----------
In set theory, a binary relation on a set \ab{A} is simply a subset of the product \ab A \af × \ab A. As such, we could model these as predicates over the type \ab A \af × \ab A, or as relations of type \ab A \as → \ab A \as → \ab 𝓡 \af ̇ (for some universe \ab 𝓡). A generalization of this notion is a binary relation is a \emph{relation from} \ab{A} \emph{to} \ab{B}, which we define first and treat binary relations on a single \ab{A} as a special case.
\ccpad
\begin{code}%
\>[1]\AgdaFunction{REL}\AgdaSpace{}%
\AgdaSymbol{:}\AgdaSpace{}%
\AgdaSymbol{\{}\AgdaBound{𝓡}\AgdaSpace{}%
\AgdaSymbol{:}\AgdaSpace{}%
\AgdaPostulate{Universe}\AgdaSymbol{\}}\AgdaSpace{}%
\AgdaSymbol{→}\AgdaSpace{}%
\AgdaBound{𝓤}\AgdaSpace{}%
\AgdaOperator{\AgdaFunction{̇}}\AgdaSpace{}%
\AgdaSymbol{→}\AgdaSpace{}%
\AgdaBound{𝓡}\AgdaSpace{}%
\AgdaOperator{\AgdaFunction{̇}}\AgdaSpace{}%
\AgdaSymbol{→}\AgdaSpace{}%
\AgdaSymbol{(}\AgdaBound{𝓝}\AgdaSpace{}%
\AgdaSymbol{:}\AgdaSpace{}%
\AgdaPostulate{Universe}\AgdaSymbol{)}\AgdaSpace{}%
\AgdaSymbol{→}\AgdaSpace{}%
\AgdaSymbol{(}\AgdaBound{𝓤}\AgdaSpace{}%
\AgdaOperator{\AgdaPrimitive{⊔}}\AgdaSpace{}%
\AgdaBound{𝓡}\AgdaSpace{}%
\AgdaOperator{\AgdaPrimitive{⊔}}\AgdaSpace{}%
\AgdaBound{𝓝}\AgdaSpace{}%
\AgdaOperator{\AgdaPrimitive{⁺}}\AgdaSymbol{)}\AgdaSpace{}%
\AgdaOperator{\AgdaFunction{̇}}\<%
\\
%
\>[1]\AgdaFunction{REL}\AgdaSpace{}%
\AgdaBound{A}\AgdaSpace{}%
\AgdaBound{B}\AgdaSpace{}%
\AgdaBound{𝓝}\AgdaSpace{}%
\AgdaSymbol{=}\AgdaSpace{}%
\AgdaBound{A}\AgdaSpace{}%
\AgdaSymbol{→}\AgdaSpace{}%
\AgdaBound{B}\AgdaSpace{}%
\AgdaSymbol{→}\AgdaSpace{}%
\AgdaBound{𝓝}\AgdaSpace{}%
\AgdaOperator{\AgdaFunction{̇}}\<%
\end{code}

\subsubsection{Kernels}\label{Binary.sssec:kernels}
The kernel of a function can be defined in many ways. For example,
\ccpad
\begin{code}%
\>[0][@{}l@{\AgdaIndent{1}}]%
\>[1]\AgdaFunction{KER}\AgdaSpace{}%
\AgdaSymbol{:}\AgdaSpace{}%
\AgdaSymbol{\{}\AgdaBound{𝓡}\AgdaSpace{}%
\AgdaSymbol{:}\AgdaSpace{}%
\AgdaPostulate{Universe}\AgdaSymbol{\}}\AgdaSpace{}%
\AgdaSymbol{\{}\AgdaBound{A}\AgdaSpace{}%
\AgdaSymbol{:}\AgdaSpace{}%
\AgdaBound{𝓤}\AgdaSpace{}%
\AgdaOperator{\AgdaFunction{̇}}\AgdaSpace{}%
\AgdaSymbol{\}}\AgdaSpace{}%
\AgdaSymbol{\{}\AgdaBound{B}\AgdaSpace{}%
\AgdaSymbol{:}\AgdaSpace{}%
\AgdaBound{𝓡}\AgdaSpace{}%
\AgdaOperator{\AgdaFunction{̇}}\AgdaSpace{}%
\AgdaSymbol{\}}\AgdaSpace{}%
\AgdaSymbol{→}\AgdaSpace{}%
\AgdaSymbol{(}\AgdaBound{A}\AgdaSpace{}%
\AgdaSymbol{→}\AgdaSpace{}%
\AgdaBound{B}\AgdaSymbol{)}\AgdaSpace{}%
\AgdaSymbol{→}\AgdaSpace{}%
\AgdaBound{𝓤}\AgdaSpace{}%
\AgdaOperator{\AgdaPrimitive{⊔}}\AgdaSpace{}%
\AgdaBound{𝓡}\AgdaSpace{}%
\AgdaOperator{\AgdaFunction{̇}}\<%
\\
%
\>[1]\AgdaFunction{KER}\AgdaSpace{}%
\AgdaSymbol{\{}\AgdaBound{𝓡}\AgdaSymbol{\}}\AgdaSpace{}%
\AgdaSymbol{\{}\AgdaBound{A}\AgdaSymbol{\}}\AgdaSpace{}%
\AgdaBound{g}\AgdaSpace{}%
\AgdaSymbol{=}\AgdaSpace{}%
\AgdaFunction{Σ}\AgdaSpace{}%
\AgdaBound{x}\AgdaSpace{}%
\AgdaFunction{꞉}\AgdaSpace{}%
\AgdaBound{A}\AgdaSpace{}%
\AgdaFunction{,}\AgdaSpace{}%
\AgdaFunction{Σ}\AgdaSpace{}%
\AgdaBound{y}\AgdaSpace{}%
\AgdaFunction{꞉}\AgdaSpace{}%
\AgdaBound{A}\AgdaSpace{}%
\AgdaFunction{,}\AgdaSpace{}%
\AgdaBound{g}\AgdaSpace{}%
\AgdaBound{x}\AgdaSpace{}%
\AgdaOperator{\AgdaDatatype{≡}}\AgdaSpace{}%
\AgdaBound{g}\AgdaSpace{}%
\AgdaBound{y}\<%
\end{code}
\ccpad
or as a unary relation (predicate) over the Cartesian product,
\ccpad
\begin{code}%
\>[0][@{}l@{\AgdaIndent{1}}]%
\>[1]\AgdaFunction{KER-pred}\AgdaSpace{}%
\AgdaSymbol{:}\AgdaSpace{}%
\AgdaSymbol{\{}\AgdaBound{𝓡}\AgdaSpace{}%
\AgdaSymbol{:}\AgdaSpace{}%
\AgdaPostulate{Universe}\AgdaSymbol{\}}\AgdaSpace{}%
\AgdaSymbol{\{}\AgdaBound{A}\AgdaSpace{}%
\AgdaSymbol{:}\AgdaSpace{}%
\AgdaBound{𝓤}\AgdaSpace{}%
\AgdaOperator{\AgdaFunction{̇}}\AgdaSymbol{\}\{}\AgdaBound{B}\AgdaSpace{}%
\AgdaSymbol{:}\AgdaSpace{}%
\AgdaBound{𝓡}\AgdaSpace{}%
\AgdaOperator{\AgdaFunction{̇}}\AgdaSymbol{\}}\AgdaSpace{}%
\AgdaSymbol{→}\AgdaSpace{}%
\AgdaSymbol{(}\AgdaBound{A}\AgdaSpace{}%
\AgdaSymbol{→}\AgdaSpace{}%
\AgdaBound{B}\AgdaSymbol{)}\AgdaSpace{}%
\AgdaSymbol{→}\AgdaSpace{}%
\AgdaFunction{Pred}\AgdaSpace{}%
\AgdaSymbol{(}\AgdaBound{A}\AgdaSpace{}%
\AgdaOperator{\AgdaFunction{×}}\AgdaSpace{}%
\AgdaBound{A}\AgdaSymbol{)}\AgdaSpace{}%
\AgdaBound{𝓡}\<%
\\
%
\>[1]\AgdaFunction{KER-pred}\AgdaSpace{}%
\AgdaBound{g}\AgdaSpace{}%
\AgdaSymbol{(}\AgdaBound{x}\AgdaSpace{}%
\AgdaOperator{\AgdaInductiveConstructor{,}}\AgdaSpace{}%
\AgdaBound{y}\AgdaSymbol{)}\AgdaSpace{}%
\AgdaSymbol{=}\AgdaSpace{}%
\AgdaBound{g}\AgdaSpace{}%
\AgdaBound{x}\AgdaSpace{}%
\AgdaOperator{\AgdaDatatype{≡}}\AgdaSpace{}%
\AgdaBound{g}\AgdaSpace{}%
\AgdaBound{y}\<%
\end{code}
\ccpad
or as a relation from \ab{A} to \ab{B},
\ccpad
\begin{code}%
\>[0][@{}l@{\AgdaIndent{1}}]%
\>[1]\AgdaFunction{Rel}\AgdaSpace{}%
\AgdaSymbol{:}\AgdaSpace{}%
\AgdaBound{𝓤}\AgdaSpace{}%
\AgdaOperator{\AgdaFunction{̇}}\AgdaSpace{}%
\AgdaSymbol{→}\AgdaSpace{}%
\AgdaSymbol{(}\AgdaBound{𝓝}\AgdaSpace{}%
\AgdaSymbol{:}\AgdaSpace{}%
\AgdaPostulate{Universe}\AgdaSymbol{)}\AgdaSpace{}%
\AgdaSymbol{→}\AgdaSpace{}%
\AgdaBound{𝓤}\AgdaSpace{}%
\AgdaOperator{\AgdaPrimitive{⊔}}\AgdaSpace{}%
\AgdaBound{𝓝}\AgdaSpace{}%
\AgdaOperator{\AgdaPrimitive{⁺}}\AgdaSpace{}%
\AgdaOperator{\AgdaFunction{̇}}\<%
\\
%
\>[1]\AgdaFunction{Rel}\AgdaSpace{}%
\AgdaBound{A}\AgdaSpace{}%
\AgdaBound{𝓝}\AgdaSpace{}%
\AgdaSymbol{=}\AgdaSpace{}%
\AgdaFunction{REL}\AgdaSpace{}%
\AgdaBound{A}\AgdaSpace{}%
\AgdaBound{A}\AgdaSpace{}%
\AgdaBound{𝓝}\<%
\\
%
\\[\AgdaEmptyExtraSkip]%
%
\>[1]\AgdaFunction{KER-rel}\AgdaSpace{}%
\AgdaSymbol{:}\AgdaSpace{}%
\AgdaSymbol{\{}\AgdaBound{𝓡}\AgdaSpace{}%
\AgdaSymbol{:}\AgdaSpace{}%
\AgdaPostulate{Universe}\AgdaSymbol{\}\{}\AgdaBound{A}\AgdaSpace{}%
\AgdaSymbol{:}\AgdaSpace{}%
\AgdaBound{𝓤}\AgdaSpace{}%
\AgdaOperator{\AgdaFunction{̇}}\AgdaSpace{}%
\AgdaSymbol{\}}\AgdaSpace{}%
\AgdaSymbol{\{}\AgdaBound{B}\AgdaSpace{}%
\AgdaSymbol{:}\AgdaSpace{}%
\AgdaBound{𝓡}\AgdaSpace{}%
\AgdaOperator{\AgdaFunction{̇}}\AgdaSpace{}%
\AgdaSymbol{\}}\AgdaSpace{}%
\AgdaSymbol{→}\AgdaSpace{}%
\AgdaSymbol{(}\AgdaBound{A}\AgdaSpace{}%
\AgdaSymbol{→}\AgdaSpace{}%
\AgdaBound{B}\AgdaSymbol{)}\AgdaSpace{}%
\AgdaSymbol{→}\AgdaSpace{}%
\AgdaFunction{Rel}\AgdaSpace{}%
\AgdaBound{A}\AgdaSpace{}%
\AgdaBound{𝓡}\<%
\\
%
\>[1]\AgdaFunction{KER-rel}\AgdaSpace{}%
\AgdaBound{g}\AgdaSpace{}%
\AgdaBound{x}\AgdaSpace{}%
\AgdaBound{y}\AgdaSpace{}%
\AgdaSymbol{=}\AgdaSpace{}%
\AgdaBound{g}\AgdaSpace{}%
\AgdaBound{x}\AgdaSpace{}%
\AgdaOperator{\AgdaDatatype{≡}}\AgdaSpace{}%
\AgdaBound{g}\AgdaSpace{}%
\AgdaBound{y}\<%
\end{code}



\subsubsection{Properties of binary relations}\label{Binary.sssec:properties-of-binary-relations}
The notions of reflexivity, symmetry, and transitivity are defined as one would hope and expect, so we present them here without further explanation.
\begin{code}%
\>[1]\AgdaFunction{reflexive}\AgdaSpace{}%
\AgdaSymbol{:}\AgdaSpace{}%
\AgdaSymbol{\{}\AgdaBound{𝓡}\AgdaSpace{}%
\AgdaSymbol{:}\AgdaSpace{}%
\AgdaPostulate{Universe}\AgdaSymbol{\}\{}\AgdaBound{X}\AgdaSpace{}%
\AgdaSymbol{:}\AgdaSpace{}%
\AgdaBound{𝓤}\AgdaSpace{}%
\AgdaOperator{\AgdaFunction{̇}}\AgdaSpace{}%
\AgdaSymbol{\}}\AgdaSpace{}%
\AgdaSymbol{→}\AgdaSpace{}%
\AgdaFunction{Rel}\AgdaSpace{}%
\AgdaBound{X}\AgdaSpace{}%
\AgdaBound{𝓡}\AgdaSpace{}%
\AgdaSymbol{→}\AgdaSpace{}%
\AgdaBound{𝓤}\AgdaSpace{}%
\AgdaOperator{\AgdaPrimitive{⊔}}\AgdaSpace{}%
\AgdaBound{𝓡}\AgdaSpace{}%
\AgdaOperator{\AgdaFunction{̇}}\<%
\\
%
\>[1]\AgdaFunction{reflexive}\AgdaSpace{}%
\AgdaOperator{\AgdaBound{\AgdaUnderscore{}≈\AgdaUnderscore{}}}\AgdaSpace{}%
\AgdaSymbol{=}\AgdaSpace{}%
\AgdaSymbol{∀}\AgdaSpace{}%
\AgdaBound{x}\AgdaSpace{}%
\AgdaSymbol{→}\AgdaSpace{}%
\AgdaBound{x}\AgdaSpace{}%
\AgdaOperator{\AgdaBound{≈}}\AgdaSpace{}%
\AgdaBound{x}\<%
\\
%
\\[\AgdaEmptyExtraSkip]%
%
\>[1]\AgdaFunction{symmetric}\AgdaSpace{}%
\AgdaSymbol{:}\AgdaSpace{}%
\AgdaSymbol{\{}\AgdaBound{𝓡}\AgdaSpace{}%
\AgdaSymbol{:}\AgdaSpace{}%
\AgdaPostulate{Universe}\AgdaSymbol{\}\{}\AgdaBound{X}\AgdaSpace{}%
\AgdaSymbol{:}\AgdaSpace{}%
\AgdaBound{𝓤}\AgdaSpace{}%
\AgdaOperator{\AgdaFunction{̇}}\AgdaSpace{}%
\AgdaSymbol{\}}\AgdaSpace{}%
\AgdaSymbol{→}\AgdaSpace{}%
\AgdaFunction{Rel}\AgdaSpace{}%
\AgdaBound{X}\AgdaSpace{}%
\AgdaBound{𝓡}\AgdaSpace{}%
\AgdaSymbol{→}\AgdaSpace{}%
\AgdaBound{𝓤}\AgdaSpace{}%
\AgdaOperator{\AgdaPrimitive{⊔}}\AgdaSpace{}%
\AgdaBound{𝓡}\AgdaSpace{}%
\AgdaOperator{\AgdaFunction{̇}}\<%
\\
%
\>[1]\AgdaFunction{symmetric}\AgdaSpace{}%
\AgdaOperator{\AgdaBound{\AgdaUnderscore{}≈\AgdaUnderscore{}}}\AgdaSpace{}%
\AgdaSymbol{=}\AgdaSpace{}%
\AgdaSymbol{∀}\AgdaSpace{}%
\AgdaBound{x}\AgdaSpace{}%
\AgdaBound{y}\AgdaSpace{}%
\AgdaSymbol{→}\AgdaSpace{}%
\AgdaBound{x}\AgdaSpace{}%
\AgdaOperator{\AgdaBound{≈}}\AgdaSpace{}%
\AgdaBound{y}\AgdaSpace{}%
\AgdaSymbol{→}\AgdaSpace{}%
\AgdaBound{y}\AgdaSpace{}%
\AgdaOperator{\AgdaBound{≈}}\AgdaSpace{}%
\AgdaBound{x}\<%
\\
%
\\[\AgdaEmptyExtraSkip]%
%
\>[1]\AgdaFunction{transitive}\AgdaSpace{}%
\AgdaSymbol{:}\AgdaSpace{}%
\AgdaSymbol{\{}\AgdaBound{𝓡}\AgdaSpace{}%
\AgdaSymbol{:}\AgdaSpace{}%
\AgdaPostulate{Universe}\AgdaSymbol{\}\{}\AgdaBound{X}\AgdaSpace{}%
\AgdaSymbol{:}\AgdaSpace{}%
\AgdaBound{𝓤}\AgdaSpace{}%
\AgdaOperator{\AgdaFunction{̇}}\AgdaSpace{}%
\AgdaSymbol{\}}\AgdaSpace{}%
\AgdaSymbol{→}\AgdaSpace{}%
\AgdaFunction{Rel}\AgdaSpace{}%
\AgdaBound{X}\AgdaSpace{}%
\AgdaBound{𝓡}\AgdaSpace{}%
\AgdaSymbol{→}\AgdaSpace{}%
\AgdaBound{𝓤}\AgdaSpace{}%
\AgdaOperator{\AgdaPrimitive{⊔}}\AgdaSpace{}%
\AgdaBound{𝓡}\AgdaSpace{}%
\AgdaOperator{\AgdaFunction{̇}}\<%
\\
%
\>[1]\AgdaFunction{transitive}\AgdaSpace{}%
\AgdaOperator{\AgdaBound{\AgdaUnderscore{}≈\AgdaUnderscore{}}}\AgdaSpace{}%
\AgdaSymbol{=}\AgdaSpace{}%
\AgdaSymbol{∀}\AgdaSpace{}%
\AgdaBound{x}\AgdaSpace{}%
\AgdaBound{y}\AgdaSpace{}%
\AgdaBound{z}\AgdaSpace{}%
\AgdaSymbol{→}\AgdaSpace{}%
\AgdaBound{x}\AgdaSpace{}%
\AgdaOperator{\AgdaBound{≈}}\AgdaSpace{}%
\AgdaBound{y}\AgdaSpace{}%
\AgdaSymbol{→}\AgdaSpace{}%
\AgdaBound{y}\AgdaSpace{}%
\AgdaOperator{\AgdaBound{≈}}\AgdaSpace{}%
\AgdaBound{z}\AgdaSpace{}%
\AgdaSymbol{→}\AgdaSpace{}%
\AgdaBound{x}\AgdaSpace{}%
\AgdaOperator{\AgdaBound{≈}}\AgdaSpace{}%
\AgdaBound{z}\<%
\end{code}

\subsubsection{Binary relation truncation}\label{sssec:binary-relation-truncation}
Recall, in Section~\ref{sssec:truncation} we described the concept of truncation as it relates to ``proof-relevant'' mathematics.  Given a binary relation \af P, it may be necessary or desirable to assume that there is at most one way to prove that a given pair of elements is \af P-related\footnote{This is another example of ``proof-irrelevance''; indeed, proofs of \af P \ab x \ab y are indistinguishable, or rather any distinctions are irrelevant in the context of interest.} We use Escardo's \af{is-subsingleton} type to express this strong (truncation at level 1) assumption in the following definition: We say that (\ab x  \ac \ab y) belongs to \af P, or that \ab x and \ab y are \af P-related if and only if both \af P \ab x \ab y \emph{and} \af{is-subsingleton} (\af P \ab x \ab y) holds.
\ccpad
\begin{code}%
\>[1]\AgdaFunction{Rel₀}\AgdaSpace{}%
\AgdaSymbol{:}\AgdaSpace{}%
\AgdaBound{𝓤}\AgdaSpace{}%
\AgdaOperator{\AgdaFunction{̇}}\AgdaSpace{}%
\AgdaSymbol{→}\AgdaSpace{}%
\AgdaSymbol{(}\AgdaBound{𝓝}\AgdaSpace{}%
\AgdaSymbol{:}\AgdaSpace{}%
\AgdaPostulate{Universe}\AgdaSymbol{)}\AgdaSpace{}%
\AgdaSymbol{→}\AgdaSpace{}%
\AgdaBound{𝓤}\AgdaSpace{}%
\AgdaOperator{\AgdaPrimitive{⊔}}\AgdaSpace{}%
\AgdaBound{𝓝}\AgdaSpace{}%
\AgdaOperator{\AgdaPrimitive{⁺}}\AgdaSpace{}%
\AgdaOperator{\AgdaFunction{̇}}\<%
\\
%
\>[1]\AgdaFunction{Rel₀}\AgdaSpace{}%
\AgdaBound{A}\AgdaSpace{}%
\AgdaBound{𝓝}\AgdaSpace{}%
\AgdaSymbol{=}\AgdaSpace{}%
\AgdaFunction{Σ}\AgdaSpace{}%
\AgdaBound{P}\AgdaSpace{}%
\AgdaFunction{꞉}\AgdaSpace{}%
\AgdaSymbol{(}\AgdaBound{A}\AgdaSpace{}%
\AgdaSymbol{→}\AgdaSpace{}%
\AgdaBound{A}\AgdaSpace{}%
\AgdaSymbol{→}\AgdaSpace{}%
\AgdaBound{𝓝}\AgdaSpace{}%
\AgdaOperator{\AgdaFunction{̇}}\AgdaSymbol{)}\AgdaSpace{}%
\AgdaFunction{,}\AgdaSpace{}%
\AgdaSymbol{∀}\AgdaSpace{}%
\AgdaBound{x}\AgdaSpace{}%
\AgdaBound{y}\AgdaSpace{}%
\AgdaSymbol{→}\AgdaSpace{}%
\AgdaFunction{is-subsingleton}\AgdaSpace{}%
\AgdaSymbol{(}\AgdaBound{P}\AgdaSpace{}%
\AgdaBound{x}\AgdaSpace{}%
\AgdaBound{y}\AgdaSymbol{)}\<%
\end{code}
\ccpad
Thus, a set \textit{set}, as defined in \S\ref{sssec:truncation-and-sets}, is a type \ab{X} with an equality relation \af ≡ : \af{Rel₀} \ab X \ab 𝓝, for some \ab 𝓝.
%%-- END: In set theory, a binary relation on a set \ab{A} is simply a subset of the product \ab A \af × \ab A. As such, we could model these as predicates over the type \ab A \af × \ab A, or as relations of type \ab A \as → \ab A \as → \ab 𝓡 \af ̇ (for some universe \ab 𝓡). We define these below.

A generalization of the notion of binary relation is a \emph{relation from} \ab{A} \emph{to} \ab{B}, which we define first and treat binary relations on a single \ab{A} as a special case.
\ccpad
\begin{code}%
\>[0]\AgdaKeyword{module}\AgdaSpace{}%
\AgdaModule{UALib.Relations.Binary}\AgdaSpace{}%
\AgdaKeyword{where}\<%
\\
%
\\[\AgdaEmptyExtraSkip]%
\>[0]\AgdaKeyword{open}\AgdaSpace{}%
\AgdaKeyword{import}\AgdaSpace{}%
\AgdaModule{UALib.Relations.Unary}\AgdaSpace{}%
\AgdaKeyword{public}\<%
\\
%
\\[\AgdaEmptyExtraSkip]%
\>[0]\AgdaKeyword{module}\AgdaSpace{}%
\AgdaModule{\AgdaUnderscore{}}\AgdaSpace{}%
\AgdaSymbol{\{}\AgdaBound{𝓤}\AgdaSpace{}%
\AgdaSymbol{:}\AgdaSpace{}%
\AgdaPostulate{Universe}\AgdaSymbol{\}}\AgdaSpace{}%
\AgdaKeyword{where}\<%
\\
%
\\[\AgdaEmptyExtraSkip]%
\>[0][@{}l@{\AgdaIndent{0}}]%
\>[1]\AgdaFunction{REL}\AgdaSpace{}%
\AgdaSymbol{:}\AgdaSpace{}%
\AgdaSymbol{\{}\AgdaBound{𝓡}\AgdaSpace{}%
\AgdaSymbol{:}\AgdaSpace{}%
\AgdaPostulate{Universe}\AgdaSymbol{\}}\AgdaSpace{}%
\AgdaSymbol{→}\AgdaSpace{}%
\AgdaBound{𝓤}\AgdaSpace{}%
\AgdaOperator{\AgdaFunction{̇}}\AgdaSpace{}%
\AgdaSymbol{→}\AgdaSpace{}%
\AgdaBound{𝓡}\AgdaSpace{}%
\AgdaOperator{\AgdaFunction{̇}}\AgdaSpace{}%
\AgdaSymbol{→}\AgdaSpace{}%
\AgdaSymbol{(}\AgdaBound{𝓝}\AgdaSpace{}%
\AgdaSymbol{:}\AgdaSpace{}%
\AgdaPostulate{Universe}\AgdaSymbol{)}\AgdaSpace{}%
\AgdaSymbol{→}\AgdaSpace{}%
\AgdaSymbol{(}\AgdaBound{𝓤}\AgdaSpace{}%
\AgdaOperator{\AgdaPrimitive{⊔}}\AgdaSpace{}%
\AgdaBound{𝓡}\AgdaSpace{}%
\AgdaOperator{\AgdaPrimitive{⊔}}\AgdaSpace{}%
\AgdaBound{𝓝}\AgdaSpace{}%
\AgdaOperator{\AgdaPrimitive{⁺}}\AgdaSymbol{)}\AgdaSpace{}%
\AgdaOperator{\AgdaFunction{̇}}\<%
\\
%
\>[1]\AgdaFunction{REL}\AgdaSpace{}%
\AgdaBound{A}\AgdaSpace{}%
\AgdaBound{B}\AgdaSpace{}%
\AgdaBound{𝓝}\AgdaSpace{}%
\AgdaSymbol{=}\AgdaSpace{}%
\AgdaBound{A}\AgdaSpace{}%
\AgdaSymbol{→}\AgdaSpace{}%
\AgdaBound{B}\AgdaSpace{}%
\AgdaSymbol{→}\AgdaSpace{}%
\AgdaBound{𝓝}\AgdaSpace{}%
\AgdaOperator{\AgdaFunction{̇}}\<%
\end{code}

\subsubsection{Kernels}\label{Binary.sssec:kernels}
The kernel of a function can be defined in many ways. For example,
\ccpad
\begin{code}%
\>[0][@{}l@{\AgdaIndent{1}}]%
\>[1]\AgdaFunction{KER}\AgdaSpace{}%
\AgdaSymbol{:}\AgdaSpace{}%
\AgdaSymbol{\{}\AgdaBound{𝓡}\AgdaSpace{}%
\AgdaSymbol{:}\AgdaSpace{}%
\AgdaPostulate{Universe}\AgdaSymbol{\}}\AgdaSpace{}%
\AgdaSymbol{\{}\AgdaBound{A}\AgdaSpace{}%
\AgdaSymbol{:}\AgdaSpace{}%
\AgdaBound{𝓤}\AgdaSpace{}%
\AgdaOperator{\AgdaFunction{̇}}\AgdaSpace{}%
\AgdaSymbol{\}}\AgdaSpace{}%
\AgdaSymbol{\{}\AgdaBound{B}\AgdaSpace{}%
\AgdaSymbol{:}\AgdaSpace{}%
\AgdaBound{𝓡}\AgdaSpace{}%
\AgdaOperator{\AgdaFunction{̇}}\AgdaSpace{}%
\AgdaSymbol{\}}\AgdaSpace{}%
\AgdaSymbol{→}\AgdaSpace{}%
\AgdaSymbol{(}\AgdaBound{A}\AgdaSpace{}%
\AgdaSymbol{→}\AgdaSpace{}%
\AgdaBound{B}\AgdaSymbol{)}\AgdaSpace{}%
\AgdaSymbol{→}\AgdaSpace{}%
\AgdaBound{𝓤}\AgdaSpace{}%
\AgdaOperator{\AgdaPrimitive{⊔}}\AgdaSpace{}%
\AgdaBound{𝓡}\AgdaSpace{}%
\AgdaOperator{\AgdaFunction{̇}}\<%
\\
%
\>[1]\AgdaFunction{KER}\AgdaSpace{}%
\AgdaSymbol{\{}\AgdaBound{𝓡}\AgdaSymbol{\}}\AgdaSpace{}%
\AgdaSymbol{\{}\AgdaBound{A}\AgdaSymbol{\}}\AgdaSpace{}%
\AgdaBound{g}\AgdaSpace{}%
\AgdaSymbol{=}\AgdaSpace{}%
\AgdaFunction{Σ}\AgdaSpace{}%
\AgdaBound{x}\AgdaSpace{}%
\AgdaFunction{꞉}\AgdaSpace{}%
\AgdaBound{A}\AgdaSpace{}%
\AgdaFunction{,}\AgdaSpace{}%
\AgdaFunction{Σ}\AgdaSpace{}%
\AgdaBound{y}\AgdaSpace{}%
\AgdaFunction{꞉}\AgdaSpace{}%
\AgdaBound{A}\AgdaSpace{}%
\AgdaFunction{,}\AgdaSpace{}%
\AgdaBound{g}\AgdaSpace{}%
\AgdaBound{x}\AgdaSpace{}%
\AgdaOperator{\AgdaDatatype{≡}}\AgdaSpace{}%
\AgdaBound{g}\AgdaSpace{}%
\AgdaBound{y}\<%
\end{code}
\ccpad
or as a unary relation (predicate) over the Cartesian product,
\ccpad
\begin{code}%
\>[0][@{}l@{\AgdaIndent{1}}]%
\>[1]\AgdaFunction{KER-pred}\AgdaSpace{}%
\AgdaSymbol{:}\AgdaSpace{}%
\AgdaSymbol{\{}\AgdaBound{𝓡}\AgdaSpace{}%
\AgdaSymbol{:}\AgdaSpace{}%
\AgdaPostulate{Universe}\AgdaSymbol{\}}\AgdaSpace{}%
\AgdaSymbol{\{}\AgdaBound{A}\AgdaSpace{}%
\AgdaSymbol{:}\AgdaSpace{}%
\AgdaBound{𝓤}\AgdaSpace{}%
\AgdaOperator{\AgdaFunction{̇}}\AgdaSymbol{\}\{}\AgdaBound{B}\AgdaSpace{}%
\AgdaSymbol{:}\AgdaSpace{}%
\AgdaBound{𝓡}\AgdaSpace{}%
\AgdaOperator{\AgdaFunction{̇}}\AgdaSymbol{\}}\AgdaSpace{}%
\AgdaSymbol{→}\AgdaSpace{}%
\AgdaSymbol{(}\AgdaBound{A}\AgdaSpace{}%
\AgdaSymbol{→}\AgdaSpace{}%
\AgdaBound{B}\AgdaSymbol{)}\AgdaSpace{}%
\AgdaSymbol{→}\AgdaSpace{}%
\AgdaFunction{Pred}\AgdaSpace{}%
\AgdaSymbol{(}\AgdaBound{A}\AgdaSpace{}%
\AgdaOperator{\AgdaFunction{×}}\AgdaSpace{}%
\AgdaBound{A}\AgdaSymbol{)}\AgdaSpace{}%
\AgdaBound{𝓡}\<%
\\
%
\>[1]\AgdaFunction{KER-pred}\AgdaSpace{}%
\AgdaBound{g}\AgdaSpace{}%
\AgdaSymbol{(}\AgdaBound{x}\AgdaSpace{}%
\AgdaOperator{\AgdaInductiveConstructor{,}}\AgdaSpace{}%
\AgdaBound{y}\AgdaSymbol{)}\AgdaSpace{}%
\AgdaSymbol{=}\AgdaSpace{}%
\AgdaBound{g}\AgdaSpace{}%
\AgdaBound{x}\AgdaSpace{}%
\AgdaOperator{\AgdaDatatype{≡}}\AgdaSpace{}%
\AgdaBound{g}\AgdaSpace{}%
\AgdaBound{y}\<%
\end{code}
\ccpad
or as a relation from \ab{A} to \ab{B},
\ccpad
\begin{code}%
\>[0][@{}l@{\AgdaIndent{1}}]%
\>[1]\AgdaFunction{Rel}\AgdaSpace{}%
\AgdaSymbol{:}\AgdaSpace{}%
\AgdaBound{𝓤}\AgdaSpace{}%
\AgdaOperator{\AgdaFunction{̇}}\AgdaSpace{}%
\AgdaSymbol{→}\AgdaSpace{}%
\AgdaSymbol{(}\AgdaBound{𝓝}\AgdaSpace{}%
\AgdaSymbol{:}\AgdaSpace{}%
\AgdaPostulate{Universe}\AgdaSymbol{)}\AgdaSpace{}%
\AgdaSymbol{→}\AgdaSpace{}%
\AgdaBound{𝓤}\AgdaSpace{}%
\AgdaOperator{\AgdaPrimitive{⊔}}\AgdaSpace{}%
\AgdaBound{𝓝}\AgdaSpace{}%
\AgdaOperator{\AgdaPrimitive{⁺}}\AgdaSpace{}%
\AgdaOperator{\AgdaFunction{̇}}\<%
\\
%
\>[1]\AgdaFunction{Rel}\AgdaSpace{}%
\AgdaBound{A}\AgdaSpace{}%
\AgdaBound{𝓝}\AgdaSpace{}%
\AgdaSymbol{=}\AgdaSpace{}%
\AgdaFunction{REL}\AgdaSpace{}%
\AgdaBound{A}\AgdaSpace{}%
\AgdaBound{A}\AgdaSpace{}%
\AgdaBound{𝓝}\<%
\\
%
\\[\AgdaEmptyExtraSkip]%
%
\>[1]\AgdaFunction{KER-rel}\AgdaSpace{}%
\AgdaSymbol{:}\AgdaSpace{}%
\AgdaSymbol{\{}\AgdaBound{𝓡}\AgdaSpace{}%
\AgdaSymbol{:}\AgdaSpace{}%
\AgdaPostulate{Universe}\AgdaSymbol{\}\{}\AgdaBound{A}\AgdaSpace{}%
\AgdaSymbol{:}\AgdaSpace{}%
\AgdaBound{𝓤}\AgdaSpace{}%
\AgdaOperator{\AgdaFunction{̇}}\AgdaSpace{}%
\AgdaSymbol{\}}\AgdaSpace{}%
\AgdaSymbol{\{}\AgdaBound{B}\AgdaSpace{}%
\AgdaSymbol{:}\AgdaSpace{}%
\AgdaBound{𝓡}\AgdaSpace{}%
\AgdaOperator{\AgdaFunction{̇}}\AgdaSpace{}%
\AgdaSymbol{\}}\AgdaSpace{}%
\AgdaSymbol{→}\AgdaSpace{}%
\AgdaSymbol{(}\AgdaBound{A}\AgdaSpace{}%
\AgdaSymbol{→}\AgdaSpace{}%
\AgdaBound{B}\AgdaSymbol{)}\AgdaSpace{}%
\AgdaSymbol{→}\AgdaSpace{}%
\AgdaFunction{Rel}\AgdaSpace{}%
\AgdaBound{A}\AgdaSpace{}%
\AgdaBound{𝓡}\<%
\\
%
\>[1]\AgdaFunction{KER-rel}\AgdaSpace{}%
\AgdaBound{g}\AgdaSpace{}%
\AgdaBound{x}\AgdaSpace{}%
\AgdaBound{y}\AgdaSpace{}%
\AgdaSymbol{=}\AgdaSpace{}%
\AgdaBound{g}\AgdaSpace{}%
\AgdaBound{x}\AgdaSpace{}%
\AgdaOperator{\AgdaDatatype{≡}}\AgdaSpace{}%
\AgdaBound{g}\AgdaSpace{}%
\AgdaBound{y}\<%
\end{code}

\subsubsection{Examples}\label{Binary.sssec:examples}

\begin{code}%
\>[0][@{}l@{\AgdaIndent{1}}]%
\>[1]\AgdaFunction{ker}\AgdaSpace{}%
\AgdaSymbol{:}\AgdaSpace{}%
\AgdaSymbol{\{}\AgdaBound{A}\AgdaSpace{}%
\AgdaBound{B}\AgdaSpace{}%
\AgdaSymbol{:}\AgdaSpace{}%
\AgdaBound{𝓤}\AgdaSpace{}%
\AgdaOperator{\AgdaFunction{̇}}\AgdaSpace{}%
\AgdaSymbol{\}}\AgdaSpace{}%
\AgdaSymbol{→}\AgdaSpace{}%
\AgdaSymbol{(}\AgdaBound{A}\AgdaSpace{}%
\AgdaSymbol{→}\AgdaSpace{}%
\AgdaBound{B}\AgdaSymbol{)}\AgdaSpace{}%
\AgdaSymbol{→}\AgdaSpace{}%
\AgdaBound{𝓤}\AgdaSpace{}%
\AgdaOperator{\AgdaFunction{̇}}\<%
\\
%
\>[1]\AgdaFunction{ker}\AgdaSpace{}%
\AgdaSymbol{=}\AgdaSpace{}%
\AgdaFunction{KER}\AgdaSymbol{\{}\AgdaBound{𝓤}\AgdaSymbol{\}}\<%
\\
%
\\[\AgdaEmptyExtraSkip]%
%
\>[1]\AgdaFunction{ker-rel}\AgdaSpace{}%
\AgdaSymbol{:}\AgdaSpace{}%
\AgdaSymbol{\{}\AgdaBound{A}\AgdaSpace{}%
\AgdaBound{B}\AgdaSpace{}%
\AgdaSymbol{:}\AgdaSpace{}%
\AgdaBound{𝓤}\AgdaSpace{}%
\AgdaOperator{\AgdaFunction{̇}}\AgdaSpace{}%
\AgdaSymbol{\}}\AgdaSpace{}%
\AgdaSymbol{→}\AgdaSpace{}%
\AgdaSymbol{(}\AgdaBound{A}\AgdaSpace{}%
\AgdaSymbol{→}\AgdaSpace{}%
\AgdaBound{B}\AgdaSymbol{)}\AgdaSpace{}%
\AgdaSymbol{→}\AgdaSpace{}%
\AgdaFunction{Rel}\AgdaSpace{}%
\AgdaBound{A}\AgdaSpace{}%
\AgdaBound{𝓤}\<%
\\
%
\>[1]\AgdaFunction{ker-rel}\AgdaSpace{}%
\AgdaSymbol{=}\AgdaSpace{}%
\AgdaFunction{KER-rel}\AgdaSpace{}%
\AgdaSymbol{\{}\AgdaBound{𝓤}\AgdaSymbol{\}}\<%
\\
%
\\[\AgdaEmptyExtraSkip]%
%
\>[1]\AgdaFunction{ker-pred}\AgdaSpace{}%
\AgdaSymbol{:}\AgdaSpace{}%
\AgdaSymbol{\{}\AgdaBound{A}\AgdaSpace{}%
\AgdaBound{B}\AgdaSpace{}%
\AgdaSymbol{:}\AgdaSpace{}%
\AgdaBound{𝓤}\AgdaSpace{}%
\AgdaOperator{\AgdaFunction{̇}}\AgdaSpace{}%
\AgdaSymbol{\}}\AgdaSpace{}%
\AgdaSymbol{→}\AgdaSpace{}%
\AgdaSymbol{(}\AgdaBound{A}\AgdaSpace{}%
\AgdaSymbol{→}\AgdaSpace{}%
\AgdaBound{B}\AgdaSymbol{)}\AgdaSpace{}%
\AgdaSymbol{→}\AgdaSpace{}%
\AgdaFunction{Pred}\AgdaSpace{}%
\AgdaSymbol{(}\AgdaBound{A}\AgdaSpace{}%
\AgdaOperator{\AgdaFunction{×}}\AgdaSpace{}%
\AgdaBound{A}\AgdaSymbol{)}\AgdaSpace{}%
\AgdaBound{𝓤}\<%
\\
%
\>[1]\AgdaFunction{ker-pred}\AgdaSpace{}%
\AgdaSymbol{=}\AgdaSpace{}%
\AgdaFunction{KER-pred}\AgdaSpace{}%
\AgdaSymbol{\{}\AgdaBound{𝓤}\AgdaSymbol{\}}\<%
\\
%
\\[\AgdaEmptyExtraSkip]%
%
\>[1]\AgdaComment{--The identity relation.}\<%
\\
%
\>[1]\AgdaFunction{𝟎}\AgdaSpace{}%
\AgdaSymbol{:}\AgdaSpace{}%
\AgdaSymbol{\{}\AgdaBound{A}\AgdaSpace{}%
\AgdaSymbol{:}\AgdaSpace{}%
\AgdaBound{𝓤}\AgdaSpace{}%
\AgdaOperator{\AgdaFunction{̇}}\AgdaSpace{}%
\AgdaSymbol{\}}\AgdaSpace{}%
\AgdaSymbol{→}\AgdaSpace{}%
\AgdaBound{𝓤}\AgdaSpace{}%
\AgdaOperator{\AgdaFunction{̇}}\<%
\\
%
\>[1]\AgdaFunction{𝟎}\AgdaSpace{}%
\AgdaSymbol{\{}\AgdaBound{A}\AgdaSymbol{\}}\AgdaSpace{}%
\AgdaSymbol{=}\AgdaSpace{}%
\AgdaFunction{Σ}\AgdaSpace{}%
\AgdaBound{a}\AgdaSpace{}%
\AgdaFunction{꞉}\AgdaSpace{}%
\AgdaBound{A}\AgdaSpace{}%
\AgdaFunction{,}\AgdaSpace{}%
\AgdaFunction{Σ}\AgdaSpace{}%
\AgdaBound{b}\AgdaSpace{}%
\AgdaFunction{꞉}\AgdaSpace{}%
\AgdaBound{A}\AgdaSpace{}%
\AgdaFunction{,}\AgdaSpace{}%
\AgdaBound{a}\AgdaSpace{}%
\AgdaOperator{\AgdaDatatype{≡}}\AgdaSpace{}%
\AgdaBound{b}\<%
\\
%
\\[\AgdaEmptyExtraSkip]%
%
\>[1]\AgdaComment{--...as a binary relation...}\<%
\\
%
\>[1]\AgdaFunction{𝟎-rel}\AgdaSpace{}%
\AgdaSymbol{:}\AgdaSpace{}%
\AgdaSymbol{\{}\AgdaBound{A}\AgdaSpace{}%
\AgdaSymbol{:}\AgdaSpace{}%
\AgdaBound{𝓤}\AgdaSpace{}%
\AgdaOperator{\AgdaFunction{̇}}\AgdaSpace{}%
\AgdaSymbol{\}}\AgdaSpace{}%
\AgdaSymbol{→}\AgdaSpace{}%
\AgdaFunction{Rel}\AgdaSpace{}%
\AgdaBound{A}\AgdaSpace{}%
\AgdaBound{𝓤}\<%
\\
%
\>[1]\AgdaFunction{𝟎-rel}\AgdaSpace{}%
\AgdaBound{a}\AgdaSpace{}%
\AgdaBound{b}\AgdaSpace{}%
\AgdaSymbol{=}\AgdaSpace{}%
\AgdaBound{a}\AgdaSpace{}%
\AgdaOperator{\AgdaDatatype{≡}}\AgdaSpace{}%
\AgdaBound{b}\<%
\\
%
\\[\AgdaEmptyExtraSkip]%
%
\>[1]\AgdaComment{--...as a binary predicate...}\<%
\\
%
\>[1]\AgdaFunction{𝟎-pred}\AgdaSpace{}%
\AgdaSymbol{:}\AgdaSpace{}%
\AgdaSymbol{\{}\AgdaBound{A}\AgdaSpace{}%
\AgdaSymbol{:}\AgdaSpace{}%
\AgdaBound{𝓤}\AgdaSpace{}%
\AgdaOperator{\AgdaFunction{̇}}\AgdaSpace{}%
\AgdaSymbol{\}}\AgdaSpace{}%
\AgdaSymbol{→}\AgdaSpace{}%
\AgdaFunction{Pred}\AgdaSpace{}%
\AgdaSymbol{(}\AgdaBound{A}\AgdaSpace{}%
\AgdaOperator{\AgdaFunction{×}}\AgdaSpace{}%
\AgdaBound{A}\AgdaSymbol{)}\AgdaSpace{}%
\AgdaBound{𝓤}\<%
\\
%
\>[1]\AgdaFunction{𝟎-pred}\AgdaSpace{}%
\AgdaSymbol{(}\AgdaBound{a}\AgdaSpace{}%
\AgdaOperator{\AgdaInductiveConstructor{,}}\AgdaSpace{}%
\AgdaBound{a'}\AgdaSymbol{)}\AgdaSpace{}%
\AgdaSymbol{=}\AgdaSpace{}%
\AgdaBound{a}\AgdaSpace{}%
\AgdaOperator{\AgdaDatatype{≡}}\AgdaSpace{}%
\AgdaBound{a'}\<%
\\
%
\\[\AgdaEmptyExtraSkip]%
%
\>[1]\AgdaFunction{𝟎-pred'}\AgdaSpace{}%
\AgdaSymbol{:}\AgdaSpace{}%
\AgdaSymbol{\{}\AgdaBound{A}\AgdaSpace{}%
\AgdaSymbol{:}\AgdaSpace{}%
\AgdaBound{𝓤}\AgdaSpace{}%
\AgdaOperator{\AgdaFunction{̇}}\AgdaSpace{}%
\AgdaSymbol{\}}\AgdaSpace{}%
\AgdaSymbol{→}\AgdaSpace{}%
\AgdaBound{𝓤}\AgdaSpace{}%
\AgdaOperator{\AgdaFunction{̇}}\<%
\\
%
\>[1]\AgdaFunction{𝟎-pred'}\AgdaSpace{}%
\AgdaSymbol{\{}\AgdaBound{A}\AgdaSymbol{\}}\AgdaSpace{}%
\AgdaSymbol{=}\AgdaSpace{}%
\AgdaFunction{Σ}\AgdaSpace{}%
\AgdaBound{p}\AgdaSpace{}%
\AgdaFunction{꞉}\AgdaSpace{}%
\AgdaSymbol{(}\AgdaBound{A}\AgdaSpace{}%
\AgdaOperator{\AgdaFunction{×}}\AgdaSpace{}%
\AgdaBound{A}\AgdaSymbol{)}\AgdaSpace{}%
\AgdaFunction{,}\AgdaSpace{}%
\AgdaOperator{\AgdaFunction{∣}}\AgdaSpace{}%
\AgdaBound{p}\AgdaSpace{}%
\AgdaOperator{\AgdaFunction{∣}}\AgdaSpace{}%
\AgdaOperator{\AgdaDatatype{≡}}\AgdaSpace{}%
\AgdaOperator{\AgdaFunction{∥}}\AgdaSpace{}%
\AgdaBound{p}\AgdaSpace{}%
\AgdaOperator{\AgdaFunction{∥}}\<%
\\
%
\\[\AgdaEmptyExtraSkip]%
%
\>[1]\AgdaComment{--...on the domain of an algebra...}\<%
\\
%
\\[\AgdaEmptyExtraSkip]%
%
\>[1]\AgdaFunction{𝟎-alg-rel}\AgdaSpace{}%
\AgdaSymbol{:}\AgdaSpace{}%
\AgdaSymbol{\{}\AgdaBound{𝑆}\AgdaSpace{}%
\AgdaSymbol{:}\AgdaSpace{}%
\AgdaFunction{Signature}\AgdaSpace{}%
\AgdaGeneralizable{𝓞}\AgdaSpace{}%
\AgdaGeneralizable{𝓥}\AgdaSymbol{\}\{}\AgdaBound{𝑨}\AgdaSpace{}%
\AgdaSymbol{:}\AgdaSpace{}%
\AgdaFunction{Algebra}\AgdaSpace{}%
\AgdaBound{𝓤}\AgdaSpace{}%
\AgdaBound{𝑆}\AgdaSymbol{\}}\AgdaSpace{}%
\AgdaSymbol{→}\AgdaSpace{}%
\AgdaBound{𝓤}\AgdaSpace{}%
\AgdaOperator{\AgdaFunction{̇}}\<%
\\
%
\>[1]\AgdaFunction{𝟎-alg-rel}\AgdaSpace{}%
\AgdaSymbol{\{}\AgdaArgument{𝑨}\AgdaSpace{}%
\AgdaSymbol{=}\AgdaSpace{}%
\AgdaBound{𝑨}\AgdaSymbol{\}}\AgdaSpace{}%
\AgdaSymbol{=}\AgdaSpace{}%
\AgdaFunction{Σ}\AgdaSpace{}%
\AgdaBound{a}\AgdaSpace{}%
\AgdaFunction{꞉}\AgdaSpace{}%
\AgdaOperator{\AgdaFunction{∣}}\AgdaSpace{}%
\AgdaBound{𝑨}\AgdaSpace{}%
\AgdaOperator{\AgdaFunction{∣}}\AgdaSpace{}%
\AgdaFunction{,}\AgdaSpace{}%
\AgdaFunction{Σ}\AgdaSpace{}%
\AgdaBound{b}\AgdaSpace{}%
\AgdaFunction{꞉}\AgdaSpace{}%
\AgdaOperator{\AgdaFunction{∣}}\AgdaSpace{}%
\AgdaBound{𝑨}\AgdaSpace{}%
\AgdaOperator{\AgdaFunction{∣}}\AgdaSpace{}%
\AgdaFunction{,}\AgdaSpace{}%
\AgdaBound{a}\AgdaSpace{}%
\AgdaOperator{\AgdaDatatype{≡}}\AgdaSpace{}%
\AgdaBound{b}\<%
\\
%
\\[\AgdaEmptyExtraSkip]%
%
\>[1]\AgdaComment{-- The total relation A × A}\<%
\\
%
\>[1]\AgdaFunction{𝟏}\AgdaSpace{}%
\AgdaSymbol{:}\AgdaSpace{}%
\AgdaSymbol{\{}\AgdaBound{A}\AgdaSpace{}%
\AgdaSymbol{:}\AgdaSpace{}%
\AgdaBound{𝓤}\AgdaSpace{}%
\AgdaOperator{\AgdaFunction{̇}}\AgdaSpace{}%
\AgdaSymbol{\}}\AgdaSpace{}%
\AgdaSymbol{→}\AgdaSpace{}%
\AgdaFunction{Rel}\AgdaSpace{}%
\AgdaBound{A}\AgdaSpace{}%
\AgdaPrimitive{𝓤₀}\<%
\\
%
\>[1]\AgdaFunction{𝟏}\AgdaSpace{}%
\AgdaBound{a}\AgdaSpace{}%
\AgdaBound{b}\AgdaSpace{}%
\AgdaSymbol{=}\AgdaSpace{}%
\AgdaFunction{𝟙}\<%
\end{code}

\subsubsection{Properties of binary relations}\label{Binary.sssec:properties-of-binary-relations}
\begin{code}%
\>[1]\AgdaFunction{reflexive}\AgdaSpace{}%
\AgdaSymbol{:}\AgdaSpace{}%
\AgdaSymbol{\{}\AgdaBound{𝓡}\AgdaSpace{}%
\AgdaSymbol{:}\AgdaSpace{}%
\AgdaPostulate{Universe}\AgdaSymbol{\}\{}\AgdaBound{X}\AgdaSpace{}%
\AgdaSymbol{:}\AgdaSpace{}%
\AgdaBound{𝓤}\AgdaSpace{}%
\AgdaOperator{\AgdaFunction{̇}}\AgdaSpace{}%
\AgdaSymbol{\}}\AgdaSpace{}%
\AgdaSymbol{→}\AgdaSpace{}%
\AgdaFunction{Rel}\AgdaSpace{}%
\AgdaBound{X}\AgdaSpace{}%
\AgdaBound{𝓡}\AgdaSpace{}%
\AgdaSymbol{→}\AgdaSpace{}%
\AgdaBound{𝓤}\AgdaSpace{}%
\AgdaOperator{\AgdaPrimitive{⊔}}\AgdaSpace{}%
\AgdaBound{𝓡}\AgdaSpace{}%
\AgdaOperator{\AgdaFunction{̇}}\<%
\\
%
\>[1]\AgdaFunction{reflexive}\AgdaSpace{}%
\AgdaOperator{\AgdaBound{\AgdaUnderscore{}≈\AgdaUnderscore{}}}\AgdaSpace{}%
\AgdaSymbol{=}\AgdaSpace{}%
\AgdaSymbol{∀}\AgdaSpace{}%
\AgdaBound{x}\AgdaSpace{}%
\AgdaSymbol{→}\AgdaSpace{}%
\AgdaBound{x}\AgdaSpace{}%
\AgdaOperator{\AgdaBound{≈}}\AgdaSpace{}%
\AgdaBound{x}\<%
\\
%
\\[\AgdaEmptyExtraSkip]%
%
\>[1]\AgdaFunction{symmetric}\AgdaSpace{}%
\AgdaSymbol{:}\AgdaSpace{}%
\AgdaSymbol{\{}\AgdaBound{𝓡}\AgdaSpace{}%
\AgdaSymbol{:}\AgdaSpace{}%
\AgdaPostulate{Universe}\AgdaSymbol{\}\{}\AgdaBound{X}\AgdaSpace{}%
\AgdaSymbol{:}\AgdaSpace{}%
\AgdaBound{𝓤}\AgdaSpace{}%
\AgdaOperator{\AgdaFunction{̇}}\AgdaSpace{}%
\AgdaSymbol{\}}\AgdaSpace{}%
\AgdaSymbol{→}\AgdaSpace{}%
\AgdaFunction{Rel}\AgdaSpace{}%
\AgdaBound{X}\AgdaSpace{}%
\AgdaBound{𝓡}\AgdaSpace{}%
\AgdaSymbol{→}\AgdaSpace{}%
\AgdaBound{𝓤}\AgdaSpace{}%
\AgdaOperator{\AgdaPrimitive{⊔}}\AgdaSpace{}%
\AgdaBound{𝓡}\AgdaSpace{}%
\AgdaOperator{\AgdaFunction{̇}}\<%
\\
%
\>[1]\AgdaFunction{symmetric}\AgdaSpace{}%
\AgdaOperator{\AgdaBound{\AgdaUnderscore{}≈\AgdaUnderscore{}}}\AgdaSpace{}%
\AgdaSymbol{=}\AgdaSpace{}%
\AgdaSymbol{∀}\AgdaSpace{}%
\AgdaBound{x}\AgdaSpace{}%
\AgdaBound{y}\AgdaSpace{}%
\AgdaSymbol{→}\AgdaSpace{}%
\AgdaBound{x}\AgdaSpace{}%
\AgdaOperator{\AgdaBound{≈}}\AgdaSpace{}%
\AgdaBound{y}\AgdaSpace{}%
\AgdaSymbol{→}\AgdaSpace{}%
\AgdaBound{y}\AgdaSpace{}%
\AgdaOperator{\AgdaBound{≈}}\AgdaSpace{}%
\AgdaBound{x}\<%
\\
%
\\[\AgdaEmptyExtraSkip]%
%
\>[1]\AgdaFunction{transitive}\AgdaSpace{}%
\AgdaSymbol{:}\AgdaSpace{}%
\AgdaSymbol{\{}\AgdaBound{𝓡}\AgdaSpace{}%
\AgdaSymbol{:}\AgdaSpace{}%
\AgdaPostulate{Universe}\AgdaSymbol{\}\{}\AgdaBound{X}\AgdaSpace{}%
\AgdaSymbol{:}\AgdaSpace{}%
\AgdaBound{𝓤}\AgdaSpace{}%
\AgdaOperator{\AgdaFunction{̇}}\AgdaSpace{}%
\AgdaSymbol{\}}\AgdaSpace{}%
\AgdaSymbol{→}\AgdaSpace{}%
\AgdaFunction{Rel}\AgdaSpace{}%
\AgdaBound{X}\AgdaSpace{}%
\AgdaBound{𝓡}\AgdaSpace{}%
\AgdaSymbol{→}\AgdaSpace{}%
\AgdaBound{𝓤}\AgdaSpace{}%
\AgdaOperator{\AgdaPrimitive{⊔}}\AgdaSpace{}%
\AgdaBound{𝓡}\AgdaSpace{}%
\AgdaOperator{\AgdaFunction{̇}}\<%
\\
%
\>[1]\AgdaFunction{transitive}\AgdaSpace{}%
\AgdaOperator{\AgdaBound{\AgdaUnderscore{}≈\AgdaUnderscore{}}}\AgdaSpace{}%
\AgdaSymbol{=}\AgdaSpace{}%
\AgdaSymbol{∀}\AgdaSpace{}%
\AgdaBound{x}\AgdaSpace{}%
\AgdaBound{y}\AgdaSpace{}%
\AgdaBound{z}\AgdaSpace{}%
\AgdaSymbol{→}\AgdaSpace{}%
\AgdaBound{x}\AgdaSpace{}%
\AgdaOperator{\AgdaBound{≈}}\AgdaSpace{}%
\AgdaBound{y}\AgdaSpace{}%
\AgdaSymbol{→}\AgdaSpace{}%
\AgdaBound{y}\AgdaSpace{}%
\AgdaOperator{\AgdaBound{≈}}\AgdaSpace{}%
\AgdaBound{z}\AgdaSpace{}%
\AgdaSymbol{→}\AgdaSpace{}%
\AgdaBound{x}\AgdaSpace{}%
\AgdaOperator{\AgdaBound{≈}}\AgdaSpace{}%
\AgdaBound{z}\<%
\\
%
\\[\AgdaEmptyExtraSkip]%
%
\>[1]\AgdaFunction{is-subsingleton-valued}\AgdaSpace{}%
\AgdaSymbol{:}\AgdaSpace{}%
\AgdaSymbol{\{}\AgdaBound{𝓡}\AgdaSpace{}%
\AgdaSymbol{:}\AgdaSpace{}%
\AgdaPostulate{Universe}\AgdaSymbol{\}\{}\AgdaBound{A}\AgdaSpace{}%
\AgdaSymbol{:}\AgdaSpace{}%
\AgdaBound{𝓤}\AgdaSpace{}%
\AgdaOperator{\AgdaFunction{̇}}\AgdaSpace{}%
\AgdaSymbol{\}}\AgdaSpace{}%
\AgdaSymbol{→}\AgdaSpace{}%
\AgdaFunction{Rel}\AgdaSpace{}%
\AgdaBound{A}\AgdaSpace{}%
\AgdaBound{𝓡}\AgdaSpace{}%
\AgdaSymbol{→}\AgdaSpace{}%
\AgdaBound{𝓤}\AgdaSpace{}%
\AgdaOperator{\AgdaPrimitive{⊔}}\AgdaSpace{}%
\AgdaBound{𝓡}\AgdaSpace{}%
\AgdaOperator{\AgdaFunction{̇}}\<%
\\
%
\>[1]\AgdaFunction{is-subsingleton-valued}%
\>[25]\AgdaOperator{\AgdaBound{\AgdaUnderscore{}≈\AgdaUnderscore{}}}\AgdaSpace{}%
\AgdaSymbol{=}\AgdaSpace{}%
\AgdaSymbol{∀}\AgdaSpace{}%
\AgdaBound{x}\AgdaSpace{}%
\AgdaBound{y}\AgdaSpace{}%
\AgdaSymbol{→}\AgdaSpace{}%
\AgdaFunction{is-prop}\AgdaSpace{}%
\AgdaSymbol{(}\AgdaBound{x}\AgdaSpace{}%
\AgdaOperator{\AgdaBound{≈}}\AgdaSpace{}%
\AgdaBound{y}\AgdaSymbol{)}\<%
\end{code}

\subsubsection{Binary relation truncation}\label{Binary.sssec:binary-relation-truncation}
Recall, in Section~\ref{Preliminaries.sssec:truncation} we described the concept of truncation as it relates to ``proof-relevant'' mathematics.  Given a binary relation \af P, it may be necessary or desirable to assume that there is at most one way to prove that a given pair of elements is \af P-related\footnote{This is another example of ``proof-irrelevance''; indeed, proofs of \af P \ab x \ab y are indistinguishable, or rather any distinctions are irrelevant in the context of interest.} We use Escardo's \af{is-subsingleton} type to express this strong (truncation at level 1) assumption in the following definition: We say that (\ab x  \ac \ab y) belongs to \af P, or that \ab x and \ab y are \af P-related if and only if both \af P \ab x \ab y \emph{and} \af{is-subsingleton} (\af P \ab x \ab y) holds.
\ccpad
\begin{code}%
\>[1]\AgdaFunction{Rel₀}\AgdaSpace{}%
\AgdaSymbol{:}\AgdaSpace{}%
\AgdaBound{𝓤}\AgdaSpace{}%
\AgdaOperator{\AgdaFunction{̇}}\AgdaSpace{}%
\AgdaSymbol{→}\AgdaSpace{}%
\AgdaSymbol{(}\AgdaBound{𝓝}\AgdaSpace{}%
\AgdaSymbol{:}\AgdaSpace{}%
\AgdaPostulate{Universe}\AgdaSymbol{)}\AgdaSpace{}%
\AgdaSymbol{→}\AgdaSpace{}%
\AgdaBound{𝓤}\AgdaSpace{}%
\AgdaOperator{\AgdaPrimitive{⊔}}\AgdaSpace{}%
\AgdaBound{𝓝}\AgdaSpace{}%
\AgdaOperator{\AgdaPrimitive{⁺}}\AgdaSpace{}%
\AgdaOperator{\AgdaFunction{̇}}\<%
\\
%
\>[1]\AgdaFunction{Rel₀}\AgdaSpace{}%
\AgdaBound{A}\AgdaSpace{}%
\AgdaBound{𝓝}\AgdaSpace{}%
\AgdaSymbol{=}\AgdaSpace{}%
\AgdaFunction{Σ}\AgdaSpace{}%
\AgdaBound{P}\AgdaSpace{}%
\AgdaFunction{꞉}\AgdaSpace{}%
\AgdaSymbol{(}\AgdaBound{A}\AgdaSpace{}%
\AgdaSymbol{→}\AgdaSpace{}%
\AgdaBound{A}\AgdaSpace{}%
\AgdaSymbol{→}\AgdaSpace{}%
\AgdaBound{𝓝}\AgdaSpace{}%
\AgdaOperator{\AgdaFunction{̇}}\AgdaSymbol{)}\AgdaSpace{}%
\AgdaFunction{,}\AgdaSpace{}%
\AgdaSymbol{∀}\AgdaSpace{}%
\AgdaBound{x}\AgdaSpace{}%
\AgdaBound{y}\AgdaSpace{}%
\AgdaSymbol{→}\AgdaSpace{}%
\AgdaFunction{is-subsingleton}\AgdaSpace{}%
\AgdaSymbol{(}\AgdaBound{P}\AgdaSpace{}%
\AgdaBound{x}\AgdaSpace{}%
\AgdaBound{y}\AgdaSymbol{)}\<%
\end{code}
\ccpad
As above we define a \textbf{set} to be a type \ab{X} with the following property: for all \ab x \ab y \as : \ab X there is at most one proof that \ab x \ad ≡ \ab y. In other words, \ab{X} is a set if and only if it satisfies the following:
\ccpad

\as ∀ \ab x \ab y \as : \ab X  \as → \af{is-subsingleton} ( \ab x \af ≡ \ab y )

\subsubsection{Implication}\label{implication}

We denote and define implication as follows.

\begin{code}%
\>[0]\<%
\\
\>[0]\AgdaComment{-- (syntactic sugar)}\<%
\\
\>[0]\AgdaOperator{\AgdaFunction{\AgdaUnderscore{}on\AgdaUnderscore{}}}\AgdaSpace{}%
\AgdaSymbol{:}\AgdaSpace{}%
\AgdaSymbol{\{}\AgdaBound{𝓤}\AgdaSpace{}%
\AgdaBound{𝓥}\AgdaSpace{}%
\AgdaBound{𝓦}\AgdaSpace{}%
\AgdaSymbol{:}\AgdaSpace{}%
\AgdaPostulate{Universe}\AgdaSymbol{\}\{}\AgdaBound{A}\AgdaSpace{}%
\AgdaSymbol{:}\AgdaSpace{}%
\AgdaBound{𝓤}\AgdaSpace{}%
\AgdaOperator{\AgdaFunction{̇}}\AgdaSymbol{\}\{}\AgdaBound{B}\AgdaSpace{}%
\AgdaSymbol{:}\AgdaSpace{}%
\AgdaBound{𝓥}\AgdaSpace{}%
\AgdaOperator{\AgdaFunction{̇}}\AgdaSymbol{\}\{}\AgdaBound{C}\AgdaSpace{}%
\AgdaSymbol{:}\AgdaSpace{}%
\AgdaBound{𝓦}\AgdaSpace{}%
\AgdaOperator{\AgdaFunction{̇}}\AgdaSymbol{\}}\<%
\\
\>[0][@{}l@{\AgdaIndent{0}}]%
\>[1]\AgdaSymbol{→}%
\>[7]\AgdaSymbol{(}\AgdaBound{B}\AgdaSpace{}%
\AgdaSymbol{→}\AgdaSpace{}%
\AgdaBound{B}\AgdaSpace{}%
\AgdaSymbol{→}\AgdaSpace{}%
\AgdaBound{C}\AgdaSymbol{)}\AgdaSpace{}%
\AgdaSymbol{→}\AgdaSpace{}%
\AgdaSymbol{(}\AgdaBound{A}\AgdaSpace{}%
\AgdaSymbol{→}\AgdaSpace{}%
\AgdaBound{B}\AgdaSymbol{)}\AgdaSpace{}%
\AgdaSymbol{→}\AgdaSpace{}%
\AgdaSymbol{(}\AgdaBound{A}\AgdaSpace{}%
\AgdaSymbol{→}\AgdaSpace{}%
\AgdaBound{A}\AgdaSpace{}%
\AgdaSymbol{→}\AgdaSpace{}%
\AgdaBound{C}\AgdaSymbol{)}\<%
\\
%
\\[\AgdaEmptyExtraSkip]%
\>[0]\AgdaOperator{\AgdaBound{\AgdaUnderscore{}*\AgdaUnderscore{}}}\AgdaSpace{}%
\AgdaOperator{\AgdaFunction{on}}\AgdaSpace{}%
\AgdaBound{g}\AgdaSpace{}%
\AgdaSymbol{=}\AgdaSpace{}%
\AgdaSymbol{λ}\AgdaSpace{}%
\AgdaBound{x}\AgdaSpace{}%
\AgdaBound{y}\AgdaSpace{}%
\AgdaSymbol{→}\AgdaSpace{}%
\AgdaBound{g}\AgdaSpace{}%
\AgdaBound{x}\AgdaSpace{}%
\AgdaOperator{\AgdaBound{*}}\AgdaSpace{}%
\AgdaBound{g}\AgdaSpace{}%
\AgdaBound{y}\<%
\\
%
\\[\AgdaEmptyExtraSkip]%
%
\\[\AgdaEmptyExtraSkip]%
\>[0]\AgdaOperator{\AgdaFunction{\AgdaUnderscore{}⇒\AgdaUnderscore{}}}\AgdaSpace{}%
\AgdaSymbol{:}\AgdaSpace{}%
\AgdaSymbol{\{}\AgdaBound{𝓤}\AgdaSpace{}%
\AgdaBound{𝓥}\AgdaSpace{}%
\AgdaBound{𝓦}\AgdaSpace{}%
\AgdaBound{𝓧}\AgdaSpace{}%
\AgdaSymbol{:}\AgdaSpace{}%
\AgdaPostulate{Universe}\AgdaSymbol{\}\{}\AgdaBound{A}\AgdaSpace{}%
\AgdaSymbol{:}\AgdaSpace{}%
\AgdaBound{𝓤}\AgdaSpace{}%
\AgdaOperator{\AgdaFunction{̇}}\AgdaSpace{}%
\AgdaSymbol{\}}\AgdaSpace{}%
\AgdaSymbol{\{}\AgdaBound{B}\AgdaSpace{}%
\AgdaSymbol{:}\AgdaSpace{}%
\AgdaBound{𝓥}\AgdaSpace{}%
\AgdaOperator{\AgdaFunction{̇}}\AgdaSpace{}%
\AgdaSymbol{\}}\<%
\\
\>[0][@{}l@{\AgdaIndent{0}}]%
\>[1]\AgdaSymbol{→}%
\>[6]\AgdaFunction{REL}\AgdaSpace{}%
\AgdaBound{A}\AgdaSpace{}%
\AgdaBound{B}\AgdaSpace{}%
\AgdaBound{𝓦}\AgdaSpace{}%
\AgdaSymbol{→}\AgdaSpace{}%
\AgdaFunction{REL}\AgdaSpace{}%
\AgdaBound{A}\AgdaSpace{}%
\AgdaBound{B}\AgdaSpace{}%
\AgdaBound{𝓧}\AgdaSpace{}%
\AgdaSymbol{→}\AgdaSpace{}%
\AgdaBound{𝓤}\AgdaSpace{}%
\AgdaOperator{\AgdaPrimitive{⊔}}\AgdaSpace{}%
\AgdaBound{𝓥}\AgdaSpace{}%
\AgdaOperator{\AgdaPrimitive{⊔}}\AgdaSpace{}%
\AgdaBound{𝓦}\AgdaSpace{}%
\AgdaOperator{\AgdaPrimitive{⊔}}\AgdaSpace{}%
\AgdaBound{𝓧}\AgdaSpace{}%
\AgdaOperator{\AgdaFunction{̇}}\<%
\\
%
\\[\AgdaEmptyExtraSkip]%
\>[0]\AgdaBound{P}\AgdaSpace{}%
\AgdaOperator{\AgdaFunction{⇒}}\AgdaSpace{}%
\AgdaBound{Q}\AgdaSpace{}%
\AgdaSymbol{=}\AgdaSpace{}%
\AgdaSymbol{∀}\AgdaSpace{}%
\AgdaSymbol{\{}\AgdaBound{i}\AgdaSpace{}%
\AgdaBound{j}\AgdaSymbol{\}}\AgdaSpace{}%
\AgdaSymbol{→}\AgdaSpace{}%
\AgdaBound{P}\AgdaSpace{}%
\AgdaBound{i}\AgdaSpace{}%
\AgdaBound{j}\AgdaSpace{}%
\AgdaSymbol{→}\AgdaSpace{}%
\AgdaBound{Q}\AgdaSpace{}%
\AgdaBound{i}\AgdaSpace{}%
\AgdaBound{j}\<%
\\
%
\\[\AgdaEmptyExtraSkip]%
\>[0]\AgdaKeyword{infixr}\AgdaSpace{}%
\AgdaNumber{4}\AgdaSpace{}%
\AgdaOperator{\AgdaFunction{\AgdaUnderscore{}⇒\AgdaUnderscore{}}}\<%
\end{code}
\ccpad
Here is a more general version that we borrow from the standard library and translate into MHE/UALib notation.
\ccpad
\begin{code}%
\>[0]\AgdaOperator{\AgdaFunction{\AgdaUnderscore{}=[\AgdaUnderscore{}]⇒\AgdaUnderscore{}}}\AgdaSpace{}%
\AgdaSymbol{:}\AgdaSpace{}%
\AgdaSymbol{\{}\AgdaBound{𝓤}\AgdaSpace{}%
\AgdaBound{𝓥}\AgdaSpace{}%
\AgdaBound{𝓡}\AgdaSpace{}%
\AgdaBound{𝓢}\AgdaSpace{}%
\AgdaSymbol{:}\AgdaSpace{}%
\AgdaPostulate{Universe}\AgdaSymbol{\}\{}\AgdaBound{A}\AgdaSpace{}%
\AgdaSymbol{:}\AgdaSpace{}%
\AgdaBound{𝓤}\AgdaSpace{}%
\AgdaOperator{\AgdaFunction{̇}}\AgdaSpace{}%
\AgdaSymbol{\}}\AgdaSpace{}%
\AgdaSymbol{\{}\AgdaBound{B}\AgdaSpace{}%
\AgdaSymbol{:}\AgdaSpace{}%
\AgdaBound{𝓥}\AgdaSpace{}%
\AgdaOperator{\AgdaFunction{̇}}\AgdaSpace{}%
\AgdaSymbol{\}}\<%
\\
\>[0][@{}l@{\AgdaIndent{0}}]%
\>[1]\AgdaSymbol{→}%
\>[10]\AgdaFunction{Rel}\AgdaSpace{}%
\AgdaBound{A}\AgdaSpace{}%
\AgdaBound{𝓡}\AgdaSpace{}%
\AgdaSymbol{→}\AgdaSpace{}%
\AgdaSymbol{(}\AgdaBound{A}\AgdaSpace{}%
\AgdaSymbol{→}\AgdaSpace{}%
\AgdaBound{B}\AgdaSymbol{)}\AgdaSpace{}%
\AgdaSymbol{→}\AgdaSpace{}%
\AgdaFunction{Rel}\AgdaSpace{}%
\AgdaBound{B}\AgdaSpace{}%
\AgdaBound{𝓢}\AgdaSpace{}%
\AgdaSymbol{→}\AgdaSpace{}%
\AgdaBound{𝓤}\AgdaSpace{}%
\AgdaOperator{\AgdaPrimitive{⊔}}\AgdaSpace{}%
\AgdaBound{𝓡}\AgdaSpace{}%
\AgdaOperator{\AgdaPrimitive{⊔}}\AgdaSpace{}%
\AgdaBound{𝓢}\AgdaSpace{}%
\AgdaOperator{\AgdaFunction{̇}}\<%
\\
%
\\[\AgdaEmptyExtraSkip]%
\>[0]\AgdaBound{P}\AgdaSpace{}%
\AgdaOperator{\AgdaFunction{=[}}\AgdaSpace{}%
\AgdaBound{g}\AgdaSpace{}%
\AgdaOperator{\AgdaFunction{]⇒}}\AgdaSpace{}%
\AgdaBound{Q}\AgdaSpace{}%
\AgdaSymbol{=}\AgdaSpace{}%
\AgdaBound{P}\AgdaSpace{}%
\AgdaOperator{\AgdaFunction{⇒}}\AgdaSpace{}%
\AgdaSymbol{(}\AgdaBound{Q}\AgdaSpace{}%
\AgdaOperator{\AgdaFunction{on}}\AgdaSpace{}%
\AgdaBound{g}\AgdaSymbol{)}\<%
\\
%
\\[\AgdaEmptyExtraSkip]%
\>[0]\AgdaKeyword{infixr}\AgdaSpace{}%
\AgdaNumber{4}\AgdaSpace{}%
\AgdaOperator{\AgdaFunction{\AgdaUnderscore{}=[\AgdaUnderscore{}]⇒\AgdaUnderscore{}}}\<%
\\
\>[0]\<%
\end{code}

----------








\subsection{Equivalence Relation Types}\label{sec:equiv-relat}
Here we describe part of the \ualibEquivalences module of the \agdaualib.
We define an equivalence relation using a \AgdaKeyword{record} type, as follows:
%% BEGIN: \subsubsection{Equivalence Relation
Types}\label{equivalence-relation-types}

This section presents the {[}UALib.Relations.Equivalences{]}{[}{]}
module of the {[}Agda Universal Algebra Library{]}{[}{]}.

This is all pretty standard. The notions of reflexivity, symmetry, and
transitivity are defined as one would expect, so we need not dwell on
them.

\begin{code}%
\>[0]\<%
\\
\>[0]\AgdaSymbol{\{-\#}\AgdaSpace{}%
\AgdaKeyword{OPTIONS}\AgdaSpace{}%
\AgdaPragma{--without-K}\AgdaSpace{}%
\AgdaPragma{--exact-split}\AgdaSpace{}%
\AgdaPragma{--safe}\AgdaSpace{}%
\AgdaSymbol{\#-\}}\<%
\\
%
\\[\AgdaEmptyExtraSkip]%
\>[0]\AgdaKeyword{module}\AgdaSpace{}%
\AgdaModule{UALib.Relations.Equivalences}\AgdaSpace{}%
\AgdaKeyword{where}\<%
\\
%
\\[\AgdaEmptyExtraSkip]%
\>[0]\AgdaKeyword{open}\AgdaSpace{}%
\AgdaKeyword{import}\AgdaSpace{}%
\AgdaModule{UALib.Relations.Binary}\AgdaSpace{}%
\AgdaKeyword{public}\<%
\\
%
\\[\AgdaEmptyExtraSkip]%
\>[0]\AgdaKeyword{module}\AgdaSpace{}%
\AgdaModule{\AgdaUnderscore{}}\AgdaSpace{}%
\AgdaSymbol{\{}\AgdaBound{𝓤}\AgdaSpace{}%
\AgdaBound{𝓡}\AgdaSpace{}%
\AgdaSymbol{:}\AgdaSpace{}%
\AgdaPostulate{Universe}\AgdaSymbol{\}}\AgdaSpace{}%
\AgdaKeyword{where}\<%
\\
%
\\[\AgdaEmptyExtraSkip]%
\>[0][@{}l@{\AgdaIndent{0}}]%
\>[1]\AgdaKeyword{record}\AgdaSpace{}%
\AgdaRecord{IsEquivalence}\AgdaSpace{}%
\AgdaSymbol{\{}\AgdaBound{A}\AgdaSpace{}%
\AgdaSymbol{:}\AgdaSpace{}%
\AgdaBound{𝓤}\AgdaSpace{}%
\AgdaOperator{\AgdaFunction{̇}}\AgdaSpace{}%
\AgdaSymbol{\}}\AgdaSpace{}%
\AgdaSymbol{(}\AgdaOperator{\AgdaBound{\AgdaUnderscore{}≈\AgdaUnderscore{}}}\AgdaSpace{}%
\AgdaSymbol{:}\AgdaSpace{}%
\AgdaFunction{Rel}\AgdaSpace{}%
\AgdaBound{A}\AgdaSpace{}%
\AgdaBound{𝓡}\AgdaSymbol{)}\AgdaSpace{}%
\AgdaSymbol{:}\AgdaSpace{}%
\AgdaBound{𝓤}\AgdaSpace{}%
\AgdaOperator{\AgdaPrimitive{⊔}}\AgdaSpace{}%
\AgdaBound{𝓡}\AgdaSpace{}%
\AgdaOperator{\AgdaFunction{̇}}\AgdaSpace{}%
\AgdaKeyword{where}\<%
\\
\>[1][@{}l@{\AgdaIndent{0}}]%
\>[2]\AgdaKeyword{field}\<%
\\
\>[2][@{}l@{\AgdaIndent{0}}]%
\>[3]\AgdaField{rfl}%
\>[9]\AgdaSymbol{:}\AgdaSpace{}%
\AgdaFunction{reflexive}\AgdaSpace{}%
\AgdaOperator{\AgdaBound{\AgdaUnderscore{}≈\AgdaUnderscore{}}}\<%
\\
%
\>[3]\AgdaField{sym}%
\>[9]\AgdaSymbol{:}\AgdaSpace{}%
\AgdaFunction{symmetric}\AgdaSpace{}%
\AgdaOperator{\AgdaBound{\AgdaUnderscore{}≈\AgdaUnderscore{}}}\<%
\\
%
\>[3]\AgdaField{trans}\AgdaSpace{}%
\AgdaSymbol{:}\AgdaSpace{}%
\AgdaFunction{transitive}\AgdaSpace{}%
\AgdaOperator{\AgdaBound{\AgdaUnderscore{}≈\AgdaUnderscore{}}}\<%
\\
%
\\[\AgdaEmptyExtraSkip]%
%
\>[1]\AgdaFunction{is-equivalence-relation}\AgdaSpace{}%
\AgdaSymbol{:}\AgdaSpace{}%
\AgdaSymbol{\{}\AgdaBound{X}\AgdaSpace{}%
\AgdaSymbol{:}\AgdaSpace{}%
\AgdaBound{𝓤}\AgdaSpace{}%
\AgdaOperator{\AgdaFunction{̇}}\AgdaSpace{}%
\AgdaSymbol{\}}\AgdaSpace{}%
\AgdaSymbol{→}\AgdaSpace{}%
\AgdaFunction{Rel}\AgdaSpace{}%
\AgdaBound{X}\AgdaSpace{}%
\AgdaBound{𝓡}\AgdaSpace{}%
\AgdaSymbol{→}\AgdaSpace{}%
\AgdaBound{𝓤}\AgdaSpace{}%
\AgdaOperator{\AgdaPrimitive{⊔}}\AgdaSpace{}%
\AgdaBound{𝓡}\AgdaSpace{}%
\AgdaOperator{\AgdaFunction{̇}}\<%
\\
%
\>[1]\AgdaFunction{is-equivalence-relation}\AgdaSpace{}%
\AgdaOperator{\AgdaBound{\AgdaUnderscore{}≈\AgdaUnderscore{}}}\AgdaSpace{}%
\AgdaSymbol{=}%
\>[57I]\AgdaFunction{is-subsingleton-valued}\AgdaSpace{}%
\AgdaOperator{\AgdaBound{\AgdaUnderscore{}≈\AgdaUnderscore{}}}\<%
\\
\>[.][@{}l@{}]\<[57I]%
\>[31]\AgdaOperator{\AgdaFunction{×}}\AgdaSpace{}%
\AgdaFunction{reflexive}\AgdaSpace{}%
\AgdaOperator{\AgdaBound{\AgdaUnderscore{}≈\AgdaUnderscore{}}}\AgdaSpace{}%
\AgdaOperator{\AgdaFunction{×}}\AgdaSpace{}%
\AgdaFunction{symmetric}\AgdaSpace{}%
\AgdaOperator{\AgdaBound{\AgdaUnderscore{}≈\AgdaUnderscore{}}}\AgdaSpace{}%
\AgdaOperator{\AgdaFunction{×}}\AgdaSpace{}%
\AgdaFunction{transitive}\AgdaSpace{}%
\AgdaOperator{\AgdaBound{\AgdaUnderscore{}≈\AgdaUnderscore{}}}\<%
\\
\>[0]\<%
\end{code}

\paragraph{Examples}\label{examples}

The zero relation 𝟎 is equivalent to the identity relation \texttt{≡}
and, of course, these are both equivalence relations. (In fact, we
already proved reflexivity, symmetry, and transitivity of \texttt{≡} in
the {[}UALib.Prelude.Equality{]}{[}{]} module, so we simply apply the
corresponding proofs where appropriate.)

\begin{code}%
\>[0]\<%
\\
\>[0]\AgdaKeyword{module}\AgdaSpace{}%
\AgdaModule{\AgdaUnderscore{}}\AgdaSpace{}%
\AgdaSymbol{\{}\AgdaBound{𝓤}\AgdaSpace{}%
\AgdaSymbol{:}\AgdaSpace{}%
\AgdaPostulate{Universe}\AgdaSymbol{\}}\AgdaSpace{}%
\AgdaKeyword{where}\<%
\\
%
\\[\AgdaEmptyExtraSkip]%
\>[0][@{}l@{\AgdaIndent{0}}]%
\>[1]\AgdaFunction{𝟎-IsEquivalence}\AgdaSpace{}%
\AgdaSymbol{:}\AgdaSpace{}%
\AgdaSymbol{\{}\AgdaBound{A}\AgdaSpace{}%
\AgdaSymbol{:}\AgdaSpace{}%
\AgdaBound{𝓤}\AgdaSpace{}%
\AgdaOperator{\AgdaFunction{̇}}\AgdaSpace{}%
\AgdaSymbol{\}}\AgdaSpace{}%
\AgdaSymbol{→}\AgdaSpace{}%
\AgdaRecord{IsEquivalence}\AgdaSymbol{\{}\AgdaBound{𝓤}\AgdaSymbol{\}\{}\AgdaArgument{A}\AgdaSpace{}%
\AgdaSymbol{=}\AgdaSpace{}%
\AgdaBound{A}\AgdaSymbol{\}}\AgdaSpace{}%
\AgdaFunction{𝟎-rel}\<%
\\
%
\>[1]\AgdaFunction{𝟎-IsEquivalence}\AgdaSpace{}%
\AgdaSymbol{=}\AgdaSpace{}%
\AgdaKeyword{record}\AgdaSpace{}%
\AgdaSymbol{\{}\AgdaSpace{}%
\AgdaField{rfl}\AgdaSpace{}%
\AgdaSymbol{=}\AgdaSpace{}%
\AgdaFunction{≡-rfl}\AgdaSymbol{;}\AgdaSpace{}%
\AgdaField{sym}\AgdaSpace{}%
\AgdaSymbol{=}\AgdaSpace{}%
\AgdaFunction{≡-sym}\AgdaSymbol{;}\AgdaSpace{}%
\AgdaField{trans}\AgdaSpace{}%
\AgdaSymbol{=}\AgdaSpace{}%
\AgdaFunction{≡-trans}\AgdaSpace{}%
\AgdaSymbol{\}}\<%
\\
%
\\[\AgdaEmptyExtraSkip]%
%
\>[1]\AgdaFunction{≡-IsEquivalence}\AgdaSpace{}%
\AgdaSymbol{:}\AgdaSpace{}%
\AgdaSymbol{\{}\AgdaBound{A}\AgdaSpace{}%
\AgdaSymbol{:}\AgdaSpace{}%
\AgdaBound{𝓤}\AgdaSpace{}%
\AgdaOperator{\AgdaFunction{̇}}\AgdaSymbol{\}}\AgdaSpace{}%
\AgdaSymbol{→}\AgdaSpace{}%
\AgdaRecord{IsEquivalence}\AgdaSymbol{\{}\AgdaBound{𝓤}\AgdaSymbol{\}\{}\AgdaArgument{A}\AgdaSpace{}%
\AgdaSymbol{=}\AgdaSpace{}%
\AgdaBound{A}\AgdaSymbol{\}}\AgdaSpace{}%
\AgdaOperator{\AgdaDatatype{\AgdaUnderscore{}≡\AgdaUnderscore{}}}\<%
\\
%
\>[1]\AgdaFunction{≡-IsEquivalence}\AgdaSpace{}%
\AgdaSymbol{=}\AgdaSpace{}%
\AgdaKeyword{record}\AgdaSpace{}%
\AgdaSymbol{\{}\AgdaSpace{}%
\AgdaField{rfl}\AgdaSpace{}%
\AgdaSymbol{=}\AgdaSpace{}%
\AgdaFunction{≡-rfl}\AgdaSpace{}%
\AgdaSymbol{;}\AgdaSpace{}%
\AgdaField{sym}\AgdaSpace{}%
\AgdaSymbol{=}\AgdaSpace{}%
\AgdaFunction{≡-sym}\AgdaSpace{}%
\AgdaSymbol{;}\AgdaSpace{}%
\AgdaField{trans}\AgdaSpace{}%
\AgdaSymbol{=}\AgdaSpace{}%
\AgdaFunction{≡-trans}\AgdaSpace{}%
\AgdaSymbol{\}}\<%
\\
\>[0]\<%
\end{code}

Finally, we should have at our disposal a proof of the fact that the
kernel of a function is an equivalence relation.

\begin{code}%
\>[0]\<%
\\
\>[0][@{}l@{\AgdaIndent{1}}]%
\>[1]\AgdaFunction{map-kernel-IsEquivalence}\AgdaSpace{}%
\AgdaSymbol{:}%
\>[122I]\AgdaSymbol{\{}\AgdaBound{𝓦}\AgdaSpace{}%
\AgdaSymbol{:}\AgdaSpace{}%
\AgdaPostulate{Universe}\AgdaSymbol{\}\{}\AgdaBound{A}\AgdaSpace{}%
\AgdaSymbol{:}\AgdaSpace{}%
\AgdaBound{𝓤}\AgdaSpace{}%
\AgdaOperator{\AgdaFunction{̇}}\AgdaSymbol{\}\{}\AgdaBound{B}\AgdaSpace{}%
\AgdaSymbol{:}\AgdaSpace{}%
\AgdaBound{𝓦}\AgdaSpace{}%
\AgdaOperator{\AgdaFunction{̇}}\AgdaSymbol{\}}\<%
\\
\>[.][@{}l@{}]\<[122I]%
\>[28]\AgdaSymbol{(}\AgdaBound{f}\AgdaSpace{}%
\AgdaSymbol{:}\AgdaSpace{}%
\AgdaBound{A}\AgdaSpace{}%
\AgdaSymbol{→}\AgdaSpace{}%
\AgdaBound{B}\AgdaSymbol{)}\AgdaSpace{}%
\AgdaSymbol{→}\AgdaSpace{}%
\AgdaRecord{IsEquivalence}\AgdaSpace{}%
\AgdaSymbol{(}\AgdaFunction{KER-rel}\AgdaSpace{}%
\AgdaBound{f}\AgdaSymbol{)}\<%
\\
%
\\[\AgdaEmptyExtraSkip]%
%
\>[1]\AgdaFunction{map-kernel-IsEquivalence}\AgdaSpace{}%
\AgdaSymbol{\{}\AgdaBound{𝓦}\AgdaSymbol{\}}\AgdaSpace{}%
\AgdaBound{f}\AgdaSpace{}%
\AgdaSymbol{=}\<%
\\
\>[1][@{}l@{\AgdaIndent{0}}]%
\>[2]\AgdaKeyword{record}%
\>[142I]\AgdaSymbol{\{}\AgdaSpace{}%
\AgdaField{rfl}\AgdaSpace{}%
\AgdaSymbol{=}\AgdaSpace{}%
\AgdaSymbol{λ}\AgdaSpace{}%
\AgdaBound{x}\AgdaSpace{}%
\AgdaSymbol{→}\AgdaSpace{}%
\AgdaInductiveConstructor{𝓇ℯ𝒻𝓁}\<%
\\
\>[.][@{}l@{}]\<[142I]%
\>[9]\AgdaSymbol{;}\AgdaSpace{}%
\AgdaField{sym}\AgdaSpace{}%
\AgdaSymbol{=}\AgdaSpace{}%
\AgdaSymbol{λ}\AgdaSpace{}%
\AgdaBound{x}\AgdaSpace{}%
\AgdaBound{y}\AgdaSpace{}%
\AgdaBound{x₁}\AgdaSpace{}%
\AgdaSymbol{→}\AgdaSpace{}%
\AgdaFunction{≡-sym}\AgdaSymbol{\{}\AgdaBound{𝓦}\AgdaSymbol{\}}\AgdaSpace{}%
\AgdaSymbol{(}\AgdaBound{f}\AgdaSpace{}%
\AgdaBound{x}\AgdaSymbol{)}\AgdaSpace{}%
\AgdaSymbol{(}\AgdaBound{f}\AgdaSpace{}%
\AgdaBound{y}\AgdaSymbol{)}\AgdaSpace{}%
\AgdaBound{x₁}\<%
\\
%
\>[9]\AgdaSymbol{;}\AgdaSpace{}%
\AgdaField{trans}\AgdaSpace{}%
\AgdaSymbol{=}\AgdaSpace{}%
\AgdaSymbol{λ}\AgdaSpace{}%
\AgdaBound{x}\AgdaSpace{}%
\AgdaBound{y}\AgdaSpace{}%
\AgdaBound{z}\AgdaSpace{}%
\AgdaBound{x₁}\AgdaSpace{}%
\AgdaBound{x₂}\AgdaSpace{}%
\AgdaSymbol{→}\AgdaSpace{}%
\AgdaFunction{≡-trans}\AgdaSpace{}%
\AgdaSymbol{(}\AgdaBound{f}\AgdaSpace{}%
\AgdaBound{x}\AgdaSymbol{)}\AgdaSpace{}%
\AgdaSymbol{(}\AgdaBound{f}\AgdaSpace{}%
\AgdaBound{y}\AgdaSymbol{)}\AgdaSpace{}%
\AgdaSymbol{(}\AgdaBound{f}\AgdaSpace{}%
\AgdaBound{z}\AgdaSymbol{)}\AgdaSpace{}%
\AgdaBound{x₁}\AgdaSpace{}%
\AgdaBound{x₂}\AgdaSpace{}%
\AgdaSymbol{\}}\<%
\\
\>[0]\<%
\end{code}

\begin{center}\rule{0.5\linewidth}{\linethickness}\end{center}

\href{UALib.Relations.Binary.html}{← UALib.Relations.Binary}
{\href{UALib.Relations.Equivalences.html}{UALib.Relations.Equivalences
→}}

\{\% include UALib.Links.md \%\}

\ccpad
\begin{code}%
\>[1]\AgdaKeyword{record}\AgdaSpace{}%
\AgdaRecord{IsEquivalence}\AgdaSpace{}%
\AgdaSymbol{\{}\AgdaBound{A}\AgdaSpace{}%
\AgdaSymbol{:}\AgdaSpace{}%
\AgdaBound{𝓤}\AgdaSpace{}%
\AgdaOperator{\AgdaFunction{̇}}\AgdaSpace{}%
\AgdaSymbol{\}}\AgdaSpace{}%
\AgdaSymbol{(}\AgdaOperator{\AgdaBound{\AgdaUnderscore{}≈\AgdaUnderscore{}}}\AgdaSpace{}%
\AgdaSymbol{:}\AgdaSpace{}%
\AgdaFunction{Rel}\AgdaSpace{}%
\AgdaBound{A}\AgdaSpace{}%
\AgdaBound{𝓡}\AgdaSymbol{)}\AgdaSpace{}%
\AgdaSymbol{:}\AgdaSpace{}%
\AgdaBound{𝓤}\AgdaSpace{}%
\AgdaOperator{\AgdaPrimitive{⊔}}\AgdaSpace{}%
\AgdaBound{𝓡}\AgdaSpace{}%
\AgdaOperator{\AgdaFunction{̇}}\AgdaSpace{}%
\AgdaKeyword{where}\<%
\\
\>[1][@{}l@{\AgdaIndent{0}}]%
\>[2]\AgdaKeyword{field}\<%
\\
\>[2][@{}l@{\AgdaIndent{0}}]%
\>[3]\AgdaField{rfl}%
\>[9]\AgdaSymbol{:}\AgdaSpace{}%
\AgdaFunction{reflexive}\AgdaSpace{}%
\AgdaOperator{\AgdaBound{\AgdaUnderscore{}≈\AgdaUnderscore{}}}\<%
\\
%
\>[3]\AgdaField{sym}%
\>[9]\AgdaSymbol{:}\AgdaSpace{}%
\AgdaFunction{symmetric}\AgdaSpace{}%
\AgdaOperator{\AgdaBound{\AgdaUnderscore{}≈\AgdaUnderscore{}}}\<%
\\
%
\>[3]\AgdaField{trans}\AgdaSpace{}%
\AgdaSymbol{:}\AgdaSpace{}%
\AgdaFunction{transitive}\AgdaSpace{}%
\AgdaOperator{\AgdaBound{\AgdaUnderscore{}≈\AgdaUnderscore{}}}\<%
\end{code}

\subsubsection{Examples}\label{Equivalences.sssec:examples}
The zero relation \af{𝟎-rel} is equivalent to the identity relation \aod ≡ and, of course, these are both equivalence relations. Indeed, we saw in \S\ref{ssec:equality} that \aod ≡ is reflexive, symmetric, and transitive, so we simply apply the corresponding proofs where appropriate.
\ccpad
\begin{code}%
\>[1]\AgdaFunction{𝟎-IsEquivalence}\AgdaSpace{}%
\AgdaSymbol{:}\AgdaSpace{}%
\AgdaSymbol{\{}\AgdaBound{A}\AgdaSpace{}%
\AgdaSymbol{:}\AgdaSpace{}%
\AgdaBound{𝓤}\AgdaSpace{}%
\AgdaOperator{\AgdaFunction{̇}}\AgdaSpace{}%
\AgdaSymbol{\}}\AgdaSpace{}%
\AgdaSymbol{→}\AgdaSpace{}%
\AgdaRecord{IsEquivalence}\AgdaSymbol{\{}\AgdaBound{𝓤}\AgdaSymbol{\}\{}\AgdaArgument{A}\AgdaSpace{}%
\AgdaSymbol{=}\AgdaSpace{}%
\AgdaBound{A}\AgdaSymbol{\}}\AgdaSpace{}%
\AgdaFunction{𝟎-rel}\<%
\\
%
\>[1]\AgdaFunction{𝟎-IsEquivalence}\AgdaSpace{}%
\AgdaSymbol{=}\AgdaSpace{}%
\AgdaKeyword{record}\AgdaSpace{}%
\AgdaSymbol{\{}\AgdaSpace{}%
\AgdaField{rfl}\AgdaSpace{}%
\AgdaSymbol{=}\AgdaSpace{}%
\AgdaFunction{≡-rfl}\AgdaSymbol{;}\AgdaSpace{}%
\AgdaField{sym}\AgdaSpace{}%
\AgdaSymbol{=}\AgdaSpace{}%
\AgdaFunction{≡-sym}\AgdaSymbol{;}\AgdaSpace{}%
\AgdaField{trans}\AgdaSpace{}%
\AgdaSymbol{=}\AgdaSpace{}%
\AgdaFunction{≡-trans}\AgdaSpace{}%
\AgdaSymbol{\}}\<%
\\
%
\\[\AgdaEmptyExtraSkip]%
%
\>[1]\AgdaFunction{≡-IsEquivalence}\AgdaSpace{}%
\AgdaSymbol{:}\AgdaSpace{}%
\AgdaSymbol{\{}\AgdaBound{A}\AgdaSpace{}%
\AgdaSymbol{:}\AgdaSpace{}%
\AgdaBound{𝓤}\AgdaSpace{}%
\AgdaOperator{\AgdaFunction{̇}}\AgdaSymbol{\}}\AgdaSpace{}%
\AgdaSymbol{→}\AgdaSpace{}%
\AgdaRecord{IsEquivalence}\AgdaSymbol{\{}\AgdaBound{𝓤}\AgdaSymbol{\}\{}\AgdaArgument{A}\AgdaSpace{}%
\AgdaSymbol{=}\AgdaSpace{}%
\AgdaBound{A}\AgdaSymbol{\}}\AgdaSpace{}%
\AgdaOperator{\AgdaDatatype{\AgdaUnderscore{}≡\AgdaUnderscore{}}}\<%
\\
%
\>[1]\AgdaFunction{≡-IsEquivalence}\AgdaSpace{}%
\AgdaSymbol{=}\AgdaSpace{}%
\AgdaKeyword{record}\AgdaSpace{}%
\AgdaSymbol{\{}\AgdaSpace{}%
\AgdaField{rfl}\AgdaSpace{}%
\AgdaSymbol{=}\AgdaSpace{}%
\AgdaFunction{≡-rfl}\AgdaSpace{}%
\AgdaSymbol{;}\AgdaSpace{}%
\AgdaField{sym}\AgdaSpace{}%
\AgdaSymbol{=}\AgdaSpace{}%
\AgdaFunction{≡-sym}\AgdaSpace{}%
\AgdaSymbol{;}\AgdaSpace{}%
\AgdaField{trans}\AgdaSpace{}%
\AgdaSymbol{=}\AgdaSpace{}%
\AgdaFunction{≡-trans}\AgdaSpace{}%
\AgdaSymbol{\}}\<%
\end{code}
\ccpad
Another obvious example of an equivalence relation is the kernel of a function.
\ccpad
\begin{code}%
\>[0][@{}l@{\AgdaIndent{1}}]%
\>[1]\AgdaFunction{map-kernel-IsEquivalence}\AgdaSpace{}%
\AgdaSymbol{:}%
\>[122I]\AgdaSymbol{\{}\AgdaBound{𝓦}\AgdaSpace{}%
\AgdaSymbol{:}\AgdaSpace{}%
\AgdaPostulate{Universe}\AgdaSymbol{\}\{}\AgdaBound{A}\AgdaSpace{}%
\AgdaSymbol{:}\AgdaSpace{}%
\AgdaBound{𝓤}\AgdaSpace{}%
\AgdaOperator{\AgdaFunction{̇}}\AgdaSymbol{\}\{}\AgdaBound{B}\AgdaSpace{}%
\AgdaSymbol{:}\AgdaSpace{}%
\AgdaBound{𝓦}\AgdaSpace{}%
\AgdaOperator{\AgdaFunction{̇}}\AgdaSymbol{\}}\<%
\\
\>[.][@{}l@{}]\<[122I]%
\>[28]\AgdaSymbol{(}\AgdaBound{f}\AgdaSpace{}%
\AgdaSymbol{:}\AgdaSpace{}%
\AgdaBound{A}\AgdaSpace{}%
\AgdaSymbol{→}\AgdaSpace{}%
\AgdaBound{B}\AgdaSymbol{)}\AgdaSpace{}%
\AgdaSymbol{→}\AgdaSpace{}%
\AgdaRecord{IsEquivalence}\AgdaSpace{}%
\AgdaSymbol{(}\AgdaFunction{KER-rel}\AgdaSpace{}%
\AgdaBound{f}\AgdaSymbol{)}\<%
\\
%
\\[\AgdaEmptyExtraSkip]%
%
\>[1]\AgdaFunction{map-kernel-IsEquivalence}\AgdaSpace{}%
\AgdaSymbol{\{}\AgdaBound{𝓦}\AgdaSymbol{\}}\AgdaSpace{}%
\AgdaBound{f}\AgdaSpace{}%
\AgdaSymbol{=}\<%
\\
\>[1][@{}l@{\AgdaIndent{0}}]%
\>[2]\AgdaKeyword{record}%
\>[142I]\AgdaSymbol{\{}\AgdaSpace{}%
\AgdaField{rfl}\AgdaSpace{}%
\AgdaSymbol{=}\AgdaSpace{}%
\AgdaSymbol{λ}\AgdaSpace{}%
\AgdaBound{x}\AgdaSpace{}%
\AgdaSymbol{→}\AgdaSpace{}%
\AgdaInductiveConstructor{𝓇ℯ𝒻𝓁}\<%
\\
\>[.][@{}l@{}]\<[142I]%
\>[9]\AgdaSymbol{;}\AgdaSpace{}%
\AgdaField{sym}\AgdaSpace{}%
\AgdaSymbol{=}\AgdaSpace{}%
\AgdaSymbol{λ}\AgdaSpace{}%
\AgdaBound{x}\AgdaSpace{}%
\AgdaBound{y}\AgdaSpace{}%
\AgdaBound{x₁}\AgdaSpace{}%
\AgdaSymbol{→}\AgdaSpace{}%
\AgdaFunction{≡-sym}\AgdaSymbol{\{}\AgdaBound{𝓦}\AgdaSymbol{\}}\AgdaSpace{}%
\AgdaSymbol{(}\AgdaBound{f}\AgdaSpace{}%
\AgdaBound{x}\AgdaSymbol{)}\AgdaSpace{}%
\AgdaSymbol{(}\AgdaBound{f}\AgdaSpace{}%
\AgdaBound{y}\AgdaSymbol{)}\AgdaSpace{}%
\AgdaBound{x₁}\<%
\\
%
\>[9]\AgdaSymbol{;}\AgdaSpace{}%
\AgdaField{trans}\AgdaSpace{}%
\AgdaSymbol{=}\AgdaSpace{}%
\AgdaSymbol{λ}\AgdaSpace{}%
\AgdaBound{x}\AgdaSpace{}%
\AgdaBound{y}\AgdaSpace{}%
\AgdaBound{z}\AgdaSpace{}%
\AgdaBound{x₁}\AgdaSpace{}%
\AgdaBound{x₂}\AgdaSpace{}%
\AgdaSymbol{→}\AgdaSpace{}%
\AgdaFunction{≡-trans}\AgdaSpace{}%
\AgdaSymbol{(}\AgdaBound{f}\AgdaSpace{}%
\AgdaBound{x}\AgdaSymbol{)}\AgdaSpace{}%
\AgdaSymbol{(}\AgdaBound{f}\AgdaSpace{}%
\AgdaBound{y}\AgdaSymbol{)}\AgdaSpace{}%
\AgdaSymbol{(}\AgdaBound{f}\AgdaSpace{}%
\AgdaBound{z}\AgdaSymbol{)}\AgdaSpace{}%
\AgdaBound{x₁}\AgdaSpace{}%
\AgdaBound{x₂}\AgdaSpace{}%
\AgdaSymbol{\}}\<%
\end{code}
%% END: \subsubsection{Equivalence Relation
Types}\label{equivalence-relation-types}

This section presents the {[}UALib.Relations.Equivalences{]}{[}{]}
module of the {[}Agda Universal Algebra Library{]}{[}{]}.

This is all pretty standard. The notions of reflexivity, symmetry, and
transitivity are defined as one would expect, so we need not dwell on
them.

\begin{code}%
\>[0]\<%
\\
\>[0]\AgdaSymbol{\{-\#}\AgdaSpace{}%
\AgdaKeyword{OPTIONS}\AgdaSpace{}%
\AgdaPragma{--without-K}\AgdaSpace{}%
\AgdaPragma{--exact-split}\AgdaSpace{}%
\AgdaPragma{--safe}\AgdaSpace{}%
\AgdaSymbol{\#-\}}\<%
\\
%
\\[\AgdaEmptyExtraSkip]%
\>[0]\AgdaKeyword{module}\AgdaSpace{}%
\AgdaModule{UALib.Relations.Equivalences}\AgdaSpace{}%
\AgdaKeyword{where}\<%
\\
%
\\[\AgdaEmptyExtraSkip]%
\>[0]\AgdaKeyword{open}\AgdaSpace{}%
\AgdaKeyword{import}\AgdaSpace{}%
\AgdaModule{UALib.Relations.Binary}\AgdaSpace{}%
\AgdaKeyword{public}\<%
\\
%
\\[\AgdaEmptyExtraSkip]%
\>[0]\AgdaKeyword{module}\AgdaSpace{}%
\AgdaModule{\AgdaUnderscore{}}\AgdaSpace{}%
\AgdaSymbol{\{}\AgdaBound{𝓤}\AgdaSpace{}%
\AgdaBound{𝓡}\AgdaSpace{}%
\AgdaSymbol{:}\AgdaSpace{}%
\AgdaPostulate{Universe}\AgdaSymbol{\}}\AgdaSpace{}%
\AgdaKeyword{where}\<%
\\
%
\\[\AgdaEmptyExtraSkip]%
\>[0][@{}l@{\AgdaIndent{0}}]%
\>[1]\AgdaKeyword{record}\AgdaSpace{}%
\AgdaRecord{IsEquivalence}\AgdaSpace{}%
\AgdaSymbol{\{}\AgdaBound{A}\AgdaSpace{}%
\AgdaSymbol{:}\AgdaSpace{}%
\AgdaBound{𝓤}\AgdaSpace{}%
\AgdaOperator{\AgdaFunction{̇}}\AgdaSpace{}%
\AgdaSymbol{\}}\AgdaSpace{}%
\AgdaSymbol{(}\AgdaOperator{\AgdaBound{\AgdaUnderscore{}≈\AgdaUnderscore{}}}\AgdaSpace{}%
\AgdaSymbol{:}\AgdaSpace{}%
\AgdaFunction{Rel}\AgdaSpace{}%
\AgdaBound{A}\AgdaSpace{}%
\AgdaBound{𝓡}\AgdaSymbol{)}\AgdaSpace{}%
\AgdaSymbol{:}\AgdaSpace{}%
\AgdaBound{𝓤}\AgdaSpace{}%
\AgdaOperator{\AgdaPrimitive{⊔}}\AgdaSpace{}%
\AgdaBound{𝓡}\AgdaSpace{}%
\AgdaOperator{\AgdaFunction{̇}}\AgdaSpace{}%
\AgdaKeyword{where}\<%
\\
\>[1][@{}l@{\AgdaIndent{0}}]%
\>[2]\AgdaKeyword{field}\<%
\\
\>[2][@{}l@{\AgdaIndent{0}}]%
\>[3]\AgdaField{rfl}%
\>[9]\AgdaSymbol{:}\AgdaSpace{}%
\AgdaFunction{reflexive}\AgdaSpace{}%
\AgdaOperator{\AgdaBound{\AgdaUnderscore{}≈\AgdaUnderscore{}}}\<%
\\
%
\>[3]\AgdaField{sym}%
\>[9]\AgdaSymbol{:}\AgdaSpace{}%
\AgdaFunction{symmetric}\AgdaSpace{}%
\AgdaOperator{\AgdaBound{\AgdaUnderscore{}≈\AgdaUnderscore{}}}\<%
\\
%
\>[3]\AgdaField{trans}\AgdaSpace{}%
\AgdaSymbol{:}\AgdaSpace{}%
\AgdaFunction{transitive}\AgdaSpace{}%
\AgdaOperator{\AgdaBound{\AgdaUnderscore{}≈\AgdaUnderscore{}}}\<%
\\
%
\\[\AgdaEmptyExtraSkip]%
%
\>[1]\AgdaFunction{is-equivalence-relation}\AgdaSpace{}%
\AgdaSymbol{:}\AgdaSpace{}%
\AgdaSymbol{\{}\AgdaBound{X}\AgdaSpace{}%
\AgdaSymbol{:}\AgdaSpace{}%
\AgdaBound{𝓤}\AgdaSpace{}%
\AgdaOperator{\AgdaFunction{̇}}\AgdaSpace{}%
\AgdaSymbol{\}}\AgdaSpace{}%
\AgdaSymbol{→}\AgdaSpace{}%
\AgdaFunction{Rel}\AgdaSpace{}%
\AgdaBound{X}\AgdaSpace{}%
\AgdaBound{𝓡}\AgdaSpace{}%
\AgdaSymbol{→}\AgdaSpace{}%
\AgdaBound{𝓤}\AgdaSpace{}%
\AgdaOperator{\AgdaPrimitive{⊔}}\AgdaSpace{}%
\AgdaBound{𝓡}\AgdaSpace{}%
\AgdaOperator{\AgdaFunction{̇}}\<%
\\
%
\>[1]\AgdaFunction{is-equivalence-relation}\AgdaSpace{}%
\AgdaOperator{\AgdaBound{\AgdaUnderscore{}≈\AgdaUnderscore{}}}\AgdaSpace{}%
\AgdaSymbol{=}%
\>[57I]\AgdaFunction{is-subsingleton-valued}\AgdaSpace{}%
\AgdaOperator{\AgdaBound{\AgdaUnderscore{}≈\AgdaUnderscore{}}}\<%
\\
\>[.][@{}l@{}]\<[57I]%
\>[31]\AgdaOperator{\AgdaFunction{×}}\AgdaSpace{}%
\AgdaFunction{reflexive}\AgdaSpace{}%
\AgdaOperator{\AgdaBound{\AgdaUnderscore{}≈\AgdaUnderscore{}}}\AgdaSpace{}%
\AgdaOperator{\AgdaFunction{×}}\AgdaSpace{}%
\AgdaFunction{symmetric}\AgdaSpace{}%
\AgdaOperator{\AgdaBound{\AgdaUnderscore{}≈\AgdaUnderscore{}}}\AgdaSpace{}%
\AgdaOperator{\AgdaFunction{×}}\AgdaSpace{}%
\AgdaFunction{transitive}\AgdaSpace{}%
\AgdaOperator{\AgdaBound{\AgdaUnderscore{}≈\AgdaUnderscore{}}}\<%
\\
\>[0]\<%
\end{code}

\paragraph{Examples}\label{examples}

The zero relation 𝟎 is equivalent to the identity relation \texttt{≡}
and, of course, these are both equivalence relations. (In fact, we
already proved reflexivity, symmetry, and transitivity of \texttt{≡} in
the {[}UALib.Prelude.Equality{]}{[}{]} module, so we simply apply the
corresponding proofs where appropriate.)

\begin{code}%
\>[0]\<%
\\
\>[0]\AgdaKeyword{module}\AgdaSpace{}%
\AgdaModule{\AgdaUnderscore{}}\AgdaSpace{}%
\AgdaSymbol{\{}\AgdaBound{𝓤}\AgdaSpace{}%
\AgdaSymbol{:}\AgdaSpace{}%
\AgdaPostulate{Universe}\AgdaSymbol{\}}\AgdaSpace{}%
\AgdaKeyword{where}\<%
\\
%
\\[\AgdaEmptyExtraSkip]%
\>[0][@{}l@{\AgdaIndent{0}}]%
\>[1]\AgdaFunction{𝟎-IsEquivalence}\AgdaSpace{}%
\AgdaSymbol{:}\AgdaSpace{}%
\AgdaSymbol{\{}\AgdaBound{A}\AgdaSpace{}%
\AgdaSymbol{:}\AgdaSpace{}%
\AgdaBound{𝓤}\AgdaSpace{}%
\AgdaOperator{\AgdaFunction{̇}}\AgdaSpace{}%
\AgdaSymbol{\}}\AgdaSpace{}%
\AgdaSymbol{→}\AgdaSpace{}%
\AgdaRecord{IsEquivalence}\AgdaSymbol{\{}\AgdaBound{𝓤}\AgdaSymbol{\}\{}\AgdaArgument{A}\AgdaSpace{}%
\AgdaSymbol{=}\AgdaSpace{}%
\AgdaBound{A}\AgdaSymbol{\}}\AgdaSpace{}%
\AgdaFunction{𝟎-rel}\<%
\\
%
\>[1]\AgdaFunction{𝟎-IsEquivalence}\AgdaSpace{}%
\AgdaSymbol{=}\AgdaSpace{}%
\AgdaKeyword{record}\AgdaSpace{}%
\AgdaSymbol{\{}\AgdaSpace{}%
\AgdaField{rfl}\AgdaSpace{}%
\AgdaSymbol{=}\AgdaSpace{}%
\AgdaFunction{≡-rfl}\AgdaSymbol{;}\AgdaSpace{}%
\AgdaField{sym}\AgdaSpace{}%
\AgdaSymbol{=}\AgdaSpace{}%
\AgdaFunction{≡-sym}\AgdaSymbol{;}\AgdaSpace{}%
\AgdaField{trans}\AgdaSpace{}%
\AgdaSymbol{=}\AgdaSpace{}%
\AgdaFunction{≡-trans}\AgdaSpace{}%
\AgdaSymbol{\}}\<%
\\
%
\\[\AgdaEmptyExtraSkip]%
%
\>[1]\AgdaFunction{≡-IsEquivalence}\AgdaSpace{}%
\AgdaSymbol{:}\AgdaSpace{}%
\AgdaSymbol{\{}\AgdaBound{A}\AgdaSpace{}%
\AgdaSymbol{:}\AgdaSpace{}%
\AgdaBound{𝓤}\AgdaSpace{}%
\AgdaOperator{\AgdaFunction{̇}}\AgdaSymbol{\}}\AgdaSpace{}%
\AgdaSymbol{→}\AgdaSpace{}%
\AgdaRecord{IsEquivalence}\AgdaSymbol{\{}\AgdaBound{𝓤}\AgdaSymbol{\}\{}\AgdaArgument{A}\AgdaSpace{}%
\AgdaSymbol{=}\AgdaSpace{}%
\AgdaBound{A}\AgdaSymbol{\}}\AgdaSpace{}%
\AgdaOperator{\AgdaDatatype{\AgdaUnderscore{}≡\AgdaUnderscore{}}}\<%
\\
%
\>[1]\AgdaFunction{≡-IsEquivalence}\AgdaSpace{}%
\AgdaSymbol{=}\AgdaSpace{}%
\AgdaKeyword{record}\AgdaSpace{}%
\AgdaSymbol{\{}\AgdaSpace{}%
\AgdaField{rfl}\AgdaSpace{}%
\AgdaSymbol{=}\AgdaSpace{}%
\AgdaFunction{≡-rfl}\AgdaSpace{}%
\AgdaSymbol{;}\AgdaSpace{}%
\AgdaField{sym}\AgdaSpace{}%
\AgdaSymbol{=}\AgdaSpace{}%
\AgdaFunction{≡-sym}\AgdaSpace{}%
\AgdaSymbol{;}\AgdaSpace{}%
\AgdaField{trans}\AgdaSpace{}%
\AgdaSymbol{=}\AgdaSpace{}%
\AgdaFunction{≡-trans}\AgdaSpace{}%
\AgdaSymbol{\}}\<%
\\
\>[0]\<%
\end{code}

Finally, we should have at our disposal a proof of the fact that the
kernel of a function is an equivalence relation.

\begin{code}%
\>[0]\<%
\\
\>[0][@{}l@{\AgdaIndent{1}}]%
\>[1]\AgdaFunction{map-kernel-IsEquivalence}\AgdaSpace{}%
\AgdaSymbol{:}%
\>[122I]\AgdaSymbol{\{}\AgdaBound{𝓦}\AgdaSpace{}%
\AgdaSymbol{:}\AgdaSpace{}%
\AgdaPostulate{Universe}\AgdaSymbol{\}\{}\AgdaBound{A}\AgdaSpace{}%
\AgdaSymbol{:}\AgdaSpace{}%
\AgdaBound{𝓤}\AgdaSpace{}%
\AgdaOperator{\AgdaFunction{̇}}\AgdaSymbol{\}\{}\AgdaBound{B}\AgdaSpace{}%
\AgdaSymbol{:}\AgdaSpace{}%
\AgdaBound{𝓦}\AgdaSpace{}%
\AgdaOperator{\AgdaFunction{̇}}\AgdaSymbol{\}}\<%
\\
\>[.][@{}l@{}]\<[122I]%
\>[28]\AgdaSymbol{(}\AgdaBound{f}\AgdaSpace{}%
\AgdaSymbol{:}\AgdaSpace{}%
\AgdaBound{A}\AgdaSpace{}%
\AgdaSymbol{→}\AgdaSpace{}%
\AgdaBound{B}\AgdaSymbol{)}\AgdaSpace{}%
\AgdaSymbol{→}\AgdaSpace{}%
\AgdaRecord{IsEquivalence}\AgdaSpace{}%
\AgdaSymbol{(}\AgdaFunction{KER-rel}\AgdaSpace{}%
\AgdaBound{f}\AgdaSymbol{)}\<%
\\
%
\\[\AgdaEmptyExtraSkip]%
%
\>[1]\AgdaFunction{map-kernel-IsEquivalence}\AgdaSpace{}%
\AgdaSymbol{\{}\AgdaBound{𝓦}\AgdaSymbol{\}}\AgdaSpace{}%
\AgdaBound{f}\AgdaSpace{}%
\AgdaSymbol{=}\<%
\\
\>[1][@{}l@{\AgdaIndent{0}}]%
\>[2]\AgdaKeyword{record}%
\>[142I]\AgdaSymbol{\{}\AgdaSpace{}%
\AgdaField{rfl}\AgdaSpace{}%
\AgdaSymbol{=}\AgdaSpace{}%
\AgdaSymbol{λ}\AgdaSpace{}%
\AgdaBound{x}\AgdaSpace{}%
\AgdaSymbol{→}\AgdaSpace{}%
\AgdaInductiveConstructor{𝓇ℯ𝒻𝓁}\<%
\\
\>[.][@{}l@{}]\<[142I]%
\>[9]\AgdaSymbol{;}\AgdaSpace{}%
\AgdaField{sym}\AgdaSpace{}%
\AgdaSymbol{=}\AgdaSpace{}%
\AgdaSymbol{λ}\AgdaSpace{}%
\AgdaBound{x}\AgdaSpace{}%
\AgdaBound{y}\AgdaSpace{}%
\AgdaBound{x₁}\AgdaSpace{}%
\AgdaSymbol{→}\AgdaSpace{}%
\AgdaFunction{≡-sym}\AgdaSymbol{\{}\AgdaBound{𝓦}\AgdaSymbol{\}}\AgdaSpace{}%
\AgdaSymbol{(}\AgdaBound{f}\AgdaSpace{}%
\AgdaBound{x}\AgdaSymbol{)}\AgdaSpace{}%
\AgdaSymbol{(}\AgdaBound{f}\AgdaSpace{}%
\AgdaBound{y}\AgdaSymbol{)}\AgdaSpace{}%
\AgdaBound{x₁}\<%
\\
%
\>[9]\AgdaSymbol{;}\AgdaSpace{}%
\AgdaField{trans}\AgdaSpace{}%
\AgdaSymbol{=}\AgdaSpace{}%
\AgdaSymbol{λ}\AgdaSpace{}%
\AgdaBound{x}\AgdaSpace{}%
\AgdaBound{y}\AgdaSpace{}%
\AgdaBound{z}\AgdaSpace{}%
\AgdaBound{x₁}\AgdaSpace{}%
\AgdaBound{x₂}\AgdaSpace{}%
\AgdaSymbol{→}\AgdaSpace{}%
\AgdaFunction{≡-trans}\AgdaSpace{}%
\AgdaSymbol{(}\AgdaBound{f}\AgdaSpace{}%
\AgdaBound{x}\AgdaSymbol{)}\AgdaSpace{}%
\AgdaSymbol{(}\AgdaBound{f}\AgdaSpace{}%
\AgdaBound{y}\AgdaSymbol{)}\AgdaSpace{}%
\AgdaSymbol{(}\AgdaBound{f}\AgdaSpace{}%
\AgdaBound{z}\AgdaSymbol{)}\AgdaSpace{}%
\AgdaBound{x₁}\AgdaSpace{}%
\AgdaBound{x₂}\AgdaSpace{}%
\AgdaSymbol{\}}\<%
\\
\>[0]\<%
\end{code}

\begin{center}\rule{0.5\linewidth}{\linethickness}\end{center}

\href{UALib.Relations.Binary.html}{← UALib.Relations.Binary}
{\href{UALib.Relations.Equivalences.html}{UALib.Relations.Equivalences
→}}

\{\% include UALib.Links.md \%\}









\subsection{Quotient Types}\label{ssec:quotient-types}
Here we is section describes some key definitions and theorems from the \ualibQuotients module of the \agdaualib.
%% -- BEGIN: \begin{code}%
\>[0]\<%
\\
\>[0]\AgdaSymbol{\{-\#}\AgdaSpace{}%
\AgdaKeyword{OPTIONS}\AgdaSpace{}%
\AgdaPragma{--without-K}\AgdaSpace{}%
\AgdaPragma{--exact-split}\AgdaSpace{}%
\AgdaPragma{--safe}\AgdaSpace{}%
\AgdaSymbol{\#-\}}\<%
\\
%
\\[\AgdaEmptyExtraSkip]%
\>[0]\AgdaKeyword{module}\AgdaSpace{}%
\AgdaModule{UALib.Relations.Quotients}\AgdaSpace{}%
\AgdaKeyword{where}\<%
\\
%
\\[\AgdaEmptyExtraSkip]%
\>[0]\AgdaKeyword{open}\AgdaSpace{}%
\AgdaKeyword{import}\AgdaSpace{}%
\AgdaModule{UALib.Relations.Equivalences}\AgdaSpace{}%
\AgdaKeyword{public}\<%
\\
\>[0]\AgdaKeyword{open}\AgdaSpace{}%
\AgdaKeyword{import}\AgdaSpace{}%
\AgdaModule{UALib.Prelude.Preliminaries}\AgdaSpace{}%
\AgdaKeyword{using}\AgdaSpace{}%
\AgdaSymbol{(}\AgdaOperator{\AgdaFunction{\AgdaUnderscore{}⇔\AgdaUnderscore{}}}\AgdaSymbol{;}\AgdaSpace{}%
\AgdaFunction{id}\AgdaSymbol{)}\AgdaSpace{}%
\AgdaKeyword{public}\<%
\\
%
\\[\AgdaEmptyExtraSkip]%
\>[0]\AgdaKeyword{module}\AgdaSpace{}%
\AgdaModule{\AgdaUnderscore{}}\AgdaSpace{}%
\AgdaSymbol{\{}\AgdaBound{𝓤}\AgdaSpace{}%
\AgdaBound{𝓡}\AgdaSpace{}%
\AgdaSymbol{:}\AgdaSpace{}%
\AgdaPostulate{Universe}\AgdaSymbol{\}}\AgdaSpace{}%
\AgdaKeyword{where}\<%
\\
\>[0]\<%
\end{code}

For a binary relation \texttt{R} on \texttt{A}, we denote a single
R-class as \texttt{{[}\ a\ {]}\ R} (the class containing \texttt{a}).
This notation is defined in UALib as follows.

\begin{code}%
\>[0]\<%
\\
\>[0]\AgdaComment{-- relation class}\<%
\\
\>[0][@{}l@{\AgdaIndent{0}}]%
\>[1]\AgdaOperator{\AgdaFunction{[\AgdaUnderscore{}]}}\AgdaSpace{}%
\AgdaSymbol{:}\AgdaSpace{}%
\AgdaSymbol{\{}\AgdaBound{A}\AgdaSpace{}%
\AgdaSymbol{:}\AgdaSpace{}%
\AgdaBound{𝓤}\AgdaSpace{}%
\AgdaOperator{\AgdaFunction{̇}}\AgdaSpace{}%
\AgdaSymbol{\}}\AgdaSpace{}%
\AgdaSymbol{→}\AgdaSpace{}%
\AgdaBound{A}\AgdaSpace{}%
\AgdaSymbol{→}\AgdaSpace{}%
\AgdaFunction{Rel}\AgdaSpace{}%
\AgdaBound{A}\AgdaSpace{}%
\AgdaBound{𝓡}\AgdaSpace{}%
\AgdaSymbol{→}\AgdaSpace{}%
\AgdaFunction{Pred}\AgdaSpace{}%
\AgdaBound{A}\AgdaSpace{}%
\AgdaBound{𝓡}\<%
\\
%
\>[1]\AgdaOperator{\AgdaFunction{[}}\AgdaSpace{}%
\AgdaBound{a}\AgdaSpace{}%
\AgdaOperator{\AgdaFunction{]}}\AgdaSpace{}%
\AgdaBound{R}\AgdaSpace{}%
\AgdaSymbol{=}\AgdaSpace{}%
\AgdaSymbol{λ}\AgdaSpace{}%
\AgdaBound{x}\AgdaSpace{}%
\AgdaSymbol{→}\AgdaSpace{}%
\AgdaBound{R}\AgdaSpace{}%
\AgdaBound{a}\AgdaSpace{}%
\AgdaBound{x}\<%
\\
\>[0]\<%
\end{code}

So, \texttt{x\ ∈\ {[}\ a\ {]}\ R} iff \texttt{R\ a\ x}, and the
following elimination rule is a tautology.

\begin{code}%
\>[0]\<%
\\
\>[0][@{}l@{\AgdaIndent{1}}]%
\>[1]\AgdaFunction{[]-elim}\AgdaSpace{}%
\AgdaSymbol{:}\AgdaSpace{}%
\AgdaSymbol{\{}\AgdaBound{A}\AgdaSpace{}%
\AgdaSymbol{:}\AgdaSpace{}%
\AgdaBound{𝓤}\AgdaSpace{}%
\AgdaOperator{\AgdaFunction{̇}}\AgdaSpace{}%
\AgdaSymbol{\}\{}\AgdaBound{a}\AgdaSpace{}%
\AgdaBound{x}\AgdaSpace{}%
\AgdaSymbol{:}\AgdaSpace{}%
\AgdaBound{A}\AgdaSymbol{\}\{}\AgdaBound{R}\AgdaSpace{}%
\AgdaSymbol{:}\AgdaSpace{}%
\AgdaFunction{Rel}\AgdaSpace{}%
\AgdaBound{A}\AgdaSpace{}%
\AgdaBound{𝓡}\AgdaSymbol{\}}\<%
\\
\>[1][@{}l@{\AgdaIndent{0}}]%
\>[2]\AgdaSymbol{→}%
\>[12]\AgdaBound{R}\AgdaSpace{}%
\AgdaBound{a}\AgdaSpace{}%
\AgdaBound{x}\AgdaSpace{}%
\AgdaOperator{\AgdaFunction{⇔}}\AgdaSpace{}%
\AgdaSymbol{(}\AgdaBound{x}\AgdaSpace{}%
\AgdaOperator{\AgdaFunction{∈}}\AgdaSpace{}%
\AgdaOperator{\AgdaFunction{[}}\AgdaSpace{}%
\AgdaBound{a}\AgdaSpace{}%
\AgdaOperator{\AgdaFunction{]}}\AgdaSpace{}%
\AgdaBound{R}\AgdaSymbol{)}\<%
\\
%
\>[1]\AgdaFunction{[]-elim}\AgdaSpace{}%
\AgdaSymbol{=}\AgdaSpace{}%
\AgdaFunction{id}\AgdaSpace{}%
\AgdaOperator{\AgdaInductiveConstructor{,}}\AgdaSpace{}%
\AgdaFunction{id}\<%
\\
\>[0]\<%
\end{code}

We define type of all classes of a relation \texttt{R} as follows.

\begin{code}%
\>[0]\<%
\\
\>[0][@{}l@{\AgdaIndent{1}}]%
\>[1]\AgdaFunction{𝒞}\AgdaSpace{}%
\AgdaSymbol{:}\AgdaSpace{}%
\AgdaSymbol{\{}\AgdaBound{A}\AgdaSpace{}%
\AgdaSymbol{:}\AgdaSpace{}%
\AgdaBound{𝓤}\AgdaSpace{}%
\AgdaOperator{\AgdaFunction{̇}}\AgdaSymbol{\}\{}\AgdaBound{R}\AgdaSpace{}%
\AgdaSymbol{:}\AgdaSpace{}%
\AgdaFunction{Rel}\AgdaSpace{}%
\AgdaBound{A}\AgdaSpace{}%
\AgdaBound{𝓡}\AgdaSymbol{\}}\AgdaSpace{}%
\AgdaSymbol{→}\AgdaSpace{}%
\AgdaFunction{Pred}\AgdaSpace{}%
\AgdaBound{A}\AgdaSpace{}%
\AgdaBound{𝓡}\AgdaSpace{}%
\AgdaSymbol{→}\AgdaSpace{}%
\AgdaSymbol{(}\AgdaBound{𝓤}\AgdaSpace{}%
\AgdaOperator{\AgdaPrimitive{⊔}}\AgdaSpace{}%
\AgdaBound{𝓡}\AgdaSpace{}%
\AgdaOperator{\AgdaPrimitive{⁺}}\AgdaSymbol{)}\AgdaSpace{}%
\AgdaOperator{\AgdaFunction{̇}}\<%
\\
%
\>[1]\AgdaFunction{𝒞}\AgdaSpace{}%
\AgdaSymbol{\{}\AgdaBound{A}\AgdaSymbol{\}\{}\AgdaBound{R}\AgdaSymbol{\}}\AgdaSpace{}%
\AgdaSymbol{=}\AgdaSpace{}%
\AgdaSymbol{λ}\AgdaSpace{}%
\AgdaSymbol{(}\AgdaBound{C}\AgdaSpace{}%
\AgdaSymbol{:}\AgdaSpace{}%
\AgdaFunction{Pred}\AgdaSpace{}%
\AgdaBound{A}\AgdaSpace{}%
\AgdaBound{𝓡}\AgdaSymbol{)}\AgdaSpace{}%
\AgdaSymbol{→}\AgdaSpace{}%
\AgdaFunction{Σ}\AgdaSpace{}%
\AgdaBound{a}\AgdaSpace{}%
\AgdaFunction{꞉}\AgdaSpace{}%
\AgdaBound{A}\AgdaSpace{}%
\AgdaFunction{,}\AgdaSpace{}%
\AgdaBound{C}\AgdaSpace{}%
\AgdaOperator{\AgdaDatatype{≡}}\AgdaSpace{}%
\AgdaSymbol{(}\AgdaSpace{}%
\AgdaOperator{\AgdaFunction{[}}\AgdaSpace{}%
\AgdaBound{a}\AgdaSpace{}%
\AgdaOperator{\AgdaFunction{]}}\AgdaSpace{}%
\AgdaBound{R}\AgdaSymbol{)}\<%
\\
\>[0]\<%
\end{code}

There are a few ways we could define the quotient with respect to a
relation. We have found the following to be the most convenient.

\begin{code}%
\>[0]\<%
\\
\>[0][@{}l@{\AgdaIndent{1}}]%
\>[1]\AgdaComment{-- relation quotient (predicate version)}\<%
\\
%
\>[1]\AgdaOperator{\AgdaFunction{\AgdaUnderscore{}/\AgdaUnderscore{}}}\AgdaSpace{}%
\AgdaSymbol{:}\AgdaSpace{}%
\AgdaSymbol{(}\AgdaBound{A}\AgdaSpace{}%
\AgdaSymbol{:}\AgdaSpace{}%
\AgdaBound{𝓤}\AgdaSpace{}%
\AgdaOperator{\AgdaFunction{̇}}\AgdaSpace{}%
\AgdaSymbol{)}\AgdaSpace{}%
\AgdaSymbol{→}\AgdaSpace{}%
\AgdaFunction{Rel}\AgdaSpace{}%
\AgdaBound{A}\AgdaSpace{}%
\AgdaBound{𝓡}\AgdaSpace{}%
\AgdaSymbol{→}\AgdaSpace{}%
\AgdaBound{𝓤}\AgdaSpace{}%
\AgdaOperator{\AgdaPrimitive{⊔}}\AgdaSpace{}%
\AgdaSymbol{(}\AgdaBound{𝓡}\AgdaSpace{}%
\AgdaOperator{\AgdaPrimitive{⁺}}\AgdaSymbol{)}\AgdaSpace{}%
\AgdaOperator{\AgdaFunction{̇}}\<%
\\
%
\>[1]\AgdaBound{A}\AgdaSpace{}%
\AgdaOperator{\AgdaFunction{/}}\AgdaSpace{}%
\AgdaBound{R}\AgdaSpace{}%
\AgdaSymbol{=}\AgdaSpace{}%
\AgdaFunction{Σ}\AgdaSpace{}%
\AgdaBound{C}\AgdaSpace{}%
\AgdaFunction{꞉}\AgdaSpace{}%
\AgdaFunction{Pred}\AgdaSpace{}%
\AgdaBound{A}\AgdaSpace{}%
\AgdaBound{𝓡}\AgdaSpace{}%
\AgdaFunction{,}%
\>[27]\AgdaFunction{𝒞}\AgdaSymbol{\{}\AgdaBound{A}\AgdaSymbol{\}\{}\AgdaBound{R}\AgdaSymbol{\}}\AgdaSpace{}%
\AgdaBound{C}\<%
\\
%
\>[1]\AgdaComment{-- old version:  A / R = Σ C ꞉ Pred A 𝓡 ,  Σ a ꞉ A ,  C ≡ ( [ a ] R )}\<%
\\
\>[0]\<%
\end{code}

We then define the following introduction rule for a relation class with
designated representative.

\begin{code}%
\>[0]\<%
\\
\>[0][@{}l@{\AgdaIndent{1}}]%
\>[1]\AgdaOperator{\AgdaFunction{⟦\AgdaUnderscore{}⟧}}\AgdaSpace{}%
\AgdaSymbol{:}\AgdaSpace{}%
\AgdaSymbol{\{}\AgdaBound{A}\AgdaSpace{}%
\AgdaSymbol{:}\AgdaSpace{}%
\AgdaBound{𝓤}\AgdaSpace{}%
\AgdaOperator{\AgdaFunction{̇}}\AgdaSymbol{\}}\AgdaSpace{}%
\AgdaSymbol{→}\AgdaSpace{}%
\AgdaBound{A}\AgdaSpace{}%
\AgdaSymbol{→}\AgdaSpace{}%
\AgdaSymbol{\{}\AgdaBound{R}\AgdaSpace{}%
\AgdaSymbol{:}\AgdaSpace{}%
\AgdaFunction{Rel}\AgdaSpace{}%
\AgdaBound{A}\AgdaSpace{}%
\AgdaBound{𝓡}\AgdaSymbol{\}}\AgdaSpace{}%
\AgdaSymbol{→}\AgdaSpace{}%
\AgdaBound{A}\AgdaSpace{}%
\AgdaOperator{\AgdaFunction{/}}\AgdaSpace{}%
\AgdaBound{R}\<%
\\
%
\>[1]\AgdaOperator{\AgdaFunction{⟦}}\AgdaSpace{}%
\AgdaBound{a}\AgdaSpace{}%
\AgdaOperator{\AgdaFunction{⟧}}\AgdaSpace{}%
\AgdaSymbol{\{}\AgdaBound{R}\AgdaSymbol{\}}\AgdaSpace{}%
\AgdaSymbol{=}\AgdaSpace{}%
\AgdaSymbol{(}\AgdaOperator{\AgdaFunction{[}}\AgdaSpace{}%
\AgdaBound{a}\AgdaSpace{}%
\AgdaOperator{\AgdaFunction{]}}\AgdaSpace{}%
\AgdaBound{R}\AgdaSymbol{)}\AgdaSpace{}%
\AgdaOperator{\AgdaInductiveConstructor{,}}\AgdaSpace{}%
\AgdaBound{a}\AgdaSpace{}%
\AgdaOperator{\AgdaInductiveConstructor{,}}\AgdaSpace{}%
\AgdaInductiveConstructor{𝓇ℯ𝒻𝓁}\<%
\\
%
\\[\AgdaEmptyExtraSkip]%
%
\>[1]\AgdaComment{--So, x ∈ [ a ]ₚ R iff R a x, and the following elimination rule is a tautology.}\<%
\\
%
\>[1]\AgdaFunction{⟦⟧-elim}\AgdaSpace{}%
\AgdaSymbol{:}\AgdaSpace{}%
\AgdaSymbol{\{}\AgdaBound{A}\AgdaSpace{}%
\AgdaSymbol{:}\AgdaSpace{}%
\AgdaBound{𝓤}\AgdaSpace{}%
\AgdaOperator{\AgdaFunction{̇}}\AgdaSpace{}%
\AgdaSymbol{\}\{}\AgdaBound{a}\AgdaSpace{}%
\AgdaBound{x}\AgdaSpace{}%
\AgdaSymbol{:}\AgdaSpace{}%
\AgdaBound{A}\AgdaSymbol{\}\{}\AgdaBound{R}\AgdaSpace{}%
\AgdaSymbol{:}\AgdaSpace{}%
\AgdaFunction{Rel}\AgdaSpace{}%
\AgdaBound{A}\AgdaSpace{}%
\AgdaBound{𝓡}\AgdaSymbol{\}}\<%
\\
\>[1][@{}l@{\AgdaIndent{0}}]%
\>[2]\AgdaSymbol{→}%
\>[12]\AgdaBound{R}\AgdaSpace{}%
\AgdaBound{a}\AgdaSpace{}%
\AgdaBound{x}\AgdaSpace{}%
\AgdaOperator{\AgdaFunction{⇔}}\AgdaSpace{}%
\AgdaSymbol{(}\AgdaBound{x}\AgdaSpace{}%
\AgdaOperator{\AgdaFunction{∈}}\AgdaSpace{}%
\AgdaOperator{\AgdaFunction{[}}\AgdaSpace{}%
\AgdaBound{a}\AgdaSpace{}%
\AgdaOperator{\AgdaFunction{]}}\AgdaSpace{}%
\AgdaBound{R}\AgdaSymbol{)}\<%
\\
%
\>[1]\AgdaFunction{⟦⟧-elim}\AgdaSpace{}%
\AgdaSymbol{=}\AgdaSpace{}%
\AgdaFunction{id}\AgdaSpace{}%
\AgdaOperator{\AgdaInductiveConstructor{,}}\AgdaSpace{}%
\AgdaFunction{id}\<%
\\
\>[0]\<%
\end{code}

If the relation is reflexive, then we have the following elimination
rules.

\begin{code}%
\>[0]\<%
\\
\>[0][@{}l@{\AgdaIndent{1}}]%
\>[1]\AgdaFunction{/-refl}\AgdaSpace{}%
\AgdaSymbol{:}\AgdaSpace{}%
\AgdaSymbol{\{}\AgdaBound{A}\AgdaSpace{}%
\AgdaSymbol{:}\AgdaSpace{}%
\AgdaBound{𝓤}\AgdaSpace{}%
\AgdaOperator{\AgdaFunction{̇}}\AgdaSymbol{\}\{}\AgdaBound{a}\AgdaSpace{}%
\AgdaBound{a'}\AgdaSpace{}%
\AgdaSymbol{:}\AgdaSpace{}%
\AgdaBound{A}\AgdaSymbol{\}\{}\AgdaBound{R}\AgdaSpace{}%
\AgdaSymbol{:}\AgdaSpace{}%
\AgdaFunction{Rel}\AgdaSpace{}%
\AgdaBound{A}\AgdaSpace{}%
\AgdaBound{𝓡}\AgdaSymbol{\}}\<%
\\
\>[1][@{}l@{\AgdaIndent{0}}]%
\>[2]\AgdaSymbol{→}%
\>[6]\AgdaFunction{reflexive}\AgdaSpace{}%
\AgdaBound{R}\AgdaSpace{}%
\AgdaSymbol{→}\AgdaSpace{}%
\AgdaOperator{\AgdaFunction{[}}\AgdaSpace{}%
\AgdaBound{a}\AgdaSpace{}%
\AgdaOperator{\AgdaFunction{]}}\AgdaSpace{}%
\AgdaBound{R}\AgdaSpace{}%
\AgdaOperator{\AgdaDatatype{≡}}\AgdaSpace{}%
\AgdaOperator{\AgdaFunction{[}}\AgdaSpace{}%
\AgdaBound{a'}\AgdaSpace{}%
\AgdaOperator{\AgdaFunction{]}}\AgdaSpace{}%
\AgdaBound{R}\AgdaSpace{}%
\AgdaSymbol{→}\AgdaSpace{}%
\AgdaBound{R}\AgdaSpace{}%
\AgdaBound{a}\AgdaSpace{}%
\AgdaBound{a'}\<%
\\
%
\>[1]\AgdaFunction{/-refl}\AgdaSymbol{\{}\AgdaArgument{A}\AgdaSpace{}%
\AgdaSymbol{=}\AgdaSpace{}%
\AgdaBound{A}\AgdaSymbol{\}\{}\AgdaBound{a}\AgdaSymbol{\}\{}\AgdaBound{a'}\AgdaSymbol{\}\{}\AgdaBound{R}\AgdaSymbol{\}}\AgdaSpace{}%
\AgdaBound{rfl}\AgdaSpace{}%
\AgdaBound{x}%
\>[32]\AgdaSymbol{=}\AgdaSpace{}%
\AgdaFunction{γ}\<%
\\
\>[1][@{}l@{\AgdaIndent{0}}]%
\>[2]\AgdaKeyword{where}\<%
\\
\>[2][@{}l@{\AgdaIndent{0}}]%
\>[3]\AgdaFunction{a'in}\AgdaSpace{}%
\AgdaSymbol{:}\AgdaSpace{}%
\AgdaBound{a'}\AgdaSpace{}%
\AgdaOperator{\AgdaFunction{∈}}\AgdaSpace{}%
\AgdaOperator{\AgdaFunction{[}}\AgdaSpace{}%
\AgdaBound{a'}\AgdaSpace{}%
\AgdaOperator{\AgdaFunction{]}}\AgdaSpace{}%
\AgdaBound{R}\<%
\\
%
\>[3]\AgdaFunction{a'in}\AgdaSpace{}%
\AgdaSymbol{=}\AgdaSpace{}%
\AgdaBound{rfl}\AgdaSpace{}%
\AgdaBound{a'}\<%
\\
%
\>[3]\AgdaFunction{γ}\AgdaSpace{}%
\AgdaSymbol{:}\AgdaSpace{}%
\AgdaBound{a'}\AgdaSpace{}%
\AgdaOperator{\AgdaFunction{∈}}\AgdaSpace{}%
\AgdaOperator{\AgdaFunction{[}}\AgdaSpace{}%
\AgdaBound{a}\AgdaSpace{}%
\AgdaOperator{\AgdaFunction{]}}\AgdaSpace{}%
\AgdaBound{R}\<%
\\
%
\>[3]\AgdaFunction{γ}\AgdaSpace{}%
\AgdaSymbol{=}\AgdaSpace{}%
\AgdaFunction{cong-app-pred}\AgdaSpace{}%
\AgdaBound{a'}\AgdaSpace{}%
\AgdaFunction{a'in}\AgdaSpace{}%
\AgdaSymbol{(}\AgdaBound{x}\AgdaSpace{}%
\AgdaOperator{\AgdaFunction{⁻¹}}\AgdaSymbol{)}\<%
\\
%
\\[\AgdaEmptyExtraSkip]%
%
\>[1]\AgdaFunction{/-refl'}\AgdaSpace{}%
\AgdaSymbol{:}\AgdaSpace{}%
\AgdaSymbol{\{}\AgdaBound{A}\AgdaSpace{}%
\AgdaSymbol{:}\AgdaSpace{}%
\AgdaBound{𝓤}\AgdaSpace{}%
\AgdaOperator{\AgdaFunction{̇}}\AgdaSymbol{\}\{}\AgdaBound{a}\AgdaSpace{}%
\AgdaBound{a'}\AgdaSpace{}%
\AgdaSymbol{:}\AgdaSpace{}%
\AgdaBound{A}\AgdaSymbol{\}\{}\AgdaBound{R}\AgdaSpace{}%
\AgdaSymbol{:}\AgdaSpace{}%
\AgdaFunction{Rel}\AgdaSpace{}%
\AgdaBound{A}\AgdaSpace{}%
\AgdaBound{𝓡}\AgdaSymbol{\}}\<%
\\
\>[1][@{}l@{\AgdaIndent{0}}]%
\>[2]\AgdaSymbol{→}%
\>[6]\AgdaFunction{transitive}\AgdaSpace{}%
\AgdaBound{R}\AgdaSpace{}%
\AgdaSymbol{→}\AgdaSpace{}%
\AgdaBound{R}\AgdaSpace{}%
\AgdaBound{a'}\AgdaSpace{}%
\AgdaBound{a}\AgdaSpace{}%
\AgdaSymbol{→}\AgdaSpace{}%
\AgdaSymbol{(}\AgdaOperator{\AgdaFunction{[}}\AgdaSpace{}%
\AgdaBound{a}\AgdaSpace{}%
\AgdaOperator{\AgdaFunction{]}}\AgdaSpace{}%
\AgdaBound{R}\AgdaSymbol{)}\AgdaSpace{}%
\AgdaOperator{\AgdaFunction{⊆}}\AgdaSpace{}%
\AgdaSymbol{(}\AgdaOperator{\AgdaFunction{[}}\AgdaSpace{}%
\AgdaBound{a'}\AgdaSpace{}%
\AgdaOperator{\AgdaFunction{]}}\AgdaSpace{}%
\AgdaBound{R}\AgdaSymbol{)}\<%
\\
%
\>[1]\AgdaFunction{/-refl'}\AgdaSymbol{\{}\AgdaArgument{A}\AgdaSpace{}%
\AgdaSymbol{=}\AgdaSpace{}%
\AgdaBound{A}\AgdaSymbol{\}\{}\AgdaBound{a}\AgdaSymbol{\}\{}\AgdaBound{a'}\AgdaSymbol{\}\{}\AgdaBound{R}\AgdaSymbol{\}}\AgdaSpace{}%
\AgdaBound{trn}\AgdaSpace{}%
\AgdaBound{Ra'a}\AgdaSpace{}%
\AgdaSymbol{\{}\AgdaBound{x}\AgdaSymbol{\}}\AgdaSpace{}%
\AgdaBound{aRx}\AgdaSpace{}%
\AgdaSymbol{=}\AgdaSpace{}%
\AgdaBound{trn}\AgdaSpace{}%
\AgdaBound{a'}\AgdaSpace{}%
\AgdaBound{a}\AgdaSpace{}%
\AgdaBound{x}\AgdaSpace{}%
\AgdaBound{Ra'a}\AgdaSpace{}%
\AgdaBound{aRx}\<%
\\
%
\\[\AgdaEmptyExtraSkip]%
%
\>[1]\AgdaOperator{\AgdaFunction{⌜\AgdaUnderscore{}⌝}}\AgdaSpace{}%
\AgdaSymbol{:}\AgdaSpace{}%
\AgdaSymbol{\{}\AgdaBound{A}\AgdaSpace{}%
\AgdaSymbol{:}\AgdaSpace{}%
\AgdaBound{𝓤}\AgdaSpace{}%
\AgdaOperator{\AgdaFunction{̇}}\AgdaSymbol{\}\{}\AgdaBound{R}\AgdaSpace{}%
\AgdaSymbol{:}\AgdaSpace{}%
\AgdaFunction{Rel}\AgdaSpace{}%
\AgdaBound{A}\AgdaSpace{}%
\AgdaBound{𝓡}\AgdaSymbol{\}}\AgdaSpace{}%
\AgdaSymbol{→}\AgdaSpace{}%
\AgdaBound{A}\AgdaSpace{}%
\AgdaOperator{\AgdaFunction{/}}\AgdaSpace{}%
\AgdaBound{R}%
\>[39]\AgdaSymbol{→}\AgdaSpace{}%
\AgdaBound{A}\<%
\\
%
\>[1]\AgdaOperator{\AgdaFunction{⌜}}\AgdaSpace{}%
\AgdaBound{𝒂}\AgdaSpace{}%
\AgdaOperator{\AgdaFunction{⌝}}\AgdaSpace{}%
\AgdaSymbol{=}\AgdaSpace{}%
\AgdaOperator{\AgdaFunction{∣}}\AgdaSpace{}%
\AgdaOperator{\AgdaFunction{∥}}\AgdaSpace{}%
\AgdaBound{𝒂}\AgdaSpace{}%
\AgdaOperator{\AgdaFunction{∥}}\AgdaSpace{}%
\AgdaOperator{\AgdaFunction{∣}}%
\>[22]\AgdaComment{-- type ⌜ and ⌝ as `\textbackslash{}cul` and `\textbackslash{}cur`}\<%
\\
\>[0]\<%
\end{code}

and an elimination rule for relation class representative, defined as
follows.

\begin{code}%
\>[0]\<%
\\
\>[0][@{}l@{\AgdaIndent{1}}]%
\>[1]\AgdaFunction{/-Refl}\AgdaSpace{}%
\AgdaSymbol{:}\AgdaSpace{}%
\AgdaSymbol{\{}\AgdaBound{A}\AgdaSpace{}%
\AgdaSymbol{:}\AgdaSpace{}%
\AgdaBound{𝓤}\AgdaSpace{}%
\AgdaOperator{\AgdaFunction{̇}}\AgdaSymbol{\}\{}\AgdaBound{a}\AgdaSpace{}%
\AgdaBound{a'}\AgdaSpace{}%
\AgdaSymbol{:}\AgdaSpace{}%
\AgdaBound{A}\AgdaSymbol{\}\{}\AgdaBound{R}\AgdaSpace{}%
\AgdaSymbol{:}\AgdaSpace{}%
\AgdaFunction{Rel}\AgdaSpace{}%
\AgdaBound{A}\AgdaSpace{}%
\AgdaBound{𝓡}\AgdaSymbol{\}}\<%
\\
\>[1][@{}l@{\AgdaIndent{0}}]%
\>[2]\AgdaSymbol{→}%
\>[6]\AgdaFunction{reflexive}\AgdaSpace{}%
\AgdaBound{R}\AgdaSpace{}%
\AgdaSymbol{→}\AgdaSpace{}%
\AgdaOperator{\AgdaFunction{⟦}}\AgdaSpace{}%
\AgdaBound{a}\AgdaSpace{}%
\AgdaOperator{\AgdaFunction{⟧}}\AgdaSymbol{\{}\AgdaBound{R}\AgdaSymbol{\}}\AgdaSpace{}%
\AgdaOperator{\AgdaDatatype{≡}}\AgdaSpace{}%
\AgdaOperator{\AgdaFunction{⟦}}\AgdaSpace{}%
\AgdaBound{a'}\AgdaSpace{}%
\AgdaOperator{\AgdaFunction{⟧}}\AgdaSpace{}%
\AgdaSymbol{→}\AgdaSpace{}%
\AgdaBound{R}\AgdaSpace{}%
\AgdaBound{a}\AgdaSpace{}%
\AgdaBound{a'}\<%
\\
%
\>[1]\AgdaFunction{/-Refl}\AgdaSpace{}%
\AgdaBound{rfl}\AgdaSpace{}%
\AgdaSymbol{(}\AgdaInductiveConstructor{refl}\AgdaSpace{}%
\AgdaSymbol{\AgdaUnderscore{})}%
\>[22]\AgdaSymbol{=}\AgdaSpace{}%
\AgdaBound{rfl}\AgdaSpace{}%
\AgdaSymbol{\AgdaUnderscore{}}\<%
\\
\>[0]\<%
\end{code}

Later we will need the following additional quotient tools.

\begin{code}%
\>[0]\<%
\\
\>[0][@{}l@{\AgdaIndent{1}}]%
\>[1]\AgdaKeyword{open}\AgdaSpace{}%
\AgdaModule{IsEquivalence}\AgdaSymbol{\{}\AgdaBound{𝓤}\AgdaSymbol{\}\{}\AgdaBound{𝓡}\AgdaSymbol{\}}\<%
\\
%
\\[\AgdaEmptyExtraSkip]%
%
\>[1]\AgdaFunction{/-subset}\AgdaSpace{}%
\AgdaSymbol{:}\AgdaSpace{}%
\AgdaSymbol{\{}\AgdaBound{A}\AgdaSpace{}%
\AgdaSymbol{:}\AgdaSpace{}%
\AgdaBound{𝓤}\AgdaSpace{}%
\AgdaOperator{\AgdaFunction{̇}}\AgdaSymbol{\}\{}\AgdaBound{a}\AgdaSpace{}%
\AgdaBound{a'}\AgdaSpace{}%
\AgdaSymbol{:}\AgdaSpace{}%
\AgdaBound{A}\AgdaSymbol{\}\{}\AgdaBound{R}\AgdaSpace{}%
\AgdaSymbol{:}\AgdaSpace{}%
\AgdaFunction{Rel}\AgdaSpace{}%
\AgdaBound{A}\AgdaSpace{}%
\AgdaBound{𝓡}\AgdaSymbol{\}}\<%
\\
\>[1][@{}l@{\AgdaIndent{0}}]%
\>[2]\AgdaSymbol{→}%
\>[6]\AgdaRecord{IsEquivalence}\AgdaSpace{}%
\AgdaBound{R}\AgdaSpace{}%
\AgdaSymbol{→}\AgdaSpace{}%
\AgdaBound{R}\AgdaSpace{}%
\AgdaBound{a}\AgdaSpace{}%
\AgdaBound{a'}\AgdaSpace{}%
\AgdaSymbol{→}\AgdaSpace{}%
\AgdaSymbol{(}\AgdaOperator{\AgdaFunction{[}}\AgdaSpace{}%
\AgdaBound{a}\AgdaSpace{}%
\AgdaOperator{\AgdaFunction{]}}\AgdaSpace{}%
\AgdaBound{R}\AgdaSymbol{)}\AgdaSpace{}%
\AgdaOperator{\AgdaFunction{⊆}}\AgdaSpace{}%
\AgdaSymbol{(}\AgdaOperator{\AgdaFunction{[}}\AgdaSpace{}%
\AgdaBound{a'}\AgdaSpace{}%
\AgdaOperator{\AgdaFunction{]}}\AgdaSpace{}%
\AgdaBound{R}\AgdaSymbol{)}\<%
\\
%
\>[1]\AgdaFunction{/-subset}\AgdaSpace{}%
\AgdaSymbol{\{}\AgdaArgument{A}\AgdaSpace{}%
\AgdaSymbol{=}\AgdaSpace{}%
\AgdaBound{A}\AgdaSymbol{\}\{}\AgdaBound{a}\AgdaSymbol{\}\{}\AgdaBound{a'}\AgdaSymbol{\}\{}\AgdaBound{R}\AgdaSymbol{\}}\AgdaSpace{}%
\AgdaBound{Req}\AgdaSpace{}%
\AgdaBound{Raa'}\AgdaSpace{}%
\AgdaSymbol{\{}\AgdaBound{x}\AgdaSymbol{\}}\AgdaSpace{}%
\AgdaBound{Rax}\AgdaSpace{}%
\AgdaSymbol{=}\AgdaSpace{}%
\AgdaSymbol{(}\AgdaField{trans}\AgdaSpace{}%
\AgdaBound{Req}\AgdaSymbol{)}\AgdaSpace{}%
\AgdaBound{a'}\AgdaSpace{}%
\AgdaBound{a}\AgdaSpace{}%
\AgdaBound{x}\AgdaSpace{}%
\AgdaSymbol{(}\AgdaField{sym}\AgdaSpace{}%
\AgdaBound{Req}\AgdaSpace{}%
\AgdaBound{a}\AgdaSpace{}%
\AgdaBound{a'}\AgdaSpace{}%
\AgdaBound{Raa'}\AgdaSymbol{)}\AgdaSpace{}%
\AgdaBound{Rax}\<%
\\
%
\\[\AgdaEmptyExtraSkip]%
%
\>[1]\AgdaFunction{/-supset}\AgdaSpace{}%
\AgdaSymbol{:}\AgdaSpace{}%
\AgdaSymbol{\{}\AgdaBound{A}\AgdaSpace{}%
\AgdaSymbol{:}\AgdaSpace{}%
\AgdaBound{𝓤}\AgdaSpace{}%
\AgdaOperator{\AgdaFunction{̇}}\AgdaSymbol{\}\{}\AgdaBound{a}\AgdaSpace{}%
\AgdaBound{a'}\AgdaSpace{}%
\AgdaSymbol{:}\AgdaSpace{}%
\AgdaBound{A}\AgdaSymbol{\}\{}\AgdaBound{R}\AgdaSpace{}%
\AgdaSymbol{:}\AgdaSpace{}%
\AgdaFunction{Rel}\AgdaSpace{}%
\AgdaBound{A}\AgdaSpace{}%
\AgdaBound{𝓡}\AgdaSymbol{\}}\<%
\\
\>[1][@{}l@{\AgdaIndent{0}}]%
\>[2]\AgdaSymbol{→}%
\>[6]\AgdaRecord{IsEquivalence}\AgdaSpace{}%
\AgdaBound{R}\AgdaSpace{}%
\AgdaSymbol{→}\AgdaSpace{}%
\AgdaBound{R}\AgdaSpace{}%
\AgdaBound{a}\AgdaSpace{}%
\AgdaBound{a'}\AgdaSpace{}%
\AgdaSymbol{→}\AgdaSpace{}%
\AgdaSymbol{(}\AgdaOperator{\AgdaFunction{[}}\AgdaSpace{}%
\AgdaBound{a}\AgdaSpace{}%
\AgdaOperator{\AgdaFunction{]}}\AgdaSpace{}%
\AgdaBound{R}\AgdaSymbol{)}\AgdaSpace{}%
\AgdaOperator{\AgdaFunction{⊇}}\AgdaSpace{}%
\AgdaSymbol{(}\AgdaOperator{\AgdaFunction{[}}\AgdaSpace{}%
\AgdaBound{a'}\AgdaSpace{}%
\AgdaOperator{\AgdaFunction{]}}\AgdaSpace{}%
\AgdaBound{R}\AgdaSymbol{)}\<%
\\
%
\>[1]\AgdaFunction{/-supset}\AgdaSpace{}%
\AgdaSymbol{\{}\AgdaArgument{A}\AgdaSpace{}%
\AgdaSymbol{=}\AgdaSpace{}%
\AgdaBound{A}\AgdaSymbol{\}\{}\AgdaBound{a}\AgdaSymbol{\}\{}\AgdaBound{a'}\AgdaSymbol{\}\{}\AgdaBound{R}\AgdaSymbol{\}}\AgdaSpace{}%
\AgdaBound{Req}\AgdaSpace{}%
\AgdaBound{Raa'}\AgdaSpace{}%
\AgdaSymbol{\{}\AgdaBound{x}\AgdaSymbol{\}}\AgdaSpace{}%
\AgdaBound{Ra'x}\AgdaSpace{}%
\AgdaSymbol{=}\AgdaSpace{}%
\AgdaSymbol{(}\AgdaField{trans}\AgdaSpace{}%
\AgdaBound{Req}\AgdaSymbol{)}\AgdaSpace{}%
\AgdaBound{a}\AgdaSpace{}%
\AgdaBound{a'}\AgdaSpace{}%
\AgdaBound{x}\AgdaSpace{}%
\AgdaBound{Raa'}\AgdaSpace{}%
\AgdaBound{Ra'x}\<%
\\
%
\\[\AgdaEmptyExtraSkip]%
%
\>[1]\AgdaFunction{/-=̇}\AgdaSpace{}%
\AgdaSymbol{:}\AgdaSpace{}%
\AgdaSymbol{\{}\AgdaBound{A}\AgdaSpace{}%
\AgdaSymbol{:}\AgdaSpace{}%
\AgdaBound{𝓤}\AgdaSpace{}%
\AgdaOperator{\AgdaFunction{̇}}\AgdaSymbol{\}\{}\AgdaBound{a}\AgdaSpace{}%
\AgdaBound{a'}\AgdaSpace{}%
\AgdaSymbol{:}\AgdaSpace{}%
\AgdaBound{A}\AgdaSymbol{\}\{}\AgdaBound{R}\AgdaSpace{}%
\AgdaSymbol{:}\AgdaSpace{}%
\AgdaFunction{Rel}\AgdaSpace{}%
\AgdaBound{A}\AgdaSpace{}%
\AgdaBound{𝓡}\AgdaSymbol{\}}\<%
\\
\>[1][@{}l@{\AgdaIndent{0}}]%
\>[2]\AgdaSymbol{→}%
\>[6]\AgdaRecord{IsEquivalence}\AgdaSpace{}%
\AgdaBound{R}\AgdaSpace{}%
\AgdaSymbol{→}\AgdaSpace{}%
\AgdaBound{R}\AgdaSpace{}%
\AgdaBound{a}\AgdaSpace{}%
\AgdaBound{a'}\AgdaSpace{}%
\AgdaSymbol{→}\AgdaSpace{}%
\AgdaSymbol{(}\AgdaOperator{\AgdaFunction{[}}\AgdaSpace{}%
\AgdaBound{a}\AgdaSpace{}%
\AgdaOperator{\AgdaFunction{]}}\AgdaSpace{}%
\AgdaBound{R}\AgdaSymbol{)}\AgdaSpace{}%
\AgdaOperator{\AgdaFunction{=̇}}\AgdaSpace{}%
\AgdaSymbol{(}\AgdaOperator{\AgdaFunction{[}}\AgdaSpace{}%
\AgdaBound{a'}\AgdaSpace{}%
\AgdaOperator{\AgdaFunction{]}}\AgdaSpace{}%
\AgdaBound{R}\AgdaSymbol{)}\<%
\\
%
\>[1]\AgdaFunction{/-=̇}\AgdaSpace{}%
\AgdaSymbol{\{}\AgdaArgument{A}\AgdaSpace{}%
\AgdaSymbol{=}\AgdaSpace{}%
\AgdaBound{A}\AgdaSymbol{\}\{}\AgdaBound{a}\AgdaSymbol{\}\{}\AgdaBound{a'}\AgdaSymbol{\}\{}\AgdaBound{R}\AgdaSymbol{\}}\AgdaSpace{}%
\AgdaBound{Req}\AgdaSpace{}%
\AgdaBound{Raa'}\AgdaSpace{}%
\AgdaSymbol{=}\AgdaSpace{}%
\AgdaFunction{/-subset}\AgdaSpace{}%
\AgdaBound{Req}\AgdaSpace{}%
\AgdaBound{Raa'}\AgdaSpace{}%
\AgdaOperator{\AgdaInductiveConstructor{,}}\AgdaSpace{}%
\AgdaFunction{/-supset}\AgdaSpace{}%
\AgdaBound{Req}\AgdaSpace{}%
\AgdaBound{Raa'}\<%
\\
\>[0]\<%
\end{code}

\subsubsection{Quotient extensionality}\label{quotient-extensionality}

We need a (subsingleton) identity type for congruence classes over sets
so that we can equate two classes even when they are presented using
different representatives. For this we assume that our relations are on
sets, rather than arbitrary types. As mentioned earlier, this is
equivalent to assuming that there is at most one proof that two elements
of a set are the same.

(Recall, a type is called a \textbf{set} if it has \emph{unique identity
proofs}; as a general principle, this is sometimes referred to as
``proof irrelevance'' or ``uniqueness of identity proofs''---two proofs
of a single identity are the same.)

\begin{code}%
\>[0]\<%
\\
\>[0][@{}l@{\AgdaIndent{1}}]%
\>[1]\AgdaFunction{class-extensionality}\AgdaSpace{}%
\AgdaSymbol{:}\AgdaSpace{}%
\AgdaFunction{propext}\AgdaSpace{}%
\AgdaBound{𝓡}\AgdaSpace{}%
\AgdaSymbol{→}\AgdaSpace{}%
\AgdaFunction{global-dfunext}\<%
\\
\>[1][@{}l@{\AgdaIndent{0}}]%
\>[2]\AgdaSymbol{→}%
\>[10]\AgdaSymbol{\{}\AgdaBound{A}\AgdaSpace{}%
\AgdaSymbol{:}\AgdaSpace{}%
\AgdaBound{𝓤}\AgdaSpace{}%
\AgdaOperator{\AgdaFunction{̇}}\AgdaSymbol{\}\{}\AgdaBound{a}\AgdaSpace{}%
\AgdaBound{a'}\AgdaSpace{}%
\AgdaSymbol{:}\AgdaSpace{}%
\AgdaBound{A}\AgdaSymbol{\}\{}\AgdaBound{R}\AgdaSpace{}%
\AgdaSymbol{:}\AgdaSpace{}%
\AgdaFunction{Rel}\AgdaSpace{}%
\AgdaBound{A}\AgdaSpace{}%
\AgdaBound{𝓡}\AgdaSymbol{\}}\<%
\\
%
\>[2]\AgdaSymbol{→}%
\>[10]\AgdaSymbol{(∀}\AgdaSpace{}%
\AgdaBound{a}\AgdaSpace{}%
\AgdaBound{x}\AgdaSpace{}%
\AgdaSymbol{→}\AgdaSpace{}%
\AgdaFunction{is-subsingleton}\AgdaSpace{}%
\AgdaSymbol{(}\AgdaBound{R}\AgdaSpace{}%
\AgdaBound{a}\AgdaSpace{}%
\AgdaBound{x}\AgdaSymbol{))}\<%
\\
%
\>[2]\AgdaSymbol{→}%
\>[10]\AgdaRecord{IsEquivalence}\AgdaSpace{}%
\AgdaBound{R}\<%
\\
\>[2][@{}l@{\AgdaIndent{0}}]%
\>[9]\AgdaComment{---------------------------------------}\<%
\\
%
\>[2]\AgdaSymbol{→}%
\>[11]\AgdaBound{R}\AgdaSpace{}%
\AgdaBound{a}\AgdaSpace{}%
\AgdaBound{a'}\AgdaSpace{}%
\AgdaSymbol{→}\AgdaSpace{}%
\AgdaSymbol{(}\AgdaOperator{\AgdaFunction{[}}\AgdaSpace{}%
\AgdaBound{a}\AgdaSpace{}%
\AgdaOperator{\AgdaFunction{]}}\AgdaSpace{}%
\AgdaBound{R}\AgdaSymbol{)}\AgdaSpace{}%
\AgdaOperator{\AgdaDatatype{≡}}\AgdaSpace{}%
\AgdaSymbol{(}\AgdaOperator{\AgdaFunction{[}}\AgdaSpace{}%
\AgdaBound{a'}\AgdaSpace{}%
\AgdaOperator{\AgdaFunction{]}}\AgdaSpace{}%
\AgdaBound{R}\AgdaSymbol{)}\<%
\\
%
\\[\AgdaEmptyExtraSkip]%
%
\>[1]\AgdaFunction{class-extensionality}\AgdaSpace{}%
\AgdaBound{pe}\AgdaSpace{}%
\AgdaBound{gfe}\AgdaSpace{}%
\AgdaSymbol{\{}\AgdaArgument{A}\AgdaSpace{}%
\AgdaSymbol{=}\AgdaSpace{}%
\AgdaBound{A}\AgdaSymbol{\}\{}\AgdaBound{a}\AgdaSymbol{\}\{}\AgdaBound{a'}\AgdaSymbol{\}\{}\AgdaBound{R}\AgdaSymbol{\}}\AgdaSpace{}%
\AgdaBound{ssR}\AgdaSpace{}%
\AgdaBound{Req}\AgdaSpace{}%
\AgdaBound{Raa'}\AgdaSpace{}%
\AgdaSymbol{=}\<%
\\
\>[1][@{}l@{\AgdaIndent{0}}]%
\>[2]\AgdaFunction{Pred-=̇-≡}\AgdaSpace{}%
\AgdaBound{pe}\AgdaSpace{}%
\AgdaBound{gfe}\AgdaSpace{}%
\AgdaSymbol{\{}\AgdaBound{A}\AgdaSymbol{\}\{}\AgdaOperator{\AgdaFunction{[}}\AgdaSpace{}%
\AgdaBound{a}\AgdaSpace{}%
\AgdaOperator{\AgdaFunction{]}}\AgdaSpace{}%
\AgdaBound{R}\AgdaSymbol{\}\{}\AgdaOperator{\AgdaFunction{[}}\AgdaSpace{}%
\AgdaBound{a'}\AgdaSpace{}%
\AgdaOperator{\AgdaFunction{]}}\AgdaSpace{}%
\AgdaBound{R}\AgdaSymbol{\}}\AgdaSpace{}%
\AgdaSymbol{(}\AgdaBound{ssR}\AgdaSpace{}%
\AgdaBound{a}\AgdaSymbol{)}\AgdaSpace{}%
\AgdaSymbol{(}\AgdaBound{ssR}\AgdaSpace{}%
\AgdaBound{a'}\AgdaSymbol{)}\AgdaSpace{}%
\AgdaSymbol{(}\AgdaFunction{/-=̇}\AgdaSpace{}%
\AgdaBound{Req}\AgdaSpace{}%
\AgdaBound{Raa'}\AgdaSymbol{)}\<%
\\
%
\\[\AgdaEmptyExtraSkip]%
%
\>[1]\AgdaFunction{to-subtype-⟦⟧}\AgdaSpace{}%
\AgdaSymbol{:}%
\>[533I]\AgdaSymbol{\{}\AgdaBound{A}\AgdaSpace{}%
\AgdaSymbol{:}\AgdaSpace{}%
\AgdaBound{𝓤}\AgdaSpace{}%
\AgdaOperator{\AgdaFunction{̇}}\AgdaSymbol{\}\{}\AgdaBound{R}\AgdaSpace{}%
\AgdaSymbol{:}\AgdaSpace{}%
\AgdaFunction{Rel}\AgdaSpace{}%
\AgdaBound{A}\AgdaSpace{}%
\AgdaBound{𝓡}\AgdaSymbol{\}\{}\AgdaBound{C}\AgdaSpace{}%
\AgdaBound{D}\AgdaSpace{}%
\AgdaSymbol{:}\AgdaSpace{}%
\AgdaFunction{Pred}\AgdaSpace{}%
\AgdaBound{A}\AgdaSpace{}%
\AgdaBound{𝓡}\AgdaSymbol{\}}\<%
\\
\>[.][@{}l@{}]\<[533I]%
\>[17]\AgdaSymbol{\{}\AgdaBound{c}\AgdaSpace{}%
\AgdaSymbol{:}\AgdaSpace{}%
\AgdaFunction{𝒞}\AgdaSpace{}%
\AgdaBound{C}\AgdaSymbol{\}\{}\AgdaBound{d}\AgdaSpace{}%
\AgdaSymbol{:}\AgdaSpace{}%
\AgdaFunction{𝒞}\AgdaSpace{}%
\AgdaBound{D}\AgdaSymbol{\}}\<%
\\
\>[1][@{}l@{\AgdaIndent{0}}]%
\>[2]\AgdaSymbol{→}%
\>[17]\AgdaSymbol{(∀}\AgdaSpace{}%
\AgdaBound{C}\AgdaSpace{}%
\AgdaSymbol{→}\AgdaSpace{}%
\AgdaFunction{is-subsingleton}\AgdaSpace{}%
\AgdaSymbol{(}\AgdaFunction{𝒞}\AgdaSymbol{\{}\AgdaBound{A}\AgdaSymbol{\}\{}\AgdaBound{R}\AgdaSymbol{\}}\AgdaSpace{}%
\AgdaBound{C}\AgdaSymbol{))}\<%
\\
%
\>[2]\AgdaSymbol{→}%
\>[17]\AgdaBound{C}\AgdaSpace{}%
\AgdaOperator{\AgdaDatatype{≡}}\AgdaSpace{}%
\AgdaBound{D}%
\>[24]\AgdaSymbol{→}%
\>[27]\AgdaSymbol{(}\AgdaBound{C}\AgdaSpace{}%
\AgdaOperator{\AgdaInductiveConstructor{,}}\AgdaSpace{}%
\AgdaBound{c}\AgdaSymbol{)}\AgdaSpace{}%
\AgdaOperator{\AgdaDatatype{≡}}\AgdaSpace{}%
\AgdaSymbol{(}\AgdaBound{D}\AgdaSpace{}%
\AgdaOperator{\AgdaInductiveConstructor{,}}\AgdaSpace{}%
\AgdaBound{d}\AgdaSymbol{)}\<%
\\
%
\\[\AgdaEmptyExtraSkip]%
%
\>[1]\AgdaFunction{to-subtype-⟦⟧}\AgdaSpace{}%
\AgdaSymbol{\{}\AgdaArgument{D}\AgdaSpace{}%
\AgdaSymbol{=}\AgdaSpace{}%
\AgdaBound{D}\AgdaSymbol{\}\{}\AgdaBound{c}\AgdaSymbol{\}\{}\AgdaBound{d}\AgdaSymbol{\}}\AgdaSpace{}%
\AgdaBound{ssA}\AgdaSpace{}%
\AgdaBound{CD}\AgdaSpace{}%
\AgdaSymbol{=}\AgdaSpace{}%
\AgdaFunction{to-Σ-≡}\AgdaSpace{}%
\AgdaSymbol{(}\AgdaBound{CD}\AgdaSpace{}%
\AgdaOperator{\AgdaInductiveConstructor{,}}\AgdaSpace{}%
\AgdaBound{ssA}\AgdaSpace{}%
\AgdaBound{D}\AgdaSpace{}%
\AgdaSymbol{(}\AgdaFunction{transport}\AgdaSpace{}%
\AgdaFunction{𝒞}\AgdaSpace{}%
\AgdaBound{CD}\AgdaSpace{}%
\AgdaBound{c}\AgdaSymbol{)}\AgdaSpace{}%
\AgdaBound{d}\AgdaSymbol{)}\<%
\\
%
\\[\AgdaEmptyExtraSkip]%
%
\>[1]\AgdaFunction{class-extensionality'}\AgdaSpace{}%
\AgdaSymbol{:}\AgdaSpace{}%
\AgdaFunction{propext}\AgdaSpace{}%
\AgdaBound{𝓡}\AgdaSpace{}%
\AgdaSymbol{→}\AgdaSpace{}%
\AgdaFunction{global-dfunext}\<%
\\
\>[1][@{}l@{\AgdaIndent{0}}]%
\>[2]\AgdaSymbol{→}%
\>[10]\AgdaSymbol{\{}\AgdaBound{A}\AgdaSpace{}%
\AgdaSymbol{:}\AgdaSpace{}%
\AgdaBound{𝓤}\AgdaSpace{}%
\AgdaOperator{\AgdaFunction{̇}}\AgdaSymbol{\}\{}\AgdaBound{a}\AgdaSpace{}%
\AgdaBound{a'}\AgdaSpace{}%
\AgdaSymbol{:}\AgdaSpace{}%
\AgdaBound{A}\AgdaSymbol{\}\{}\AgdaBound{R}\AgdaSpace{}%
\AgdaSymbol{:}\AgdaSpace{}%
\AgdaFunction{Rel}\AgdaSpace{}%
\AgdaBound{A}\AgdaSpace{}%
\AgdaBound{𝓡}\AgdaSymbol{\}}\<%
\\
%
\>[2]\AgdaSymbol{→}%
\>[10]\AgdaSymbol{(∀}\AgdaSpace{}%
\AgdaBound{a}\AgdaSpace{}%
\AgdaBound{x}\AgdaSpace{}%
\AgdaSymbol{→}\AgdaSpace{}%
\AgdaFunction{is-subsingleton}\AgdaSpace{}%
\AgdaSymbol{(}\AgdaBound{R}\AgdaSpace{}%
\AgdaBound{a}\AgdaSpace{}%
\AgdaBound{x}\AgdaSymbol{))}\<%
\\
%
\>[2]\AgdaSymbol{→}%
\>[10]\AgdaSymbol{(∀}\AgdaSpace{}%
\AgdaBound{C}\AgdaSpace{}%
\AgdaSymbol{→}\AgdaSpace{}%
\AgdaFunction{is-subsingleton}\AgdaSpace{}%
\AgdaSymbol{(}\AgdaFunction{𝒞}\AgdaSpace{}%
\AgdaBound{C}\AgdaSymbol{))}\<%
\\
%
\>[2]\AgdaSymbol{→}%
\>[10]\AgdaRecord{IsEquivalence}\AgdaSpace{}%
\AgdaBound{R}\<%
\\
\>[2][@{}l@{\AgdaIndent{0}}]%
\>[9]\AgdaComment{---------------------------------------}\<%
\\
%
\>[2]\AgdaSymbol{→}%
\>[11]\AgdaBound{R}\AgdaSpace{}%
\AgdaBound{a}\AgdaSpace{}%
\AgdaBound{a'}\AgdaSpace{}%
\AgdaSymbol{→}\AgdaSpace{}%
\AgdaSymbol{(}\AgdaOperator{\AgdaFunction{⟦}}\AgdaSpace{}%
\AgdaBound{a}\AgdaSpace{}%
\AgdaOperator{\AgdaFunction{⟧}}\AgdaSpace{}%
\AgdaSymbol{\{}\AgdaBound{R}\AgdaSymbol{\})}\AgdaSpace{}%
\AgdaOperator{\AgdaDatatype{≡}}\AgdaSpace{}%
\AgdaSymbol{(}\AgdaOperator{\AgdaFunction{⟦}}\AgdaSpace{}%
\AgdaBound{a'}\AgdaSpace{}%
\AgdaOperator{\AgdaFunction{⟧}}\AgdaSpace{}%
\AgdaSymbol{\{}\AgdaBound{R}\AgdaSymbol{\})}\<%
\\
%
\\[\AgdaEmptyExtraSkip]%
%
\>[1]\AgdaFunction{class-extensionality'}\AgdaSpace{}%
\AgdaBound{pe}\AgdaSpace{}%
\AgdaBound{gfe}\AgdaSpace{}%
\AgdaSymbol{\{}\AgdaArgument{A}\AgdaSpace{}%
\AgdaSymbol{=}\AgdaSpace{}%
\AgdaBound{A}\AgdaSymbol{\}\{}\AgdaBound{a}\AgdaSymbol{\}\{}\AgdaBound{a'}\AgdaSymbol{\}\{}\AgdaBound{R}\AgdaSymbol{\}}\AgdaSpace{}%
\AgdaBound{ssR}\AgdaSpace{}%
\AgdaBound{ssA}\AgdaSpace{}%
\AgdaBound{Req}\AgdaSpace{}%
\AgdaBound{Raa'}\AgdaSpace{}%
\AgdaSymbol{=}\AgdaSpace{}%
\AgdaFunction{γ}\<%
\\
\>[1][@{}l@{\AgdaIndent{0}}]%
\>[2]\AgdaKeyword{where}\<%
\\
\>[2][@{}l@{\AgdaIndent{0}}]%
\>[3]\AgdaFunction{CD}\AgdaSpace{}%
\AgdaSymbol{:}\AgdaSpace{}%
\AgdaSymbol{(}\AgdaOperator{\AgdaFunction{[}}\AgdaSpace{}%
\AgdaBound{a}\AgdaSpace{}%
\AgdaOperator{\AgdaFunction{]}}\AgdaSpace{}%
\AgdaBound{R}\AgdaSymbol{)}\AgdaSpace{}%
\AgdaOperator{\AgdaDatatype{≡}}\AgdaSpace{}%
\AgdaSymbol{(}\AgdaOperator{\AgdaFunction{[}}\AgdaSpace{}%
\AgdaBound{a'}\AgdaSpace{}%
\AgdaOperator{\AgdaFunction{]}}\AgdaSpace{}%
\AgdaBound{R}\AgdaSymbol{)}\<%
\\
%
\>[3]\AgdaFunction{CD}\AgdaSpace{}%
\AgdaSymbol{=}\AgdaSpace{}%
\AgdaFunction{class-extensionality}\AgdaSpace{}%
\AgdaBound{pe}\AgdaSpace{}%
\AgdaBound{gfe}\AgdaSpace{}%
\AgdaSymbol{\{}\AgdaBound{A}\AgdaSymbol{\}\{}\AgdaBound{a}\AgdaSymbol{\}\{}\AgdaBound{a'}\AgdaSymbol{\}\{}\AgdaBound{R}\AgdaSymbol{\}}\AgdaSpace{}%
\AgdaBound{ssR}\AgdaSpace{}%
\AgdaBound{Req}\AgdaSpace{}%
\AgdaBound{Raa'}\<%
\\
%
\\[\AgdaEmptyExtraSkip]%
%
\>[3]\AgdaFunction{γ}\AgdaSpace{}%
\AgdaSymbol{:}\AgdaSpace{}%
\AgdaSymbol{(}\AgdaOperator{\AgdaFunction{⟦}}\AgdaSpace{}%
\AgdaBound{a}\AgdaSpace{}%
\AgdaOperator{\AgdaFunction{⟧}}\AgdaSpace{}%
\AgdaSymbol{\{}\AgdaBound{R}\AgdaSymbol{\})}\AgdaSpace{}%
\AgdaOperator{\AgdaDatatype{≡}}\AgdaSpace{}%
\AgdaSymbol{(}\AgdaOperator{\AgdaFunction{⟦}}\AgdaSpace{}%
\AgdaBound{a'}\AgdaSpace{}%
\AgdaOperator{\AgdaFunction{⟧}}\AgdaSpace{}%
\AgdaSymbol{\{}\AgdaBound{R}\AgdaSymbol{\})}\<%
\\
%
\>[3]\AgdaFunction{γ}\AgdaSpace{}%
\AgdaSymbol{=}\AgdaSpace{}%
\AgdaFunction{to-subtype-⟦⟧}\AgdaSpace{}%
\AgdaBound{ssA}\AgdaSpace{}%
\AgdaFunction{CD}\<%
\\
\>[0]\<%
\end{code}

\subsubsection{Compatibility}\label{compatibility}

The following definitions and lemmas are useful for asserting and
proving facts about \textbf{compatibility} of relations and functions.

\begin{code}%
\>[0]\AgdaKeyword{module}\AgdaSpace{}%
\AgdaModule{\AgdaUnderscore{}}\AgdaSpace{}%
\AgdaSymbol{\{}\AgdaBound{𝓤}\AgdaSpace{}%
\AgdaBound{𝓥}\AgdaSpace{}%
\AgdaBound{𝓦}\AgdaSpace{}%
\AgdaSymbol{:}\AgdaSpace{}%
\AgdaPostulate{Universe}\AgdaSymbol{\}}\AgdaSpace{}%
\AgdaSymbol{\{}\AgdaBound{γ}\AgdaSpace{}%
\AgdaSymbol{:}\AgdaSpace{}%
\AgdaBound{𝓥}\AgdaSpace{}%
\AgdaOperator{\AgdaFunction{̇}}\AgdaSpace{}%
\AgdaSymbol{\}}\AgdaSpace{}%
\AgdaSymbol{\{}\AgdaBound{Z}\AgdaSpace{}%
\AgdaSymbol{:}\AgdaSpace{}%
\AgdaBound{𝓤}\AgdaSpace{}%
\AgdaOperator{\AgdaFunction{̇}}\AgdaSpace{}%
\AgdaSymbol{\}}\AgdaSpace{}%
\AgdaKeyword{where}\<%
\\
%
\\[\AgdaEmptyExtraSkip]%
\>[0][@{}l@{\AgdaIndent{0}}]%
\>[1]\AgdaFunction{lift-rel}\AgdaSpace{}%
\AgdaSymbol{:}\AgdaSpace{}%
\AgdaFunction{Rel}\AgdaSpace{}%
\AgdaBound{Z}\AgdaSpace{}%
\AgdaBound{𝓦}\AgdaSpace{}%
\AgdaSymbol{→}\AgdaSpace{}%
\AgdaSymbol{(}\AgdaBound{γ}\AgdaSpace{}%
\AgdaSymbol{→}\AgdaSpace{}%
\AgdaBound{Z}\AgdaSymbol{)}\AgdaSpace{}%
\AgdaSymbol{→}\AgdaSpace{}%
\AgdaSymbol{(}\AgdaBound{γ}\AgdaSpace{}%
\AgdaSymbol{→}\AgdaSpace{}%
\AgdaBound{Z}\AgdaSymbol{)}\AgdaSpace{}%
\AgdaSymbol{→}\AgdaSpace{}%
\AgdaBound{𝓥}\AgdaSpace{}%
\AgdaOperator{\AgdaPrimitive{⊔}}\AgdaSpace{}%
\AgdaBound{𝓦}\AgdaSpace{}%
\AgdaOperator{\AgdaFunction{̇}}\<%
\\
%
\>[1]\AgdaFunction{lift-rel}\AgdaSpace{}%
\AgdaBound{R}\AgdaSpace{}%
\AgdaBound{f}\AgdaSpace{}%
\AgdaBound{g}\AgdaSpace{}%
\AgdaSymbol{=}\AgdaSpace{}%
\AgdaSymbol{∀}\AgdaSpace{}%
\AgdaBound{x}\AgdaSpace{}%
\AgdaSymbol{→}\AgdaSpace{}%
\AgdaBound{R}\AgdaSpace{}%
\AgdaSymbol{(}\AgdaBound{f}\AgdaSpace{}%
\AgdaBound{x}\AgdaSymbol{)}\AgdaSpace{}%
\AgdaSymbol{(}\AgdaBound{g}\AgdaSpace{}%
\AgdaBound{x}\AgdaSymbol{)}\<%
\\
%
\\[\AgdaEmptyExtraSkip]%
%
\>[1]\AgdaFunction{compatible-fun}\AgdaSpace{}%
\AgdaSymbol{:}\AgdaSpace{}%
\AgdaSymbol{(}\AgdaBound{f}\AgdaSpace{}%
\AgdaSymbol{:}\AgdaSpace{}%
\AgdaSymbol{(}\AgdaBound{γ}\AgdaSpace{}%
\AgdaSymbol{→}\AgdaSpace{}%
\AgdaBound{Z}\AgdaSymbol{)}\AgdaSpace{}%
\AgdaSymbol{→}\AgdaSpace{}%
\AgdaBound{Z}\AgdaSymbol{)(}\AgdaBound{R}\AgdaSpace{}%
\AgdaSymbol{:}\AgdaSpace{}%
\AgdaFunction{Rel}\AgdaSpace{}%
\AgdaBound{Z}\AgdaSpace{}%
\AgdaBound{𝓦}\AgdaSymbol{)}\AgdaSpace{}%
\AgdaSymbol{→}\AgdaSpace{}%
\AgdaBound{𝓥}\AgdaSpace{}%
\AgdaOperator{\AgdaPrimitive{⊔}}\AgdaSpace{}%
\AgdaBound{𝓤}\AgdaSpace{}%
\AgdaOperator{\AgdaPrimitive{⊔}}\AgdaSpace{}%
\AgdaBound{𝓦}\AgdaSpace{}%
\AgdaOperator{\AgdaFunction{̇}}\<%
\\
%
\>[1]\AgdaFunction{compatible-fun}\AgdaSpace{}%
\AgdaBound{f}\AgdaSpace{}%
\AgdaBound{R}%
\>[21]\AgdaSymbol{=}\AgdaSpace{}%
\AgdaSymbol{(}\AgdaFunction{lift-rel}\AgdaSpace{}%
\AgdaBound{R}\AgdaSymbol{)}\AgdaSpace{}%
\AgdaOperator{\AgdaFunction{=[}}\AgdaSpace{}%
\AgdaBound{f}\AgdaSpace{}%
\AgdaOperator{\AgdaFunction{]⇒}}\AgdaSpace{}%
\AgdaBound{R}\<%
\\
%
\\[\AgdaEmptyExtraSkip]%
\>[0]\AgdaComment{-- relation compatible with an operation}\<%
\\
\>[0]\AgdaKeyword{module}\AgdaSpace{}%
\AgdaModule{\AgdaUnderscore{}}\AgdaSpace{}%
\AgdaSymbol{\{}\AgdaBound{𝓤}\AgdaSpace{}%
\AgdaBound{𝓦}\AgdaSpace{}%
\AgdaSymbol{:}\AgdaSpace{}%
\AgdaPostulate{Universe}\AgdaSymbol{\}}\AgdaSpace{}%
\AgdaSymbol{\{}\AgdaBound{𝑆}\AgdaSpace{}%
\AgdaSymbol{:}\AgdaSpace{}%
\AgdaFunction{Signature}\AgdaSpace{}%
\AgdaGeneralizable{𝓞}\AgdaSpace{}%
\AgdaGeneralizable{𝓥}\AgdaSymbol{\}}\AgdaSpace{}%
\AgdaKeyword{where}\<%
\\
\>[0][@{}l@{\AgdaIndent{0}}]%
\>[1]\AgdaFunction{compatible-op}\AgdaSpace{}%
\AgdaSymbol{:}\AgdaSpace{}%
\AgdaSymbol{\{}\AgdaBound{𝑨}\AgdaSpace{}%
\AgdaSymbol{:}\AgdaSpace{}%
\AgdaFunction{Algebra}\AgdaSpace{}%
\AgdaBound{𝓤}\AgdaSpace{}%
\AgdaBound{𝑆}\AgdaSymbol{\}}\AgdaSpace{}%
\AgdaSymbol{→}\AgdaSpace{}%
\AgdaOperator{\AgdaFunction{∣}}\AgdaSpace{}%
\AgdaBound{𝑆}\AgdaSpace{}%
\AgdaOperator{\AgdaFunction{∣}}\AgdaSpace{}%
\AgdaSymbol{→}\AgdaSpace{}%
\AgdaFunction{Rel}\AgdaSpace{}%
\AgdaOperator{\AgdaFunction{∣}}\AgdaSpace{}%
\AgdaBound{𝑨}\AgdaSpace{}%
\AgdaOperator{\AgdaFunction{∣}}\AgdaSpace{}%
\AgdaBound{𝓦}\AgdaSpace{}%
\AgdaSymbol{→}\AgdaSpace{}%
\AgdaBound{𝓤}\AgdaSpace{}%
\AgdaOperator{\AgdaPrimitive{⊔}}\AgdaSpace{}%
\AgdaBound{𝓥}\AgdaSpace{}%
\AgdaOperator{\AgdaPrimitive{⊔}}\AgdaSpace{}%
\AgdaBound{𝓦}\AgdaSpace{}%
\AgdaOperator{\AgdaFunction{̇}}\<%
\\
%
\>[1]\AgdaFunction{compatible-op}\AgdaSpace{}%
\AgdaSymbol{\{}\AgdaBound{𝑨}\AgdaSymbol{\}}\AgdaSpace{}%
\AgdaBound{f}\AgdaSpace{}%
\AgdaBound{R}\AgdaSpace{}%
\AgdaSymbol{=}\AgdaSpace{}%
\AgdaSymbol{∀\{}\AgdaBound{𝒂}\AgdaSymbol{\}\{}\AgdaBound{𝒃}\AgdaSymbol{\}}\AgdaSpace{}%
\AgdaSymbol{→}\AgdaSpace{}%
\AgdaSymbol{(}\AgdaFunction{lift-rel}\AgdaSpace{}%
\AgdaBound{R}\AgdaSymbol{)}\AgdaSpace{}%
\AgdaBound{𝒂}\AgdaSpace{}%
\AgdaBound{𝒃}%
\>[53]\AgdaSymbol{→}\AgdaSpace{}%
\AgdaBound{R}\AgdaSpace{}%
\AgdaSymbol{((}\AgdaBound{f}\AgdaSpace{}%
\AgdaOperator{\AgdaFunction{̂}}\AgdaSpace{}%
\AgdaBound{𝑨}\AgdaSymbol{)}\AgdaSpace{}%
\AgdaBound{𝒂}\AgdaSymbol{)}\AgdaSpace{}%
\AgdaSymbol{((}\AgdaBound{f}\AgdaSpace{}%
\AgdaOperator{\AgdaFunction{̂}}\AgdaSpace{}%
\AgdaBound{𝑨}\AgdaSymbol{)}\AgdaSpace{}%
\AgdaBound{𝒃}\AgdaSymbol{)}\<%
\\
%
\>[1]\AgdaComment{-- alternative notation: (lift-rel R) =[ f ̂ 𝑨 ]⇒ R}\<%
\\
%
\\[\AgdaEmptyExtraSkip]%
\>[0]\AgdaComment{--The given relation is compatible with all ops of an algebra.}\<%
\\
\>[0][@{}l@{\AgdaIndent{0}}]%
\>[1]\AgdaFunction{compatible}\AgdaSpace{}%
\AgdaSymbol{:(}\AgdaBound{𝑨}\AgdaSpace{}%
\AgdaSymbol{:}\AgdaSpace{}%
\AgdaFunction{Algebra}\AgdaSpace{}%
\AgdaBound{𝓤}\AgdaSpace{}%
\AgdaBound{𝑆}\AgdaSymbol{)}\AgdaSpace{}%
\AgdaSymbol{→}\AgdaSpace{}%
\AgdaFunction{Rel}\AgdaSpace{}%
\AgdaOperator{\AgdaFunction{∣}}\AgdaSpace{}%
\AgdaBound{𝑨}\AgdaSpace{}%
\AgdaOperator{\AgdaFunction{∣}}\AgdaSpace{}%
\AgdaBound{𝓦}\AgdaSpace{}%
\AgdaSymbol{→}\AgdaSpace{}%
\AgdaBound{𝓞}\AgdaSpace{}%
\AgdaOperator{\AgdaPrimitive{⊔}}\AgdaSpace{}%
\AgdaBound{𝓤}\AgdaSpace{}%
\AgdaOperator{\AgdaPrimitive{⊔}}\AgdaSpace{}%
\AgdaBound{𝓥}\AgdaSpace{}%
\AgdaOperator{\AgdaPrimitive{⊔}}\AgdaSpace{}%
\AgdaBound{𝓦}\AgdaSpace{}%
\AgdaOperator{\AgdaFunction{̇}}\<%
\\
%
\>[1]\AgdaFunction{compatible}%
\>[13]\AgdaBound{𝑨}\AgdaSpace{}%
\AgdaBound{R}\AgdaSpace{}%
\AgdaSymbol{=}\AgdaSpace{}%
\AgdaSymbol{∀}\AgdaSpace{}%
\AgdaBound{f}\AgdaSpace{}%
\AgdaSymbol{→}\AgdaSpace{}%
\AgdaFunction{compatible-op}\AgdaSymbol{\{}\AgdaBound{𝑨}\AgdaSymbol{\}}\AgdaSpace{}%
\AgdaBound{f}\AgdaSpace{}%
\AgdaBound{R}\<%
\end{code}

\subsubsection{Examples}\label{examples}

\begin{code}%
\>[0]\<%
\\
\>[0]\AgdaKeyword{module}\AgdaSpace{}%
\AgdaModule{\AgdaUnderscore{}}\AgdaSpace{}%
\AgdaSymbol{\{}\AgdaBound{𝓤}\AgdaSpace{}%
\AgdaSymbol{:}\AgdaSpace{}%
\AgdaPostulate{Universe}\AgdaSymbol{\}}\AgdaSpace{}%
\AgdaSymbol{\{}\AgdaBound{𝑆}\AgdaSpace{}%
\AgdaSymbol{:}\AgdaSpace{}%
\AgdaFunction{Signature}\AgdaSpace{}%
\AgdaGeneralizable{𝓞}\AgdaSpace{}%
\AgdaGeneralizable{𝓥}\AgdaSymbol{\}}\AgdaSpace{}%
\AgdaKeyword{where}\<%
\\
%
\\[\AgdaEmptyExtraSkip]%
\>[0][@{}l@{\AgdaIndent{0}}]%
\>[1]\AgdaFunction{𝟎-compatible-op}\AgdaSpace{}%
\AgdaSymbol{:}\AgdaSpace{}%
\AgdaFunction{funext}\AgdaSpace{}%
\AgdaBound{𝓥}\AgdaSpace{}%
\AgdaBound{𝓤}\AgdaSpace{}%
\AgdaSymbol{→}\AgdaSpace{}%
\AgdaSymbol{\{}\AgdaBound{𝑨}\AgdaSpace{}%
\AgdaSymbol{:}\AgdaSpace{}%
\AgdaFunction{Algebra}\AgdaSpace{}%
\AgdaBound{𝓤}\AgdaSpace{}%
\AgdaBound{𝑆}\AgdaSymbol{\}}\AgdaSpace{}%
\AgdaSymbol{(}\AgdaBound{f}\AgdaSpace{}%
\AgdaSymbol{:}\AgdaSpace{}%
\AgdaOperator{\AgdaFunction{∣}}\AgdaSpace{}%
\AgdaBound{𝑆}\AgdaSpace{}%
\AgdaOperator{\AgdaFunction{∣}}\AgdaSymbol{)}\<%
\\
\>[1][@{}l@{\AgdaIndent{0}}]%
\>[2]\AgdaSymbol{→}%
\>[22]\AgdaFunction{compatible-op}\AgdaSpace{}%
\AgdaSymbol{\{}\AgdaArgument{𝓤}\AgdaSpace{}%
\AgdaSymbol{=}\AgdaSpace{}%
\AgdaBound{𝓤}\AgdaSymbol{\}\{}\AgdaArgument{𝑨}\AgdaSpace{}%
\AgdaSymbol{=}\AgdaSpace{}%
\AgdaBound{𝑨}\AgdaSymbol{\}}\AgdaSpace{}%
\AgdaBound{f}\AgdaSpace{}%
\AgdaFunction{𝟎-rel}\<%
\\
%
\>[1]\AgdaFunction{𝟎-compatible-op}\AgdaSpace{}%
\AgdaBound{fe}\AgdaSpace{}%
\AgdaSymbol{\{}\AgdaBound{𝑨}\AgdaSymbol{\}}\AgdaSpace{}%
\AgdaBound{f}\AgdaSpace{}%
\AgdaBound{ptws0}%
\>[33]\AgdaSymbol{=}\AgdaSpace{}%
\AgdaFunction{ap}\AgdaSpace{}%
\AgdaSymbol{(}\AgdaBound{f}\AgdaSpace{}%
\AgdaOperator{\AgdaFunction{̂}}\AgdaSpace{}%
\AgdaBound{𝑨}\AgdaSymbol{)}\AgdaSpace{}%
\AgdaSymbol{(}\AgdaBound{fe}\AgdaSpace{}%
\AgdaSymbol{(λ}\AgdaSpace{}%
\AgdaBound{x}\AgdaSpace{}%
\AgdaSymbol{→}\AgdaSpace{}%
\AgdaBound{ptws0}\AgdaSpace{}%
\AgdaBound{x}\AgdaSymbol{))}\<%
\\
%
\\[\AgdaEmptyExtraSkip]%
%
\>[1]\AgdaFunction{𝟎-compatible}\AgdaSpace{}%
\AgdaSymbol{:}\AgdaSpace{}%
\AgdaFunction{funext}\AgdaSpace{}%
\AgdaBound{𝓥}\AgdaSpace{}%
\AgdaBound{𝓤}\AgdaSpace{}%
\AgdaSymbol{→}\AgdaSpace{}%
\AgdaSymbol{\{}\AgdaBound{A}\AgdaSpace{}%
\AgdaSymbol{:}\AgdaSpace{}%
\AgdaFunction{Algebra}\AgdaSpace{}%
\AgdaBound{𝓤}\AgdaSpace{}%
\AgdaBound{𝑆}\AgdaSymbol{\}}\AgdaSpace{}%
\AgdaSymbol{→}\AgdaSpace{}%
\AgdaFunction{compatible}\AgdaSpace{}%
\AgdaBound{A}\AgdaSpace{}%
\AgdaFunction{𝟎-rel}\<%
\\
%
\>[1]\AgdaFunction{𝟎-compatible}\AgdaSpace{}%
\AgdaBound{fe}\AgdaSpace{}%
\AgdaSymbol{\{}\AgdaBound{A}\AgdaSymbol{\}}\AgdaSpace{}%
\AgdaSymbol{=}\AgdaSpace{}%
\AgdaSymbol{λ}\AgdaSpace{}%
\AgdaBound{f}\AgdaSpace{}%
\AgdaBound{args}\AgdaSpace{}%
\AgdaSymbol{→}\AgdaSpace{}%
\AgdaFunction{𝟎-compatible-op}\AgdaSpace{}%
\AgdaBound{fe}\AgdaSpace{}%
\AgdaSymbol{\{}\AgdaBound{A}\AgdaSymbol{\}}\AgdaSpace{}%
\AgdaBound{f}\AgdaSpace{}%
\AgdaBound{args}\<%
\end{code}

For a binary relation \ab R on \ab A, we denote a single \ab R-class by \af [ \ab a \af ] \ab R (this denotes the class containing \ab a). We denote the type of all classes of a relation \ab R on \ab A by \af 𝒞 \as{\{} \ab A \as{\}} \as{\{} \ab R \as{\}}.  These are defined as in the \ualib as follows.
\ccpad
\begin{code}%
\>[1]\AgdaOperator{\AgdaFunction{[\AgdaUnderscore{}]}}\AgdaSpace{}%
\AgdaSymbol{:}\AgdaSpace{}%
\AgdaSymbol{\{}\AgdaBound{A}\AgdaSpace{}%
\AgdaSymbol{:}\AgdaSpace{}%
\AgdaBound{𝓤}\AgdaSpace{}%
\AgdaOperator{\AgdaFunction{̇}}\AgdaSpace{}%
\AgdaSymbol{\}}\AgdaSpace{}%
\AgdaSymbol{→}\AgdaSpace{}%
\AgdaBound{A}\AgdaSpace{}%
\AgdaSymbol{→}\AgdaSpace{}%
\AgdaFunction{Rel}\AgdaSpace{}%
\AgdaBound{A}\AgdaSpace{}%
\AgdaBound{𝓡}\AgdaSpace{}%
\AgdaSymbol{→}\AgdaSpace{}%
\AgdaFunction{Pred}\AgdaSpace{}%
\AgdaBound{A}\AgdaSpace{}%
\AgdaBound{𝓡}\<%
\\
%
\>[1]\AgdaOperator{\AgdaFunction{[}}\AgdaSpace{}%
\AgdaBound{a}\AgdaSpace{}%
\AgdaOperator{\AgdaFunction{]}}\AgdaSpace{}%
\AgdaBound{R}\AgdaSpace{}%
\AgdaSymbol{=}\AgdaSpace{}%
\AgdaSymbol{λ}\AgdaSpace{}%
\AgdaBound{x}\AgdaSpace{}%
\AgdaSymbol{→}\AgdaSpace{}%
\AgdaBound{R}\AgdaSpace{}%
\AgdaBound{a}\AgdaSpace{}%
\AgdaBound{x}\<%
\\
\\
\>[1]\AgdaFunction{𝒞}\AgdaSpace{}%
\AgdaSymbol{:}\AgdaSpace{}%
\AgdaSymbol{\{}\AgdaBound{A}\AgdaSpace{}%
\AgdaSymbol{:}\AgdaSpace{}%
\AgdaBound{𝓤}\AgdaSpace{}%
\AgdaOperator{\AgdaFunction{̇}}\AgdaSymbol{\}\{}\AgdaBound{R}\AgdaSpace{}%
\AgdaSymbol{:}\AgdaSpace{}%
\AgdaFunction{Rel}\AgdaSpace{}%
\AgdaBound{A}\AgdaSpace{}%
\AgdaBound{𝓡}\AgdaSymbol{\}}\AgdaSpace{}%
\AgdaSymbol{→}\AgdaSpace{}%
\AgdaFunction{Pred}\AgdaSpace{}%
\AgdaBound{A}\AgdaSpace{}%
\AgdaBound{𝓡}\AgdaSpace{}%
\AgdaSymbol{→}\AgdaSpace{}%
\AgdaSymbol{(}\AgdaBound{𝓤}\AgdaSpace{}%
\AgdaOperator{\AgdaPrimitive{⊔}}\AgdaSpace{}%
\AgdaBound{𝓡}\AgdaSpace{}%
\AgdaOperator{\AgdaPrimitive{⁺}}\AgdaSymbol{)}\AgdaSpace{}%
\AgdaOperator{\AgdaFunction{̇}}\<%
\\
%
\>[1]\AgdaFunction{𝒞}\AgdaSpace{}%
\AgdaSymbol{\{}\AgdaBound{A}\AgdaSymbol{\}\{}\AgdaBound{R}\AgdaSymbol{\}}\AgdaSpace{}%
\AgdaSymbol{=}\AgdaSpace{}%
\AgdaSymbol{λ}\AgdaSpace{}%
\AgdaSymbol{(}\AgdaBound{C}\AgdaSpace{}%
\AgdaSymbol{:}\AgdaSpace{}%
\AgdaFunction{Pred}\AgdaSpace{}%
\AgdaBound{A}\AgdaSpace{}%
\AgdaBound{𝓡}\AgdaSymbol{)}\AgdaSpace{}%
\AgdaSymbol{→}\AgdaSpace{}%
\AgdaFunction{Σ}\AgdaSpace{}%
\AgdaBound{a}\AgdaSpace{}%
\AgdaFunction{꞉}\AgdaSpace{}%
\AgdaBound{A}\AgdaSpace{}%
\AgdaFunction{,}\AgdaSpace{}%
\AgdaBound{C}\AgdaSpace{}%
\AgdaOperator{\AgdaDatatype{≡}}\AgdaSpace{}%
\AgdaSymbol{(}\AgdaSpace{}%
\AgdaOperator{\AgdaFunction{[}}\AgdaSpace{}%
\AgdaBound{a}\AgdaSpace{}%
\AgdaOperator{\AgdaFunction{]}}\AgdaSpace{}%
\AgdaBound{R}\AgdaSymbol{)}\<%
\end{code}
\ccpad
There are a few ways we could define the quotient with respect to a relation. We have found the following to be the most convenient.
\ccpad
\begin{code}%
\>[1]\AgdaOperator{\AgdaFunction{\AgdaUnderscore{}/\AgdaUnderscore{}}}\AgdaSpace{}%
\AgdaSymbol{:}\AgdaSpace{}%
\AgdaSymbol{(}\AgdaBound{A}\AgdaSpace{}%
\AgdaSymbol{:}\AgdaSpace{}%
\AgdaBound{𝓤}\AgdaSpace{}%
\AgdaOperator{\AgdaFunction{̇}}\AgdaSpace{}%
\AgdaSymbol{)}\AgdaSpace{}%
\AgdaSymbol{→}\AgdaSpace{}%
\AgdaFunction{Rel}\AgdaSpace{}%
\AgdaBound{A}\AgdaSpace{}%
\AgdaBound{𝓡}\AgdaSpace{}%
\AgdaSymbol{→}\AgdaSpace{}%
\AgdaBound{𝓤}\AgdaSpace{}%
\AgdaOperator{\AgdaPrimitive{⊔}}\AgdaSpace{}%
\AgdaSymbol{(}\AgdaBound{𝓡}\AgdaSpace{}%
\AgdaOperator{\AgdaPrimitive{⁺}}\AgdaSymbol{)}\AgdaSpace{}%
\AgdaOperator{\AgdaFunction{̇}}\<%
\\
%
\>[1]\AgdaBound{A}\AgdaSpace{}%
\AgdaOperator{\AgdaFunction{/}}\AgdaSpace{}%
\AgdaBound{R}\AgdaSpace{}%
\AgdaSymbol{=}\AgdaSpace{}%
\AgdaFunction{Σ}\AgdaSpace{}%
\AgdaBound{C}\AgdaSpace{}%
\AgdaFunction{꞉}\AgdaSpace{}%
\AgdaFunction{Pred}\AgdaSpace{}%
\AgdaBound{A}\AgdaSpace{}%
\AgdaBound{𝓡}\AgdaSpace{}%
\AgdaFunction{,}%
\>[27]\AgdaFunction{𝒞}\AgdaSymbol{\{}\AgdaBound{A}\AgdaSymbol{\}\{}\AgdaBound{R}\AgdaSymbol{\}}\AgdaSpace{}%
\AgdaBound{C}\<%
\\
%
\>[1]\AgdaComment{-- old version:  A / R = Σ C ꞉ Pred A 𝓡 ,  Σ a ꞉ A ,  C ≡ ( [ a ] R )}\<%
\end{code}
\ccpad
We then define the following introduction rule for a relation class with designated representative.
\ccpad
\begin{code}%
\>[0][@{}l@{\AgdaIndent{1}}]%
\>[1]\AgdaOperator{\AgdaFunction{⟦\AgdaUnderscore{}⟧}}\AgdaSpace{}%
\AgdaSymbol{:}\AgdaSpace{}%
\AgdaSymbol{\{}\AgdaBound{A}\AgdaSpace{}%
\AgdaSymbol{:}\AgdaSpace{}%
\AgdaBound{𝓤}\AgdaSpace{}%
\AgdaOperator{\AgdaFunction{̇}}\AgdaSymbol{\}}\AgdaSpace{}%
\AgdaSymbol{→}\AgdaSpace{}%
\AgdaBound{A}\AgdaSpace{}%
\AgdaSymbol{→}\AgdaSpace{}%
\AgdaSymbol{\{}\AgdaBound{R}\AgdaSpace{}%
\AgdaSymbol{:}\AgdaSpace{}%
\AgdaFunction{Rel}\AgdaSpace{}%
\AgdaBound{A}\AgdaSpace{}%
\AgdaBound{𝓡}\AgdaSymbol{\}}\AgdaSpace{}%
\AgdaSymbol{→}\AgdaSpace{}%
\AgdaBound{A}\AgdaSpace{}%
\AgdaOperator{\AgdaFunction{/}}\AgdaSpace{}%
\AgdaBound{R}\<%
\\
%
\>[1]\AgdaOperator{\AgdaFunction{⟦}}\AgdaSpace{}%
\AgdaBound{a}\AgdaSpace{}%
\AgdaOperator{\AgdaFunction{⟧}}\AgdaSpace{}%
\AgdaSymbol{\{}\AgdaBound{R}\AgdaSymbol{\}}\AgdaSpace{}%
\AgdaSymbol{=}\AgdaSpace{}%
\AgdaSymbol{(}\AgdaOperator{\AgdaFunction{[}}\AgdaSpace{}%
\AgdaBound{a}\AgdaSpace{}%
\AgdaOperator{\AgdaFunction{]}}\AgdaSpace{}%
\AgdaBound{R}\AgdaSymbol{)}\AgdaSpace{}%
\AgdaOperator{\AgdaInductiveConstructor{,}}\AgdaSpace{}%
\AgdaBound{a}\AgdaSpace{}%
\AgdaOperator{\AgdaInductiveConstructor{,}}\AgdaSpace{}%
\AgdaInductiveConstructor{𝓇ℯ𝒻𝓁}\<%
\end{code}
\ccpad


\begin{code}
\>[1]\AgdaOperator{\AgdaFunction{⌜\AgdaUnderscore{}⌝}}\AgdaSpace{}%
\AgdaSymbol{:}\AgdaSpace{}%
\AgdaSymbol{\{}\AgdaBound{A}\AgdaSpace{}%
\AgdaSymbol{:}\AgdaSpace{}%
\AgdaBound{𝓤}\AgdaSpace{}%
\AgdaOperator{\AgdaFunction{̇}}\AgdaSymbol{\}\{}\AgdaBound{R}\AgdaSpace{}%
\AgdaSymbol{:}\AgdaSpace{}%
\AgdaFunction{Rel}\AgdaSpace{}%
\AgdaBound{A}\AgdaSpace{}%
\AgdaBound{𝓡}\AgdaSymbol{\}}\AgdaSpace{}%
\AgdaSymbol{→}\AgdaSpace{}%
\AgdaBound{A}\AgdaSpace{}%
\AgdaOperator{\AgdaFunction{/}}\AgdaSpace{}%
\AgdaBound{R}%
\>[39]\AgdaSymbol{→}\AgdaSpace{}%
\AgdaBound{A}\<%
\\
%
\>[1]\AgdaOperator{\AgdaFunction{⌜}}\AgdaSpace{}%
\AgdaBound{𝒂}\AgdaSpace{}%
\AgdaOperator{\AgdaFunction{⌝}}\AgdaSpace{}%
\AgdaSymbol{=}\AgdaSpace{}%
\AgdaOperator{\AgdaFunction{∣}}\AgdaSpace{}%
\AgdaOperator{\AgdaFunction{∥}}\AgdaSpace{}%
\AgdaBound{𝒂}\AgdaSpace{}%
\AgdaOperator{\AgdaFunction{∥}}\AgdaSpace{}%
\AgdaOperator{\AgdaFunction{∣}}%
\>[22]\AgdaComment{-- type ⌜ and ⌝ as `\textbackslash{}cul` and `\textbackslash{}cur`}\<%
\end{code}
\ccpad
Later we will need the following tools for handling quotients.
\ccpad
\begin{code}%
\>[0][@{}l@{\AgdaIndent{1}}]%
\>[1]\AgdaKeyword{open}\AgdaSpace{}%
\AgdaModule{IsEquivalence}\AgdaSymbol{\{}\AgdaBound{𝓤}\AgdaSymbol{\}\{}\AgdaBound{𝓡}\AgdaSymbol{\}}\<%
\\
%
\\[\AgdaEmptyExtraSkip]%
%
\>[1]\AgdaFunction{/-subset}\AgdaSpace{}%
\AgdaSymbol{:}\AgdaSpace{}%
\AgdaSymbol{\{}\AgdaBound{A}\AgdaSpace{}%
\AgdaSymbol{:}\AgdaSpace{}%
\AgdaBound{𝓤}\AgdaSpace{}%
\AgdaOperator{\AgdaFunction{̇}}\AgdaSymbol{\}\{}\AgdaBound{a}\AgdaSpace{}%
\AgdaBound{a'}\AgdaSpace{}%
\AgdaSymbol{:}\AgdaSpace{}%
\AgdaBound{A}\AgdaSymbol{\}\{}\AgdaBound{R}\AgdaSpace{}%
\AgdaSymbol{:}\AgdaSpace{}%
\AgdaFunction{Rel}\AgdaSpace{}%
\AgdaBound{A}\AgdaSpace{}%
\AgdaBound{𝓡}\AgdaSymbol{\}}\<%
\\
\>[1][@{}l@{\AgdaIndent{0}}]%
\>[2]\AgdaSymbol{→}%
\>[6]\AgdaRecord{IsEquivalence}\AgdaSpace{}%
\AgdaBound{R}\AgdaSpace{}%
\AgdaSymbol{→}\AgdaSpace{}%
\AgdaBound{R}\AgdaSpace{}%
\AgdaBound{a}\AgdaSpace{}%
\AgdaBound{a'}\AgdaSpace{}%
\AgdaSymbol{→}\AgdaSpace{}%
\AgdaSymbol{(}\AgdaOperator{\AgdaFunction{[}}\AgdaSpace{}%
\AgdaBound{a}\AgdaSpace{}%
\AgdaOperator{\AgdaFunction{]}}\AgdaSpace{}%
\AgdaBound{R}\AgdaSymbol{)}\AgdaSpace{}%
\AgdaOperator{\AgdaFunction{⊆}}\AgdaSpace{}%
\AgdaSymbol{(}\AgdaOperator{\AgdaFunction{[}}\AgdaSpace{}%
\AgdaBound{a'}\AgdaSpace{}%
\AgdaOperator{\AgdaFunction{]}}\AgdaSpace{}%
\AgdaBound{R}\AgdaSymbol{)}\<%
\\
%
\>[1]\AgdaFunction{/-subset}\AgdaSpace{}%
\AgdaSymbol{\{}\AgdaArgument{A}\AgdaSpace{}%
\AgdaSymbol{=}\AgdaSpace{}%
\AgdaBound{A}\AgdaSymbol{\}\{}\AgdaBound{a}\AgdaSymbol{\}\{}\AgdaBound{a'}\AgdaSymbol{\}\{}\AgdaBound{R}\AgdaSymbol{\}}\AgdaSpace{}%
\AgdaBound{Req}\AgdaSpace{}%
\AgdaBound{Raa'}\AgdaSpace{}%
\AgdaSymbol{\{}\AgdaBound{x}\AgdaSymbol{\}}\AgdaSpace{}%
\AgdaBound{Rax}\AgdaSpace{}%
\AgdaSymbol{=}\AgdaSpace{}%
\AgdaSymbol{(}\AgdaField{trans}\AgdaSpace{}%
\AgdaBound{Req}\AgdaSymbol{)}\AgdaSpace{}%
\AgdaBound{a'}\AgdaSpace{}%
\AgdaBound{a}\AgdaSpace{}%
\AgdaBound{x}\AgdaSpace{}%
\AgdaSymbol{(}\AgdaField{sym}\AgdaSpace{}%
\AgdaBound{Req}\AgdaSpace{}%
\AgdaBound{a}\AgdaSpace{}%
\AgdaBound{a'}\AgdaSpace{}%
\AgdaBound{Raa'}\AgdaSymbol{)}\AgdaSpace{}%
\AgdaBound{Rax}\<%
\\
%
\\[\AgdaEmptyExtraSkip]%
%
\>[1]\AgdaFunction{/-supset}\AgdaSpace{}%
\AgdaSymbol{:}\AgdaSpace{}%
\AgdaSymbol{\{}\AgdaBound{A}\AgdaSpace{}%
\AgdaSymbol{:}\AgdaSpace{}%
\AgdaBound{𝓤}\AgdaSpace{}%
\AgdaOperator{\AgdaFunction{̇}}\AgdaSymbol{\}\{}\AgdaBound{a}\AgdaSpace{}%
\AgdaBound{a'}\AgdaSpace{}%
\AgdaSymbol{:}\AgdaSpace{}%
\AgdaBound{A}\AgdaSymbol{\}\{}\AgdaBound{R}\AgdaSpace{}%
\AgdaSymbol{:}\AgdaSpace{}%
\AgdaFunction{Rel}\AgdaSpace{}%
\AgdaBound{A}\AgdaSpace{}%
\AgdaBound{𝓡}\AgdaSymbol{\}}\<%
\\
\>[1][@{}l@{\AgdaIndent{0}}]%
\>[2]\AgdaSymbol{→}%
\>[6]\AgdaRecord{IsEquivalence}\AgdaSpace{}%
\AgdaBound{R}\AgdaSpace{}%
\AgdaSymbol{→}\AgdaSpace{}%
\AgdaBound{R}\AgdaSpace{}%
\AgdaBound{a}\AgdaSpace{}%
\AgdaBound{a'}\AgdaSpace{}%
\AgdaSymbol{→}\AgdaSpace{}%
\AgdaSymbol{(}\AgdaOperator{\AgdaFunction{[}}\AgdaSpace{}%
\AgdaBound{a}\AgdaSpace{}%
\AgdaOperator{\AgdaFunction{]}}\AgdaSpace{}%
\AgdaBound{R}\AgdaSymbol{)}\AgdaSpace{}%
\AgdaOperator{\AgdaFunction{⊇}}\AgdaSpace{}%
\AgdaSymbol{(}\AgdaOperator{\AgdaFunction{[}}\AgdaSpace{}%
\AgdaBound{a'}\AgdaSpace{}%
\AgdaOperator{\AgdaFunction{]}}\AgdaSpace{}%
\AgdaBound{R}\AgdaSymbol{)}\<%
\\
%
\>[1]\AgdaFunction{/-supset}\AgdaSpace{}%
\AgdaSymbol{\{}\AgdaArgument{A}\AgdaSpace{}%
\AgdaSymbol{=}\AgdaSpace{}%
\AgdaBound{A}\AgdaSymbol{\}\{}\AgdaBound{a}\AgdaSymbol{\}\{}\AgdaBound{a'}\AgdaSymbol{\}\{}\AgdaBound{R}\AgdaSymbol{\}}\AgdaSpace{}%
\AgdaBound{Req}\AgdaSpace{}%
\AgdaBound{Raa'}\AgdaSpace{}%
\AgdaSymbol{\{}\AgdaBound{x}\AgdaSymbol{\}}\AgdaSpace{}%
\AgdaBound{Ra'x}\AgdaSpace{}%
\AgdaSymbol{=}\AgdaSpace{}%
\AgdaSymbol{(}\AgdaField{trans}\AgdaSpace{}%
\AgdaBound{Req}\AgdaSymbol{)}\AgdaSpace{}%
\AgdaBound{a}\AgdaSpace{}%
\AgdaBound{a'}\AgdaSpace{}%
\AgdaBound{x}\AgdaSpace{}%
\AgdaBound{Raa'}\AgdaSpace{}%
\AgdaBound{Ra'x}\<%
\\
%
\\[\AgdaEmptyExtraSkip]%
%
\>[1]\AgdaFunction{/-=̇}\AgdaSpace{}%
\AgdaSymbol{:}\AgdaSpace{}%
\AgdaSymbol{\{}\AgdaBound{A}\AgdaSpace{}%
\AgdaSymbol{:}\AgdaSpace{}%
\AgdaBound{𝓤}\AgdaSpace{}%
\AgdaOperator{\AgdaFunction{̇}}\AgdaSymbol{\}\{}\AgdaBound{a}\AgdaSpace{}%
\AgdaBound{a'}\AgdaSpace{}%
\AgdaSymbol{:}\AgdaSpace{}%
\AgdaBound{A}\AgdaSymbol{\}\{}\AgdaBound{R}\AgdaSpace{}%
\AgdaSymbol{:}\AgdaSpace{}%
\AgdaFunction{Rel}\AgdaSpace{}%
\AgdaBound{A}\AgdaSpace{}%
\AgdaBound{𝓡}\AgdaSymbol{\}}\<%
\\
\>[1][@{}l@{\AgdaIndent{0}}]%
\>[2]\AgdaSymbol{→}%
\>[6]\AgdaRecord{IsEquivalence}\AgdaSpace{}%
\AgdaBound{R}\AgdaSpace{}%
\AgdaSymbol{→}\AgdaSpace{}%
\AgdaBound{R}\AgdaSpace{}%
\AgdaBound{a}\AgdaSpace{}%
\AgdaBound{a'}\AgdaSpace{}%
\AgdaSymbol{→}\AgdaSpace{}%
\AgdaSymbol{(}\AgdaOperator{\AgdaFunction{[}}\AgdaSpace{}%
\AgdaBound{a}\AgdaSpace{}%
\AgdaOperator{\AgdaFunction{]}}\AgdaSpace{}%
\AgdaBound{R}\AgdaSymbol{)}\AgdaSpace{}%
\AgdaOperator{\AgdaFunction{=̇}}\AgdaSpace{}%
\AgdaSymbol{(}\AgdaOperator{\AgdaFunction{[}}\AgdaSpace{}%
\AgdaBound{a'}\AgdaSpace{}%
\AgdaOperator{\AgdaFunction{]}}\AgdaSpace{}%
\AgdaBound{R}\AgdaSymbol{)}\<%
\\
%
\>[1]\AgdaFunction{/-=̇}\AgdaSpace{}%
\AgdaSymbol{\{}\AgdaArgument{A}\AgdaSpace{}%
\AgdaSymbol{=}\AgdaSpace{}%
\AgdaBound{A}\AgdaSymbol{\}\{}\AgdaBound{a}\AgdaSymbol{\}\{}\AgdaBound{a'}\AgdaSymbol{\}\{}\AgdaBound{R}\AgdaSymbol{\}}\AgdaSpace{}%
\AgdaBound{Req}\AgdaSpace{}%
\AgdaBound{Raa'}\AgdaSpace{}%
\AgdaSymbol{=}\AgdaSpace{}%
\AgdaFunction{/-subset}\AgdaSpace{}%
\AgdaBound{Req}\AgdaSpace{}%
\AgdaBound{Raa'}\AgdaSpace{}%
\AgdaOperator{\AgdaInductiveConstructor{,}}\AgdaSpace{}%
\AgdaFunction{/-supset}\AgdaSpace{}%
\AgdaBound{Req}\AgdaSpace{}%
\AgdaBound{Raa'}\<%
\end{code}



\subsubsection{Quotient extensionality}\label{quotient-extensionality}
We need a (subsingleton) identity type for congruence classes over sets so that we can equate two classes even when they are presented using different representatives. For this we assume that our relations are on sets, rather than arbitrary types. As mentioned earlier, this is equivalent to assuming that there is at most one proof that two elements of a set are the same.
\ccpad
\begin{code}%
\>[0][@{}l@{\AgdaIndent{1}}]%
\>[1]\AgdaFunction{class-extensionality}\AgdaSpace{}%
\AgdaSymbol{:}%
\>[200I]\AgdaFunction{propext}\AgdaSpace{}%
\AgdaBound{𝓡}\AgdaSpace{}%
\AgdaSymbol{→}\AgdaSpace{}%
\AgdaFunction{global-dfunext}\<%
\\
\>[2]\AgdaSymbol{→}%
\>[.][@{}l@{}]\<[200I]%
\>[14]\AgdaSymbol{\{}\AgdaBound{A}\AgdaSpace{}%
\AgdaSymbol{:}\AgdaSpace{}%
\AgdaBound{𝓤}\AgdaSpace{}%
\AgdaOperator{\AgdaFunction{̇}}\AgdaSymbol{\}\{}\AgdaBound{a}\AgdaSpace{}%
\AgdaBound{a'}\AgdaSpace{}%
\AgdaSymbol{:}\AgdaSpace{}%
\AgdaBound{A}\AgdaSymbol{\}\{}\AgdaBound{R}\AgdaSpace{}%
\AgdaSymbol{:}\AgdaSpace{}%
\AgdaFunction{Rel}\AgdaSpace{}%
\AgdaBound{A}\AgdaSpace{}%
\AgdaBound{𝓡}\AgdaSymbol{\}}\<%
\\
\>[2]\AgdaSymbol{→}%
\>[14]\AgdaSymbol{(∀}\AgdaSpace{}%
\AgdaBound{a}\AgdaSpace{}%
\AgdaBound{x}\AgdaSpace{}%
\AgdaSymbol{→}\AgdaSpace{}%
\AgdaFunction{is-subsingleton}\AgdaSpace{}%
\AgdaSymbol{(}\AgdaBound{R}\AgdaSpace{}%
\AgdaBound{a}\AgdaSpace{}%
\AgdaBound{x}\AgdaSymbol{))}\<%
\\
\>[2]\AgdaSymbol{→}%
\>[14]\AgdaRecord{IsEquivalence}\AgdaSpace{}%
\AgdaBound{R}\<%
\\
\>[14]\AgdaComment{------------------------}\<%
\\
%
\>[2]\AgdaSymbol{→}%
\>[14]\AgdaBound{R}\AgdaSpace{}%
\AgdaBound{a}\AgdaSpace{}%
\AgdaBound{a'}\AgdaSpace{}%
\AgdaSymbol{→}\AgdaSpace{}%
\AgdaSymbol{(}\AgdaOperator{\AgdaFunction{[}}\AgdaSpace{}%
\AgdaBound{a}\AgdaSpace{}%
\AgdaOperator{\AgdaFunction{]}}\AgdaSpace{}%
\AgdaBound{R}\AgdaSymbol{)}\AgdaSpace{}%
\AgdaOperator{\AgdaDatatype{≡}}\AgdaSpace{}%
\AgdaSymbol{(}\AgdaOperator{\AgdaFunction{[}}\AgdaSpace{}%
\AgdaBound{a'}\AgdaSpace{}%
\AgdaOperator{\AgdaFunction{]}}\AgdaSpace{}%
\AgdaBound{R}\AgdaSymbol{)}\<%
\\
%
\\[\AgdaEmptyExtraSkip]%
%
\>[1]\AgdaFunction{class-extensionality}\AgdaSpace{}%
\AgdaBound{pe}\AgdaSpace{}%
\AgdaBound{gfe}\AgdaSpace{}%
\AgdaSymbol{\{}\AgdaArgument{A}\AgdaSpace{}%
\AgdaSymbol{=}\AgdaSpace{}%
\AgdaBound{A}\AgdaSymbol{\}\{}\AgdaBound{a}\AgdaSymbol{\}\{}\AgdaBound{a'}\AgdaSymbol{\}\{}\AgdaBound{R}\AgdaSymbol{\}}\AgdaSpace{}%
\AgdaBound{ssR}\AgdaSpace{}%
\AgdaBound{Req}\AgdaSpace{}%
\AgdaBound{Raa'}\AgdaSpace{}%
\AgdaSymbol{=}\<%
\\
\>[1][@{}l@{\AgdaIndent{0}}]%
\>[2]\AgdaFunction{Pred-=̇-≡}\AgdaSpace{}%
\AgdaBound{pe}\AgdaSpace{}%
\AgdaBound{gfe}\AgdaSpace{}%
\AgdaSymbol{\{}\AgdaBound{A}\AgdaSymbol{\}\{}\AgdaOperator{\AgdaFunction{[}}\AgdaSpace{}%
\AgdaBound{a}\AgdaSpace{}%
\AgdaOperator{\AgdaFunction{]}}\AgdaSpace{}%
\AgdaBound{R}\AgdaSymbol{\}\{}\AgdaOperator{\AgdaFunction{[}}\AgdaSpace{}%
\AgdaBound{a'}\AgdaSpace{}%
\AgdaOperator{\AgdaFunction{]}}\AgdaSpace{}%
\AgdaBound{R}\AgdaSymbol{\}}\AgdaSpace{}%
\AgdaSymbol{(}\AgdaBound{ssR}\AgdaSpace{}%
\AgdaBound{a}\AgdaSymbol{)}\AgdaSpace{}%
\AgdaSymbol{(}\AgdaBound{ssR}\AgdaSpace{}%
\AgdaBound{a'}\AgdaSymbol{)}\AgdaSpace{}%
\AgdaSymbol{(}\AgdaFunction{/-=̇}\AgdaSpace{}%
\AgdaBound{Req}\AgdaSpace{}%
\AgdaBound{Raa'}\AgdaSymbol{)}\<%
\\
%
\\[\AgdaEmptyExtraSkip]%
%
\>[1]\AgdaFunction{to-subtype-⟦⟧}\AgdaSpace{}%
\AgdaSymbol{:}%
\>[533I]\AgdaSymbol{\{}\AgdaBound{A}\AgdaSpace{}%
\AgdaSymbol{:}\AgdaSpace{}%
\AgdaBound{𝓤}\AgdaSpace{}%
\AgdaOperator{\AgdaFunction{̇}}\AgdaSymbol{\}\{}\AgdaBound{R}\AgdaSpace{}%
\AgdaSymbol{:}\AgdaSpace{}%
\AgdaFunction{Rel}\AgdaSpace{}%
\AgdaBound{A}\AgdaSpace{}%
\AgdaBound{𝓡}\AgdaSymbol{\}\{}\AgdaBound{C}\AgdaSpace{}%
\AgdaBound{D}\AgdaSpace{}%
\AgdaSymbol{:}\AgdaSpace{}%
\AgdaFunction{Pred}\AgdaSpace{}%
\AgdaBound{A}\AgdaSpace{}%
\AgdaBound{𝓡}\AgdaSymbol{\}}\<%
\\
\>[.][@{}l@{}]\<[533I]%
\>[17]\AgdaSymbol{\{}\AgdaBound{c}\AgdaSpace{}%
\AgdaSymbol{:}\AgdaSpace{}%
\AgdaFunction{𝒞}\AgdaSpace{}%
\AgdaBound{C}\AgdaSymbol{\}\{}\AgdaBound{d}\AgdaSpace{}%
\AgdaSymbol{:}\AgdaSpace{}%
\AgdaFunction{𝒞}\AgdaSpace{}%
\AgdaBound{D}\AgdaSymbol{\}}\<%
\\
\>[1][@{}l@{\AgdaIndent{0}}]%
\>[2]\AgdaSymbol{→}%
\>[17]\AgdaSymbol{(∀}\AgdaSpace{}%
\AgdaBound{C}\AgdaSpace{}%
\AgdaSymbol{→}\AgdaSpace{}%
\AgdaFunction{is-subsingleton}\AgdaSpace{}%
\AgdaSymbol{(}\AgdaFunction{𝒞}\AgdaSymbol{\{}\AgdaBound{A}\AgdaSymbol{\}\{}\AgdaBound{R}\AgdaSymbol{\}}\AgdaSpace{}%
\AgdaBound{C}\AgdaSymbol{))}\<%
\\
%
\>[2]\AgdaSymbol{→}%
\>[17]\AgdaBound{C}\AgdaSpace{}%
\AgdaOperator{\AgdaDatatype{≡}}\AgdaSpace{}%
\AgdaBound{D}%
\>[24]\AgdaSymbol{→}%
\>[27]\AgdaSymbol{(}\AgdaBound{C}\AgdaSpace{}%
\AgdaOperator{\AgdaInductiveConstructor{,}}\AgdaSpace{}%
\AgdaBound{c}\AgdaSymbol{)}\AgdaSpace{}%
\AgdaOperator{\AgdaDatatype{≡}}\AgdaSpace{}%
\AgdaSymbol{(}\AgdaBound{D}\AgdaSpace{}%
\AgdaOperator{\AgdaInductiveConstructor{,}}\AgdaSpace{}%
\AgdaBound{d}\AgdaSymbol{)}\<%
\\
%
\\[\AgdaEmptyExtraSkip]%
%
\>[1]\AgdaFunction{to-subtype-⟦⟧}\AgdaSpace{}%
\AgdaSymbol{\{}\AgdaArgument{D}\AgdaSpace{}%
\AgdaSymbol{=}\AgdaSpace{}%
\AgdaBound{D}\AgdaSymbol{\}\{}\AgdaBound{c}\AgdaSymbol{\}\{}\AgdaBound{d}\AgdaSymbol{\}}\AgdaSpace{}%
\AgdaBound{ssA}\AgdaSpace{}%
\AgdaBound{CD}\AgdaSpace{}%
\AgdaSymbol{=}\AgdaSpace{}%
\AgdaFunction{to-Σ-≡}\AgdaSpace{}%
\AgdaSymbol{(}\AgdaBound{CD}\AgdaSpace{}%
\AgdaOperator{\AgdaInductiveConstructor{,}}\AgdaSpace{}%
\AgdaBound{ssA}\AgdaSpace{}%
\AgdaBound{D}\AgdaSpace{}%
\AgdaSymbol{(}\AgdaFunction{transport}\AgdaSpace{}%
\AgdaFunction{𝒞}\AgdaSpace{}%
\AgdaBound{CD}\AgdaSpace{}%
\AgdaBound{c}\AgdaSymbol{)}\AgdaSpace{}%
\AgdaBound{d}\AgdaSymbol{)}\<%
\\
%
\\[\AgdaEmptyExtraSkip]%
%
\>[1]\AgdaFunction{class-extensionality'}\AgdaSpace{}%
\AgdaSymbol{:}%
\>[200I]\AgdaFunction{propext}\AgdaSpace{}%
\AgdaBound{𝓡}\AgdaSpace{}%
\AgdaSymbol{→}\AgdaSpace{}%
\AgdaFunction{global-dfunext}\<%
\\
\>[2]\AgdaSymbol{→}%
\>[.][@{}l@{}]\<[200I]%
\>[14]\AgdaSymbol{\{}\AgdaBound{A}\AgdaSpace{}%
\AgdaSymbol{:}\AgdaSpace{}%
\AgdaBound{𝓤}\AgdaSpace{}%
\AgdaOperator{\AgdaFunction{̇}}\AgdaSymbol{\}\{}\AgdaBound{a}\AgdaSpace{}%
\AgdaBound{a'}\AgdaSpace{}%
\AgdaSymbol{:}\AgdaSpace{}%
\AgdaBound{A}\AgdaSymbol{\}\{}\AgdaBound{R}\AgdaSpace{}%
\AgdaSymbol{:}\AgdaSpace{}%
\AgdaFunction{Rel}\AgdaSpace{}%
\AgdaBound{A}\AgdaSpace{}%
\AgdaBound{𝓡}\AgdaSymbol{\}}\<%
\\
\>[2]\AgdaSymbol{→}%
\>[14]\AgdaSymbol{(∀}\AgdaSpace{}%
\AgdaBound{a}\AgdaSpace{}%
\AgdaBound{x}\AgdaSpace{}%
\AgdaSymbol{→}\AgdaSpace{}%
\AgdaFunction{is-subsingleton}\AgdaSpace{}%
\AgdaSymbol{(}\AgdaBound{R}\AgdaSpace{}%
\AgdaBound{a}\AgdaSpace{}%
\AgdaBound{x}\AgdaSymbol{))}\<%
\\
\>[2]\AgdaSymbol{→}%
\>[14]\AgdaSymbol{(∀}\AgdaSpace{}%
\AgdaBound{C}\AgdaSpace{}%
\AgdaSymbol{→}\AgdaSpace{}%
\AgdaFunction{is-subsingleton}\AgdaSpace{}%
\AgdaSymbol{(}\AgdaFunction{𝒞}\AgdaSpace{}%
\AgdaBound{C}\AgdaSymbol{))}\<%
\\
\>[2]\AgdaSymbol{→}%
\>[14]\AgdaRecord{IsEquivalence}\AgdaSpace{}%
\AgdaBound{R}\<%
\\
\>[14]\AgdaComment{---------------------------------------}\<%
\\
%
\>[2]\AgdaSymbol{→}%
\>[14]\AgdaBound{R}\AgdaSpace{}%
\AgdaBound{a}\AgdaSpace{}%
\AgdaBound{a'}\AgdaSpace{}%
\AgdaSymbol{→}\AgdaSpace{}%
\AgdaSymbol{(}\AgdaOperator{\AgdaFunction{⟦}}\AgdaSpace{}%
\AgdaBound{a}\AgdaSpace{}%
\AgdaOperator{\AgdaFunction{⟧}}\AgdaSpace{}%
\AgdaSymbol{\{}\AgdaBound{R}\AgdaSymbol{\})}\AgdaSpace{}%
\AgdaOperator{\AgdaDatatype{≡}}\AgdaSpace{}%
\AgdaSymbol{(}\AgdaOperator{\AgdaFunction{⟦}}\AgdaSpace{}%
\AgdaBound{a'}\AgdaSpace{}%
\AgdaOperator{\AgdaFunction{⟧}}\AgdaSpace{}%
\AgdaSymbol{\{}\AgdaBound{R}\AgdaSymbol{\})}\<%
\\
%
\\[\AgdaEmptyExtraSkip]%
%
\>[1]\AgdaFunction{class-extensionality'}\AgdaSpace{}%
\AgdaBound{pe}\AgdaSpace{}%
\AgdaBound{gfe}\AgdaSpace{}%
\AgdaSymbol{\{}\AgdaArgument{A}\AgdaSpace{}%
\AgdaSymbol{=}\AgdaSpace{}%
\AgdaBound{A}\AgdaSymbol{\}\{}\AgdaBound{a}\AgdaSymbol{\}\{}\AgdaBound{a'}\AgdaSymbol{\}\{}\AgdaBound{R}\AgdaSymbol{\}}\AgdaSpace{}%
\AgdaBound{ssR}\AgdaSpace{}%
\AgdaBound{ssA}\AgdaSpace{}%
\AgdaBound{Req}\AgdaSpace{}%
\AgdaBound{Raa'}\AgdaSpace{}%
\AgdaSymbol{=}\AgdaSpace{}%
\AgdaFunction{γ}\<%
\\
\>[1][@{}l@{\AgdaIndent{0}}]%
\>[2]\AgdaKeyword{where}\<%
\\
\>[2][@{}l@{\AgdaIndent{0}}]%
\>[3]\AgdaFunction{CD}\AgdaSpace{}%
\AgdaSymbol{:}\AgdaSpace{}%
\AgdaSymbol{(}\AgdaOperator{\AgdaFunction{[}}\AgdaSpace{}%
\AgdaBound{a}\AgdaSpace{}%
\AgdaOperator{\AgdaFunction{]}}\AgdaSpace{}%
\AgdaBound{R}\AgdaSymbol{)}\AgdaSpace{}%
\AgdaOperator{\AgdaDatatype{≡}}\AgdaSpace{}%
\AgdaSymbol{(}\AgdaOperator{\AgdaFunction{[}}\AgdaSpace{}%
\AgdaBound{a'}\AgdaSpace{}%
\AgdaOperator{\AgdaFunction{]}}\AgdaSpace{}%
\AgdaBound{R}\AgdaSymbol{)}\<%
\\
%
\>[3]\AgdaFunction{CD}\AgdaSpace{}%
\AgdaSymbol{=}\AgdaSpace{}%
\AgdaFunction{class-extensionality}\AgdaSpace{}%
\AgdaBound{pe}\AgdaSpace{}%
\AgdaBound{gfe}\AgdaSpace{}%
\AgdaSymbol{\{}\AgdaBound{A}\AgdaSymbol{\}\{}\AgdaBound{a}\AgdaSymbol{\}\{}\AgdaBound{a'}\AgdaSymbol{\}\{}\AgdaBound{R}\AgdaSymbol{\}}\AgdaSpace{}%
\AgdaBound{ssR}\AgdaSpace{}%
\AgdaBound{Req}\AgdaSpace{}%
\AgdaBound{Raa'}\<%
\\
%
\\[\AgdaEmptyExtraSkip]%
%
\>[3]\AgdaFunction{γ}\AgdaSpace{}%
\AgdaSymbol{:}\AgdaSpace{}%
\AgdaSymbol{(}\AgdaOperator{\AgdaFunction{⟦}}\AgdaSpace{}%
\AgdaBound{a}\AgdaSpace{}%
\AgdaOperator{\AgdaFunction{⟧}}\AgdaSpace{}%
\AgdaSymbol{\{}\AgdaBound{R}\AgdaSymbol{\})}\AgdaSpace{}%
\AgdaOperator{\AgdaDatatype{≡}}\AgdaSpace{}%
\AgdaSymbol{(}\AgdaOperator{\AgdaFunction{⟦}}\AgdaSpace{}%
\AgdaBound{a'}\AgdaSpace{}%
\AgdaOperator{\AgdaFunction{⟧}}\AgdaSpace{}%
\AgdaSymbol{\{}\AgdaBound{R}\AgdaSymbol{\})}\<%
\\
%
\>[3]\AgdaFunction{γ}\AgdaSpace{}%
\AgdaSymbol{=}\AgdaSpace{}%
\AgdaFunction{to-subtype-⟦⟧}\AgdaSpace{}%
\AgdaBound{ssA}\AgdaSpace{}%
\AgdaFunction{CD}\<%
\end{code}



\subsubsection{Compatibility}\label{compatibility}
The following definitions and lemmas are useful for asserting and proving facts about \textbf{compatibility} of relations and functions.
\ccpad
\begin{code}%
\>[0]\AgdaKeyword{module}\AgdaSpace{}%
\AgdaModule{\AgdaUnderscore{}}\AgdaSpace{}%
\AgdaSymbol{\{}\AgdaBound{𝓤}\AgdaSpace{}%
\AgdaBound{𝓥}\AgdaSpace{}%
\AgdaBound{𝓦}\AgdaSpace{}%
\AgdaSymbol{:}\AgdaSpace{}%
\AgdaPostulate{Universe}\AgdaSymbol{\}}\AgdaSpace{}%
\AgdaSymbol{\{}\AgdaBound{γ}\AgdaSpace{}%
\AgdaSymbol{:}\AgdaSpace{}%
\AgdaBound{𝓥}\AgdaSpace{}%
\AgdaOperator{\AgdaFunction{̇}}\AgdaSpace{}%
\AgdaSymbol{\}}\AgdaSpace{}%
\AgdaSymbol{\{}\AgdaBound{Z}\AgdaSpace{}%
\AgdaSymbol{:}\AgdaSpace{}%
\AgdaBound{𝓤}\AgdaSpace{}%
\AgdaOperator{\AgdaFunction{̇}}\AgdaSpace{}%
\AgdaSymbol{\}}\AgdaSpace{}%
\AgdaKeyword{where}\<%
\\
%
\\[\AgdaEmptyExtraSkip]%
\>[0][@{}l@{\AgdaIndent{0}}]%
\>[1]\AgdaFunction{lift-rel}\AgdaSpace{}%
\AgdaSymbol{:}\AgdaSpace{}%
\AgdaFunction{Rel}\AgdaSpace{}%
\AgdaBound{Z}\AgdaSpace{}%
\AgdaBound{𝓦}\AgdaSpace{}%
\AgdaSymbol{→}\AgdaSpace{}%
\AgdaSymbol{(}\AgdaBound{γ}\AgdaSpace{}%
\AgdaSymbol{→}\AgdaSpace{}%
\AgdaBound{Z}\AgdaSymbol{)}\AgdaSpace{}%
\AgdaSymbol{→}\AgdaSpace{}%
\AgdaSymbol{(}\AgdaBound{γ}\AgdaSpace{}%
\AgdaSymbol{→}\AgdaSpace{}%
\AgdaBound{Z}\AgdaSymbol{)}\AgdaSpace{}%
\AgdaSymbol{→}\AgdaSpace{}%
\AgdaBound{𝓥}\AgdaSpace{}%
\AgdaOperator{\AgdaPrimitive{⊔}}\AgdaSpace{}%
\AgdaBound{𝓦}\AgdaSpace{}%
\AgdaOperator{\AgdaFunction{̇}}\<%
\\
%
\>[1]\AgdaFunction{lift-rel}\AgdaSpace{}%
\AgdaBound{R}\AgdaSpace{}%
\AgdaBound{f}\AgdaSpace{}%
\AgdaBound{g}\AgdaSpace{}%
\AgdaSymbol{=}\AgdaSpace{}%
\AgdaSymbol{∀}\AgdaSpace{}%
\AgdaBound{x}\AgdaSpace{}%
\AgdaSymbol{→}\AgdaSpace{}%
\AgdaBound{R}\AgdaSpace{}%
\AgdaSymbol{(}\AgdaBound{f}\AgdaSpace{}%
\AgdaBound{x}\AgdaSymbol{)}\AgdaSpace{}%
\AgdaSymbol{(}\AgdaBound{g}\AgdaSpace{}%
\AgdaBound{x}\AgdaSymbol{)}\<%
\\
%
\\[\AgdaEmptyExtraSkip]%
%
\>[1]\AgdaFunction{compatible-fun}\AgdaSpace{}%
\AgdaSymbol{:}\AgdaSpace{}%
\AgdaSymbol{(}\AgdaBound{f}\AgdaSpace{}%
\AgdaSymbol{:}\AgdaSpace{}%
\AgdaSymbol{(}\AgdaBound{γ}\AgdaSpace{}%
\AgdaSymbol{→}\AgdaSpace{}%
\AgdaBound{Z}\AgdaSymbol{)}\AgdaSpace{}%
\AgdaSymbol{→}\AgdaSpace{}%
\AgdaBound{Z}\AgdaSymbol{)(}\AgdaBound{R}\AgdaSpace{}%
\AgdaSymbol{:}\AgdaSpace{}%
\AgdaFunction{Rel}\AgdaSpace{}%
\AgdaBound{Z}\AgdaSpace{}%
\AgdaBound{𝓦}\AgdaSymbol{)}\AgdaSpace{}%
\AgdaSymbol{→}\AgdaSpace{}%
\AgdaBound{𝓥}\AgdaSpace{}%
\AgdaOperator{\AgdaPrimitive{⊔}}\AgdaSpace{}%
\AgdaBound{𝓤}\AgdaSpace{}%
\AgdaOperator{\AgdaPrimitive{⊔}}\AgdaSpace{}%
\AgdaBound{𝓦}\AgdaSpace{}%
\AgdaOperator{\AgdaFunction{̇}}\<%
\\
%
\>[1]\AgdaFunction{compatible-fun}\AgdaSpace{}%
\AgdaBound{f}\AgdaSpace{}%
\AgdaBound{R}%
\>[21]\AgdaSymbol{=}\AgdaSpace{}%
\AgdaSymbol{(}\AgdaFunction{lift-rel}\AgdaSpace{}%
\AgdaBound{R}\AgdaSymbol{)}\AgdaSpace{}%
\AgdaOperator{\AgdaFunction{=[}}\AgdaSpace{}%
\AgdaBound{f}\AgdaSpace{}%
\AgdaOperator{\AgdaFunction{]⇒}}\AgdaSpace{}%
\AgdaBound{R}\<%
\\
%
\\[\AgdaEmptyExtraSkip]%
\>[0]\AgdaComment{-- relation compatible with an operation}\<%
\\
\>[0]\AgdaKeyword{module}\AgdaSpace{}%
\AgdaModule{\AgdaUnderscore{}}\AgdaSpace{}%
\AgdaSymbol{\{}\AgdaBound{𝓤}\AgdaSpace{}%
\AgdaBound{𝓦}\AgdaSpace{}%
\AgdaSymbol{:}\AgdaSpace{}%
\AgdaPostulate{Universe}\AgdaSymbol{\}}\AgdaSpace{}%
\AgdaSymbol{\{}\AgdaBound{𝑆}\AgdaSpace{}%
\AgdaSymbol{:}\AgdaSpace{}%
\AgdaFunction{Signature}\AgdaSpace{}%
\AgdaGeneralizable{𝓞}\AgdaSpace{}%
\AgdaGeneralizable{𝓥}\AgdaSymbol{\}}\AgdaSpace{}%
\AgdaKeyword{where}\<%
\\
\>[0][@{}l@{\AgdaIndent{0}}]%
\>[1]\AgdaFunction{compatible-op}\AgdaSpace{}%
\AgdaSymbol{:}\AgdaSpace{}%
\AgdaSymbol{\{}\AgdaBound{𝑨}\AgdaSpace{}%
\AgdaSymbol{:}\AgdaSpace{}%
\AgdaFunction{Algebra}\AgdaSpace{}%
\AgdaBound{𝓤}\AgdaSpace{}%
\AgdaBound{𝑆}\AgdaSymbol{\}}\AgdaSpace{}%
\AgdaSymbol{→}\AgdaSpace{}%
\AgdaOperator{\AgdaFunction{∣}}\AgdaSpace{}%
\AgdaBound{𝑆}\AgdaSpace{}%
\AgdaOperator{\AgdaFunction{∣}}\AgdaSpace{}%
\AgdaSymbol{→}\AgdaSpace{}%
\AgdaFunction{Rel}\AgdaSpace{}%
\AgdaOperator{\AgdaFunction{∣}}\AgdaSpace{}%
\AgdaBound{𝑨}\AgdaSpace{}%
\AgdaOperator{\AgdaFunction{∣}}\AgdaSpace{}%
\AgdaBound{𝓦}\AgdaSpace{}%
\AgdaSymbol{→}\AgdaSpace{}%
\AgdaBound{𝓤}\AgdaSpace{}%
\AgdaOperator{\AgdaPrimitive{⊔}}\AgdaSpace{}%
\AgdaBound{𝓥}\AgdaSpace{}%
\AgdaOperator{\AgdaPrimitive{⊔}}\AgdaSpace{}%
\AgdaBound{𝓦}\AgdaSpace{}%
\AgdaOperator{\AgdaFunction{̇}}\<%
\\
%
\>[1]\AgdaFunction{compatible-op}\AgdaSpace{}%
\AgdaSymbol{\{}\AgdaBound{𝑨}\AgdaSymbol{\}}\AgdaSpace{}%
\AgdaBound{f}\AgdaSpace{}%
\AgdaBound{R}\AgdaSpace{}%
\AgdaSymbol{=}\AgdaSpace{}%
\AgdaSymbol{∀\{}\AgdaBound{𝒂}\AgdaSymbol{\}\{}\AgdaBound{𝒃}\AgdaSymbol{\}}\AgdaSpace{}%
\AgdaSymbol{→}\AgdaSpace{}%
\AgdaSymbol{(}\AgdaFunction{lift-rel}\AgdaSpace{}%
\AgdaBound{R}\AgdaSymbol{)}\AgdaSpace{}%
\AgdaBound{𝒂}\AgdaSpace{}%
\AgdaBound{𝒃}%
\>[53]\AgdaSymbol{→}\AgdaSpace{}%
\AgdaBound{R}\AgdaSpace{}%
\AgdaSymbol{((}\AgdaBound{f}\AgdaSpace{}%
\AgdaOperator{\AgdaFunction{̂}}\AgdaSpace{}%
\AgdaBound{𝑨}\AgdaSymbol{)}\AgdaSpace{}%
\AgdaBound{𝒂}\AgdaSymbol{)}\AgdaSpace{}%
\AgdaSymbol{((}\AgdaBound{f}\AgdaSpace{}%
\AgdaOperator{\AgdaFunction{̂}}\AgdaSpace{}%
\AgdaBound{𝑨}\AgdaSymbol{)}\AgdaSpace{}%
\AgdaBound{𝒃}\AgdaSymbol{)}\<%
\\
%
\>[1]\AgdaComment{-- alternative notation: (lift-rel R) =[ f ̂ 𝑨 ]⇒ R}\<%
\\
%
\\[\AgdaEmptyExtraSkip]%
\>[0]\AgdaComment{--The given relation is compatible with all ops of an algebra.}\<%
\\
\>[0][@{}l@{\AgdaIndent{0}}]%
\>[1]\AgdaFunction{compatible}\AgdaSpace{}%
\AgdaSymbol{:}{\AgdaSpace{}}%
\AgdaSymbol{(}\AgdaBound{𝑨}\AgdaSpace{}%
\AgdaSymbol{:}\AgdaSpace{}%
\AgdaFunction{Algebra}\AgdaSpace{}%
\AgdaBound{𝓤}\AgdaSpace{}%
\AgdaBound{𝑆}\AgdaSymbol{)}\AgdaSpace{}%
\AgdaSymbol{→}\AgdaSpace{}%
\AgdaFunction{Rel}\AgdaSpace{}%
\AgdaOperator{\AgdaFunction{∣}}\AgdaSpace{}%
\AgdaBound{𝑨}\AgdaSpace{}%
\AgdaOperator{\AgdaFunction{∣}}\AgdaSpace{}%
\AgdaBound{𝓦}\AgdaSpace{}%
\AgdaSymbol{→}\AgdaSpace{}%
\AgdaBound{𝓞}\AgdaSpace{}%
\AgdaOperator{\AgdaPrimitive{⊔}}\AgdaSpace{}%
\AgdaBound{𝓤}\AgdaSpace{}%
\AgdaOperator{\AgdaPrimitive{⊔}}\AgdaSpace{}%
\AgdaBound{𝓥}\AgdaSpace{}%
\AgdaOperator{\AgdaPrimitive{⊔}}\AgdaSpace{}%
\AgdaBound{𝓦}\AgdaSpace{}%
\AgdaOperator{\AgdaFunction{̇}}\<%
\\
%
\>[1]\AgdaFunction{compatible}%
\>[13]\AgdaBound{𝑨}\AgdaSpace{}%
\AgdaBound{R}\AgdaSpace{}%
\AgdaSymbol{=}\AgdaSpace{}%
\AgdaSymbol{∀}\AgdaSpace{}%
\AgdaBound{f}\AgdaSpace{}%
\AgdaSymbol{→}\AgdaSpace{}%
\AgdaFunction{compatible-op}\AgdaSymbol{\{}\AgdaBound{𝑨}\AgdaSymbol{\}}\AgdaSpace{}%
\AgdaBound{f}\AgdaSpace{}%
\AgdaBound{R}\<%
\end{code}
\ccpad
We'll see this definition of compatibility at work very soon when we define congruence relations in the next section.


\newcommand\siOV{\AgdaFunction{Signature}\AgdaSpace{}\AgdaBound{𝓞}\AgdaSpace{}\AgdaBound{𝓥}\xspace}

\subsection{Congruence Relation Types}\label{sec:congr-relat}
This section describes  parts of the \ualibCongruences module of the \agdaualib.
%% -- BEGIN: \>[0]\AgdaFunction{Con}\AgdaSpace{}%
\AgdaSymbol{:}\AgdaSpace{}%
\AgdaSymbol{\{}\AgdaBound{𝓤}\AgdaSpace{}%
\AgdaSymbol{:}\AgdaSpace{}%
\AgdaPostulate{Universe}\AgdaSymbol{\}(}\AgdaBound{A}\AgdaSpace{}%
\AgdaSymbol{:}\AgdaSpace{}%
\AgdaFunction{Algebra}\AgdaSpace{}%
\AgdaBound{𝓤}\AgdaSpace{}%
\AgdaBound{𝑆}\AgdaSymbol{)}\AgdaSpace{}%
\AgdaSymbol{→}\AgdaSpace{}%
\AgdaBound{𝓞}\AgdaSpace{}%
\AgdaOperator{\AgdaPrimitive{⊔}}\AgdaSpace{}%
\AgdaBound{𝓥}\AgdaSpace{}%
\AgdaOperator{\AgdaPrimitive{⊔}}\AgdaSpace{}%
\AgdaBound{𝓤}\AgdaSpace{}%
\AgdaOperator{\AgdaPrimitive{⁺}}\AgdaSpace{}%
\AgdaOperator{\AgdaFunction{̇}}\<%
\\
\>[0]\AgdaFunction{Con}\AgdaSpace{}%
\AgdaSymbol{\{}\AgdaBound{𝓤}\AgdaSymbol{\}}\AgdaSpace{}%
\AgdaBound{A}\AgdaSpace{}%
\AgdaSymbol{=}\AgdaSpace{}%
\AgdaFunction{Σ}\AgdaSpace{}%
\AgdaBound{θ}\AgdaSpace{}%
\AgdaFunction{꞉}\AgdaSpace{}%
\AgdaSymbol{(}\AgdaSpace{}%
\AgdaFunction{Rel}\AgdaSpace{}%
\AgdaOperator{\AgdaFunction{∣}}\AgdaSpace{}%
\AgdaBound{A}\AgdaSpace{}%
\AgdaOperator{\AgdaFunction{∣}}\AgdaSpace{}%
\AgdaBound{𝓤}\AgdaSpace{}%
\AgdaSymbol{)}\AgdaSpace{}%
\AgdaFunction{,}\AgdaSpace{}%
\AgdaRecord{IsEquivalence}\AgdaSpace{}%
\AgdaBound{θ}\AgdaSpace{}%
\AgdaOperator{\AgdaFunction{×}}\AgdaSpace{}%
\AgdaFunction{compatible}\AgdaSpace{}%
\AgdaBound{A}\AgdaSpace{}%
\AgdaBound{θ}\<%
\\
%
\\[\AgdaEmptyExtraSkip]%
\>[0]\AgdaFunction{con}\AgdaSpace{}%
\AgdaSymbol{:}\AgdaSpace{}%
\AgdaSymbol{\{}\AgdaBound{𝓤}\AgdaSpace{}%
\AgdaSymbol{:}\AgdaSpace{}%
\AgdaPostulate{Universe}\AgdaSymbol{\}(}\AgdaBound{A}\AgdaSpace{}%
\AgdaSymbol{:}\AgdaSpace{}%
\AgdaFunction{Algebra}\AgdaSpace{}%
\AgdaBound{𝓤}\AgdaSpace{}%
\AgdaBound{𝑆}\AgdaSymbol{)}%
\>[39]\AgdaSymbol{→}%
\>[42]\AgdaFunction{Pred}\AgdaSpace{}%
\AgdaSymbol{(}\AgdaFunction{Rel}\AgdaSpace{}%
\AgdaOperator{\AgdaFunction{∣}}\AgdaSpace{}%
\AgdaBound{A}\AgdaSpace{}%
\AgdaOperator{\AgdaFunction{∣}}\AgdaSpace{}%
\AgdaBound{𝓤}\AgdaSymbol{)}\AgdaSpace{}%
\AgdaSymbol{(}\AgdaBound{𝓞}\AgdaSpace{}%
\AgdaOperator{\AgdaPrimitive{⊔}}\AgdaSpace{}%
\AgdaBound{𝓥}\AgdaSpace{}%
\AgdaOperator{\AgdaPrimitive{⊔}}\AgdaSpace{}%
\AgdaBound{𝓤}\AgdaSymbol{)}\<%
\\
\>[0]\AgdaFunction{con}\AgdaSpace{}%
\AgdaBound{A}\AgdaSpace{}%
\AgdaSymbol{=}\AgdaSpace{}%
\AgdaSymbol{λ}\AgdaSpace{}%
\AgdaBound{θ}\AgdaSpace{}%
\AgdaSymbol{→}\AgdaSpace{}%
\AgdaRecord{IsEquivalence}\AgdaSpace{}%
\AgdaBound{θ}\AgdaSpace{}%
\AgdaOperator{\AgdaFunction{×}}\AgdaSpace{}%
\AgdaFunction{compatible}\AgdaSpace{}%
\AgdaBound{A}\AgdaSpace{}%
\AgdaBound{θ}\<%
\\
%
\\[\AgdaEmptyExtraSkip]%
\>[0]\AgdaKeyword{record}\AgdaSpace{}%
\AgdaRecord{Congruence}\AgdaSpace{}%
\AgdaSymbol{\{}\AgdaBound{𝓤}\AgdaSpace{}%
\AgdaBound{𝓦}\AgdaSpace{}%
\AgdaSymbol{:}\AgdaSpace{}%
\AgdaPostulate{Universe}\AgdaSymbol{\}}\AgdaSpace{}%
\AgdaSymbol{(}\AgdaBound{A}\AgdaSpace{}%
\AgdaSymbol{:}\AgdaSpace{}%
\AgdaFunction{Algebra}\AgdaSpace{}%
\AgdaBound{𝓤}\AgdaSpace{}%
\AgdaBound{𝑆}\AgdaSymbol{)}\AgdaSpace{}%
\AgdaSymbol{:}\AgdaSpace{}%
\AgdaBound{𝓞}\AgdaSpace{}%
\AgdaOperator{\AgdaPrimitive{⊔}}\AgdaSpace{}%
\AgdaBound{𝓥}\AgdaSpace{}%
\AgdaOperator{\AgdaPrimitive{⊔}}\AgdaSpace{}%
\AgdaBound{𝓤}\AgdaSpace{}%
\AgdaOperator{\AgdaPrimitive{⊔}}\AgdaSpace{}%
\AgdaBound{𝓦}\AgdaSpace{}%
\AgdaOperator{\AgdaPrimitive{⁺}}\AgdaSpace{}%
\AgdaOperator{\AgdaFunction{̇}}%
\>[74]\AgdaKeyword{where}\<%
\\
\>[0][@{}l@{\AgdaIndent{0}}]%
\>[1]\AgdaKeyword{constructor}\AgdaSpace{}%
\AgdaInductiveConstructor{mkcon}\<%
\\
%
\>[1]\AgdaKeyword{field}\<%
\\
\>[1][@{}l@{\AgdaIndent{0}}]%
\>[2]\AgdaOperator{\AgdaField{⟨\AgdaUnderscore{}⟩}}\AgdaSpace{}%
\AgdaSymbol{:}\AgdaSpace{}%
\AgdaFunction{Rel}\AgdaSpace{}%
\AgdaOperator{\AgdaFunction{∣}}\AgdaSpace{}%
\AgdaBound{A}\AgdaSpace{}%
\AgdaOperator{\AgdaFunction{∣}}\AgdaSpace{}%
\AgdaBound{𝓦}\<%
\\
%
\>[2]\AgdaField{Compatible}\AgdaSpace{}%
\AgdaSymbol{:}\AgdaSpace{}%
\AgdaFunction{compatible}\AgdaSpace{}%
\AgdaBound{A}\AgdaSpace{}%
\AgdaOperator{\AgdaField{⟨\AgdaUnderscore{}⟩}}\<%
\\
%
\>[2]\AgdaField{IsEquiv}\AgdaSpace{}%
\AgdaSymbol{:}\AgdaSpace{}%
\AgdaRecord{IsEquivalence}\AgdaSpace{}%
\AgdaOperator{\AgdaField{⟨\AgdaUnderscore{}⟩}}\<%
\\
%
\\[\AgdaEmptyExtraSkip]%
\>[0]\AgdaKeyword{open}\AgdaSpace{}%
\AgdaModule{Congruence}\<%
\\
%
\\[\AgdaEmptyExtraSkip]%
\>[0]\AgdaFunction{compatible-equivalence}\AgdaSpace{}%
\AgdaSymbol{:}\AgdaSpace{}%
\AgdaSymbol{\{}\AgdaBound{𝓤}\AgdaSpace{}%
\AgdaBound{𝓦}\AgdaSpace{}%
\AgdaSymbol{:}\AgdaSpace{}%
\AgdaPostulate{Universe}\AgdaSymbol{\}\{}\AgdaBound{𝑨}\AgdaSpace{}%
\AgdaSymbol{:}\AgdaSpace{}%
\AgdaFunction{Algebra}\AgdaSpace{}%
\AgdaBound{𝓤}\AgdaSpace{}%
\AgdaBound{𝑆}\AgdaSymbol{\}}\AgdaSpace{}%
\AgdaSymbol{→}\AgdaSpace{}%
\AgdaFunction{Rel}\AgdaSpace{}%
\AgdaOperator{\AgdaFunction{∣}}\AgdaSpace{}%
\AgdaBound{𝑨}\AgdaSpace{}%
\AgdaOperator{\AgdaFunction{∣}}\AgdaSpace{}%
\AgdaBound{𝓦}\AgdaSpace{}%
\AgdaSymbol{→}\AgdaSpace{}%
\AgdaBound{𝓞}\AgdaSpace{}%
\AgdaOperator{\AgdaPrimitive{⊔}}\AgdaSpace{}%
\AgdaBound{𝓥}\AgdaSpace{}%
\AgdaOperator{\AgdaPrimitive{⊔}}\AgdaSpace{}%
\AgdaBound{𝓦}\AgdaSpace{}%
\AgdaOperator{\AgdaPrimitive{⊔}}\AgdaSpace{}%
\AgdaBound{𝓤}\AgdaSpace{}%
\AgdaOperator{\AgdaFunction{̇}}\<%
\\
\>[0]\AgdaFunction{compatible-equivalence}\AgdaSpace{}%
\AgdaSymbol{\{}\AgdaBound{𝓤}\AgdaSymbol{\}\{}\AgdaBound{𝓦}\AgdaSymbol{\}}\AgdaSpace{}%
\AgdaSymbol{\{}\AgdaBound{𝑨}\AgdaSymbol{\}}\AgdaSpace{}%
\AgdaBound{R}\AgdaSpace{}%
\AgdaSymbol{=}\AgdaSpace{}%
\AgdaFunction{compatible}\AgdaSpace{}%
\AgdaBound{𝑨}\AgdaSpace{}%
\AgdaBound{R}\AgdaSpace{}%
\AgdaOperator{\AgdaFunction{×}}\AgdaSpace{}%
\AgdaRecord{IsEquivalence}\AgdaSpace{}%
\AgdaBound{R}\<%

We first present three ways to represent a congruence relation of an algebra---first as a function, then as a predicate on relations, and finally as a record.
%% Notice that module begins by assuming a signature \ab 𝑆 \as : \af{Signature} \ab 𝓞 \ab 𝓥 which is then present and available throughout the module.
\ccpad
\begin{code}%
\>[0]\AgdaFunction{Con}\AgdaSpace{}%
\AgdaSymbol{:}\AgdaSpace{}%
\AgdaSymbol{\{}\AgdaBound{𝓤}\AgdaSpace{}%
\AgdaSymbol{:}\AgdaSpace{}%
\AgdaPostulate{Universe}\AgdaSymbol{\}(}\AgdaBound{A}\AgdaSpace{}%
\AgdaSymbol{:}\AgdaSpace{}%
\AgdaFunction{Algebra}\AgdaSpace{}%
\AgdaBound{𝓤}\AgdaSpace{}%
\AgdaBound{𝑆}\AgdaSymbol{)}\AgdaSpace{}%
\AgdaSymbol{→}\AgdaSpace{}%
\AgdaBound{𝓞}\AgdaSpace{}%
\AgdaOperator{\AgdaPrimitive{⊔}}\AgdaSpace{}%
\AgdaBound{𝓥}\AgdaSpace{}%
\AgdaOperator{\AgdaPrimitive{⊔}}\AgdaSpace{}%
\AgdaBound{𝓤}\AgdaSpace{}%
\AgdaOperator{\AgdaPrimitive{⁺}}\AgdaSpace{}%
\AgdaOperator{\AgdaFunction{̇}}\<%
\\
\>[0]\AgdaFunction{Con}\AgdaSpace{}%
\AgdaSymbol{\{}\AgdaBound{𝓤}\AgdaSymbol{\}}\AgdaSpace{}%
\AgdaBound{A}\AgdaSpace{}%
\AgdaSymbol{=}\AgdaSpace{}%
\AgdaFunction{Σ}\AgdaSpace{}%
\AgdaBound{θ}\AgdaSpace{}%
\AgdaFunction{꞉}\AgdaSpace{}%
\AgdaSymbol{(}\AgdaSpace{}%
\AgdaFunction{Rel}\AgdaSpace{}%
\AgdaOperator{\AgdaFunction{∣}}\AgdaSpace{}%
\AgdaBound{A}\AgdaSpace{}%
\AgdaOperator{\AgdaFunction{∣}}\AgdaSpace{}%
\AgdaBound{𝓤}\AgdaSpace{}%
\AgdaSymbol{)}\AgdaSpace{}%
\AgdaFunction{,}\AgdaSpace{}%
\AgdaRecord{IsEquivalence}\AgdaSpace{}%
\AgdaBound{θ}\AgdaSpace{}%
\AgdaOperator{\AgdaFunction{×}}\AgdaSpace{}%
\AgdaFunction{compatible}\AgdaSpace{}%
\AgdaBound{A}\AgdaSpace{}%
\AgdaBound{θ}\<%
\\
%
\\[\AgdaEmptyExtraSkip]%
\>[0]\AgdaFunction{con}\AgdaSpace{}%
\AgdaSymbol{:}\AgdaSpace{}%
\AgdaSymbol{\{}\AgdaBound{𝓤}\AgdaSpace{}%
\AgdaSymbol{:}\AgdaSpace{}%
\AgdaPostulate{Universe}\AgdaSymbol{\}(}\AgdaBound{A}\AgdaSpace{}%
\AgdaSymbol{:}\AgdaSpace{}%
\AgdaFunction{Algebra}\AgdaSpace{}%
\AgdaBound{𝓤}\AgdaSpace{}%
\AgdaBound{𝑆}\AgdaSymbol{)}%
\>[39]\AgdaSymbol{→}%
\>[42]\AgdaFunction{Pred}\AgdaSpace{}%
\AgdaSymbol{(}\AgdaFunction{Rel}\AgdaSpace{}%
\AgdaOperator{\AgdaFunction{∣}}\AgdaSpace{}%
\AgdaBound{A}\AgdaSpace{}%
\AgdaOperator{\AgdaFunction{∣}}\AgdaSpace{}%
\AgdaBound{𝓤}\AgdaSymbol{)}\AgdaSpace{}%
\AgdaSymbol{(}\AgdaBound{𝓞}\AgdaSpace{}%
\AgdaOperator{\AgdaPrimitive{⊔}}\AgdaSpace{}%
\AgdaBound{𝓥}\AgdaSpace{}%
\AgdaOperator{\AgdaPrimitive{⊔}}\AgdaSpace{}%
\AgdaBound{𝓤}\AgdaSymbol{)}\<%
\\
\>[0]\AgdaFunction{con}\AgdaSpace{}%
\AgdaBound{A}\AgdaSpace{}%
\AgdaSymbol{=}\AgdaSpace{}%
\AgdaSymbol{λ}\AgdaSpace{}%
\AgdaBound{θ}\AgdaSpace{}%
\AgdaSymbol{→}\AgdaSpace{}%
\AgdaRecord{IsEquivalence}\AgdaSpace{}%
\AgdaBound{θ}\AgdaSpace{}%
\AgdaOperator{\AgdaFunction{×}}\AgdaSpace{}%
\AgdaFunction{compatible}\AgdaSpace{}%
\AgdaBound{A}\AgdaSpace{}%
\AgdaBound{θ}\<%
\\
%
\\[\AgdaEmptyExtraSkip]%
\>[0]\AgdaKeyword{record}\AgdaSpace{}%
\AgdaRecord{Congruence}\AgdaSpace{}%
\AgdaSymbol{\{}\AgdaBound{𝓤}\AgdaSpace{}%
\AgdaBound{𝓦}\AgdaSpace{}%
\AgdaSymbol{:}\AgdaSpace{}%
\AgdaPostulate{Universe}\AgdaSymbol{\}}\AgdaSpace{}%
\AgdaSymbol{(}\AgdaBound{A}\AgdaSpace{}%
\AgdaSymbol{:}\AgdaSpace{}%
\AgdaFunction{Algebra}\AgdaSpace{}%
\AgdaBound{𝓤}\AgdaSpace{}%
\AgdaBound{𝑆}\AgdaSymbol{)}\AgdaSpace{}%
\AgdaSymbol{:}\AgdaSpace{}%
\AgdaBound{𝓞}\AgdaSpace{}%
\AgdaOperator{\AgdaPrimitive{⊔}}\AgdaSpace{}%
\AgdaBound{𝓥}\AgdaSpace{}%
\AgdaOperator{\AgdaPrimitive{⊔}}\AgdaSpace{}%
\AgdaBound{𝓤}\AgdaSpace{}%
\AgdaOperator{\AgdaPrimitive{⊔}}\AgdaSpace{}%
\AgdaBound{𝓦}\AgdaSpace{}%
\AgdaOperator{\AgdaPrimitive{⁺}}\AgdaSpace{}%
\AgdaOperator{\AgdaFunction{̇}}%
\>[74]\AgdaKeyword{where}\<%
\\
\>[0][@{}l@{\AgdaIndent{0}}]%
\>[1]\AgdaKeyword{constructor}\AgdaSpace{}%
\AgdaInductiveConstructor{mkcon}\<%
\\
%
\>[1]\AgdaKeyword{field}\<%
\\
\>[1][@{}l@{\AgdaIndent{0}}]%
\>[2]\AgdaOperator{\AgdaField{⟨\AgdaUnderscore{}⟩}}\AgdaSpace{}%
\AgdaSymbol{:}\AgdaSpace{}%
\AgdaFunction{Rel}\AgdaSpace{}%
\AgdaOperator{\AgdaFunction{∣}}\AgdaSpace{}%
\AgdaBound{A}\AgdaSpace{}%
\AgdaOperator{\AgdaFunction{∣}}\AgdaSpace{}%
\AgdaBound{𝓦}\<%
\\
%
\>[2]\AgdaField{Compatible}\AgdaSpace{}%
\AgdaSymbol{:}\AgdaSpace{}%
\AgdaFunction{compatible}\AgdaSpace{}%
\AgdaBound{A}\AgdaSpace{}%
\AgdaOperator{\AgdaField{⟨\AgdaUnderscore{}⟩}}\<%
\\
%
\>[2]\AgdaField{IsEquiv}\AgdaSpace{}%
\AgdaSymbol{:}\AgdaSpace{}%
\AgdaRecord{IsEquivalence}\AgdaSpace{}%
\AgdaOperator{\AgdaField{⟨\AgdaUnderscore{}⟩}}\<%
\\
%
\\[\AgdaEmptyExtraSkip]%
\>[0]\AgdaKeyword{open}\AgdaSpace{}%
\AgdaModule{Congruence}\<%
\\
%
\\[\AgdaEmptyExtraSkip]%
\>[0]\AgdaFunction{compatible-equivalence}\AgdaSpace{}%
\AgdaSymbol{:}\AgdaSpace{}%
\AgdaSymbol{\{}\AgdaBound{𝓤}\AgdaSpace{}%
\AgdaBound{𝓦}\AgdaSpace{}%
\AgdaSymbol{:}\AgdaSpace{}%
\AgdaPostulate{Universe}\AgdaSymbol{\}\{}\AgdaBound{𝑨}\AgdaSpace{}%
\AgdaSymbol{:}\AgdaSpace{}%
\AgdaFunction{Algebra}\AgdaSpace{}%
\AgdaBound{𝓤}\AgdaSpace{}%
\AgdaBound{𝑆}\AgdaSymbol{\}}\AgdaSpace{}%
\AgdaSymbol{→}\AgdaSpace{}%
\AgdaFunction{Rel}\AgdaSpace{}%
\AgdaOperator{\AgdaFunction{∣}}\AgdaSpace{}%
\AgdaBound{𝑨}\AgdaSpace{}%
\AgdaOperator{\AgdaFunction{∣}}\AgdaSpace{}%
\AgdaBound{𝓦}\AgdaSpace{}%
\AgdaSymbol{→}\AgdaSpace{}%
\AgdaBound{𝓞}\AgdaSpace{}%
\AgdaOperator{\AgdaPrimitive{⊔}}\AgdaSpace{}%
\AgdaBound{𝓥}\AgdaSpace{}%
\AgdaOperator{\AgdaPrimitive{⊔}}\AgdaSpace{}%
\AgdaBound{𝓦}\AgdaSpace{}%
\AgdaOperator{\AgdaPrimitive{⊔}}\AgdaSpace{}%
\AgdaBound{𝓤}\AgdaSpace{}%
\AgdaOperator{\AgdaFunction{̇}}\<%
\\
\>[0]\AgdaFunction{compatible-equivalence}\AgdaSpace{}%
\AgdaSymbol{\{}\AgdaBound{𝓤}\AgdaSymbol{\}\{}\AgdaBound{𝓦}\AgdaSymbol{\}}\AgdaSpace{}%
\AgdaSymbol{\{}\AgdaBound{𝑨}\AgdaSymbol{\}}\AgdaSpace{}%
\AgdaBound{R}\AgdaSpace{}%
\AgdaSymbol{=}\AgdaSpace{}%
\AgdaFunction{compatible}\AgdaSpace{}%
\AgdaBound{𝑨}\AgdaSpace{}%
\AgdaBound{R}\AgdaSpace{}%
\AgdaOperator{\AgdaFunction{×}}\AgdaSpace{}%
\AgdaRecord{IsEquivalence}\AgdaSpace{}%
\AgdaBound{R}\<%
\end{code}

\subsubsection{Example}\label{Congruences.sssec:example}
We defined the \emph{trivial} (or ``diagonal'' or ``identity'' or ``zero'') relation \af{𝟎-rel} in \S\ref{sssec:examples}, and we observed in \S\ref{Equivalences.sssec:examples} that \af{𝟎-rel} is equivalent to the identity relation \aod ≡ and that these are both equivalence relations. Therefore, in order to build a congruence of some algebra \ab 𝑨 out of the trivial relation, it remains to show that \af{𝟎-rel} is compatible with all operations of \ab 𝑨.  We do this now and then construct the corresponding congruence.
%% \af{𝟎-rel}
\ccpad
\begin{code}%
\>[0]\AgdaKeyword{module}\AgdaSpace{}\AgdaModule{\AgdaUnderscore{}}\AgdaSpace{}%
\AgdaSymbol{\{}\AgdaBound{𝓤}\AgdaSpace{}\AgdaSymbol{:}\AgdaSpace{}%
\AgdaPostulate{Universe}\AgdaSymbol{\}}\AgdaSpace{}\AgdaSymbol{\{}\AgdaBound{𝑆}\AgdaSpace{}%
\AgdaSymbol{:}\AgdaSpace{}\AgdaFunction{Signature}\AgdaSpace{}\AgdaBound{𝓞}\AgdaSpace{}%
\AgdaBound{𝓥}\AgdaSymbol{\}}\AgdaSpace{}\AgdaKeyword{where}\<%
\\
%
\\[\AgdaEmptyExtraSkip]%
\>[0][@{}l@{\AgdaIndent{0}}]%
\>[1]\AgdaFunction{𝟎-compatible-op}\AgdaSpace{}%
\AgdaSymbol{:}%
\>[200I]\AgdaFunction{funext}\AgdaSpace{}%
\AgdaBound{𝓥}\AgdaSpace{}%
\AgdaBound{𝓤}\AgdaSpace{}%
\AgdaSymbol{→}\AgdaSpace{}%
\AgdaSymbol{\{}\AgdaBound{𝑨}\AgdaSpace{}%
\AgdaSymbol{:}\AgdaSpace{}%
\AgdaFunction{Algebra}\AgdaSpace{}%
\AgdaBound{𝓤}\AgdaSpace{}%
\AgdaBound{𝑆}\AgdaSymbol{\}}\AgdaSpace{}%
\AgdaSymbol{(}\AgdaBound{f}\AgdaSpace{}%
\AgdaSymbol{:}\AgdaSpace{}%
\AgdaOperator{\AgdaFunction{∣}}\AgdaSpace{}%
\AgdaBound{𝑆}\AgdaSpace{}%
\AgdaOperator{\AgdaFunction{∣}}\AgdaSymbol{)}\<%
\\
\>[1][@{}l@{\AgdaIndent{0}}]%
\>[3]\AgdaSymbol{→}%
\>[.][@{}l@{}]\<[200I]%
\>[14]\AgdaFunction{compatible-op}\AgdaSpace{}%
\AgdaSymbol{\{}\AgdaArgument{𝓤}\AgdaSpace{}%
\AgdaSymbol{=}\AgdaSpace{}%
\AgdaBound{𝓤}\AgdaSymbol{\}\{}\AgdaArgument{𝑨}\AgdaSpace{}%
\AgdaSymbol{=}\AgdaSpace{}%
\AgdaBound{𝑨}\AgdaSymbol{\}}\AgdaSpace{}%
\AgdaBound{f}\AgdaSpace{}%
\AgdaFunction{𝟎-rel}\<%
\\
%
\>[1]\AgdaFunction{𝟎-compatible-op}\AgdaSpace{}%
\AgdaBound{fe}\AgdaSpace{}%
\AgdaSymbol{\{}\AgdaBound{𝑨}\AgdaSymbol{\}}\AgdaSpace{}%
\AgdaBound{f}\AgdaSpace{}%
\AgdaBound{ptws0}%
\>[33]\AgdaSymbol{=}\AgdaSpace{}%
\AgdaFunction{ap}\AgdaSpace{}%
\AgdaSymbol{(}\AgdaBound{f}\AgdaSpace{}%
\AgdaOperator{\AgdaFunction{̂}}\AgdaSpace{}%
\AgdaBound{𝑨}\AgdaSymbol{)}\AgdaSpace{}%
\AgdaSymbol{(}\AgdaBound{fe}\AgdaSpace{}%
\AgdaSymbol{(λ}\AgdaSpace{}%
\AgdaBound{x}\AgdaSpace{}%
\AgdaSymbol{→}\AgdaSpace{}%
\AgdaBound{ptws0}\AgdaSpace{}%
\AgdaBound{x}\AgdaSymbol{))}\<%
\\
%
\\[\AgdaEmptyExtraSkip]%
%
\>[1]\AgdaFunction{𝟎-compatible}\AgdaSpace{}%
\AgdaSymbol{:}\AgdaSpace{}%
\AgdaFunction{funext}\AgdaSpace{}%
\AgdaBound{𝓥}\AgdaSpace{}%
\AgdaBound{𝓤}\AgdaSpace{}%
\AgdaSymbol{→}\AgdaSpace{}%
\AgdaSymbol{\{}\AgdaBound{A}\AgdaSpace{}%
\AgdaSymbol{:}\AgdaSpace{}%
\AgdaFunction{Algebra}\AgdaSpace{}%
\AgdaBound{𝓤}\AgdaSpace{}%
\AgdaBound{𝑆}\AgdaSymbol{\}}\AgdaSpace{}%
\AgdaSymbol{→}\AgdaSpace{}%
\AgdaFunction{compatible}\AgdaSpace{}%
\AgdaBound{A}\AgdaSpace{}%
\AgdaFunction{𝟎-rel}\<%
\\
%
\>[1]\AgdaFunction{𝟎-compatible}\AgdaSpace{}%
\AgdaBound{fe}\AgdaSpace{}%
\AgdaSymbol{\{}\AgdaBound{A}\AgdaSymbol{\}}\AgdaSpace{}%
\AgdaSymbol{=}\AgdaSpace{}%
\AgdaSymbol{λ}\AgdaSpace{}%
\AgdaBound{f}\AgdaSpace{}%
\AgdaBound{args}\AgdaSpace{}%
\AgdaSymbol{→}\AgdaSpace{}%
\AgdaFunction{𝟎-compatible-op}\AgdaSpace{}%
\AgdaBound{fe}\AgdaSpace{}%
\AgdaSymbol{\{}\AgdaBound{A}\AgdaSymbol{\}}\AgdaSpace{}%
\AgdaBound{f}\AgdaSpace{}%
\AgdaBound{args}\<%
\end{code}
\ccpad
Now that we have the ingredients required to construct a congruence, we carry out the construction as follows.
\ccpad
\begin{code}%
\>[0]\AgdaFunction{Δ}\AgdaSpace{}%
\AgdaSymbol{:}\AgdaSpace{}%
\AgdaSymbol{\{}\AgdaBound{𝓤}\AgdaSpace{}%
\AgdaSymbol{:}\AgdaSpace{}%
\AgdaPostulate{Universe}\AgdaSymbol{\}}\AgdaSpace{}%
\AgdaSymbol{→}\AgdaSpace{}%
\AgdaFunction{funext}\AgdaSpace{}%
\AgdaBound{𝓥}\AgdaSpace{}%
\AgdaBound{𝓤}\AgdaSpace{}%
\AgdaSymbol{→}\AgdaSpace{}%
\AgdaSymbol{(}\AgdaBound{A}\AgdaSpace{}%
\AgdaSymbol{:}\AgdaSpace{}%
\AgdaFunction{Algebra}\AgdaSpace{}%
\AgdaBound{𝓤}\AgdaSpace{}%
\AgdaBound{𝑆}\AgdaSymbol{)}\AgdaSpace{}%
\AgdaSymbol{→}\AgdaSpace{}%
\AgdaRecord{Congruence}\AgdaSpace{}%
\AgdaBound{A}\<%
\\
\>[0]\AgdaFunction{Δ}\AgdaSpace{}%
\AgdaBound{fe}\AgdaSpace{}%
\AgdaBound{A}\AgdaSpace{}%
\AgdaSymbol{=}\AgdaSpace{}%
\AgdaInductiveConstructor{mkcon}\AgdaSpace{}%
\AgdaFunction{𝟎-rel}\AgdaSpace{}%
\AgdaSymbol{(}\AgdaSpace{}%
\AgdaFunction{𝟎-compatible}\AgdaSpace{}%
\AgdaBound{fe}\AgdaSpace{}%
\AgdaSymbol{)}\AgdaSpace{}%
\AgdaSymbol{(}\AgdaSpace{}%
\AgdaFunction{𝟎-IsEquivalence}\AgdaSpace{}%
\AgdaSymbol{)}\<%
\end{code}

\subsubsection{Quotient algebras}\label{sssec:quotient-algebras}
An important construction in universal algebra is the quotient of an algebra \ab 𝑨 with respect to a congruence relation \ab θ of \ab 𝑨. This quotient is typically denote by \ab 𝑨 \as / \ab θ and Agda allows us to define and express quotients using the standard notation.
\ccpad
\begin{code}%
\>[0]\AgdaOperator{\AgdaFunction{\AgdaUnderscore{}╱\AgdaUnderscore{}}}\AgdaSpace{}%
\AgdaSymbol{:}\AgdaSpace{}%
\AgdaSymbol{\{}\AgdaBound{𝓤}\AgdaSpace{}%
\AgdaBound{𝓡}\AgdaSpace{}%
\AgdaSymbol{:}\AgdaSpace{}%
\AgdaPostulate{Universe}\AgdaSymbol{\}(}\AgdaBound{A}\AgdaSpace{}%
\AgdaSymbol{:}\AgdaSpace{}%
\AgdaFunction{Algebra}\AgdaSpace{}%
\AgdaBound{𝓤}\AgdaSpace{}%
\AgdaBound{𝑆}\AgdaSymbol{)}\AgdaSpace{}%
\AgdaComment{-- type ╱ with `\textbackslash{}---`}\<%
\\
\>[0][@{}l@{\AgdaIndent{0}}]%
\>[1]\AgdaSymbol{→}%
\>[8]\AgdaRecord{Congruence}\AgdaSymbol{\{}\AgdaBound{𝓤}\AgdaSymbol{\}\{}\AgdaBound{𝓡}\AgdaSymbol{\}}\AgdaSpace{}%
\AgdaBound{A}%
\>[41]\AgdaComment{-- a number of times`}\<%
\\
\>[1][@{}l@{\AgdaIndent{0}}]%
\>[8]\AgdaComment{-----------------------}\<%
\\
%
\>[1]\AgdaSymbol{→}%
\>[8]\AgdaFunction{Algebra}\AgdaSpace{}%
\AgdaSymbol{(}\AgdaBound{𝓤}\AgdaSpace{}%
\AgdaOperator{\AgdaPrimitive{⊔}}\AgdaSpace{}%
\AgdaBound{𝓡}\AgdaSpace{}%
\AgdaOperator{\AgdaPrimitive{⁺}}\AgdaSymbol{)}\AgdaSpace{}%
\AgdaBound{𝑆}\<%
\\
\>[0]\AgdaBound{A}\AgdaSpace{}%
\AgdaOperator{\AgdaFunction{╱}}\AgdaSpace{}%
\AgdaBound{θ}\AgdaSpace{}%
\AgdaSymbol{=}%
\>[207I]\AgdaSymbol{((}\AgdaSpace{}%
\AgdaOperator{\AgdaFunction{∣}}\AgdaSpace{}%
\AgdaBound{A}\AgdaSpace{}%
\AgdaOperator{\AgdaFunction{∣}}\AgdaSpace{}%
\AgdaOperator{\AgdaFunction{/}}\AgdaSpace{}%
\AgdaOperator{\AgdaField{⟨}}\AgdaSpace{}%
\AgdaBound{θ}\AgdaSpace{}%
\AgdaOperator{\AgdaField{⟩}}\AgdaSpace{}%
\AgdaSymbol{)}\AgdaSpace{}%
\AgdaOperator{\AgdaInductiveConstructor{,}}\AgdaSpace{}%
\AgdaComment{-- carrier (i.e. domain or universe))}\<%
\\
\>[207I][@{}l@{\AgdaIndent{0}}]%
\>[10]\AgdaSymbol{(λ}\AgdaSpace{}%
\AgdaBound{f}\AgdaSpace{}%
\AgdaBound{args}%
\>[28]\AgdaComment{-- operations}\<%
\\
\>[10][@{}l@{\AgdaIndent{0}}]%
\>[11]\AgdaSymbol{→}%
\>[220I]\AgdaSymbol{(}\AgdaOperator{\AgdaFunction{[}}\AgdaSpace{}%
\AgdaSymbol{(}\AgdaBound{f}\AgdaSpace{}%
\AgdaOperator{\AgdaFunction{̂}}\AgdaSpace{}%
\AgdaBound{A}\AgdaSymbol{)}\AgdaSpace{}%
\AgdaSymbol{(λ}\AgdaSpace{}%
\AgdaBound{i₁}\AgdaSpace{}%
\AgdaSymbol{→}\AgdaSpace{}%
\AgdaOperator{\AgdaFunction{∣}}\AgdaSpace{}%
\AgdaOperator{\AgdaFunction{∥}}\AgdaSpace{}%
\AgdaBound{args}\AgdaSpace{}%
\AgdaBound{i₁}\AgdaSpace{}%
\AgdaOperator{\AgdaFunction{∥}}\AgdaSpace{}%
\AgdaOperator{\AgdaFunction{∣}}\AgdaSymbol{)}\AgdaSpace{}%
\AgdaOperator{\AgdaFunction{]}}\AgdaSpace{}%
\AgdaOperator{\AgdaField{⟨}}\AgdaSpace{}%
\AgdaBound{θ}\AgdaSpace{}%
\AgdaOperator{\AgdaField{⟩}}\AgdaSymbol{)}\AgdaSpace{}%
\AgdaOperator{\AgdaInductiveConstructor{,}}\<%
\\
\>[.][@{}l@{}]\<[220I]%
\>[13]\AgdaSymbol{((}\AgdaBound{f}\AgdaSpace{}%
\AgdaOperator{\AgdaFunction{̂}}\AgdaSpace{}%
\AgdaBound{A}\AgdaSymbol{)}\AgdaSpace{}%
\AgdaSymbol{(λ}\AgdaSpace{}%
\AgdaBound{i₁}\AgdaSpace{}%
\AgdaSymbol{→}\AgdaSpace{}%
\AgdaOperator{\AgdaFunction{∣}}\AgdaSpace{}%
\AgdaOperator{\AgdaFunction{∥}}\AgdaSpace{}%
\AgdaBound{args}\AgdaSpace{}%
\AgdaBound{i₁}\AgdaSpace{}%
\AgdaOperator{\AgdaFunction{∥}}\AgdaSpace{}%
\AgdaOperator{\AgdaFunction{∣}}\AgdaSymbol{)}\AgdaSpace{}%
\AgdaOperator{\AgdaInductiveConstructor{,}}\AgdaSpace{}%
\AgdaInductiveConstructor{refl}\AgdaSpace{}%
\AgdaSymbol{\AgdaUnderscore{}}\AgdaSpace{}%
\AgdaSymbol{)}\<%
\\
%
\>[10]\AgdaSymbol{)}\<%
\\
\>[.][@{}l@{}]\<[207I]%
\>[8]\AgdaSymbol{)}\<%
\end{code}

\subsubsection{Examples}\label{examples-zero-of-quotient}
The zero element of a quotient can be expressed as follows.
\ccpad
\begin{code}%
\>[0]\AgdaFunction{Zero╱}\AgdaSpace{}%
\AgdaSymbol{:}%
\>[380I]\AgdaSymbol{\{}\AgdaBound{𝓤}\AgdaSpace{}%
\AgdaBound{𝓡}\AgdaSpace{}%
\AgdaSymbol{:}\AgdaSpace{}%
\AgdaPostulate{Universe}\AgdaSymbol{\}\{}\AgdaBound{A}\AgdaSpace{}%
\AgdaSymbol{:}\AgdaSpace{}%
\AgdaFunction{Algebra}\AgdaSpace{}%
\AgdaBound{𝓤}\AgdaSpace{}%
\AgdaBound{𝑆}\AgdaSymbol{\}}\<%
\\
\>[.][@{}l@{}]\<[380I]%
\>[8]\AgdaSymbol{(}\AgdaBound{θ}\AgdaSpace{}%
\AgdaSymbol{:}\AgdaSpace{}%
\AgdaRecord{Congruence}\AgdaSymbol{\{}\AgdaBound{𝓤}\AgdaSymbol{\}\{}\AgdaBound{𝓡}\AgdaSymbol{\}}\AgdaSpace{}%
\AgdaBound{A}\AgdaSymbol{)}\<%
\\
\>[0][@{}l@{\AgdaIndent{0}}]%
\>[2]\AgdaSymbol{→}%
\>[8]\AgdaFunction{Rel}\AgdaSpace{}%
\AgdaSymbol{(}\AgdaOperator{\AgdaFunction{∣}}\AgdaSpace{}%
\AgdaBound{A}\AgdaSpace{}%
\AgdaOperator{\AgdaFunction{∣}}\AgdaSpace{}%
\AgdaOperator{\AgdaFunction{/}}\AgdaSpace{}%
\AgdaOperator{\AgdaField{⟨}}\AgdaSpace{}%
\AgdaBound{θ}\AgdaSpace{}%
\AgdaOperator{\AgdaField{⟩}}\AgdaSymbol{)}\AgdaSpace{}%
\AgdaSymbol{(}\AgdaBound{𝓤}\AgdaSpace{}%
\AgdaOperator{\AgdaPrimitive{⊔}}\AgdaSpace{}%
\AgdaBound{𝓡}\AgdaSpace{}%
\AgdaOperator{\AgdaPrimitive{⁺}}\AgdaSymbol{)}\<%
\\
%
\\[\AgdaEmptyExtraSkip]%
\>[0]\AgdaFunction{Zero╱}\AgdaSpace{}%
\AgdaBound{θ}\AgdaSpace{}%
\AgdaSymbol{=}\AgdaSpace{}%
\AgdaSymbol{λ}\AgdaSpace{}%
\AgdaBound{x}\AgdaSpace{}%
\AgdaBound{x₁}\AgdaSpace{}%
\AgdaSymbol{→}\AgdaSpace{}%
\AgdaBound{x}\AgdaSpace{}%
\AgdaOperator{\AgdaDatatype{≡}}\AgdaSpace{}%
\AgdaBound{x₁}\<%
\end{code}
\ccpad
Finally, the following elimination rule is sometimes useful.
\ccpad
\begin{code}%
\>[0]\AgdaFunction{╱-refl}%
\>[411I]\AgdaSymbol{:\{}\AgdaBound{𝓤}\AgdaSpace{}%
\AgdaBound{𝓡}\AgdaSpace{}%
\AgdaSymbol{:}\AgdaSpace{}%
\AgdaPostulate{Universe}\AgdaSymbol{\}}\AgdaSpace{}%
\AgdaSymbol{(}\AgdaBound{A}\AgdaSpace{}%
\AgdaSymbol{:}\AgdaSpace{}%
\AgdaFunction{Algebra}\AgdaSpace{}%
\AgdaBound{𝓤}\AgdaSpace{}%
\AgdaBound{𝑆}\AgdaSymbol{)}\<%
\\
\>[411I][@{}l@{\AgdaIndent{0}}]%
\>[8]\AgdaSymbol{\{}\AgdaBound{θ}\AgdaSpace{}%
\AgdaSymbol{:}\AgdaSpace{}%
\AgdaRecord{Congruence}\AgdaSymbol{\{}\AgdaBound{𝓤}\AgdaSymbol{\}\{}\AgdaBound{𝓡}\AgdaSymbol{\}}\AgdaSpace{}%
\AgdaBound{A}\AgdaSymbol{\}}\AgdaSpace{}%
\AgdaSymbol{\{}\AgdaBound{a}\AgdaSpace{}%
\AgdaBound{a'}\AgdaSpace{}%
\AgdaSymbol{:}\AgdaSpace{}%
\AgdaOperator{\AgdaFunction{∣}}\AgdaSpace{}%
\AgdaBound{A}\AgdaSpace{}%
\AgdaOperator{\AgdaFunction{∣}}\AgdaSymbol{\}}\<%
\\
\>[0][@{}l@{\AgdaIndent{0}}]%
\>[2]\AgdaSymbol{→}%
\>[8]\AgdaOperator{\AgdaFunction{⟦}}\AgdaSpace{}%
\AgdaBound{a}\AgdaSpace{}%
\AgdaOperator{\AgdaFunction{⟧}}\AgdaSymbol{\{}\AgdaOperator{\AgdaField{⟨}}\AgdaSpace{}%
\AgdaBound{θ}\AgdaSpace{}%
\AgdaOperator{\AgdaField{⟩}}\AgdaSymbol{\}}\AgdaSpace{}%
\AgdaOperator{\AgdaDatatype{≡}}\AgdaSpace{}%
\AgdaOperator{\AgdaFunction{⟦}}\AgdaSpace{}%
\AgdaBound{a'}\AgdaSpace{}%
\AgdaOperator{\AgdaFunction{⟧}}\AgdaSpace{}%
\AgdaSymbol{→}\AgdaSpace{}%
\AgdaOperator{\AgdaField{⟨}}\AgdaSpace{}%
\AgdaBound{θ}\AgdaSpace{}%
\AgdaOperator{\AgdaField{⟩}}\AgdaSpace{}%
\AgdaBound{a}\AgdaSpace{}%
\AgdaBound{a'}\<%
\\
%
\\[\AgdaEmptyExtraSkip]%
\>[0]\AgdaFunction{╱-refl}\AgdaSpace{}%
\AgdaBound{A}\AgdaSpace{}%
\AgdaSymbol{\{}\AgdaBound{θ}\AgdaSymbol{\}}\AgdaSpace{}%
\AgdaSymbol{(}\AgdaInductiveConstructor{refl}\AgdaSpace{}%
\AgdaSymbol{\AgdaUnderscore{})}%
\>[23]\AgdaSymbol{=}\AgdaSpace{}%
\AgdaField{IsEquivalence.rfl}\AgdaSpace{}%
\AgdaSymbol{(}\AgdaField{IsEquiv}\AgdaSpace{}%
\AgdaBound{θ}\AgdaSymbol{)}\AgdaSpace{}%
\AgdaSymbol{\AgdaUnderscore{}}\<%
\end{code}
%% -- END: \>[0]\AgdaFunction{Con}\AgdaSpace{}%
\AgdaSymbol{:}\AgdaSpace{}%
\AgdaSymbol{\{}\AgdaBound{𝓤}\AgdaSpace{}%
\AgdaSymbol{:}\AgdaSpace{}%
\AgdaPostulate{Universe}\AgdaSymbol{\}(}\AgdaBound{A}\AgdaSpace{}%
\AgdaSymbol{:}\AgdaSpace{}%
\AgdaFunction{Algebra}\AgdaSpace{}%
\AgdaBound{𝓤}\AgdaSpace{}%
\AgdaBound{𝑆}\AgdaSymbol{)}\AgdaSpace{}%
\AgdaSymbol{→}\AgdaSpace{}%
\AgdaBound{𝓞}\AgdaSpace{}%
\AgdaOperator{\AgdaPrimitive{⊔}}\AgdaSpace{}%
\AgdaBound{𝓥}\AgdaSpace{}%
\AgdaOperator{\AgdaPrimitive{⊔}}\AgdaSpace{}%
\AgdaBound{𝓤}\AgdaSpace{}%
\AgdaOperator{\AgdaPrimitive{⁺}}\AgdaSpace{}%
\AgdaOperator{\AgdaFunction{̇}}\<%
\\
\>[0]\AgdaFunction{Con}\AgdaSpace{}%
\AgdaSymbol{\{}\AgdaBound{𝓤}\AgdaSymbol{\}}\AgdaSpace{}%
\AgdaBound{A}\AgdaSpace{}%
\AgdaSymbol{=}\AgdaSpace{}%
\AgdaFunction{Σ}\AgdaSpace{}%
\AgdaBound{θ}\AgdaSpace{}%
\AgdaFunction{꞉}\AgdaSpace{}%
\AgdaSymbol{(}\AgdaSpace{}%
\AgdaFunction{Rel}\AgdaSpace{}%
\AgdaOperator{\AgdaFunction{∣}}\AgdaSpace{}%
\AgdaBound{A}\AgdaSpace{}%
\AgdaOperator{\AgdaFunction{∣}}\AgdaSpace{}%
\AgdaBound{𝓤}\AgdaSpace{}%
\AgdaSymbol{)}\AgdaSpace{}%
\AgdaFunction{,}\AgdaSpace{}%
\AgdaRecord{IsEquivalence}\AgdaSpace{}%
\AgdaBound{θ}\AgdaSpace{}%
\AgdaOperator{\AgdaFunction{×}}\AgdaSpace{}%
\AgdaFunction{compatible}\AgdaSpace{}%
\AgdaBound{A}\AgdaSpace{}%
\AgdaBound{θ}\<%
\\
%
\\[\AgdaEmptyExtraSkip]%
\>[0]\AgdaFunction{con}\AgdaSpace{}%
\AgdaSymbol{:}\AgdaSpace{}%
\AgdaSymbol{\{}\AgdaBound{𝓤}\AgdaSpace{}%
\AgdaSymbol{:}\AgdaSpace{}%
\AgdaPostulate{Universe}\AgdaSymbol{\}(}\AgdaBound{A}\AgdaSpace{}%
\AgdaSymbol{:}\AgdaSpace{}%
\AgdaFunction{Algebra}\AgdaSpace{}%
\AgdaBound{𝓤}\AgdaSpace{}%
\AgdaBound{𝑆}\AgdaSymbol{)}%
\>[39]\AgdaSymbol{→}%
\>[42]\AgdaFunction{Pred}\AgdaSpace{}%
\AgdaSymbol{(}\AgdaFunction{Rel}\AgdaSpace{}%
\AgdaOperator{\AgdaFunction{∣}}\AgdaSpace{}%
\AgdaBound{A}\AgdaSpace{}%
\AgdaOperator{\AgdaFunction{∣}}\AgdaSpace{}%
\AgdaBound{𝓤}\AgdaSymbol{)}\AgdaSpace{}%
\AgdaSymbol{(}\AgdaBound{𝓞}\AgdaSpace{}%
\AgdaOperator{\AgdaPrimitive{⊔}}\AgdaSpace{}%
\AgdaBound{𝓥}\AgdaSpace{}%
\AgdaOperator{\AgdaPrimitive{⊔}}\AgdaSpace{}%
\AgdaBound{𝓤}\AgdaSymbol{)}\<%
\\
\>[0]\AgdaFunction{con}\AgdaSpace{}%
\AgdaBound{A}\AgdaSpace{}%
\AgdaSymbol{=}\AgdaSpace{}%
\AgdaSymbol{λ}\AgdaSpace{}%
\AgdaBound{θ}\AgdaSpace{}%
\AgdaSymbol{→}\AgdaSpace{}%
\AgdaRecord{IsEquivalence}\AgdaSpace{}%
\AgdaBound{θ}\AgdaSpace{}%
\AgdaOperator{\AgdaFunction{×}}\AgdaSpace{}%
\AgdaFunction{compatible}\AgdaSpace{}%
\AgdaBound{A}\AgdaSpace{}%
\AgdaBound{θ}\<%
\\
%
\\[\AgdaEmptyExtraSkip]%
\>[0]\AgdaKeyword{record}\AgdaSpace{}%
\AgdaRecord{Congruence}\AgdaSpace{}%
\AgdaSymbol{\{}\AgdaBound{𝓤}\AgdaSpace{}%
\AgdaBound{𝓦}\AgdaSpace{}%
\AgdaSymbol{:}\AgdaSpace{}%
\AgdaPostulate{Universe}\AgdaSymbol{\}}\AgdaSpace{}%
\AgdaSymbol{(}\AgdaBound{A}\AgdaSpace{}%
\AgdaSymbol{:}\AgdaSpace{}%
\AgdaFunction{Algebra}\AgdaSpace{}%
\AgdaBound{𝓤}\AgdaSpace{}%
\AgdaBound{𝑆}\AgdaSymbol{)}\AgdaSpace{}%
\AgdaSymbol{:}\AgdaSpace{}%
\AgdaBound{𝓞}\AgdaSpace{}%
\AgdaOperator{\AgdaPrimitive{⊔}}\AgdaSpace{}%
\AgdaBound{𝓥}\AgdaSpace{}%
\AgdaOperator{\AgdaPrimitive{⊔}}\AgdaSpace{}%
\AgdaBound{𝓤}\AgdaSpace{}%
\AgdaOperator{\AgdaPrimitive{⊔}}\AgdaSpace{}%
\AgdaBound{𝓦}\AgdaSpace{}%
\AgdaOperator{\AgdaPrimitive{⁺}}\AgdaSpace{}%
\AgdaOperator{\AgdaFunction{̇}}%
\>[74]\AgdaKeyword{where}\<%
\\
\>[0][@{}l@{\AgdaIndent{0}}]%
\>[1]\AgdaKeyword{constructor}\AgdaSpace{}%
\AgdaInductiveConstructor{mkcon}\<%
\\
%
\>[1]\AgdaKeyword{field}\<%
\\
\>[1][@{}l@{\AgdaIndent{0}}]%
\>[2]\AgdaOperator{\AgdaField{⟨\AgdaUnderscore{}⟩}}\AgdaSpace{}%
\AgdaSymbol{:}\AgdaSpace{}%
\AgdaFunction{Rel}\AgdaSpace{}%
\AgdaOperator{\AgdaFunction{∣}}\AgdaSpace{}%
\AgdaBound{A}\AgdaSpace{}%
\AgdaOperator{\AgdaFunction{∣}}\AgdaSpace{}%
\AgdaBound{𝓦}\<%
\\
%
\>[2]\AgdaField{Compatible}\AgdaSpace{}%
\AgdaSymbol{:}\AgdaSpace{}%
\AgdaFunction{compatible}\AgdaSpace{}%
\AgdaBound{A}\AgdaSpace{}%
\AgdaOperator{\AgdaField{⟨\AgdaUnderscore{}⟩}}\<%
\\
%
\>[2]\AgdaField{IsEquiv}\AgdaSpace{}%
\AgdaSymbol{:}\AgdaSpace{}%
\AgdaRecord{IsEquivalence}\AgdaSpace{}%
\AgdaOperator{\AgdaField{⟨\AgdaUnderscore{}⟩}}\<%
\\
%
\\[\AgdaEmptyExtraSkip]%
\>[0]\AgdaKeyword{open}\AgdaSpace{}%
\AgdaModule{Congruence}\<%
\\
%
\\[\AgdaEmptyExtraSkip]%
\>[0]\AgdaFunction{compatible-equivalence}\AgdaSpace{}%
\AgdaSymbol{:}\AgdaSpace{}%
\AgdaSymbol{\{}\AgdaBound{𝓤}\AgdaSpace{}%
\AgdaBound{𝓦}\AgdaSpace{}%
\AgdaSymbol{:}\AgdaSpace{}%
\AgdaPostulate{Universe}\AgdaSymbol{\}\{}\AgdaBound{𝑨}\AgdaSpace{}%
\AgdaSymbol{:}\AgdaSpace{}%
\AgdaFunction{Algebra}\AgdaSpace{}%
\AgdaBound{𝓤}\AgdaSpace{}%
\AgdaBound{𝑆}\AgdaSymbol{\}}\AgdaSpace{}%
\AgdaSymbol{→}\AgdaSpace{}%
\AgdaFunction{Rel}\AgdaSpace{}%
\AgdaOperator{\AgdaFunction{∣}}\AgdaSpace{}%
\AgdaBound{𝑨}\AgdaSpace{}%
\AgdaOperator{\AgdaFunction{∣}}\AgdaSpace{}%
\AgdaBound{𝓦}\AgdaSpace{}%
\AgdaSymbol{→}\AgdaSpace{}%
\AgdaBound{𝓞}\AgdaSpace{}%
\AgdaOperator{\AgdaPrimitive{⊔}}\AgdaSpace{}%
\AgdaBound{𝓥}\AgdaSpace{}%
\AgdaOperator{\AgdaPrimitive{⊔}}\AgdaSpace{}%
\AgdaBound{𝓦}\AgdaSpace{}%
\AgdaOperator{\AgdaPrimitive{⊔}}\AgdaSpace{}%
\AgdaBound{𝓤}\AgdaSpace{}%
\AgdaOperator{\AgdaFunction{̇}}\<%
\\
\>[0]\AgdaFunction{compatible-equivalence}\AgdaSpace{}%
\AgdaSymbol{\{}\AgdaBound{𝓤}\AgdaSymbol{\}\{}\AgdaBound{𝓦}\AgdaSymbol{\}}\AgdaSpace{}%
\AgdaSymbol{\{}\AgdaBound{𝑨}\AgdaSymbol{\}}\AgdaSpace{}%
\AgdaBound{R}\AgdaSpace{}%
\AgdaSymbol{=}\AgdaSpace{}%
\AgdaFunction{compatible}\AgdaSpace{}%
\AgdaBound{𝑨}\AgdaSpace{}%
\AgdaBound{R}\AgdaSpace{}%
\AgdaOperator{\AgdaFunction{×}}\AgdaSpace{}%
\AgdaRecord{IsEquivalence}\AgdaSpace{}%
\AgdaBound{R}\<%



\appendix 

\section{Some Components of the Type Topology Library}
Here we collect some of the components from the \typetopology library that we used above but did not have space to discuss.  They are collected here for the reader's convenience and to keep the paper somewhat self-contained.

\begin{code}
\>[0]\AgdaFunction{transport}%
\>[1275I]\AgdaSymbol{:}\AgdaSpace{}%
\AgdaSymbol{\{}\AgdaBound{X}\AgdaSpace{}%
\AgdaSymbol{:}\AgdaSpace{}%
\AgdaGeneralizable{𝓤}\AgdaSpace{}%
\AgdaOperator{\AgdaFunction{̇}}\AgdaSpace{}%
\AgdaSymbol{\}}\AgdaSpace{}%
\AgdaSymbol{(}\AgdaBound{A}\AgdaSpace{}%
\AgdaSymbol{:}\AgdaSpace{}%
\AgdaBound{X}\AgdaSpace{}%
\AgdaSymbol{→}\AgdaSpace{}%
\AgdaGeneralizable{𝓥}\AgdaSpace{}%
\AgdaOperator{\AgdaFunction{̇}}\AgdaSpace{}%
\AgdaSymbol{)}\AgdaSpace{}%
\AgdaSymbol{\{}\AgdaBound{x}\AgdaSpace{}%
\AgdaBound{y}\AgdaSpace{}%
\AgdaSymbol{:}\AgdaSpace{}%
\AgdaBound{X}\AgdaSymbol{\}}\<%
\\
\>[.][@{}l@{}]\<[1275I]%
\>[10]\AgdaSymbol{→}\AgdaSpace{}%
\AgdaBound{x}\AgdaSpace{}%
\AgdaOperator{\AgdaDatatype{≡}}\AgdaSpace{}%
\AgdaBound{y}\AgdaSpace{}%
\AgdaSymbol{→}\AgdaSpace{}%
\AgdaBound{A}\AgdaSpace{}%
\AgdaBound{x}\AgdaSpace{}%
\AgdaSymbol{→}\AgdaSpace{}%
\AgdaBound{A}\AgdaSpace{}%
\AgdaBound{y}\<%
\\
%
\\[\AgdaEmptyExtraSkip]%
\>[0]\AgdaFunction{transport}\AgdaSpace{}%
\AgdaBound{A}\AgdaSpace{}%
\AgdaSymbol{(}\AgdaInductiveConstructor{refl}\AgdaSpace{}%
\AgdaBound{x}\AgdaSymbol{)}\AgdaSpace{}%
\AgdaSymbol{=}\AgdaSpace{}%
\AgdaFunction{𝑖𝑑}\AgdaSpace{}%
\AgdaSymbol{(}\AgdaBound{A}\AgdaSpace{}%
\AgdaBound{x}\AgdaSymbol{)}\<%
\end{code}
\scpad
\begin{code}
  \>[0]\AgdaFunction{to-Σ-≡}%
\>[1596I]\AgdaSymbol{:}\AgdaSpace{}%
\AgdaSymbol{\{}\AgdaBound{X}\AgdaSpace{}%
\AgdaSymbol{:}\AgdaSpace{}%
\AgdaGeneralizable{𝓤}\AgdaSpace{}%
\AgdaOperator{\AgdaFunction{̇}}\AgdaSpace{}%
\AgdaSymbol{\}}\AgdaSpace{}%
\AgdaSymbol{\{}\AgdaBound{A}\AgdaSpace{}%
\AgdaSymbol{:}\AgdaSpace{}%
\AgdaBound{X}\AgdaSpace{}%
\AgdaSymbol{→}\AgdaSpace{}%
\AgdaGeneralizable{𝓥}\AgdaSpace{}%
\AgdaOperator{\AgdaFunction{̇}}\AgdaSpace{}%
\AgdaSymbol{\}}\AgdaSpace{}%
\AgdaSymbol{\{}\AgdaBound{σ}\AgdaSpace{}%
\AgdaBound{τ}\AgdaSpace{}%
\AgdaSymbol{:}\AgdaSpace{}%
\AgdaRecord{Σ}\AgdaSpace{}%
\AgdaBound{A}\AgdaSymbol{\}}\<%
\\
\>[.][@{}l@{}]\<[1596I]%
\>[7]\AgdaSymbol{→}\AgdaSpace{}%
\AgdaSymbol{(}\AgdaFunction{Σ}\AgdaSpace{}%
\AgdaBound{p}\AgdaSpace{}%
\AgdaFunction{꞉}\AgdaSpace{}%
\AgdaFunction{pr₁}\AgdaSpace{}%
\AgdaBound{σ}\AgdaSpace{}%
\AgdaOperator{\AgdaDatatype{≡}}\AgdaSpace{}%
\AgdaFunction{pr₁}\AgdaSpace{}%
\AgdaBound{τ}\AgdaSpace{}%
\AgdaFunction{,}\AgdaSpace{}%
\AgdaFunction{transport}\AgdaSpace{}%
\AgdaBound{A}\AgdaSpace{}%
\AgdaBound{p}\AgdaSpace{}%
\AgdaSymbol{(}\AgdaFunction{pr₂}\AgdaSpace{}%
\AgdaBound{σ}\AgdaSymbol{)}\AgdaSpace{}%
\AgdaOperator{\AgdaDatatype{≡}}\AgdaSpace{}%
\AgdaFunction{pr₂}\AgdaSpace{}%
\AgdaBound{τ}\AgdaSymbol{)}\<%
\\
%
\>[7]\AgdaSymbol{→}\AgdaSpace{}%
\AgdaBound{σ}\AgdaSpace{}%
\AgdaOperator{\AgdaDatatype{≡}}\AgdaSpace{}%
\AgdaBound{τ}\<%
\\
%
\\[\AgdaEmptyExtraSkip]%
\>[0]\AgdaFunction{to-Σ-≡}\AgdaSpace{}%
\AgdaSymbol{(}\AgdaInductiveConstructor{refl}\AgdaSpace{}%
\AgdaBound{x}\AgdaSpace{}%
\AgdaOperator{\AgdaInductiveConstructor{,}}\AgdaSpace{}%
\AgdaInductiveConstructor{refl}\AgdaSpace{}%
\AgdaBound{a}\AgdaSymbol{)}\AgdaSpace{}%
\AgdaSymbol{=}\AgdaSpace{}%
\AgdaInductiveConstructor{refl}\AgdaSpace{}%
\AgdaSymbol{(}\AgdaBound{x}\AgdaSpace{}%
\AgdaOperator{\AgdaInductiveConstructor{,}}\AgdaSpace{}%
\AgdaBound{a}\AgdaSymbol{)}\<%
\end{code}


\subsubsection{Dependent function extensionality}\label{dependent-function-extensionality}

dfunext : ∀ 𝓤 𝓥 → (𝓤 ⊔ 𝓥)⁺ ̇
dfunext 𝓤 𝓥 = {X : 𝓤 ̇ } {A : X → 𝓥 ̇ } {f g : Π A} → f ∼ g → f ≡ g

global-dfunext : 𝓤ω
global-dfunext = ∀ {𝓤 𝓥} → dfunext 𝓤 𝓥

happly : {X : 𝓤 ̇ } {A : X → 𝓥 ̇ } (f g : Π A) → f ≡ g → f ∼ g
happly f g p x = ap (λ - → - x) p

Extensionality for dependent function types is defined as follows.
\ccpad
\begin{code}%
\>[0]\AgdaFunction{dep-extensionality}\AgdaSpace{}%
\AgdaSymbol{:}\AgdaSpace{}%
\AgdaSymbol{∀}\AgdaSpace{}%
\AgdaBound{𝓤}\AgdaSpace{}%
\AgdaBound{𝓦}\AgdaSpace{}%
\AgdaSymbol{→}\AgdaSpace{}%
\AgdaBound{𝓤}\AgdaSpace{}%
\AgdaOperator{\AgdaPrimitive{⁺}}\AgdaSpace{}%
\AgdaOperator{\AgdaPrimitive{⊔}}\AgdaSpace{}%
\AgdaBound{𝓦}\AgdaSpace{}%
\AgdaOperator{\AgdaPrimitive{⁺}}\AgdaSpace{}%
\AgdaOperator{\AgdaFunction{̇}}\<%
\\
\>[0]\AgdaFunction{dep-extensionality}\AgdaSpace{}%
\AgdaBound{𝓤}\AgdaSpace{}%
\AgdaBound{𝓦}\AgdaSpace{}%
\AgdaSymbol{=}\AgdaSpace{}%
\AgdaSymbol{\{}\AgdaBound{A}\AgdaSpace{}%
\AgdaSymbol{:}\AgdaSpace{}%
\AgdaBound{𝓤}\AgdaSpace{}%
\AgdaOperator{\AgdaFunction{̇}}\AgdaSpace{}%
\AgdaSymbol{\}}\AgdaSpace{}%
\AgdaSymbol{\{}\AgdaBound{B}\AgdaSpace{}%
\AgdaSymbol{:}\AgdaSpace{}%
\AgdaBound{A}\AgdaSpace{}%
\AgdaSymbol{→}\AgdaSpace{}%
\AgdaBound{𝓦}\AgdaSpace{}%
\AgdaOperator{\AgdaFunction{̇}}\AgdaSpace{}%
\AgdaSymbol{\}}\<%
\\
\>[0][@{}l@{\AgdaIndent{0}}]%
\>[2]\AgdaSymbol{\{}\AgdaBound{f}\AgdaSpace{}%
\AgdaBound{g}\AgdaSpace{}%
\AgdaSymbol{:}\AgdaSpace{}%
\AgdaSymbol{∀(}\AgdaBound{x}\AgdaSpace{}%
\AgdaSymbol{:}\AgdaSpace{}%
\AgdaBound{A}\AgdaSymbol{)}\AgdaSpace{}%
\AgdaSymbol{→}\AgdaSpace{}%
\AgdaBound{B}\AgdaSpace{}%
\AgdaBound{x}\AgdaSymbol{\}}\AgdaSpace{}%
\AgdaSymbol{→}%
\>[28]\AgdaBound{f}\AgdaSpace{}%
\AgdaOperator{\AgdaFunction{∼}}\AgdaSpace{}%
\AgdaBound{g}%
\>[35]\AgdaSymbol{→}%
\>[38]\AgdaBound{f}\AgdaSpace{}%
\AgdaOperator{\AgdaDatatype{≡}}\AgdaSpace{}%
\AgdaBound{g}\<%
\end{code}
\ccpad
Sometimes we need extensionality principles that work at all universe levels, and Agda is capable of expressing such principles, which belong to the special 𝓤ω type, as follows:
\ccpad
\begin{code}%
\>[0]\AgdaFunction{∀-extensionality}\AgdaSpace{}%
\AgdaSymbol{:}\AgdaSpace{}%
\AgdaPrimitive{𝓤ω}\<%
\\
\>[0]\AgdaFunction{∀-extensionality}\AgdaSpace{}%
\AgdaSymbol{=}\AgdaSpace{}%
\AgdaSymbol{∀}%
\>[22]\AgdaSymbol{\{}\AgdaBound{𝓤}\AgdaSpace{}%
\AgdaBound{𝓥}\AgdaSymbol{\}}\AgdaSpace{}%
\AgdaSymbol{→}\AgdaSpace{}%
\AgdaFunction{extensionality}\AgdaSpace{}%
\AgdaBound{𝓤}\AgdaSpace{}%
\AgdaBound{𝓥}\<%
\\
%
\\[\AgdaEmptyExtraSkip]%
\>[0]\AgdaFunction{∀-dep-extensionality}\AgdaSpace{}%
\AgdaSymbol{:}\AgdaSpace{}%
\AgdaPrimitive{𝓤ω}\<%
\\
\>[0]\AgdaFunction{∀-dep-extensionality}\AgdaSpace{}%
\AgdaSymbol{=}\AgdaSpace{}%
\AgdaSymbol{∀}\AgdaSpace{}%
\AgdaSymbol{\{}\AgdaBound{𝓤}\AgdaSpace{}%
\AgdaBound{𝓥}\AgdaSymbol{\}}\AgdaSpace{}%
\AgdaSymbol{→}\AgdaSpace{}%
\AgdaFunction{dep-extensionality}\AgdaSpace{}%
\AgdaBound{𝓤}\AgdaSpace{}%
\AgdaBound{𝓥}\<%
\end{code}
\ccpad
More details about the 𝓤ω type are available at \href{https://agda.readthedocs.io/en/latest/language/universe-levels.html#expressions-of-kind-set}{agda.readthedocs.io}.
\ccpad
\begin{code}%
\>[0]\AgdaFunction{extensionality-lemma}\AgdaSpace{}%
\AgdaSymbol{:}%
\>[119I]\AgdaSymbol{∀}\AgdaSpace{}%
\AgdaSymbol{\{}\AgdaBound{𝓘}\AgdaSpace{}%
\AgdaBound{𝓤}\AgdaSpace{}%
\AgdaBound{𝓥}\AgdaSpace{}%
\AgdaBound{𝓣}\AgdaSymbol{\}}\AgdaSpace{}%
\AgdaSymbol{→}\<%
\\
\>[.][@{}l@{}]\<[119I]%
\>[23]\AgdaSymbol{\{}\AgdaBound{I}\AgdaSpace{}%
\AgdaSymbol{:}\AgdaSpace{}%
\AgdaBound{𝓘}\AgdaSpace{}%
\AgdaOperator{\AgdaFunction{̇}}\AgdaSpace{}%
\AgdaSymbol{\}\{}\AgdaBound{X}\AgdaSpace{}%
\AgdaSymbol{:}\AgdaSpace{}%
\AgdaBound{𝓤}\AgdaSpace{}%
\AgdaOperator{\AgdaFunction{̇}}\AgdaSpace{}%
\AgdaSymbol{\}\{}\AgdaBound{A}\AgdaSpace{}%
\AgdaSymbol{:}\AgdaSpace{}%
\AgdaBound{I}\AgdaSpace{}%
\AgdaSymbol{→}\AgdaSpace{}%
\AgdaBound{𝓥}\AgdaSpace{}%
\AgdaOperator{\AgdaFunction{̇}}\AgdaSpace{}%
\AgdaSymbol{\}}\<%
\\
%
\>[23]\AgdaSymbol{(}\AgdaBound{p}\AgdaSpace{}%
\AgdaBound{q}\AgdaSpace{}%
\AgdaSymbol{:}\AgdaSpace{}%
\AgdaSymbol{(}\AgdaBound{i}\AgdaSpace{}%
\AgdaSymbol{:}\AgdaSpace{}%
\AgdaBound{I}\AgdaSymbol{)}\AgdaSpace{}%
\AgdaSymbol{→}\AgdaSpace{}%
\AgdaSymbol{(}\AgdaBound{X}\AgdaSpace{}%
\AgdaSymbol{→}\AgdaSpace{}%
\AgdaBound{A}\AgdaSpace{}%
\AgdaBound{i}\AgdaSymbol{)}\AgdaSpace{}%
\AgdaSymbol{→}\AgdaSpace{}%
\AgdaBound{𝓣}\AgdaSpace{}%
\AgdaOperator{\AgdaFunction{̇}}\AgdaSpace{}%
\AgdaSymbol{)}\<%
\\
%
\>[23]\AgdaSymbol{(}\AgdaBound{args}\AgdaSpace{}%
\AgdaSymbol{:}\AgdaSpace{}%
\AgdaBound{X}\AgdaSpace{}%
\AgdaSymbol{→}\AgdaSpace{}%
\AgdaSymbol{(}\AgdaFunction{Π}\AgdaSpace{}%
\AgdaBound{A}\AgdaSymbol{))}\<%
\\
\>[0][@{}l@{\AgdaIndent{0}}]%
\>[1]\AgdaSymbol{→}%
\>[23]\AgdaBound{p}\AgdaSpace{}%
\AgdaOperator{\AgdaDatatype{≡}}\AgdaSpace{}%
\AgdaBound{q}\<%
\\
\>[1][@{}l@{\AgdaIndent{0}}]%
\>[23]\AgdaComment{-------------------------------------------------}\<%
\\
\>[0][@{}l@{\AgdaIndent{0}}]%
\>[1]\AgdaSymbol{→}%
\>[23]\AgdaSymbol{(λ}\AgdaSpace{}%
\AgdaBound{i}\AgdaSpace{}%
\AgdaSymbol{→}\AgdaSpace{}%
\AgdaSymbol{(}\AgdaBound{p}\AgdaSpace{}%
\AgdaBound{i}\AgdaSymbol{)(λ}\AgdaSpace{}%
\AgdaBound{x}\AgdaSpace{}%
\AgdaSymbol{→}\AgdaSpace{}%
\AgdaBound{args}\AgdaSpace{}%
\AgdaBound{x}\AgdaSpace{}%
\AgdaBound{i}\AgdaSymbol{))}\AgdaSpace{}%
\AgdaOperator{\AgdaDatatype{≡}}\AgdaSpace{}%
\AgdaSymbol{(λ}\AgdaSpace{}%
\AgdaBound{i}\AgdaSpace{}%
\AgdaSymbol{→}\AgdaSpace{}%
\AgdaSymbol{(}\AgdaBound{q}\AgdaSpace{}%
\AgdaBound{i}\AgdaSymbol{)(λ}\AgdaSpace{}%
\AgdaBound{x}\AgdaSpace{}%
\AgdaSymbol{→}\AgdaSpace{}%
\AgdaBound{args}\AgdaSpace{}%
\AgdaBound{x}\AgdaSpace{}%
\AgdaBound{i}\AgdaSymbol{))}\<%
\\
%
\\[\AgdaEmptyExtraSkip]%
\>[0]\AgdaFunction{extensionality-lemma}\AgdaSpace{}%
\AgdaBound{p}\AgdaSpace{}%
\AgdaBound{q}\AgdaSpace{}%
\AgdaBound{args}\AgdaSpace{}%
\AgdaBound{p≡q}\AgdaSpace{}%
\AgdaSymbol{=}\<%
\\
\>[0][@{}l@{\AgdaIndent{0}}]%
\>[1]\AgdaFunction{ap}\AgdaSpace{}%
\AgdaSymbol{(λ}\AgdaSpace{}%
\AgdaBound{-}\AgdaSpace{}%
\AgdaSymbol{→}\AgdaSpace{}%
\AgdaSymbol{λ}\AgdaSpace{}%
\AgdaBound{i}\AgdaSpace{}%
\AgdaSymbol{→}\AgdaSpace{}%
\AgdaSymbol{(}\AgdaBound{-}\AgdaSpace{}%
\AgdaBound{i}\AgdaSymbol{)}\AgdaSpace{}%
\AgdaSymbol{(λ}\AgdaSpace{}%
\AgdaBound{x}\AgdaSpace{}%
\AgdaSymbol{→}\AgdaSpace{}%
\AgdaBound{args}\AgdaSpace{}%
\AgdaBound{x}\AgdaSpace{}%
\AgdaBound{i}\AgdaSymbol{))}\AgdaSpace{}%
\AgdaBound{p≡q}\<%
\end{code}


%% Bibliography
\bibliography{ualib_refs}

%%%%%%%%%%%%%%%%%%%%%%%%%%%%%%%%%%%%%%%%%%%%%%%%%%%%%%%%%%%%%%%%%%%%%%%%%%%
\end{document} %%%%%%%%%%%%%%%%%%%%%%%%%%%%%%%%%%%%%%%%%%%%%%%%%%%%%%%%%%%%
%%%%%%%%%%%%%%%%%%%%%%%%%%%%%%%%%%%%%%%%%%%%%%%%%%%%%%%%%%%%%%%%%%%%%%%%%%%

