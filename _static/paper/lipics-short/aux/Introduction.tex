To support formalization in type theory of research level mathematics in universal algebra and related fields, we present the Agda Universal Algebra Library (\agdaualib), a software library containing formal statements and proofs of the core definitions and results of universal algebra. 
The \ualib is written in \agda~\cite{Norell:2009}, a programming language and proof assistant based on Martin-L\"of Type Theory that not only supports dependent and inductive types, but also provides powerful \emph{proof tactics} for proving things about the objects that inhabit these types.

There have been a number of prior efforts to formalize parts of universal algebra in type theory, most notably
\begin{itemize}
  \item Capretta~\cite{Capretta:1999} (1999) formalized the basics of universal algebra in the Calculus of Inductive Constructions using the Coq proof assistant;
    \item Spitters and van der Weegen~\cite{Spitters:2011} (2011) formalized the basics of universal algebra and some classical algebraic structures, also in the Calculus of Inductive Constructions using the Coq proof assistant, promoting the use of type classes as a preferable alternative to setoids;
 \item Gunther, et al~\cite{Gunther:2018} (2018) developed what seems to be (prior to the \ualib) the most extensive library of formal universal algebra to date; in particular, this work includes a formalization of some basic equational logic; the project (like the \ualib) uses Martin-L\"of Type Theory and the Agda proof assistant.
\end{itemize}
Some other projects aimed at formalizing mathematics generally, and algebra in particular, have developed into very extensive libraries that include definitions, theorems, and proofs about algebraic structures, such as groups, rings, modules, etc.  However, goals of this prior work seem to be the formalization of special classical algebraic structures, as opposed to the general theory of (universal) algebras.

Although the \agdaualib project was initiated relatively recently (in 2018), the part of universal algebra and equational logic that it formalizes extends beyond the scope of prior efforts.  In particular, the \ualib now includes the only formal, constructive, machine-checked proof of Birkhoff's variety theorem that we know of. We remark that, with the exception of~\cite{Carlstrom:2008}, all other proofs of Birkhoff's theorem we have seen are informal and not known to be constructive.

%% \subsection{Vision and Goals}\label{vision-and-goals}
The seminal idea for the \agdaualib project was the observation that, on the one hand, a number of fundamental constructions in universal algebra can be defined recursively, and theorems about them proved by structural induction, while, on the other hand, inductive and dependent types make possible very precise formal representations of recursively defined objects, which often admit elegant constructive proofs of properties of such objects.  An important feature of such proofs in type theory is that they are total functional programs and, as such, they are computable and composable.
%% These observations suggested that there would be much to gain from implementing universal algebra in a language, such as Martin-L\"of type theory, that features dependent and inductive types.

Finally, our own research experience has taught us that a proof assistant and programming language (like Agda), when equipped with specialized libraries and domain-specific tactics to automate proof idioms of a particular field, can be an extremely powerful and effective asset. We believe that such libraries, and the proof assistants they support, will eventually become indispensable tools in the working mathematician's toolkit.



\subsection{Contributions and organization}
\label{sec:contributions}
Apart from the library itself, we describes the formal implementation and proof of a deep result, Garrett Birkhoff's celebrated HSP theorem~\cite{Birkhoff:1935}, which was among the first major results of universal algebra.  The theorem states that a \defn{variety} (a class of algebras closed under quotients, subalgebras, and products) is an equational class (defined by the set of identities satisfied by all its members).  %% More precisely, a class 𝒦 of algebras is closed under the taking of quotients, subalgebras, and arbitrary products if and only if 𝒦 is the class of algebras satisfying a certain set of equations.
The fact that we now have a formal proof of this is noteworthy, not only because this is the first time the theorem has been proved in dependent type theory and verified with a proof assistant, but also because the proof is constructive. As the paper~\cite{Carlstrom:2008} of Carlstr\"om makes clear, it is a highly nontrivial exercise to take a well-known informal proof of a theorem like Birkhoff's and show that it can be formalized using only constructive logic and natural deduction, without appealing to, say, the Law of the Excluded Middle or the Axiom of Choice.

Each of the sections that follow describes the most important or noteworthy components of the \ualib. We cover just enough to keep the paper somewhat self-contained.  Of course, space does not permit us to cover every definition and theorem required to present a complete formal proof of a theorem like Birkhoff's.  We remedy this in two ways. First, throughout the paper we include pointers to places in the documentation where the omitted material can be found.  Second, we include an appendix containing \agda background, discussion of the foundational assumptions of the \ualib, and definitions of some important types of dependent type theory and how they are represented in \agda and in the \ualib.  We hope this appendix is especially useful to readers who are not already proficient users of \agda.



Finally, the official sources of information about the \agdaualib are
\begin{itemize}
  \item \href{https://ualib.gitlab.io}{\texttt{ualib.org}} (the web site) includes every line of code in the library, rendered as html and accompanied by documentation, and
  \item \href{https://gitlab.com/ualib/ualib.gitlab.io}{\texttt{gitlab.com/ualib/ualib.gitlab.io}} (the source code) freely available and licensed under the \href{https://creativecommons.org/licenses/by-sa/4.0/}{Creative Commons Attribution-ShareAlike 4.0 International License}.
\end{itemize}
