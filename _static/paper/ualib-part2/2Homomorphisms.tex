% -*- TeX-master: "ualib-part2.tex" -*-
%%% Local Variables: 
%%% mode: latex
%%% TeX-engine: 'xetex
%%% End:
%%%%%%%%%%%%%%%%%%%%%%%%%%%%%%%%%%%%%%%%%%%%%%%%%%%%%%%%%%%%%%%%%%%%%%%%%%%%%%%%%%%%%%%%
\section{Homomorphism Types}\label{sec:homomorphism-types}
In this section we define types that represent some of the most important concepts from general (universal) algebra.  Note that the Agda modules we describe here, and in succeeding sections depend on and import the modules that were presented in Part 1 of this series of three papers describing the \agdaualib (see~\cite{DeMeo:2021-1}). 

We begin in Subsection~\ref{sec:basic-definitions} with the basic definition of \emph{homomorphism}. In \S\ref{sec:homom-theor} we formalize the statement and proof of the first fundamental theorem about homomorphisms, which is sometimes referred to as the \emph{First Isomorphism Theorem}. This is followed by \S\ref{sec:isomorphisms}, in which we define the type of \emph{isomorphisms} between algebraic structures.  Finally, in \S\ref{sec:hom-images}, we define types that manifest the notion of \emph{homomorphic image}.

\subsection{Basic definitions}\label{sec:basic-definitions}\firstsentence{\ualibHomBasic}{\urlHomBasic}
Since this is the first module we introduce in the installment of our series of papers documenting the \agdaualib, we will begin by showing the start of the module in full.  In later modules, we will leave such details implicit.  Here is how the file \texttt{Homomorphisms/Basic.lagda} of the \ualib begins.
\ccpad
\begin{code}%
\>[0]\AgdaSymbol{\{-\#}\AgdaSpace{}%
\AgdaKeyword{OPTIONS}\AgdaSpace{}%
\AgdaPragma{--without-K}\AgdaSpace{}%
\AgdaPragma{--exact-split}\AgdaSpace{}%
\AgdaPragma{--safe}\AgdaSpace{}%
\AgdaSymbol{\#-\}}\<%
\\
%
\\[\AgdaEmptyExtraSkip]%
\>[0]\AgdaKeyword{open}\AgdaSpace{}%
\AgdaKeyword{import}\AgdaSpace{}%
\AgdaModule{Algebras.Signatures}\AgdaSpace{}%
\AgdaKeyword{using}\AgdaSpace{}%
\AgdaSymbol{(}\AgdaFunction{Signature}\AgdaSymbol{;}\AgdaSpace{}%
\AgdaGeneralizable{𝓞}\AgdaSymbol{;}\AgdaSpace{}%
\AgdaGeneralizable{𝓥}\AgdaSymbol{)}\<%
\\
\>[0]\AgdaKeyword{open}\AgdaSpace{}%
\AgdaKeyword{import}\AgdaSpace{}%
\AgdaModule{MGS-Subsingleton-Theorems}\AgdaSpace{}%
\AgdaKeyword{using}\AgdaSpace{}%
\AgdaSymbol{(}\AgdaFunction{global-dfunext}\AgdaSymbol{)}\<%
\\
%
\\[\AgdaEmptyExtraSkip]%
\>[0]\AgdaKeyword{module}\AgdaSpace{}%
\AgdaModule{Homomorphisms.Basic}\AgdaSpace{}%
\AgdaSymbol{\{}\AgdaBound{𝑆}\AgdaSpace{}%
\AgdaSymbol{:}\AgdaSpace{}%
\AgdaFunction{Signature}\AgdaSpace{}%
\AgdaGeneralizable{𝓞}\AgdaSpace{}%
\AgdaGeneralizable{𝓥}\AgdaSymbol{\}\{}\AgdaBound{gfe}\AgdaSpace{}%
\AgdaSymbol{:}\AgdaSpace{}%
\AgdaFunction{global-dfunext}\AgdaSymbol{\}}\AgdaSpace{}%
\AgdaKeyword{where}\<%
\\
%
\\[\AgdaEmptyExtraSkip]%
\>[0]\AgdaKeyword{open}\AgdaSpace{}%
\AgdaKeyword{import}\AgdaSpace{}%
\AgdaModule{Algebras.Congruences}\AgdaSymbol{\{}\AgdaArgument{𝑆}\AgdaSpace{}%
\AgdaSymbol{=}\AgdaSpace{}%
\AgdaBound{𝑆}\AgdaSymbol{\}}\AgdaSpace{}%
\AgdaKeyword{public}\<%
\\
\>[0]\AgdaKeyword{open}\AgdaSpace{}%
\AgdaKeyword{import}\AgdaSpace{}%
\AgdaModule{MGS-MLTT}\AgdaSpace{}%
\AgdaKeyword{using}\AgdaSpace{}%
\AgdaSymbol{(}\AgdaOperator{\AgdaFunction{\AgdaUnderscore{}≡⟨\AgdaUnderscore{}⟩\AgdaUnderscore{}}}\AgdaSymbol{;}\AgdaSpace{}%
\AgdaOperator{\AgdaFunction{\AgdaUnderscore{}∎}}\AgdaSymbol{)}\AgdaSpace{}%
\AgdaKeyword{public}\<%
\end{code}
This is a lot to process, and we won't go through every line of code.  Instead we highlight some key points.

\textbf{Notation}. As in~\cite{DeMeo:2021-1}, MLTT stands for Martin-L\"of Type Theory.

First, the \AgdaKeyword{OPTIONS} \emph{pragma} sets some parameters that effect the type theoretic foundations that we assume.  These are discussed in~\cite[\S2.1]{DeMeo:2021-1}, but let us briefly review: \AgdaPragma{without-K} disables \axiomk~(\cite{agdaref-axiomk}); \AgdaPragma{exact-split} makes Agda accept only definitions that use \emph{definitional} equalities~(\cite{agdatools-patternmatching); \AgdaPragma{safe} ensures that nothing is postulated outright, so that every non-MLTT axiom has to be an explicit assumption)~(see \cite{agdaref-safeagda} and~\cite{agdatools-patternmatching}).
  --- (e.g., an argument to a function or module); see



\paragraph{Homomorphisms}\label{homomorphisms}

If \texttt{𝑨} and \texttt{𝑩} are algebraic structures in the signature
\texttt{𝑆}, then a \textbf{homomorphism} is a function
\texttt{h\ :\ ∣\ 𝑨\ ∣\ →\ ∣\ 𝑩\ ∣} from the domain of \texttt{𝑨} to the
domain of \texttt{𝑩} that is compatible (or commutes) with all of the
basic operations of the signature; that is, for all
\texttt{𝑓\ :\ ∣\ 𝑆\ ∣} and all tuples
\texttt{𝒂\ :\ ∥\ 𝑆\ ∥\ 𝑓\ →\ ∣\ 𝑨\ ∣} with values in \texttt{∣\ 𝑨\ ∣},
the following equality holds:

\texttt{h\ ((𝑓\ ̂\ 𝑨)\ 𝒂)\ ≡\ (𝑓\ ̂\ 𝑩)\ (h\ ∘\ 𝒂)}.

Recall, \texttt{h\ ∘\ 𝒂} is the tuple whose i-th component is
\texttt{h\ (𝒂\ i)}.

To formalize this concept, we first define a type representing the
assertion that a function \texttt{h\ :\ ∣\ 𝑨\ ∣\ →\ ∣\ 𝑩\ ∣}, from the
domain of \texttt{𝑨} to the domain of \texttt{𝑩}, \emph{commutes} (or is
\emph{compatible}) with an operation 𝑓, interpreted in the algebras
\texttt{𝑨} and \texttt{𝑩}. Pleasingly, the defining equation of the
previous paragraph can be expressed in Agda without any adulteration.

\begin{code}%
\>[0]\<%
\\
\>[0]\AgdaKeyword{module}\AgdaSpace{}%
\AgdaModule{\AgdaUnderscore{}}\AgdaSpace{}%
\AgdaSymbol{\{}\AgdaBound{𝓤}\AgdaSpace{}%
\AgdaBound{𝓦}\AgdaSpace{}%
\AgdaSymbol{:}\AgdaSpace{}%
\AgdaFunction{Universe}\AgdaSymbol{\}}\AgdaSpace{}%
\AgdaKeyword{where}\<%
\\
%
\\[\AgdaEmptyExtraSkip]%
\>[0][@{}l@{\AgdaIndent{0}}]%
\>[1]\AgdaFunction{compatible-op-map}\AgdaSpace{}%
\AgdaSymbol{:}%
\>[42I]\AgdaSymbol{(}\AgdaBound{𝑨}\AgdaSpace{}%
\AgdaSymbol{:}\AgdaSpace{}%
\AgdaFunction{Algebra}\AgdaSpace{}%
\AgdaBound{𝓤}\AgdaSpace{}%
\AgdaBound{𝑆}\AgdaSymbol{)(}\AgdaBound{𝑩}\AgdaSpace{}%
\AgdaSymbol{:}\AgdaSpace{}%
\AgdaFunction{Algebra}\AgdaSpace{}%
\AgdaBound{𝓦}\AgdaSpace{}%
\AgdaBound{𝑆}\AgdaSymbol{)}\<%
\\
\>[.][@{}l@{}]\<[42I]%
\>[21]\AgdaSymbol{(}\AgdaBound{𝑓}\AgdaSpace{}%
\AgdaSymbol{:}\AgdaSpace{}%
\AgdaOperator{\AgdaFunction{∣}}\AgdaSpace{}%
\AgdaBound{𝑆}\AgdaSpace{}%
\AgdaOperator{\AgdaFunction{∣}}\AgdaSymbol{)(}\AgdaBound{h}\AgdaSpace{}%
\AgdaSymbol{:}\AgdaSpace{}%
\AgdaOperator{\AgdaFunction{∣}}\AgdaSpace{}%
\AgdaBound{𝑨}\AgdaSpace{}%
\AgdaOperator{\AgdaFunction{∣}}%
\>[44]\AgdaSymbol{→}\AgdaSpace{}%
\AgdaOperator{\AgdaFunction{∣}}\AgdaSpace{}%
\AgdaBound{𝑩}\AgdaSpace{}%
\AgdaOperator{\AgdaFunction{∣}}\AgdaSymbol{)}\AgdaSpace{}%
\AgdaSymbol{→}\AgdaSpace{}%
\AgdaBound{𝓥}\AgdaSpace{}%
\AgdaOperator{\AgdaFunction{⊔}}\AgdaSpace{}%
\AgdaBound{𝓤}\AgdaSpace{}%
\AgdaOperator{\AgdaFunction{⊔}}\AgdaSpace{}%
\AgdaBound{𝓦}\AgdaSpace{}%
\AgdaOperator{\AgdaFunction{̇}}\<%
\\
%
\\[\AgdaEmptyExtraSkip]%
%
\>[1]\AgdaFunction{compatible-op-map}\AgdaSpace{}%
\AgdaBound{𝑨}\AgdaSpace{}%
\AgdaBound{𝑩}\AgdaSpace{}%
\AgdaBound{𝑓}\AgdaSpace{}%
\AgdaBound{h}\AgdaSpace{}%
\AgdaSymbol{=}\AgdaSpace{}%
\AgdaSymbol{∀}\AgdaSpace{}%
\AgdaBound{𝑎}\AgdaSpace{}%
\AgdaSymbol{→}\AgdaSpace{}%
\AgdaBound{h}\AgdaSpace{}%
\AgdaSymbol{((}\AgdaBound{𝑓}\AgdaSpace{}%
\AgdaOperator{\AgdaFunction{̂}}\AgdaSpace{}%
\AgdaBound{𝑨}\AgdaSymbol{)}\AgdaSpace{}%
\AgdaBound{𝑎}\AgdaSymbol{)}\AgdaSpace{}%
\AgdaOperator{\AgdaDatatype{≡}}\AgdaSpace{}%
\AgdaSymbol{(}\AgdaBound{𝑓}\AgdaSpace{}%
\AgdaOperator{\AgdaFunction{̂}}\AgdaSpace{}%
\AgdaBound{𝑩}\AgdaSymbol{)}\AgdaSpace{}%
\AgdaSymbol{(}\AgdaBound{h}\AgdaSpace{}%
\AgdaOperator{\AgdaFunction{∘}}\AgdaSpace{}%
\AgdaBound{𝑎}\AgdaSymbol{)}\<%
\\
\>[0]\<%
\end{code}

Note the appearance of the shorthand \texttt{∀\ 𝑎} in the definition of
\texttt{compatible-op-map}. We can get away with this in place of
\texttt{(𝑎\ :\ ∥\ 𝑆\ ∥\ 𝑓\ →\ ∣\ 𝑨\ ∣)} since Agda is able to infer that
the \texttt{𝑎} here must be a tuple on \texttt{∣\ 𝑨\ ∣} of ``length''
\texttt{∥\ 𝑆\ ∥\ 𝑓} (the arity of \texttt{𝑓}).

We now define the type \texttt{hom\ 𝑨\ 𝑩} of homomorphisms from
\texttt{𝑨} to \texttt{𝑩} by first defining the property
\texttt{is-homomorphism}.

\begin{code}%
\>[0]\<%
\\
\>[0][@{}l@{\AgdaIndent{1}}]%
\>[1]\AgdaFunction{is-homomorphism}\AgdaSpace{}%
\AgdaSymbol{:}\AgdaSpace{}%
\AgdaSymbol{(}\AgdaBound{𝑨}\AgdaSpace{}%
\AgdaSymbol{:}\AgdaSpace{}%
\AgdaFunction{Algebra}\AgdaSpace{}%
\AgdaBound{𝓤}\AgdaSpace{}%
\AgdaBound{𝑆}\AgdaSymbol{)(}\AgdaBound{𝑩}\AgdaSpace{}%
\AgdaSymbol{:}\AgdaSpace{}%
\AgdaFunction{Algebra}\AgdaSpace{}%
\AgdaBound{𝓦}\AgdaSpace{}%
\AgdaBound{𝑆}\AgdaSymbol{)}\AgdaSpace{}%
\AgdaSymbol{→}\AgdaSpace{}%
\AgdaSymbol{(}\AgdaOperator{\AgdaFunction{∣}}\AgdaSpace{}%
\AgdaBound{𝑨}\AgdaSpace{}%
\AgdaOperator{\AgdaFunction{∣}}\AgdaSpace{}%
\AgdaSymbol{→}\AgdaSpace{}%
\AgdaOperator{\AgdaFunction{∣}}\AgdaSpace{}%
\AgdaBound{𝑩}\AgdaSpace{}%
\AgdaOperator{\AgdaFunction{∣}}\AgdaSymbol{)}\AgdaSpace{}%
\AgdaSymbol{→}\AgdaSpace{}%
\AgdaBound{𝓞}\AgdaSpace{}%
\AgdaOperator{\AgdaFunction{⊔}}\AgdaSpace{}%
\AgdaBound{𝓥}\AgdaSpace{}%
\AgdaOperator{\AgdaFunction{⊔}}\AgdaSpace{}%
\AgdaBound{𝓤}\AgdaSpace{}%
\AgdaOperator{\AgdaFunction{⊔}}\AgdaSpace{}%
\AgdaBound{𝓦}\AgdaSpace{}%
\AgdaOperator{\AgdaFunction{̇}}\<%
\\
%
\>[1]\AgdaFunction{is-homomorphism}\AgdaSpace{}%
\AgdaBound{𝑨}\AgdaSpace{}%
\AgdaBound{𝑩}\AgdaSpace{}%
\AgdaBound{g}\AgdaSpace{}%
\AgdaSymbol{=}\AgdaSpace{}%
\AgdaSymbol{∀}\AgdaSpace{}%
\AgdaSymbol{(}\AgdaBound{𝑓}\AgdaSpace{}%
\AgdaSymbol{:}\AgdaSpace{}%
\AgdaOperator{\AgdaFunction{∣}}\AgdaSpace{}%
\AgdaBound{𝑆}\AgdaSpace{}%
\AgdaOperator{\AgdaFunction{∣}}\AgdaSymbol{)}\AgdaSpace{}%
\AgdaSymbol{→}\AgdaSpace{}%
\AgdaFunction{compatible-op-map}\AgdaSpace{}%
\AgdaBound{𝑨}\AgdaSpace{}%
\AgdaBound{𝑩}\AgdaSpace{}%
\AgdaBound{𝑓}\AgdaSpace{}%
\AgdaBound{g}\<%
\\
%
\\[\AgdaEmptyExtraSkip]%
%
\>[1]\AgdaFunction{hom}\AgdaSpace{}%
\AgdaSymbol{:}\AgdaSpace{}%
\AgdaFunction{Algebra}\AgdaSpace{}%
\AgdaBound{𝓤}\AgdaSpace{}%
\AgdaBound{𝑆}\AgdaSpace{}%
\AgdaSymbol{→}\AgdaSpace{}%
\AgdaFunction{Algebra}\AgdaSpace{}%
\AgdaBound{𝓦}\AgdaSpace{}%
\AgdaBound{𝑆}%
\>[34]\AgdaSymbol{→}\AgdaSpace{}%
\AgdaBound{𝓞}\AgdaSpace{}%
\AgdaOperator{\AgdaFunction{⊔}}\AgdaSpace{}%
\AgdaBound{𝓥}\AgdaSpace{}%
\AgdaOperator{\AgdaFunction{⊔}}\AgdaSpace{}%
\AgdaBound{𝓤}\AgdaSpace{}%
\AgdaOperator{\AgdaFunction{⊔}}\AgdaSpace{}%
\AgdaBound{𝓦}\AgdaSpace{}%
\AgdaOperator{\AgdaFunction{̇}}\<%
\\
%
\>[1]\AgdaFunction{hom}\AgdaSpace{}%
\AgdaBound{𝑨}\AgdaSpace{}%
\AgdaBound{𝑩}\AgdaSpace{}%
\AgdaSymbol{=}\AgdaSpace{}%
\AgdaFunction{Σ}\AgdaSpace{}%
\AgdaBound{g}\AgdaSpace{}%
\AgdaFunction{꞉}\AgdaSpace{}%
\AgdaSymbol{(}\AgdaOperator{\AgdaFunction{∣}}\AgdaSpace{}%
\AgdaBound{𝑨}\AgdaSpace{}%
\AgdaOperator{\AgdaFunction{∣}}\AgdaSpace{}%
\AgdaSymbol{→}\AgdaSpace{}%
\AgdaOperator{\AgdaFunction{∣}}\AgdaSpace{}%
\AgdaBound{𝑩}\AgdaSpace{}%
\AgdaOperator{\AgdaFunction{∣}}\AgdaSpace{}%
\AgdaSymbol{)}\AgdaSpace{}%
\AgdaFunction{,}\AgdaSpace{}%
\AgdaFunction{is-homomorphism}\AgdaSpace{}%
\AgdaBound{𝑨}\AgdaSpace{}%
\AgdaBound{𝑩}\AgdaSpace{}%
\AgdaBound{g}\<%
\\
\>[0]\<%
\end{code}

A simple example is the identity map, which is proved to be a
homomorphism as follows.

\begin{code}%
\>[0]\<%
\\
\>[0]\AgdaFunction{𝒾𝒹}\AgdaSpace{}%
\AgdaSymbol{:}\AgdaSpace{}%
\AgdaSymbol{\{}\AgdaBound{𝓤}\AgdaSpace{}%
\AgdaSymbol{:}\AgdaSpace{}%
\AgdaFunction{Universe}\AgdaSymbol{\}}\AgdaSpace{}%
\AgdaSymbol{(}\AgdaBound{A}\AgdaSpace{}%
\AgdaSymbol{:}\AgdaSpace{}%
\AgdaFunction{Algebra}\AgdaSpace{}%
\AgdaBound{𝓤}\AgdaSpace{}%
\AgdaBound{𝑆}\AgdaSymbol{)}\AgdaSpace{}%
\AgdaSymbol{→}\AgdaSpace{}%
\AgdaFunction{hom}\AgdaSpace{}%
\AgdaBound{A}\AgdaSpace{}%
\AgdaBound{A}\<%
\\
\>[0]\AgdaFunction{𝒾𝒹}\AgdaSpace{}%
\AgdaSymbol{\AgdaUnderscore{}}\AgdaSpace{}%
\AgdaSymbol{=}\AgdaSpace{}%
\AgdaSymbol{(λ}\AgdaSpace{}%
\AgdaBound{x}\AgdaSpace{}%
\AgdaSymbol{→}\AgdaSpace{}%
\AgdaBound{x}\AgdaSymbol{)}\AgdaSpace{}%
\AgdaOperator{\AgdaInductiveConstructor{,}}\AgdaSpace{}%
\AgdaSymbol{λ}\AgdaSpace{}%
\AgdaBound{\AgdaUnderscore{}}\AgdaSpace{}%
\AgdaBound{\AgdaUnderscore{}}\AgdaSpace{}%
\AgdaSymbol{→}\AgdaSpace{}%
\AgdaInductiveConstructor{𝓇ℯ𝒻𝓁}\<%
\\
%
\\[\AgdaEmptyExtraSkip]%
\>[0]\AgdaFunction{id-is-hom}\AgdaSpace{}%
\AgdaSymbol{:}\AgdaSpace{}%
\AgdaSymbol{\{}\AgdaBound{𝓤}\AgdaSpace{}%
\AgdaSymbol{:}\AgdaSpace{}%
\AgdaFunction{Universe}\AgdaSymbol{\}\{}\AgdaBound{𝑨}\AgdaSpace{}%
\AgdaSymbol{:}\AgdaSpace{}%
\AgdaFunction{Algebra}\AgdaSpace{}%
\AgdaBound{𝓤}\AgdaSpace{}%
\AgdaBound{𝑆}\AgdaSymbol{\}}\AgdaSpace{}%
\AgdaSymbol{→}\AgdaSpace{}%
\AgdaFunction{is-homomorphism}\AgdaSpace{}%
\AgdaBound{𝑨}\AgdaSpace{}%
\AgdaBound{𝑨}\AgdaSpace{}%
\AgdaSymbol{(}\AgdaFunction{𝑖𝑑}\AgdaSpace{}%
\AgdaOperator{\AgdaFunction{∣}}\AgdaSpace{}%
\AgdaBound{𝑨}\AgdaSpace{}%
\AgdaOperator{\AgdaFunction{∣}}\AgdaSymbol{)}\<%
\\
\>[0]\AgdaFunction{id-is-hom}\AgdaSpace{}%
\AgdaSymbol{=}\AgdaSpace{}%
\AgdaSymbol{λ}\AgdaSpace{}%
\AgdaBound{\AgdaUnderscore{}}\AgdaSpace{}%
\AgdaBound{\AgdaUnderscore{}}\AgdaSpace{}%
\AgdaSymbol{→}\AgdaSpace{}%
\AgdaInductiveConstructor{𝓇ℯ𝒻𝓁}\<%
\\
\>[0]\<%
\end{code}

Similarly, we represent \textbf{monomorphisms} (injective homomorphisms)
and \textbf{epimorphisms} (surjective homomorphisms) with the following
types.

\begin{code}%
\>[0]\<%
\\
\>[0]\AgdaKeyword{module}\AgdaSpace{}%
\AgdaModule{\AgdaUnderscore{}}\AgdaSpace{}%
\AgdaSymbol{\{}\AgdaBound{𝓤}\AgdaSpace{}%
\AgdaBound{𝓦}\AgdaSpace{}%
\AgdaSymbol{:}\AgdaSpace{}%
\AgdaFunction{Universe}\AgdaSymbol{\}}\AgdaSpace{}%
\AgdaKeyword{where}\<%
\\
%
\\[\AgdaEmptyExtraSkip]%
\>[0][@{}l@{\AgdaIndent{0}}]%
\>[1]\AgdaFunction{is-monomorphism}\AgdaSpace{}%
\AgdaSymbol{:}\AgdaSpace{}%
\AgdaSymbol{(}\AgdaBound{𝑨}\AgdaSpace{}%
\AgdaSymbol{:}\AgdaSpace{}%
\AgdaFunction{Algebra}\AgdaSpace{}%
\AgdaBound{𝓤}\AgdaSpace{}%
\AgdaBound{𝑆}\AgdaSymbol{)(}\AgdaBound{𝑩}\AgdaSpace{}%
\AgdaSymbol{:}\AgdaSpace{}%
\AgdaFunction{Algebra}\AgdaSpace{}%
\AgdaBound{𝓦}\AgdaSpace{}%
\AgdaBound{𝑆}\AgdaSymbol{)}\AgdaSpace{}%
\AgdaSymbol{→}\AgdaSpace{}%
\AgdaSymbol{(}\AgdaOperator{\AgdaFunction{∣}}\AgdaSpace{}%
\AgdaBound{𝑨}\AgdaSpace{}%
\AgdaOperator{\AgdaFunction{∣}}\AgdaSpace{}%
\AgdaSymbol{→}\AgdaSpace{}%
\AgdaOperator{\AgdaFunction{∣}}\AgdaSpace{}%
\AgdaBound{𝑩}\AgdaSpace{}%
\AgdaOperator{\AgdaFunction{∣}}\AgdaSymbol{)}\AgdaSpace{}%
\AgdaSymbol{→}\AgdaSpace{}%
\AgdaBound{𝓞}\AgdaSpace{}%
\AgdaOperator{\AgdaFunction{⊔}}\AgdaSpace{}%
\AgdaBound{𝓥}\AgdaSpace{}%
\AgdaOperator{\AgdaFunction{⊔}}\AgdaSpace{}%
\AgdaBound{𝓤}\AgdaSpace{}%
\AgdaOperator{\AgdaFunction{⊔}}\AgdaSpace{}%
\AgdaBound{𝓦}\AgdaSpace{}%
\AgdaOperator{\AgdaFunction{̇}}\<%
\\
%
\>[1]\AgdaFunction{is-monomorphism}\AgdaSpace{}%
\AgdaBound{𝑨}\AgdaSpace{}%
\AgdaBound{𝑩}\AgdaSpace{}%
\AgdaBound{g}\AgdaSpace{}%
\AgdaSymbol{=}\AgdaSpace{}%
\AgdaFunction{is-homomorphism}\AgdaSpace{}%
\AgdaBound{𝑨}\AgdaSpace{}%
\AgdaBound{𝑩}\AgdaSpace{}%
\AgdaBound{g}\AgdaSpace{}%
\AgdaOperator{\AgdaFunction{×}}\AgdaSpace{}%
\AgdaFunction{Monic}\AgdaSpace{}%
\AgdaBound{g}\<%
\\
%
\\[\AgdaEmptyExtraSkip]%
%
\>[1]\AgdaFunction{mon}\AgdaSpace{}%
\AgdaSymbol{:}\AgdaSpace{}%
\AgdaFunction{Algebra}\AgdaSpace{}%
\AgdaBound{𝓤}\AgdaSpace{}%
\AgdaBound{𝑆}\AgdaSpace{}%
\AgdaSymbol{→}\AgdaSpace{}%
\AgdaFunction{Algebra}\AgdaSpace{}%
\AgdaBound{𝓦}\AgdaSpace{}%
\AgdaBound{𝑆}%
\>[34]\AgdaSymbol{→}\AgdaSpace{}%
\AgdaBound{𝓞}\AgdaSpace{}%
\AgdaOperator{\AgdaFunction{⊔}}\AgdaSpace{}%
\AgdaBound{𝓥}\AgdaSpace{}%
\AgdaOperator{\AgdaFunction{⊔}}\AgdaSpace{}%
\AgdaBound{𝓤}\AgdaSpace{}%
\AgdaOperator{\AgdaFunction{⊔}}\AgdaSpace{}%
\AgdaBound{𝓦}\AgdaSpace{}%
\AgdaOperator{\AgdaFunction{̇}}\<%
\\
%
\>[1]\AgdaFunction{mon}\AgdaSpace{}%
\AgdaBound{𝑨}\AgdaSpace{}%
\AgdaBound{𝑩}\AgdaSpace{}%
\AgdaSymbol{=}\AgdaSpace{}%
\AgdaFunction{Σ}\AgdaSpace{}%
\AgdaBound{g}\AgdaSpace{}%
\AgdaFunction{꞉}\AgdaSpace{}%
\AgdaSymbol{(}\AgdaOperator{\AgdaFunction{∣}}\AgdaSpace{}%
\AgdaBound{𝑨}\AgdaSpace{}%
\AgdaOperator{\AgdaFunction{∣}}\AgdaSpace{}%
\AgdaSymbol{→}\AgdaSpace{}%
\AgdaOperator{\AgdaFunction{∣}}\AgdaSpace{}%
\AgdaBound{𝑩}\AgdaSpace{}%
\AgdaOperator{\AgdaFunction{∣}}\AgdaSymbol{)}\AgdaSpace{}%
\AgdaFunction{,}\AgdaSpace{}%
\AgdaFunction{is-monomorphism}\AgdaSpace{}%
\AgdaBound{𝑨}\AgdaSpace{}%
\AgdaBound{𝑩}\AgdaSpace{}%
\AgdaBound{g}\<%
\\
%
\\[\AgdaEmptyExtraSkip]%
%
\>[1]\AgdaFunction{is-epimorphism}\AgdaSpace{}%
\AgdaSymbol{:}\AgdaSpace{}%
\AgdaSymbol{(}\AgdaBound{𝑨}\AgdaSpace{}%
\AgdaSymbol{:}\AgdaSpace{}%
\AgdaFunction{Algebra}\AgdaSpace{}%
\AgdaBound{𝓤}\AgdaSpace{}%
\AgdaBound{𝑆}\AgdaSymbol{)(}\AgdaBound{𝑩}\AgdaSpace{}%
\AgdaSymbol{:}\AgdaSpace{}%
\AgdaFunction{Algebra}\AgdaSpace{}%
\AgdaBound{𝓦}\AgdaSpace{}%
\AgdaBound{𝑆}\AgdaSymbol{)}\AgdaSpace{}%
\AgdaSymbol{→}\AgdaSpace{}%
\AgdaSymbol{(}\AgdaOperator{\AgdaFunction{∣}}\AgdaSpace{}%
\AgdaBound{𝑨}\AgdaSpace{}%
\AgdaOperator{\AgdaFunction{∣}}\AgdaSpace{}%
\AgdaSymbol{→}\AgdaSpace{}%
\AgdaOperator{\AgdaFunction{∣}}\AgdaSpace{}%
\AgdaBound{𝑩}\AgdaSpace{}%
\AgdaOperator{\AgdaFunction{∣}}\AgdaSymbol{)}\AgdaSpace{}%
\AgdaSymbol{→}\AgdaSpace{}%
\AgdaBound{𝓞}\AgdaSpace{}%
\AgdaOperator{\AgdaFunction{⊔}}\AgdaSpace{}%
\AgdaBound{𝓥}\AgdaSpace{}%
\AgdaOperator{\AgdaFunction{⊔}}\AgdaSpace{}%
\AgdaBound{𝓤}\AgdaSpace{}%
\AgdaOperator{\AgdaFunction{⊔}}\AgdaSpace{}%
\AgdaBound{𝓦}\AgdaSpace{}%
\AgdaOperator{\AgdaFunction{̇}}\<%
\\
%
\>[1]\AgdaFunction{is-epimorphism}\AgdaSpace{}%
\AgdaBound{𝑨}\AgdaSpace{}%
\AgdaBound{𝑩}\AgdaSpace{}%
\AgdaBound{g}\AgdaSpace{}%
\AgdaSymbol{=}\AgdaSpace{}%
\AgdaFunction{is-homomorphism}\AgdaSpace{}%
\AgdaBound{𝑨}\AgdaSpace{}%
\AgdaBound{𝑩}\AgdaSpace{}%
\AgdaBound{g}\AgdaSpace{}%
\AgdaOperator{\AgdaFunction{×}}\AgdaSpace{}%
\AgdaFunction{Epic}\AgdaSpace{}%
\AgdaBound{g}\<%
\\
%
\\[\AgdaEmptyExtraSkip]%
%
\>[1]\AgdaFunction{epi}\AgdaSpace{}%
\AgdaSymbol{:}\AgdaSpace{}%
\AgdaFunction{Algebra}\AgdaSpace{}%
\AgdaBound{𝓤}\AgdaSpace{}%
\AgdaBound{𝑆}\AgdaSpace{}%
\AgdaSymbol{→}\AgdaSpace{}%
\AgdaFunction{Algebra}\AgdaSpace{}%
\AgdaBound{𝓦}\AgdaSpace{}%
\AgdaBound{𝑆}%
\>[34]\AgdaSymbol{→}\AgdaSpace{}%
\AgdaBound{𝓞}\AgdaSpace{}%
\AgdaOperator{\AgdaFunction{⊔}}\AgdaSpace{}%
\AgdaBound{𝓥}\AgdaSpace{}%
\AgdaOperator{\AgdaFunction{⊔}}\AgdaSpace{}%
\AgdaBound{𝓤}\AgdaSpace{}%
\AgdaOperator{\AgdaFunction{⊔}}\AgdaSpace{}%
\AgdaBound{𝓦}\AgdaSpace{}%
\AgdaOperator{\AgdaFunction{̇}}\<%
\\
%
\>[1]\AgdaFunction{epi}\AgdaSpace{}%
\AgdaBound{𝑨}\AgdaSpace{}%
\AgdaBound{𝑩}\AgdaSpace{}%
\AgdaSymbol{=}\AgdaSpace{}%
\AgdaFunction{Σ}\AgdaSpace{}%
\AgdaBound{g}\AgdaSpace{}%
\AgdaFunction{꞉}\AgdaSpace{}%
\AgdaSymbol{(}\AgdaOperator{\AgdaFunction{∣}}\AgdaSpace{}%
\AgdaBound{𝑨}\AgdaSpace{}%
\AgdaOperator{\AgdaFunction{∣}}\AgdaSpace{}%
\AgdaSymbol{→}\AgdaSpace{}%
\AgdaOperator{\AgdaFunction{∣}}\AgdaSpace{}%
\AgdaBound{𝑩}\AgdaSpace{}%
\AgdaOperator{\AgdaFunction{∣}}\AgdaSymbol{)}\AgdaSpace{}%
\AgdaFunction{,}\AgdaSpace{}%
\AgdaFunction{is-epimorphism}\AgdaSpace{}%
\AgdaBound{𝑨}\AgdaSpace{}%
\AgdaBound{𝑩}\AgdaSpace{}%
\AgdaBound{g}\<%
\\
\>[0]\<%
\end{code}

Finally, it will be convenient to have functions that return the
\emph{hom reduct} of an inhabitant of \texttt{mon} or \texttt{epi}.

\begin{code}%
\>[0]\<%
\\
\>[0][@{}l@{\AgdaIndent{1}}]%
\>[1]\AgdaFunction{mon-to-hom}\AgdaSpace{}%
\AgdaSymbol{:}\AgdaSpace{}%
\AgdaSymbol{(}\AgdaBound{𝑨}\AgdaSpace{}%
\AgdaSymbol{:}\AgdaSpace{}%
\AgdaFunction{Algebra}\AgdaSpace{}%
\AgdaBound{𝓤}\AgdaSpace{}%
\AgdaBound{𝑆}\AgdaSymbol{)\{}\AgdaBound{𝑩}\AgdaSpace{}%
\AgdaSymbol{:}\AgdaSpace{}%
\AgdaFunction{Algebra}\AgdaSpace{}%
\AgdaBound{𝓦}\AgdaSpace{}%
\AgdaBound{𝑆}\AgdaSymbol{\}}\AgdaSpace{}%
\AgdaSymbol{→}\AgdaSpace{}%
\AgdaFunction{mon}\AgdaSpace{}%
\AgdaBound{𝑨}\AgdaSpace{}%
\AgdaBound{𝑩}\AgdaSpace{}%
\AgdaSymbol{→}\AgdaSpace{}%
\AgdaFunction{hom}\AgdaSpace{}%
\AgdaBound{𝑨}\AgdaSpace{}%
\AgdaBound{𝑩}\<%
\\
%
\>[1]\AgdaFunction{mon-to-hom}\AgdaSpace{}%
\AgdaBound{𝑨}\AgdaSpace{}%
\AgdaBound{ϕ}\AgdaSpace{}%
\AgdaSymbol{=}\AgdaSpace{}%
\AgdaOperator{\AgdaFunction{∣}}\AgdaSpace{}%
\AgdaBound{ϕ}\AgdaSpace{}%
\AgdaOperator{\AgdaFunction{∣}}\AgdaSpace{}%
\AgdaOperator{\AgdaInductiveConstructor{,}}\AgdaSpace{}%
\AgdaFunction{fst}\AgdaSpace{}%
\AgdaOperator{\AgdaFunction{∥}}\AgdaSpace{}%
\AgdaBound{ϕ}\AgdaSpace{}%
\AgdaOperator{\AgdaFunction{∥}}\<%
\\
%
\\[\AgdaEmptyExtraSkip]%
%
\>[1]\AgdaFunction{epi-to-hom}\AgdaSpace{}%
\AgdaSymbol{:}\AgdaSpace{}%
\AgdaSymbol{\{}\AgdaBound{𝑨}\AgdaSpace{}%
\AgdaSymbol{:}\AgdaSpace{}%
\AgdaFunction{Algebra}\AgdaSpace{}%
\AgdaBound{𝓤}\AgdaSpace{}%
\AgdaBound{𝑆}\AgdaSymbol{\}(}\AgdaBound{𝑩}\AgdaSpace{}%
\AgdaSymbol{:}\AgdaSpace{}%
\AgdaFunction{Algebra}\AgdaSpace{}%
\AgdaBound{𝓦}\AgdaSpace{}%
\AgdaBound{𝑆}\AgdaSymbol{)}\AgdaSpace{}%
\AgdaSymbol{→}\AgdaSpace{}%
\AgdaFunction{epi}\AgdaSpace{}%
\AgdaBound{𝑨}\AgdaSpace{}%
\AgdaBound{𝑩}\AgdaSpace{}%
\AgdaSymbol{→}\AgdaSpace{}%
\AgdaFunction{hom}\AgdaSpace{}%
\AgdaBound{𝑨}\AgdaSpace{}%
\AgdaBound{𝑩}\<%
\\
%
\>[1]\AgdaFunction{epi-to-hom}\AgdaSpace{}%
\AgdaSymbol{\AgdaUnderscore{}}\AgdaSpace{}%
\AgdaBound{ϕ}\AgdaSpace{}%
\AgdaSymbol{=}\AgdaSpace{}%
\AgdaOperator{\AgdaFunction{∣}}\AgdaSpace{}%
\AgdaBound{ϕ}\AgdaSpace{}%
\AgdaOperator{\AgdaFunction{∣}}\AgdaSpace{}%
\AgdaOperator{\AgdaInductiveConstructor{,}}\AgdaSpace{}%
\AgdaFunction{fst}\AgdaSpace{}%
\AgdaOperator{\AgdaFunction{∥}}\AgdaSpace{}%
\AgdaBound{ϕ}\AgdaSpace{}%
\AgdaOperator{\AgdaFunction{∥}}\<%
\\
\>[0]\<%
\end{code}

\paragraph{Equalizers in Agda}\label{equalizers-in-agda}

Recall, the equalizer of two functions (resp., homomorphisms)
\texttt{g\ h\ :\ A\ →\ B} is the subset of \texttt{A} on which the
values of the functions \texttt{g} and \texttt{h} agree. We define the
equalizer of functions and homomorphisms in Agda as follows.

\begin{code}%
\>[0]\<%
\\
\>[0]\AgdaKeyword{module}\AgdaSpace{}%
\AgdaModule{\AgdaUnderscore{}}\AgdaSpace{}%
\AgdaSymbol{\{}\AgdaBound{𝓤}\AgdaSpace{}%
\AgdaBound{𝓦}\AgdaSpace{}%
\AgdaSymbol{:}\AgdaSpace{}%
\AgdaFunction{Universe}\AgdaSymbol{\}\{}\AgdaBound{𝑨}\AgdaSpace{}%
\AgdaSymbol{:}\AgdaSpace{}%
\AgdaFunction{Algebra}\AgdaSpace{}%
\AgdaBound{𝓤}\AgdaSpace{}%
\AgdaBound{𝑆}\AgdaSymbol{\}}\AgdaSpace{}%
\AgdaKeyword{where}\<%
\\
%
\\[\AgdaEmptyExtraSkip]%
\>[0][@{}l@{\AgdaIndent{0}}]%
\>[1]\AgdaFunction{𝑬}\AgdaSpace{}%
\AgdaSymbol{:}\AgdaSpace{}%
\AgdaSymbol{\{}\AgdaBound{𝑩}\AgdaSpace{}%
\AgdaSymbol{:}\AgdaSpace{}%
\AgdaFunction{Algebra}\AgdaSpace{}%
\AgdaBound{𝓦}\AgdaSpace{}%
\AgdaBound{𝑆}\AgdaSymbol{\}(}\AgdaBound{g}\AgdaSpace{}%
\AgdaBound{h}\AgdaSpace{}%
\AgdaSymbol{:}\AgdaSpace{}%
\AgdaOperator{\AgdaFunction{∣}}\AgdaSpace{}%
\AgdaBound{𝑨}\AgdaSpace{}%
\AgdaOperator{\AgdaFunction{∣}}\AgdaSpace{}%
\AgdaSymbol{→}\AgdaSpace{}%
\AgdaOperator{\AgdaFunction{∣}}\AgdaSpace{}%
\AgdaBound{𝑩}\AgdaSpace{}%
\AgdaOperator{\AgdaFunction{∣}}\AgdaSymbol{)}\AgdaSpace{}%
\AgdaSymbol{→}\AgdaSpace{}%
\AgdaFunction{Pred}\AgdaSpace{}%
\AgdaOperator{\AgdaFunction{∣}}\AgdaSpace{}%
\AgdaBound{𝑨}\AgdaSpace{}%
\AgdaOperator{\AgdaFunction{∣}}\AgdaSpace{}%
\AgdaBound{𝓦}\<%
\\
%
\>[1]\AgdaFunction{𝑬}\AgdaSpace{}%
\AgdaBound{g}\AgdaSpace{}%
\AgdaBound{h}\AgdaSpace{}%
\AgdaBound{x}\AgdaSpace{}%
\AgdaSymbol{=}\AgdaSpace{}%
\AgdaBound{g}\AgdaSpace{}%
\AgdaBound{x}\AgdaSpace{}%
\AgdaOperator{\AgdaDatatype{≡}}\AgdaSpace{}%
\AgdaBound{h}\AgdaSpace{}%
\AgdaBound{x}\<%
\\
%
\\[\AgdaEmptyExtraSkip]%
%
\>[1]\AgdaFunction{𝑬𝑯}\AgdaSpace{}%
\AgdaSymbol{:}\AgdaSpace{}%
\AgdaSymbol{(}\AgdaBound{𝑩}\AgdaSpace{}%
\AgdaSymbol{:}\AgdaSpace{}%
\AgdaFunction{Algebra}\AgdaSpace{}%
\AgdaBound{𝓦}\AgdaSpace{}%
\AgdaBound{𝑆}\AgdaSymbol{)(}\AgdaBound{g}\AgdaSpace{}%
\AgdaBound{h}\AgdaSpace{}%
\AgdaSymbol{:}\AgdaSpace{}%
\AgdaFunction{hom}\AgdaSpace{}%
\AgdaBound{𝑨}\AgdaSpace{}%
\AgdaBound{𝑩}\AgdaSymbol{)}\AgdaSpace{}%
\AgdaSymbol{→}\AgdaSpace{}%
\AgdaFunction{Pred}\AgdaSpace{}%
\AgdaOperator{\AgdaFunction{∣}}\AgdaSpace{}%
\AgdaBound{𝑨}\AgdaSpace{}%
\AgdaOperator{\AgdaFunction{∣}}\AgdaSpace{}%
\AgdaBound{𝓦}\<%
\\
%
\>[1]\AgdaFunction{𝑬𝑯}\AgdaSpace{}%
\AgdaBound{𝑩}\AgdaSpace{}%
\AgdaBound{g}\AgdaSpace{}%
\AgdaBound{h}\AgdaSpace{}%
\AgdaBound{x}\AgdaSpace{}%
\AgdaSymbol{=}\AgdaSpace{}%
\AgdaOperator{\AgdaFunction{∣}}\AgdaSpace{}%
\AgdaBound{g}\AgdaSpace{}%
\AgdaOperator{\AgdaFunction{∣}}\AgdaSpace{}%
\AgdaBound{x}\AgdaSpace{}%
\AgdaOperator{\AgdaDatatype{≡}}\AgdaSpace{}%
\AgdaOperator{\AgdaFunction{∣}}\AgdaSpace{}%
\AgdaBound{h}\AgdaSpace{}%
\AgdaOperator{\AgdaFunction{∣}}\AgdaSpace{}%
\AgdaBound{x}\<%
\\
\>[0]\<%
\end{code}

We will define subuniverses in the {[}Subalgebras.Subuniverses{]}
module, but we note here that the equalizer of homomorphisms from
\texttt{𝑨} to \texttt{𝑩} will turn out to be subuniverse of \texttt{𝑨}.
Indeed, this is easily proved as follows.

\begin{code}%
\>[0]\<%
\\
\>[0][@{}l@{\AgdaIndent{1}}]%
\>[1]\AgdaFunction{𝑬𝑯-closed}\AgdaSpace{}%
\AgdaSymbol{:}%
\>[494I]\AgdaSymbol{(}\AgdaBound{𝑩}\AgdaSpace{}%
\AgdaSymbol{:}\AgdaSpace{}%
\AgdaFunction{Algebra}\AgdaSpace{}%
\AgdaBound{𝓦}\AgdaSpace{}%
\AgdaBound{𝑆}\AgdaSymbol{)(}\AgdaBound{g}\AgdaSpace{}%
\AgdaBound{h}\AgdaSpace{}%
\AgdaSymbol{:}\AgdaSpace{}%
\AgdaFunction{hom}\AgdaSpace{}%
\AgdaBound{𝑨}\AgdaSpace{}%
\AgdaBound{𝑩}\AgdaSymbol{)}\<%
\\
\>[.][@{}l@{}]\<[494I]%
\>[13]\AgdaSymbol{(}\AgdaBound{𝑓}\AgdaSpace{}%
\AgdaSymbol{:}\AgdaSpace{}%
\AgdaOperator{\AgdaFunction{∣}}\AgdaSpace{}%
\AgdaBound{𝑆}\AgdaSpace{}%
\AgdaOperator{\AgdaFunction{∣}}\AgdaSymbol{)}\AgdaSpace{}%
\AgdaSymbol{(}\AgdaBound{𝒂}\AgdaSpace{}%
\AgdaSymbol{:}\AgdaSpace{}%
\AgdaOperator{\AgdaFunction{∥}}\AgdaSpace{}%
\AgdaBound{𝑆}\AgdaSpace{}%
\AgdaOperator{\AgdaFunction{∥}}\AgdaSpace{}%
\AgdaBound{𝑓}\AgdaSpace{}%
\AgdaSymbol{→}\AgdaSpace{}%
\AgdaOperator{\AgdaFunction{∣}}\AgdaSpace{}%
\AgdaBound{𝑨}\AgdaSpace{}%
\AgdaOperator{\AgdaFunction{∣}}\AgdaSymbol{)}\AgdaSpace{}%
\AgdaSymbol{→}\AgdaSpace{}%
\AgdaSymbol{(∀}\AgdaSpace{}%
\AgdaBound{x}\AgdaSpace{}%
\AgdaSymbol{→}\AgdaSpace{}%
\AgdaSymbol{(}\AgdaBound{𝒂}\AgdaSpace{}%
\AgdaBound{x}\AgdaSymbol{)}\AgdaSpace{}%
\AgdaOperator{\AgdaFunction{∈}}\AgdaSpace{}%
\AgdaFunction{𝑬𝑯}\AgdaSpace{}%
\AgdaBound{𝑩}\AgdaSpace{}%
\AgdaBound{g}\AgdaSpace{}%
\AgdaBound{h}\AgdaSymbol{)}\<%
\\
%
\>[13]\AgdaComment{-----------------------------------------------------}\<%
\\
\>[1][@{}l@{\AgdaIndent{0}}]%
\>[3]\AgdaSymbol{→}%
\>[13]\AgdaOperator{\AgdaFunction{∣}}\AgdaSpace{}%
\AgdaBound{g}\AgdaSpace{}%
\AgdaOperator{\AgdaFunction{∣}}\AgdaSpace{}%
\AgdaSymbol{((}\AgdaBound{𝑓}\AgdaSpace{}%
\AgdaOperator{\AgdaFunction{̂}}\AgdaSpace{}%
\AgdaBound{𝑨}\AgdaSymbol{)}\AgdaSpace{}%
\AgdaBound{𝒂}\AgdaSymbol{)}\AgdaSpace{}%
\AgdaOperator{\AgdaDatatype{≡}}\AgdaSpace{}%
\AgdaOperator{\AgdaFunction{∣}}\AgdaSpace{}%
\AgdaBound{h}\AgdaSpace{}%
\AgdaOperator{\AgdaFunction{∣}}\AgdaSpace{}%
\AgdaSymbol{((}\AgdaBound{𝑓}\AgdaSpace{}%
\AgdaOperator{\AgdaFunction{̂}}\AgdaSpace{}%
\AgdaBound{𝑨}\AgdaSymbol{)}\AgdaSpace{}%
\AgdaBound{𝒂}\AgdaSymbol{)}\<%
\\
%
\\[\AgdaEmptyExtraSkip]%
%
\>[1]\AgdaFunction{𝑬𝑯-closed}\AgdaSpace{}%
\AgdaBound{𝑩}\AgdaSpace{}%
\AgdaBound{g}\AgdaSpace{}%
\AgdaBound{h}\AgdaSpace{}%
\AgdaBound{𝑓}\AgdaSpace{}%
\AgdaBound{𝒂}\AgdaSpace{}%
\AgdaBound{p}\AgdaSpace{}%
\AgdaSymbol{=}%
\>[550I]\AgdaOperator{\AgdaFunction{∣}}\AgdaSpace{}%
\AgdaBound{g}\AgdaSpace{}%
\AgdaOperator{\AgdaFunction{∣}}\AgdaSpace{}%
\AgdaSymbol{((}\AgdaBound{𝑓}\AgdaSpace{}%
\AgdaOperator{\AgdaFunction{̂}}\AgdaSpace{}%
\AgdaBound{𝑨}\AgdaSymbol{)}\AgdaSpace{}%
\AgdaBound{𝒂}\AgdaSymbol{)}%
\>[45]\AgdaOperator{\AgdaFunction{≡⟨}}\AgdaSpace{}%
\AgdaOperator{\AgdaFunction{∥}}\AgdaSpace{}%
\AgdaBound{g}\AgdaSpace{}%
\AgdaOperator{\AgdaFunction{∥}}\AgdaSpace{}%
\AgdaBound{𝑓}\AgdaSpace{}%
\AgdaBound{𝒂}\AgdaSpace{}%
\AgdaOperator{\AgdaFunction{⟩}}\<%
\\
\>[.][@{}l@{}]\<[550I]%
\>[25]\AgdaSymbol{(}\AgdaBound{𝑓}\AgdaSpace{}%
\AgdaOperator{\AgdaFunction{̂}}\AgdaSpace{}%
\AgdaBound{𝑩}\AgdaSymbol{)(}\AgdaOperator{\AgdaFunction{∣}}\AgdaSpace{}%
\AgdaBound{g}\AgdaSpace{}%
\AgdaOperator{\AgdaFunction{∣}}\AgdaSpace{}%
\AgdaOperator{\AgdaFunction{∘}}\AgdaSpace{}%
\AgdaBound{𝒂}\AgdaSymbol{)}%
\>[45]\AgdaOperator{\AgdaFunction{≡⟨}}\AgdaSpace{}%
\AgdaFunction{ap}\AgdaSpace{}%
\AgdaSymbol{(}\AgdaBound{𝑓}\AgdaSpace{}%
\AgdaOperator{\AgdaFunction{̂}}\AgdaSpace{}%
\AgdaBound{𝑩}\AgdaSymbol{)(}\AgdaBound{gfe}\AgdaSpace{}%
\AgdaBound{p}\AgdaSymbol{)}\AgdaSpace{}%
\AgdaOperator{\AgdaFunction{⟩}}\<%
\\
%
\>[25]\AgdaSymbol{(}\AgdaBound{𝑓}\AgdaSpace{}%
\AgdaOperator{\AgdaFunction{̂}}\AgdaSpace{}%
\AgdaBound{𝑩}\AgdaSymbol{)(}\AgdaOperator{\AgdaFunction{∣}}\AgdaSpace{}%
\AgdaBound{h}\AgdaSpace{}%
\AgdaOperator{\AgdaFunction{∣}}\AgdaSpace{}%
\AgdaOperator{\AgdaFunction{∘}}\AgdaSpace{}%
\AgdaBound{𝒂}\AgdaSymbol{)}%
\>[45]\AgdaOperator{\AgdaFunction{≡⟨}}\AgdaSpace{}%
\AgdaSymbol{(}\AgdaOperator{\AgdaFunction{∥}}\AgdaSpace{}%
\AgdaBound{h}\AgdaSpace{}%
\AgdaOperator{\AgdaFunction{∥}}\AgdaSpace{}%
\AgdaBound{𝑓}\AgdaSpace{}%
\AgdaBound{𝒂}\AgdaSymbol{)}\AgdaOperator{\AgdaFunction{⁻¹}}\AgdaSpace{}%
\AgdaOperator{\AgdaFunction{⟩}}\<%
\\
%
\>[25]\AgdaOperator{\AgdaFunction{∣}}\AgdaSpace{}%
\AgdaBound{h}\AgdaSpace{}%
\AgdaOperator{\AgdaFunction{∣}}\AgdaSpace{}%
\AgdaSymbol{((}\AgdaBound{𝑓}\AgdaSpace{}%
\AgdaOperator{\AgdaFunction{̂}}\AgdaSpace{}%
\AgdaBound{𝑨}\AgdaSymbol{)}\AgdaSpace{}%
\AgdaBound{𝒂}\AgdaSymbol{)}%
\>[45]\AgdaOperator{\AgdaFunction{∎}}\<%
\\
\>[0]\<%
\end{code}

\paragraph{Kernels of Homomorphisms}\label{kernels-of-homomorphisms}

The kernel of a homomorphism is a congruence relation and conversely for
every congruence relation θ, there exists a homomorphism with kernel θ
(namely, that canonical projection onto the quotient modulo θ).

\begin{code}%
\>[0]\<%
\\
%
\\[\AgdaEmptyExtraSkip]%
\>[0]\AgdaKeyword{open}\AgdaSpace{}%
\AgdaModule{Congruence}\<%
\\
%
\\[\AgdaEmptyExtraSkip]%
\>[0]\AgdaKeyword{module}\AgdaSpace{}%
\AgdaModule{\AgdaUnderscore{}}\AgdaSpace{}%
\AgdaSymbol{\{}\AgdaBound{𝓤}\AgdaSpace{}%
\AgdaBound{𝓦}\AgdaSpace{}%
\AgdaSymbol{:}\AgdaSpace{}%
\AgdaFunction{Universe}\AgdaSymbol{\}}\AgdaSpace{}%
\AgdaKeyword{where}\<%
\\
%
\\[\AgdaEmptyExtraSkip]%
\>[0][@{}l@{\AgdaIndent{0}}]%
\>[1]\AgdaFunction{homker-compatible}\AgdaSpace{}%
\AgdaSymbol{:}\AgdaSpace{}%
\AgdaSymbol{\{}\AgdaBound{𝑨}\AgdaSpace{}%
\AgdaSymbol{:}\AgdaSpace{}%
\AgdaFunction{Algebra}\AgdaSpace{}%
\AgdaBound{𝓤}\AgdaSpace{}%
\AgdaBound{𝑆}\AgdaSymbol{\}(}\AgdaBound{𝑩}\AgdaSpace{}%
\AgdaSymbol{:}\AgdaSpace{}%
\AgdaFunction{Algebra}\AgdaSpace{}%
\AgdaBound{𝓦}\AgdaSpace{}%
\AgdaBound{𝑆}\AgdaSymbol{)(}\AgdaBound{h}\AgdaSpace{}%
\AgdaSymbol{:}\AgdaSpace{}%
\AgdaFunction{hom}\AgdaSpace{}%
\AgdaBound{𝑨}\AgdaSpace{}%
\AgdaBound{𝑩}\AgdaSymbol{)}\<%
\\
\>[1][@{}l@{\AgdaIndent{0}}]%
\>[2]\AgdaSymbol{→}%
\>[21]\AgdaFunction{compatible}\AgdaSpace{}%
\AgdaBound{𝑨}\AgdaSpace{}%
\AgdaSymbol{(}\AgdaFunction{KER-rel}\AgdaSpace{}%
\AgdaOperator{\AgdaFunction{∣}}\AgdaSpace{}%
\AgdaBound{h}\AgdaSpace{}%
\AgdaOperator{\AgdaFunction{∣}}\AgdaSymbol{)}\<%
\\
%
\\[\AgdaEmptyExtraSkip]%
%
\>[1]\AgdaFunction{homker-compatible}\AgdaSpace{}%
\AgdaSymbol{\{}\AgdaBound{𝑨}\AgdaSymbol{\}}\AgdaSpace{}%
\AgdaBound{𝑩}\AgdaSpace{}%
\AgdaBound{h}\AgdaSpace{}%
\AgdaBound{f}\AgdaSpace{}%
\AgdaSymbol{\{}\AgdaBound{𝒂}\AgdaSymbol{\}\{}\AgdaBound{𝒂'}\AgdaSymbol{\}}\AgdaSpace{}%
\AgdaBound{Kerhab}\AgdaSpace{}%
\AgdaSymbol{=}\AgdaSpace{}%
\AgdaFunction{γ}\AgdaSpace{}%
\AgdaKeyword{where}\<%
\\
\>[1][@{}l@{\AgdaIndent{0}}]%
\>[3]\AgdaFunction{γ}\AgdaSpace{}%
\AgdaSymbol{:}\AgdaSpace{}%
\AgdaOperator{\AgdaFunction{∣}}\AgdaSpace{}%
\AgdaBound{h}\AgdaSpace{}%
\AgdaOperator{\AgdaFunction{∣}}\AgdaSpace{}%
\AgdaSymbol{((}\AgdaBound{f}\AgdaSpace{}%
\AgdaOperator{\AgdaFunction{̂}}\AgdaSpace{}%
\AgdaBound{𝑨}\AgdaSymbol{)}\AgdaSpace{}%
\AgdaBound{𝒂}\AgdaSymbol{)}%
\>[28]\AgdaOperator{\AgdaDatatype{≡}}\AgdaSpace{}%
\AgdaOperator{\AgdaFunction{∣}}\AgdaSpace{}%
\AgdaBound{h}\AgdaSpace{}%
\AgdaOperator{\AgdaFunction{∣}}\AgdaSpace{}%
\AgdaSymbol{((}\AgdaBound{f}\AgdaSpace{}%
\AgdaOperator{\AgdaFunction{̂}}\AgdaSpace{}%
\AgdaBound{𝑨}\AgdaSymbol{)}\AgdaSpace{}%
\AgdaBound{𝒂'}\AgdaSymbol{)}\<%
\\
%
\>[3]\AgdaFunction{γ}\AgdaSpace{}%
\AgdaSymbol{=}%
\>[644I]\AgdaOperator{\AgdaFunction{∣}}\AgdaSpace{}%
\AgdaBound{h}\AgdaSpace{}%
\AgdaOperator{\AgdaFunction{∣}}\AgdaSpace{}%
\AgdaSymbol{((}\AgdaBound{f}\AgdaSpace{}%
\AgdaOperator{\AgdaFunction{̂}}\AgdaSpace{}%
\AgdaBound{𝑨}\AgdaSymbol{)}\AgdaSpace{}%
\AgdaBound{𝒂}\AgdaSymbol{)}%
\>[28]\AgdaOperator{\AgdaFunction{≡⟨}}\AgdaSpace{}%
\AgdaOperator{\AgdaFunction{∥}}\AgdaSpace{}%
\AgdaBound{h}\AgdaSpace{}%
\AgdaOperator{\AgdaFunction{∥}}\AgdaSpace{}%
\AgdaBound{f}\AgdaSpace{}%
\AgdaBound{𝒂}\AgdaSpace{}%
\AgdaOperator{\AgdaFunction{⟩}}\<%
\\
\>[.][@{}l@{}]\<[644I]%
\>[7]\AgdaSymbol{(}\AgdaBound{f}\AgdaSpace{}%
\AgdaOperator{\AgdaFunction{̂}}\AgdaSpace{}%
\AgdaBound{𝑩}\AgdaSymbol{)}\AgdaSpace{}%
\AgdaSymbol{(}\AgdaOperator{\AgdaFunction{∣}}\AgdaSpace{}%
\AgdaBound{h}\AgdaSpace{}%
\AgdaOperator{\AgdaFunction{∣}}\AgdaSpace{}%
\AgdaOperator{\AgdaFunction{∘}}\AgdaSpace{}%
\AgdaBound{𝒂}\AgdaSymbol{)}%
\>[28]\AgdaOperator{\AgdaFunction{≡⟨}}\AgdaSpace{}%
\AgdaFunction{ap}\AgdaSpace{}%
\AgdaSymbol{(λ}\AgdaSpace{}%
\AgdaBound{-}\AgdaSpace{}%
\AgdaSymbol{→}\AgdaSpace{}%
\AgdaSymbol{(}\AgdaBound{f}\AgdaSpace{}%
\AgdaOperator{\AgdaFunction{̂}}\AgdaSpace{}%
\AgdaBound{𝑩}\AgdaSymbol{)}\AgdaSpace{}%
\AgdaBound{-}\AgdaSymbol{)}\AgdaSpace{}%
\AgdaSymbol{(}\AgdaBound{gfe}\AgdaSpace{}%
\AgdaSymbol{λ}\AgdaSpace{}%
\AgdaBound{x}\AgdaSpace{}%
\AgdaSymbol{→}\AgdaSpace{}%
\AgdaBound{Kerhab}\AgdaSpace{}%
\AgdaBound{x}\AgdaSymbol{)}\AgdaSpace{}%
\AgdaOperator{\AgdaFunction{⟩}}\<%
\\
%
\>[7]\AgdaSymbol{(}\AgdaBound{f}\AgdaSpace{}%
\AgdaOperator{\AgdaFunction{̂}}\AgdaSpace{}%
\AgdaBound{𝑩}\AgdaSymbol{)}\AgdaSpace{}%
\AgdaSymbol{(}\AgdaOperator{\AgdaFunction{∣}}\AgdaSpace{}%
\AgdaBound{h}\AgdaSpace{}%
\AgdaOperator{\AgdaFunction{∣}}\AgdaSpace{}%
\AgdaOperator{\AgdaFunction{∘}}\AgdaSpace{}%
\AgdaBound{𝒂'}\AgdaSymbol{)}\AgdaSpace{}%
\AgdaOperator{\AgdaFunction{≡⟨}}\AgdaSpace{}%
\AgdaSymbol{(}\AgdaOperator{\AgdaFunction{∥}}\AgdaSpace{}%
\AgdaBound{h}\AgdaSpace{}%
\AgdaOperator{\AgdaFunction{∥}}\AgdaSpace{}%
\AgdaBound{f}\AgdaSpace{}%
\AgdaBound{𝒂'}\AgdaSymbol{)}\AgdaOperator{\AgdaFunction{⁻¹}}\AgdaSpace{}%
\AgdaOperator{\AgdaFunction{⟩}}\<%
\\
%
\>[7]\AgdaOperator{\AgdaFunction{∣}}\AgdaSpace{}%
\AgdaBound{h}\AgdaSpace{}%
\AgdaOperator{\AgdaFunction{∣}}\AgdaSpace{}%
\AgdaSymbol{((}\AgdaBound{f}\AgdaSpace{}%
\AgdaOperator{\AgdaFunction{̂}}\AgdaSpace{}%
\AgdaBound{𝑨}\AgdaSymbol{)}\AgdaSpace{}%
\AgdaBound{𝒂'}\AgdaSymbol{)}%
\>[28]\AgdaOperator{\AgdaFunction{∎}}\<%
\\
%
\\[\AgdaEmptyExtraSkip]%
%
\\[\AgdaEmptyExtraSkip]%
%
\>[1]\AgdaFunction{homker-equivalence}\AgdaSpace{}%
\AgdaSymbol{:}\AgdaSpace{}%
\AgdaSymbol{\{}\AgdaBound{𝑨}\AgdaSpace{}%
\AgdaSymbol{:}\AgdaSpace{}%
\AgdaFunction{Algebra}\AgdaSpace{}%
\AgdaBound{𝓤}\AgdaSpace{}%
\AgdaBound{𝑆}\AgdaSymbol{\}(}\AgdaBound{𝑩}\AgdaSpace{}%
\AgdaSymbol{:}\AgdaSpace{}%
\AgdaFunction{Algebra}\AgdaSpace{}%
\AgdaBound{𝓦}\AgdaSpace{}%
\AgdaBound{𝑆}\AgdaSymbol{)(}\AgdaBound{h}\AgdaSpace{}%
\AgdaSymbol{:}\AgdaSpace{}%
\AgdaFunction{hom}\AgdaSpace{}%
\AgdaBound{𝑨}\AgdaSpace{}%
\AgdaBound{𝑩}\AgdaSymbol{)}\<%
\\
\>[1][@{}l@{\AgdaIndent{0}}]%
\>[2]\AgdaSymbol{→}%
\>[22]\AgdaRecord{IsEquivalence}\AgdaSpace{}%
\AgdaSymbol{(}\AgdaFunction{KER-rel}\AgdaSpace{}%
\AgdaOperator{\AgdaFunction{∣}}\AgdaSpace{}%
\AgdaBound{h}\AgdaSpace{}%
\AgdaOperator{\AgdaFunction{∣}}\AgdaSymbol{)}\<%
\\
%
\\[\AgdaEmptyExtraSkip]%
%
\>[1]\AgdaFunction{homker-equivalence}\AgdaSpace{}%
\AgdaBound{𝑨}\AgdaSpace{}%
\AgdaBound{h}\AgdaSpace{}%
\AgdaSymbol{=}\AgdaSpace{}%
\AgdaFunction{map-kernel-IsEquivalence}\AgdaSpace{}%
\AgdaOperator{\AgdaFunction{∣}}\AgdaSpace{}%
\AgdaBound{h}\AgdaSpace{}%
\AgdaOperator{\AgdaFunction{∣}}\<%
\\
\>[0]\<%
\end{code}

It is convenient to define a function that takes a homomorphism and
constructs a congruence from its kernel. We call this function
\texttt{kercon}.

\begin{code}%
\>[0]\<%
\\
\>[0][@{}l@{\AgdaIndent{1}}]%
\>[1]\AgdaFunction{kercon}\AgdaSpace{}%
\AgdaSymbol{:}\AgdaSpace{}%
\AgdaSymbol{\{}\AgdaBound{𝑨}\AgdaSpace{}%
\AgdaSymbol{:}\AgdaSpace{}%
\AgdaFunction{Algebra}\AgdaSpace{}%
\AgdaBound{𝓤}\AgdaSpace{}%
\AgdaBound{𝑆}\AgdaSymbol{\}(}\AgdaBound{𝑩}\AgdaSpace{}%
\AgdaSymbol{:}\AgdaSpace{}%
\AgdaFunction{Algebra}\AgdaSpace{}%
\AgdaBound{𝓦}\AgdaSpace{}%
\AgdaBound{𝑆}\AgdaSymbol{)(}\AgdaBound{h}\AgdaSpace{}%
\AgdaSymbol{:}\AgdaSpace{}%
\AgdaFunction{hom}\AgdaSpace{}%
\AgdaBound{𝑨}\AgdaSpace{}%
\AgdaBound{𝑩}\AgdaSymbol{)}\AgdaSpace{}%
\AgdaSymbol{→}\AgdaSpace{}%
\AgdaRecord{Congruence}\AgdaSpace{}%
\AgdaBound{𝑨}\<%
\\
%
\\[\AgdaEmptyExtraSkip]%
%
\>[1]\AgdaFunction{kercon}\AgdaSpace{}%
\AgdaBound{𝑩}\AgdaSpace{}%
\AgdaBound{h}\AgdaSpace{}%
\AgdaSymbol{=}\AgdaSpace{}%
\AgdaInductiveConstructor{mkcon}\AgdaSpace{}%
\AgdaSymbol{(}\AgdaFunction{KER-rel}\AgdaSpace{}%
\AgdaOperator{\AgdaFunction{∣}}\AgdaSpace{}%
\AgdaBound{h}\AgdaSpace{}%
\AgdaOperator{\AgdaFunction{∣}}\AgdaSymbol{)(}\AgdaFunction{homker-compatible}\AgdaSpace{}%
\AgdaBound{𝑩}\AgdaSpace{}%
\AgdaBound{h}\AgdaSymbol{)(}\AgdaFunction{homker-equivalence}\AgdaSpace{}%
\AgdaBound{𝑩}\AgdaSpace{}%
\AgdaBound{h}\AgdaSymbol{)}\<%
\\
\>[0]\<%
\end{code}

From this congruence we construct the corresponding quotient.

\begin{code}%
\>[0]\<%
\\
\>[0][@{}l@{\AgdaIndent{1}}]%
\>[1]\AgdaFunction{kerquo}\AgdaSpace{}%
\AgdaSymbol{:}\AgdaSpace{}%
\AgdaSymbol{\{}\AgdaBound{𝑨}\AgdaSpace{}%
\AgdaSymbol{:}\AgdaSpace{}%
\AgdaFunction{Algebra}\AgdaSpace{}%
\AgdaBound{𝓤}\AgdaSpace{}%
\AgdaBound{𝑆}\AgdaSymbol{\}(}\AgdaBound{𝑩}\AgdaSpace{}%
\AgdaSymbol{:}\AgdaSpace{}%
\AgdaFunction{Algebra}\AgdaSpace{}%
\AgdaBound{𝓦}\AgdaSpace{}%
\AgdaBound{𝑆}\AgdaSymbol{)(}\AgdaBound{h}\AgdaSpace{}%
\AgdaSymbol{:}\AgdaSpace{}%
\AgdaFunction{hom}\AgdaSpace{}%
\AgdaBound{𝑨}\AgdaSpace{}%
\AgdaBound{𝑩}\AgdaSymbol{)}\AgdaSpace{}%
\AgdaSymbol{→}\AgdaSpace{}%
\AgdaFunction{Algebra}\AgdaSpace{}%
\AgdaSymbol{(}\AgdaBound{𝓤}\AgdaSpace{}%
\AgdaOperator{\AgdaFunction{⊔}}\AgdaSpace{}%
\AgdaBound{𝓦}\AgdaSpace{}%
\AgdaOperator{\AgdaFunction{⁺}}\AgdaSymbol{)}\AgdaSpace{}%
\AgdaBound{𝑆}\<%
\\
%
\\[\AgdaEmptyExtraSkip]%
%
\>[1]\AgdaFunction{kerquo}\AgdaSpace{}%
\AgdaSymbol{\{}\AgdaBound{𝑨}\AgdaSymbol{\}}\AgdaSpace{}%
\AgdaBound{𝑩}\AgdaSpace{}%
\AgdaBound{h}\AgdaSpace{}%
\AgdaSymbol{=}\AgdaSpace{}%
\AgdaBound{𝑨}\AgdaSpace{}%
\AgdaOperator{\AgdaFunction{╱}}\AgdaSpace{}%
\AgdaSymbol{(}\AgdaFunction{kercon}\AgdaSpace{}%
\AgdaBound{𝑩}\AgdaSpace{}%
\AgdaBound{h}\AgdaSymbol{)}\<%
\\
%
\\[\AgdaEmptyExtraSkip]%
%
\>[1]\AgdaComment{-- NOTATION.}\<%
\\
%
\>[1]\AgdaOperator{\AgdaFunction{\AgdaUnderscore{}[\AgdaUnderscore{}]/ker\AgdaUnderscore{}}}\AgdaSpace{}%
\AgdaSymbol{:}\AgdaSpace{}%
\AgdaSymbol{(}\AgdaBound{𝑨}\AgdaSpace{}%
\AgdaSymbol{:}\AgdaSpace{}%
\AgdaFunction{Algebra}\AgdaSpace{}%
\AgdaBound{𝓤}\AgdaSpace{}%
\AgdaBound{𝑆}\AgdaSymbol{)(}\AgdaBound{𝑩}\AgdaSpace{}%
\AgdaSymbol{:}\AgdaSpace{}%
\AgdaFunction{Algebra}\AgdaSpace{}%
\AgdaBound{𝓦}\AgdaSpace{}%
\AgdaBound{𝑆}\AgdaSymbol{)(}\AgdaBound{h}\AgdaSpace{}%
\AgdaSymbol{:}\AgdaSpace{}%
\AgdaFunction{hom}\AgdaSpace{}%
\AgdaBound{𝑨}\AgdaSpace{}%
\AgdaBound{𝑩}\AgdaSymbol{)}\AgdaSpace{}%
\AgdaSymbol{→}\AgdaSpace{}%
\AgdaFunction{Algebra}\AgdaSpace{}%
\AgdaSymbol{(}\AgdaBound{𝓤}\AgdaSpace{}%
\AgdaOperator{\AgdaFunction{⊔}}\AgdaSpace{}%
\AgdaBound{𝓦}\AgdaSpace{}%
\AgdaOperator{\AgdaFunction{⁺}}\AgdaSymbol{)}\AgdaSpace{}%
\AgdaBound{𝑆}\<%
\\
%
\>[1]\AgdaBound{𝑨}\AgdaSpace{}%
\AgdaOperator{\AgdaFunction{[}}\AgdaSpace{}%
\AgdaBound{𝑩}\AgdaSpace{}%
\AgdaOperator{\AgdaFunction{]/ker}}\AgdaSpace{}%
\AgdaBound{h}\AgdaSpace{}%
\AgdaSymbol{=}\AgdaSpace{}%
\AgdaFunction{kerquo}\AgdaSpace{}%
\AgdaSymbol{\{}\AgdaBound{𝑨}\AgdaSymbol{\}}\AgdaSpace{}%
\AgdaBound{𝑩}\AgdaSpace{}%
\AgdaBound{h}\<%
\\
%
\\[\AgdaEmptyExtraSkip]%
%
\>[1]\AgdaKeyword{infix}\AgdaSpace{}%
\AgdaNumber{60}\AgdaSpace{}%
\AgdaOperator{\AgdaFunction{\AgdaUnderscore{}[\AgdaUnderscore{}]/ker\AgdaUnderscore{}}}\<%
\\
\>[0]\<%
\end{code}

Given an algebra \texttt{𝑨} and a congruence \texttt{θ}, the canonical
epimorphism from an algebra \texttt{𝑨} to \texttt{𝑨\ ╱\ θ} is defined as
follows.

\begin{code}%
\>[0]\<%
\\
\>[0][@{}l@{\AgdaIndent{1}}]%
\>[1]\AgdaFunction{πepi}\AgdaSpace{}%
\AgdaSymbol{:}\AgdaSpace{}%
\AgdaSymbol{\{}\AgdaBound{𝑨}\AgdaSpace{}%
\AgdaSymbol{:}\AgdaSpace{}%
\AgdaFunction{Algebra}\AgdaSpace{}%
\AgdaBound{𝓤}\AgdaSpace{}%
\AgdaBound{𝑆}\AgdaSymbol{\}}\AgdaSpace{}%
\AgdaSymbol{(}\AgdaBound{θ}\AgdaSpace{}%
\AgdaSymbol{:}\AgdaSpace{}%
\AgdaRecord{Congruence}\AgdaSymbol{\{}\AgdaBound{𝓤}\AgdaSymbol{\}\{}\AgdaBound{𝓦}\AgdaSymbol{\}}\AgdaSpace{}%
\AgdaBound{𝑨}\AgdaSymbol{)}\AgdaSpace{}%
\AgdaSymbol{→}\AgdaSpace{}%
\AgdaFunction{epi}\AgdaSpace{}%
\AgdaBound{𝑨}\AgdaSpace{}%
\AgdaSymbol{(}\AgdaBound{𝑨}\AgdaSpace{}%
\AgdaOperator{\AgdaFunction{╱}}\AgdaSpace{}%
\AgdaBound{θ}\AgdaSymbol{)}\<%
\\
%
\>[1]\AgdaFunction{πepi}\AgdaSpace{}%
\AgdaSymbol{\{}\AgdaBound{𝑨}\AgdaSymbol{\}}\AgdaSpace{}%
\AgdaBound{θ}\AgdaSpace{}%
\AgdaSymbol{=}\AgdaSpace{}%
\AgdaFunction{cπ}\AgdaSpace{}%
\AgdaOperator{\AgdaInductiveConstructor{,}}\AgdaSpace{}%
\AgdaFunction{cπ-is-hom}\AgdaSpace{}%
\AgdaOperator{\AgdaInductiveConstructor{,}}\AgdaSpace{}%
\AgdaFunction{cπ-is-epic}\AgdaSpace{}%
\AgdaKeyword{where}\<%
\\
%
\\[\AgdaEmptyExtraSkip]%
\>[1][@{}l@{\AgdaIndent{0}}]%
\>[2]\AgdaFunction{cπ}\AgdaSpace{}%
\AgdaSymbol{:}\AgdaSpace{}%
\AgdaOperator{\AgdaFunction{∣}}\AgdaSpace{}%
\AgdaBound{𝑨}\AgdaSpace{}%
\AgdaOperator{\AgdaFunction{∣}}\AgdaSpace{}%
\AgdaSymbol{→}\AgdaSpace{}%
\AgdaOperator{\AgdaFunction{∣}}\AgdaSpace{}%
\AgdaBound{𝑨}\AgdaSpace{}%
\AgdaOperator{\AgdaFunction{╱}}\AgdaSpace{}%
\AgdaBound{θ}\AgdaSpace{}%
\AgdaOperator{\AgdaFunction{∣}}\<%
\\
%
\>[2]\AgdaFunction{cπ}\AgdaSpace{}%
\AgdaBound{a}\AgdaSpace{}%
\AgdaSymbol{=}\AgdaSpace{}%
\AgdaOperator{\AgdaFunction{⟦}}\AgdaSpace{}%
\AgdaBound{a}\AgdaSpace{}%
\AgdaOperator{\AgdaFunction{⟧}}\AgdaSymbol{\{}\AgdaOperator{\AgdaField{⟨}}\AgdaSpace{}%
\AgdaBound{θ}\AgdaSpace{}%
\AgdaOperator{\AgdaField{⟩}}\AgdaSymbol{\}}\<%
\\
%
\\[\AgdaEmptyExtraSkip]%
%
\>[2]\AgdaFunction{cπ-is-hom}\AgdaSpace{}%
\AgdaSymbol{:}\AgdaSpace{}%
\AgdaFunction{is-homomorphism}\AgdaSpace{}%
\AgdaBound{𝑨}\AgdaSpace{}%
\AgdaSymbol{(}\AgdaBound{𝑨}\AgdaSpace{}%
\AgdaOperator{\AgdaFunction{╱}}\AgdaSpace{}%
\AgdaBound{θ}\AgdaSymbol{)}\AgdaSpace{}%
\AgdaFunction{cπ}\<%
\\
%
\>[2]\AgdaFunction{cπ-is-hom}\AgdaSpace{}%
\AgdaSymbol{\AgdaUnderscore{}}\AgdaSpace{}%
\AgdaSymbol{\AgdaUnderscore{}}\AgdaSpace{}%
\AgdaSymbol{=}\AgdaSpace{}%
\AgdaInductiveConstructor{𝓇ℯ𝒻𝓁}\<%
\\
%
\\[\AgdaEmptyExtraSkip]%
%
\>[2]\AgdaFunction{cπ-is-epic}\AgdaSpace{}%
\AgdaSymbol{:}\AgdaSpace{}%
\AgdaFunction{Epic}\AgdaSpace{}%
\AgdaFunction{cπ}\<%
\\
%
\>[2]\AgdaFunction{cπ-is-epic}\AgdaSpace{}%
\AgdaSymbol{(}\AgdaDottedPattern{\AgdaSymbol{.(}}\AgdaDottedPattern{\AgdaOperator{\AgdaField{⟨}}}\AgdaSpace{}%
\AgdaDottedPattern{\AgdaBound{θ}}\AgdaSpace{}%
\AgdaDottedPattern{\AgdaOperator{\AgdaField{⟩}}}\AgdaSpace{}%
\AgdaDottedPattern{\AgdaBound{a}}\AgdaDottedPattern{\AgdaSymbol{)}}\AgdaSpace{}%
\AgdaOperator{\AgdaInductiveConstructor{,}}\AgdaSpace{}%
\AgdaBound{a}\AgdaSpace{}%
\AgdaOperator{\AgdaInductiveConstructor{,}}\AgdaSpace{}%
\AgdaInductiveConstructor{refl}\AgdaSpace{}%
\AgdaSymbol{\AgdaUnderscore{})}\AgdaSpace{}%
\AgdaSymbol{=}\AgdaSpace{}%
\AgdaInductiveConstructor{Image\AgdaUnderscore{}∋\AgdaUnderscore{}.im}\AgdaSpace{}%
\AgdaBound{a}\<%
\\
\>[0]\<%
\end{code}

To obtain the homomorphism part (or ``hom reduct'') of the canonical
epimorphism, we apply \texttt{epi-to-hom}.

\begin{code}%
\>[0]\<%
\\
\>[0][@{}l@{\AgdaIndent{1}}]%
\>[1]\AgdaFunction{πhom}\AgdaSpace{}%
\AgdaSymbol{:}\AgdaSpace{}%
\AgdaSymbol{\{}\AgdaBound{𝑨}\AgdaSpace{}%
\AgdaSymbol{:}\AgdaSpace{}%
\AgdaFunction{Algebra}\AgdaSpace{}%
\AgdaBound{𝓤}\AgdaSpace{}%
\AgdaBound{𝑆}\AgdaSymbol{\}(}\AgdaBound{θ}\AgdaSpace{}%
\AgdaSymbol{:}\AgdaSpace{}%
\AgdaRecord{Congruence}\AgdaSymbol{\{}\AgdaBound{𝓤}\AgdaSymbol{\}\{}\AgdaBound{𝓦}\AgdaSymbol{\}}\AgdaSpace{}%
\AgdaBound{𝑨}\AgdaSymbol{)}\AgdaSpace{}%
\AgdaSymbol{→}\AgdaSpace{}%
\AgdaFunction{hom}\AgdaSpace{}%
\AgdaBound{𝑨}\AgdaSpace{}%
\AgdaSymbol{(}\AgdaBound{𝑨}\AgdaSpace{}%
\AgdaOperator{\AgdaFunction{╱}}\AgdaSpace{}%
\AgdaBound{θ}\AgdaSymbol{)}\<%
\\
%
\>[1]\AgdaFunction{πhom}\AgdaSpace{}%
\AgdaSymbol{\{}\AgdaBound{𝑨}\AgdaSymbol{\}}\AgdaSpace{}%
\AgdaBound{θ}\AgdaSpace{}%
\AgdaSymbol{=}\AgdaSpace{}%
\AgdaFunction{epi-to-hom}\AgdaSpace{}%
\AgdaSymbol{(}\AgdaBound{𝑨}\AgdaSpace{}%
\AgdaOperator{\AgdaFunction{╱}}\AgdaSpace{}%
\AgdaBound{θ}\AgdaSymbol{)}\AgdaSpace{}%
\AgdaSymbol{(}\AgdaFunction{πepi}\AgdaSpace{}%
\AgdaBound{θ}\AgdaSymbol{)}\<%
\\
\>[0]\<%
\end{code}

We combine the foregoing to define a function that takes 𝑆-algebras
\texttt{𝑨} and \texttt{𝑩}, and a homomorphism \texttt{h\ :\ hom\ 𝑨\ 𝑩}
and returns the canonical epimorphism from \texttt{𝑨} onto
\texttt{𝑨\ {[}\ 𝑩\ {]}/ker\ h}. (Recall, the latter is the special
notation we defined above for the quotient of \texttt{𝑨} modulo the
kernel of \texttt{h}.)

\begin{code}%
\>[0]\<%
\\
\>[0][@{}l@{\AgdaIndent{1}}]%
\>[1]\AgdaFunction{πker}\AgdaSpace{}%
\AgdaSymbol{:}\AgdaSpace{}%
\AgdaSymbol{\{}\AgdaBound{𝑨}\AgdaSpace{}%
\AgdaSymbol{:}\AgdaSpace{}%
\AgdaFunction{Algebra}\AgdaSpace{}%
\AgdaBound{𝓤}\AgdaSpace{}%
\AgdaBound{𝑆}\AgdaSymbol{\}(}\AgdaBound{𝑩}\AgdaSpace{}%
\AgdaSymbol{:}\AgdaSpace{}%
\AgdaFunction{Algebra}\AgdaSpace{}%
\AgdaBound{𝓦}\AgdaSpace{}%
\AgdaBound{𝑆}\AgdaSymbol{)(}\AgdaBound{h}\AgdaSpace{}%
\AgdaSymbol{:}\AgdaSpace{}%
\AgdaFunction{hom}\AgdaSpace{}%
\AgdaBound{𝑨}\AgdaSpace{}%
\AgdaBound{𝑩}\AgdaSymbol{)}%
\>[57]\AgdaSymbol{→}%
\>[60]\AgdaFunction{epi}\AgdaSpace{}%
\AgdaBound{𝑨}\AgdaSpace{}%
\AgdaSymbol{(}\AgdaBound{𝑨}\AgdaSpace{}%
\AgdaOperator{\AgdaFunction{[}}\AgdaSpace{}%
\AgdaBound{𝑩}\AgdaSpace{}%
\AgdaOperator{\AgdaFunction{]/ker}}\AgdaSpace{}%
\AgdaBound{h}\AgdaSymbol{)}\<%
\\
%
\\[\AgdaEmptyExtraSkip]%
%
\>[1]\AgdaFunction{πker}\AgdaSpace{}%
\AgdaSymbol{\{}\AgdaBound{𝑨}\AgdaSymbol{\}}\AgdaSpace{}%
\AgdaBound{𝑩}\AgdaSpace{}%
\AgdaBound{h}\AgdaSpace{}%
\AgdaSymbol{=}\AgdaSpace{}%
\AgdaFunction{πepi}\AgdaSpace{}%
\AgdaSymbol{(}\AgdaFunction{kercon}\AgdaSpace{}%
\AgdaBound{𝑩}\AgdaSpace{}%
\AgdaBound{h}\AgdaSymbol{)}\<%
\\
\>[0]\<%
\end{code}

The kernel of the canonical projection of \texttt{𝑨} onto
\texttt{𝑨\ /\ θ} is equal to \texttt{θ}, but since equality of
inhabitants of certain types (like \texttt{Congruence} or \texttt{Rel})
can be a tricky business, we settle for proving the containment
\texttt{𝑨\ /\ θ\ ⊆\ θ}. Of the two containments, this is the easier one
to prove; luckily it is also the one we need later.

\begin{code}%
\>[0]\<%
\\
\>[0]\AgdaFunction{ker-in-con}\AgdaSpace{}%
\AgdaSymbol{:}\AgdaSpace{}%
\AgdaSymbol{\{}\AgdaBound{𝓤}\AgdaSpace{}%
\AgdaBound{𝓦}\AgdaSpace{}%
\AgdaSymbol{:}\AgdaSpace{}%
\AgdaFunction{Universe}\AgdaSymbol{\}(}\AgdaBound{𝑨}\AgdaSpace{}%
\AgdaSymbol{:}\AgdaSpace{}%
\AgdaFunction{Algebra}\AgdaSpace{}%
\AgdaBound{𝓤}\AgdaSpace{}%
\AgdaBound{𝑆}\AgdaSymbol{)(}\AgdaBound{θ}\AgdaSpace{}%
\AgdaSymbol{:}\AgdaSpace{}%
\AgdaRecord{Congruence}\AgdaSymbol{\{}\AgdaBound{𝓤}\AgdaSymbol{\}\{}\AgdaBound{𝓦}\AgdaSymbol{\}}\AgdaSpace{}%
\AgdaBound{𝑨}\AgdaSymbol{)(}\AgdaBound{x}\AgdaSpace{}%
\AgdaBound{y}\AgdaSpace{}%
\AgdaSymbol{:}\AgdaSpace{}%
\AgdaOperator{\AgdaFunction{∣}}\AgdaSpace{}%
\AgdaBound{𝑨}\AgdaSpace{}%
\AgdaOperator{\AgdaFunction{∣}}\AgdaSymbol{)}\<%
\\
\>[0][@{}l@{\AgdaIndent{0}}]%
\>[1]\AgdaSymbol{→}%
\>[13]\AgdaOperator{\AgdaField{⟨}}\AgdaSpace{}%
\AgdaFunction{kercon}\AgdaSpace{}%
\AgdaSymbol{(}\AgdaBound{𝑨}\AgdaSpace{}%
\AgdaOperator{\AgdaFunction{╱}}\AgdaSpace{}%
\AgdaBound{θ}\AgdaSymbol{)}\AgdaSpace{}%
\AgdaSymbol{(}\AgdaFunction{πhom}\AgdaSpace{}%
\AgdaBound{θ}\AgdaSymbol{)}\AgdaSpace{}%
\AgdaOperator{\AgdaField{⟩}}\AgdaSpace{}%
\AgdaBound{x}\AgdaSpace{}%
\AgdaBound{y}%
\>[46]\AgdaSymbol{→}%
\>[49]\AgdaOperator{\AgdaField{⟨}}\AgdaSpace{}%
\AgdaBound{θ}\AgdaSpace{}%
\AgdaOperator{\AgdaField{⟩}}\AgdaSpace{}%
\AgdaBound{x}\AgdaSpace{}%
\AgdaBound{y}\<%
\\
%
\\[\AgdaEmptyExtraSkip]%
\>[0]\AgdaFunction{ker-in-con}\AgdaSpace{}%
\AgdaBound{𝑨}\AgdaSpace{}%
\AgdaBound{θ}\AgdaSpace{}%
\AgdaBound{x}\AgdaSpace{}%
\AgdaBound{y}\AgdaSpace{}%
\AgdaBound{hyp}\AgdaSpace{}%
\AgdaSymbol{=}\AgdaSpace{}%
\AgdaFunction{╱-refl}\AgdaSpace{}%
\AgdaBound{θ}\AgdaSpace{}%
\AgdaBound{hyp}\<%
\\
\>[0]\<%
\end{code}

\paragraph{Product homomorphisms}\label{product-homomorphisms}

\begin{code}%
\>[0]\<%
\\
\>[0]\AgdaFunction{⨅-hom-co}\AgdaSpace{}%
\AgdaSymbol{:}\AgdaSpace{}%
\AgdaSymbol{\{}\AgdaBound{𝓠}\AgdaSpace{}%
\AgdaBound{𝓤}\AgdaSpace{}%
\AgdaBound{𝓘}\AgdaSpace{}%
\AgdaSymbol{:}\AgdaSpace{}%
\AgdaFunction{Universe}\AgdaSymbol{\}(}\AgdaBound{𝑨}\AgdaSpace{}%
\AgdaSymbol{:}\AgdaSpace{}%
\AgdaFunction{Algebra}\AgdaSpace{}%
\AgdaBound{𝓠}\AgdaSpace{}%
\AgdaBound{𝑆}\AgdaSymbol{)\{}\AgdaBound{I}\AgdaSpace{}%
\AgdaSymbol{:}\AgdaSpace{}%
\AgdaBound{𝓘}\AgdaSpace{}%
\AgdaOperator{\AgdaFunction{̇}}\AgdaSymbol{\}(}\AgdaBound{ℬ}\AgdaSpace{}%
\AgdaSymbol{:}\AgdaSpace{}%
\AgdaBound{I}\AgdaSpace{}%
\AgdaSymbol{→}\AgdaSpace{}%
\AgdaFunction{Algebra}\AgdaSpace{}%
\AgdaBound{𝓤}\AgdaSpace{}%
\AgdaBound{𝑆}\AgdaSymbol{)}\<%
\\
\>[0][@{}l@{\AgdaIndent{0}}]%
\>[1]\AgdaSymbol{→}%
\>[11]\AgdaSymbol{(∀}\AgdaSpace{}%
\AgdaBound{i}\AgdaSpace{}%
\AgdaSymbol{→}\AgdaSpace{}%
\AgdaFunction{hom}\AgdaSpace{}%
\AgdaBound{𝑨}\AgdaSpace{}%
\AgdaSymbol{(}\AgdaBound{ℬ}\AgdaSpace{}%
\AgdaBound{i}\AgdaSymbol{))}\<%
\\
%
\>[11]\AgdaComment{--------------------}\<%
\\
%
\>[1]\AgdaSymbol{→}%
\>[11]\AgdaFunction{hom}\AgdaSpace{}%
\AgdaBound{𝑨}\AgdaSpace{}%
\AgdaSymbol{(}\AgdaFunction{⨅}\AgdaSpace{}%
\AgdaBound{ℬ}\AgdaSymbol{)}\<%
\\
%
\\[\AgdaEmptyExtraSkip]%
\>[0]\AgdaFunction{⨅-hom-co}\AgdaSpace{}%
\AgdaBound{𝑨}\AgdaSpace{}%
\AgdaSymbol{\{}\AgdaBound{I}\AgdaSymbol{\}}\AgdaSpace{}%
\AgdaBound{ℬ}\AgdaSpace{}%
\AgdaBound{homs}\AgdaSpace{}%
\AgdaSymbol{=}\AgdaSpace{}%
\AgdaFunction{ϕ}\AgdaSpace{}%
\AgdaOperator{\AgdaInductiveConstructor{,}}\AgdaSpace{}%
\AgdaFunction{ϕhom}\<%
\\
\>[0][@{}l@{\AgdaIndent{0}}]%
\>[1]\AgdaKeyword{where}\<%
\\
\>[1][@{}l@{\AgdaIndent{0}}]%
\>[2]\AgdaFunction{ϕ}\AgdaSpace{}%
\AgdaSymbol{:}\AgdaSpace{}%
\AgdaOperator{\AgdaFunction{∣}}\AgdaSpace{}%
\AgdaBound{𝑨}\AgdaSpace{}%
\AgdaOperator{\AgdaFunction{∣}}\AgdaSpace{}%
\AgdaSymbol{→}\AgdaSpace{}%
\AgdaOperator{\AgdaFunction{∣}}\AgdaSpace{}%
\AgdaFunction{⨅}\AgdaSpace{}%
\AgdaBound{ℬ}\AgdaSpace{}%
\AgdaOperator{\AgdaFunction{∣}}\<%
\\
%
\>[2]\AgdaFunction{ϕ}\AgdaSpace{}%
\AgdaBound{a}\AgdaSpace{}%
\AgdaSymbol{=}\AgdaSpace{}%
\AgdaSymbol{λ}\AgdaSpace{}%
\AgdaBound{i}\AgdaSpace{}%
\AgdaSymbol{→}\AgdaSpace{}%
\AgdaOperator{\AgdaFunction{∣}}\AgdaSpace{}%
\AgdaBound{homs}\AgdaSpace{}%
\AgdaBound{i}\AgdaSpace{}%
\AgdaOperator{\AgdaFunction{∣}}\AgdaSpace{}%
\AgdaBound{a}\<%
\\
%
\\[\AgdaEmptyExtraSkip]%
%
\>[2]\AgdaFunction{ϕhom}\AgdaSpace{}%
\AgdaSymbol{:}\AgdaSpace{}%
\AgdaFunction{is-homomorphism}\AgdaSpace{}%
\AgdaBound{𝑨}\AgdaSpace{}%
\AgdaSymbol{(}\AgdaFunction{⨅}\AgdaSpace{}%
\AgdaBound{ℬ}\AgdaSymbol{)}\AgdaSpace{}%
\AgdaFunction{ϕ}\<%
\\
%
\>[2]\AgdaFunction{ϕhom}\AgdaSpace{}%
\AgdaBound{𝑓}\AgdaSpace{}%
\AgdaBound{𝒂}\AgdaSpace{}%
\AgdaSymbol{=}\AgdaSpace{}%
\AgdaBound{gfe}\AgdaSpace{}%
\AgdaSymbol{λ}\AgdaSpace{}%
\AgdaBound{i}\AgdaSpace{}%
\AgdaSymbol{→}\AgdaSpace{}%
\AgdaOperator{\AgdaFunction{∥}}\AgdaSpace{}%
\AgdaBound{homs}\AgdaSpace{}%
\AgdaBound{i}\AgdaSpace{}%
\AgdaOperator{\AgdaFunction{∥}}\AgdaSpace{}%
\AgdaBound{𝑓}\AgdaSpace{}%
\AgdaSymbol{(λ}\AgdaSpace{}%
\AgdaBound{x}\AgdaSpace{}%
\AgdaSymbol{→}\AgdaSpace{}%
\AgdaBound{𝒂}\AgdaSpace{}%
\AgdaBound{x}\AgdaSymbol{)}\<%
\\
%
\\[\AgdaEmptyExtraSkip]%
%
\\[\AgdaEmptyExtraSkip]%
\>[0]\AgdaFunction{⨅-hom}\AgdaSpace{}%
\AgdaSymbol{:}\AgdaSpace{}%
\AgdaSymbol{\{}\AgdaBound{𝓠}\AgdaSpace{}%
\AgdaBound{𝓤}\AgdaSpace{}%
\AgdaBound{𝓘}\AgdaSpace{}%
\AgdaSymbol{:}\AgdaSpace{}%
\AgdaFunction{Universe}\AgdaSymbol{\}\{}\AgdaBound{I}\AgdaSpace{}%
\AgdaSymbol{:}\AgdaSpace{}%
\AgdaBound{𝓘}\AgdaSpace{}%
\AgdaOperator{\AgdaFunction{̇}}\AgdaSymbol{\}(}\AgdaBound{𝒜}\AgdaSpace{}%
\AgdaSymbol{:}\AgdaSpace{}%
\AgdaBound{I}\AgdaSpace{}%
\AgdaSymbol{→}\AgdaSpace{}%
\AgdaFunction{Algebra}\AgdaSpace{}%
\AgdaBound{𝓠}\AgdaSpace{}%
\AgdaBound{𝑆}\AgdaSymbol{)(}\AgdaBound{ℬ}\AgdaSpace{}%
\AgdaSymbol{:}\AgdaSpace{}%
\AgdaBound{I}\AgdaSpace{}%
\AgdaSymbol{→}\AgdaSpace{}%
\AgdaFunction{Algebra}\AgdaSpace{}%
\AgdaBound{𝓤}\AgdaSpace{}%
\AgdaBound{𝑆}\AgdaSymbol{)}\<%
\\
\>[0][@{}l@{\AgdaIndent{0}}]%
\>[1]\AgdaSymbol{→}%
\>[8]\AgdaSymbol{((}\AgdaBound{i}\AgdaSpace{}%
\AgdaSymbol{:}\AgdaSpace{}%
\AgdaBound{I}\AgdaSymbol{)}\AgdaSpace{}%
\AgdaSymbol{→}\AgdaSpace{}%
\AgdaFunction{hom}\AgdaSpace{}%
\AgdaSymbol{(}\AgdaBound{𝒜}\AgdaSpace{}%
\AgdaBound{i}\AgdaSymbol{)(}\AgdaBound{ℬ}\AgdaSpace{}%
\AgdaBound{i}\AgdaSymbol{))}\<%
\\
%
\>[8]\AgdaComment{---------------------------}\<%
\\
%
\>[1]\AgdaSymbol{→}%
\>[8]\AgdaFunction{hom}\AgdaSpace{}%
\AgdaSymbol{(}\AgdaFunction{⨅}\AgdaSpace{}%
\AgdaBound{𝒜}\AgdaSymbol{)}\AgdaSpace{}%
\AgdaSymbol{(}\AgdaFunction{⨅}\AgdaSpace{}%
\AgdaBound{ℬ}\AgdaSymbol{)}\<%
\\
%
\\[\AgdaEmptyExtraSkip]%
\>[0]\AgdaFunction{⨅-hom}\AgdaSpace{}%
\AgdaBound{𝒜}\AgdaSpace{}%
\AgdaBound{ℬ}\AgdaSpace{}%
\AgdaBound{homs}\AgdaSpace{}%
\AgdaSymbol{=}\AgdaSpace{}%
\AgdaFunction{ϕ}\AgdaSpace{}%
\AgdaOperator{\AgdaInductiveConstructor{,}}\AgdaSpace{}%
\AgdaFunction{ϕhom}\<%
\\
\>[0][@{}l@{\AgdaIndent{0}}]%
\>[1]\AgdaKeyword{where}\<%
\\
\>[1][@{}l@{\AgdaIndent{0}}]%
\>[2]\AgdaFunction{ϕ}\AgdaSpace{}%
\AgdaSymbol{:}\AgdaSpace{}%
\AgdaOperator{\AgdaFunction{∣}}\AgdaSpace{}%
\AgdaFunction{⨅}\AgdaSpace{}%
\AgdaBound{𝒜}\AgdaSpace{}%
\AgdaOperator{\AgdaFunction{∣}}\AgdaSpace{}%
\AgdaSymbol{→}\AgdaSpace{}%
\AgdaOperator{\AgdaFunction{∣}}\AgdaSpace{}%
\AgdaFunction{⨅}\AgdaSpace{}%
\AgdaBound{ℬ}\AgdaSpace{}%
\AgdaOperator{\AgdaFunction{∣}}\<%
\\
%
\>[2]\AgdaFunction{ϕ}\AgdaSpace{}%
\AgdaSymbol{=}\AgdaSpace{}%
\AgdaSymbol{λ}\AgdaSpace{}%
\AgdaBound{x}\AgdaSpace{}%
\AgdaBound{i}\AgdaSpace{}%
\AgdaSymbol{→}\AgdaSpace{}%
\AgdaOperator{\AgdaFunction{∣}}\AgdaSpace{}%
\AgdaBound{homs}\AgdaSpace{}%
\AgdaBound{i}\AgdaSpace{}%
\AgdaOperator{\AgdaFunction{∣}}\AgdaSpace{}%
\AgdaSymbol{(}\AgdaBound{x}\AgdaSpace{}%
\AgdaBound{i}\AgdaSymbol{)}\<%
\\
%
\\[\AgdaEmptyExtraSkip]%
%
\>[2]\AgdaFunction{ϕhom}\AgdaSpace{}%
\AgdaSymbol{:}\AgdaSpace{}%
\AgdaFunction{is-homomorphism}\AgdaSpace{}%
\AgdaSymbol{(}\AgdaFunction{⨅}\AgdaSpace{}%
\AgdaBound{𝒜}\AgdaSymbol{)}\AgdaSpace{}%
\AgdaSymbol{(}\AgdaFunction{⨅}\AgdaSpace{}%
\AgdaBound{ℬ}\AgdaSymbol{)}\AgdaSpace{}%
\AgdaFunction{ϕ}\<%
\\
%
\>[2]\AgdaFunction{ϕhom}\AgdaSpace{}%
\AgdaBound{𝑓}\AgdaSpace{}%
\AgdaBound{𝒂}\AgdaSpace{}%
\AgdaSymbol{=}\AgdaSpace{}%
\AgdaBound{gfe}\AgdaSpace{}%
\AgdaSymbol{λ}\AgdaSpace{}%
\AgdaBound{i}\AgdaSpace{}%
\AgdaSymbol{→}\AgdaSpace{}%
\AgdaOperator{\AgdaFunction{∥}}\AgdaSpace{}%
\AgdaBound{homs}\AgdaSpace{}%
\AgdaBound{i}\AgdaSpace{}%
\AgdaOperator{\AgdaFunction{∥}}\AgdaSpace{}%
\AgdaBound{𝑓}\AgdaSpace{}%
\AgdaSymbol{(λ}\AgdaSpace{}%
\AgdaBound{x}\AgdaSpace{}%
\AgdaSymbol{→}\AgdaSpace{}%
\AgdaBound{𝒂}\AgdaSpace{}%
\AgdaBound{x}\AgdaSpace{}%
\AgdaBound{i}\AgdaSymbol{)}\<%
\\
\>[0]\<%
\end{code}

\paragraph{Projection homomorphisms}\label{projection-homomorphisms}

Later we will need a proof of the fact that projecting out of a product
algebra onto one of its factors is a homomorphism.

\begin{code}%
\>[0]\<%
\\
\>[0]\AgdaFunction{⨅-projection-hom}\AgdaSpace{}%
\AgdaSymbol{:}%
\>[1138I]\AgdaSymbol{\{}\AgdaBound{𝓤}\AgdaSpace{}%
\AgdaBound{𝓘}\AgdaSpace{}%
\AgdaSymbol{:}\AgdaSpace{}%
\AgdaFunction{Universe}\AgdaSymbol{\}}\<%
\\
\>[.][@{}l@{}]\<[1138I]%
\>[19]\AgdaSymbol{\{}\AgdaBound{I}\AgdaSpace{}%
\AgdaSymbol{:}\AgdaSpace{}%
\AgdaBound{𝓘}\AgdaSpace{}%
\AgdaOperator{\AgdaFunction{̇}}\AgdaSymbol{\}\{}\AgdaBound{𝒜}\AgdaSpace{}%
\AgdaSymbol{:}\AgdaSpace{}%
\AgdaBound{I}\AgdaSpace{}%
\AgdaSymbol{→}\AgdaSpace{}%
\AgdaFunction{Algebra}\AgdaSpace{}%
\AgdaBound{𝓤}\AgdaSpace{}%
\AgdaBound{𝑆}\AgdaSymbol{\}}\<%
\\
%
\>[19]\AgdaComment{--------------------------------}\<%
\\
\>[0][@{}l@{\AgdaIndent{0}}]%
\>[1]\AgdaSymbol{→}%
\>[19]\AgdaSymbol{(}\AgdaBound{i}\AgdaSpace{}%
\AgdaSymbol{:}\AgdaSpace{}%
\AgdaBound{I}\AgdaSymbol{)}\AgdaSpace{}%
\AgdaSymbol{→}\AgdaSpace{}%
\AgdaFunction{hom}\AgdaSpace{}%
\AgdaSymbol{(}\AgdaFunction{⨅}\AgdaSpace{}%
\AgdaBound{𝒜}\AgdaSymbol{)}\AgdaSpace{}%
\AgdaSymbol{(}\AgdaBound{𝒜}\AgdaSpace{}%
\AgdaBound{i}\AgdaSymbol{)}\<%
\\
%
\\[\AgdaEmptyExtraSkip]%
\>[0]\AgdaFunction{⨅-projection-hom}\AgdaSpace{}%
\AgdaSymbol{\{}\AgdaBound{𝓤}\AgdaSymbol{\}\{}\AgdaBound{𝓘}\AgdaSymbol{\}\{}\AgdaBound{I}\AgdaSymbol{\}\{}\AgdaBound{𝒜}\AgdaSymbol{\}}\AgdaSpace{}%
\AgdaBound{i}\AgdaSpace{}%
\AgdaSymbol{=}\AgdaSpace{}%
\AgdaFunction{ϕi}\AgdaSpace{}%
\AgdaOperator{\AgdaInductiveConstructor{,}}\AgdaSpace{}%
\AgdaFunction{ϕihom}\<%
\\
\>[0][@{}l@{\AgdaIndent{0}}]%
\>[1]\AgdaKeyword{where}\<%
\\
\>[1][@{}l@{\AgdaIndent{0}}]%
\>[2]\AgdaFunction{ϕi}\AgdaSpace{}%
\AgdaSymbol{:}\AgdaSpace{}%
\AgdaOperator{\AgdaFunction{∣}}\AgdaSpace{}%
\AgdaFunction{⨅}\AgdaSpace{}%
\AgdaBound{𝒜}\AgdaSpace{}%
\AgdaOperator{\AgdaFunction{∣}}\AgdaSpace{}%
\AgdaSymbol{→}\AgdaSpace{}%
\AgdaOperator{\AgdaFunction{∣}}\AgdaSpace{}%
\AgdaBound{𝒜}\AgdaSpace{}%
\AgdaBound{i}\AgdaSpace{}%
\AgdaOperator{\AgdaFunction{∣}}\<%
\\
%
\>[2]\AgdaFunction{ϕi}\AgdaSpace{}%
\AgdaSymbol{=}\AgdaSpace{}%
\AgdaSymbol{λ}\AgdaSpace{}%
\AgdaBound{x}\AgdaSpace{}%
\AgdaSymbol{→}\AgdaSpace{}%
\AgdaBound{x}\AgdaSpace{}%
\AgdaBound{i}\<%
\\
%
\\[\AgdaEmptyExtraSkip]%
%
\>[2]\AgdaFunction{ϕihom}\AgdaSpace{}%
\AgdaSymbol{:}\AgdaSpace{}%
\AgdaFunction{is-homomorphism}\AgdaSpace{}%
\AgdaSymbol{(}\AgdaFunction{⨅}\AgdaSpace{}%
\AgdaBound{𝒜}\AgdaSymbol{)}\AgdaSpace{}%
\AgdaSymbol{(}\AgdaBound{𝒜}\AgdaSpace{}%
\AgdaBound{i}\AgdaSymbol{)}\AgdaSpace{}%
\AgdaFunction{ϕi}\<%
\\
%
\>[2]\AgdaFunction{ϕihom}\AgdaSpace{}%
\AgdaBound{𝑓}\AgdaSpace{}%
\AgdaBound{𝒂}\AgdaSpace{}%
\AgdaSymbol{=}\AgdaSpace{}%
\AgdaInductiveConstructor{𝓇ℯ𝒻𝓁}\<%
\end{code}
\ccpad
Of course, we could prove a more general result involving projections onto multiple factors, but so far the single-factor result has sufficed.


\subsection{Homomorphism Theorems}\label{sec:homom-theor}\firstsentence{\ualibNoether}{\urlNoether}
% -*- TeX-master: "ualib-part2.tex" -*-
%%% Local Variables: 
%%% mode: latex
%%% TeX-engine: 'xetex
%%% End: 
\subsubsection{The First Homomorphism Theorem}\label{the-first-homomorphism-theorem}

Here is a version of the so-called \emph{First Homomorphism theorem}, sometimes called Noether's First Homomorphism theorem, after Emmy Noether who was among the first proponents of the abstract approach to algebra that we now refer to as ``modern algebra.''
\ccpad
\begin{code}%
\>[0]\AgdaKeyword{open}\AgdaSpace{}%
\AgdaModule{Congruence}\<%
\\
%
\>[0]\AgdaKeyword{module}\AgdaSpace{}%
\AgdaModule{\AgdaUnderscore{}}%
\>[31I]\AgdaSymbol{\{}\AgdaBound{𝓤}\AgdaSpace{}%
\AgdaBound{𝓦}\AgdaSpace{}%
\AgdaSymbol{:}\AgdaSpace{}%
\AgdaFunction{Universe}\AgdaSymbol{\}}\<%
\\
\>[.][@{}l@{}]\<[31I]%
\>[9]\AgdaComment{-- extensionality assumptions --}\<%
\\
\>[9][@{}l@{\AgdaIndent{0}}]%
\>[12]\AgdaSymbol{(}\AgdaBound{fe}\AgdaSpace{}%
\AgdaSymbol{:}\AgdaSpace{}%
\AgdaFunction{dfunext}\AgdaSpace{}%
\AgdaBound{𝓥}\AgdaSpace{}%
\AgdaBound{𝓦}\AgdaSymbol{)}\AgdaSpace{}%
\AgdaSymbol{(}\AgdaBound{pe}\AgdaSpace{}%
\AgdaSymbol{:}\AgdaSpace{}%
\AgdaFunction{prop-ext}\AgdaSpace{}%
\AgdaBound{𝓤}\AgdaSpace{}%
\AgdaBound{𝓦}\AgdaSymbol{)}\<%
\\
%
\\[\AgdaEmptyExtraSkip]%
%
\>[9]\AgdaSymbol{(}\AgdaBound{𝑨}\AgdaSpace{}%
\AgdaSymbol{:}\AgdaSpace{}%
\AgdaFunction{Algebra}\AgdaSpace{}%
\AgdaBound{𝓤}\AgdaSpace{}%
\AgdaBound{𝑆}\AgdaSymbol{)(}\AgdaBound{𝑩}\AgdaSpace{}%
\AgdaSymbol{:}\AgdaSpace{}%
\AgdaFunction{Algebra}\AgdaSpace{}%
\AgdaBound{𝓦}\AgdaSpace{}%
\AgdaBound{𝑆}\AgdaSymbol{)(}\AgdaBound{h}\AgdaSpace{}%
\AgdaSymbol{:}\AgdaSpace{}%
\AgdaFunction{hom}\AgdaSpace{}%
\AgdaBound{𝑨}\AgdaSpace{}%
\AgdaBound{𝑩}\AgdaSymbol{)}\<%
\\
%
\\[\AgdaEmptyExtraSkip]%
%
\>[9]\AgdaComment{-- truncation assumptions --}\<%
\\
\>[9][@{}l@{\AgdaIndent{0}}]%
\>[12]\AgdaSymbol{(}\AgdaBound{Bset}\AgdaSpace{}%
\AgdaSymbol{:}\AgdaSpace{}%
\AgdaFunction{is-set}\AgdaSpace{}%
\AgdaOperator{\AgdaFunction{∣}}\AgdaSpace{}%
\AgdaBound{𝑩}\AgdaSpace{}%
\AgdaOperator{\AgdaFunction{∣}}\AgdaSymbol{)}\<%
\\
%
\>[12]\AgdaSymbol{(}\AgdaBound{ssR}\AgdaSpace{}%
\AgdaSymbol{:}\AgdaSpace{}%
\AgdaSymbol{∀}\AgdaSpace{}%
\AgdaBound{a}\AgdaSpace{}%
\AgdaBound{x}\AgdaSpace{}%
\AgdaSymbol{→}\AgdaSpace{}%
\AgdaFunction{is-subsingleton}\AgdaSpace{}%
\AgdaSymbol{(}\AgdaOperator{\AgdaField{⟨}}\AgdaSpace{}%
\AgdaFunction{kercon}\AgdaSpace{}%
\AgdaBound{𝑩}\AgdaSpace{}%
\AgdaBound{h}\AgdaSpace{}%
\AgdaOperator{\AgdaField{⟩}}\AgdaSpace{}%
\AgdaBound{a}\AgdaSpace{}%
\AgdaBound{x}\AgdaSymbol{))}\<%
\\
%
\>[12]\AgdaSymbol{(}\AgdaBound{ssA}\AgdaSpace{}%
\AgdaSymbol{:}\AgdaSpace{}%
\AgdaSymbol{∀}\AgdaSpace{}%
\AgdaBound{C}\AgdaSpace{}%
\AgdaSymbol{→}\AgdaSpace{}%
\AgdaFunction{is-subsingleton}\AgdaSpace{}%
\AgdaSymbol{(}\AgdaFunction{𝒞}\AgdaSymbol{\{}\AgdaArgument{A}\AgdaSpace{}%
\AgdaSymbol{=}\AgdaSpace{}%
\AgdaOperator{\AgdaFunction{∣}}\AgdaSpace{}%
\AgdaBound{𝑨}\AgdaSpace{}%
\AgdaOperator{\AgdaFunction{∣}}\AgdaSymbol{\}\{}\AgdaOperator{\AgdaField{⟨}}\AgdaSpace{}%
\AgdaFunction{kercon}\AgdaSpace{}%
\AgdaBound{𝑩}\AgdaSpace{}%
\AgdaBound{h}\AgdaSpace{}%
\AgdaOperator{\AgdaField{⟩}}\AgdaSymbol{\}}\AgdaSpace{}%
\AgdaBound{C}\AgdaSymbol{))}\<%
\\
%
\\[\AgdaEmptyExtraSkip]%
\>[0][@{}l@{\AgdaIndent{0}}]%
\>[1]\AgdaKeyword{where}\<%
\\
%
\\[\AgdaEmptyExtraSkip]%
%
\>[1]\AgdaFunction{FirstHomomorphismTheorem}\AgdaSpace{}%
\AgdaSymbol{:}\<%
\\
%
\\[\AgdaEmptyExtraSkip]%
\>[1][@{}l@{\AgdaIndent{0}}]%
\>[2]\AgdaFunction{Σ}\AgdaSpace{}%
\AgdaBound{ϕ}%
\>[90I]\AgdaFunction{꞉}\AgdaSpace{}%
\AgdaFunction{hom}\AgdaSpace{}%
\AgdaSymbol{(}\AgdaBound{𝑨}\AgdaSpace{}%
\AgdaOperator{\AgdaFunction{[}}\AgdaSpace{}%
\AgdaBound{𝑩}\AgdaSpace{}%
\AgdaOperator{\AgdaFunction{]/ker}}\AgdaSpace{}%
\AgdaBound{h}\AgdaSymbol{)}\AgdaSpace{}%
\AgdaBound{𝑩}\AgdaSpace{}%
\AgdaFunction{,}\<%
\\
\>[90I][@{}l@{\AgdaIndent{0}}]%
\>[7]\AgdaSymbol{(}\AgdaOperator{\AgdaFunction{∣}}\AgdaSpace{}%
\AgdaBound{h}\AgdaSpace{}%
\AgdaOperator{\AgdaFunction{∣}}\AgdaSpace{}%
\AgdaOperator{\AgdaDatatype{≡}}\AgdaSpace{}%
\AgdaOperator{\AgdaFunction{∣}}\AgdaSpace{}%
\AgdaBound{ϕ}\AgdaSpace{}%
\AgdaOperator{\AgdaFunction{∣}}\AgdaSpace{}%
\AgdaOperator{\AgdaFunction{∘}}\AgdaSpace{}%
\AgdaOperator{\AgdaFunction{∣}}\AgdaSpace{}%
\AgdaFunction{πker}\AgdaSpace{}%
\AgdaBound{𝑩}\AgdaSpace{}%
\AgdaBound{h}\AgdaSpace{}%
\AgdaOperator{\AgdaFunction{∣}}\AgdaSymbol{)}\AgdaSpace{}%
\AgdaOperator{\AgdaFunction{×}}\AgdaSpace{}%
\AgdaFunction{Monic}\AgdaSpace{}%
\AgdaOperator{\AgdaFunction{∣}}\AgdaSpace{}%
\AgdaBound{ϕ}\AgdaSpace{}%
\AgdaOperator{\AgdaFunction{∣}}\AgdaSpace{}%
\AgdaOperator{\AgdaFunction{×}}\AgdaSpace{}%
\AgdaFunction{is-embedding}\AgdaSpace{}%
\AgdaOperator{\AgdaFunction{∣}}\AgdaSpace{}%
\AgdaBound{ϕ}\AgdaSpace{}%
\AgdaOperator{\AgdaFunction{∣}}\<%
\\
%
\\[\AgdaEmptyExtraSkip]%
%
\>[1]\AgdaFunction{FirstHomomorphismTheorem}\AgdaSpace{}%
\AgdaSymbol{=}\AgdaSpace{}%
\AgdaSymbol{(}\AgdaFunction{ϕ}\AgdaSpace{}%
\AgdaOperator{\AgdaInductiveConstructor{,}}\AgdaSpace{}%
\AgdaFunction{ϕhom}\AgdaSymbol{)}\AgdaSpace{}%
\AgdaOperator{\AgdaInductiveConstructor{,}}\AgdaSpace{}%
\AgdaFunction{ϕcom}\AgdaSpace{}%
\AgdaOperator{\AgdaInductiveConstructor{,}}\AgdaSpace{}%
\AgdaFunction{ϕmon}\AgdaSpace{}%
\AgdaOperator{\AgdaInductiveConstructor{,}}\AgdaSpace{}%
\AgdaFunction{ϕemb}\<%
\\
\>[1][@{}l@{\AgdaIndent{0}}]%
\>[2]\AgdaKeyword{where}\<%
\\
%
\>[2]\AgdaFunction{θ}\AgdaSpace{}%
\AgdaSymbol{:}\AgdaSpace{}%
\AgdaRecord{Congruence}\AgdaSpace{}%
\AgdaBound{𝑨}\<%
\\
%
\>[2]\AgdaFunction{θ}\AgdaSpace{}%
\AgdaSymbol{=}\AgdaSpace{}%
\AgdaFunction{kercon}\AgdaSpace{}%
\AgdaBound{𝑩}\AgdaSpace{}%
\AgdaBound{h}\<%
\\
%
\\[\AgdaEmptyExtraSkip]%
%
\>[2]\AgdaFunction{ϕ}\AgdaSpace{}%
\AgdaSymbol{:}\AgdaSpace{}%
\AgdaOperator{\AgdaFunction{∣}}\AgdaSpace{}%
\AgdaBound{𝑨}\AgdaSpace{}%
\AgdaOperator{\AgdaFunction{[}}\AgdaSpace{}%
\AgdaBound{𝑩}\AgdaSpace{}%
\AgdaOperator{\AgdaFunction{]/ker}}\AgdaSpace{}%
\AgdaBound{h}\AgdaSpace{}%
\AgdaOperator{\AgdaFunction{∣}}\AgdaSpace{}%
\AgdaSymbol{→}\AgdaSpace{}%
\AgdaOperator{\AgdaFunction{∣}}\AgdaSpace{}%
\AgdaBound{𝑩}\AgdaSpace{}%
\AgdaOperator{\AgdaFunction{∣}}\<%
\\
%
\>[2]\AgdaFunction{ϕ}\AgdaSpace{}%
\AgdaBound{a}\AgdaSpace{}%
\AgdaSymbol{=}\AgdaSpace{}%
\AgdaOperator{\AgdaFunction{∣}}\AgdaSpace{}%
\AgdaBound{h}\AgdaSpace{}%
\AgdaOperator{\AgdaFunction{∣}}\AgdaSpace{}%
\AgdaOperator{\AgdaFunction{⌜}}\AgdaSpace{}%
\AgdaBound{a}\AgdaSpace{}%
\AgdaOperator{\AgdaFunction{⌝}}\<%
\\
%
\\[\AgdaEmptyExtraSkip]%
%
\>[2]\AgdaFunction{𝑹}\AgdaSpace{}%
\AgdaSymbol{:}\AgdaSpace{}%
\AgdaFunction{Pred₂}\AgdaSpace{}%
\AgdaOperator{\AgdaFunction{∣}}\AgdaSpace{}%
\AgdaBound{𝑨}\AgdaSpace{}%
\AgdaOperator{\AgdaFunction{∣}}\AgdaSpace{}%
\AgdaBound{𝓦}\<%
\\
%
\>[2]\AgdaFunction{𝑹}\AgdaSpace{}%
\AgdaSymbol{=}\AgdaSpace{}%
\AgdaOperator{\AgdaField{⟨}}\AgdaSpace{}%
\AgdaFunction{kercon}\AgdaSpace{}%
\AgdaBound{𝑩}\AgdaSpace{}%
\AgdaBound{h}\AgdaSpace{}%
\AgdaOperator{\AgdaField{⟩}}\AgdaSpace{}%
\AgdaOperator{\AgdaInductiveConstructor{,}}\AgdaSpace{}%
\AgdaBound{ssR}\<%
\\
%
\\[\AgdaEmptyExtraSkip]%
%
\>[2]\AgdaFunction{ϕhom}\AgdaSpace{}%
\AgdaSymbol{:}\AgdaSpace{}%
\AgdaFunction{is-homomorphism}\AgdaSpace{}%
\AgdaSymbol{(}\AgdaBound{𝑨}\AgdaSpace{}%
\AgdaOperator{\AgdaFunction{[}}\AgdaSpace{}%
\AgdaBound{𝑩}\AgdaSpace{}%
\AgdaOperator{\AgdaFunction{]/ker}}\AgdaSpace{}%
\AgdaBound{h}\AgdaSymbol{)}\AgdaSpace{}%
\AgdaBound{𝑩}\AgdaSpace{}%
\AgdaFunction{ϕ}\<%
\\
%
\>[2]\AgdaFunction{ϕhom}\AgdaSpace{}%
\AgdaBound{𝑓}\AgdaSpace{}%
\AgdaBound{𝒂}%
\>[183I]\AgdaSymbol{=}%
\>[14]\AgdaOperator{\AgdaFunction{∣}}\AgdaSpace{}%
\AgdaBound{h}\AgdaSpace{}%
\AgdaOperator{\AgdaFunction{∣}}\AgdaSpace{}%
\AgdaSymbol{(}\AgdaSpace{}%
\AgdaSymbol{(}\AgdaBound{𝑓}\AgdaSpace{}%
\AgdaOperator{\AgdaFunction{̂}}\AgdaSpace{}%
\AgdaBound{𝑨}\AgdaSymbol{)}\AgdaSpace{}%
\AgdaSymbol{(λ}\AgdaSpace{}%
\AgdaBound{x}\AgdaSpace{}%
\AgdaSymbol{→}\AgdaSpace{}%
\AgdaOperator{\AgdaFunction{⌜}}\AgdaSpace{}%
\AgdaBound{𝒂}\AgdaSpace{}%
\AgdaBound{x}\AgdaSpace{}%
\AgdaOperator{\AgdaFunction{⌝}}\AgdaSymbol{)}\AgdaSpace{}%
\AgdaSymbol{)}\AgdaSpace{}%
\AgdaOperator{\AgdaFunction{≡⟨}}\AgdaSpace{}%
\AgdaOperator{\AgdaFunction{∥}}\AgdaSpace{}%
\AgdaBound{h}\AgdaSpace{}%
\AgdaOperator{\AgdaFunction{∥}}\AgdaSpace{}%
\AgdaBound{𝑓}\AgdaSpace{}%
\AgdaSymbol{(λ}\AgdaSpace{}%
\AgdaBound{x}\AgdaSpace{}%
\AgdaSymbol{→}\AgdaSpace{}%
\AgdaOperator{\AgdaFunction{⌜}}\AgdaSpace{}%
\AgdaBound{𝒂}\AgdaSpace{}%
\AgdaBound{x}\AgdaSpace{}%
\AgdaOperator{\AgdaFunction{⌝}}\AgdaSymbol{)}%
\>[76]\AgdaOperator{\AgdaFunction{⟩}}\<%
\\
\>[183I][@{}l@{\AgdaIndent{0}}]%
\>[13]\AgdaSymbol{(}\AgdaBound{𝑓}\AgdaSpace{}%
\AgdaOperator{\AgdaFunction{̂}}\AgdaSpace{}%
\AgdaBound{𝑩}\AgdaSymbol{)}\AgdaSpace{}%
\AgdaSymbol{(}\AgdaOperator{\AgdaFunction{∣}}\AgdaSpace{}%
\AgdaBound{h}\AgdaSpace{}%
\AgdaOperator{\AgdaFunction{∣}}\AgdaSpace{}%
\AgdaOperator{\AgdaFunction{∘}}\AgdaSpace{}%
\AgdaSymbol{(λ}\AgdaSpace{}%
\AgdaBound{x}\AgdaSpace{}%
\AgdaSymbol{→}\AgdaSpace{}%
\AgdaOperator{\AgdaFunction{⌜}}\AgdaSpace{}%
\AgdaBound{𝒂}\AgdaSpace{}%
\AgdaBound{x}\AgdaSpace{}%
\AgdaOperator{\AgdaFunction{⌝}}\AgdaSymbol{))}\AgdaSpace{}%
\AgdaOperator{\AgdaFunction{≡⟨}}\AgdaSpace{}%
\AgdaFunction{ap}\AgdaSpace{}%
\AgdaSymbol{(}\AgdaBound{𝑓}\AgdaSpace{}%
\AgdaOperator{\AgdaFunction{̂}}\AgdaSpace{}%
\AgdaBound{𝑩}\AgdaSymbol{)}\AgdaSpace{}%
\AgdaSymbol{(}\AgdaBound{fe}\AgdaSpace{}%
\AgdaSymbol{λ}\AgdaSpace{}%
\AgdaBound{x}\AgdaSpace{}%
\AgdaSymbol{→}\AgdaSpace{}%
\AgdaInductiveConstructor{𝓇ℯ𝒻𝓁}\AgdaSymbol{)}\AgdaSpace{}%
\AgdaOperator{\AgdaFunction{⟩}}\<%
\\
%
\>[13]\AgdaSymbol{(}\AgdaBound{𝑓}\AgdaSpace{}%
\AgdaOperator{\AgdaFunction{̂}}\AgdaSpace{}%
\AgdaBound{𝑩}\AgdaSymbol{)}\AgdaSpace{}%
\AgdaSymbol{(λ}\AgdaSpace{}%
\AgdaBound{x}\AgdaSpace{}%
\AgdaSymbol{→}\AgdaSpace{}%
\AgdaFunction{ϕ}\AgdaSpace{}%
\AgdaSymbol{(}\AgdaBound{𝒂}\AgdaSpace{}%
\AgdaBound{x}\AgdaSymbol{))}%
\>[49]\AgdaOperator{\AgdaFunction{∎}}\<%
\\
%
\\[\AgdaEmptyExtraSkip]%
%
\>[2]\AgdaFunction{ϕmon}\AgdaSpace{}%
\AgdaSymbol{:}\AgdaSpace{}%
\AgdaFunction{Monic}\AgdaSpace{}%
\AgdaFunction{ϕ}\<%
\\
%
\>[2]\AgdaFunction{ϕmon}\AgdaSpace{}%
\AgdaSymbol{(}\AgdaDottedPattern{\AgdaSymbol{.(}}\AgdaDottedPattern{\AgdaOperator{\AgdaField{⟨}}}\AgdaSpace{}%
\AgdaDottedPattern{\AgdaFunction{θ}}\AgdaSpace{}%
\AgdaDottedPattern{\AgdaOperator{\AgdaField{⟩}}}\AgdaSpace{}%
\AgdaDottedPattern{\AgdaBound{u}}\AgdaDottedPattern{\AgdaSymbol{)}}\AgdaSpace{}%
\AgdaOperator{\AgdaInductiveConstructor{,}}\AgdaSpace{}%
\AgdaBound{u}\AgdaSpace{}%
\AgdaOperator{\AgdaInductiveConstructor{,}}\AgdaSpace{}%
\AgdaInductiveConstructor{refl}\AgdaSpace{}%
\AgdaSymbol{\AgdaUnderscore{})}\AgdaSpace{}%
\AgdaSymbol{(}\AgdaDottedPattern{\AgdaSymbol{.(}}\AgdaDottedPattern{\AgdaOperator{\AgdaField{⟨}}}\AgdaSpace{}%
\AgdaDottedPattern{\AgdaFunction{θ}}\AgdaSpace{}%
\AgdaDottedPattern{\AgdaOperator{\AgdaField{⟩}}}\AgdaSpace{}%
\AgdaDottedPattern{\AgdaBound{v}}\AgdaDottedPattern{\AgdaSymbol{)}}\AgdaSpace{}%
\AgdaOperator{\AgdaInductiveConstructor{,}}\AgdaSpace{}%
\AgdaBound{v}\AgdaSpace{}%
\AgdaOperator{\AgdaInductiveConstructor{,}}\AgdaSpace{}%
\AgdaInductiveConstructor{refl}\AgdaSpace{}%
\AgdaSymbol{\AgdaUnderscore{})}\AgdaSpace{}%
\AgdaBound{ϕuv}\AgdaSpace{}%
\AgdaSymbol{=}\<%
\\
\>[2][@{}l@{\AgdaIndent{0}}]%
\>[3]\AgdaFunction{class-extensionality'}\AgdaSpace{}%
\AgdaSymbol{\{}\AgdaArgument{𝑹}\AgdaSpace{}%
\AgdaSymbol{=}\AgdaSpace{}%
\AgdaFunction{𝑹}\AgdaSymbol{\}}\AgdaSpace{}%
\AgdaBound{pe}\AgdaSpace{}%
\AgdaBound{ssA}\AgdaSpace{}%
\AgdaSymbol{(}\AgdaField{IsEquiv}\AgdaSpace{}%
\AgdaFunction{θ}\AgdaSymbol{)}\AgdaSpace{}%
\AgdaBound{ϕuv}\<%
\\
%
\\[\AgdaEmptyExtraSkip]%
%
\>[2]\AgdaFunction{ϕcom}\AgdaSpace{}%
\AgdaSymbol{:}\AgdaSpace{}%
\AgdaOperator{\AgdaFunction{∣}}\AgdaSpace{}%
\AgdaBound{h}\AgdaSpace{}%
\AgdaOperator{\AgdaFunction{∣}}\AgdaSpace{}%
\AgdaOperator{\AgdaDatatype{≡}}\AgdaSpace{}%
\AgdaFunction{ϕ}\AgdaSpace{}%
\AgdaOperator{\AgdaFunction{∘}}\AgdaSpace{}%
\AgdaOperator{\AgdaFunction{∣}}\AgdaSpace{}%
\AgdaFunction{πker}\AgdaSpace{}%
\AgdaBound{𝑩}\AgdaSpace{}%
\AgdaBound{h}\AgdaSpace{}%
\AgdaOperator{\AgdaFunction{∣}}\<%
\\
%
\>[2]\AgdaFunction{ϕcom}\AgdaSpace{}%
\AgdaSymbol{=}\AgdaSpace{}%
\AgdaInductiveConstructor{𝓇ℯ𝒻𝓁}\<%
\\
%
\\[\AgdaEmptyExtraSkip]%
%
\>[2]\AgdaFunction{ϕemb}\AgdaSpace{}%
\AgdaSymbol{:}\AgdaSpace{}%
\AgdaFunction{is-embedding}\AgdaSpace{}%
\AgdaFunction{ϕ}\<%
\\
%
\>[2]\AgdaFunction{ϕemb}\AgdaSpace{}%
\AgdaSymbol{=}\AgdaSpace{}%
\AgdaFunction{monic-is-embedding|sets}\AgdaSpace{}%
\AgdaFunction{ϕ}\AgdaSpace{}%
\AgdaBound{Bset}\AgdaSpace{}%
\AgdaFunction{ϕmon}\<%
\end{code}
\ccpad
Next we show that the homomorphism \ab{ϕ}, whose existence we just proved, is unique.
\ccpad
\begin{code}%
\>[0][@{}l@{\AgdaIndent{1}}]%
\>[1]\AgdaFunction{NoetherHomUnique}\AgdaSpace{}%
\AgdaSymbol{:}\AgdaSpace{}%
\>[234I]\AgdaSymbol{(}\AgdaBound{f}\AgdaSpace{}%
\AgdaBound{g}\AgdaSpace{}%
\AgdaSymbol{:}\AgdaSpace{}%
\AgdaFunction{hom}\AgdaSpace{}%
\AgdaSymbol{(}\AgdaBound{𝑨}\AgdaSpace{}%
\AgdaOperator{\AgdaFunction{[}}\AgdaSpace{}%
\AgdaBound{𝑩}\AgdaSpace{}%
\AgdaOperator{\AgdaFunction{]/ker}}\AgdaSpace{}%
\AgdaBound{h}\AgdaSymbol{)}\AgdaSpace{}%
\AgdaBound{𝑩}\AgdaSymbol{)}\<%
\\
\>[1][@{}l@{\AgdaIndent{0}}]%
\>[2]\AgdaSymbol{→}%
\>[.][@{}l@{}]\<[234I]%
\>[20]\AgdaOperator{\AgdaFunction{∣}}\AgdaSpace{}%
\AgdaBound{h}\AgdaSpace{}%
\AgdaOperator{\AgdaFunction{∣}}\AgdaSpace{}%
\AgdaOperator{\AgdaDatatype{≡}}\AgdaSpace{}%
\AgdaOperator{\AgdaFunction{∣}}\AgdaSpace{}%
\AgdaBound{f}\AgdaSpace{}%
\AgdaOperator{\AgdaFunction{∣}}\AgdaSpace{}%
\AgdaOperator{\AgdaFunction{∘}}\AgdaSpace{}%
\AgdaOperator{\AgdaFunction{∣}}\AgdaSpace{}%
\AgdaFunction{πker}\AgdaSpace{}%
\AgdaBound{𝑩}\AgdaSpace{}%
\AgdaBound{h}\AgdaSpace{}%
\AgdaOperator{\AgdaFunction{∣}}\AgdaSpace{}%
\AgdaSymbol{→}\AgdaSpace{}%
\AgdaOperator{\AgdaFunction{∣}}\AgdaSpace{}%
\AgdaBound{h}\AgdaSpace{}%
\AgdaOperator{\AgdaFunction{∣}}\AgdaSpace{}%
\AgdaOperator{\AgdaDatatype{≡}}\AgdaSpace{}%
\AgdaOperator{\AgdaFunction{∣}}\AgdaSpace{}%
\AgdaBound{g}\AgdaSpace{}%
\AgdaOperator{\AgdaFunction{∣}}\AgdaSpace{}%
\AgdaOperator{\AgdaFunction{∘}}\AgdaSpace{}%
\AgdaOperator{\AgdaFunction{∣}}\AgdaSpace{}%
\AgdaFunction{πker}\AgdaSpace{}%
\AgdaBound{𝑩}\AgdaSpace{}%
\AgdaBound{h}\AgdaSpace{}%
\AgdaOperator{\AgdaFunction{∣}}\<%
\\
%
% \>[20]\AgdaComment{--------------------------------------------------------------------}\<%
% \\
%
\>[2]\AgdaSymbol{→}%
\>[20]\AgdaSymbol{∀}\AgdaSpace{}%
\AgdaBound{a}%
\>[25]\AgdaSymbol{→}%
\>[28]\AgdaOperator{\AgdaFunction{∣}}\AgdaSpace{}%
\AgdaBound{f}\AgdaSpace{}%
\AgdaOperator{\AgdaFunction{∣}}\AgdaSpace{}%
\AgdaBound{a}\AgdaSpace{}%
\AgdaOperator{\AgdaDatatype{≡}}\AgdaSpace{}%
\AgdaOperator{\AgdaFunction{∣}}\AgdaSpace{}%
\AgdaBound{g}\AgdaSpace{}%
\AgdaOperator{\AgdaFunction{∣}}\AgdaSpace{}%
\AgdaBound{a}\<%
\\
%
\\[\AgdaEmptyExtraSkip]%
%
\>[1]\AgdaFunction{NoetherHomUnique}\AgdaSpace{}%
\AgdaBound{f}\AgdaSpace{}%
\AgdaBound{g}\AgdaSpace{}%
\AgdaBound{hfk}\AgdaSpace{}%
\AgdaBound{hgk}\AgdaSpace{}%
\AgdaSymbol{(}\AgdaDottedPattern{\AgdaSymbol{.(}}\AgdaDottedPattern{\AgdaOperator{\AgdaField{⟨}}}\AgdaSpace{}%
\AgdaDottedPattern{\AgdaFunction{kercon}}\AgdaSpace{}%
\AgdaDottedPattern{\AgdaBound{𝑩}}\AgdaSpace{}%
\AgdaDottedPattern{\AgdaBound{h}}\AgdaSpace{}%
\AgdaDottedPattern{\AgdaOperator{\AgdaField{⟩}}}\AgdaSpace{}%
\AgdaDottedPattern{\AgdaBound{a}}\AgdaDottedPattern{\AgdaSymbol{)}}\AgdaSpace{}%
\AgdaOperator{\AgdaInductiveConstructor{,}}\AgdaSpace{}%
\AgdaBound{a}\AgdaSpace{}%
\AgdaOperator{\AgdaInductiveConstructor{,}}\AgdaSpace{}%
\AgdaInductiveConstructor{𝓇ℯ𝒻𝓁}\AgdaSymbol{)}\AgdaSpace{}%
\AgdaSymbol{=}\<%
\\
%
\\[\AgdaEmptyExtraSkip]%
\>[1][@{}l@{\AgdaIndent{0}}]%
\>[2]\AgdaKeyword{let}\AgdaSpace{}%
\AgdaBound{θ}\AgdaSpace{}%
\AgdaSymbol{=}\AgdaSpace{}%
\AgdaSymbol{(}\AgdaOperator{\AgdaField{⟨}}\AgdaSpace{}%
\AgdaFunction{kercon}\AgdaSpace{}%
\AgdaBound{𝑩}\AgdaSpace{}%
\AgdaBound{h}\AgdaSpace{}%
\AgdaOperator{\AgdaField{⟩}}\AgdaSpace{}%
\AgdaBound{a}\AgdaSpace{}%
\AgdaOperator{\AgdaInductiveConstructor{,}}\AgdaSpace{}%
\AgdaBound{a}\AgdaSpace{}%
\AgdaOperator{\AgdaInductiveConstructor{,}}\AgdaSpace{}%
\AgdaInductiveConstructor{𝓇ℯ𝒻𝓁}\AgdaSymbol{)}\AgdaSpace{}%
\AgdaKeyword{in}\<%
\\
%
\\[\AgdaEmptyExtraSkip]%
\>[2][@{}l@{\AgdaIndent{0}}]%
\>[3]\AgdaOperator{\AgdaFunction{∣}}\AgdaSpace{}%
\AgdaBound{f}\AgdaSpace{}%
\AgdaOperator{\AgdaFunction{∣}}\AgdaSpace{}%
\AgdaBound{θ}%
\>[13]\AgdaOperator{\AgdaFunction{≡⟨}}\AgdaSpace{}%
\AgdaFunction{cong-app}\AgdaSpace{}%
\AgdaSymbol{(}\AgdaBound{hfk}\AgdaSpace{}%
\AgdaOperator{\AgdaFunction{⁻¹}}\AgdaSymbol{)}\AgdaSpace{}%
\AgdaBound{a}\AgdaSpace{}%
\AgdaOperator{\AgdaFunction{⟩}}%
\>[37]\AgdaOperator{\AgdaFunction{∣}}\AgdaSpace{}%
\AgdaBound{h}\AgdaSpace{}%
\AgdaOperator{\AgdaFunction{∣}}\AgdaSpace{}%
\AgdaBound{a}%
\>[47]\AgdaOperator{\AgdaFunction{≡⟨}}\AgdaSpace{}%
\AgdaFunction{cong-app}\AgdaSpace{}%
\AgdaSymbol{(}\AgdaBound{hgk}\AgdaSymbol{)}\AgdaSpace{}%
\AgdaBound{a}\AgdaSpace{}%
\AgdaOperator{\AgdaFunction{⟩}}%
\>[69]\AgdaOperator{\AgdaFunction{∣}}\AgdaSpace{}%
\AgdaBound{g}\AgdaSpace{}%
\AgdaOperator{\AgdaFunction{∣}}\AgdaSpace{}%
\AgdaBound{θ}%
\>[79]\AgdaOperator{\AgdaFunction{∎}}\<%
\end{code}
\ccpad
If we postulate function extensionality, then we have\footnote{%
We already assumed \emph{global} function extensionality in this module, so we could just appeal to that in this case.  However, we make a local function extensionality assumption explicit here merely to highlight where and how the principle is applied.}
\ccpad
\begin{code}%
\>[0][@{}l@{\AgdaIndent{1}}]%
\>[1]\AgdaFunction{fe-NoetherHomUnique}\AgdaSpace{}%
\AgdaSymbol{:}\AgdaSpace{}%
\>[123I]\AgdaFunction{funext}\AgdaSpace{}%
\AgdaSymbol{(}\AgdaBound{𝓤}\AgdaSpace{}%
\AgdaOperator{\AgdaFunction{⊔}}\AgdaSpace{}%
\AgdaBound{𝓦}\AgdaSpace{}%
\AgdaOperator{\AgdaFunction{⁺}}\AgdaSymbol{)}\AgdaSpace{}%
\AgdaBound{𝓦}\AgdaSpace{}%
\AgdaSymbol{→}\AgdaSpace{}%
\AgdaSymbol{(}\AgdaBound{f}\AgdaSpace{}%
\AgdaBound{g}\AgdaSpace{}%
\AgdaSymbol{:}\AgdaSpace{}%
\AgdaFunction{hom}\AgdaSpace{}%
\AgdaSymbol{(}\AgdaBound{𝑨}\AgdaSpace{}%
\AgdaOperator{\AgdaFunction{[}}\AgdaSpace{}%
\AgdaBound{𝑩}\AgdaSpace{}%
\AgdaOperator{\AgdaFunction{]/ker}}\AgdaSpace{}%
\AgdaBound{h}\AgdaSymbol{)}\AgdaSpace{}%
\AgdaBound{𝑩}\AgdaSymbol{)}\<%
\\
\>[1][@{}l@{\AgdaIndent{0}}]%
\>[2]\AgdaSymbol{→}%
\>[.][@{}l@{}]\<[123I]%
\>[23]\AgdaOperator{\AgdaFunction{∣}}\AgdaSpace{}%
\AgdaBound{h}\AgdaSpace{}%
\AgdaOperator{\AgdaFunction{∣}}\AgdaSpace{}%
\AgdaOperator{\AgdaDatatype{≡}}\AgdaSpace{}%
\AgdaOperator{\AgdaFunction{∣}}\AgdaSpace{}%
\AgdaBound{f}\AgdaSpace{}%
\AgdaOperator{\AgdaFunction{∣}}\AgdaSpace{}%
\AgdaOperator{\AgdaFunction{∘}}\AgdaSpace{}%
\AgdaOperator{\AgdaFunction{∣}}\AgdaSpace{}%
\AgdaFunction{πker}\AgdaSpace{}%
\AgdaBound{𝑩}\AgdaSpace{}%
\AgdaBound{h}\AgdaSpace{}%
\AgdaOperator{\AgdaFunction{∣}}\AgdaSpace{}%
\AgdaSymbol{→}\AgdaSpace{}%
\AgdaOperator{\AgdaFunction{∣}}\AgdaSpace{}%
\AgdaBound{h}\AgdaSpace{}%
\AgdaOperator{\AgdaFunction{∣}}\AgdaSpace{}%
\AgdaOperator{\AgdaDatatype{≡}}\AgdaSpace{}%
\AgdaOperator{\AgdaFunction{∣}}\AgdaSpace{}%
\AgdaBound{g}\AgdaSpace{}%
\AgdaOperator{\AgdaFunction{∣}}\AgdaSpace{}%
\AgdaOperator{\AgdaFunction{∘}}\AgdaSpace{}%
\AgdaOperator{\AgdaFunction{∣}}\AgdaSpace{}%
\AgdaFunction{πker}\AgdaSpace{}%
\AgdaBound{𝑩}\AgdaSpace{}%
\AgdaBound{h}\AgdaSpace{}%
\AgdaOperator{\AgdaFunction{∣}}\<%
\\
% %
% \>[23]\AgdaComment{---------------------------------------------------------------------}\<%
% \\
%
\>[2]\AgdaSymbol{→}%
\>[23]\AgdaOperator{\AgdaFunction{∣}}\AgdaSpace{}%
\AgdaBound{f}\AgdaSpace{}%
\AgdaOperator{\AgdaFunction{∣}}\AgdaSpace{}%
\AgdaOperator{\AgdaDatatype{≡}}\AgdaSpace{}%
\AgdaOperator{\AgdaFunction{∣}}\AgdaSpace{}%
\AgdaBound{g}\AgdaSpace{}%
\AgdaOperator{\AgdaFunction{∣}}\<%
\\
%
\\[\AgdaEmptyExtraSkip]%
%
\>[1]\AgdaFunction{fe-NoetherHomUnique}\AgdaSpace{}%
\AgdaBound{fe}\AgdaSpace{}%
\AgdaBound{f}\AgdaSpace{}%
\AgdaBound{g}\AgdaSpace{}%
\AgdaBound{hfk}\AgdaSpace{}%
\AgdaBound{hgk}\AgdaSpace{}%
\AgdaSymbol{=}\AgdaSpace{}%
\AgdaBound{fe}\AgdaSpace{}%
\AgdaSymbol{(}\AgdaFunction{NoetherHomUnique}\AgdaSpace{}%
\AgdaBound{f}\AgdaSpace{}%
\AgdaBound{g}\AgdaSpace{}%
\AgdaBound{hfk}\AgdaSpace{}%
\AgdaBound{hgk}\AgdaSymbol{)}\<%
\end{code}
\ccpad
Assuming the hypotheses of the First Homomorphism theorem, if we add the assumption that \ab{h} is epic, then we get the so-called \defn{First Isomorphism theorem}.
\ccpad
\begin{code}%
\>[0][@{}l@{\AgdaIndent{1}}]%
\>[1]\AgdaFunction{FirstIsomorphismTheorem}\AgdaSpace{}%
\AgdaSymbol{:}\<%
\\
%
\\[\AgdaEmptyExtraSkip]%
\>[1][@{}l@{\AgdaIndent{0}}]%
\>[2]\AgdaFunction{Epic}\AgdaSpace{}%
\AgdaOperator{\AgdaFunction{∣}}\AgdaSpace{}%
\AgdaBound{h}\AgdaSpace{}%
\AgdaOperator{\AgdaFunction{∣}}\AgdaSpace{}%
\AgdaSymbol{→}\AgdaSpace{}%
\AgdaFunction{Σ}\AgdaSpace{}%
\AgdaBound{f}\AgdaSpace{}%
\AgdaFunction{꞉}\AgdaSpace{}%
\AgdaSymbol{(}\AgdaFunction{epi}\AgdaSpace{}%
\AgdaSymbol{(}\AgdaBound{𝑨}\AgdaSpace{}%
\AgdaOperator{\AgdaFunction{[}}\AgdaSpace{}%
\AgdaBound{𝑩}\AgdaSpace{}%
\AgdaOperator{\AgdaFunction{]/ker}}\AgdaSpace{}%
\AgdaBound{h}\AgdaSymbol{)}\AgdaSpace{}%
\AgdaBound{𝑩}\AgdaSymbol{)}\AgdaSpace{}%
\AgdaFunction{,}\AgdaSpace{}%
\AgdaSymbol{(}\AgdaOperator{\AgdaFunction{∣}}\AgdaSpace{}%
\AgdaBound{h}\AgdaSpace{}%
\AgdaOperator{\AgdaFunction{∣}}\AgdaSpace{}%
\AgdaOperator{\AgdaDatatype{≡}}\AgdaSpace{}%
\AgdaOperator{\AgdaFunction{∣}}\AgdaSpace{}%
\AgdaBound{f}\AgdaSpace{}%
\AgdaOperator{\AgdaFunction{∣}}\AgdaSpace{}%
\AgdaOperator{\AgdaFunction{∘}}\AgdaSpace{}%
\AgdaOperator{\AgdaFunction{∣}}\AgdaSpace{}%
\AgdaFunction{πker}\AgdaSpace{}%
\AgdaBound{𝑩}\AgdaSpace{}%
\AgdaBound{h}\AgdaSpace{}%
\AgdaOperator{\AgdaFunction{∣}}\AgdaSymbol{)}\AgdaSpace{}%
\AgdaOperator{\AgdaFunction{×}}\AgdaSpace{}%
\AgdaFunction{is-embedding}\AgdaSpace{}%
\AgdaOperator{\AgdaFunction{∣}}\AgdaSpace{}%
\AgdaBound{f}\AgdaSpace{}%
\AgdaOperator{\AgdaFunction{∣}}\<%
\\
%
\\[\AgdaEmptyExtraSkip]%
%
\>[1]\AgdaFunction{FirstIsomorphismTheorem}\AgdaSpace{}%
\AgdaBound{hE}\AgdaSpace{}%
\AgdaSymbol{=}\AgdaSpace{}%
\AgdaSymbol{(}\AgdaFunction{fmap}\AgdaSpace{}%
\AgdaOperator{\AgdaInductiveConstructor{,}}\AgdaSpace{}%
\AgdaFunction{fhom}\AgdaSpace{}%
\AgdaOperator{\AgdaInductiveConstructor{,}}\AgdaSpace{}%
\AgdaFunction{fepic}\AgdaSymbol{)}\AgdaSpace{}%
\AgdaOperator{\AgdaInductiveConstructor{,}}\AgdaSpace{}%
\AgdaInductiveConstructor{𝓇ℯ𝒻𝓁}\AgdaSpace{}%
\AgdaOperator{\AgdaInductiveConstructor{,}}\AgdaSpace{}%
\AgdaFunction{femb}\<%
\\
\>[1][@{}l@{\AgdaIndent{0}}]%
\>[2]\AgdaKeyword{where}\<%
\\
%
\>[2]\AgdaFunction{θ}\AgdaSpace{}%
\AgdaSymbol{:}\AgdaSpace{}%
\AgdaRecord{Congruence}\AgdaSpace{}%
\AgdaBound{𝑨}\<%
\\
%
\>[2]\AgdaFunction{θ}\AgdaSpace{}%
\AgdaSymbol{=}\AgdaSpace{}%
\AgdaFunction{kercon}\AgdaSpace{}%
\AgdaBound{𝑩}\AgdaSpace{}%
\AgdaBound{h}\<%
\\
%
\\[\AgdaEmptyExtraSkip]%
%
\>[2]\AgdaFunction{fmap}\AgdaSpace{}%
\AgdaSymbol{:}\AgdaSpace{}%
\AgdaOperator{\AgdaFunction{∣}}\AgdaSpace{}%
\AgdaBound{𝑨}\AgdaSpace{}%
\AgdaOperator{\AgdaFunction{[}}\AgdaSpace{}%
\AgdaBound{𝑩}\AgdaSpace{}%
\AgdaOperator{\AgdaFunction{]/ker}}\AgdaSpace{}%
\AgdaBound{h}\AgdaSpace{}%
\AgdaOperator{\AgdaFunction{∣}}\AgdaSpace{}%
\AgdaSymbol{→}\AgdaSpace{}%
\AgdaOperator{\AgdaFunction{∣}}\AgdaSpace{}%
\AgdaBound{𝑩}\AgdaSpace{}%
\AgdaOperator{\AgdaFunction{∣}}\<%
\\
%
\>[2]\AgdaFunction{fmap}\AgdaSpace{}%
\AgdaBound{⟦a⟧}\AgdaSpace{}%
\AgdaSymbol{=}\AgdaSpace{}%
\AgdaOperator{\AgdaFunction{∣}}\AgdaSpace{}%
\AgdaBound{h}\AgdaSpace{}%
\AgdaOperator{\AgdaFunction{∣}}\AgdaSpace{}%
\AgdaOperator{\AgdaFunction{⌜}}\AgdaSpace{}%
\AgdaBound{⟦a⟧}\AgdaSpace{}%
\AgdaOperator{\AgdaFunction{⌝}}\<%
\\
%
\\[\AgdaEmptyExtraSkip]%
%
\>[2]\AgdaFunction{fhom}\AgdaSpace{}%
\AgdaSymbol{:}\AgdaSpace{}%
\AgdaFunction{is-homomorphism}\AgdaSpace{}%
\AgdaSymbol{(}\AgdaBound{𝑨}\AgdaSpace{}%
\AgdaOperator{\AgdaFunction{[}}\AgdaSpace{}%
\AgdaBound{𝑩}\AgdaSpace{}%
\AgdaOperator{\AgdaFunction{]/ker}}\AgdaSpace{}%
\AgdaBound{h}\AgdaSymbol{)}\AgdaSpace{}%
\AgdaBound{𝑩}\AgdaSpace{}%
\AgdaFunction{fmap}\<%
\\
%
\>[2]\AgdaFunction{fhom}\AgdaSpace{}%
\AgdaBound{𝑓}\AgdaSpace{}%
\AgdaBound{a}\AgdaSpace{}%
\AgdaSymbol{=}%
\>[14]\AgdaOperator{\AgdaFunction{∣}}\AgdaSpace{}%
\AgdaBound{h}\AgdaSpace{}%
\AgdaOperator{\AgdaFunction{∣}}\AgdaSymbol{((}\AgdaBound{𝑓}\AgdaSpace{}%
\AgdaOperator{\AgdaFunction{̂}}\AgdaSpace{}%
\AgdaBound{𝑨}\AgdaSymbol{)}\AgdaSpace{}%
\AgdaSymbol{λ}\AgdaSpace{}%
\AgdaBound{x}\AgdaSpace{}%
\AgdaSymbol{→}\AgdaSpace{}%
\AgdaOperator{\AgdaFunction{⌜}}\AgdaSpace{}%
\AgdaBound{a}\AgdaSpace{}%
\AgdaBound{x}\AgdaSpace{}%
\AgdaOperator{\AgdaFunction{⌝}}\AgdaSymbol{)}%
\>[45]\AgdaOperator{\AgdaFunction{≡⟨}}\AgdaSpace{}%
\AgdaOperator{\AgdaFunction{∥}}\AgdaSpace{}%
\AgdaBound{h}\AgdaSpace{}%
\AgdaOperator{\AgdaFunction{∥}}\AgdaSpace{}%
\AgdaBound{𝑓}\AgdaSpace{}%
\AgdaSymbol{(λ}\AgdaSpace{}%
\AgdaBound{x}\AgdaSpace{}%
\AgdaSymbol{→}\AgdaSpace{}%
\AgdaOperator{\AgdaFunction{⌜}}\AgdaSpace{}%
\AgdaBound{a}\AgdaSpace{}%
\AgdaBound{x}\AgdaSpace{}%
\AgdaOperator{\AgdaFunction{⌝}}\AgdaSymbol{)}%
\>[73]\AgdaOperator{\AgdaFunction{⟩}}\<%
\\
%
\>[14]\AgdaSymbol{(}\AgdaBound{𝑓}\AgdaSpace{}%
\AgdaOperator{\AgdaFunction{̂}}\AgdaSpace{}%
\AgdaBound{𝑩}\AgdaSymbol{)(}\AgdaOperator{\AgdaFunction{∣}}\AgdaSpace{}%
\AgdaBound{h}\AgdaSpace{}%
\AgdaOperator{\AgdaFunction{∣}}\AgdaSpace{}%
\AgdaOperator{\AgdaFunction{∘}}\AgdaSpace{}%
\AgdaSymbol{λ}\AgdaSpace{}%
\AgdaBound{x}\AgdaSpace{}%
\AgdaSymbol{→}\AgdaSpace{}%
\AgdaOperator{\AgdaFunction{⌜}}\AgdaSpace{}%
\AgdaBound{a}\AgdaSpace{}%
\AgdaBound{x}\AgdaSpace{}%
\AgdaOperator{\AgdaFunction{⌝}}\AgdaSymbol{)}\AgdaSpace{}%
\AgdaOperator{\AgdaFunction{≡⟨}}\AgdaSpace{}%
\AgdaFunction{ap}\AgdaSymbol{(}\AgdaBound{𝑓}\AgdaSpace{}%
\AgdaOperator{\AgdaFunction{̂}}\AgdaSpace{}%
\AgdaBound{𝑩}\AgdaSymbol{)(}\AgdaBound{gfe}\AgdaSpace{}%
\AgdaSymbol{λ}\AgdaSpace{}%
\AgdaBound{\AgdaUnderscore{}}\AgdaSpace{}%
\AgdaSymbol{→}\AgdaSpace{}%
\AgdaInductiveConstructor{𝓇ℯ𝒻𝓁}\AgdaSymbol{)}\AgdaOperator{\AgdaFunction{⟩}}\<%
\\
%
\>[14]\AgdaSymbol{(}\AgdaBound{𝑓}\AgdaSpace{}%
\AgdaOperator{\AgdaFunction{̂}}\AgdaSpace{}%
\AgdaBound{𝑩}\AgdaSymbol{)}\AgdaSpace{}%
\AgdaSymbol{(}\AgdaFunction{fmap}\AgdaSpace{}%
\AgdaOperator{\AgdaFunction{∘}}\AgdaSpace{}%
\AgdaBound{a}\AgdaSymbol{)}%
\>[46]\AgdaOperator{\AgdaFunction{∎}}\<%
\\
%
\\[\AgdaEmptyExtraSkip]%
%
\>[2]\AgdaFunction{fepic}\AgdaSpace{}%
\AgdaSymbol{:}\AgdaSpace{}%
\AgdaFunction{Epic}\AgdaSpace{}%
\AgdaFunction{fmap}\<%
\\
%
\>[2]\AgdaFunction{fepic}\AgdaSpace{}%
\AgdaBound{b}\AgdaSpace{}%
\AgdaSymbol{=}\AgdaSpace{}%
\AgdaFunction{γ}\AgdaSpace{}%
\AgdaKeyword{where}\<%
\\
\>[2][@{}l@{\AgdaIndent{0}}]%
\>[3]\AgdaFunction{a}\AgdaSpace{}%
\AgdaSymbol{:}\AgdaSpace{}%
\AgdaOperator{\AgdaFunction{∣}}\AgdaSpace{}%
\AgdaBound{𝑨}\AgdaSpace{}%
\AgdaOperator{\AgdaFunction{∣}}\<%
\\
%
\>[3]\AgdaFunction{a}\AgdaSpace{}%
\AgdaSymbol{=}\AgdaSpace{}%
\AgdaFunction{EpicInv}\AgdaSpace{}%
\AgdaOperator{\AgdaFunction{∣}}\AgdaSpace{}%
\AgdaBound{h}\AgdaSpace{}%
\AgdaOperator{\AgdaFunction{∣}}\AgdaSpace{}%
\AgdaBound{hE}\AgdaSpace{}%
\AgdaBound{b}\<%
\\
%
\\[\AgdaEmptyExtraSkip]%
%
\>[3]\AgdaFunction{bfa}\AgdaSpace{}%
\AgdaSymbol{:}\AgdaSpace{}%
\AgdaBound{b}\AgdaSpace{}%
\AgdaOperator{\AgdaDatatype{≡}}\AgdaSpace{}%
\AgdaFunction{fmap}\AgdaSpace{}%
\AgdaOperator{\AgdaFunction{⟦}}\AgdaSpace{}%
\AgdaFunction{a}\AgdaSpace{}%
\AgdaOperator{\AgdaFunction{⟧}}\<%
\\
%
\>[3]\AgdaFunction{bfa}\AgdaSpace{}%
\AgdaSymbol{=}\AgdaSpace{}%
\AgdaSymbol{(}\AgdaFunction{cong-app}\AgdaSpace{}%
\AgdaSymbol{(}\AgdaFunction{EpicInvIsRightInv}\AgdaSpace{}%
\AgdaBound{gfe}\AgdaSpace{}%
\AgdaOperator{\AgdaFunction{∣}}\AgdaSpace{}%
\AgdaBound{h}\AgdaSpace{}%
\AgdaOperator{\AgdaFunction{∣}}\AgdaSpace{}%
\AgdaBound{hE}\AgdaSymbol{)}\AgdaSpace{}%
\AgdaBound{b}\AgdaSymbol{)}\AgdaOperator{\AgdaFunction{⁻¹}}\<%
\\
%
\\[\AgdaEmptyExtraSkip]%
%
\>[3]\AgdaFunction{γ}\AgdaSpace{}%
\AgdaSymbol{:}\AgdaSpace{}%
\AgdaOperator{\AgdaDatatype{Image}}\AgdaSpace{}%
\AgdaFunction{fmap}\AgdaSpace{}%
\AgdaOperator{\AgdaDatatype{∋}}\AgdaSpace{}%
\AgdaBound{b}\<%
\\
%
\>[3]\AgdaFunction{γ}\AgdaSpace{}%
\AgdaSymbol{=}\AgdaSpace{}%
\AgdaInductiveConstructor{Image\AgdaUnderscore{}∋\AgdaUnderscore{}.eq}\AgdaSpace{}%
\AgdaBound{b}\AgdaSpace{}%
\AgdaOperator{\AgdaFunction{⟦}}\AgdaSpace{}%
\AgdaFunction{a}\AgdaSpace{}%
\AgdaOperator{\AgdaFunction{⟧}}\AgdaSpace{}%
\AgdaFunction{bfa}\<%
\\
%
\\[\AgdaEmptyExtraSkip]%
%
\>[2]\AgdaFunction{fmon}\AgdaSpace{}%
\AgdaSymbol{:}\AgdaSpace{}%
\AgdaFunction{Monic}\AgdaSpace{}%
\AgdaFunction{fmap}\<%
\\
%
\>[2]\AgdaFunction{fmon}\AgdaSpace{}%
\AgdaSymbol{(}\AgdaDottedPattern{\AgdaSymbol{.(}}\AgdaDottedPattern{\AgdaOperator{\AgdaField{⟨}}}\AgdaSpace{}%
\AgdaDottedPattern{\AgdaFunction{θ}}\AgdaSpace{}%
\AgdaDottedPattern{\AgdaOperator{\AgdaField{⟩}}}\AgdaSpace{}%
\AgdaDottedPattern{\AgdaBound{a}}\AgdaDottedPattern{\AgdaSymbol{)}}\AgdaSpace{}%
\AgdaOperator{\AgdaInductiveConstructor{,}}\AgdaSpace{}%
\AgdaBound{a}\AgdaSpace{}%
\AgdaOperator{\AgdaInductiveConstructor{,}}\AgdaSpace{}%
\AgdaInductiveConstructor{𝓇ℯ𝒻𝓁}\AgdaSymbol{)}\AgdaSpace{}%
\AgdaSymbol{(}\AgdaDottedPattern{\AgdaSymbol{.(}}\AgdaDottedPattern{\AgdaOperator{\AgdaField{⟨}}}\AgdaSpace{}%
\AgdaDottedPattern{\AgdaFunction{θ}}\AgdaSpace{}%
\AgdaDottedPattern{\AgdaOperator{\AgdaField{⟩}}}\AgdaSpace{}%
\AgdaDottedPattern{\AgdaBound{a'}}\AgdaDottedPattern{\AgdaSymbol{)}}\AgdaSpace{}%
\AgdaOperator{\AgdaInductiveConstructor{,}}\AgdaSpace{}%
\AgdaBound{a'}\AgdaSpace{}%
\AgdaOperator{\AgdaInductiveConstructor{,}}\AgdaSpace{}%
\AgdaInductiveConstructor{𝓇ℯ𝒻𝓁}\AgdaSymbol{)}\AgdaSpace{}%
\AgdaBound{faa'}\AgdaSpace{}%
\AgdaSymbol{=}\<%
\\
\>[2][@{}l@{\AgdaIndent{0}}]%
\>[3]\AgdaFunction{class-extensionality'}\AgdaSpace{}%
\AgdaSymbol{\{}\AgdaArgument{𝑹}\AgdaSpace{}%
\AgdaSymbol{=}\AgdaSpace{}%
\AgdaOperator{\AgdaField{⟨}}\AgdaSpace{}%
\AgdaFunction{kercon}\AgdaSpace{}%
\AgdaBound{𝑩}\AgdaSpace{}%
\AgdaBound{h}\AgdaSpace{}%
\AgdaOperator{\AgdaField{⟩}}\AgdaSpace{}%
\AgdaOperator{\AgdaInductiveConstructor{,}}\AgdaSpace{}%
\AgdaBound{ssR}\AgdaSymbol{\}}\AgdaSpace{}%
\AgdaBound{pe}\AgdaSpace{}%
\AgdaBound{ssA}\AgdaSpace{}%
\AgdaSymbol{(}\AgdaField{IsEquiv}\AgdaSpace{}%
\AgdaFunction{θ}\AgdaSymbol{)}\AgdaSpace{}%
\AgdaBound{faa'}\<%
\\
%
\\[\AgdaEmptyExtraSkip]%
%
\>[2]\AgdaFunction{femb}\AgdaSpace{}%
\AgdaSymbol{:}\AgdaSpace{}%
\AgdaFunction{is-embedding}\AgdaSpace{}%
\AgdaFunction{fmap}\<%
\\
%
\>[2]\AgdaFunction{femb}\AgdaSpace{}%
\AgdaSymbol{=}\AgdaSpace{}%
\AgdaFunction{monic-is-embedding|sets}\AgdaSpace{}%
\AgdaFunction{fmap}\AgdaSpace{}%
\AgdaBound{Bset}\AgdaSpace{}%
\AgdaFunction{fmon}\<%
\end{code}
\ccpad
The argument used above to prove \af{NoetherHomUnique} can also be used to prove uniqueness of the epimorphism \ab f found in the isomorphism theorem.
\ccpad
\begin{code}%
\>[0][@{}l@{\AgdaIndent{1}}]%
\>[1]\AgdaFunction{NoetherIsoUnique}\AgdaSpace{}%
\AgdaSymbol{:}\AgdaSpace{}%
\>[334I]\AgdaSymbol{(}\AgdaBound{f}\AgdaSpace{}%
\AgdaBound{g}\AgdaSpace{}%
\AgdaSymbol{:}\AgdaSpace{}%
\AgdaFunction{epi}\AgdaSpace{}%
\AgdaSymbol{(}\AgdaBound{𝑨}\AgdaSpace{}%
\AgdaOperator{\AgdaFunction{[}}\AgdaSpace{}%
\AgdaBound{𝑩}\AgdaSpace{}%
\AgdaOperator{\AgdaFunction{]/ker}}\AgdaSpace{}%
\AgdaBound{h}\AgdaSymbol{)}\AgdaSpace{}%
\AgdaBound{𝑩}\AgdaSymbol{)}\AgdaSpace{}%
\AgdaSymbol{→}\AgdaSpace{}%
\AgdaOperator{\AgdaFunction{∣}}\AgdaSpace{}%
\AgdaBound{h}\AgdaSpace{}%
\AgdaOperator{\AgdaFunction{∣}}\AgdaSpace{}%
\AgdaOperator{\AgdaDatatype{≡}}\AgdaSpace{}%
\AgdaOperator{\AgdaFunction{∣}}\AgdaSpace{}%
\AgdaBound{f}\AgdaSpace{}%
\AgdaOperator{\AgdaFunction{∣}}\AgdaSpace{}%
\AgdaOperator{\AgdaFunction{∘}}\AgdaSpace{}%
\AgdaOperator{\AgdaFunction{∣}}\AgdaSpace{}%
\AgdaFunction{πker}\AgdaSpace{}%
\AgdaBound{𝑩}\AgdaSpace{}%
\AgdaBound{h}\AgdaSpace{}%
\AgdaOperator{\AgdaFunction{∣}}\<%
\\
\>[1][@{}l@{\AgdaIndent{0}}]%
\>[2]\AgdaSymbol{→}%
\>[.][@{}l@{}]\<[334I]%
\>[20]\AgdaOperator{\AgdaFunction{∣}}\AgdaSpace{}%
\AgdaBound{h}\AgdaSpace{}%
\AgdaOperator{\AgdaFunction{∣}}\AgdaSpace{}%
\AgdaOperator{\AgdaDatatype{≡}}\AgdaSpace{}%
\AgdaOperator{\AgdaFunction{∣}}\AgdaSpace{}%
\AgdaBound{g}\AgdaSpace{}%
\AgdaOperator{\AgdaFunction{∣}}\AgdaSpace{}%
\AgdaOperator{\AgdaFunction{∘}}\AgdaSpace{}%
\AgdaOperator{\AgdaFunction{∣}}\AgdaSpace{}%
\AgdaFunction{πker}\AgdaSpace{}%
\AgdaBound{𝑩}\AgdaSpace{}%
\AgdaBound{h}\AgdaSpace{}%
\AgdaOperator{\AgdaFunction{∣}}\AgdaSpace{}%
\AgdaSymbol{→}\AgdaSpace{}%
\AgdaSymbol{∀}\AgdaSpace{}%
\AgdaBound{a}\AgdaSpace{}%
\AgdaSymbol{→}\AgdaSpace{}%
\AgdaOperator{\AgdaFunction{∣}}\AgdaSpace{}%
\AgdaBound{f}\AgdaSpace{}%
\AgdaOperator{\AgdaFunction{∣}}\AgdaSpace{}%
\AgdaBound{a}\AgdaSpace{}%
\AgdaOperator{\AgdaDatatype{≡}}\AgdaSpace{}%
\AgdaOperator{\AgdaFunction{∣}}\AgdaSpace{}%
\AgdaBound{g}\AgdaSpace{}%
\AgdaOperator{\AgdaFunction{∣}}\AgdaSpace{}%
\AgdaBound{a}\<%
\\
%
\\[\AgdaEmptyExtraSkip]%
%
\>[1]\AgdaFunction{NoetherIsoUnique}\AgdaSpace{}%
\AgdaBound{f}\AgdaSpace{}%
\AgdaBound{g}\AgdaSpace{}%
\AgdaBound{hfk}\AgdaSpace{}%
\AgdaBound{hgk}\AgdaSpace{}%
\AgdaSymbol{(}\AgdaDottedPattern{\AgdaSymbol{.(}}\AgdaDottedPattern{\AgdaOperator{\AgdaField{⟨}}}\AgdaSpace{}%
\AgdaDottedPattern{\AgdaFunction{kercon}}\AgdaSpace{}%
\AgdaDottedPattern{\AgdaBound{𝑩}}\AgdaSpace{}%
\AgdaDottedPattern{\AgdaBound{h}}\AgdaSpace{}%
\AgdaDottedPattern{\AgdaOperator{\AgdaField{⟩}}}\AgdaSpace{}%
\AgdaDottedPattern{\AgdaBound{a}}\AgdaDottedPattern{\AgdaSymbol{)}}\AgdaSpace{}%
\AgdaOperator{\AgdaInductiveConstructor{,}}\AgdaSpace{}%
\AgdaBound{a}\AgdaSpace{}%
\AgdaOperator{\AgdaInductiveConstructor{,}}\AgdaSpace{}%
\AgdaInductiveConstructor{𝓇ℯ𝒻𝓁}\AgdaSymbol{)}\AgdaSpace{}%
\AgdaSymbol{=}\<%
\\
%
\\[\AgdaEmptyExtraSkip]%
\>[1][@{}l@{\AgdaIndent{0}}]%
\>[2]\AgdaKeyword{let}\AgdaSpace{}%
\AgdaBound{θ}\AgdaSpace{}%
\AgdaSymbol{=}\AgdaSpace{}%
\AgdaSymbol{(}\AgdaOperator{\AgdaField{⟨}}\AgdaSpace{}%
\AgdaFunction{kercon}\AgdaSpace{}%
\AgdaBound{𝑩}\AgdaSpace{}%
\AgdaBound{h}\AgdaSpace{}%
\AgdaOperator{\AgdaField{⟩}}\AgdaSpace{}%
\AgdaBound{a}\AgdaSpace{}%
\AgdaOperator{\AgdaInductiveConstructor{,}}\AgdaSpace{}%
\AgdaBound{a}\AgdaSpace{}%
\AgdaOperator{\AgdaInductiveConstructor{,}}\AgdaSpace{}%
\AgdaInductiveConstructor{𝓇ℯ𝒻𝓁}\AgdaSymbol{)}\AgdaSpace{}%
\AgdaKeyword{in}\<%
\\
%
\\[\AgdaEmptyExtraSkip]%
\>[2][@{}l@{\AgdaIndent{0}}]%
\>[3]\AgdaOperator{\AgdaFunction{∣}}\AgdaSpace{}%
\AgdaBound{f}\AgdaSpace{}%
\AgdaOperator{\AgdaFunction{∣}}\AgdaSpace{}%
\AgdaBound{θ}%
\>[13]\AgdaOperator{\AgdaFunction{≡⟨}}\AgdaSpace{}%
\AgdaFunction{cong-app}\AgdaSymbol{(}\AgdaBound{hfk}\AgdaSpace{}%
\AgdaOperator{\AgdaFunction{⁻¹}}\AgdaSymbol{)}\AgdaBound{a}\AgdaSpace{}%
\AgdaOperator{\AgdaFunction{⟩}}%
\>[37]\AgdaOperator{\AgdaFunction{∣}}\AgdaSpace{}%
\AgdaBound{h}\AgdaSpace{}%
\AgdaOperator{\AgdaFunction{∣}}\AgdaSpace{}%
\AgdaBound{a}%
\>[47]\AgdaOperator{\AgdaFunction{≡⟨}}\AgdaSpace{}%
\AgdaFunction{cong-app}\AgdaSymbol{(}\AgdaBound{hgk}\AgdaSymbol{)}\AgdaBound{a}\AgdaSpace{}%
\AgdaOperator{\AgdaFunction{⟩}}%
\>[69]\AgdaOperator{\AgdaFunction{∣}}\AgdaSpace{}%
\AgdaBound{g}\AgdaSpace{}%
\AgdaOperator{\AgdaFunction{∣}}\AgdaSpace{}%
\AgdaBound{θ}%
\>[79]\AgdaOperator{\AgdaFunction{∎}}\<%
\end{code}

\subsubsection{Composition of homomorphisms}\label{homomorphism-composition}

The composition of homomorphisms is again a homomorphism.  There are a number of
alternative ways to formalize this fact in Agda. The two representations
included in the \ualib are the following.
\ccpad
\begin{code}%
\>[1]\AgdaFunction{∘-hom}\AgdaSpace{}%
\AgdaSymbol{:}\AgdaSpace{}%
\AgdaSymbol{(}\AgdaBound{𝑨}\AgdaSpace{}%
\AgdaSymbol{:}\AgdaSpace{}%
\AgdaFunction{Algebra}\AgdaSpace{}%
\AgdaBound{𝓧}\AgdaSpace{}%
\AgdaBound{𝑆}\AgdaSymbol{)\{}\AgdaBound{𝑩}\AgdaSpace{}%
\AgdaSymbol{:}\AgdaSpace{}%
\AgdaFunction{Algebra}\AgdaSpace{}%
\AgdaBound{𝓨}\AgdaSpace{}%
\AgdaBound{𝑆}\AgdaSymbol{\}(}\AgdaBound{𝑪}\AgdaSpace{}%
\AgdaSymbol{:}\AgdaSpace{}%
\AgdaFunction{Algebra}\AgdaSpace{}%
\AgdaBound{𝓩}\AgdaSpace{}%
\AgdaBound{𝑆}\AgdaSymbol{)}\<%
\\
\>[1][@{}l@{\AgdaIndent{0}}]%
\>[2]\AgdaSymbol{→}%
\>[10]\AgdaFunction{hom}\AgdaSpace{}%
\AgdaBound{𝑨}\AgdaSpace{}%
\AgdaBound{𝑩}%
\>[19]\AgdaSymbol{→}%
\>[22]\AgdaFunction{hom}\AgdaSpace{}%
\AgdaBound{𝑩}\AgdaSpace{}%
\AgdaBound{𝑪}%
\>[31]\AgdaSymbol{→}%
\>[34]\AgdaFunction{hom}\AgdaSpace{}%
\AgdaBound{𝑨}\AgdaSpace{}%
\AgdaBound{𝑪}\<%
\\
%
\\[\AgdaEmptyExtraSkip]%
%
\>[1]\AgdaFunction{∘-hom}\AgdaSpace{}%
\AgdaBound{𝑨}\AgdaSpace{}%
\AgdaSymbol{\{}\AgdaBound{𝑩}\AgdaSymbol{\}}\AgdaSpace{}%
\AgdaBound{𝑪}\AgdaSpace{}%
\AgdaSymbol{(}\AgdaBound{g}\AgdaSpace{}%
\AgdaOperator{\AgdaInductiveConstructor{,}}\AgdaSpace{}%
\AgdaBound{ghom}\AgdaSymbol{)}\AgdaSpace{}%
\AgdaSymbol{(}\AgdaBound{h}\AgdaSpace{}%
\AgdaOperator{\AgdaInductiveConstructor{,}}\AgdaSpace{}%
\AgdaBound{hhom}\AgdaSymbol{)}\AgdaSpace{}%
\AgdaSymbol{=}\AgdaSpace{}%
\AgdaBound{h}\AgdaSpace{}%
\AgdaOperator{\AgdaFunction{∘}}\AgdaSpace{}%
\AgdaBound{g}\AgdaSpace{}%
\AgdaOperator{\AgdaInductiveConstructor{,}}\AgdaSpace{}%
\AgdaFunction{γ}\AgdaSpace{}%
\AgdaKeyword{where}\<%
\\
%
\\[\AgdaEmptyExtraSkip]%
\>[1][@{}l@{\AgdaIndent{0}}]%
\>[2]\AgdaFunction{γ}\AgdaSpace{}%
\AgdaSymbol{:}\AgdaSpace{}%
\AgdaSymbol{∀}\AgdaSpace{}%
\AgdaBound{𝑓}\AgdaSpace{}%
\AgdaBound{a}\AgdaSpace{}%
\AgdaSymbol{→}\AgdaSpace{}%
\AgdaSymbol{(}\AgdaBound{h}\AgdaSpace{}%
\AgdaOperator{\AgdaFunction{∘}}\AgdaSpace{}%
\AgdaBound{g}\AgdaSymbol{)((}\AgdaBound{𝑓}\AgdaSpace{}%
\AgdaOperator{\AgdaFunction{̂}}\AgdaSpace{}%
\AgdaBound{𝑨}\AgdaSymbol{)}\AgdaSpace{}%
\AgdaBound{a}\AgdaSymbol{)}\AgdaSpace{}%
\AgdaOperator{\AgdaDatatype{≡}}\AgdaSpace{}%
\AgdaSymbol{(}\AgdaBound{𝑓}\AgdaSpace{}%
\AgdaOperator{\AgdaFunction{̂}}\AgdaSpace{}%
\AgdaBound{𝑪}\AgdaSymbol{)(}\AgdaBound{h}\AgdaSpace{}%
\AgdaOperator{\AgdaFunction{∘}}\AgdaSpace{}%
\AgdaBound{g}\AgdaSpace{}%
\AgdaOperator{\AgdaFunction{∘}}\AgdaSpace{}%
\AgdaBound{a}\AgdaSymbol{)}\<%
\\
%
\\[\AgdaEmptyExtraSkip]%
%
\>[2]\AgdaFunction{γ}\AgdaSpace{}%
\AgdaBound{𝑓}\AgdaSpace{}%
\AgdaBound{a}\AgdaSpace{}%
\AgdaSymbol{=}%
\>[824I]\AgdaSymbol{(}\AgdaBound{h}\AgdaSpace{}%
\AgdaOperator{\AgdaFunction{∘}}\AgdaSpace{}%
\AgdaBound{g}\AgdaSymbol{)}\AgdaSpace{}%
\AgdaSymbol{((}\AgdaBound{𝑓}\AgdaSpace{}%
\AgdaOperator{\AgdaFunction{̂}}\AgdaSpace{}%
\AgdaBound{𝑨}\AgdaSymbol{)}\AgdaSpace{}%
\AgdaBound{a}\AgdaSymbol{)}\AgdaSpace{}%
\AgdaOperator{\AgdaFunction{≡⟨}}\AgdaSpace{}%
\AgdaFunction{ap}\AgdaSpace{}%
\AgdaBound{h}\AgdaSpace{}%
\AgdaSymbol{(}\AgdaSpace{}%
\AgdaBound{ghom}\AgdaSpace{}%
\AgdaBound{𝑓}\AgdaSpace{}%
\AgdaBound{a}\AgdaSpace{}%
\AgdaSymbol{)}\AgdaSpace{}%
\AgdaOperator{\AgdaFunction{⟩}}\<%
\\
\>[.][@{}l@{}]\<[824I]%
\>[10]\AgdaBound{h}\AgdaSpace{}%
\AgdaSymbol{((}\AgdaBound{𝑓}\AgdaSpace{}%
\AgdaOperator{\AgdaFunction{̂}}\AgdaSpace{}%
\AgdaBound{𝑩}\AgdaSymbol{)}\AgdaSpace{}%
\AgdaSymbol{(}\AgdaBound{g}\AgdaSpace{}%
\AgdaOperator{\AgdaFunction{∘}}\AgdaSpace{}%
\AgdaBound{a}\AgdaSymbol{))}\AgdaSpace{}%
\AgdaOperator{\AgdaFunction{≡⟨}}\AgdaSpace{}%
\AgdaBound{hhom}\AgdaSpace{}%
\AgdaBound{𝑓}\AgdaSpace{}%
\AgdaSymbol{(}\AgdaSpace{}%
\AgdaBound{g}\AgdaSpace{}%
\AgdaOperator{\AgdaFunction{∘}}\AgdaSpace{}%
\AgdaBound{a}\AgdaSpace{}%
\AgdaSymbol{)}\AgdaSpace{}%
\AgdaOperator{\AgdaFunction{⟩}}\<%
\\
%
\>[10]\AgdaSymbol{(}\AgdaBound{𝑓}\AgdaSpace{}%
\AgdaOperator{\AgdaFunction{̂}}\AgdaSpace{}%
\AgdaBound{𝑪}\AgdaSymbol{)}\AgdaSpace{}%
\AgdaSymbol{(}\AgdaBound{h}\AgdaSpace{}%
\AgdaOperator{\AgdaFunction{∘}}\AgdaSpace{}%
\AgdaBound{g}\AgdaSpace{}%
\AgdaOperator{\AgdaFunction{∘}}\AgdaSpace{}%
\AgdaBound{a}\AgdaSymbol{)}\AgdaSpace{}%
\AgdaOperator{\AgdaFunction{∎}}\<%
\\
%
\\[\AgdaEmptyExtraSkip]%
%
\\[\AgdaEmptyExtraSkip]%
%
\>[1]\AgdaFunction{∘-is-hom}\AgdaSpace{}%
\AgdaSymbol{:}%
\>[864I]\AgdaSymbol{(}\AgdaBound{𝑨}\AgdaSpace{}%
\AgdaSymbol{:}\AgdaSpace{}%
\AgdaFunction{Algebra}\AgdaSpace{}%
\AgdaBound{𝓧}\AgdaSpace{}%
\AgdaBound{𝑆}\AgdaSymbol{)\{}\AgdaBound{𝑩}\AgdaSpace{}%
\AgdaSymbol{:}\AgdaSpace{}%
\AgdaFunction{Algebra}\AgdaSpace{}%
\AgdaBound{𝓨}\AgdaSpace{}%
\AgdaBound{𝑆}\AgdaSymbol{\}(}\AgdaBound{𝑪}\AgdaSpace{}%
\AgdaSymbol{:}\AgdaSpace{}%
\AgdaFunction{Algebra}\AgdaSpace{}%
\AgdaBound{𝓩}\AgdaSpace{}%
\AgdaBound{𝑆}\AgdaSymbol{)}\<%
\\
\>[.][@{}l@{}]\<[864I]%
\>[12]\AgdaSymbol{\{}\AgdaBound{f}\AgdaSpace{}%
\AgdaSymbol{:}\AgdaSpace{}%
\AgdaOperator{\AgdaFunction{∣}}\AgdaSpace{}%
\AgdaBound{𝑨}\AgdaSpace{}%
\AgdaOperator{\AgdaFunction{∣}}\AgdaSpace{}%
\AgdaSymbol{→}\AgdaSpace{}%
\AgdaOperator{\AgdaFunction{∣}}\AgdaSpace{}%
\AgdaBound{𝑩}\AgdaSpace{}%
\AgdaOperator{\AgdaFunction{∣}}\AgdaSymbol{\}}\AgdaSpace{}%
\AgdaSymbol{\{}\AgdaBound{g}\AgdaSpace{}%
\AgdaSymbol{:}\AgdaSpace{}%
\AgdaOperator{\AgdaFunction{∣}}\AgdaSpace{}%
\AgdaBound{𝑩}\AgdaSpace{}%
\AgdaOperator{\AgdaFunction{∣}}\AgdaSpace{}%
\AgdaSymbol{→}\AgdaSpace{}%
\AgdaOperator{\AgdaFunction{∣}}\AgdaSpace{}%
\AgdaBound{𝑪}\AgdaSpace{}%
\AgdaOperator{\AgdaFunction{∣}}\AgdaSymbol{\}}\<%
\\
\>[1][@{}l@{\AgdaIndent{0}}]%
\>[2]\AgdaSymbol{→}%
\>[12]\AgdaFunction{is-homomorphism}\AgdaSpace{}%
\AgdaBound{𝑨}\AgdaSpace{}%
\AgdaBound{𝑩}\AgdaSpace{}%
\AgdaBound{f}\AgdaSpace{}%
\AgdaSymbol{→}\AgdaSpace{}%
\AgdaFunction{is-homomorphism}\AgdaSpace{}%
\AgdaBound{𝑩}\AgdaSpace{}%
\AgdaBound{𝑪}\AgdaSpace{}%
\AgdaBound{g}\<%
\\
%
\>[2]\AgdaSymbol{→}%
\>[12]\AgdaFunction{is-homomorphism}\AgdaSpace{}%
\AgdaBound{𝑨}\AgdaSpace{}%
\AgdaBound{𝑪}\AgdaSpace{}%
\AgdaSymbol{(}\AgdaBound{g}\AgdaSpace{}%
\AgdaOperator{\AgdaFunction{∘}}\AgdaSpace{}%
\AgdaBound{f}\AgdaSymbol{)}\<%
\\
%
\\[\AgdaEmptyExtraSkip]%
%
\>[1]\AgdaFunction{∘-is-hom}\AgdaSpace{}%
\AgdaBound{𝑨}\AgdaSpace{}%
\AgdaBound{𝑪}\AgdaSpace{}%
\AgdaSymbol{\{}\AgdaBound{f}\AgdaSymbol{\}}\AgdaSpace{}%
\AgdaSymbol{\{}\AgdaBound{g}\AgdaSymbol{\}}\AgdaSpace{}%
\AgdaBound{fhom}\AgdaSpace{}%
\AgdaBound{ghom}\AgdaSpace{}%
\AgdaSymbol{=}\AgdaSpace{}%
\AgdaOperator{\AgdaFunction{∥}}\AgdaSpace{}%
\AgdaFunction{∘-hom}\AgdaSpace{}%
\AgdaBound{𝑨}\AgdaSpace{}%
\AgdaBound{𝑪}\AgdaSpace{}%
\AgdaSymbol{(}\AgdaBound{f}\AgdaSpace{}%
\AgdaOperator{\AgdaInductiveConstructor{,}}\AgdaSpace{}%
\AgdaBound{fhom}\AgdaSymbol{)}\AgdaSpace{}%
\AgdaSymbol{(}\AgdaBound{g}\AgdaSpace{}%
\AgdaOperator{\AgdaInductiveConstructor{,}}\AgdaSpace{}%
\AgdaBound{ghom}\AgdaSymbol{)}\AgdaSpace{}%
\AgdaOperator{\AgdaFunction{∥}}\<%
\end{code}

\subsubsection{Homomorphism decomposition}\label{homomorphism-decomposition}

If \ab{g} \as : \af{hom} \ab 𝑨 \ab 𝑩, \ab{h} \as : \af{hom} \ab 𝑨 \ab 𝑪, \ab{h} is surjective, and \af{ker} \ab h \af ⊆ \af{ker} \ab g, then there exists \ab{ϕ} \as : \af{hom} \ab 𝑪 \ab 𝑩 such that \ab{g} \as = \ab ϕ \af ∘ \ab h, that is, such that the following diagram commutes.\\

\begin{tikzpicture} %\draw[gray] (0,4.5) node {{\tt TikzRepresentableFunctor3}};
  % \drawgrid{-2}{2}{-1}{4}
\tikzset{node distance=2cm, scale=1}
    \node (A) at (-1,2) {\ab 𝑨};
    \node (A') at (1,2) {\ab 𝑪};
    \node (B) at (0,0) {\ab 𝑩};
    \path[->>] (A) edge node [fill=white] {\ab h} (A');

    \path[->] (A) edge node [fill=white] {\ab g} (B);
    \path[dotted,->] (A') edge node [fill=white] {\as ∃\ab ϕ} (B);
    % \draw[|->,bend left] (-.37,1.1) to node [above] {$\mathcal C(\mathrm{inr} f, B)$} (.37,1.1);
  \end{tikzpicture}\\
% \begin{verbatim}
% 𝑨---- h -->>𝑪
%  \         .
%   \       .
%    g     ∃ϕ
%     \   .
%      \ .
%       V
%       𝑩
% \end{verbatim}
This, or some variation of it, is sometimes referred to as the \defn{Second Isomorphism Theorem}. We formalize its statement and proof as follows. (Notice that the proof is constructive.)
\ccpad
\begin{code}%
\>[0]\AgdaFunction{homFactor}\AgdaSpace{}%
\AgdaSymbol{:}%
\>[1069I]\AgdaSymbol{\{}\AgdaBound{𝓤}\AgdaSpace{}%
\AgdaSymbol{:}\AgdaSpace{}%
\AgdaFunction{Universe}\AgdaSymbol{\}}\AgdaSpace{}%
\AgdaSymbol{→}\AgdaSpace{}%
\AgdaFunction{funext}\AgdaSpace{}%
\AgdaBound{𝓤}\AgdaSpace{}%
\AgdaBound{𝓤}\AgdaSpace{}%
\AgdaSymbol{→}\AgdaSpace{}%
\AgdaSymbol{\{}\AgdaBound{𝑨}\AgdaSpace{}%
\AgdaBound{𝑩}\AgdaSpace{}%
\AgdaBound{𝑪}\AgdaSpace{}%
\AgdaSymbol{:}\AgdaSpace{}%
\AgdaFunction{Algebra}\AgdaSpace{}%
\AgdaBound{𝓤}\AgdaSpace{}%
\AgdaBound{𝑆}\AgdaSymbol{\}}\<%
\\
\>[.][@{}l@{}]\<[1069I]%
\>[12]\AgdaSymbol{(}\AgdaBound{g}\AgdaSpace{}%
\AgdaSymbol{:}\AgdaSpace{}%
\AgdaFunction{hom}\AgdaSpace{}%
\AgdaBound{𝑨}\AgdaSpace{}%
\AgdaBound{𝑩}\AgdaSymbol{)}\AgdaSpace{}%
\AgdaSymbol{(}\AgdaBound{h}\AgdaSpace{}%
\AgdaSymbol{:}\AgdaSpace{}%
\AgdaFunction{hom}\AgdaSpace{}%
\AgdaBound{𝑨}\AgdaSpace{}%
\AgdaBound{𝑪}\AgdaSymbol{)}\<%
\\
\>[0][@{}l@{\AgdaIndent{0}}]%
\>[1]\AgdaSymbol{→}%
\>[12]\AgdaFunction{ker-pred}\AgdaSpace{}%
\AgdaOperator{\AgdaFunction{∣}}\AgdaSpace{}%
\AgdaBound{h}\AgdaSpace{}%
\AgdaOperator{\AgdaFunction{∣}}\AgdaSpace{}%
\AgdaOperator{\AgdaFunction{⊆}}\AgdaSpace{}%
\AgdaFunction{ker-pred}\AgdaSpace{}%
\AgdaOperator{\AgdaFunction{∣}}\AgdaSpace{}%
\AgdaBound{g}\AgdaSpace{}%
\AgdaOperator{\AgdaFunction{∣}}%
\>[45]\AgdaSymbol{→}%
\>[49]\AgdaFunction{Epic}\AgdaSpace{}%
\AgdaOperator{\AgdaFunction{∣}}\AgdaSpace{}%
\AgdaBound{h}\AgdaSpace{}%
\AgdaOperator{\AgdaFunction{∣}}\<%
\\
%
\>[12]\AgdaComment{-------------------------------------------------------}\<%
\\
%
\>[1]\AgdaSymbol{→}%
\>[12]\AgdaFunction{Σ}\AgdaSpace{}%
\AgdaBound{ϕ}\AgdaSpace{}%
\AgdaFunction{꞉}\AgdaSpace{}%
\AgdaSymbol{(}\AgdaFunction{hom}\AgdaSpace{}%
\AgdaBound{𝑪}\AgdaSpace{}%
\AgdaBound{𝑩}\AgdaSymbol{)}\AgdaSpace{}%
\AgdaFunction{,}\AgdaSpace{}%
\AgdaOperator{\AgdaFunction{∣}}\AgdaSpace{}%
\AgdaBound{g}\AgdaSpace{}%
\AgdaOperator{\AgdaFunction{∣}}\AgdaSpace{}%
\AgdaOperator{\AgdaDatatype{≡}}\AgdaSpace{}%
\AgdaOperator{\AgdaFunction{∣}}\AgdaSpace{}%
\AgdaBound{ϕ}\AgdaSpace{}%
\AgdaOperator{\AgdaFunction{∣}}\AgdaSpace{}%
\AgdaOperator{\AgdaFunction{∘}}\AgdaSpace{}%
\AgdaOperator{\AgdaFunction{∣}}\AgdaSpace{}%
\AgdaBound{h}\AgdaSpace{}%
\AgdaOperator{\AgdaFunction{∣}}\<%
\\
%
\\[\AgdaEmptyExtraSkip]%
\>[0]\AgdaFunction{homFactor}\AgdaSpace{}%
\AgdaBound{fe}\AgdaSymbol{\{}\AgdaBound{𝑨}\AgdaSymbol{\}\{}\AgdaBound{𝑩}\AgdaSymbol{\}\{}\AgdaBound{𝑪}\AgdaSymbol{\}(}\AgdaBound{g}\AgdaSpace{}%
\AgdaOperator{\AgdaInductiveConstructor{,}}\AgdaSpace{}%
\AgdaBound{ghom}\AgdaSymbol{)(}\AgdaBound{h}\AgdaSpace{}%
\AgdaOperator{\AgdaInductiveConstructor{,}}\AgdaSpace{}%
\AgdaBound{hhom}\AgdaSymbol{)}\AgdaSpace{}%
\AgdaBound{Kh⊆Kg}\AgdaSpace{}%
\AgdaBound{hEpi}\AgdaSpace{}%
\AgdaSymbol{=}\AgdaSpace{}%
\AgdaSymbol{(}\AgdaFunction{ϕ}\AgdaSpace{}%
\AgdaOperator{\AgdaInductiveConstructor{,}}\AgdaSpace{}%
\AgdaFunction{ϕIsHomCB}\AgdaSymbol{)}\AgdaSpace{}%
\AgdaOperator{\AgdaInductiveConstructor{,}}\AgdaSpace{}%
\AgdaFunction{g≡ϕ∘h}\<%
\\
\>[0][@{}l@{\AgdaIndent{0}}]%
\>[1]\AgdaKeyword{where}\<%
\\
%
\>[1]\AgdaFunction{hInv}\AgdaSpace{}%
\AgdaSymbol{:}\AgdaSpace{}%
\AgdaOperator{\AgdaFunction{∣}}\AgdaSpace{}%
\AgdaBound{𝑪}\AgdaSpace{}%
\AgdaOperator{\AgdaFunction{∣}}\AgdaSpace{}%
\AgdaSymbol{→}\AgdaSpace{}%
\AgdaOperator{\AgdaFunction{∣}}\AgdaSpace{}%
\AgdaBound{𝑨}\AgdaSpace{}%
\AgdaOperator{\AgdaFunction{∣}}\<%
\\
%
\>[1]\AgdaFunction{hInv}\AgdaSpace{}%
\AgdaSymbol{=}\AgdaSpace{}%
\AgdaSymbol{λ}\AgdaSpace{}%
\AgdaBound{c}\AgdaSpace{}%
\AgdaSymbol{→}\AgdaSpace{}%
\AgdaSymbol{(}\AgdaFunction{EpicInv}\AgdaSpace{}%
\AgdaBound{h}\AgdaSpace{}%
\AgdaBound{hEpi}\AgdaSymbol{)}\AgdaSpace{}%
\AgdaBound{c}\<%
\\
%
\\[\AgdaEmptyExtraSkip]%
%
\>[1]\AgdaFunction{ϕ}\AgdaSpace{}%
\AgdaSymbol{:}\AgdaSpace{}%
\AgdaOperator{\AgdaFunction{∣}}\AgdaSpace{}%
\AgdaBound{𝑪}\AgdaSpace{}%
\AgdaOperator{\AgdaFunction{∣}}\AgdaSpace{}%
\AgdaSymbol{→}\AgdaSpace{}%
\AgdaOperator{\AgdaFunction{∣}}\AgdaSpace{}%
\AgdaBound{𝑩}\AgdaSpace{}%
\AgdaOperator{\AgdaFunction{∣}}\<%
\\
%
\>[1]\AgdaFunction{ϕ}\AgdaSpace{}%
\AgdaSymbol{=}\AgdaSpace{}%
\AgdaSymbol{λ}\AgdaSpace{}%
\AgdaBound{c}\AgdaSpace{}%
\AgdaSymbol{→}\AgdaSpace{}%
\AgdaBound{g}\AgdaSpace{}%
\AgdaSymbol{(}\AgdaSpace{}%
\AgdaFunction{hInv}\AgdaSpace{}%
\AgdaBound{c}\AgdaSpace{}%
\AgdaSymbol{)}\<%
\\
%
\\[\AgdaEmptyExtraSkip]%
%
\>[1]\AgdaFunction{ξ}\AgdaSpace{}%
\AgdaSymbol{:}\AgdaSpace{}%
\AgdaSymbol{∀}\AgdaSpace{}%
\AgdaBound{x}\AgdaSpace{}%
\AgdaSymbol{→}\AgdaSpace{}%
\AgdaFunction{ker-pred}\AgdaSpace{}%
\AgdaBound{h}\AgdaSpace{}%
\AgdaSymbol{(}\AgdaBound{x}\AgdaSpace{}%
\AgdaOperator{\AgdaInductiveConstructor{,}}\AgdaSpace{}%
\AgdaFunction{hInv}\AgdaSpace{}%
\AgdaSymbol{(}\AgdaBound{h}\AgdaSpace{}%
\AgdaBound{x}\AgdaSymbol{))}\<%
\\
%
\>[1]\AgdaFunction{ξ}\AgdaSpace{}%
\AgdaBound{x}\AgdaSpace{}%
\AgdaSymbol{=}\AgdaSpace{}%
\AgdaSymbol{(}\AgdaFunction{cong-app}\AgdaSpace{}%
\AgdaSymbol{(}\AgdaFunction{EpicInvIsRightInv}\AgdaSpace{}%
\AgdaBound{fe}\AgdaSpace{}%
\AgdaBound{h}\AgdaSpace{}%
\AgdaBound{hEpi}\AgdaSymbol{)}\AgdaSpace{}%
\AgdaSymbol{(}\AgdaBound{h}\AgdaSpace{}%
\AgdaBound{x}\AgdaSymbol{))}\AgdaOperator{\AgdaFunction{⁻¹}}\<%
\\
%
\\[\AgdaEmptyExtraSkip]%
%
\>[1]\AgdaFunction{g≡ϕ∘h}\AgdaSpace{}%
\AgdaSymbol{:}\AgdaSpace{}%
\AgdaBound{g}\AgdaSpace{}%
\AgdaOperator{\AgdaDatatype{≡}}\AgdaSpace{}%
\AgdaFunction{ϕ}\AgdaSpace{}%
\AgdaOperator{\AgdaFunction{∘}}\AgdaSpace{}%
\AgdaBound{h}\<%
\\
%
\>[1]\AgdaFunction{g≡ϕ∘h}\AgdaSpace{}%
\AgdaSymbol{=}\AgdaSpace{}%
\AgdaBound{fe}%
\>[13]\AgdaSymbol{λ}\AgdaSpace{}%
\AgdaBound{x}\AgdaSpace{}%
\AgdaSymbol{→}\AgdaSpace{}%
\AgdaBound{Kh⊆Kg}\AgdaSpace{}%
\AgdaSymbol{(}\AgdaFunction{ξ}\AgdaSpace{}%
\AgdaBound{x}\AgdaSymbol{)}\<%
\\
%
\\[\AgdaEmptyExtraSkip]%
%
\>[1]\AgdaFunction{ζ}\AgdaSpace{}%
\AgdaSymbol{:}\AgdaSpace{}%
\AgdaSymbol{(}\AgdaBound{𝑓}\AgdaSpace{}%
\AgdaSymbol{:}\AgdaSpace{}%
\AgdaOperator{\AgdaFunction{∣}}\AgdaSpace{}%
\AgdaBound{𝑆}\AgdaSpace{}%
\AgdaOperator{\AgdaFunction{∣}}\AgdaSymbol{)(}\AgdaBound{𝒄}\AgdaSpace{}%
\AgdaSymbol{:}\AgdaSpace{}%
\AgdaOperator{\AgdaFunction{∥}}\AgdaSpace{}%
\AgdaBound{𝑆}\AgdaSpace{}%
\AgdaOperator{\AgdaFunction{∥}}\AgdaSpace{}%
\AgdaBound{𝑓}\AgdaSpace{}%
\AgdaSymbol{→}\AgdaSpace{}%
\AgdaOperator{\AgdaFunction{∣}}\AgdaSpace{}%
\AgdaBound{𝑪}\AgdaSpace{}%
\AgdaOperator{\AgdaFunction{∣}}\AgdaSymbol{)(}\AgdaBound{x}\AgdaSpace{}%
\AgdaSymbol{:}\AgdaSpace{}%
\AgdaOperator{\AgdaFunction{∥}}\AgdaSpace{}%
\AgdaBound{𝑆}\AgdaSpace{}%
\AgdaOperator{\AgdaFunction{∥}}\AgdaSpace{}%
\AgdaBound{𝑓}\AgdaSymbol{)}\AgdaSpace{}%
\AgdaSymbol{→}%
\>[54]\AgdaBound{𝒄}\AgdaSpace{}%
\AgdaBound{x}\AgdaSpace{}%
\AgdaOperator{\AgdaDatatype{≡}}\AgdaSpace{}%
\AgdaSymbol{(}\AgdaBound{h}\AgdaSpace{}%
\AgdaOperator{\AgdaFunction{∘}}\AgdaSpace{}%
\AgdaFunction{hInv}\AgdaSymbol{)(}\AgdaBound{𝒄}\AgdaSpace{}%
\AgdaBound{x}\AgdaSymbol{)}\<%
\\
%
\>[1]\AgdaFunction{ζ}%
\>[4]\AgdaBound{𝑓}\AgdaSpace{}%
\AgdaBound{𝒄}\AgdaSpace{}%
\AgdaBound{x}\AgdaSpace{}%
\AgdaSymbol{=}\AgdaSpace{}%
\AgdaSymbol{(}\AgdaFunction{cong-app}\AgdaSpace{}%
\AgdaSymbol{(}\AgdaFunction{EpicInvIsRightInv}\AgdaSpace{}%
\AgdaBound{fe}\AgdaSpace{}%
\AgdaBound{h}\AgdaSpace{}%
\AgdaBound{hEpi}\AgdaSymbol{)}\AgdaSpace{}%
\AgdaSymbol{(}\AgdaBound{𝒄}\AgdaSpace{}%
\AgdaBound{x}\AgdaSymbol{))}\AgdaOperator{\AgdaFunction{⁻¹}}\<%
\\
%
\\[\AgdaEmptyExtraSkip]%
%
\>[1]\AgdaFunction{ι}\AgdaSpace{}%
\AgdaSymbol{:}\AgdaSpace{}%
\AgdaSymbol{(}\AgdaBound{𝑓}\AgdaSpace{}%
\AgdaSymbol{:}\AgdaSpace{}%
\AgdaOperator{\AgdaFunction{∣}}\AgdaSpace{}%
\AgdaBound{𝑆}\AgdaSpace{}%
\AgdaOperator{\AgdaFunction{∣}}\AgdaSymbol{)(}\AgdaBound{𝒄}\AgdaSpace{}%
\AgdaSymbol{:}\AgdaSpace{}%
\AgdaOperator{\AgdaFunction{∥}}\AgdaSpace{}%
\AgdaBound{𝑆}\AgdaSpace{}%
\AgdaOperator{\AgdaFunction{∥}}\AgdaSpace{}%
\AgdaBound{𝑓}\AgdaSpace{}%
\AgdaSymbol{→}\AgdaSpace{}%
\AgdaOperator{\AgdaFunction{∣}}\AgdaSpace{}%
\AgdaBound{𝑪}\AgdaSpace{}%
\AgdaOperator{\AgdaFunction{∣}}\AgdaSymbol{)}\AgdaSpace{}%
\AgdaSymbol{→}%
\>[41]\AgdaBound{𝒄}\AgdaSpace{}%
\AgdaOperator{\AgdaDatatype{≡}}\AgdaSpace{}%
\AgdaBound{h}\AgdaSpace{}%
\AgdaOperator{\AgdaFunction{∘}}\AgdaSpace{}%
\AgdaSymbol{(}\AgdaFunction{hInv}\AgdaSpace{}%
\AgdaOperator{\AgdaFunction{∘}}\AgdaSpace{}%
\AgdaBound{𝒄}\AgdaSymbol{)}\<%
\\
%
\>[1]\AgdaFunction{ι}\AgdaSpace{}%
\AgdaBound{𝑓}\AgdaSpace{}%
\AgdaBound{𝒄}\AgdaSpace{}%
\AgdaSymbol{=}\AgdaSpace{}%
\AgdaFunction{ap}\AgdaSpace{}%
\AgdaSymbol{(λ}\AgdaSpace{}%
\AgdaBound{-}\AgdaSpace{}%
\AgdaSymbol{→}\AgdaSpace{}%
\AgdaBound{-}\AgdaSpace{}%
\AgdaOperator{\AgdaFunction{∘}}\AgdaSpace{}%
\AgdaBound{𝒄}\AgdaSymbol{)(}\AgdaFunction{EpicInvIsRightInv}\AgdaSpace{}%
\AgdaBound{fe}\AgdaSpace{}%
\AgdaBound{h}\AgdaSpace{}%
\AgdaBound{hEpi}\AgdaSymbol{)}\AgdaOperator{\AgdaFunction{⁻¹}}\<%
\\
%
\\[\AgdaEmptyExtraSkip]%
%
\>[1]\AgdaFunction{useker}\AgdaSpace{}%
\AgdaSymbol{:}\AgdaSpace{}%
\AgdaSymbol{∀}\AgdaSpace{}%
\AgdaBound{𝑓}\AgdaSpace{}%
\AgdaBound{𝒄}\AgdaSpace{}%
\AgdaSymbol{→}\AgdaSpace{}%
\AgdaBound{g}\AgdaSymbol{(}\AgdaFunction{hInv}\AgdaSpace{}%
\AgdaSymbol{(}\AgdaBound{h}\AgdaSymbol{((}\AgdaBound{𝑓}\AgdaSpace{}%
\AgdaOperator{\AgdaFunction{̂}}\AgdaSpace{}%
\AgdaBound{𝑨}\AgdaSymbol{)(}\AgdaFunction{hInv}\AgdaSpace{}%
\AgdaOperator{\AgdaFunction{∘}}\AgdaSpace{}%
\AgdaBound{𝒄}\AgdaSymbol{))))}\AgdaSpace{}%
\AgdaOperator{\AgdaDatatype{≡}}\AgdaSpace{}%
\AgdaBound{g}\AgdaSymbol{((}\AgdaBound{𝑓}\AgdaSpace{}%
\AgdaOperator{\AgdaFunction{̂}}\AgdaSpace{}%
\AgdaBound{𝑨}\AgdaSymbol{)(}\AgdaFunction{hInv}\AgdaSpace{}%
\AgdaOperator{\AgdaFunction{∘}}\AgdaSpace{}%
\AgdaBound{𝒄}\AgdaSymbol{))}\<%
\\
%
\>[1]\AgdaFunction{useker}\AgdaSpace{}%
\AgdaBound{𝑓}\AgdaSpace{}%
\AgdaBound{c}\AgdaSpace{}%
\AgdaSymbol{=}\AgdaSpace{}%
\AgdaBound{Kh⊆Kg}\AgdaSpace{}%
\AgdaSymbol{(}\AgdaFunction{cong-app}\AgdaSymbol{(}\AgdaFunction{EpicInvIsRightInv}\AgdaSpace{}%
\AgdaBound{fe}\AgdaSpace{}%
\AgdaBound{h}\AgdaSpace{}%
\AgdaBound{hEpi}\AgdaSymbol{)(}\AgdaBound{h}\AgdaSpace{}%
\AgdaSymbol{((}\AgdaBound{𝑓}\AgdaSpace{}%
\AgdaOperator{\AgdaFunction{̂}}\AgdaSpace{}%
\AgdaBound{𝑨}\AgdaSymbol{)(}\AgdaFunction{hInv}\AgdaSpace{}%
\AgdaOperator{\AgdaFunction{∘}}\AgdaSpace{}%
\AgdaBound{c}\AgdaSymbol{))))}\<%
\\
%
\\[\AgdaEmptyExtraSkip]%
%
\>[1]\AgdaFunction{ϕIsHomCB}\AgdaSpace{}%
\AgdaSymbol{:}\AgdaSpace{}%
\AgdaSymbol{(}\AgdaBound{𝑓}\AgdaSpace{}%
\AgdaSymbol{:}\AgdaSpace{}%
\AgdaOperator{\AgdaFunction{∣}}\AgdaSpace{}%
\AgdaBound{𝑆}\AgdaSpace{}%
\AgdaOperator{\AgdaFunction{∣}}\AgdaSymbol{)(}\AgdaBound{𝒄}\AgdaSpace{}%
\AgdaSymbol{:}\AgdaSpace{}%
\AgdaOperator{\AgdaFunction{∥}}\AgdaSpace{}%
\AgdaBound{𝑆}\AgdaSpace{}%
\AgdaOperator{\AgdaFunction{∥}}\AgdaSpace{}%
\AgdaBound{𝑓}\AgdaSpace{}%
\AgdaSymbol{→}\AgdaSpace{}%
\AgdaOperator{\AgdaFunction{∣}}\AgdaSpace{}%
\AgdaBound{𝑪}\AgdaSpace{}%
\AgdaOperator{\AgdaFunction{∣}}\AgdaSymbol{)}\AgdaSpace{}%
\AgdaSymbol{→}\AgdaSpace{}%
\AgdaFunction{ϕ}\AgdaSymbol{((}\AgdaBound{𝑓}\AgdaSpace{}%
\AgdaOperator{\AgdaFunction{̂}}\AgdaSpace{}%
\AgdaBound{𝑪}\AgdaSymbol{)}\AgdaSpace{}%
\AgdaBound{𝒄}\AgdaSymbol{)}\AgdaSpace{}%
\>[666I]\AgdaOperator{\AgdaDatatype{≡}}\AgdaSpace{}%
\AgdaSymbol{(}\AgdaBound{𝑓}\AgdaSpace{}%
\AgdaOperator{\AgdaFunction{̂}}\AgdaSpace{}%
\AgdaBound{𝑩}\AgdaSymbol{)(}\AgdaFunction{ϕ}\AgdaSpace{}%
\AgdaOperator{\AgdaFunction{∘}}\AgdaSpace{}%
\AgdaBound{𝒄}\AgdaSymbol{)}\<%
\\
%
% \\[\AgdaEmptyExtraSkip]%
% %
\>[1]\AgdaFunction{ϕIsHomCB}\AgdaSpace{}%
\AgdaBound{𝑓}\AgdaSpace{}%
\AgdaBound{𝒄}%
\>[1330I]\AgdaSymbol{=}%
\>[17]\AgdaBound{g}\AgdaSpace{}%
\AgdaSymbol{(}\AgdaFunction{hInv}\AgdaSpace{}%
\AgdaSymbol{((}\AgdaBound{𝑓}\AgdaSpace{}%
\AgdaOperator{\AgdaFunction{̂}}\AgdaSpace{}%
\AgdaBound{𝑪}\AgdaSymbol{)}\AgdaSpace{}%
\AgdaBound{𝒄}\AgdaSymbol{))}%
\>[666I][@{}l@{\AgdaIndent{0}}]%
\>[52]\AgdaOperator{\AgdaFunction{≡⟨}}\AgdaSpace{}%
\AgdaFunction{i}%
\>[60]\AgdaOperator{\AgdaFunction{⟩}}\<%
\\
\>[1330I][@{}l@{\AgdaIndent{0}}]%
\>[16]\AgdaBound{g}\AgdaSpace{}%
\AgdaSymbol{(}\AgdaFunction{hInv}\AgdaSpace{}%
\AgdaSymbol{((}\AgdaBound{𝑓}\AgdaSpace{}%
\AgdaOperator{\AgdaFunction{̂}}\AgdaSpace{}%
\AgdaBound{𝑪}\AgdaSymbol{)(}\AgdaBound{h}\AgdaSpace{}%
\AgdaOperator{\AgdaFunction{∘}}\AgdaSpace{}%
\AgdaSymbol{(}\AgdaFunction{hInv}\AgdaSpace{}%
\AgdaOperator{\AgdaFunction{∘}}\AgdaSpace{}%
\AgdaBound{𝒄}\AgdaSymbol{))))}\AgdaSpace{}%
\>[52]\AgdaOperator{\AgdaFunction{≡⟨}}\AgdaSpace{}%
\AgdaFunction{ii}%
\>[60]\AgdaOperator{\AgdaFunction{⟩}}\<%
\\
%
\>[16]\AgdaBound{g}\AgdaSpace{}%
\AgdaSymbol{(}\AgdaFunction{hInv}\AgdaSpace{}%
\AgdaSymbol{(}\AgdaBound{h}\AgdaSpace{}%
\AgdaSymbol{((}\AgdaBound{𝑓}\AgdaSpace{}%
\AgdaOperator{\AgdaFunction{̂}}\AgdaSpace{}%
\AgdaBound{𝑨}\AgdaSymbol{)(}\AgdaFunction{hInv}\AgdaSpace{}%
\AgdaOperator{\AgdaFunction{∘}}\AgdaSpace{}%
\AgdaBound{𝒄}\AgdaSymbol{))))}%
\>[52]\AgdaOperator{\AgdaFunction{≡⟨}}\AgdaSpace{}%
\AgdaFunction{iii}\AgdaSpace{}%
\>[60]\AgdaOperator{\AgdaFunction{⟩}}\<%
\\
%
\>[16]\AgdaBound{g}\AgdaSpace{}%
\AgdaSymbol{((}\AgdaBound{𝑓}\AgdaSpace{}%
\AgdaOperator{\AgdaFunction{̂}}\AgdaSpace{}%
\AgdaBound{𝑨}\AgdaSymbol{)(}\AgdaFunction{hInv}\AgdaSpace{}%
\AgdaOperator{\AgdaFunction{∘}}\AgdaSpace{}%
\AgdaBound{𝒄}\AgdaSymbol{))}%
\>[52]\AgdaOperator{\AgdaFunction{≡⟨}}\AgdaSpace{}%
\AgdaFunction{iv}%
\>[60]\AgdaOperator{\AgdaFunction{⟩}}\<%
\\
%
\>[16]\AgdaSymbol{(}\AgdaBound{𝑓}\AgdaSpace{}%
\AgdaOperator{\AgdaFunction{̂}}\AgdaSpace{}%
\AgdaBound{𝑩}\AgdaSymbol{)(λ}\AgdaSpace{}%
\AgdaBound{x}\AgdaSpace{}%
\AgdaSymbol{→}\AgdaSpace{}%
\AgdaBound{g}\AgdaSpace{}%
\AgdaSymbol{(}\AgdaFunction{hInv}\AgdaSpace{}%
\AgdaSymbol{(}\AgdaBound{𝒄}\AgdaSpace{}%
\AgdaBound{x}\AgdaSymbol{)))}%
\>[52]\AgdaOperator{\AgdaFunction{∎}}\<%
\\
\>[1][@{}l@{\AgdaIndent{0}}]%
\>[2]\AgdaKeyword{where}\<%
\\
%
\>[2]\AgdaFunction{i}%
\>[6]\AgdaSymbol{=}\AgdaSpace{}%
\AgdaFunction{ap}\AgdaSpace{}%
\AgdaSymbol{(}\AgdaBound{g}\AgdaSpace{}%
\AgdaOperator{\AgdaFunction{∘}}\AgdaSpace{}%
\AgdaFunction{hInv}\AgdaSymbol{)}\AgdaSpace{}%
\AgdaSymbol{(}\AgdaFunction{ap}\AgdaSpace{}%
\AgdaSymbol{(}\AgdaBound{𝑓}\AgdaSpace{}%
\AgdaOperator{\AgdaFunction{̂}}\AgdaSpace{}%
\AgdaBound{𝑪}\AgdaSymbol{)}\AgdaSpace{}%
\AgdaSymbol{(}\AgdaFunction{ι}\AgdaSpace{}%
\AgdaBound{𝑓}\AgdaSpace{}%
\AgdaBound{𝒄}\AgdaSymbol{))}\<%
\\
%
\>[2]\AgdaFunction{ii}%
\>[6]\AgdaSymbol{=}\AgdaSpace{}%
\AgdaFunction{ap}\AgdaSpace{}%
\AgdaSymbol{(}\AgdaBound{g}\AgdaSpace{}%
\AgdaOperator{\AgdaFunction{∘}}\AgdaSpace{}%
\AgdaFunction{hInv}\AgdaSymbol{)}\AgdaSpace{}%
\AgdaSymbol{(}\AgdaBound{hhom}\AgdaSpace{}%
\AgdaBound{𝑓}\AgdaSpace{}%
\AgdaSymbol{(}\AgdaFunction{hInv}\AgdaSpace{}%
\AgdaOperator{\AgdaFunction{∘}}\AgdaSpace{}%
\AgdaBound{𝒄}\AgdaSymbol{))}\AgdaOperator{\AgdaFunction{⁻¹}}\<%
\\
%
\>[2]\AgdaFunction{iii}\AgdaSpace{}%
\AgdaSymbol{=}\AgdaSpace{}%
\AgdaFunction{useker}\AgdaSpace{}%
\AgdaBound{𝑓}\AgdaSpace{}%
\AgdaBound{𝒄}\<%
\\
%
\>[2]\AgdaFunction{iv}%
\>[6]\AgdaSymbol{=}\AgdaSpace{}%
\AgdaBound{ghom}\AgdaSpace{}%
\AgdaBound{𝑓}\AgdaSpace{}%
\AgdaSymbol{(}\AgdaFunction{hInv}\AgdaSpace{}%
\AgdaOperator{\AgdaFunction{∘}}\AgdaSpace{}%
\AgdaBound{𝒄}\AgdaSymbol{)}\<%
\end{code}
\ccpad
Here's a more general version.
\ccpad
\begin{code}%
\>[1]\AgdaFunction{HomFactor}\AgdaSpace{}%
\AgdaSymbol{:}%
\>[1407I]\AgdaSymbol{(}\AgdaBound{𝑨}\AgdaSpace{}%
\AgdaSymbol{:}\AgdaSpace{}%
\AgdaFunction{Algebra}\AgdaSpace{}%
\AgdaBound{𝓧}\AgdaSpace{}%
\AgdaBound{𝑆}\AgdaSymbol{)\{}\AgdaBound{𝑩}\AgdaSpace{}%
\AgdaSymbol{:}\AgdaSpace{}%
\AgdaFunction{Algebra}\AgdaSpace{}%
\AgdaBound{𝓨}\AgdaSpace{}%
\AgdaBound{𝑆}\AgdaSymbol{\}\{}\AgdaBound{𝑪}\AgdaSpace{}%
\AgdaSymbol{:}\AgdaSpace{}%
\AgdaFunction{Algebra}\AgdaSpace{}%
\AgdaBound{𝓩}\AgdaSpace{}%
\AgdaBound{𝑆}\AgdaSymbol{\}}\<%
\\
\>[.][@{}l@{}]\<[1407I]%
\>[13]\AgdaSymbol{(}\AgdaBound{β}\AgdaSpace{}%
\AgdaSymbol{:}\AgdaSpace{}%
\AgdaFunction{hom}\AgdaSpace{}%
\AgdaBound{𝑨}\AgdaSpace{}%
\AgdaBound{𝑩}\AgdaSymbol{)}\AgdaSpace{}%
\AgdaSymbol{(}\AgdaBound{γ}\AgdaSpace{}%
\AgdaSymbol{:}\AgdaSpace{}%
\AgdaFunction{hom}\AgdaSpace{}%
\AgdaBound{𝑨}\AgdaSpace{}%
\AgdaBound{𝑪}\AgdaSymbol{)}\<%
\\
\>[1][@{}l@{\AgdaIndent{0}}]%
\>[2]\AgdaSymbol{→}%
\>[13]\AgdaFunction{Epic}\AgdaSpace{}%
\AgdaOperator{\AgdaFunction{∣}}\AgdaSpace{}%
\AgdaBound{γ}\AgdaSpace{}%
\AgdaOperator{\AgdaFunction{∣}}\AgdaSpace{}%
\AgdaSymbol{→}\AgdaSpace{}%
\AgdaSymbol{(}\AgdaFunction{KER-pred}\AgdaSpace{}%
\AgdaOperator{\AgdaFunction{∣}}\AgdaSpace{}%
\AgdaBound{γ}\AgdaSpace{}%
\AgdaOperator{\AgdaFunction{∣}}\AgdaSymbol{)}\AgdaSpace{}%
\AgdaOperator{\AgdaFunction{⊆}}\AgdaSpace{}%
\AgdaSymbol{(}\AgdaFunction{KER-pred}\AgdaSpace{}%
\AgdaOperator{\AgdaFunction{∣}}\AgdaSpace{}%
\AgdaBound{β}\AgdaSpace{}%
\AgdaOperator{\AgdaFunction{∣}}\AgdaSymbol{)}\<%
\\
%
\>[13]\AgdaComment{-----------------------------------------------------------}\<%
\\
%
\>[2]\AgdaSymbol{→}%
\>[13]\AgdaFunction{Σ}\AgdaSpace{}%
\AgdaBound{ϕ}\AgdaSpace{}%
\AgdaFunction{꞉}\AgdaSpace{}%
\AgdaSymbol{(}\AgdaFunction{hom}\AgdaSpace{}%
\AgdaBound{𝑪}\AgdaSpace{}%
\AgdaBound{𝑩}\AgdaSymbol{)}\AgdaSpace{}%
\AgdaFunction{,}\AgdaSpace{}%
\AgdaOperator{\AgdaFunction{∣}}\AgdaSpace{}%
\AgdaBound{β}\AgdaSpace{}%
\AgdaOperator{\AgdaFunction{∣}}\AgdaSpace{}%
\AgdaOperator{\AgdaDatatype{≡}}\AgdaSpace{}%
\AgdaOperator{\AgdaFunction{∣}}\AgdaSpace{}%
\AgdaBound{ϕ}\AgdaSpace{}%
\AgdaOperator{\AgdaFunction{∣}}\AgdaSpace{}%
\AgdaOperator{\AgdaFunction{∘}}\AgdaSpace{}%
\AgdaOperator{\AgdaFunction{∣}}\AgdaSpace{}%
\AgdaBound{γ}\AgdaSpace{}%
\AgdaOperator{\AgdaFunction{∣}}\<%
\\
%
\\[\AgdaEmptyExtraSkip]%
%
\>[1]\AgdaFunction{HomFactor}\AgdaSpace{}%
\AgdaBound{𝑨}\AgdaSpace{}%
\AgdaSymbol{\{}\AgdaBound{𝑩}\AgdaSymbol{\}\{}\AgdaBound{𝑪}\AgdaSymbol{\}}\AgdaSpace{}%
\AgdaBound{β}\AgdaSpace{}%
\AgdaBound{γ}\AgdaSpace{}%
\AgdaBound{γE}\AgdaSpace{}%
\AgdaBound{Kγβ}\AgdaSpace{}%
\AgdaSymbol{=}\AgdaSpace{}%
\AgdaSymbol{(}\AgdaFunction{ϕ}\AgdaSpace{}%
\AgdaOperator{\AgdaInductiveConstructor{,}}\AgdaSpace{}%
\AgdaFunction{ϕIsHomCB}\AgdaSymbol{)}\AgdaSpace{}%
\AgdaOperator{\AgdaInductiveConstructor{,}}\AgdaSpace{}%
\AgdaFunction{βϕγ}\<%
\\
\>[1][@{}l@{\AgdaIndent{0}}]%
\>[2]\AgdaKeyword{where}\<%
\\
%
\>[2]\AgdaFunction{γInv}\AgdaSpace{}%
\AgdaSymbol{:}\AgdaSpace{}%
\AgdaOperator{\AgdaFunction{∣}}\AgdaSpace{}%
\AgdaBound{𝑪}\AgdaSpace{}%
\AgdaOperator{\AgdaFunction{∣}}\AgdaSpace{}%
\AgdaSymbol{→}\AgdaSpace{}%
\AgdaOperator{\AgdaFunction{∣}}\AgdaSpace{}%
\AgdaBound{𝑨}\AgdaSpace{}%
\AgdaOperator{\AgdaFunction{∣}}\<%
\\
%
\>[2]\AgdaFunction{γInv}\AgdaSpace{}%
\AgdaSymbol{=}\AgdaSpace{}%
\AgdaSymbol{λ}\AgdaSpace{}%
\AgdaBound{y}\AgdaSpace{}%
\AgdaSymbol{→}\AgdaSpace{}%
\AgdaSymbol{(}\AgdaFunction{EpicInv}\AgdaSpace{}%
\AgdaOperator{\AgdaFunction{∣}}\AgdaSpace{}%
\AgdaBound{γ}\AgdaSpace{}%
\AgdaOperator{\AgdaFunction{∣}}\AgdaSpace{}%
\AgdaBound{γE}\AgdaSymbol{)}\AgdaSpace{}%
\AgdaBound{y}\<%
\\
%
\\[\AgdaEmptyExtraSkip]%
%
\>[2]\AgdaFunction{ϕ}\AgdaSpace{}%
\AgdaSymbol{:}\AgdaSpace{}%
\AgdaOperator{\AgdaFunction{∣}}\AgdaSpace{}%
\AgdaBound{𝑪}\AgdaSpace{}%
\AgdaOperator{\AgdaFunction{∣}}\AgdaSpace{}%
\AgdaSymbol{→}\AgdaSpace{}%
\AgdaOperator{\AgdaFunction{∣}}\AgdaSpace{}%
\AgdaBound{𝑩}\AgdaSpace{}%
\AgdaOperator{\AgdaFunction{∣}}\<%
\\
%
\>[2]\AgdaFunction{ϕ}\AgdaSpace{}%
\AgdaSymbol{=}\AgdaSpace{}%
\AgdaSymbol{λ}\AgdaSpace{}%
\AgdaBound{y}\AgdaSpace{}%
\AgdaSymbol{→}\AgdaSpace{}%
\AgdaOperator{\AgdaFunction{∣}}\AgdaSpace{}%
\AgdaBound{β}\AgdaSpace{}%
\AgdaOperator{\AgdaFunction{∣}}\AgdaSpace{}%
\AgdaSymbol{(}\AgdaSpace{}%
\AgdaFunction{γInv}\AgdaSpace{}%
\AgdaBound{y}\AgdaSpace{}%
\AgdaSymbol{)}\<%
\\
%
\\[\AgdaEmptyExtraSkip]%
%
\>[2]\AgdaFunction{ξ}\AgdaSpace{}%
\AgdaSymbol{:}\AgdaSpace{}%
\AgdaSymbol{(}\AgdaBound{x}\AgdaSpace{}%
\AgdaSymbol{:}\AgdaSpace{}%
\AgdaOperator{\AgdaFunction{∣}}\AgdaSpace{}%
\AgdaBound{𝑨}\AgdaSpace{}%
\AgdaOperator{\AgdaFunction{∣}}\AgdaSymbol{)}\AgdaSpace{}%
\AgdaSymbol{→}\AgdaSpace{}%
\AgdaFunction{KER-pred}\AgdaSpace{}%
\AgdaOperator{\AgdaFunction{∣}}\AgdaSpace{}%
\AgdaBound{γ}\AgdaSpace{}%
\AgdaOperator{\AgdaFunction{∣}}\AgdaSpace{}%
\AgdaSymbol{(}\AgdaBound{x}\AgdaSpace{}%
\AgdaOperator{\AgdaInductiveConstructor{,}}\AgdaSpace{}%
\AgdaFunction{γInv}\AgdaSpace{}%
\AgdaSymbol{(}\AgdaOperator{\AgdaFunction{∣}}\AgdaSpace{}%
\AgdaBound{γ}\AgdaSpace{}%
\AgdaOperator{\AgdaFunction{∣}}\AgdaSpace{}%
\AgdaBound{x}\AgdaSymbol{))}\<%
\\
%
\>[2]\AgdaFunction{ξ}\AgdaSpace{}%
\AgdaBound{x}\AgdaSpace{}%
\AgdaSymbol{=}%
\>[9]\AgdaSymbol{(}\AgdaSpace{}%
\AgdaFunction{cong-app}\AgdaSpace{}%
\AgdaSymbol{(}\AgdaFunction{EpicInvIsRightInv}\AgdaSpace{}%
\AgdaBound{gfe}\AgdaSpace{}%
\AgdaOperator{\AgdaFunction{∣}}\AgdaSpace{}%
\AgdaBound{γ}\AgdaSpace{}%
\AgdaOperator{\AgdaFunction{∣}}\AgdaSpace{}%
\AgdaBound{γE}\AgdaSymbol{)}\AgdaSpace{}%
\AgdaSymbol{(}\AgdaSpace{}%
\AgdaOperator{\AgdaFunction{∣}}\AgdaSpace{}%
\AgdaBound{γ}\AgdaSpace{}%
\AgdaOperator{\AgdaFunction{∣}}\AgdaSpace{}%
\AgdaBound{x}\AgdaSpace{}%
\AgdaSymbol{)}\AgdaSpace{}%
\AgdaSymbol{)}\AgdaOperator{\AgdaFunction{⁻¹}}\<%
\\
%
\\[\AgdaEmptyExtraSkip]%
%
\>[2]\AgdaFunction{βϕγ}\AgdaSpace{}%
\AgdaSymbol{:}\AgdaSpace{}%
\AgdaOperator{\AgdaFunction{∣}}\AgdaSpace{}%
\AgdaBound{β}\AgdaSpace{}%
\AgdaOperator{\AgdaFunction{∣}}\AgdaSpace{}%
\AgdaOperator{\AgdaDatatype{≡}}\AgdaSpace{}%
\AgdaFunction{ϕ}\AgdaSpace{}%
\AgdaOperator{\AgdaFunction{∘}}\AgdaSpace{}%
\AgdaOperator{\AgdaFunction{∣}}\AgdaSpace{}%
\AgdaBound{γ}\AgdaSpace{}%
\AgdaOperator{\AgdaFunction{∣}}\<%
\\
%
\>[2]\AgdaFunction{βϕγ}\AgdaSpace{}%
\AgdaSymbol{=}\AgdaSpace{}%
\AgdaBound{gfe}\AgdaSpace{}%
\AgdaSymbol{λ}\AgdaSpace{}%
\AgdaBound{x}\AgdaSpace{}%
\AgdaSymbol{→}\AgdaSpace{}%
\AgdaBound{Kγβ}\AgdaSpace{}%
\AgdaSymbol{(}\AgdaFunction{ξ}\AgdaSpace{}%
\AgdaBound{x}\AgdaSymbol{)}\<%
\\
%
\\[\AgdaEmptyExtraSkip]%
%
\>[2]\AgdaFunction{ι}\AgdaSpace{}%
\AgdaSymbol{:}\AgdaSpace{}%
\AgdaSymbol{(}\AgdaBound{𝑓}\AgdaSpace{}%
\AgdaSymbol{:}\AgdaSpace{}%
\AgdaOperator{\AgdaFunction{∣}}\AgdaSpace{}%
\AgdaBound{𝑆}\AgdaSpace{}%
\AgdaOperator{\AgdaFunction{∣}}\AgdaSymbol{)(}\AgdaBound{𝒄}\AgdaSpace{}%
\AgdaSymbol{:}\AgdaSpace{}%
\AgdaOperator{\AgdaFunction{∥}}\AgdaSpace{}%
\AgdaBound{𝑆}\AgdaSpace{}%
\AgdaOperator{\AgdaFunction{∥}}\AgdaSpace{}%
\AgdaBound{𝑓}\AgdaSpace{}%
\AgdaSymbol{→}\AgdaSpace{}%
\AgdaOperator{\AgdaFunction{∣}}\AgdaSpace{}%
\AgdaBound{𝑪}\AgdaSpace{}%
\AgdaOperator{\AgdaFunction{∣}}\AgdaSymbol{)}\AgdaSpace{}%
\AgdaSymbol{→}\AgdaSpace{}%
\AgdaBound{𝒄}\AgdaSpace{}%
\AgdaOperator{\AgdaDatatype{≡}}%
\>[46]\AgdaOperator{\AgdaFunction{∣}}\AgdaSpace{}%
\AgdaBound{γ}\AgdaSpace{}%
\AgdaOperator{\AgdaFunction{∣}}\AgdaSpace{}%
\AgdaOperator{\AgdaFunction{∘}}\AgdaSpace{}%
\AgdaSymbol{(}\AgdaFunction{γInv}\AgdaSpace{}%
\AgdaOperator{\AgdaFunction{∘}}\AgdaSpace{}%
\AgdaBound{𝒄}\AgdaSymbol{)}\<%
\\
%
\>[2]\AgdaFunction{ι}\AgdaSpace{}%
\AgdaBound{𝑓}\AgdaSpace{}%
\AgdaBound{𝒄}\AgdaSpace{}%
\AgdaSymbol{=}\AgdaSpace{}%
\AgdaFunction{ap}\AgdaSpace{}%
\AgdaSymbol{(λ}\AgdaSpace{}%
\AgdaBound{-}\AgdaSpace{}%
\AgdaSymbol{→}\AgdaSpace{}%
\AgdaBound{-}\AgdaSpace{}%
\AgdaOperator{\AgdaFunction{∘}}\AgdaSpace{}%
\AgdaBound{𝒄}\AgdaSymbol{)(}\AgdaFunction{EpicInvIsRightInv}\AgdaSpace{}%
\AgdaBound{gfe}\AgdaSpace{}%
\AgdaOperator{\AgdaFunction{∣}}\AgdaSpace{}%
\AgdaBound{γ}\AgdaSpace{}%
\AgdaOperator{\AgdaFunction{∣}}\AgdaSpace{}%
\AgdaBound{γE}\AgdaSymbol{)}\AgdaOperator{\AgdaFunction{⁻¹}}\<%
\\
%
\\[\AgdaEmptyExtraSkip]%
%
\>[2]\AgdaFunction{useker}\AgdaSpace{}%
\AgdaSymbol{:}\AgdaSpace{}%
\AgdaSymbol{∀}\AgdaSpace{}%
\AgdaBound{𝑓}\AgdaSpace{}%
\AgdaBound{𝒄}\AgdaSpace{}%
\AgdaSymbol{→}\AgdaSpace{}%
\AgdaOperator{\AgdaFunction{∣}}\AgdaSpace{}%
\AgdaBound{β}\AgdaSpace{}%
\AgdaOperator{\AgdaFunction{∣}}\AgdaSpace{}%
\AgdaSymbol{(}\AgdaFunction{γInv}\AgdaSpace{}%
\AgdaSymbol{(}\AgdaOperator{\AgdaFunction{∣}}\AgdaSpace{}%
\AgdaBound{γ}\AgdaSpace{}%
\AgdaOperator{\AgdaFunction{∣}}\AgdaSpace{}%
\AgdaSymbol{((}\AgdaBound{𝑓}\AgdaSpace{}%
\AgdaOperator{\AgdaFunction{̂}}\AgdaSpace{}%
\AgdaBound{𝑨}\AgdaSymbol{)}\AgdaSpace{}%
\AgdaSymbol{(}\AgdaFunction{γInv}\AgdaSpace{}%
\AgdaOperator{\AgdaFunction{∘}}\AgdaSpace{}%
\AgdaBound{𝒄}\AgdaSymbol{))))}\AgdaSpace{}%
\AgdaOperator{\AgdaDatatype{≡}}\AgdaSpace{}%
\AgdaOperator{\AgdaFunction{∣}}\AgdaSpace{}%
\>[667I]\AgdaBound{β}\AgdaSpace{}%
\AgdaOperator{\AgdaFunction{∣}}\AgdaSymbol{((}\AgdaBound{𝑓}\AgdaSpace{}%
\AgdaOperator{\AgdaFunction{̂}}\AgdaSpace{}%
\AgdaBound{𝑨}\AgdaSymbol{)}\AgdaSpace{}%
\AgdaSymbol{(}\AgdaFunction{γInv}\AgdaSpace{}%
\AgdaOperator{\AgdaFunction{∘}}\AgdaSpace{}%
\AgdaBound{𝒄}\AgdaSymbol{))}\<%
\\
%
\>[2]\AgdaFunction{useker}\AgdaSpace{}%
\AgdaBound{𝑓}\AgdaSpace{}%
\AgdaBound{𝒄}\AgdaSpace{}%
\AgdaSymbol{=}\AgdaSpace{}%
\AgdaBound{Kγβ}\AgdaSpace{}%
\AgdaSymbol{(}\AgdaFunction{cong-app}\AgdaSpace{}%
\AgdaSymbol{(}\AgdaFunction{EpicInvIsRightInv}\AgdaSpace{}%
\AgdaBound{gfe}\AgdaSpace{}%
\AgdaOperator{\AgdaFunction{∣}}\AgdaSpace{}%
\AgdaBound{γ}\AgdaSpace{}%
\AgdaOperator{\AgdaFunction{∣}}\AgdaSpace{}%
\AgdaBound{γE}\AgdaSymbol{)(}\AgdaOperator{\AgdaFunction{∣}}\AgdaSpace{}%
\AgdaBound{γ}\AgdaSpace{}%
\AgdaOperator{\AgdaFunction{∣}}\AgdaSpace{}%
\AgdaSymbol{((}\AgdaBound{𝑓}\AgdaSpace{}%
\AgdaOperator{\AgdaFunction{̂}}\AgdaSpace{}%
\AgdaBound{𝑨}\AgdaSymbol{)(}\AgdaFunction{γInv}\AgdaSpace{}%
\AgdaOperator{\AgdaFunction{∘}}\AgdaSpace{}%
\AgdaBound{𝒄}\AgdaSymbol{))))}\<%
\\
%
\\[\AgdaEmptyExtraSkip]%
%
\>[2]\AgdaFunction{ϕIsHomCB}\AgdaSpace{}%
\AgdaSymbol{:}\AgdaSpace{}%
\AgdaSymbol{∀}\AgdaSpace{}%
\AgdaBound{𝑓}\AgdaSpace{}%
\AgdaBound{𝒄}\AgdaSpace{}%
\AgdaSymbol{→}\AgdaSpace{}%
\AgdaFunction{ϕ}\AgdaSpace{}%
\AgdaSymbol{((}\AgdaBound{𝑓}\AgdaSpace{}%
\AgdaOperator{\AgdaFunction{̂}}\AgdaSpace{}%
\AgdaBound{𝑪}\AgdaSymbol{)}\AgdaSpace{}%
\AgdaBound{𝒄}\AgdaSymbol{)}\AgdaSpace{}%
\AgdaOperator{\AgdaDatatype{≡}}\AgdaSpace{}%
\AgdaSymbol{((}\AgdaBound{𝑓}\AgdaSpace{}%
\AgdaOperator{\AgdaFunction{̂}}\AgdaSpace{}%
\AgdaBound{𝑩}\AgdaSymbol{)(}\AgdaFunction{ϕ}\AgdaSpace{}%
\AgdaOperator{\AgdaFunction{∘}}\AgdaSpace{}%
\AgdaBound{𝒄}\AgdaSymbol{))}\<%
\\
%
% \\[\AgdaEmptyExtraSkip]%
% %
\>[2]\AgdaFunction{ϕIsHomCB}\AgdaSpace{}%
\AgdaBound{𝑓}\AgdaSpace{}%
\AgdaBound{𝒄}%
\>[1662I]\AgdaSymbol{=}\AgdaSpace{}%
\AgdaOperator{\AgdaFunction{∣}}\AgdaSpace{}%
\AgdaBound{β}\AgdaSpace{}%
\AgdaOperator{\AgdaFunction{∣}}\AgdaSpace{}%
\AgdaSymbol{(}\AgdaFunction{γInv}\AgdaSpace{}%
\AgdaSymbol{((}\AgdaBound{𝑓}\AgdaSpace{}%
\AgdaOperator{\AgdaFunction{̂}}\AgdaSpace{}%
\AgdaBound{𝑪}\AgdaSymbol{)}\AgdaSpace{}%
\AgdaBound{𝒄}\AgdaSymbol{))}%
\>[667I][@{}l@{\AgdaIndent{0}}]%
\>[60]\AgdaOperator{\AgdaFunction{≡⟨}}\AgdaSpace{}%
\AgdaFunction{i}%
\>[67]\AgdaOperator{\AgdaFunction{⟩}}\<%
\\
\>[1662I][@{}l@{\AgdaIndent{0}}]%
\>[16]\AgdaOperator{\AgdaFunction{∣}}\AgdaSpace{}%
\AgdaBound{β}\AgdaSpace{}%
\AgdaOperator{\AgdaFunction{∣}}\AgdaSpace{}%
\AgdaSymbol{(}\AgdaFunction{γInv}\AgdaSpace{}%
\AgdaSymbol{((}\AgdaBound{𝑓}\AgdaSpace{}%
\AgdaOperator{\AgdaFunction{̂}}\AgdaSpace{}%
\AgdaBound{𝑪}\AgdaSymbol{)(}\AgdaOperator{\AgdaFunction{∣}}\AgdaSpace{}%
\AgdaBound{γ}\AgdaSpace{}%
\AgdaOperator{\AgdaFunction{∣}}\AgdaSpace{}%
\AgdaOperator{\AgdaFunction{∘}}\AgdaSpace{}%
\AgdaSymbol{(}\AgdaFunction{γInv}\AgdaSpace{}%
\AgdaOperator{\AgdaFunction{∘}}\AgdaSpace{}%
\AgdaBound{𝒄}\AgdaSymbol{))))}\AgdaSpace{}%
\>[60]\AgdaOperator{\AgdaFunction{≡⟨}}\AgdaSpace{}%
\AgdaFunction{ii}%
\>[67]\AgdaOperator{\AgdaFunction{⟩}}\<%
\\
%
\>[16]\AgdaOperator{\AgdaFunction{∣}}\AgdaSpace{}%
\AgdaBound{β}\AgdaSpace{}%
\AgdaOperator{\AgdaFunction{∣}}\AgdaSpace{}%
\AgdaSymbol{(}\AgdaFunction{γInv}\AgdaSpace{}%
\AgdaSymbol{(}\AgdaOperator{\AgdaFunction{∣}}\AgdaSpace{}%
\AgdaBound{γ}\AgdaSpace{}%
\AgdaOperator{\AgdaFunction{∣}}\AgdaSpace{}%
\AgdaSymbol{((}\AgdaBound{𝑓}\AgdaSpace{}%
\AgdaOperator{\AgdaFunction{̂}}\AgdaSpace{}%
\AgdaBound{𝑨}\AgdaSymbol{)(}\AgdaFunction{γInv}\AgdaSpace{}%
\AgdaOperator{\AgdaFunction{∘}}\AgdaSpace{}%
\AgdaBound{𝒄}\AgdaSymbol{))))}%
\>[60]\AgdaOperator{\AgdaFunction{≡⟨}}\AgdaSpace{}%
\AgdaFunction{iii}\AgdaSpace{}%
\>[67]\AgdaOperator{\AgdaFunction{⟩}}\<%
\\
%
\>[16]\AgdaOperator{\AgdaFunction{∣}}\AgdaSpace{}%
\AgdaBound{β}\AgdaSpace{}%
\AgdaOperator{\AgdaFunction{∣}}\AgdaSpace{}%
\AgdaSymbol{((}\AgdaBound{𝑓}\AgdaSpace{}%
\AgdaOperator{\AgdaFunction{̂}}\AgdaSpace{}%
\AgdaBound{𝑨}\AgdaSymbol{)(}\AgdaFunction{γInv}\AgdaSpace{}%
\AgdaOperator{\AgdaFunction{∘}}\AgdaSpace{}%
\AgdaBound{𝒄}\AgdaSymbol{))}%
\>[60]\AgdaOperator{\AgdaFunction{≡⟨}}\AgdaSpace{}%
\AgdaFunction{iv}%
\>[67]\AgdaOperator{\AgdaFunction{⟩}}\<%
\\
%
\>[16]\AgdaSymbol{((}\AgdaBound{𝑓}\AgdaSpace{}%
\AgdaOperator{\AgdaFunction{̂}}\AgdaSpace{}%
\AgdaBound{𝑩}\AgdaSymbol{)(λ}\AgdaSpace{}%
\AgdaBound{x}\AgdaSpace{}%
\AgdaSymbol{→}\AgdaSpace{}%
\AgdaOperator{\AgdaFunction{∣}}\AgdaSpace{}%
\AgdaBound{β}\AgdaSpace{}%
\AgdaOperator{\AgdaFunction{∣}}\AgdaSpace{}%
\AgdaSymbol{(}\AgdaFunction{γInv}\AgdaSpace{}%
\AgdaSymbol{(}\AgdaBound{𝒄}\AgdaSpace{}%
\AgdaBound{x}\AgdaSymbol{))))}%
\>[60]\AgdaOperator{\AgdaFunction{∎}}\<%
\\
\>[2][@{}l@{\AgdaIndent{0}}]%
\>[3]\AgdaKeyword{where}\<%
\\
%
\>[3]\AgdaFunction{i}%
\>[7]\AgdaSymbol{=}\AgdaSpace{}%
\AgdaFunction{ap}\AgdaSpace{}%
\AgdaSymbol{(}\AgdaOperator{\AgdaFunction{∣}}\AgdaSpace{}%
\AgdaBound{β}\AgdaSpace{}%
\AgdaOperator{\AgdaFunction{∣}}\AgdaSpace{}%
\AgdaOperator{\AgdaFunction{∘}}\AgdaSpace{}%
\AgdaFunction{γInv}\AgdaSymbol{)}\AgdaSpace{}%
\AgdaSymbol{(}\AgdaFunction{ap}\AgdaSpace{}%
\AgdaSymbol{(}\AgdaBound{𝑓}\AgdaSpace{}%
\AgdaOperator{\AgdaFunction{̂}}\AgdaSpace{}%
\AgdaBound{𝑪}\AgdaSymbol{)}\AgdaSpace{}%
\AgdaSymbol{(}\AgdaFunction{ι}\AgdaSpace{}%
\AgdaBound{𝑓}\AgdaSpace{}%
\AgdaBound{𝒄}\AgdaSymbol{))}\<%
\\
%
\>[3]\AgdaFunction{ii}%
\>[7]\AgdaSymbol{=}\AgdaSpace{}%
\AgdaFunction{ap}\AgdaSpace{}%
\AgdaSymbol{(}\AgdaOperator{\AgdaFunction{∣}}\AgdaSpace{}%
\AgdaBound{β}\AgdaSpace{}%
\AgdaOperator{\AgdaFunction{∣}}\AgdaSpace{}%
\AgdaOperator{\AgdaFunction{∘}}\AgdaSpace{}%
\AgdaFunction{γInv}\AgdaSymbol{)}\AgdaSpace{}%
\AgdaSymbol{(}\AgdaOperator{\AgdaFunction{∥}}\AgdaSpace{}%
\AgdaBound{γ}\AgdaSpace{}%
\AgdaOperator{\AgdaFunction{∥}}\AgdaSpace{}%
\AgdaBound{𝑓}\AgdaSpace{}%
\AgdaSymbol{(}\AgdaFunction{γInv}\AgdaSpace{}%
\AgdaOperator{\AgdaFunction{∘}}\AgdaSpace{}%
\AgdaBound{𝒄}\AgdaSymbol{))}\AgdaOperator{\AgdaFunction{⁻¹}}\<%
\\
%
\>[3]\AgdaFunction{iii}\AgdaSpace{}%
\AgdaSymbol{=}\AgdaSpace{}%
\AgdaFunction{useker}\AgdaSpace{}%
\AgdaBound{𝑓}\AgdaSpace{}%
\AgdaBound{𝒄}\<%
\\
%
\>[3]\AgdaFunction{iv}%
\>[7]\AgdaSymbol{=}\AgdaSpace{}%
\AgdaOperator{\AgdaFunction{∥}}\AgdaSpace{}%
\AgdaBound{β}\AgdaSpace{}%
\AgdaOperator{\AgdaFunction{∥}}\AgdaSpace{}%
\AgdaBound{𝑓}\AgdaSpace{}%
\AgdaSymbol{(}\AgdaFunction{γInv}\AgdaSpace{}%
\AgdaOperator{\AgdaFunction{∘}}\AgdaSpace{}%
\AgdaBound{𝒄}\AgdaSymbol{)}\<%
\end{code}
\ccpad
If, in addition, both \ab β and \ab γ are epic, then so is \ab ϕ.
\ccpad
\begin{code}%
\>[0][@{}l@{\AgdaIndent{1}}]%
\>[1]\AgdaFunction{HomFactorEpi}\AgdaSpace{}%
\AgdaSymbol{:}%
\>[1755I]\AgdaSymbol{(}\AgdaBound{𝑨}\AgdaSpace{}%
\AgdaSymbol{:}\AgdaSpace{}%
\AgdaFunction{Algebra}\AgdaSpace{}%
\AgdaBound{𝓧}\AgdaSpace{}%
\AgdaBound{𝑆}\AgdaSymbol{)\{}\AgdaBound{𝑩}\AgdaSpace{}%
\AgdaSymbol{:}\AgdaSpace{}%
\AgdaFunction{Algebra}\AgdaSpace{}%
\AgdaBound{𝓨}\AgdaSpace{}%
\AgdaBound{𝑆}\AgdaSymbol{\}\{}\AgdaBound{𝑪}\AgdaSpace{}%
\AgdaSymbol{:}\AgdaSpace{}%
\AgdaFunction{Algebra}\AgdaSpace{}%
\AgdaBound{𝓩}\AgdaSpace{}%
\AgdaBound{𝑆}\AgdaSymbol{\}}\<%
\\
\>[.][@{}l@{}]\<[1755I]%
\>[16]\AgdaSymbol{(}\AgdaBound{β}\AgdaSpace{}%
\AgdaSymbol{:}\AgdaSpace{}%
\AgdaFunction{hom}\AgdaSpace{}%
\AgdaBound{𝑨}\AgdaSpace{}%
\AgdaBound{𝑩}\AgdaSymbol{)}\AgdaSpace{}%
\AgdaSymbol{(}\AgdaBound{βe}\AgdaSpace{}%
\AgdaSymbol{:}\AgdaSpace{}%
\AgdaFunction{Epic}\AgdaSpace{}%
\AgdaOperator{\AgdaFunction{∣}}\AgdaSpace{}%
\AgdaBound{β}\AgdaSpace{}%
\AgdaOperator{\AgdaFunction{∣}}\AgdaSymbol{)}\<%
\\
%
\>[16]\AgdaSymbol{(}\AgdaBound{ξ}\AgdaSpace{}%
\AgdaSymbol{:}\AgdaSpace{}%
\AgdaFunction{hom}\AgdaSpace{}%
\AgdaBound{𝑨}\AgdaSpace{}%
\AgdaBound{𝑪}\AgdaSymbol{)}\AgdaSpace{}%
\AgdaSymbol{(}\AgdaBound{ξe}\AgdaSpace{}%
\AgdaSymbol{:}\AgdaSpace{}%
\AgdaFunction{Epic}\AgdaSpace{}%
\AgdaOperator{\AgdaFunction{∣}}\AgdaSpace{}%
\AgdaBound{ξ}\AgdaSpace{}%
\AgdaOperator{\AgdaFunction{∣}}\AgdaSymbol{)}\<%
\\
\>[1][@{}l@{\AgdaIndent{0}}]%
\>[2]\AgdaSymbol{→}%
\>[16]\AgdaSymbol{(}\AgdaFunction{KER-pred}\AgdaSpace{}%
\AgdaOperator{\AgdaFunction{∣}}\AgdaSpace{}%
\AgdaBound{ξ}\AgdaSpace{}%
\AgdaOperator{\AgdaFunction{∣}}\AgdaSymbol{)}\AgdaSpace{}%
\AgdaOperator{\AgdaFunction{⊆}}\AgdaSpace{}%
\AgdaSymbol{(}\AgdaFunction{KER-pred}\AgdaSpace{}%
\AgdaOperator{\AgdaFunction{∣}}\AgdaSpace{}%
\AgdaBound{β}\AgdaSpace{}%
\AgdaOperator{\AgdaFunction{∣}}\AgdaSymbol{)}\<%
\\
%
% \>[16]\AgdaComment{----------------------------------}\<%
% \\
% %
\>[2]\AgdaSymbol{→}%
\>[16]\AgdaFunction{Σ}\AgdaSpace{}%
\AgdaBound{ϕ}\AgdaSpace{}%
\AgdaFunction{꞉}\AgdaSpace{}%
\AgdaSymbol{(}\AgdaFunction{epi}\AgdaSpace{}%
\AgdaBound{𝑪}\AgdaSpace{}%
\AgdaBound{𝑩}\AgdaSymbol{)}\AgdaSpace{}%
\AgdaFunction{,}\AgdaSpace{}%
\AgdaOperator{\AgdaFunction{∣}}\AgdaSpace{}%
\AgdaBound{β}\AgdaSpace{}%
\AgdaOperator{\AgdaFunction{∣}}\AgdaSpace{}%
\AgdaOperator{\AgdaDatatype{≡}}\AgdaSpace{}%
\AgdaOperator{\AgdaFunction{∣}}\AgdaSpace{}%
\AgdaBound{ϕ}\AgdaSpace{}%
\AgdaOperator{\AgdaFunction{∣}}\AgdaSpace{}%
\AgdaOperator{\AgdaFunction{∘}}\AgdaSpace{}%
\AgdaOperator{\AgdaFunction{∣}}\AgdaSpace{}%
\AgdaBound{ξ}\AgdaSpace{}%
\AgdaOperator{\AgdaFunction{∣}}\<%
\\
%
\\[\AgdaEmptyExtraSkip]%
%
\>[1]\AgdaFunction{HomFactorEpi}\AgdaSpace{}%
\AgdaBound{𝑨}\AgdaSpace{}%
\AgdaSymbol{\{}\AgdaBound{𝑩}\AgdaSymbol{\}\{}\AgdaBound{𝑪}\AgdaSymbol{\}}\AgdaSpace{}%
\AgdaBound{β}\AgdaSpace{}%
\AgdaBound{βe}\AgdaSpace{}%
\AgdaBound{ξ}\AgdaSpace{}%
\AgdaBound{ξe}\AgdaSpace{}%
\AgdaBound{kerincl}\AgdaSpace{}%
\AgdaSymbol{=}\AgdaSpace{}%
\AgdaSymbol{(}\AgdaFunction{fst}\AgdaSpace{}%
\AgdaOperator{\AgdaFunction{∣}}\AgdaSpace{}%
\AgdaFunction{ϕF}\AgdaSpace{}%
\AgdaOperator{\AgdaFunction{∣}}\AgdaSpace{}%
\AgdaOperator{\AgdaInductiveConstructor{,}}\AgdaSpace{}%
\AgdaSymbol{(}\AgdaFunction{snd}\AgdaSpace{}%
\AgdaOperator{\AgdaFunction{∣}}\AgdaSpace{}%
\AgdaFunction{ϕF}\AgdaSpace{}%
\AgdaOperator{\AgdaFunction{∣}}\AgdaSpace{}%
\AgdaOperator{\AgdaInductiveConstructor{,}}\AgdaSpace{}%
\AgdaFunction{ϕE}\AgdaSymbol{))}\AgdaSpace{}%
\AgdaOperator{\AgdaInductiveConstructor{,}}\AgdaSpace{}%
\AgdaOperator{\AgdaFunction{∥}}\AgdaSpace{}%
\AgdaFunction{ϕF}\AgdaSpace{}%
\AgdaOperator{\AgdaFunction{∥}}\<%
\\
\>[1][@{}l@{\AgdaIndent{0}}]%
\>[2]\AgdaKeyword{where}\<%
\\
%
\>[2]\AgdaFunction{ϕF}\AgdaSpace{}%
\AgdaSymbol{:}\AgdaSpace{}%
\AgdaFunction{Σ}\AgdaSpace{}%
\AgdaBound{ϕ}\AgdaSpace{}%
\AgdaFunction{꞉}\AgdaSpace{}%
\AgdaSymbol{(}\AgdaFunction{hom}\AgdaSpace{}%
\AgdaBound{𝑪}\AgdaSpace{}%
\AgdaBound{𝑩}\AgdaSymbol{)}\AgdaSpace{}%
\AgdaFunction{,}\AgdaSpace{}%
\AgdaOperator{\AgdaFunction{∣}}\AgdaSpace{}%
\AgdaBound{β}\AgdaSpace{}%
\AgdaOperator{\AgdaFunction{∣}}\AgdaSpace{}%
\AgdaOperator{\AgdaDatatype{≡}}\AgdaSpace{}%
\AgdaOperator{\AgdaFunction{∣}}\AgdaSpace{}%
\AgdaBound{ϕ}\AgdaSpace{}%
\AgdaOperator{\AgdaFunction{∣}}\AgdaSpace{}%
\AgdaOperator{\AgdaFunction{∘}}\AgdaSpace{}%
\AgdaOperator{\AgdaFunction{∣}}\AgdaSpace{}%
\AgdaBound{ξ}\AgdaSpace{}%
\AgdaOperator{\AgdaFunction{∣}}\<%
\\
%
\>[2]\AgdaFunction{ϕF}\AgdaSpace{}%
\AgdaSymbol{=}\AgdaSpace{}%
\AgdaFunction{HomFactor}%
\>[18]\AgdaBound{𝑨}\AgdaSpace{}%
\AgdaSymbol{\{}\AgdaBound{𝑩}\AgdaSymbol{\}\{}\AgdaBound{𝑪}\AgdaSymbol{\}}\AgdaSpace{}%
\AgdaBound{β}\AgdaSpace{}%
\AgdaBound{ξ}\AgdaSpace{}%
\AgdaBound{ξe}\AgdaSpace{}%
\AgdaBound{kerincl}\<%
\\
%
\\[\AgdaEmptyExtraSkip]%
%
\>[2]\AgdaFunction{ξinv}\AgdaSpace{}%
\AgdaSymbol{:}\AgdaSpace{}%
\AgdaOperator{\AgdaFunction{∣}}\AgdaSpace{}%
\AgdaBound{𝑪}\AgdaSpace{}%
\AgdaOperator{\AgdaFunction{∣}}\AgdaSpace{}%
\AgdaSymbol{→}\AgdaSpace{}%
\AgdaOperator{\AgdaFunction{∣}}\AgdaSpace{}%
\AgdaBound{𝑨}\AgdaSpace{}%
\AgdaOperator{\AgdaFunction{∣}}\<%
\\
%
\>[2]\AgdaFunction{ξinv}\AgdaSpace{}%
\AgdaSymbol{=}\AgdaSpace{}%
\AgdaSymbol{λ}\AgdaSpace{}%
\AgdaBound{c}\AgdaSpace{}%
\AgdaSymbol{→}\AgdaSpace{}%
\AgdaSymbol{(}\AgdaFunction{EpicInv}\AgdaSpace{}%
\AgdaOperator{\AgdaFunction{∣}}\AgdaSpace{}%
\AgdaBound{ξ}\AgdaSpace{}%
\AgdaOperator{\AgdaFunction{∣}}\AgdaSpace{}%
\AgdaBound{ξe}\AgdaSymbol{)}\AgdaSpace{}%
\AgdaBound{c}\<%
\\
%
\\[\AgdaEmptyExtraSkip]%
%
\>[2]\AgdaFunction{βinv}\AgdaSpace{}%
\AgdaSymbol{:}\AgdaSpace{}%
\AgdaOperator{\AgdaFunction{∣}}\AgdaSpace{}%
\AgdaBound{𝑩}\AgdaSpace{}%
\AgdaOperator{\AgdaFunction{∣}}\AgdaSpace{}%
\AgdaSymbol{→}\AgdaSpace{}%
\AgdaOperator{\AgdaFunction{∣}}\AgdaSpace{}%
\AgdaBound{𝑨}\AgdaSpace{}%
\AgdaOperator{\AgdaFunction{∣}}\<%
\\
%
\>[2]\AgdaFunction{βinv}\AgdaSpace{}%
\AgdaSymbol{=}\AgdaSpace{}%
\AgdaSymbol{λ}\AgdaSpace{}%
\AgdaBound{b}\AgdaSpace{}%
\AgdaSymbol{→}\AgdaSpace{}%
\AgdaSymbol{(}\AgdaFunction{EpicInv}\AgdaSpace{}%
\AgdaOperator{\AgdaFunction{∣}}\AgdaSpace{}%
\AgdaBound{β}\AgdaSpace{}%
\AgdaOperator{\AgdaFunction{∣}}\AgdaSpace{}%
\AgdaBound{βe}\AgdaSymbol{)}\AgdaSpace{}%
\AgdaBound{b}\<%
\\
%
\\[\AgdaEmptyExtraSkip]%
%
\>[2]\AgdaFunction{ϕ}\AgdaSpace{}%
\AgdaSymbol{:}\AgdaSpace{}%
\AgdaOperator{\AgdaFunction{∣}}\AgdaSpace{}%
\AgdaBound{𝑪}\AgdaSpace{}%
\AgdaOperator{\AgdaFunction{∣}}\AgdaSpace{}%
\AgdaSymbol{→}\AgdaSpace{}%
\AgdaOperator{\AgdaFunction{∣}}\AgdaSpace{}%
\AgdaBound{𝑩}\AgdaSpace{}%
\AgdaOperator{\AgdaFunction{∣}}\<%
\\
%
\>[2]\AgdaFunction{ϕ}\AgdaSpace{}%
\AgdaSymbol{=}\AgdaSpace{}%
\AgdaSymbol{λ}\AgdaSpace{}%
\AgdaBound{c}\AgdaSpace{}%
\AgdaSymbol{→}\AgdaSpace{}%
\AgdaOperator{\AgdaFunction{∣}}\AgdaSpace{}%
\AgdaBound{β}\AgdaSpace{}%
\AgdaOperator{\AgdaFunction{∣}}\AgdaSpace{}%
\AgdaSymbol{(}\AgdaSpace{}%
\AgdaFunction{ξinv}\AgdaSpace{}%
\AgdaBound{c}\AgdaSpace{}%
\AgdaSymbol{)}\<%
\\
%
\\[\AgdaEmptyExtraSkip]%
%
\>[2]\AgdaFunction{ϕE}\AgdaSpace{}%
\AgdaSymbol{:}\AgdaSpace{}%
\AgdaFunction{Epic}\AgdaSpace{}%
\AgdaFunction{ϕ}\<%
\\
%
\>[2]\AgdaFunction{ϕE}\AgdaSpace{}%
\AgdaSymbol{=}\AgdaSpace{}%
\AgdaFunction{epic-factor}\AgdaSpace{}%
\AgdaBound{gfe}\AgdaSpace{}%
\AgdaOperator{\AgdaFunction{∣}}\AgdaSpace{}%
\AgdaBound{β}\AgdaSpace{}%
\AgdaOperator{\AgdaFunction{∣}}\AgdaSpace{}%
\AgdaOperator{\AgdaFunction{∣}}\AgdaSpace{}%
\AgdaBound{ξ}\AgdaSpace{}%
\AgdaOperator{\AgdaFunction{∣}}\AgdaSpace{}%
\AgdaFunction{ϕ}\AgdaSpace{}%
\AgdaOperator{\AgdaFunction{∥}}\AgdaSpace{}%
\AgdaFunction{ϕF}\AgdaSpace{}%
\AgdaOperator{\AgdaFunction{∥}}\AgdaSpace{}%
\AgdaBound{βe}\<%
\end{code}


\subsection{Isomorphisms}\label{sec:isomorphisms}\firstsentence{\ualibIsomorphisms}{\urlIsomorphisms}
% -*- TeX-master: "ualib-part2.tex" -*-
%%% Local Variables: 
%%% mode: latex
%%% TeX-engine: 'xetex
%%% End: 
Here we formalize the notion of \emph{isomorphism} between algebraic structures.

\subsubsection{Definition of isomorphism}\label{definition-of-isomorphism}

Recall, \ab{f} \AgdaOperator{\AgdaFunction{∼}} \ab g means \ab f and \ab g are   %\textasciitilde{}
\emph{extensionally} (or \emph{point-wise}) \emph{equal}; i.e., \as ∀ \ab x, \ab f \ab x \aod ≡ \ab g \ab x. We use this notion of equality of functions in the following definition of \defn{isomorphism}.
\ccpad
\begin{code}%
\>[0]\AgdaOperator{\AgdaFunction{\AgdaUnderscore{}≅\AgdaUnderscore{}}}\AgdaSpace{}%
\AgdaSymbol{:}\AgdaSpace{}%
\AgdaSymbol{\{}\AgdaBound{𝓤}\AgdaSpace{}%
\AgdaBound{𝓦}\AgdaSpace{}%
\AgdaSymbol{:}\AgdaSpace{}%
\AgdaFunction{Universe}\AgdaSymbol{\}(}\AgdaBound{𝑨}\AgdaSpace{}%
\AgdaSymbol{:}\AgdaSpace{}%
\AgdaFunction{Algebra}\AgdaSpace{}%
\AgdaBound{𝓤}\AgdaSpace{}%
\AgdaBound{𝑆}\AgdaSymbol{)(}\AgdaBound{𝑩}\AgdaSpace{}%
\AgdaSymbol{:}\AgdaSpace{}%
\AgdaFunction{Algebra}\AgdaSpace{}%
\AgdaBound{𝓦}\AgdaSpace{}%
\AgdaBound{𝑆}\AgdaSymbol{)}\AgdaSpace{}%
\AgdaSymbol{→}\AgdaSpace{}%
\AgdaBound{𝓞}\AgdaSpace{}%
\AgdaOperator{\AgdaFunction{⊔}}\AgdaSpace{}%
\AgdaBound{𝓥}\AgdaSpace{}%
\AgdaOperator{\AgdaFunction{⊔}}\AgdaSpace{}%
\AgdaBound{𝓤}\AgdaSpace{}%
\AgdaOperator{\AgdaFunction{⊔}}\AgdaSpace{}%
\AgdaBound{𝓦}\AgdaSpace{}%
\AgdaOperator{\AgdaFunction{̇}}\<%
\\
\>[0]\AgdaBound{𝑨}\AgdaSpace{}%
\AgdaOperator{\AgdaFunction{≅}}\AgdaSpace{}%
\AgdaBound{𝑩}\AgdaSpace{}%
\AgdaSymbol{=}%
\>[9]\AgdaFunction{Σ}\AgdaSpace{}%
\AgdaBound{f}\AgdaSpace{}%
\AgdaFunction{꞉}\AgdaSpace{}%
\AgdaSymbol{(}\AgdaFunction{hom}\AgdaSpace{}%
\AgdaBound{𝑨}\AgdaSpace{}%
\AgdaBound{𝑩}\AgdaSymbol{)}\AgdaSpace{}%
\AgdaFunction{,}\AgdaSpace{}%
\AgdaFunction{Σ}\AgdaSpace{}%
\AgdaBound{g}\AgdaSpace{}%
\AgdaFunction{꞉}\AgdaSpace{}%
\AgdaSymbol{(}\AgdaFunction{hom}\AgdaSpace{}%
\AgdaBound{𝑩}\AgdaSpace{}%
\AgdaBound{𝑨}\AgdaSymbol{)}%
\>[73I]\AgdaFunction{,}\AgdaSpace{}%
\AgdaSymbol{(}\AgdaOperator{\AgdaFunction{∣}}\AgdaSpace{}%
\AgdaBound{f}\AgdaSpace{}%
\AgdaOperator{\AgdaFunction{∣}}\AgdaSpace{}%
\AgdaOperator{\AgdaFunction{∘}}\AgdaSpace{}%
\AgdaOperator{\AgdaFunction{∣}}\AgdaSpace{}%
\AgdaBound{g}\AgdaSpace{}%
\AgdaOperator{\AgdaFunction{∣}}\AgdaSpace{}%
\AgdaOperator{\AgdaFunction{∼}}\AgdaSpace{}%
\AgdaOperator{\AgdaFunction{∣}}\AgdaSpace{}%
\AgdaFunction{𝒾𝒹}\AgdaSpace{}%
\AgdaBound{𝑩}\AgdaSpace{}%
\AgdaOperator{\AgdaFunction{∣}}\AgdaSymbol{)}\AgdaSpace{}%
\AgdaOperator{\AgdaFunction{×}}\AgdaSpace{}%
\AgdaSymbol{(}\AgdaOperator{\AgdaFunction{∣}}\AgdaSpace{}%
\AgdaBound{g}\AgdaSpace{}%
\AgdaOperator{\AgdaFunction{∣}}\AgdaSpace{}%
\AgdaOperator{\AgdaFunction{∘}}\AgdaSpace{}%
\AgdaOperator{\AgdaFunction{∣}}\AgdaSpace{}%
\AgdaBound{f}\AgdaSpace{}%
\AgdaOperator{\AgdaFunction{∣}}\AgdaSpace{}%
\AgdaOperator{\AgdaFunction{∼}}\AgdaSpace{}%
\AgdaOperator{\AgdaFunction{∣}}\AgdaSpace{}%
\AgdaFunction{𝒾𝒹}\AgdaSpace{}%
\AgdaBound{𝑨}\AgdaSpace{}%
\AgdaOperator{\AgdaFunction{∣}}\AgdaSymbol{)}\<%
\end{code}
\ccpad
That is, two structures are \defn{isomorphic} provided there are homomorphisms going back and forth between them which compose to the identity.

\subsubsection{Isomorphism is an equivalence relation}\label{isomorphism-is-an-equivalence-relation}

\begin{code}%
\>[0]\AgdaFunction{≅-refl}\AgdaSpace{}%
\AgdaSymbol{:}\AgdaSpace{}%
\AgdaSymbol{\{}\AgdaBound{𝓤}\AgdaSpace{}%
\AgdaSymbol{:}\AgdaSpace{}%
\AgdaFunction{Universe}\AgdaSymbol{\}}\AgdaSpace{}%
\AgdaSymbol{\{}\AgdaBound{𝑨}\AgdaSpace{}%
\AgdaSymbol{:}\AgdaSpace{}%
\AgdaFunction{Algebra}\AgdaSpace{}%
\AgdaBound{𝓤}\AgdaSpace{}%
\AgdaBound{𝑆}\AgdaSymbol{\}}\AgdaSpace{}%
\AgdaSymbol{→}\AgdaSpace{}%
\AgdaBound{𝑨}\AgdaSpace{}%
\AgdaOperator{\AgdaFunction{≅}}\AgdaSpace{}%
\AgdaBound{𝑨}\<%
\\
\>[0]\AgdaFunction{≅-refl}\AgdaSpace{}%
\AgdaSymbol{\{}\AgdaBound{𝓤}\AgdaSymbol{\}\{}\AgdaBound{𝑨}\AgdaSymbol{\}}\AgdaSpace{}%
\AgdaSymbol{=}\AgdaSpace{}%
\AgdaFunction{𝒾𝒹}\AgdaSpace{}%
\AgdaBound{𝑨}\AgdaSpace{}%
\AgdaOperator{\AgdaInductiveConstructor{,}}\AgdaSpace{}%
\AgdaFunction{𝒾𝒹}\AgdaSpace{}%
\AgdaBound{𝑨}\AgdaSpace{}%
\AgdaOperator{\AgdaInductiveConstructor{,}}\AgdaSpace{}%
\AgdaSymbol{(λ}\AgdaSpace{}%
\AgdaBound{a}\AgdaSpace{}%
\AgdaSymbol{→}\AgdaSpace{}%
\AgdaInductiveConstructor{𝓇ℯ𝒻𝓁}\AgdaSymbol{)}\AgdaSpace{}%
\AgdaOperator{\AgdaInductiveConstructor{,}}\AgdaSpace{}%
\AgdaSymbol{(λ}\AgdaSpace{}%
\AgdaBound{a}\AgdaSpace{}%
\AgdaSymbol{→}\AgdaSpace{}%
\AgdaInductiveConstructor{𝓇ℯ𝒻𝓁}\AgdaSymbol{)}\<%
\end{code}
\scpad
\begin{code}%
\>[0]\AgdaFunction{≅-sym}\AgdaSpace{}%
\AgdaSymbol{:}\AgdaSpace{}%
\AgdaSymbol{\{}\AgdaBound{𝓤}\AgdaSpace{}%
\AgdaBound{𝓦}\AgdaSpace{}%
\AgdaSymbol{:}\AgdaSpace{}%
\AgdaFunction{Universe}\AgdaSymbol{\}\{}\AgdaBound{𝑨}\AgdaSpace{}%
\AgdaSymbol{:}\AgdaSpace{}%
\AgdaFunction{Algebra}\AgdaSpace{}%
\AgdaBound{𝓤}\AgdaSpace{}%
\AgdaBound{𝑆}\AgdaSymbol{\}\{}\AgdaBound{𝑩}\AgdaSpace{}%
\AgdaSymbol{:}\AgdaSpace{}%
\AgdaFunction{Algebra}\AgdaSpace{}%
\AgdaBound{𝓦}\AgdaSpace{}%
\AgdaBound{𝑆}\AgdaSymbol{\}}\AgdaSpace{}%
\AgdaSymbol{→}\AgdaSpace{}%
\AgdaBound{𝑨}\AgdaSpace{}%
\AgdaOperator{\AgdaFunction{≅}}\AgdaSpace{}%
\AgdaBound{𝑩}\AgdaSpace{}%
\AgdaSymbol{→}\AgdaSpace{}%
\AgdaBound{𝑩}\AgdaSpace{}%
\AgdaOperator{\AgdaFunction{≅}}\AgdaSpace{}%
\AgdaBound{𝑨}\<%
\\
\>[0]\AgdaFunction{≅-sym}\AgdaSpace{}%
\AgdaBound{h}\AgdaSpace{}%
\AgdaSymbol{=}\AgdaSpace{}%
\AgdaFunction{fst}\AgdaSpace{}%
\AgdaOperator{\AgdaFunction{∥}}\AgdaSpace{}%
\AgdaBound{h}\AgdaSpace{}%
\AgdaOperator{\AgdaFunction{∥}}\AgdaSpace{}%
\AgdaOperator{\AgdaInductiveConstructor{,}}\AgdaSpace{}%
\AgdaFunction{fst}\AgdaSpace{}%
\AgdaBound{h}\AgdaSpace{}%
\AgdaOperator{\AgdaInductiveConstructor{,}}\AgdaSpace{}%
\AgdaOperator{\AgdaFunction{∥}}\AgdaSpace{}%
\AgdaFunction{snd}\AgdaSpace{}%
\AgdaOperator{\AgdaFunction{∥}}\AgdaSpace{}%
\AgdaBound{h}\AgdaSpace{}%
\AgdaOperator{\AgdaFunction{∥}}\AgdaSpace{}%
\AgdaOperator{\AgdaFunction{∥}}\AgdaSpace{}%
\AgdaOperator{\AgdaInductiveConstructor{,}}\AgdaSpace{}%
\AgdaOperator{\AgdaFunction{∣}}\AgdaSpace{}%
\AgdaFunction{snd}\AgdaSpace{}%
\AgdaOperator{\AgdaFunction{∥}}\AgdaSpace{}%
\AgdaBound{h}\AgdaSpace{}%
\AgdaOperator{\AgdaFunction{∥}}\AgdaSpace{}%
\AgdaOperator{\AgdaFunction{∣}}\<%
\end{code}
\scpad
\begin{code}%
\>[1]\AgdaFunction{≅-trans}\AgdaSpace{}%
\AgdaSymbol{:}\AgdaSpace{}%
\>[378I]\AgdaSymbol{\{}\AgdaBound{𝑨}\AgdaSpace{}%
\AgdaSymbol{:}\AgdaSpace{}%
\AgdaFunction{Algebra}\AgdaSpace{}%
\AgdaBound{𝓧}\AgdaSpace{}%
\AgdaBound{𝑆}\AgdaSymbol{\}\{}\AgdaBound{𝑩}\AgdaSpace{}%
\AgdaSymbol{:}\AgdaSpace{}%
\AgdaFunction{Algebra}\AgdaSpace{}%
\AgdaBound{𝓨}\AgdaSpace{}%
\AgdaBound{𝑆}\AgdaSymbol{\}\{}\AgdaBound{𝑪}\AgdaSpace{}%
\AgdaSymbol{:}\AgdaSpace{}%
\AgdaFunction{Algebra}\AgdaSpace{}%
\AgdaBound{𝓩}\AgdaSpace{}%
\AgdaBound{𝑆}\AgdaSymbol{\}}\<%
\\
\>[1][@{}l@{\AgdaIndent{0}}]%
\>[2]\AgdaSymbol{→}%
\>[.][@{}l@{}]\<[378I]%
\>[11]\AgdaBound{𝑨}\AgdaSpace{}%
\AgdaOperator{\AgdaFunction{≅}}\AgdaSpace{}%
\AgdaBound{𝑩}\AgdaSpace{}%
\AgdaSymbol{→}\AgdaSpace{}%
\AgdaBound{𝑩}\AgdaSpace{}%
\AgdaOperator{\AgdaFunction{≅}}\AgdaSpace{}%
\AgdaBound{𝑪}\AgdaSpace{}%
\AgdaSymbol{→}\AgdaSpace{}%
\AgdaBound{𝑨}\AgdaSpace{}%
\AgdaOperator{\AgdaFunction{≅}}\AgdaSpace{}%
\AgdaBound{𝑪}\<%
\\
%
\\[\AgdaEmptyExtraSkip]%
%
\>[1]\AgdaFunction{≅-trans}\AgdaSpace{}%
\AgdaSymbol{\{}\AgdaBound{𝑨}\AgdaSymbol{\}}\AgdaSpace{}%
\AgdaSymbol{\{}\AgdaBound{𝑩}\AgdaSymbol{\}\{}\AgdaBound{𝑪}\AgdaSymbol{\}}\AgdaSpace{}%
\AgdaBound{ab}\AgdaSpace{}%
\AgdaBound{bc}\AgdaSpace{}%
\AgdaSymbol{=}\AgdaSpace{}%
\AgdaFunction{f}\AgdaSpace{}%
\AgdaOperator{\AgdaInductiveConstructor{,}}\AgdaSpace{}%
\AgdaFunction{g}\AgdaSpace{}%
\AgdaOperator{\AgdaInductiveConstructor{,}}\AgdaSpace{}%
\AgdaFunction{α}\AgdaSpace{}%
\AgdaOperator{\AgdaInductiveConstructor{,}}\AgdaSpace{}%
\AgdaFunction{β}\<%
\\
\>[1][@{}l@{\AgdaIndent{0}}]%
\>[2]\AgdaKeyword{where}\<%
\\
%
\>[2]\AgdaFunction{f1}\AgdaSpace{}%
\AgdaSymbol{:}\AgdaSpace{}%
\AgdaFunction{hom}\AgdaSpace{}%
\AgdaBound{𝑨}\AgdaSpace{}%
\AgdaBound{𝑩}\<%
\\
%
\>[2]\AgdaFunction{f1}\AgdaSpace{}%
\AgdaSymbol{=}\AgdaSpace{}%
\AgdaOperator{\AgdaFunction{∣}}\AgdaSpace{}%
\AgdaBound{ab}\AgdaSpace{}%
\AgdaOperator{\AgdaFunction{∣}}\<%
\\
%
\>[2]\AgdaFunction{f2}\AgdaSpace{}%
\AgdaSymbol{:}\AgdaSpace{}%
\AgdaFunction{hom}\AgdaSpace{}%
\AgdaBound{𝑩}\AgdaSpace{}%
\AgdaBound{𝑪}\<%
\\
%
\>[2]\AgdaFunction{f2}\AgdaSpace{}%
\AgdaSymbol{=}\AgdaSpace{}%
\AgdaOperator{\AgdaFunction{∣}}\AgdaSpace{}%
\AgdaBound{bc}\AgdaSpace{}%
\AgdaOperator{\AgdaFunction{∣}}\<%
\\
%
\>[2]\AgdaFunction{f}\AgdaSpace{}%
\AgdaSymbol{:}\AgdaSpace{}%
\AgdaFunction{hom}\AgdaSpace{}%
\AgdaBound{𝑨}\AgdaSpace{}%
\AgdaBound{𝑪}\<%
\\
%
\>[2]\AgdaFunction{f}\AgdaSpace{}%
\AgdaSymbol{=}\AgdaSpace{}%
\AgdaFunction{∘-hom}\AgdaSpace{}%
\AgdaBound{𝑨}\AgdaSpace{}%
\AgdaBound{𝑪}\AgdaSpace{}%
\AgdaFunction{f1}\AgdaSpace{}%
\AgdaFunction{f2}\<%
\\
%
\\[\AgdaEmptyExtraSkip]%
%
\>[2]\AgdaFunction{g1}\AgdaSpace{}%
\AgdaSymbol{:}\AgdaSpace{}%
\AgdaFunction{hom}\AgdaSpace{}%
\AgdaBound{𝑪}\AgdaSpace{}%
\AgdaBound{𝑩}\<%
\\
%
\>[2]\AgdaFunction{g1}\AgdaSpace{}%
\AgdaSymbol{=}\AgdaSpace{}%
\AgdaFunction{fst}\AgdaSpace{}%
\AgdaOperator{\AgdaFunction{∥}}\AgdaSpace{}%
\AgdaBound{bc}\AgdaSpace{}%
\AgdaOperator{\AgdaFunction{∥}}\<%
\\
%
\>[2]\AgdaFunction{g2}\AgdaSpace{}%
\AgdaSymbol{:}\AgdaSpace{}%
\AgdaFunction{hom}\AgdaSpace{}%
\AgdaBound{𝑩}\AgdaSpace{}%
\AgdaBound{𝑨}\<%
\\
%
\>[2]\AgdaFunction{g2}\AgdaSpace{}%
\AgdaSymbol{=}\AgdaSpace{}%
\AgdaFunction{fst}\AgdaSpace{}%
\AgdaOperator{\AgdaFunction{∥}}\AgdaSpace{}%
\AgdaBound{ab}\AgdaSpace{}%
\AgdaOperator{\AgdaFunction{∥}}\<%
\\
%
\>[2]\AgdaFunction{g}\AgdaSpace{}%
\AgdaSymbol{:}\AgdaSpace{}%
\AgdaFunction{hom}\AgdaSpace{}%
\AgdaBound{𝑪}\AgdaSpace{}%
\AgdaBound{𝑨}\<%
\\
%
\>[2]\AgdaFunction{g}\AgdaSpace{}%
\AgdaSymbol{=}\AgdaSpace{}%
\AgdaFunction{∘-hom}\AgdaSpace{}%
\AgdaBound{𝑪}\AgdaSpace{}%
\AgdaBound{𝑨}\AgdaSpace{}%
\AgdaFunction{g1}\AgdaSpace{}%
\AgdaFunction{g2}\<%
\\
%
\\[\AgdaEmptyExtraSkip]%
%
\>[2]\AgdaFunction{α}\AgdaSpace{}%
\AgdaSymbol{:}\AgdaSpace{}%
\AgdaOperator{\AgdaFunction{∣}}\AgdaSpace{}%
\AgdaFunction{f}\AgdaSpace{}%
\AgdaOperator{\AgdaFunction{∣}}\AgdaSpace{}%
\AgdaOperator{\AgdaFunction{∘}}\AgdaSpace{}%
\AgdaOperator{\AgdaFunction{∣}}\AgdaSpace{}%
\AgdaFunction{g}\AgdaSpace{}%
\AgdaOperator{\AgdaFunction{∣}}\AgdaSpace{}%
\AgdaOperator{\AgdaFunction{∼}}\AgdaSpace{}%
\AgdaOperator{\AgdaFunction{∣}}\AgdaSpace{}%
\AgdaFunction{𝒾𝒹}\AgdaSpace{}%
\AgdaBound{𝑪}\AgdaSpace{}%
\AgdaOperator{\AgdaFunction{∣}}\<%
\\
%
\>[2]\AgdaFunction{α}\AgdaSpace{}%
\AgdaBound{x}\AgdaSpace{}%
\AgdaSymbol{=}\AgdaSpace{}%
\AgdaSymbol{(}\AgdaFunction{ap}\AgdaSpace{}%
\AgdaOperator{\AgdaFunction{∣}}\AgdaSpace{}%
\AgdaFunction{f2}\AgdaSpace{}%
\AgdaOperator{\AgdaFunction{∣}}\AgdaSymbol{(}\AgdaOperator{\AgdaFunction{∣}}\AgdaSpace{}%
\AgdaFunction{snd}\AgdaSpace{}%
\AgdaOperator{\AgdaFunction{∥}}\AgdaSpace{}%
\AgdaBound{ab}\AgdaSpace{}%
\AgdaOperator{\AgdaFunction{∥}}\AgdaSpace{}%
\AgdaOperator{\AgdaFunction{∣}}\AgdaSpace{}%
\AgdaSymbol{(}\AgdaOperator{\AgdaFunction{∣}}\AgdaSpace{}%
\AgdaFunction{g1}\AgdaSpace{}%
\AgdaOperator{\AgdaFunction{∣}}\AgdaSpace{}%
\AgdaBound{x}\AgdaSymbol{)))}\AgdaOperator{\AgdaFunction{∙}}\AgdaSymbol{(}\AgdaOperator{\AgdaFunction{∣}}\AgdaSpace{}%
\AgdaFunction{snd}\AgdaSpace{}%
\AgdaOperator{\AgdaFunction{∥}}\AgdaSpace{}%
\AgdaBound{bc}\AgdaSpace{}%
\AgdaOperator{\AgdaFunction{∥}}\AgdaSpace{}%
\AgdaOperator{\AgdaFunction{∣}}\AgdaSymbol{)}\AgdaSpace{}%
\AgdaBound{x}\<%
\\
%
\\[\AgdaEmptyExtraSkip]%
%
\>[2]\AgdaFunction{β}\AgdaSpace{}%
\AgdaSymbol{:}\AgdaSpace{}%
\AgdaOperator{\AgdaFunction{∣}}\AgdaSpace{}%
\AgdaFunction{g}\AgdaSpace{}%
\AgdaOperator{\AgdaFunction{∣}}\AgdaSpace{}%
\AgdaOperator{\AgdaFunction{∘}}\AgdaSpace{}%
\AgdaOperator{\AgdaFunction{∣}}\AgdaSpace{}%
\AgdaFunction{f}\AgdaSpace{}%
\AgdaOperator{\AgdaFunction{∣}}\AgdaSpace{}%
\AgdaOperator{\AgdaFunction{∼}}\AgdaSpace{}%
\AgdaOperator{\AgdaFunction{∣}}\AgdaSpace{}%
\AgdaFunction{𝒾𝒹}\AgdaSpace{}%
\AgdaBound{𝑨}\AgdaSpace{}%
\AgdaOperator{\AgdaFunction{∣}}\<%
\\
%
\>[2]\AgdaFunction{β}\AgdaSpace{}%
\AgdaBound{x}\AgdaSpace{}%
\AgdaSymbol{=}\AgdaSpace{}%
\AgdaSymbol{(}\AgdaFunction{ap}\AgdaSpace{}%
\AgdaOperator{\AgdaFunction{∣}}\AgdaSpace{}%
\AgdaFunction{g2}\AgdaSpace{}%
\AgdaOperator{\AgdaFunction{∣}}\AgdaSymbol{(}\AgdaOperator{\AgdaFunction{∥}}\AgdaSpace{}%
\AgdaFunction{snd}\AgdaSpace{}%
\AgdaOperator{\AgdaFunction{∥}}\AgdaSpace{}%
\AgdaBound{bc}\AgdaSpace{}%
\AgdaOperator{\AgdaFunction{∥}}\AgdaSpace{}%
\AgdaOperator{\AgdaFunction{∥}}\AgdaSpace{}%
\AgdaSymbol{(}\AgdaOperator{\AgdaFunction{∣}}\AgdaSpace{}%
\AgdaFunction{f1}\AgdaSpace{}%
\AgdaOperator{\AgdaFunction{∣}}\AgdaSpace{}%
\AgdaBound{x}\AgdaSymbol{)))}\AgdaOperator{\AgdaFunction{∙}}\AgdaSymbol{(}\AgdaOperator{\AgdaFunction{∥}}\AgdaSpace{}%
\AgdaFunction{snd}\AgdaSpace{}%
\AgdaOperator{\AgdaFunction{∥}}\AgdaSpace{}%
\AgdaBound{ab}\AgdaSpace{}%
\AgdaOperator{\AgdaFunction{∥}}\AgdaSpace{}%
\AgdaOperator{\AgdaFunction{∥}}\AgdaSymbol{)}\AgdaSpace{}%
\AgdaBound{x}\<%
\end{code}
\ccpad
To make \af{trans-≅} easier to apply in certain situations, we define a couple of alternatives where the only difference is which arguments are implicit.
\ccpad
\begin{code}%
\>[1]\AgdaFunction{TRANS-≅}\AgdaSpace{}%
\AgdaSymbol{:}\AgdaSpace{}%
\>[199I]\AgdaSymbol{\{}\AgdaBound{𝑨}\AgdaSpace{}%
\AgdaSymbol{:}\AgdaSpace{}%
\AgdaFunction{Algebra}\AgdaSpace{}%
\AgdaBound{𝓧}\AgdaSpace{}%
\AgdaBound{𝑆}\AgdaSymbol{\}\{}\AgdaBound{𝑩}\AgdaSpace{}%
\AgdaSymbol{:}\AgdaSpace{}%
\AgdaFunction{Algebra}\AgdaSpace{}%
\AgdaBound{𝓨}\AgdaSpace{}%
\AgdaBound{𝑆}\AgdaSymbol{\}\{}\AgdaBound{𝑪}\AgdaSpace{}%
\AgdaSymbol{:}\AgdaSpace{}%
\AgdaFunction{Algebra}\AgdaSpace{}%
\AgdaBound{𝓩}\AgdaSpace{}%
\AgdaBound{𝑆}\AgdaSymbol{\}}\<%
\\
\>[1][@{}l@{\AgdaIndent{0}}]%
\>[2]\AgdaSymbol{→}%
\>[.][@{}l@{}]\<[199I]%
\>[11]\AgdaBound{𝑨}\AgdaSpace{}%
\AgdaOperator{\AgdaFunction{≅}}\AgdaSpace{}%
\AgdaBound{𝑩}\AgdaSpace{}%
\AgdaSymbol{→}\AgdaSpace{}%
\AgdaBound{𝑩}\AgdaSpace{}%
\AgdaOperator{\AgdaFunction{≅}}\AgdaSpace{}%
\AgdaBound{𝑪}\AgdaSpace{}%
\AgdaSymbol{→}\AgdaSpace{}%
\AgdaBound{𝑨}\AgdaSpace{}%
\AgdaOperator{\AgdaFunction{≅}}\AgdaSpace{}%
\AgdaBound{𝑪}\<%
\\
%
\>[1]\AgdaFunction{TRANS-≅}\AgdaSpace{}%
\AgdaSymbol{\{}\AgdaBound{𝑨}\AgdaSymbol{\}\{}\AgdaBound{𝑩}\AgdaSymbol{\}\{}\AgdaBound{𝑪}\AgdaSymbol{\}}\AgdaSpace{}%
\AgdaSymbol{=}\AgdaSpace{}%
\AgdaFunction{trans-≅}\AgdaSpace{}%
\AgdaBound{𝑨}\AgdaSpace{}%
\AgdaBound{𝑩}\AgdaSpace{}%
\AgdaBound{𝑪}\<%
\end{code}
\scpad
\begin{code}
\>[1]\AgdaFunction{Trans-≅}\AgdaSpace{}%
\AgdaSymbol{:}\AgdaSpace{}%
\>[22I]\AgdaSymbol{(}\AgdaBound{𝑨}\AgdaSpace{}%
\AgdaSymbol{:}\AgdaSpace{}%
\AgdaFunction{Algebra}\AgdaSpace{}%
\AgdaBound{𝓧}\AgdaSpace{}%
\AgdaBound{𝑆}\AgdaSymbol{)\{}\AgdaBound{𝑩}\AgdaSpace{}%
\AgdaSymbol{:}\AgdaSpace{}%
\AgdaFunction{Algebra}\AgdaSpace{}%
\AgdaBound{𝓨}\AgdaSpace{}%
\AgdaBound{𝑆}\AgdaSymbol{\}(}\AgdaBound{𝑪}\AgdaSpace{}%
\AgdaSymbol{:}\AgdaSpace{}%
\AgdaFunction{Algebra}\AgdaSpace{}%
\AgdaBound{𝓩}\AgdaSpace{}%
\AgdaBound{𝑆}\AgdaSymbol{)}\<%
\\
\>[1][@{}l@{\AgdaIndent{0}}]%
\>[2]\AgdaSymbol{→}%
\>[.][@{}l@{}]\<[22I]%
\>[11]\AgdaBound{𝑨}\AgdaSpace{}%
\AgdaOperator{\AgdaFunction{≅}}\AgdaSpace{}%
\AgdaBound{𝑩}\AgdaSpace{}%
\AgdaSymbol{→}\AgdaSpace{}%
\AgdaBound{𝑩}\AgdaSpace{}%
\AgdaOperator{\AgdaFunction{≅}}\AgdaSpace{}%
\AgdaBound{𝑪}\AgdaSpace{}%
\AgdaSymbol{→}\AgdaSpace{}%
\AgdaBound{𝑨}\AgdaSpace{}%
\AgdaOperator{\AgdaFunction{≅}}\AgdaSpace{}%
\AgdaBound{𝑪}\<%
\\
%
\>[1]\AgdaFunction{Trans-≅}\AgdaSpace{}%
\AgdaBound{𝑨}\AgdaSpace{}%
\AgdaSymbol{\{}\AgdaBound{𝑩}\AgdaSymbol{\}}\AgdaSpace{}%
\AgdaBound{𝑪}\AgdaSpace{}%
\AgdaSymbol{=}\AgdaSpace{}%
\AgdaFunction{trans-≅}\AgdaSpace{}%
\AgdaBound{𝑨}\AgdaSpace{}%
\AgdaBound{𝑩}\AgdaSpace{}%
\AgdaBound{𝑪}\<%
\end{code}

\subsubsection{Lift is an algebraic invariant}\label{lift-is-an-algebraic-invariant}

Fortunately, the lift operation preserves isomorphism (i.e., it's an \defn{algebraic invariant}). As algebra is our main focus, this invariance of the lift operation is what makes it a workable solution to the technical problems that arise from the noncumulativity of the universe hierarchy discussed in \ualibLifts~\cite[\S2.5]{DeMeo:2021-1}.
\ccpad
\begin{code}%
\>[0]\AgdaKeyword{open}\AgdaSpace{}%
\AgdaModule{Lift}\<%
\\
%
\\[\AgdaEmptyExtraSkip]%
\>[1]\AgdaFunction{lift-alg-≅}\AgdaSpace{}%
\AgdaSymbol{:}\AgdaSpace{}%
\AgdaSymbol{\{}\AgdaBound{𝑨}\AgdaSpace{}%
\AgdaSymbol{:}\AgdaSpace{}%
\AgdaFunction{Algebra}\AgdaSpace{}%
\AgdaBound{𝓤}\AgdaSpace{}%
\AgdaBound{𝑆}\AgdaSymbol{\}}\AgdaSpace{}%
\AgdaSymbol{→}\AgdaSpace{}%
\AgdaBound{𝑨}\AgdaSpace{}%
\AgdaOperator{\AgdaFunction{≅}}\AgdaSpace{}%
\AgdaSymbol{(}\AgdaFunction{lift-alg}\AgdaSpace{}%
\AgdaBound{𝑨}\AgdaSpace{}%
\AgdaBound{𝓦}\AgdaSymbol{)}\<%
\\
%
\>[1]\AgdaFunction{lift-alg-≅}\AgdaSpace{}%
\AgdaSymbol{\{}\AgdaBound{𝑨}\AgdaSymbol{\}}\AgdaSpace{}%
\AgdaSymbol{=}\AgdaSpace{}%
\AgdaFunction{𝓁𝒾𝒻𝓉}\AgdaSpace{}%
\AgdaOperator{\AgdaInductiveConstructor{,}}\AgdaSpace{}%
\AgdaFunction{𝓁ℴ𝓌ℯ𝓇}\AgdaSpace{}%
\AgdaBound{𝑨}\AgdaSpace{}%
\AgdaOperator{\AgdaInductiveConstructor{,}}\AgdaSpace{}%
\AgdaFunction{extfun}\AgdaSpace{}%
\AgdaFunction{lift∼lower}\AgdaSpace{}%
\AgdaOperator{\AgdaInductiveConstructor{,}}\AgdaSpace{}%
\AgdaFunction{extfun}\AgdaSpace{}%
\AgdaSymbol{(}\AgdaFunction{lower∼lift}\AgdaSymbol{\{}\AgdaBound{𝓦}\AgdaSymbol{\})}\<%
\\
%
\\[\AgdaEmptyExtraSkip]%
%
\\[\AgdaEmptyExtraSkip]%
%
\>[1]\AgdaFunction{lift-alg-hom}\AgdaSpace{}%
\AgdaSymbol{:}\AgdaSpace{}%
\>[202I]\AgdaSymbol{(}\AgdaBound{𝓧}\AgdaSpace{}%
\AgdaSymbol{:}\AgdaSpace{}%
\AgdaFunction{Universe}\AgdaSymbol{)(}\AgdaBound{𝓨}\AgdaSpace{}%
\AgdaSymbol{:}\AgdaSpace{}%
\AgdaFunction{Universe}\AgdaSymbol{)\{}\AgdaBound{𝑨}\AgdaSpace{}%
\AgdaSymbol{:}\AgdaSpace{}%
\AgdaFunction{Algebra}\AgdaSpace{}%
\AgdaBound{𝓤}\AgdaSpace{}%
\AgdaBound{𝑆}\AgdaSymbol{\}(}\AgdaBound{𝑩}\AgdaSpace{}%
\AgdaSymbol{:}\AgdaSpace{}%
\AgdaFunction{Algebra}\AgdaSpace{}%
\AgdaBound{𝓦}\AgdaSpace{}%
\AgdaBound{𝑆}\AgdaSymbol{)}\<%
\\
\>[1][@{}l@{\AgdaIndent{0}}]%
\>[2]\AgdaSymbol{→}%
\AgdaTab{202}
\>[16]\AgdaFunction{hom}\AgdaSpace{}%
\AgdaBound{𝑨}\AgdaSpace{}%
\AgdaBound{𝑩}%
\>[25]\AgdaSymbol{→}%
\>[28]\AgdaFunction{hom}\AgdaSpace{}%
\AgdaSymbol{(}\AgdaFunction{lift-alg}\AgdaSpace{}%
\AgdaBound{𝑨}\AgdaSpace{}%
\AgdaBound{𝓧}\AgdaSymbol{)}\AgdaSpace{}%
\AgdaSymbol{(}\AgdaFunction{lift-alg}\AgdaSpace{}%
\AgdaBound{𝑩}\AgdaSpace{}%
\AgdaBound{𝓨}\AgdaSymbol{)}\<%
\\
%
\\[\AgdaEmptyExtraSkip]%
%
\>[1]\AgdaFunction{lift-alg-hom}\AgdaSpace{}%
\AgdaBound{𝓧}\AgdaSpace{}%
\AgdaBound{𝓨}\AgdaSpace{}%
\AgdaSymbol{\{}\AgdaBound{𝑨}\AgdaSymbol{\}}\AgdaSpace{}%
\AgdaBound{𝑩}\AgdaSpace{}%
\AgdaSymbol{(}\AgdaBound{f}\AgdaSpace{}%
\AgdaOperator{\AgdaInductiveConstructor{,}}\AgdaSpace{}%
\AgdaBound{fhom}\AgdaSymbol{)}\AgdaSpace{}%
\AgdaSymbol{=}\AgdaSpace{}%
\AgdaInductiveConstructor{lift}\AgdaSpace{}%
\AgdaOperator{\AgdaFunction{∘}}\AgdaSpace{}%
\AgdaBound{f}\AgdaSpace{}%
\AgdaOperator{\AgdaFunction{∘}}\AgdaSpace{}%
\AgdaField{lower}\AgdaSpace{}%
\AgdaOperator{\AgdaInductiveConstructor{,}}\AgdaSpace{}%
\AgdaFunction{γ}\<%
\\
\>[1][@{}l@{\AgdaIndent{0}}]%
\>[2]\AgdaKeyword{where}\<%
\\
%
\>[2]\AgdaFunction{lABh}\AgdaSpace{}%
\AgdaSymbol{:}\AgdaSpace{}%
\AgdaFunction{is-homomorphism}\AgdaSpace{}%
\AgdaSymbol{(}\AgdaFunction{lift-alg}\AgdaSpace{}%
\AgdaBound{𝑨}\AgdaSpace{}%
\AgdaBound{𝓧}\AgdaSymbol{)}\AgdaSpace{}%
\AgdaBound{𝑩}\AgdaSpace{}%
\AgdaSymbol{(}\AgdaBound{f}\AgdaSpace{}%
\AgdaOperator{\AgdaFunction{∘}}\AgdaSpace{}%
\AgdaField{lower}\AgdaSymbol{)}\<%
\\
%
\>[2]\AgdaFunction{lABh}\AgdaSpace{}%
\AgdaSymbol{=}\AgdaSpace{}%
\AgdaFunction{∘-is-hom}\AgdaSpace{}%
\AgdaSymbol{(}\AgdaFunction{lift-alg}\AgdaSpace{}%
\AgdaBound{𝑨}\AgdaSpace{}%
\AgdaBound{𝓧}\AgdaSymbol{)}\AgdaSpace{}%
\AgdaBound{𝑩}\AgdaSpace{}%
\AgdaSymbol{\{}\AgdaField{lower}\AgdaSymbol{\}\{}\AgdaBound{f}\AgdaSymbol{\}}\AgdaSpace{}%
\AgdaSymbol{(λ}\AgdaSpace{}%
\AgdaBound{\AgdaUnderscore{}}\AgdaSpace{}%
\AgdaBound{\AgdaUnderscore{}}\AgdaSpace{}%
\AgdaSymbol{→}\AgdaSpace{}%
\AgdaInductiveConstructor{𝓇ℯ𝒻𝓁}\AgdaSymbol{)}\AgdaSpace{}%
\AgdaBound{fhom}\<%
\\
%
\\[\AgdaEmptyExtraSkip]%
%
\>[2]\AgdaFunction{γ}\AgdaSpace{}%
\AgdaSymbol{:}\AgdaSpace{}%
\AgdaFunction{is-homomorphism}\AgdaSymbol{(}\AgdaFunction{lift-alg}\AgdaSpace{}%
\AgdaBound{𝑨}\AgdaSpace{}%
\AgdaBound{𝓧}\AgdaSymbol{)(}\AgdaFunction{lift-alg}\AgdaSpace{}%
\AgdaBound{𝑩}\AgdaSpace{}%
\AgdaBound{𝓨}\AgdaSymbol{)}\AgdaSpace{}%
\AgdaSymbol{(}\AgdaInductiveConstructor{lift}\AgdaSpace{}%
\AgdaOperator{\AgdaFunction{∘}}\AgdaSpace{}%
\AgdaSymbol{(}\AgdaBound{f}\AgdaSpace{}%
\AgdaOperator{\AgdaFunction{∘}}\AgdaSpace{}%
\AgdaField{lower}\AgdaSymbol{))}\<%
\\
%
\>[2]\AgdaFunction{γ}\AgdaSpace{}%
\AgdaSymbol{=}\AgdaSpace{}%
\AgdaFunction{∘-is-hom}\AgdaSpace{}%
\AgdaSymbol{(}\AgdaFunction{lift-alg}\AgdaSpace{}%
\AgdaBound{𝑨}\AgdaSpace{}%
\AgdaBound{𝓧}\AgdaSymbol{)}\AgdaSpace{}%
\AgdaSymbol{(}\AgdaFunction{lift-alg}\AgdaSpace{}%
\AgdaBound{𝑩}\AgdaSpace{}%
\AgdaBound{𝓨}\AgdaSymbol{)\{}\AgdaBound{f}\AgdaSpace{}%
\AgdaOperator{\AgdaFunction{∘}}\AgdaSpace{}%
\AgdaField{lower}\AgdaSymbol{\}\{}\AgdaInductiveConstructor{lift}\AgdaSymbol{\}}\AgdaSpace{}%
\AgdaFunction{lABh}\AgdaSpace{}%
\AgdaSymbol{λ}\AgdaSpace{}%
\AgdaBound{\AgdaUnderscore{}}\AgdaSpace{}%
\AgdaBound{\AgdaUnderscore{}}\AgdaSpace{}%
\AgdaSymbol{→}\AgdaSpace{}%
\AgdaInductiveConstructor{𝓇ℯ𝒻𝓁}\<%
\\
%
\\[\AgdaEmptyExtraSkip]%
%
\>[1]\AgdaFunction{lift-alg-iso}\AgdaSpace{}%
\AgdaSymbol{:}\AgdaSpace{}%
\>[234I]\AgdaSymbol{\{}\AgdaBound{𝑨}\AgdaSpace{}%
\AgdaSymbol{:}\AgdaSpace{}%
\AgdaFunction{Algebra}\AgdaSpace{}%
\AgdaBound{𝓤}\AgdaSpace{}%
\AgdaBound{𝑆}\AgdaSymbol{\}\{}\AgdaBound{𝓧}\AgdaSpace{}%
\AgdaSymbol{:}\AgdaSpace{}%
\AgdaFunction{Universe}\AgdaSymbol{\}}%
\AgdaSymbol{\{}\AgdaBound{𝑩}\AgdaSpace{}%
\AgdaSymbol{:}\AgdaSpace{}%
\AgdaFunction{Algebra}\AgdaSpace{}%
\AgdaBound{𝓦}\AgdaSpace{}%
\AgdaBound{𝑆}\AgdaSymbol{\}\{}\AgdaBound{𝓨}\AgdaSpace{}%
\AgdaSymbol{:}\AgdaSpace{}%
\AgdaFunction{Universe}\AgdaSymbol{\}}\<%
\\
\>[1][@{}l@{\AgdaIndent{0}}]%
\>[2]\AgdaSymbol{→}\AgdaSpace{}%
\AgdaTab{234}
\>[14]\AgdaBound{𝑨}\AgdaSpace{}%
\AgdaOperator{\AgdaFunction{≅}}\AgdaSpace{}%
\AgdaBound{𝑩}\AgdaSpace{}%
\AgdaSymbol{→}\AgdaSpace{}%
\AgdaSymbol{(}\AgdaFunction{lift-alg}\AgdaSpace{}%
\AgdaBound{𝑨}\AgdaSpace{}%
\AgdaBound{𝓧}\AgdaSymbol{)}\AgdaSpace{}%
\AgdaOperator{\AgdaFunction{≅}}\AgdaSpace{}%
\AgdaSymbol{(}\AgdaFunction{lift-alg}\AgdaSpace{}%
\AgdaBound{𝑩}\AgdaSpace{}%
\AgdaBound{𝓨}\AgdaSymbol{)}\<%
\\
%
\\[\AgdaEmptyExtraSkip]%
%
\>[1]\AgdaFunction{lift-alg-iso}\AgdaSpace{}%
\AgdaBound{A≅B}\AgdaSpace{}%
\AgdaSymbol{=}\AgdaSpace{}%
\AgdaFunction{≅-trans}\AgdaSpace{}%
\AgdaSymbol{(}\AgdaFunction{≅-trans}\AgdaSpace{}%
\AgdaSymbol{(}\AgdaFunction{≅-sym}\AgdaSpace{}%
\AgdaFunction{lift-alg-≅}\AgdaSymbol{)}\AgdaSpace{}%
\AgdaBound{A≅B}\AgdaSymbol{)}\AgdaSpace{}%
\AgdaFunction{lift-alg-≅}\<%
\end{code}

\subsubsection{Lift associativity}\label{lift-associativity}

The lift is also associative, up to isomorphism at least.
\ccpad
\begin{code}%
\>[1]\AgdaFunction{lift-alg-assoc}\AgdaSpace{}%
\AgdaSymbol{:}\AgdaSpace{}%
\AgdaSymbol{\{}\AgdaBound{𝑨}\AgdaSpace{}%
\AgdaSymbol{:}\AgdaSpace{}%
\AgdaFunction{Algebra}\AgdaSpace{}%
\AgdaBound{𝓤}\AgdaSpace{}%
\AgdaBound{𝑆}\AgdaSymbol{\}}\AgdaSpace{}%
\AgdaSymbol{→}\AgdaSpace{}%
\AgdaFunction{lift-alg}\AgdaSpace{}%
\AgdaBound{𝑨}\AgdaSpace{}%
\AgdaSymbol{(}\AgdaBound{𝓦}\AgdaSpace{}%
\AgdaOperator{\AgdaFunction{⊔}}\AgdaSpace{}%
\AgdaBound{𝓘}\AgdaSymbol{)}\AgdaSpace{}%
\AgdaOperator{\AgdaFunction{≅}}\AgdaSpace{}%
\AgdaSymbol{(}\AgdaFunction{lift-alg}\AgdaSpace{}%
\AgdaSymbol{(}\AgdaFunction{lift-alg}\AgdaSpace{}%
\AgdaBound{𝑨}\AgdaSpace{}%
\AgdaBound{𝓦}\AgdaSymbol{)}\AgdaSpace{}%
\AgdaBound{𝓘}\AgdaSymbol{)}\<%
\\
%
\>[1]\AgdaFunction{lift-alg-assoc}\AgdaSpace{}%
\AgdaSymbol{\{}\AgdaBound{𝑨}\AgdaSymbol{\}}\AgdaSpace{}%
\AgdaSymbol{=}\AgdaSpace{}%
\AgdaFunction{≅-trans}\AgdaSpace{}%
\AgdaSymbol{(}\AgdaFunction{≅-trans}\AgdaSpace{}%
\AgdaFunction{γ}\AgdaSpace{}%
\AgdaFunction{lift-alg-≅}\AgdaSymbol{)}\AgdaSpace{}%
\AgdaFunction{lift-alg-≅}\<%
\\
\>[1][@{}l@{\AgdaIndent{0}}]%
\>[2]\AgdaKeyword{where}\<%
\\
%
\>[2]\AgdaFunction{γ}\AgdaSpace{}%
\AgdaSymbol{:}\AgdaSpace{}%
\AgdaFunction{lift-alg}\AgdaSpace{}%
\AgdaBound{𝑨}\AgdaSpace{}%
\AgdaSymbol{(}\AgdaBound{𝓦}\AgdaSpace{}%
\AgdaOperator{\AgdaFunction{⊔}}\AgdaSpace{}%
\AgdaBound{𝓘}\AgdaSymbol{)}\AgdaSpace{}%
\AgdaOperator{\AgdaFunction{≅}}\AgdaSpace{}%
\AgdaBound{𝑨}\<%
\\
%
\>[2]\AgdaFunction{γ}\AgdaSpace{}%
\AgdaSymbol{=}\AgdaSpace{}%
\AgdaFunction{≅-sym}\AgdaSpace{}%
\AgdaFunction{lift-alg-≅}\<%
\\
%
\\[\AgdaEmptyExtraSkip]%
%
\>[1]\AgdaFunction{lift-alg-associative}\AgdaSpace{}%
\AgdaSymbol{:}\AgdaSpace{}%
\AgdaSymbol{(}\AgdaBound{𝑨}\AgdaSpace{}%
\AgdaSymbol{:}\AgdaSpace{}%
\AgdaFunction{Algebra}\AgdaSpace{}%
\AgdaBound{𝓤}\AgdaSpace{}%
\AgdaBound{𝑆}\AgdaSymbol{)}\AgdaSpace{}%
\AgdaSymbol{→}\AgdaSpace{}%
\AgdaFunction{lift-alg}\AgdaSpace{}%
\AgdaBound{𝑨}\AgdaSpace{}%
\AgdaSymbol{(}\AgdaBound{𝓦}\AgdaSpace{}%
\AgdaOperator{\AgdaFunction{⊔}}\AgdaSpace{}%
\AgdaBound{𝓘}\AgdaSymbol{)}\AgdaSpace{}%
\AgdaOperator{\AgdaFunction{≅}}\AgdaSpace{}%
\AgdaSymbol{(}\AgdaFunction{lift-alg}\AgdaSpace{}%
\AgdaSymbol{(}\AgdaFunction{lift-alg}\AgdaSpace{}%
\AgdaBound{𝑨}\AgdaSpace{}%
\AgdaBound{𝓦}\AgdaSymbol{)}\AgdaSpace{}%
\AgdaBound{𝓘}\AgdaSymbol{)}\<%
\\
%
\>[1]\AgdaFunction{lift-alg-associative}\AgdaSpace{}%
\AgdaBound{𝑨}\AgdaSpace{}%
\AgdaSymbol{=}\AgdaSpace{}%
\AgdaFunction{lift-alg-assoc}\AgdaSpace{}%
\AgdaSymbol{\{}\AgdaBound{𝑨}\AgdaSymbol{\}}\<%
\end{code}

\subsubsection{Products preserve isomorphisms}\label{products-preserve-isomorphisms}

Products of isomorphic families of algebras are themselves isomorphic. The proof looks a bit technical, but it is as straightforward as it ought to be.
\ccpad
\begin{code}%
\>[1]\AgdaFunction{⨅≅}\AgdaSpace{}%
\AgdaSymbol{:}\AgdaSpace{}%
\AgdaSymbol{\{}\AgdaBound{𝒜}\AgdaSpace{}%
\AgdaSymbol{:}\AgdaSpace{}%
\AgdaBound{I}\AgdaSpace{}%
\AgdaSymbol{→}\AgdaSpace{}%
\AgdaFunction{Algebra}\AgdaSpace{}%
\AgdaBound{𝓤}\AgdaSpace{}%
\AgdaBound{𝑆}\AgdaSymbol{\}\{}\AgdaBound{ℬ}\AgdaSpace{}%
\AgdaSymbol{:}\AgdaSpace{}%
\AgdaBound{I}\AgdaSpace{}%
\AgdaSymbol{→}\AgdaSpace{}%
\AgdaFunction{Algebra}\AgdaSpace{}%
\AgdaBound{𝓦}\AgdaSpace{}%
\AgdaBound{𝑆}\AgdaSymbol{\}}\AgdaSpace{}%
\AgdaSymbol{→}\AgdaSpace{}%
\AgdaFunction{Π}\AgdaSpace{}%
\AgdaBound{i}\AgdaSpace{}%
\AgdaFunction{꞉}\AgdaSpace{}%
\AgdaBound{I}\AgdaSpace{}%
\AgdaFunction{,}\AgdaSpace{}%
\AgdaBound{𝒜}\AgdaSpace{}%
\AgdaBound{i}\AgdaSpace{}%
\AgdaOperator{\AgdaFunction{≅}}\AgdaSpace{}%
\AgdaBound{ℬ}\AgdaSpace{}%
\AgdaBound{i}\AgdaSpace{}%
\AgdaSymbol{→}\AgdaSpace{}%
\AgdaFunction{⨅}\AgdaSpace{}%
\AgdaBound{𝒜}\AgdaSpace{}%
\AgdaOperator{\AgdaFunction{≅}}\AgdaSpace{}%
\AgdaFunction{⨅}\AgdaSpace{}%
\AgdaBound{ℬ}\<%
\\
%
\\[\AgdaEmptyExtraSkip]%
%
\>[1]\AgdaFunction{⨅≅}\AgdaSpace{}%
\AgdaSymbol{\{}\AgdaBound{𝒜}\AgdaSymbol{\}\{}\AgdaBound{ℬ}\AgdaSymbol{\}}\AgdaSpace{}%
\AgdaBound{AB}\AgdaSpace{}%
\AgdaSymbol{=}\AgdaSpace{}%
\AgdaFunction{γ}\<%
\\
\>[1][@{}l@{\AgdaIndent{0}}]%
\>[2]\AgdaKeyword{where}\<%
\\
%
\>[2]\AgdaFunction{ϕ}\AgdaSpace{}%
\AgdaSymbol{:}\AgdaSpace{}%
\AgdaOperator{\AgdaFunction{∣}}\AgdaSpace{}%
\AgdaFunction{⨅}\AgdaSpace{}%
\AgdaBound{𝒜}\AgdaSpace{}%
\AgdaOperator{\AgdaFunction{∣}}\AgdaSpace{}%
\AgdaSymbol{→}\AgdaSpace{}%
\AgdaOperator{\AgdaFunction{∣}}\AgdaSpace{}%
\AgdaFunction{⨅}\AgdaSpace{}%
\AgdaBound{ℬ}\AgdaSpace{}%
\AgdaOperator{\AgdaFunction{∣}}\<%
\\
%
\>[2]\AgdaFunction{ϕ}\AgdaSpace{}%
\AgdaBound{a}\AgdaSpace{}%
\AgdaBound{i}\AgdaSpace{}%
\AgdaSymbol{=}\AgdaSpace{}%
\AgdaOperator{\AgdaFunction{∣}}\AgdaSpace{}%
\AgdaFunction{fst}\AgdaSpace{}%
\AgdaSymbol{(}\AgdaBound{AB}\AgdaSpace{}%
\AgdaBound{i}\AgdaSymbol{)}\AgdaSpace{}%
\AgdaOperator{\AgdaFunction{∣}}\AgdaSpace{}%
\AgdaSymbol{(}\AgdaBound{a}\AgdaSpace{}%
\AgdaBound{i}\AgdaSymbol{)}\<%
\\
%
\\[\AgdaEmptyExtraSkip]%
%
\>[2]\AgdaFunction{ϕhom}\AgdaSpace{}%
\AgdaSymbol{:}\AgdaSpace{}%
\AgdaFunction{is-homomorphism}\AgdaSpace{}%
\AgdaSymbol{(}\AgdaFunction{⨅}\AgdaSpace{}%
\AgdaBound{𝒜}\AgdaSymbol{)}\AgdaSpace{}%
\AgdaSymbol{(}\AgdaFunction{⨅}\AgdaSpace{}%
\AgdaBound{ℬ}\AgdaSymbol{)}\AgdaSpace{}%
\AgdaFunction{ϕ}\<%
\\
%
\>[2]\AgdaFunction{ϕhom}\AgdaSpace{}%
\AgdaBound{𝑓}\AgdaSpace{}%
\AgdaBound{a}\AgdaSpace{}%
\AgdaSymbol{=}\AgdaSpace{}%
\AgdaBound{gfe}\AgdaSpace{}%
\AgdaSymbol{(λ}\AgdaSpace{}%
\AgdaBound{i}\AgdaSpace{}%
\AgdaSymbol{→}\AgdaSpace{}%
\AgdaOperator{\AgdaFunction{∥}}\AgdaSpace{}%
\AgdaFunction{fst}\AgdaSpace{}%
\AgdaSymbol{(}\AgdaBound{AB}\AgdaSpace{}%
\AgdaBound{i}\AgdaSymbol{)}\AgdaSpace{}%
\AgdaOperator{\AgdaFunction{∥}}\AgdaSpace{}%
\AgdaBound{𝑓}\AgdaSpace{}%
\AgdaSymbol{(λ}\AgdaSpace{}%
\AgdaBound{x}\AgdaSpace{}%
\AgdaSymbol{→}\AgdaSpace{}%
\AgdaBound{a}\AgdaSpace{}%
\AgdaBound{x}\AgdaSpace{}%
\AgdaBound{i}\AgdaSymbol{))}\<%
\\
%
\\[\AgdaEmptyExtraSkip]%
%
\>[2]\AgdaFunction{ψ}\AgdaSpace{}%
\AgdaSymbol{:}\AgdaSpace{}%
\AgdaOperator{\AgdaFunction{∣}}\AgdaSpace{}%
\AgdaFunction{⨅}\AgdaSpace{}%
\AgdaBound{ℬ}\AgdaSpace{}%
\AgdaOperator{\AgdaFunction{∣}}\AgdaSpace{}%
\AgdaSymbol{→}\AgdaSpace{}%
\AgdaOperator{\AgdaFunction{∣}}\AgdaSpace{}%
\AgdaFunction{⨅}\AgdaSpace{}%
\AgdaBound{𝒜}\AgdaSpace{}%
\AgdaOperator{\AgdaFunction{∣}}\<%
\\
%
\>[2]\AgdaFunction{ψ}\AgdaSpace{}%
\AgdaBound{b}\AgdaSpace{}%
\AgdaBound{i}\AgdaSpace{}%
\AgdaSymbol{=}\AgdaSpace{}%
\AgdaOperator{\AgdaFunction{∣}}\AgdaSpace{}%
\AgdaFunction{fst}\AgdaSpace{}%
\AgdaOperator{\AgdaFunction{∥}}\AgdaSpace{}%
\AgdaBound{AB}\AgdaSpace{}%
\AgdaBound{i}\AgdaSpace{}%
\AgdaOperator{\AgdaFunction{∥}}\AgdaSpace{}%
\AgdaOperator{\AgdaFunction{∣}}\AgdaSpace{}%
\AgdaSymbol{(}\AgdaBound{b}\AgdaSpace{}%
\AgdaBound{i}\AgdaSymbol{)}\<%
\\
%
\\[\AgdaEmptyExtraSkip]%
%
\>[2]\AgdaFunction{ψhom}\AgdaSpace{}%
\AgdaSymbol{:}\AgdaSpace{}%
\AgdaFunction{is-homomorphism}\AgdaSpace{}%
\AgdaSymbol{(}\AgdaFunction{⨅}\AgdaSpace{}%
\AgdaBound{ℬ}\AgdaSymbol{)}\AgdaSpace{}%
\AgdaSymbol{(}\AgdaFunction{⨅}\AgdaSpace{}%
\AgdaBound{𝒜}\AgdaSymbol{)}\AgdaSpace{}%
\AgdaFunction{ψ}\<%
\\
%
\>[2]\AgdaFunction{ψhom}\AgdaSpace{}%
\AgdaBound{𝑓}\AgdaSpace{}%
\AgdaBound{𝒃}\AgdaSpace{}%
\AgdaSymbol{=}\AgdaSpace{}%
\AgdaBound{gfe}\AgdaSpace{}%
\AgdaSymbol{(λ}\AgdaSpace{}%
\AgdaBound{i}\AgdaSpace{}%
\AgdaSymbol{→}\AgdaSpace{}%
\AgdaFunction{snd}\AgdaSpace{}%
\AgdaOperator{\AgdaFunction{∣}}\AgdaSpace{}%
\AgdaFunction{snd}\AgdaSpace{}%
\AgdaSymbol{(}\AgdaBound{AB}\AgdaSpace{}%
\AgdaBound{i}\AgdaSymbol{)}\AgdaSpace{}%
\AgdaOperator{\AgdaFunction{∣}}\AgdaSpace{}%
\AgdaBound{𝑓}\AgdaSpace{}%
\AgdaSymbol{(λ}\AgdaSpace{}%
\AgdaBound{x}\AgdaSpace{}%
\AgdaSymbol{→}\AgdaSpace{}%
\AgdaBound{𝒃}\AgdaSpace{}%
\AgdaBound{x}\AgdaSpace{}%
\AgdaBound{i}\AgdaSymbol{))}\<%
\\
%
\\[\AgdaEmptyExtraSkip]%
%
\>[2]\AgdaFunction{ϕ\textasciitilde{}ψ}\AgdaSpace{}%
\AgdaSymbol{:}\AgdaSpace{}%
\AgdaFunction{ϕ}\AgdaSpace{}%
\AgdaOperator{\AgdaFunction{∘}}\AgdaSpace{}%
\AgdaFunction{ψ}\AgdaSpace{}%
\AgdaOperator{\AgdaFunction{∼}}\AgdaSpace{}%
\AgdaOperator{\AgdaFunction{∣}}\AgdaSpace{}%
\AgdaFunction{𝒾𝒹}\AgdaSpace{}%
\AgdaSymbol{(}\AgdaFunction{⨅}\AgdaSpace{}%
\AgdaBound{ℬ}\AgdaSymbol{)}\AgdaSpace{}%
\AgdaOperator{\AgdaFunction{∣}}\<%
\\
%
\>[2]\AgdaFunction{ϕ\textasciitilde{}ψ}\AgdaSpace{}%
\AgdaBound{𝒃}\AgdaSpace{}%
\AgdaSymbol{=}\AgdaSpace{}%
\AgdaBound{gfe}\AgdaSpace{}%
\AgdaSymbol{λ}\AgdaSpace{}%
\AgdaBound{i}\AgdaSpace{}%
\AgdaSymbol{→}\AgdaSpace{}%
\AgdaFunction{fst}\AgdaSpace{}%
\AgdaOperator{\AgdaFunction{∥}}\AgdaSpace{}%
\AgdaFunction{snd}\AgdaSpace{}%
\AgdaSymbol{(}\AgdaBound{AB}\AgdaSpace{}%
\AgdaBound{i}\AgdaSymbol{)}\AgdaSpace{}%
\AgdaOperator{\AgdaFunction{∥}}\AgdaSpace{}%
\AgdaSymbol{(}\AgdaBound{𝒃}\AgdaSpace{}%
\AgdaBound{i}\AgdaSymbol{)}\<%
\\
%
\\[\AgdaEmptyExtraSkip]%
%
\>[2]\AgdaFunction{ψ\textasciitilde{}ϕ}\AgdaSpace{}%
\AgdaSymbol{:}\AgdaSpace{}%
\AgdaFunction{ψ}\AgdaSpace{}%
\AgdaOperator{\AgdaFunction{∘}}\AgdaSpace{}%
\AgdaFunction{ϕ}\AgdaSpace{}%
\AgdaOperator{\AgdaFunction{∼}}\AgdaSpace{}%
\AgdaOperator{\AgdaFunction{∣}}\AgdaSpace{}%
\AgdaFunction{𝒾𝒹}\AgdaSpace{}%
\AgdaSymbol{(}\AgdaFunction{⨅}\AgdaSpace{}%
\AgdaBound{𝒜}\AgdaSymbol{)}\AgdaSpace{}%
\AgdaOperator{\AgdaFunction{∣}}\<%
\\
%
\>[2]\AgdaFunction{ψ\textasciitilde{}ϕ}\AgdaSpace{}%
\AgdaBound{a}\AgdaSpace{}%
\AgdaSymbol{=}\AgdaSpace{}%
\AgdaBound{gfe}\AgdaSpace{}%
\AgdaSymbol{λ}\AgdaSpace{}%
\AgdaBound{i}\AgdaSpace{}%
\AgdaSymbol{→}\AgdaSpace{}%
\AgdaFunction{snd}\AgdaSpace{}%
\AgdaOperator{\AgdaFunction{∥}}\AgdaSpace{}%
\AgdaFunction{snd}\AgdaSpace{}%
\AgdaSymbol{(}\AgdaBound{AB}\AgdaSpace{}%
\AgdaBound{i}\AgdaSymbol{)}\AgdaSpace{}%
\AgdaOperator{\AgdaFunction{∥}}\AgdaSpace{}%
\AgdaSymbol{(}\AgdaBound{a}\AgdaSpace{}%
\AgdaBound{i}\AgdaSymbol{)}\<%
\\
%
\\[\AgdaEmptyExtraSkip]%
%
\>[2]\AgdaFunction{γ}\AgdaSpace{}%
\AgdaSymbol{:}\AgdaSpace{}%
\AgdaFunction{⨅}\AgdaSpace{}%
\AgdaBound{𝒜}\AgdaSpace{}%
\AgdaOperator{\AgdaFunction{≅}}\AgdaSpace{}%
\AgdaFunction{⨅}\AgdaSpace{}%
\AgdaBound{ℬ}\<%
\\
%
\>[2]\AgdaFunction{γ}\AgdaSpace{}%
\AgdaSymbol{=}\AgdaSpace{}%
\AgdaSymbol{(}\AgdaFunction{ϕ}\AgdaSpace{}%
\AgdaOperator{\AgdaInductiveConstructor{,}}\AgdaSpace{}%
\AgdaFunction{ϕhom}\AgdaSymbol{)}\AgdaSpace{}%
\AgdaOperator{\AgdaInductiveConstructor{,}}\AgdaSpace{}%
\AgdaSymbol{((}\AgdaFunction{ψ}\AgdaSpace{}%
\AgdaOperator{\AgdaInductiveConstructor{,}}\AgdaSpace{}%
\AgdaFunction{ψhom}\AgdaSymbol{)}\AgdaSpace{}%
\AgdaOperator{\AgdaInductiveConstructor{,}}\AgdaSpace{}%
\AgdaFunction{ϕ\textasciitilde{}ψ}\AgdaSpace{}%
\AgdaOperator{\AgdaInductiveConstructor{,}}\AgdaSpace{}%
\AgdaFunction{ψ\textasciitilde{}ϕ}\AgdaSymbol{)}\<%
\end{code}
\ccpad
A nearly identical proof goes through for isomorphisms of \emph{lifted} products (though, just for fun, we use the universal quantifier syntax here to express the dependent function type in the statement of the lemma, instead of the Pi notation we used in the statement of the previous lemma; that is, \as ∀ \ab i \as → \ab 𝒜 \ab i \af ≅ \ab ℬ (\aic{lift} \ab i) instead of \af Π \abt{i}{I} , \ab 𝒜 \ab i \af ≅ \ab ℬ (\aic{lift} \ab i)).
\ccpad
\begin{code}%
\>[1]\AgdaFunction{lift-alg-⨅≅}\AgdaSpace{}%
\AgdaSymbol{:}%
\>[1257I]\AgdaSymbol{\{}\AgdaBound{I}\AgdaSpace{}%
\AgdaSymbol{:}\AgdaSpace{}%
\AgdaBound{𝓘}\AgdaSpace{}%
\AgdaOperator{\AgdaFunction{̇}}\AgdaSymbol{\}\{}\AgdaBound{𝒜}\AgdaSpace{}%
\AgdaSymbol{:}\AgdaSpace{}%
\AgdaBound{I}\AgdaSpace{}%
\AgdaSymbol{→}\AgdaSpace{}%
\AgdaFunction{Algebra}\AgdaSpace{}%
\AgdaBound{𝓤}\AgdaSpace{}%
\AgdaBound{𝑆}\AgdaSymbol{\}\{}\AgdaBound{ℬ}\AgdaSpace{}%
\AgdaSymbol{:}\AgdaSpace{}%
\AgdaSymbol{(}\AgdaRecord{Lift}\AgdaSymbol{\{}\AgdaBound{𝓩}\AgdaSymbol{\}}\AgdaSpace{}%
\AgdaBound{I}\AgdaSymbol{)}\AgdaSpace{}%
\AgdaSymbol{→}\AgdaSpace{}%
\AgdaFunction{Algebra}\AgdaSpace{}%
\AgdaBound{𝓦}\AgdaSpace{}%
\AgdaBound{𝑆}\AgdaSymbol{\}}\<%
\\
\>[1][@{}l@{\AgdaIndent{0}}]%
\>[2]\AgdaSymbol{→}%
\AgdaTab{1257}%\>[.][@{}l@{}]\<[1257I]%
\>[15]\AgdaSymbol{(∀}\AgdaSpace{}%
\AgdaBound{i}\AgdaSpace{}%
\AgdaSymbol{→}\AgdaSpace{}%
\AgdaBound{𝒜}\AgdaSpace{}%
\AgdaBound{i}\AgdaSpace{}%
\AgdaOperator{\AgdaFunction{≅}}\AgdaSpace{}%
\AgdaBound{ℬ}\AgdaSpace{}%
\AgdaSymbol{(}\AgdaInductiveConstructor{lift}\AgdaSpace{}%
\AgdaBound{i}\AgdaSymbol{))}\AgdaSpace{}%
\AgdaSymbol{→}\AgdaSpace{}%
\AgdaFunction{lift-alg}\AgdaSpace{}%
\AgdaSymbol{(}\AgdaFunction{⨅}\AgdaSpace{}%
\AgdaBound{𝒜}\AgdaSymbol{)}\AgdaSpace{}%
\AgdaBound{𝓩}\AgdaSpace{}%
\AgdaOperator{\AgdaFunction{≅}}\AgdaSpace{}%
\AgdaFunction{⨅}\AgdaSpace{}%
\AgdaBound{ℬ}\<%
\\
%
\\[\AgdaEmptyExtraSkip]%
%
\>[1]\AgdaFunction{lift-alg-⨅≅}\AgdaSpace{}%
\AgdaSymbol{\{}\AgdaBound{I}\AgdaSymbol{\}\{}\AgdaBound{𝒜}\AgdaSymbol{\}\{}\AgdaBound{ℬ}\AgdaSymbol{\}}\AgdaSpace{}%
\AgdaBound{AB}\AgdaSpace{}%
\AgdaSymbol{=}\AgdaSpace{}%
\AgdaFunction{γ}\<%
\\
\>[1][@{}l@{\AgdaIndent{0}}]%
\>[2]\AgdaKeyword{where}\<%
\\
%
\>[2]\AgdaFunction{ϕ}\AgdaSpace{}%
\AgdaSymbol{:}\AgdaSpace{}%
\AgdaOperator{\AgdaFunction{∣}}\AgdaSpace{}%
\AgdaFunction{⨅}\AgdaSpace{}%
\AgdaBound{𝒜}\AgdaSpace{}%
\AgdaOperator{\AgdaFunction{∣}}\AgdaSpace{}%
\AgdaSymbol{→}\AgdaSpace{}%
\AgdaOperator{\AgdaFunction{∣}}\AgdaSpace{}%
\AgdaFunction{⨅}\AgdaSpace{}%
\AgdaBound{ℬ}\AgdaSpace{}%
\AgdaOperator{\AgdaFunction{∣}}\<%
\\
%
\>[2]\AgdaFunction{ϕ}\AgdaSpace{}%
\AgdaBound{a}\AgdaSpace{}%
\AgdaBound{i}\AgdaSpace{}%
\AgdaSymbol{=}\AgdaSpace{}%
\AgdaOperator{\AgdaFunction{∣}}\AgdaSpace{}%
\AgdaFunction{fst}\AgdaSpace{}%
\AgdaSymbol{(}\AgdaBound{AB}%
\>[21]\AgdaSymbol{(}\AgdaField{lower}\AgdaSpace{}%
\AgdaBound{i}\AgdaSymbol{))}\AgdaSpace{}%
\AgdaOperator{\AgdaFunction{∣}}\AgdaSpace{}%
\AgdaSymbol{(}\AgdaBound{a}\AgdaSpace{}%
\AgdaSymbol{(}\AgdaField{lower}\AgdaSpace{}%
\AgdaBound{i}\AgdaSymbol{))}\<%
\\
%
\\[\AgdaEmptyExtraSkip]%
%
\>[2]\AgdaFunction{ϕhom}\AgdaSpace{}%
\AgdaSymbol{:}\AgdaSpace{}%
\AgdaFunction{is-homomorphism}\AgdaSpace{}%
\AgdaSymbol{(}\AgdaFunction{⨅}\AgdaSpace{}%
\AgdaBound{𝒜}\AgdaSymbol{)}\AgdaSpace{}%
\AgdaSymbol{(}\AgdaFunction{⨅}\AgdaSpace{}%
\AgdaBound{ℬ}\AgdaSymbol{)}\AgdaSpace{}%
\AgdaFunction{ϕ}\<%
\\
%
\>[2]\AgdaFunction{ϕhom}\AgdaSpace{}%
\AgdaBound{𝑓}\AgdaSpace{}%
\AgdaBound{a}\AgdaSpace{}%
\AgdaSymbol{=}\AgdaSpace{}%
\AgdaBound{gfe}\AgdaSpace{}%
\AgdaSymbol{(λ}\AgdaSpace{}%
\AgdaBound{i}\AgdaSpace{}%
\AgdaSymbol{→}\AgdaSpace{}%
\AgdaSymbol{(}\AgdaOperator{\AgdaFunction{∥}}\AgdaSpace{}%
\AgdaFunction{fst}\AgdaSpace{}%
\AgdaSymbol{(}\AgdaBound{AB}\AgdaSpace{}%
\AgdaSymbol{(}\AgdaField{lower}\AgdaSpace{}%
\AgdaBound{i}\AgdaSymbol{))}\AgdaSpace{}%
\AgdaOperator{\AgdaFunction{∥}}\AgdaSymbol{)}\AgdaSpace{}%
\AgdaBound{𝑓}\AgdaSpace{}%
\AgdaSymbol{(λ}\AgdaSpace{}%
\AgdaBound{x}\AgdaSpace{}%
\AgdaSymbol{→}\AgdaSpace{}%
\AgdaBound{a}\AgdaSpace{}%
\AgdaBound{x}\AgdaSpace{}%
\AgdaSymbol{(}\AgdaField{lower}\AgdaSpace{}%
\AgdaBound{i}\AgdaSymbol{)))}\<%
\\
%
\\[\AgdaEmptyExtraSkip]%
%
\>[2]\AgdaFunction{ψ}\AgdaSpace{}%
\AgdaSymbol{:}\AgdaSpace{}%
\AgdaOperator{\AgdaFunction{∣}}\AgdaSpace{}%
\AgdaFunction{⨅}\AgdaSpace{}%
\AgdaBound{ℬ}\AgdaSpace{}%
\AgdaOperator{\AgdaFunction{∣}}\AgdaSpace{}%
\AgdaSymbol{→}\AgdaSpace{}%
\AgdaOperator{\AgdaFunction{∣}}\AgdaSpace{}%
\AgdaFunction{⨅}\AgdaSpace{}%
\AgdaBound{𝒜}\AgdaSpace{}%
\AgdaOperator{\AgdaFunction{∣}}\<%
\\
%
\>[2]\AgdaFunction{ψ}\AgdaSpace{}%
\AgdaBound{b}\AgdaSpace{}%
\AgdaBound{i}\AgdaSpace{}%
\AgdaSymbol{=}\AgdaSpace{}%
\AgdaOperator{\AgdaFunction{∣}}\AgdaSpace{}%
\AgdaFunction{fst}\AgdaSpace{}%
\AgdaOperator{\AgdaFunction{∥}}\AgdaSpace{}%
\AgdaBound{AB}\AgdaSpace{}%
\AgdaBound{i}\AgdaSpace{}%
\AgdaOperator{\AgdaFunction{∥}}\AgdaSpace{}%
\AgdaOperator{\AgdaFunction{∣}}\AgdaSpace{}%
\AgdaSymbol{(}\AgdaBound{b}\AgdaSpace{}%
\AgdaSymbol{(}\AgdaInductiveConstructor{lift}\AgdaSpace{}%
\AgdaBound{i}\AgdaSymbol{))}\<%
\\
%
\\[\AgdaEmptyExtraSkip]%
%
\>[2]\AgdaFunction{ψhom}\AgdaSpace{}%
\AgdaSymbol{:}\AgdaSpace{}%
\AgdaFunction{is-homomorphism}\AgdaSpace{}%
\AgdaSymbol{(}\AgdaFunction{⨅}\AgdaSpace{}%
\AgdaBound{ℬ}\AgdaSymbol{)}\AgdaSpace{}%
\AgdaSymbol{(}\AgdaFunction{⨅}\AgdaSpace{}%
\AgdaBound{𝒜}\AgdaSymbol{)}\AgdaSpace{}%
\AgdaFunction{ψ}\<%
\\
%
\>[2]\AgdaFunction{ψhom}\AgdaSpace{}%
\AgdaBound{𝑓}\AgdaSpace{}%
\AgdaBound{𝒃}\AgdaSpace{}%
\AgdaSymbol{=}\AgdaSpace{}%
\AgdaBound{gfe}\AgdaSpace{}%
\AgdaSymbol{(λ}\AgdaSpace{}%
\AgdaBound{i}\AgdaSpace{}%
\AgdaSymbol{→}\AgdaSpace{}%
\AgdaSymbol{(}\AgdaFunction{snd}\AgdaSpace{}%
\AgdaOperator{\AgdaFunction{∣}}\AgdaSpace{}%
\AgdaFunction{snd}\AgdaSpace{}%
\AgdaSymbol{(}\AgdaBound{AB}\AgdaSpace{}%
\AgdaBound{i}\AgdaSymbol{)}\AgdaSpace{}%
\AgdaOperator{\AgdaFunction{∣}}\AgdaSymbol{)}\AgdaSpace{}%
\AgdaBound{𝑓}\AgdaSpace{}%
\AgdaSymbol{(λ}\AgdaSpace{}%
\AgdaBound{x}\AgdaSpace{}%
\AgdaSymbol{→}\AgdaSpace{}%
\AgdaBound{𝒃}\AgdaSpace{}%
\AgdaBound{x}\AgdaSpace{}%
\AgdaSymbol{(}\AgdaInductiveConstructor{lift}\AgdaSpace{}%
\AgdaBound{i}\AgdaSymbol{)))}\<%
\\
%
\\[\AgdaEmptyExtraSkip]%
%
\>[2]\AgdaFunction{ϕ\textasciitilde{}ψ}\AgdaSpace{}%
\AgdaSymbol{:}\AgdaSpace{}%
\AgdaFunction{ϕ}\AgdaSpace{}%
\AgdaOperator{\AgdaFunction{∘}}\AgdaSpace{}%
\AgdaFunction{ψ}\AgdaSpace{}%
\AgdaOperator{\AgdaFunction{∼}}\AgdaSpace{}%
\AgdaOperator{\AgdaFunction{∣}}\AgdaSpace{}%
\AgdaFunction{𝒾𝒹}\AgdaSpace{}%
\AgdaSymbol{(}\AgdaFunction{⨅}\AgdaSpace{}%
\AgdaBound{ℬ}\AgdaSymbol{)}\AgdaSpace{}%
\AgdaOperator{\AgdaFunction{∣}}\<%
\\
%
\>[2]\AgdaFunction{ϕ\textasciitilde{}ψ}\AgdaSpace{}%
\AgdaBound{𝒃}\AgdaSpace{}%
\AgdaSymbol{=}\AgdaSpace{}%
\AgdaBound{gfe}\AgdaSpace{}%
\AgdaSymbol{λ}\AgdaSpace{}%
\AgdaBound{i}\AgdaSpace{}%
\AgdaSymbol{→}\AgdaSpace{}%
\AgdaFunction{fst}\AgdaSpace{}%
\AgdaOperator{\AgdaFunction{∥}}\AgdaSpace{}%
\AgdaFunction{snd}\AgdaSpace{}%
\AgdaSymbol{(}\AgdaBound{AB}\AgdaSpace{}%
\AgdaSymbol{(}\AgdaField{lower}\AgdaSpace{}%
\AgdaBound{i}\AgdaSymbol{))}\AgdaSpace{}%
\AgdaOperator{\AgdaFunction{∥}}\AgdaSpace{}%
\AgdaSymbol{(}\AgdaBound{𝒃}\AgdaSpace{}%
\AgdaBound{i}\AgdaSymbol{)}\<%
\\
%
\\[\AgdaEmptyExtraSkip]%
%
\>[2]\AgdaFunction{ψ\textasciitilde{}ϕ}\AgdaSpace{}%
\AgdaSymbol{:}\AgdaSpace{}%
\AgdaFunction{ψ}\AgdaSpace{}%
\AgdaOperator{\AgdaFunction{∘}}\AgdaSpace{}%
\AgdaFunction{ϕ}\AgdaSpace{}%
\AgdaOperator{\AgdaFunction{∼}}\AgdaSpace{}%
\AgdaOperator{\AgdaFunction{∣}}\AgdaSpace{}%
\AgdaFunction{𝒾𝒹}\AgdaSpace{}%
\AgdaSymbol{(}\AgdaFunction{⨅}\AgdaSpace{}%
\AgdaBound{𝒜}\AgdaSymbol{)}\AgdaSpace{}%
\AgdaOperator{\AgdaFunction{∣}}\<%
\\
%
\>[2]\AgdaFunction{ψ\textasciitilde{}ϕ}\AgdaSpace{}%
\AgdaBound{a}\AgdaSpace{}%
\AgdaSymbol{=}\AgdaSpace{}%
\AgdaBound{gfe}\AgdaSpace{}%
\AgdaSymbol{λ}\AgdaSpace{}%
\AgdaBound{i}\AgdaSpace{}%
\AgdaSymbol{→}\AgdaSpace{}%
\AgdaFunction{snd}\AgdaSpace{}%
\AgdaOperator{\AgdaFunction{∥}}\AgdaSpace{}%
\AgdaFunction{snd}\AgdaSpace{}%
\AgdaSymbol{(}\AgdaBound{AB}\AgdaSpace{}%
\AgdaBound{i}\AgdaSymbol{)}\AgdaSpace{}%
\AgdaOperator{\AgdaFunction{∥}}\AgdaSpace{}%
\AgdaSymbol{(}\AgdaBound{a}\AgdaSpace{}%
\AgdaBound{i}\AgdaSymbol{)}\<%
\\
%
\\[\AgdaEmptyExtraSkip]%
%
\>[2]\AgdaFunction{A≅B}\AgdaSpace{}%
\AgdaSymbol{:}\AgdaSpace{}%
\AgdaFunction{⨅}\AgdaSpace{}%
\AgdaBound{𝒜}\AgdaSpace{}%
\AgdaOperator{\AgdaFunction{≅}}\AgdaSpace{}%
\AgdaFunction{⨅}\AgdaSpace{}%
\AgdaBound{ℬ}\<%
\\
%
\>[2]\AgdaFunction{A≅B}\AgdaSpace{}%
\AgdaSymbol{=}\AgdaSpace{}%
\AgdaSymbol{(}\AgdaFunction{ϕ}\AgdaSpace{}%
\AgdaOperator{\AgdaInductiveConstructor{,}}\AgdaSpace{}%
\AgdaFunction{ϕhom}\AgdaSymbol{)}\AgdaSpace{}%
\AgdaOperator{\AgdaInductiveConstructor{,}}\AgdaSpace{}%
\AgdaSymbol{((}\AgdaFunction{ψ}\AgdaSpace{}%
\AgdaOperator{\AgdaInductiveConstructor{,}}\AgdaSpace{}%
\AgdaFunction{ψhom}\AgdaSymbol{)}\AgdaSpace{}%
\AgdaOperator{\AgdaInductiveConstructor{,}}\AgdaSpace{}%
\AgdaFunction{ϕ\textasciitilde{}ψ}\AgdaSpace{}%
\AgdaOperator{\AgdaInductiveConstructor{,}}\AgdaSpace{}%
\AgdaFunction{ψ\textasciitilde{}ϕ}\AgdaSymbol{)}\<%
\\
%
\\[\AgdaEmptyExtraSkip]%
%
\>[2]\AgdaFunction{γ}\AgdaSpace{}%
\AgdaSymbol{:}\AgdaSpace{}%
\AgdaFunction{lift-alg}\AgdaSpace{}%
\AgdaSymbol{(}\AgdaFunction{⨅}\AgdaSpace{}%
\AgdaBound{𝒜}\AgdaSymbol{)}\AgdaSpace{}%
\AgdaBound{𝓩}\AgdaSpace{}%
\AgdaOperator{\AgdaFunction{≅}}\AgdaSpace{}%
\AgdaFunction{⨅}\AgdaSpace{}%
\AgdaBound{ℬ}\<%
\\
%
\>[2]\AgdaFunction{γ}\AgdaSpace{}%
\AgdaSymbol{=}\AgdaSpace{}%
\AgdaFunction{≅-trans}\AgdaSpace{}%
\AgdaSymbol{(}\AgdaFunction{≅-sym}\AgdaSpace{}%
\AgdaFunction{lift-alg-≅}\AgdaSymbol{)}\AgdaSpace{}%
\AgdaFunction{A≅B}\<%
\end{code}





\subsubsection{Embedding tools}\label{embedding-tools}

Finally, we prove some useful facts about embeddings that occasionally come in handy.
\ccpad
\begin{code}%
\>[1]\AgdaFunction{embedding-lift-nat}\AgdaSpace{}%
\AgdaSymbol{:}\AgdaSpace{}%
\>[1257I]\AgdaFunction{hfunext}\AgdaSpace{}%
\AgdaBound{𝓘}\AgdaSpace{}%
\AgdaBound{𝓤}\AgdaSpace{}%
\AgdaSymbol{→}\AgdaSpace{}%
\AgdaFunction{hfunext}\AgdaSpace{}%
\AgdaBound{𝓘}\AgdaSpace{}%
\AgdaBound{𝓦}\<%
\\
\>[1][@{}l@{\AgdaIndent{0}}]%
\>[2]\AgdaSymbol{→}%
\AgdaTab{1257}%
\>[22]\AgdaSymbol{\{}\AgdaBound{I}\AgdaSpace{}%
\AgdaSymbol{:}\AgdaSpace{}%
\AgdaBound{𝓘}\AgdaSpace{}%
\AgdaOperator{\AgdaFunction{̇}}\AgdaSymbol{\}\{}\AgdaBound{A}\AgdaSpace{}%
\AgdaSymbol{:}\AgdaSpace{}%
\AgdaBound{I}\AgdaSpace{}%
\AgdaSymbol{→}\AgdaSpace{}%
\AgdaBound{𝓤}\AgdaSpace{}%
\AgdaOperator{\AgdaFunction{̇}}\AgdaSymbol{\}\{}\AgdaBound{B}\AgdaSpace{}%
\AgdaSymbol{:}\AgdaSpace{}%
\AgdaBound{I}\AgdaSpace{}%
\AgdaSymbol{→}\AgdaSpace{}%
\AgdaBound{𝓦}\AgdaSpace{}%
\AgdaOperator{\AgdaFunction{̇}}\AgdaSymbol{\}}\<%
\\
%
\>[22]\AgdaSymbol{(}\AgdaBound{h}\AgdaSpace{}%
\AgdaSymbol{:}\AgdaSpace{}%
\AgdaFunction{Nat}\AgdaSpace{}%
\AgdaBound{A}\AgdaSpace{}%
\AgdaBound{B}\AgdaSymbol{)}\AgdaSpace{}%
\AgdaSymbol{→}\AgdaSpace{}%
\AgdaSymbol{(∀}\AgdaSpace{}%
\AgdaBound{i}\AgdaSpace{}%
\AgdaSymbol{→}\AgdaSpace{}%
\AgdaFunction{is-embedding}\AgdaSpace{}%
\AgdaSymbol{(}\AgdaBound{h}\AgdaSpace{}%
\AgdaBound{i}\AgdaSymbol{))}\<%
\\
%
\>[22]\AgdaComment{--------------------------------------------------}\<%
\\
%
\>[2]\AgdaSymbol{→}%
\>[22]\AgdaFunction{is-embedding}\AgdaSymbol{(}\AgdaFunction{NatΠ}\AgdaSpace{}%
\AgdaBound{h}\AgdaSymbol{)}\<%
\\
%
\\[\AgdaEmptyExtraSkip]%
%
\>[1]\AgdaFunction{embedding-lift-nat}\AgdaSpace{}%
\AgdaBound{hfu}\AgdaSpace{}%
\AgdaBound{hfw}\AgdaSpace{}%
\AgdaBound{h}\AgdaSpace{}%
\AgdaBound{hem}\AgdaSpace{}%
\AgdaSymbol{=}\AgdaSpace{}%
\AgdaFunction{NatΠ-is-embedding}\AgdaSpace{}%
\AgdaBound{hfu}\AgdaSpace{}%
\AgdaBound{hfw}\AgdaSpace{}%
\AgdaBound{h}\AgdaSpace{}%
\AgdaBound{hem}\<%
\\
%
\\[\AgdaEmptyExtraSkip]%
%
\\[\AgdaEmptyExtraSkip]%
%
\>[1]\AgdaFunction{embedding-lift-nat'}\AgdaSpace{}%
\AgdaSymbol{:}\AgdaSpace{}%
\>[321I]\AgdaFunction{hfunext}\AgdaSpace{}%
\AgdaBound{𝓘}\AgdaSpace{}%
\AgdaBound{𝓤}\AgdaSpace{}%
\AgdaSymbol{→}\AgdaSpace{}%
\AgdaFunction{hfunext}\AgdaSpace{}%
\AgdaBound{𝓘}\AgdaSpace{}%
\AgdaBound{𝓦}\<%
\\
\>[1][@{}l@{\AgdaIndent{0}}]%
\>[2]\AgdaSymbol{→}%
\AgdaTab{321}%
\>[23]\AgdaSymbol{\{}\AgdaBound{I}\AgdaSpace{}%
\AgdaSymbol{:}\AgdaSpace{}%
\AgdaBound{𝓘}\AgdaSpace{}%
\AgdaOperator{\AgdaFunction{̇}}\AgdaSymbol{\}\{}\AgdaBound{𝒜}\AgdaSpace{}%
\AgdaSymbol{:}\AgdaSpace{}%
\AgdaBound{I}\AgdaSpace{}%
\AgdaSymbol{→}\AgdaSpace{}%
\AgdaFunction{Algebra}\AgdaSpace{}%
\AgdaBound{𝓤}\AgdaSpace{}%
\AgdaBound{𝑆}\AgdaSymbol{\}\{}\AgdaBound{ℬ}\AgdaSpace{}%
\AgdaSymbol{:}\AgdaSpace{}%
\AgdaBound{I}\AgdaSpace{}%
\AgdaSymbol{→}\AgdaSpace{}%
\AgdaFunction{Algebra}\AgdaSpace{}%
\AgdaBound{𝓦}\AgdaSpace{}%
\AgdaBound{𝑆}\AgdaSymbol{\}}\<%
\\
%
\>[23]\AgdaSymbol{(}\AgdaBound{h}\AgdaSpace{}%
\AgdaSymbol{:}\AgdaSpace{}%
\AgdaFunction{Nat}\AgdaSymbol{(}\AgdaFunction{fst}\AgdaSpace{}%
\AgdaOperator{\AgdaFunction{∘}}\AgdaSpace{}%
\AgdaBound{𝒜}\AgdaSymbol{)(}\AgdaFunction{fst}\AgdaSpace{}%
\AgdaOperator{\AgdaFunction{∘}}\AgdaSpace{}%
\AgdaBound{ℬ}\AgdaSymbol{))}\AgdaSpace{}%
\AgdaSymbol{→}\AgdaSpace{}%
\AgdaSymbol{(∀}\AgdaSpace{}%
\AgdaBound{i}\AgdaSpace{}%
\AgdaSymbol{→}\AgdaSpace{}%
\AgdaFunction{is-embedding}\AgdaSpace{}%
\AgdaSymbol{(}\AgdaBound{h}\AgdaSpace{}%
\AgdaBound{i}\AgdaSymbol{))}\<%
\\
%
\>[23]\AgdaComment{--------------------------------------------------------}\<%
\\
%
\>[2]\AgdaSymbol{→}%
\>[23]\AgdaFunction{is-embedding}\AgdaSymbol{(}\AgdaFunction{NatΠ}\AgdaSpace{}%
\AgdaBound{h}\AgdaSymbol{)}\<%
\\
%
\\[\AgdaEmptyExtraSkip]%
%
\>[1]\AgdaFunction{embedding-lift-nat'}\AgdaSpace{}%
\AgdaBound{hfu}\AgdaSpace{}%
\AgdaBound{hfw}\AgdaSpace{}%
\AgdaBound{h}\AgdaSpace{}%
\AgdaBound{hem}\AgdaSpace{}%
\AgdaSymbol{=}\AgdaSpace{}%
\AgdaFunction{NatΠ-is-embedding}\AgdaSpace{}%
\AgdaBound{hfu}\AgdaSpace{}%
\AgdaBound{hfw}\AgdaSpace{}%
\AgdaBound{h}\AgdaSpace{}%
\AgdaBound{hem}\<%
\\
%
\\[\AgdaEmptyExtraSkip]%
%
\\[\AgdaEmptyExtraSkip]%
%
\>[1]\AgdaFunction{embedding-lift}\AgdaSpace{}%
\AgdaSymbol{:}\AgdaSpace{}%
\>[201I]\AgdaFunction{hfunext}\AgdaSpace{}%
\AgdaBound{𝓘}\AgdaSpace{}%
\AgdaBound{𝓤}\AgdaSpace{}%
\AgdaSymbol{→}\AgdaSpace{}%
\AgdaFunction{hfunext}\AgdaSpace{}%
\AgdaBound{𝓘}\AgdaSpace{}%
\AgdaBound{𝓦}\<%
\\
\>[1][@{}l@{\AgdaIndent{0}}]%
\>[2]\AgdaSymbol{→}%
\AgdaTab{201}%
\>[18]\AgdaSymbol{\{}\AgdaBound{I}\AgdaSpace{}%
\AgdaSymbol{:}\AgdaSpace{}%
\AgdaBound{𝓘}\AgdaSpace{}%
\AgdaOperator{\AgdaFunction{̇}}\AgdaSymbol{\}}\AgdaSpace{}%
\AgdaSymbol{\{}\AgdaBound{𝒜}\AgdaSpace{}%
\AgdaSymbol{:}\AgdaSpace{}%
\AgdaBound{I}\AgdaSpace{}%
\AgdaSymbol{→}\AgdaSpace{}%
\AgdaFunction{Algebra}\AgdaSpace{}%
\AgdaBound{𝓤}\AgdaSpace{}%
\AgdaBound{𝑆}\AgdaSymbol{\}\{}\AgdaBound{ℬ}\AgdaSpace{}%
\AgdaSymbol{:}\AgdaSpace{}%
\AgdaBound{I}\AgdaSpace{}%
\AgdaSymbol{→}\AgdaSpace{}%
\AgdaFunction{Algebra}\AgdaSpace{}%
\AgdaBound{𝓦}\AgdaSpace{}%
\AgdaBound{𝑆}\AgdaSymbol{\}}\<%
\\
%
\>[2]\AgdaSymbol{→}%
\>[18]\AgdaSymbol{(}\AgdaBound{h}\AgdaSpace{}%
\AgdaSymbol{:}\AgdaSpace{}%
\AgdaSymbol{∀}\AgdaSpace{}%
\AgdaBound{i}\AgdaSpace{}%
\AgdaSymbol{→}\AgdaSpace{}%
\AgdaOperator{\AgdaFunction{∣}}\AgdaSpace{}%
\AgdaBound{𝒜}\AgdaSpace{}%
\AgdaBound{i}\AgdaSpace{}%
\AgdaOperator{\AgdaFunction{∣}}\AgdaSpace{}%
\AgdaSymbol{→}\AgdaSpace{}%
\AgdaOperator{\AgdaFunction{∣}}\AgdaSpace{}%
\AgdaBound{ℬ}\AgdaSpace{}%
\AgdaBound{i}\AgdaSpace{}%
\AgdaOperator{\AgdaFunction{∣}}\AgdaSymbol{)}\AgdaSpace{}%
\AgdaSymbol{→}\AgdaSpace{}%
\AgdaSymbol{(∀}\AgdaSpace{}%
\AgdaBound{i}\AgdaSpace{}%
\AgdaSymbol{→}\AgdaSpace{}%
\AgdaFunction{is-embedding}\AgdaSpace{}%
\AgdaSymbol{(}\AgdaBound{h}\AgdaSpace{}%
\AgdaBound{i}\AgdaSymbol{))}\<%
\\
%
\>[18]\AgdaComment{----------------------------------------------------------}\<%
\\
%
\>[2]\AgdaSymbol{→}%
\>[18]\AgdaFunction{is-embedding}\AgdaSymbol{(λ}\AgdaSpace{}%
\AgdaSymbol{(}\AgdaBound{x}\AgdaSpace{}%
\AgdaSymbol{:}\AgdaSpace{}%
\AgdaOperator{\AgdaFunction{∣}}\AgdaSpace{}%
\AgdaFunction{⨅}\AgdaSpace{}%
\AgdaBound{𝒜}\AgdaSpace{}%
\AgdaOperator{\AgdaFunction{∣}}\AgdaSymbol{)}\AgdaSpace{}%
\AgdaSymbol{(}\AgdaBound{i}\AgdaSpace{}%
\AgdaSymbol{:}\AgdaSpace{}%
\AgdaBound{I}\AgdaSymbol{)}\AgdaSpace{}%
\AgdaSymbol{→}\AgdaSpace{}%
\AgdaSymbol{(}\AgdaBound{h}\AgdaSpace{}%
\AgdaBound{i}\AgdaSymbol{)(}\AgdaBound{x}\AgdaSpace{}%
\AgdaBound{i}\AgdaSymbol{))}\<%
\\
%
\\[\AgdaEmptyExtraSkip]%
%
\>[1]\AgdaFunction{embedding-lift}\AgdaSpace{}%
\AgdaBound{hfu}\AgdaSpace{}%
\AgdaBound{hfw}\AgdaSpace{}%
\AgdaSymbol{\{}\AgdaBound{I}\AgdaSymbol{\}\{}\AgdaBound{𝒜}\AgdaSymbol{\}\{}\AgdaBound{ℬ}\AgdaSymbol{\}}\AgdaSpace{}%
\AgdaBound{h}\AgdaSpace{}%
\AgdaBound{hem}\AgdaSpace{}%
\AgdaSymbol{=}\AgdaSpace{}%
\AgdaFunction{embedding-lift-nat'}\AgdaSpace{}%
\AgdaBound{hfu}\AgdaSpace{}%
\AgdaBound{hfw}\AgdaSpace{}%
\AgdaSymbol{\{}\AgdaBound{I}\AgdaSymbol{\}\{}\AgdaBound{𝒜}\AgdaSymbol{\}\{}\AgdaBound{ℬ}\AgdaSymbol{\}}\AgdaSpace{}%
\AgdaBound{h}\AgdaSpace{}%
\AgdaBound{hem}\<%
\end{code}
\scpad
\begin{code}
\>[0]\AgdaFunction{iso→embedding}\AgdaSpace{}%
\AgdaSymbol{:}\AgdaSpace{}%
\>[1181I]\AgdaSymbol{\{}\AgdaBound{𝓤}\AgdaSpace{}%
\AgdaBound{𝓦}\AgdaSpace{}%
\AgdaSymbol{:}\AgdaSpace{}%
\AgdaFunction{Universe}\AgdaSymbol{\}\{}\AgdaBound{𝑨}\AgdaSpace{}%
\AgdaSymbol{:}\AgdaSpace{}%
\AgdaFunction{Algebra}\AgdaSpace{}%
\AgdaBound{𝓤}\AgdaSpace{}%
\AgdaBound{𝑆}\AgdaSymbol{\}\{}\AgdaBound{𝑩}\AgdaSpace{}%
\AgdaSymbol{:}\AgdaSpace{}%
\AgdaFunction{Algebra}\AgdaSpace{}%
\AgdaBound{𝓦}\AgdaSpace{}%
\AgdaBound{𝑆}\AgdaSymbol{\}}\<%
\\
\>[0][@{}l@{\AgdaIndent{0}}]%
\>[1]\AgdaSymbol{→}%
\AgdaTab{1181}%
\>[16]\AgdaSymbol{(}\AgdaBound{ϕ}\AgdaSpace{}%
\AgdaSymbol{:}\AgdaSpace{}%
\AgdaBound{𝑨}\AgdaSpace{}%
\AgdaOperator{\AgdaFunction{≅}}\AgdaSpace{}%
\AgdaBound{𝑩}\AgdaSymbol{)}\AgdaSpace{}%
\AgdaSymbol{→}\AgdaSpace{}%
\AgdaFunction{is-embedding}\AgdaSpace{}%
\AgdaSymbol{(}\AgdaFunction{fst}\AgdaSpace{}%
\AgdaOperator{\AgdaFunction{∣}}\AgdaSpace{}%
\AgdaBound{ϕ}\AgdaSpace{}%
\AgdaOperator{\AgdaFunction{∣}}\AgdaSymbol{)}\<%
\\
%
\\[\AgdaEmptyExtraSkip]%
\>[0]\AgdaFunction{iso→embedding}\AgdaSpace{}%
\AgdaBound{ϕ}\AgdaSpace{}%
\AgdaSymbol{=}%
\AgdaFunction{equivs-are-embeddings}\AgdaSpace{}%
\AgdaSymbol{(}\AgdaFunction{fst}\AgdaSpace{}%
\AgdaOperator{\AgdaFunction{∣}}\AgdaSpace{}%
\AgdaBound{ϕ}\AgdaSpace{}%
\AgdaOperator{\AgdaFunction{∣}}\AgdaSymbol{)}\AgdaSpace{}%
\AgdaSymbol{(}\AgdaFunction{invertibles-are-equivs}\AgdaSpace{}%
\AgdaSymbol{(}\AgdaFunction{fst}\AgdaSpace{}%
\AgdaOperator{\AgdaFunction{∣}}\AgdaSpace{}%
\AgdaBound{ϕ}\AgdaSpace{}%
\AgdaOperator{\AgdaFunction{∣}}\AgdaSymbol{)}\AgdaSpace{}%
\AgdaFunction{finv}\AgdaSymbol{)}\<%
\\
\>[0][@{}l@{\AgdaIndent{0}}]%
\>[1]\AgdaKeyword{where}\<%
\\
%
\>[1]\AgdaFunction{finv}\AgdaSpace{}%
\AgdaSymbol{:}\AgdaSpace{}%
\AgdaFunction{invertible}\AgdaSpace{}%
\AgdaSymbol{(}\AgdaFunction{fst}\AgdaSpace{}%
\AgdaOperator{\AgdaFunction{∣}}\AgdaSpace{}%
\AgdaBound{ϕ}\AgdaSpace{}%
\AgdaOperator{\AgdaFunction{∣}}\AgdaSymbol{)}\<%
\\
%
\>[1]\AgdaFunction{finv}\AgdaSpace{}%
\AgdaSymbol{=}\AgdaSpace{}%
\AgdaOperator{\AgdaFunction{∣}}\AgdaSpace{}%
\AgdaFunction{fst}\AgdaSpace{}%
\AgdaOperator{\AgdaFunction{∥}}\AgdaSpace{}%
\AgdaBound{ϕ}\AgdaSpace{}%
\AgdaOperator{\AgdaFunction{∥}}\AgdaSpace{}%
\AgdaOperator{\AgdaFunction{∣}}\AgdaSpace{}%
\AgdaOperator{\AgdaInductiveConstructor{,}}\AgdaSpace{}%
\AgdaSymbol{(}\AgdaFunction{snd}\AgdaSpace{}%
\AgdaOperator{\AgdaFunction{∥}}\AgdaSpace{}%
\AgdaFunction{snd}\AgdaSpace{}%
\AgdaBound{ϕ}\AgdaSpace{}%
\AgdaOperator{\AgdaFunction{∥}}\AgdaSpace{}%
\AgdaOperator{\AgdaInductiveConstructor{,}}\AgdaSpace{}%
\AgdaFunction{fst}\AgdaSpace{}%
\AgdaOperator{\AgdaFunction{∥}}\AgdaSpace{}%
\AgdaFunction{snd}\AgdaSpace{}%
\AgdaBound{ϕ}\AgdaSpace{}%
\AgdaOperator{\AgdaFunction{∥}}\AgdaSymbol{)}\<%
\\
\>[0]\<%
\end{code}


\subsection{Homomorphic Images}\label{sec:hom-images}\firstsentence{\ualibHomomorphicImages}{\urlHomomorphicImages}
% -*- TeX-master: "ualib-part2.tex" -*-
%%% Local Variables: 
%%% mode: latex
%%% TeX-engine: 'xetex
%%% End: 
% \subsubsection{Images of a single algebra}\label{images-of-a-single-algebra}

We begin with what seems, for our purposes, the most useful way to represent the class of \defn{homomorphic images} of an algebra in dependent type theory.
\ccpad
\begin{code}%
\>[1]\AgdaFunction{HomImage}\AgdaSpace{}%
\AgdaSymbol{:}\AgdaSpace{}%
\AgdaSymbol{\{}\AgdaBound{𝑨}\AgdaSpace{}%
\AgdaSymbol{:}\AgdaSpace{}%
\AgdaFunction{Algebra}\AgdaSpace{}%
\AgdaBound{𝓤}\AgdaSpace{}%
\AgdaBound{𝑆}\AgdaSymbol{\}(}\AgdaBound{𝑩}\AgdaSpace{}%
\AgdaSymbol{:}\AgdaSpace{}%
\AgdaFunction{Algebra}\AgdaSpace{}%
\AgdaBound{𝓦}\AgdaSpace{}%
\AgdaBound{𝑆}\AgdaSymbol{)(}\AgdaBound{ϕ}\AgdaSpace{}%
\AgdaSymbol{:}\AgdaSpace{}%
\AgdaFunction{hom}\AgdaSpace{}%
\AgdaBound{𝑨}\AgdaSpace{}%
\AgdaBound{𝑩}\AgdaSymbol{)}\AgdaSpace{}%
\AgdaSymbol{→}\AgdaSpace{}%
\AgdaOperator{\AgdaFunction{∣}}\AgdaSpace{}%
\AgdaBound{𝑩}\AgdaSpace{}%
\AgdaOperator{\AgdaFunction{∣}}\AgdaSpace{}%
\AgdaSymbol{→}\AgdaSpace{}%
\AgdaBound{𝓤}\AgdaSpace{}%
\AgdaOperator{\AgdaFunction{⊔}}\AgdaSpace{}%
\AgdaBound{𝓦}\AgdaSpace{}%
\AgdaOperator{\AgdaFunction{̇}}\<%
\\
%
\>[1]\AgdaFunction{HomImage}\AgdaSpace{}%
\AgdaBound{𝑩}\AgdaSpace{}%
\AgdaBound{ϕ}\AgdaSpace{}%
\AgdaSymbol{=}\AgdaSpace{}%
\AgdaSymbol{λ}\AgdaSpace{}%
\AgdaBound{b}\AgdaSpace{}%
\AgdaSymbol{→}\AgdaSpace{}%
\AgdaOperator{\AgdaDatatype{Image}}\AgdaSpace{}%
\AgdaOperator{\AgdaFunction{∣}}\AgdaSpace{}%
\AgdaBound{ϕ}\AgdaSpace{}%
\AgdaOperator{\AgdaFunction{∣}}\AgdaSpace{}%
\AgdaOperator{\AgdaDatatype{∋}}\AgdaSpace{}%
\AgdaBound{b}\<%
\\
%
\\[\AgdaEmptyExtraSkip]%
%
\>[1]\AgdaFunction{HomImagesOf}\AgdaSpace{}%
\AgdaSymbol{:}\AgdaSpace{}%
\AgdaFunction{Algebra}\AgdaSpace{}%
\AgdaBound{𝓤}\AgdaSpace{}%
\AgdaBound{𝑆}\AgdaSpace{}%
\AgdaSymbol{→}\AgdaSpace{}%
\AgdaBound{𝓞}\AgdaSpace{}%
\AgdaOperator{\AgdaFunction{⊔}}\AgdaSpace{}%
\AgdaBound{𝓥}\AgdaSpace{}%
\AgdaOperator{\AgdaFunction{⊔}}\AgdaSpace{}%
\AgdaBound{𝓤}\AgdaSpace{}%
\AgdaOperator{\AgdaFunction{⊔}}\AgdaSpace{}%
\AgdaBound{𝓦}\AgdaSpace{}%
\AgdaOperator{\AgdaFunction{⁺}}\AgdaSpace{}%
\AgdaOperator{\AgdaFunction{̇}}\<%
\\
%
\>[1]\AgdaFunction{HomImagesOf}\AgdaSpace{}%
\AgdaBound{𝑨}\AgdaSpace{}%
\AgdaSymbol{=}\AgdaSpace{}%
\AgdaFunction{Σ}\AgdaSpace{}%
\AgdaBound{𝑩}\AgdaSpace{}%
\AgdaFunction{꞉}\AgdaSpace{}%
\AgdaSymbol{(}\AgdaFunction{Algebra}\AgdaSpace{}%
\AgdaBound{𝓦}\AgdaSpace{}%
\AgdaBound{𝑆}\AgdaSymbol{)}\AgdaSpace{}%
\AgdaFunction{,}\AgdaSpace{}%
\AgdaFunction{Σ}\AgdaSpace{}%
\AgdaBound{ϕ}\AgdaSpace{}%
\AgdaFunction{꞉}\AgdaSpace{}%
\AgdaSymbol{(}\AgdaOperator{\AgdaFunction{∣}}\AgdaSpace{}%
\AgdaBound{𝑨}\AgdaSpace{}%
\AgdaOperator{\AgdaFunction{∣}}\AgdaSpace{}%
\AgdaSymbol{→}\AgdaSpace{}%
\AgdaOperator{\AgdaFunction{∣}}\AgdaSpace{}%
\AgdaBound{𝑩}\AgdaSpace{}%
\AgdaOperator{\AgdaFunction{∣}}\AgdaSymbol{)}\AgdaSpace{}%
\AgdaFunction{,}\AgdaSpace{}%
\AgdaFunction{is-homomorphism}\AgdaSpace{}%
\AgdaBound{𝑨}\AgdaSpace{}%
\AgdaBound{𝑩}\AgdaSpace{}%
\AgdaBound{ϕ}\AgdaSpace{}%
\AgdaOperator{\AgdaFunction{×}}\AgdaSpace{}%
\AgdaFunction{Epic}\AgdaSpace{}%
\AgdaBound{ϕ}\<%
\end{code}
\ccpad
These types should be self-explanatory, but just to be sure, let's describe the Sigma type appearing in the second definition. Given an \ab{𝑆}-algebra \ab{𝑨}, the type \af{HomImagesOf} \ab 𝑨 denotes the class of algebras \ab{𝑩} \as : \af{Algebra} \ab 𝓦 \ab 𝑆 with a map \ab{φ} \as : \af{∣} \ab 𝑨 \af ∣ \as → \af ∣ \ab 𝑩 \af ∣ such that \ab{φ} is a epimorphism.

The standard (informal) notion of the class of homomorphic images of an algebra assumes closure under isomorphism. Thus, we consider \ab{𝑩} to be a homomorphic image of \ab{𝑨} if (and only if) there exists an algebra \ab{𝑪} which is a homomorphic image of \ab{𝑨} and isomorphic to \ab{𝑩}.  In the \ualib we expresses this notion with the following type.
\ccpad
\begin{code}%
\>[1]\AgdaOperator{\AgdaFunction{\AgdaUnderscore{}is-hom-image-of\AgdaUnderscore{}}}\AgdaSpace{}%
\AgdaSymbol{:}\AgdaSpace{}%
\AgdaSymbol{(}\AgdaBound{𝑩}\AgdaSpace{}%
\AgdaSymbol{:}\AgdaSpace{}%
\AgdaFunction{Algebra}\AgdaSpace{}%
\AgdaBound{𝓦}\AgdaSpace{}%
\AgdaBound{𝑆}\AgdaSymbol{)(}\AgdaBound{𝑨}\AgdaSpace{}%
\AgdaSymbol{:}\AgdaSpace{}%
\AgdaFunction{Algebra}\AgdaSpace{}%
\AgdaBound{𝓤}\AgdaSpace{}%
\AgdaBound{𝑆}\AgdaSymbol{)}\AgdaSpace{}%
\AgdaSymbol{→}\AgdaSpace{}%
\AgdaFunction{ov}\AgdaSpace{}%
\AgdaBound{𝓦}\AgdaSpace{}%
\AgdaOperator{\AgdaFunction{⊔}}\AgdaSpace{}%
\AgdaBound{𝓤}\AgdaSpace{}%
\AgdaOperator{\AgdaFunction{̇}}\<%
\\
%
\>[1]\AgdaBound{𝑩}\AgdaSpace{}%
\AgdaOperator{\AgdaFunction{is-hom-image-of}}\AgdaSpace{}%
\AgdaBound{𝑨}\AgdaSpace{}%
\AgdaSymbol{=}\AgdaSpace{}%
\AgdaFunction{Σ}\AgdaSpace{}%
\AgdaBound{𝑪ϕ}\AgdaSpace{}%
\AgdaFunction{꞉}\AgdaSpace{}%
\AgdaSymbol{(}\AgdaFunction{HomImagesOf}\AgdaSpace{}%
\AgdaBound{𝑨}\AgdaSymbol{)}\AgdaSpace{}%
\AgdaFunction{,}\AgdaSpace{}%
\AgdaOperator{\AgdaFunction{∣}}\AgdaSpace{}%
\AgdaBound{𝑪ϕ}\AgdaSpace{}%
\AgdaOperator{\AgdaFunction{∣}}\AgdaSpace{}%
\AgdaOperator{\AgdaFunction{≅}}\AgdaSpace{}%
\AgdaBound{𝑩}\<%
\end{code}
\ccpad
\subsubsection{Images of a class of algebras}\label{images-of-a-class-of-algebras}

Given a class \ab{𝒦} of \ab{𝑆}-algebras, we need a type that expresses the assertion that a given algebra is a \emph{homomorphic image} of some algebra in the class, as well as a type that represents all such homomorphic images.
\ccpad
\begin{code}%
\>[1]\AgdaOperator{\AgdaFunction{\AgdaUnderscore{}is-hom-image-of-class\AgdaUnderscore{}}}\AgdaSpace{}%
\AgdaSymbol{:}\AgdaSpace{}%
\AgdaFunction{Algebra}\AgdaSpace{}%
\AgdaBound{𝓤}\AgdaSpace{}%
\AgdaBound{𝑆}\AgdaSpace{}%
\AgdaSymbol{→}\AgdaSpace{}%
\AgdaFunction{Pred}\AgdaSpace{}%
\AgdaSymbol{(}\AgdaFunction{Algebra}\AgdaSpace{}%
\AgdaBound{𝓤}\AgdaSpace{}%
\AgdaBound{𝑆}\AgdaSymbol{)(}\AgdaBound{𝓤}\AgdaSpace{}%
\AgdaOperator{\AgdaFunction{⁺}}\AgdaSymbol{)}\AgdaSpace{}%
\AgdaSymbol{→}\AgdaSpace{}%
\AgdaFunction{ov}\AgdaSpace{}%
\AgdaBound{𝓤}\AgdaSpace{}%
\AgdaOperator{\AgdaFunction{̇}}\<%
\\
%
\>[1]\AgdaBound{𝑩}\AgdaSpace{}%
\AgdaOperator{\AgdaFunction{is-hom-image-of-class}}\AgdaSpace{}%
\AgdaBound{𝓚}\AgdaSpace{}%
\AgdaSymbol{=}\AgdaSpace{}%
\AgdaFunction{Σ}\AgdaSpace{}%
\AgdaBound{𝑨}\AgdaSpace{}%
\AgdaFunction{꞉}\AgdaSpace{}%
\AgdaSymbol{(}\AgdaFunction{Algebra}\AgdaSpace{}%
\AgdaBound{𝓤}\AgdaSpace{}%
\AgdaBound{𝑆}\AgdaSymbol{)}\AgdaSpace{}%
\AgdaFunction{,}\AgdaSpace{}%
\AgdaSymbol{(}\AgdaBound{𝑨}\AgdaSpace{}%
\AgdaOperator{\AgdaFunction{∈}}\AgdaSpace{}%
\AgdaBound{𝓚}\AgdaSymbol{)}\AgdaSpace{}%
\AgdaOperator{\AgdaFunction{×}}\AgdaSpace{}%
\AgdaSymbol{(}\AgdaBound{𝑩}\AgdaSpace{}%
\AgdaOperator{\AgdaFunction{is-hom-image-of}}\AgdaSpace{}%
\AgdaBound{𝑨}\AgdaSymbol{)}\<%
\\
%
\\[\AgdaEmptyExtraSkip]%
%
\>[1]\AgdaFunction{HomImagesOfClass}\AgdaSpace{}%
\AgdaSymbol{:}\AgdaSpace{}%
\AgdaFunction{Pred}\AgdaSpace{}%
\AgdaSymbol{(}\AgdaFunction{Algebra}\AgdaSpace{}%
\AgdaBound{𝓤}\AgdaSpace{}%
\AgdaBound{𝑆}\AgdaSymbol{)}\AgdaSpace{}%
\AgdaSymbol{(}\AgdaBound{𝓤}\AgdaSpace{}%
\AgdaOperator{\AgdaFunction{⁺}}\AgdaSymbol{)}\AgdaSpace{}%
\AgdaSymbol{→}\AgdaSpace{}%
\AgdaFunction{ov}\AgdaSpace{}%
\AgdaBound{𝓤}\AgdaSpace{}%
\AgdaOperator{\AgdaFunction{̇}}\<%
\\
%
\>[1]\AgdaFunction{HomImagesOfClass}\AgdaSpace{}%
\AgdaBound{𝓚}\AgdaSpace{}%
\AgdaSymbol{=}\AgdaSpace{}%
\AgdaFunction{Σ}\AgdaSpace{}%
\AgdaBound{𝑩}\AgdaSpace{}%
\AgdaFunction{꞉}\AgdaSpace{}%
\AgdaSymbol{(}\AgdaFunction{Algebra}\AgdaSpace{}%
\AgdaBound{𝓤}\AgdaSpace{}%
\AgdaBound{𝑆}\AgdaSymbol{)}\AgdaSpace{}%
\AgdaFunction{,}\AgdaSpace{}%
\AgdaSymbol{(}\AgdaBound{𝑩}\AgdaSpace{}%
\AgdaOperator{\AgdaFunction{is-hom-image-of-class}}\AgdaSpace{}%
\AgdaBound{𝓚}\AgdaSymbol{)}\<%
\end{code}
\ccpad
\subsubsection{Lifting tools}\label{lifting-tools}

Here are some tools that have been useful (e.g., in the road to the proof of Birkhoff's HSP theorem).
The first states and proves the simple fact that the lift of an epimorphism is an epimorphism.
\ccpad
\begin{code}%
\>[1]\AgdaFunction{lift-of-alg-epic-is-epic}\AgdaSpace{}%
\AgdaSymbol{:}%
\>[208I]\AgdaSymbol{(}\AgdaBound{𝓩}\AgdaSpace{}%
\AgdaSymbol{:}\AgdaSpace{}%
\AgdaFunction{Universe}\AgdaSymbol{)\{}\AgdaBound{𝓦}\AgdaSpace{}%
\AgdaSymbol{:}\AgdaSpace{}%
\AgdaFunction{Universe}\AgdaSymbol{\}}\<%
\\
\>[.][@{}l@{}]\<[208I]%
\>[28]\AgdaSymbol{\{}\AgdaBound{𝑨}\AgdaSpace{}%
\AgdaSymbol{:}\AgdaSpace{}%
\AgdaFunction{Algebra}\AgdaSpace{}%
\AgdaBound{𝓧}\AgdaSpace{}%
\AgdaBound{𝑆}\AgdaSymbol{\}(}\AgdaBound{𝑩}\AgdaSpace{}%
\AgdaSymbol{:}\AgdaSpace{}%
\AgdaFunction{Algebra}\AgdaSpace{}%
\AgdaBound{𝓨}\AgdaSpace{}%
\AgdaBound{𝑆}\AgdaSymbol{)(}\AgdaBound{h}\AgdaSpace{}%
\AgdaSymbol{:}\AgdaSpace{}%
\AgdaFunction{hom}\AgdaSpace{}%
\AgdaBound{𝑨}\AgdaSpace{}%
\AgdaBound{𝑩}\AgdaSymbol{)}\<%
\\
%
\>[28]\AgdaComment{-----------------------------------------------}\<%
\\
\>[1][@{}l@{\AgdaIndent{0}}]%
\>[2]\AgdaSymbol{→}%
\>[28]\AgdaFunction{Epic}\AgdaSpace{}%
\AgdaOperator{\AgdaFunction{∣}}\AgdaSpace{}%
\AgdaBound{h}\AgdaSpace{}%
\AgdaOperator{\AgdaFunction{∣}}%
\>[40]\AgdaSymbol{→}%
\>[43]\AgdaFunction{Epic}\AgdaSpace{}%
\AgdaOperator{\AgdaFunction{∣}}\AgdaSpace{}%
\AgdaFunction{lift-alg-hom}\AgdaSpace{}%
\AgdaBound{𝓩}\AgdaSpace{}%
\AgdaBound{𝓦}\AgdaSpace{}%
\AgdaBound{𝑩}\AgdaSpace{}%
\AgdaBound{h}\AgdaSpace{}%
\AgdaOperator{\AgdaFunction{∣}}\<%
\\
%
\\[\AgdaEmptyExtraSkip]%
%
\>[1]\AgdaFunction{lift-of-alg-epic-is-epic}\AgdaSpace{}%
\AgdaBound{𝓩}\AgdaSpace{}%
\AgdaSymbol{\{}\AgdaBound{𝓦}\AgdaSymbol{\}}\AgdaSpace{}%
\AgdaSymbol{\{}\AgdaBound{𝑨}\AgdaSymbol{\}}\AgdaSpace{}%
\AgdaBound{𝑩}\AgdaSpace{}%
\AgdaBound{h}\AgdaSpace{}%
\AgdaBound{hepi}\AgdaSpace{}%
\AgdaBound{y}\AgdaSpace{}%
\AgdaSymbol{=}\AgdaSpace{}%
\AgdaInductiveConstructor{eq}\AgdaSpace{}%
\AgdaBound{y}\AgdaSpace{}%
\AgdaSymbol{(}\AgdaInductiveConstructor{lift}\AgdaSpace{}%
\AgdaFunction{a}\AgdaSymbol{)}\AgdaSpace{}%
\AgdaFunction{η}\<%
\\
\>[1][@{}l@{\AgdaIndent{0}}]%
\>[2]\AgdaKeyword{where}\<%
\\
%
\>[2]\AgdaFunction{lh}\AgdaSpace{}%
\AgdaSymbol{:}\AgdaSpace{}%
\AgdaFunction{hom}\AgdaSpace{}%
\AgdaSymbol{(}\AgdaFunction{lift-alg}\AgdaSpace{}%
\AgdaBound{𝑨}\AgdaSpace{}%
\AgdaBound{𝓩}\AgdaSymbol{)}\AgdaSpace{}%
\AgdaSymbol{(}\AgdaFunction{lift-alg}\AgdaSpace{}%
\AgdaBound{𝑩}\AgdaSpace{}%
\AgdaBound{𝓦}\AgdaSymbol{)}\<%
\\
%
\>[2]\AgdaFunction{lh}\AgdaSpace{}%
\AgdaSymbol{=}\AgdaSpace{}%
\AgdaFunction{lift-alg-hom}\AgdaSpace{}%
\AgdaBound{𝓩}\AgdaSpace{}%
\AgdaBound{𝓦}\AgdaSpace{}%
\AgdaBound{𝑩}\AgdaSpace{}%
\AgdaBound{h}\<%
\\
%
\\[\AgdaEmptyExtraSkip]%
%
\>[2]\AgdaFunction{ζ}\AgdaSpace{}%
\AgdaSymbol{:}\AgdaSpace{}%
\AgdaOperator{\AgdaDatatype{Image}}\AgdaSpace{}%
\AgdaOperator{\AgdaFunction{∣}}\AgdaSpace{}%
\AgdaBound{h}\AgdaSpace{}%
\AgdaOperator{\AgdaFunction{∣}}\AgdaSpace{}%
\AgdaOperator{\AgdaDatatype{∋}}\AgdaSpace{}%
\AgdaSymbol{(}\AgdaField{lower}\AgdaSpace{}%
\AgdaBound{y}\AgdaSymbol{)}\<%
\\
%
\>[2]\AgdaFunction{ζ}\AgdaSpace{}%
\AgdaSymbol{=}\AgdaSpace{}%
\AgdaBound{hepi}\AgdaSpace{}%
\AgdaSymbol{(}\AgdaField{lower}\AgdaSpace{}%
\AgdaBound{y}\AgdaSymbol{)}\<%
\\
%
\\[\AgdaEmptyExtraSkip]%
%
\>[2]\AgdaFunction{a}\AgdaSpace{}%
\AgdaSymbol{:}\AgdaSpace{}%
\AgdaOperator{\AgdaFunction{∣}}\AgdaSpace{}%
\AgdaBound{𝑨}\AgdaSpace{}%
\AgdaOperator{\AgdaFunction{∣}}\<%
\\
%
\>[2]\AgdaFunction{a}\AgdaSpace{}%
\AgdaSymbol{=}\AgdaSpace{}%
\AgdaFunction{Inv}\AgdaSpace{}%
\AgdaOperator{\AgdaFunction{∣}}\AgdaSpace{}%
\AgdaBound{h}\AgdaSpace{}%
\AgdaOperator{\AgdaFunction{∣}}\AgdaSpace{}%
\AgdaFunction{ζ}\<%
\\
%
\\[\AgdaEmptyExtraSkip]%
%
\>[2]\AgdaFunction{β}\AgdaSpace{}%
\AgdaSymbol{:}\AgdaSpace{}%
\AgdaInductiveConstructor{lift}\AgdaSpace{}%
\AgdaSymbol{(}\AgdaOperator{\AgdaFunction{∣}}\AgdaSpace{}%
\AgdaBound{h}\AgdaSpace{}%
\AgdaOperator{\AgdaFunction{∣}}\AgdaSpace{}%
\AgdaFunction{a}\AgdaSymbol{)}\AgdaSpace{}%
\AgdaOperator{\AgdaDatatype{≡}}\AgdaSpace{}%
\AgdaSymbol{(}\AgdaInductiveConstructor{lift}\AgdaSpace{}%
\AgdaOperator{\AgdaFunction{∘}}\AgdaSpace{}%
\AgdaOperator{\AgdaFunction{∣}}\AgdaSpace{}%
\AgdaBound{h}\AgdaSpace{}%
\AgdaOperator{\AgdaFunction{∣}}\AgdaSpace{}%
\AgdaOperator{\AgdaFunction{∘}}\AgdaSpace{}%
\AgdaField{lower}\AgdaSymbol{\{}\AgdaBound{𝓦}\AgdaSymbol{\})}\AgdaSpace{}%
\AgdaSymbol{(}\AgdaInductiveConstructor{lift}\AgdaSpace{}%
\AgdaFunction{a}\AgdaSymbol{)}\<%
\\
%
\>[2]\AgdaFunction{β}\AgdaSpace{}%
\AgdaSymbol{=}\AgdaSpace{}%
\AgdaFunction{ap}\AgdaSpace{}%
\AgdaSymbol{(λ}\AgdaSpace{}%
\AgdaBound{-}\AgdaSpace{}%
\AgdaSymbol{→}\AgdaSpace{}%
\AgdaInductiveConstructor{lift}\AgdaSpace{}%
\AgdaSymbol{(}\AgdaOperator{\AgdaFunction{∣}}\AgdaSpace{}%
\AgdaBound{h}\AgdaSpace{}%
\AgdaOperator{\AgdaFunction{∣}}\AgdaSpace{}%
\AgdaSymbol{(}\AgdaSpace{}%
\AgdaBound{-}\AgdaSpace{}%
\AgdaFunction{a}\AgdaSymbol{)))}\AgdaSpace{}%
\AgdaSymbol{(}\AgdaFunction{lower∼lift}\AgdaSpace{}%
\AgdaSymbol{\{}\AgdaBound{𝓦}\AgdaSymbol{\}}\AgdaSpace{}%
\AgdaSymbol{)}\<%
\\
%
\\[\AgdaEmptyExtraSkip]%
%
\>[2]\AgdaFunction{η}\AgdaSpace{}%
\AgdaSymbol{:}\AgdaSpace{}%
\AgdaBound{y}\AgdaSpace{}%
\AgdaOperator{\AgdaDatatype{≡}}\AgdaSpace{}%
\AgdaOperator{\AgdaFunction{∣}}\AgdaSpace{}%
\AgdaFunction{lh}\AgdaSpace{}%
\AgdaOperator{\AgdaFunction{∣}}\AgdaSpace{}%
\AgdaSymbol{(}\AgdaInductiveConstructor{lift}\AgdaSpace{}%
\AgdaFunction{a}\AgdaSymbol{)}\<%
\\
%
\>[2]\AgdaFunction{η}\AgdaSpace{}%
\AgdaSymbol{=}%
\>[324I]\AgdaBound{y}%
\>[22]\AgdaOperator{\AgdaFunction{≡⟨}}\AgdaSpace{}%
\AgdaSymbol{(}\AgdaFunction{extfun}\AgdaSpace{}%
\AgdaFunction{lift∼lower}\AgdaSymbol{)}\AgdaSpace{}%
\AgdaBound{y}\AgdaSpace{}%
\AgdaOperator{\AgdaFunction{⟩}}\<%
\\
\>[.][@{}l@{}]\<[324I]%
\>[6]\AgdaInductiveConstructor{lift}\AgdaSpace{}%
\AgdaSymbol{(}\AgdaField{lower}\AgdaSpace{}%
\AgdaBound{y}\AgdaSymbol{)}%
\>[22]\AgdaOperator{\AgdaFunction{≡⟨}}\AgdaSpace{}%
\AgdaFunction{ap}\AgdaSpace{}%
\AgdaInductiveConstructor{lift}\AgdaSpace{}%
\AgdaSymbol{(}\AgdaFunction{InvIsInv}\AgdaSpace{}%
\AgdaOperator{\AgdaFunction{∣}}\AgdaSpace{}%
\AgdaBound{h}\AgdaSpace{}%
\AgdaOperator{\AgdaFunction{∣}}\AgdaSpace{}%
\AgdaFunction{ζ}\AgdaSymbol{)}\AgdaOperator{\AgdaFunction{⁻¹}}\AgdaSpace{}%
\AgdaOperator{\AgdaFunction{⟩}}\<%
\\
%
\>[6]\AgdaInductiveConstructor{lift}\AgdaSpace{}%
\AgdaSymbol{(}\AgdaOperator{\AgdaFunction{∣}}\AgdaSpace{}%
\AgdaBound{h}\AgdaSpace{}%
\AgdaOperator{\AgdaFunction{∣}}\AgdaSpace{}%
\AgdaFunction{a}\AgdaSymbol{)}%
\>[22]\AgdaOperator{\AgdaFunction{≡⟨}}\AgdaSpace{}%
\AgdaFunction{β}\AgdaSpace{}%
\AgdaOperator{\AgdaFunction{⟩}}\<%
\\
%
\>[6]\AgdaOperator{\AgdaFunction{∣}}\AgdaSpace{}%
\AgdaFunction{lh}\AgdaSpace{}%
\AgdaOperator{\AgdaFunction{∣}}\AgdaSpace{}%
\AgdaSymbol{(}\AgdaInductiveConstructor{lift}\AgdaSpace{}%
\AgdaFunction{a}\AgdaSymbol{)}\AgdaSpace{}%
\AgdaOperator{\AgdaFunction{∎}}\<%
\\
%
\\[\AgdaEmptyExtraSkip]%
%
\\[\AgdaEmptyExtraSkip]%
%
\>[1]\AgdaFunction{lift-alg-hom-image}\AgdaSpace{}%
\AgdaSymbol{:}%
\>[351I]\AgdaSymbol{\{}\AgdaBound{𝓩}\AgdaSpace{}%
\AgdaBound{𝓦}\AgdaSpace{}%
\AgdaSymbol{:}\AgdaSpace{}%
\AgdaFunction{Universe}\AgdaSymbol{\}}\<%
\\
\>[.][@{}l@{}]\<[351I]%
\>[22]\AgdaSymbol{\{}\AgdaBound{𝑨}\AgdaSpace{}%
\AgdaSymbol{:}\AgdaSpace{}%
\AgdaFunction{Algebra}\AgdaSpace{}%
\AgdaBound{𝓧}\AgdaSpace{}%
\AgdaBound{𝑆}\AgdaSymbol{\}\{}\AgdaBound{𝑩}\AgdaSpace{}%
\AgdaSymbol{:}\AgdaSpace{}%
\AgdaFunction{Algebra}\AgdaSpace{}%
\AgdaBound{𝓨}\AgdaSpace{}%
\AgdaBound{𝑆}\AgdaSymbol{\}}\<%
\\
\>[1][@{}l@{\AgdaIndent{0}}]%
\>[2]\AgdaSymbol{→}%
\>[22]\AgdaBound{𝑩}\AgdaSpace{}%
\AgdaOperator{\AgdaFunction{is-hom-image-of}}\AgdaSpace{}%
\AgdaBound{𝑨}\<%
\\
%
\>[22]\AgdaComment{-----------------------------------------------}\<%
\\
%
\>[2]\AgdaSymbol{→}%
\>[22]\AgdaSymbol{(}\AgdaFunction{lift-alg}\AgdaSpace{}%
\AgdaBound{𝑩}\AgdaSpace{}%
\AgdaBound{𝓦}\AgdaSymbol{)}\AgdaSpace{}%
\AgdaOperator{\AgdaFunction{is-hom-image-of}}\AgdaSpace{}%
\AgdaSymbol{(}\AgdaFunction{lift-alg}\AgdaSpace{}%
\AgdaBound{𝑨}\AgdaSpace{}%
\AgdaBound{𝓩}\AgdaSymbol{)}\<%
\\
%
\\[\AgdaEmptyExtraSkip]%
%
\>[1]\AgdaFunction{lift-alg-hom-image}\AgdaSpace{}%
\AgdaSymbol{\{}\AgdaBound{𝓩}\AgdaSymbol{\}\{}\AgdaBound{𝓦}\AgdaSymbol{\}\{}\AgdaBound{𝑨}\AgdaSymbol{\}\{}\AgdaBound{𝑩}\AgdaSymbol{\}}\AgdaSpace{}%
\AgdaSymbol{((}\AgdaBound{𝑪}\AgdaSpace{}%
\AgdaOperator{\AgdaInductiveConstructor{,}}\AgdaSpace{}%
\AgdaBound{ϕ}\AgdaSpace{}%
\AgdaOperator{\AgdaInductiveConstructor{,}}\AgdaSpace{}%
\AgdaBound{ϕhom}\AgdaSpace{}%
\AgdaOperator{\AgdaInductiveConstructor{,}}\AgdaSpace{}%
\AgdaBound{ϕepic}\AgdaSymbol{)}\AgdaSpace{}%
\AgdaOperator{\AgdaInductiveConstructor{,}}\AgdaSpace{}%
\AgdaBound{C≅B}\AgdaSymbol{)}\AgdaSpace{}%
\AgdaSymbol{=}\<%
\\
\>[1][@{}l@{\AgdaIndent{0}}]%
\>[2]\AgdaSymbol{(}\AgdaFunction{lift-alg}\AgdaSpace{}%
\AgdaBound{𝑪}\AgdaSpace{}%
\AgdaBound{𝓦}\AgdaSpace{}%
\AgdaOperator{\AgdaInductiveConstructor{,}}\AgdaSpace{}%
\AgdaOperator{\AgdaFunction{∣}}\AgdaSpace{}%
\AgdaFunction{lϕ}\AgdaSpace{}%
\AgdaOperator{\AgdaFunction{∣}}\AgdaSpace{}%
\AgdaOperator{\AgdaInductiveConstructor{,}}\AgdaSpace{}%
\AgdaOperator{\AgdaFunction{∥}}\AgdaSpace{}%
\AgdaFunction{lϕ}\AgdaSpace{}%
\AgdaOperator{\AgdaFunction{∥}}\AgdaSpace{}%
\AgdaOperator{\AgdaInductiveConstructor{,}}\AgdaSpace{}%
\AgdaFunction{lϕepic}\AgdaSymbol{)}\AgdaSpace{}%
\AgdaOperator{\AgdaInductiveConstructor{,}}\AgdaSpace{}%
\AgdaFunction{lift-alg-iso}\AgdaSpace{}%
\AgdaBound{C≅B}\<%
\\
\>[2][@{}l@{\AgdaIndent{0}}]%
\>[3]\AgdaKeyword{where}\<%
\\
%
\>[3]\AgdaFunction{lϕ}\AgdaSpace{}%
\AgdaSymbol{:}\AgdaSpace{}%
\AgdaFunction{hom}\AgdaSpace{}%
\AgdaSymbol{(}\AgdaFunction{lift-alg}\AgdaSpace{}%
\AgdaBound{𝑨}\AgdaSpace{}%
\AgdaBound{𝓩}\AgdaSymbol{)}\AgdaSpace{}%
\AgdaSymbol{(}\AgdaFunction{lift-alg}\AgdaSpace{}%
\AgdaBound{𝑪}\AgdaSpace{}%
\AgdaBound{𝓦}\AgdaSymbol{)}\<%
\\
%
\>[3]\AgdaFunction{lϕ}\AgdaSpace{}%
\AgdaSymbol{=}\AgdaSpace{}%
\AgdaSymbol{(}\AgdaFunction{lift-alg-hom}\AgdaSpace{}%
\AgdaBound{𝓩}\AgdaSpace{}%
\AgdaBound{𝓦}\AgdaSpace{}%
\AgdaBound{𝑪}\AgdaSymbol{)}\AgdaSpace{}%
\AgdaSymbol{(}\AgdaBound{ϕ}\AgdaSpace{}%
\AgdaOperator{\AgdaInductiveConstructor{,}}\AgdaSpace{}%
\AgdaBound{ϕhom}\AgdaSymbol{)}\<%
\\
%
\\[\AgdaEmptyExtraSkip]%
%
\>[3]\AgdaFunction{lϕepic}\AgdaSpace{}%
\AgdaSymbol{:}\AgdaSpace{}%
\AgdaFunction{Epic}\AgdaSpace{}%
\AgdaOperator{\AgdaFunction{∣}}\AgdaSpace{}%
\AgdaFunction{lϕ}\AgdaSpace{}%
\AgdaOperator{\AgdaFunction{∣}}\<%
\\
%
\>[3]\AgdaFunction{lϕepic}\AgdaSpace{}%
\AgdaSymbol{=}\AgdaSpace{}%
\AgdaFunction{lift-of-alg-epic-is-epic}\AgdaSpace{}%
\AgdaBound{𝓩}\AgdaSpace{}%
\AgdaBound{𝑪}\AgdaSpace{}%
\AgdaSymbol{(}\AgdaBound{ϕ}\AgdaSpace{}%
\AgdaOperator{\AgdaInductiveConstructor{,}}\AgdaSpace{}%
\AgdaBound{ϕhom}\AgdaSymbol{)}\AgdaSpace{}%
\AgdaBound{ϕepic}\<%
\end{code}

