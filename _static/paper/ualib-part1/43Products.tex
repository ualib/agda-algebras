% -*- TeX-master: "ualib-part1.tex" -*-
%%% Local Variables: 
%%% mode: latex
%%% TeX-engine: 'xetex
%%% End:
We begin this module by assuming a signature \ab{𝑆} \as : \af{Signature} \ab 𝓞 \ab 𝓥 which is then present and available throughout the module.  Thus the start of the \ualibhtml{Algebras.Products} module is declared with the line
\AgdaKeyword{module}\AgdaSpace{}%
\AgdaModule{Algebras.Products}\AgdaSpace{}%
\AgdaSymbol{\{}\AgdaBound{𝑆}\AgdaSpace{}%
\AgdaSymbol{:}\AgdaSpace{}%
\AgdaFunction{Signature}\AgdaSpace{}%
\AgdaGeneralizable{𝓞}\AgdaSpace{}%
\AgdaGeneralizable{𝓥}\AgdaSymbol{\}}\AgdaSpace{}%
\AgdaKeyword{where}.%
% We begin this module by assuming a signature \ab{𝑆} \as : \af{Signature} \ab 𝓞 \ab 𝓥 which is then present and available throughout the module.  Because of this, in contrast to descriptions of previous modules, we present the first few lines of the \ualibhtml{Algebras.Products} module in full.
% \ccpad
% \begin{code}%
% \>[0]\AgdaSymbol{\{-\#}\AgdaSpace{}%
% \AgdaKeyword{OPTIONS}\AgdaSpace{}%
% \AgdaPragma{--without-K}\AgdaSpace{}%
% \AgdaPragma{--exact-split}\AgdaSpace{}%
% \AgdaPragma{--safe}\AgdaSpace{}%
% \AgdaSymbol{\#-\}}\<%
% \\
% %
% \\[\AgdaEmptyExtraSkip]%
% \>[0]\AgdaKeyword{open}\AgdaSpace{}%
% \AgdaKeyword{import}\AgdaSpace{}%
% \AgdaModule{Algebras.Signatures}\AgdaSpace{}%
% \AgdaKeyword{using}\AgdaSpace{}%
% \AgdaSymbol{(}\AgdaFunction{Signature}\AgdaSymbol{;}\AgdaSpace{}%
% \AgdaGeneralizable{𝓞}\AgdaSymbol{;}\AgdaSpace{}%
% \AgdaGeneralizable{𝓥}\AgdaSymbol{)}\<%
% \\
% %
% % \\[\AgdaEmptyExtraSkip]%
% \>[0]\AgdaKeyword{module}\AgdaSpace{}%
% \AgdaModule{Algebras.Products}\AgdaSpace{}%
% \AgdaSymbol{\{}\AgdaBound{𝑆}\AgdaSpace{}%
% \AgdaSymbol{:}\AgdaSpace{}%
% \AgdaFunction{Signature}\AgdaSpace{}%
% \AgdaGeneralizable{𝓞}\AgdaSpace{}%
% \AgdaGeneralizable{𝓥}\AgdaSymbol{\}}\AgdaSpace{}%
% \AgdaKeyword{where}\<%
% \\
% %
% % \\[\AgdaEmptyExtraSkip]%
% \>[0]\AgdaKeyword{open}\AgdaSpace{}%
% \AgdaKeyword{import}\AgdaSpace{}%
% \AgdaModule{Algebras.Algebras}\AgdaSpace{}%
% \AgdaKeyword{hiding}\AgdaSpace{}%
% \AgdaSymbol{(}\AgdaGeneralizable{𝓞}\AgdaSymbol{;}\AgdaSpace{}%
% \AgdaGeneralizable{𝓥}\AgdaSymbol{)}\AgdaSpace{}%
% \AgdaKeyword{public}\<%
% \end{code}
% \ccpad
% Notice that we must import the \af{Signature} type from \ualibhtml{Algebras.Signatures} first so that we can use it to declare the signature \AgdaBound{𝑆} as a parameter of the \ualibhtml{Algebras.Products} module.

% \>[0]\AgdaKeyword{module}\AgdaSpace{}%
% \AgdaModule{Algebras.Products}\AgdaSpace{}%
% \AgdaSymbol{\{}\AgdaBound{𝑆}\AgdaSpace{}%
% \AgdaSymbol{:}\AgdaSpace{}%
% \AgdaFunction{Signature}\AgdaSpace{}%
% \AgdaGeneralizable{𝓞}\AgdaSpace{}%
% \AgdaGeneralizable{𝓥}\AgdaSymbol{\}}\AgdaSpace{}%
% \AgdaKeyword{where}\<%
% \end{code}

In the \ualib the \defn{product of} \ab 𝑆-\defn{algebras} is represented by the following type.\footnote{Alternative equivalent notation for the domain of the product is \as ∀~\ab i~\as →~\af ∣~\ab 𝒜~\ab i~\af ∣.}
\ccpad
\begin{code}%
% \>[0]\AgdaKeyword{module}\AgdaSpace{}%
% \AgdaModule{\AgdaUnderscore{}}\AgdaSpace{}%
% \AgdaSymbol{\{}\AgdaBound{𝓤}\AgdaSpace{}%
% \AgdaBound{𝓘}\AgdaSpace{}%
% \AgdaSymbol{:}\AgdaSpace{}%
% \AgdaPostulate{Universe}\AgdaSymbol{\}\{}\AgdaBound{I}\AgdaSpace{}%
% \AgdaSymbol{:}\AgdaSpace{}%
% \AgdaBound{𝓘}\AgdaSpace{}%
% \AgdaOperator{\AgdaFunction{̇}}\AgdaSpace{}%
% \AgdaSymbol{\}}\AgdaSpace{}%
% \AgdaKeyword{where}\<%
% \\
% %
% \\[\AgdaEmptyExtraSkip]%
\>[0][@{}l@{\AgdaIndent{0}}]%
\>[1]\AgdaFunction{⨅}\AgdaSpace{}%
\AgdaSymbol{:}\AgdaSpace{}%
\AgdaSymbol{(}\AgdaBound{𝒜}\AgdaSpace{}%
\AgdaSymbol{:}\AgdaSpace{}%
\AgdaBound{I}\AgdaSpace{}%
\AgdaSymbol{→}\AgdaSpace{}%
\AgdaFunction{Algebra}\AgdaSpace{}%
\AgdaBound{𝓤}\AgdaSpace{}%
\AgdaBound{𝑆}\AgdaSpace{}%
\AgdaSymbol{)}\AgdaSpace{}%
\AgdaSymbol{→}\AgdaSpace{}%
\AgdaFunction{Algebra}\AgdaSpace{}%
\AgdaSymbol{(}\AgdaBound{𝓘}\AgdaSpace{}%
\AgdaOperator{\AgdaPrimitive{⊔}}\AgdaSpace{}%
\AgdaBound{𝓤}\AgdaSymbol{)}\AgdaSpace{}%
\AgdaBound{𝑆}\<%
\\
%
\\[\AgdaEmptyExtraSkip]%
%
\>[1]\AgdaFunction{⨅}\AgdaSpace{}%
\AgdaBound{𝒜}\AgdaSpace{}%
\AgdaSymbol{=}%
\>[51I]\AgdaSymbol{(}\AgdaFunction{Π}\AgdaSpace{}%
\AgdaBound{i}\AgdaSpace{}%
\AgdaFunction{꞉}\AgdaSpace{}%
\AgdaBound{I}\AgdaSpace{}%
\AgdaFunction{,}\AgdaSpace{}%
\AgdaOperator{\AgdaFunction{∣}}\AgdaSpace{}%
\AgdaBound{𝒜}\AgdaSpace{}%
\AgdaBound{i}\AgdaSpace{}%
\AgdaOperator{\AgdaFunction{∣}}\AgdaSymbol{)}\AgdaSpace{}%
\AgdaOperator{\AgdaInductiveConstructor{,}}%
\>[43]\AgdaComment{-- domain of the product algebra}\<%
\\
\>[.][@{}l@{}]\<[51I]%
\>[7]\AgdaSymbol{λ}\AgdaSpace{}%
\AgdaBound{𝑓}\AgdaSpace{}%
\AgdaBound{𝑎}\AgdaSpace{}%
\AgdaBound{i}\AgdaSpace{}%
\AgdaSymbol{→}\AgdaSpace{}%
\AgdaSymbol{(}\AgdaBound{𝑓}\AgdaSpace{}%
\AgdaOperator{\AgdaFunction{̂}}\AgdaSpace{}%
\AgdaBound{𝒜}\AgdaSpace{}%
\AgdaBound{i}\AgdaSymbol{)}\AgdaSpace{}%
\AgdaSymbol{λ}\AgdaSpace{}%
\AgdaBound{x}\AgdaSpace{}%
\AgdaSymbol{→}\AgdaSpace{}%
\AgdaBound{𝑎}\AgdaSpace{}%
\AgdaBound{x}\AgdaSpace{}%
\AgdaBound{i}%
\>[43]\AgdaComment{-- basic operations of the product algebra}\<%
\end{code}
\ccpad
Other modules of the \ualib will use the foregoing product type exclusively. However, for completeness, here is how one constructs a product algebra type when the factor algebras inhabit the record type \AgdaRecord{algebra}.
\ccpad
\begin{code}%
% %
% \\[\AgdaEmptyExtraSkip]%
% \>[0]\AgdaComment{-- product for algebras of record type}\<%
% \\
\>[0][@{}l@{\AgdaIndent{0}}]%
\>[1]\AgdaKeyword{open}\AgdaSpace{}%
\AgdaModule{algebra}\<%
\\
\>[1]\AgdaFunction{⨅'}\AgdaSpace{}%
\AgdaSymbol{:}\AgdaSpace{}%
% \AgdaSymbol{\{}\AgdaBound{𝓤}\AgdaSpace{}%
% \AgdaBound{𝓘}\AgdaSpace{}%
% \AgdaSymbol{:}\AgdaSpace{}%
% \AgdaPostulate{Universe}\AgdaSymbol{\}\{}\AgdaBound{I}\AgdaSpace{}%
% \AgdaSymbol{:}\AgdaSpace{}%
% \AgdaBound{𝓘}\AgdaSpace{}%
% \AgdaOperator{\AgdaFunction{̇}}\AgdaSymbol{\}\AgdaSpace{}%
\AgdaSymbol{(}\AgdaBound{𝒜}\AgdaSpace{}%
\AgdaSymbol{:}\AgdaSpace{}%
\AgdaBound{I}\AgdaSpace{}%
\AgdaSymbol{→}\AgdaSpace{}%
\AgdaRecord{algebra}\AgdaSpace{}%
\AgdaBound{𝓤}\AgdaSpace{}%
\AgdaBound{𝑆}\AgdaSymbol{)}\AgdaSpace{}%
\AgdaSymbol{→}\AgdaSpace{}%
\AgdaRecord{algebra}\AgdaSpace{}%
\AgdaSymbol{(}\AgdaBound{𝓘}\AgdaSpace{}%
\AgdaOperator{\AgdaPrimitive{⊔}}\AgdaSpace{}%
\AgdaBound{𝓤}\AgdaSymbol{)}\AgdaSpace{}%
\AgdaBound{𝑆}\<%
\\
%
% \\[\AgdaEmptyExtraSkip]%
\>[1]\AgdaFunction{⨅'}\AgdaSpace{}%
\AgdaBound{𝒜}\AgdaSpace{}%
\AgdaSymbol{=}\AgdaSpace{}%
\AgdaKeyword{record}%
\>[100I]\AgdaSymbol{\{}\AgdaSpace{}%
\AgdaField{univ}\AgdaSpace{}%
\AgdaSymbol{=}\AgdaSpace{}%
\AgdaSymbol{∀}\AgdaSpace{}%
\AgdaBound{i}\AgdaSpace{}%
\AgdaSymbol{→}\AgdaSpace{}%
\AgdaField{univ}\AgdaSpace{}%
\AgdaSymbol{(}\AgdaBound{𝒜}\AgdaSpace{}%
\AgdaBound{i}\AgdaSymbol{)}\AgdaSpace{}\AgdaSymbol{;}%
\>[55]\AgdaComment{-- domain}\<%
\\
%\AgdaTab{100}%\>[.][@{}l@{}]\<[100I]%
\>[100I][@{}l@{\AgdaIndent{0}}]%
\>[15]\AgdaField{op}\AgdaSpace{}%
\AgdaSymbol{=}\AgdaSpace{}%
\AgdaSymbol{λ}\AgdaSpace{}%
\AgdaBound{𝑓}\AgdaSpace{}%
\AgdaBound{𝑎}\AgdaSpace{}%
\AgdaBound{i}\AgdaSpace{}%
\AgdaSymbol{→}\AgdaSpace{}%
\AgdaSymbol{(}\AgdaField{op}\AgdaSpace{}%
\AgdaSymbol{(}\AgdaBound{𝒜}\AgdaSpace{}%
\AgdaBound{i}\AgdaSymbol{))}\AgdaSpace{}%
\AgdaBound{𝑓}\AgdaSpace{}%
\AgdaSymbol{λ}\AgdaSpace{}%
\AgdaBound{x}\AgdaSpace{}%
\AgdaSymbol{→}\AgdaSpace{}%
\AgdaBound{𝑎}\AgdaSpace{}%
\AgdaBound{x}\AgdaSpace{}%
\AgdaBound{i}\AgdaSpace{}\AgdaSymbol{\}}%
\>[55]\AgdaComment{-- basic operations}\<%
% \\
% \>[10]\AgdaSymbol{\}}\<%
\end{code}

\subsubsection{Products of classes of algebras}\label{products-of-classes-of-algebras}
\context{\AgdaSymbol{\{}\AgdaBound{𝓤}\AgdaSpace{}%
\AgdaBound{𝓧}\AgdaSpace{}%
\AgdaSymbol{:}\AgdaSpace{}%
\AgdaPostulate{Universe}\AgdaSymbol{\}\{}\AgdaBound{X}\AgdaSpace{}%
\AgdaSymbol{:}\AgdaSpace{}%
\AgdaBound{𝓧}\AgdaSpace{}%
\AgdaOperator{\AgdaFunction{̇}}\AgdaSymbol{\}}}

An arbitrary class \ab 𝒦 of algebras is represented as a predicate over the type \AgdaFunction{Algebra}\AgdaSpace{}\AgdaBound{𝓤}\AgdaSpace{}\AgdaBound{𝑆}, for some universe \AgdaBound{𝓤} and signature \AgdaBound{𝑆}. That is, \ab 𝒦 \as : \AgdaFunction{Pred}\AgdaSpace{}%
\AgdaSymbol{(}\AgdaFunction{Algebra}\AgdaSpace{}%
\AgdaBound{𝓤}\AgdaSpace{}%
\AgdaBound{𝑆}\AgdaSymbol{)}\AgdaSpace{}%
\AgdaUnderscore.\footnote{%
The underscore is merely a placeholder for the universe of the predicate type and needn't concern us here.} Later we will formally state and prove that the product of all subalgebras of algebras in such a class belongs to the class \af{SP}(\ab 𝒦) of subalgebras of products of algebras in \ab 𝒦. That is, \af ⨅ \af S(\ab 𝒦) \af ∈ \af{SP}(\ab 𝒦).This turns out to be a nontrivial exercise. In fact, it is not immediately clear (at least not to this author) how to even express the product of an entire class of algebras as a dependent type. (We urge the reader to take a moment and try this before reading on.)

After some concentrated thought and a few failed attempts, eventually the right type reveals itself and, as is often the case, the thing we sought seems almost obvious once we lay our hands on it.\footnote{%
This was our own experience, but readers are encouraged to try to derive for themselves a type that represents the product of all things satisfying a predicate (e.g., over a type like \AgdaFunction{Algebra}\AgdaSpace{}\AgdaBound{𝓤}\AgdaSpace{}\AgdaBound{𝑆}, or over an arbitrary type). It is a good exercise.} The solution we propose in the \ualib is the \af{class-product} type whose construction is the main goal of this section.

First, let us establish a notational convention that is used in many of the later modules of the \ualib. Given a signature \ab{𝑆} \as : \af{Signature} \ab  𝓞 \ab 𝓥, the type \af{Algebra} \ab 𝓤 \ab 𝑆 has universe level \ab{𝓞} \ap ⊔ \ab 𝓥 \ap ⊔ \ab 𝓤 \af ⁺. Such levels occur very frequently and the \ab{𝓞} \ap ⊔ \ab 𝓥 part remains fixed since \ab{𝓞} and \ab 𝓥 always denote the levels of operation and arity types, respectively. Thus, we define the following shorthand: \AgdaFunction{ov}\AgdaSpace{}%
\AgdaBound{𝓤}\AgdaSpace{}%
\AgdaSymbol{:=}\AgdaSpace{}%
\AgdaBound{𝓞}\AgdaSpace{}%
\AgdaOperator{\AgdaPrimitive{⊔}}\AgdaSpace{}%
\AgdaBound{𝓥}\AgdaSpace{}%
\AgdaOperator{\AgdaPrimitive{⊔}}\AgdaSpace{}%
\AgdaBound{𝓤}\AgdaSpace{}%
\AgdaOperator{\AgdaPrimitive{⁺}}.

% We now proceed to develop types that will represent products of classes of algebras. First
Next, we define the type \AgdaFunction{ℑ} that will serve as the \defn{index} for the class, as well as for the product of all members of the class.
\ccpad
\begin{code}%
\>[0][@{}l@{\AgdaIndent{0}}]%
\>[1]\AgdaFunction{ℑ}\AgdaSpace{}%
\AgdaSymbol{:}\AgdaSpace{}%
\AgdaFunction{Pred}\AgdaSpace{}%
\AgdaSymbol{(}\AgdaFunction{Algebra}\AgdaSpace{}%
\AgdaBound{𝓤}\AgdaSpace{}%
\AgdaBound{𝑆}\AgdaSymbol{)(}\AgdaFunction{ov}\AgdaSpace{}%
\AgdaBound{𝓤}\AgdaSymbol{)}\AgdaSpace{}%
\AgdaSymbol{→}\AgdaSpace{}%
\AgdaSymbol{(}\AgdaBound{𝓧}\AgdaSpace{}%
\AgdaOperator{\AgdaPrimitive{⊔}}\AgdaSpace{}%
\AgdaFunction{ov}\AgdaSpace{}%
\AgdaBound{𝓤}\AgdaSymbol{)}\AgdaSpace{}%
\AgdaOperator{\AgdaFunction{̇}}\<%
\\
%
% \\[\AgdaEmptyExtraSkip]%
%
\>[1]\AgdaFunction{ℑ}\AgdaSpace{}%
\AgdaBound{𝒦}\AgdaSpace{}%
\AgdaSymbol{=}\AgdaSpace{}%
\AgdaFunction{Σ}\AgdaSpace{}%
\AgdaBound{𝑨}\AgdaSpace{}%
\AgdaFunction{꞉}\AgdaSpace{}%
\AgdaSymbol{(}\AgdaFunction{Algebra}\AgdaSpace{}%
\AgdaBound{𝓤}\AgdaSpace{}%
\AgdaBound{𝑆}\AgdaSymbol{)}\AgdaSpace{}%
\AgdaFunction{,}\AgdaSpace{}%
\AgdaSymbol{(}\AgdaBound{𝑨}\AgdaSpace{}%
\AgdaOperator{\AgdaFunction{∈}}\AgdaSpace{}%
\AgdaBound{𝒦}\AgdaSymbol{)}\AgdaSpace{}%
\AgdaOperator{\AgdaFunction{×}}\AgdaSpace{}%
\AgdaSymbol{(}\AgdaBound{X}\AgdaSpace{}%
\AgdaSymbol{→}\AgdaSpace{}%
\AgdaOperator{\AgdaFunction{∣}}\AgdaSpace{}%
\AgdaBound{𝑨}\AgdaSpace{}%
\AgdaOperator{\AgdaFunction{∣}}\AgdaSymbol{)}\<%
\end{code}
\ccpad
Each inhabitant \ab{i} \as : \af ℑ \ab 𝒦 denotes a triple of the form (\ab{𝑨} , \ab p , \ab h), where \ab{𝑨} \as : \af{Algebra} \ab 𝓤 \ab 𝑆, \ab{p}~\as :~\ab 𝑨~\af ∈~\ab 𝒦, and \ab{h} \as : \ab X \as → \af ∣ \ab 𝑨 \af ∣.
% Thus, if \ab{p} \as : \ab 𝑨 \af ∈ \ab 𝒦 and \ab{h} \as : \ab X \as → \af ∣ \ab 𝑨 \af ∣, then the triple (\ab{𝑨} , \ab p , \ab h) inhabits the type \af ℑ \ab 𝒦, and w
We view such a triple as an \emph{index} identifying a particular member, \ab{𝑨}, of the class \ab 𝒦, along with proof of membership \ab p, and a \defn{context map} \ab h.

% The second component of this Sigma type is (\ab{𝑨} \af ∈ \ab 𝒦) \af × (\ab X \as → \af ∣ \ab 𝑨 \af ∣). 
In previous versions of the \ualib this second component of the Sigma type defining \af ℑ was simply \ab{𝑨} \af ∈ \ab 𝒦, and did not include \ab X \as → \af ∣ \ab 𝑨 \af ∣. However, we realized that adding the latter is quite useful. Later we will see exactly why, but for now suffice it to say that a map of type \ab X \as → \af ∣ \ab 𝑨 \af ∣ may be viewed abstractly as an \emph{ambient context}, or more concretely, as an assignment of \emph{values} in \af ∣ \ab 𝑨 \af ∣ to \emph{variable symbols} in \ab X.  When computing with or reasoning about products, while we don't want to rigidly impose a context in advance, we do want to lay our hands on whatever context is ultimately assumed.  Including the ``context map'' inside the index type \af ℑ of the product turns out to be a convenient way to achieve this flexibility.

Taking the product over the index type \af ℑ requires a function that maps an index \ab{i} \as : \af ℑ to the corresponding algebra.  the function mapping an index to the corresponding algebra is simply the first projection.
\ccpad
\begin{code}%
\>[1]\AgdaFunction{𝔄}\AgdaSpace{}%
\AgdaSymbol{:}\AgdaSpace{}%
\AgdaSymbol{(}\AgdaBound{𝒦}\AgdaSpace{}%
\AgdaSymbol{:}\AgdaSpace{}%
\AgdaFunction{Pred}\AgdaSpace{}%
\AgdaSymbol{(}\AgdaFunction{Algebra}\AgdaSpace{}%
\AgdaBound{𝓤}\AgdaSpace{}%
\AgdaBound{𝑆}\AgdaSymbol{)(}\AgdaFunction{ov}\AgdaSpace{}%
\AgdaBound{𝓤}\AgdaSymbol{))}\AgdaSpace{}%
\AgdaSymbol{→}\AgdaSpace{}%
\AgdaFunction{ℑ}\AgdaSpace{}%
\AgdaBound{𝒦}\AgdaSpace{}%
\AgdaSymbol{→}\AgdaSpace{}%
\AgdaFunction{Algebra}\AgdaSpace{}%
\AgdaBound{𝓤}\AgdaSpace{}%
\AgdaBound{𝑆}\<%
\\
%
% \\[\AgdaEmptyExtraSkip]%
%
\>[1]\AgdaFunction{𝔄}\AgdaSpace{}%
\AgdaBound{𝒦}\AgdaSpace{}%
\AgdaSymbol{=}\AgdaSpace{}%
\AgdaSymbol{λ}\AgdaSpace{}%
\AgdaSymbol{(}\AgdaBound{i}\AgdaSpace{}%
\AgdaSymbol{:}\AgdaSpace{}%
\AgdaSymbol{(}\AgdaFunction{ℑ}\AgdaSpace{}%
\AgdaBound{𝒦}\AgdaSymbol{))}\AgdaSpace{}%
\AgdaSymbol{→}\AgdaSpace{}%
\AgdaOperator{\AgdaFunction{∣}}\AgdaSpace{}%
\AgdaBound{i}\AgdaSpace{}%
\AgdaOperator{\AgdaFunction{∣}}\<%
\end{code}
\ccpad
Finally, we define \af{class-product} which represents the product of all members of \ab 𝒦.
\ccpad
\begin{code}%
\>[1]\AgdaFunction{class-product}\AgdaSpace{}%
\AgdaSymbol{:}\AgdaSpace{}%
\AgdaFunction{Pred}\AgdaSpace{}%
\AgdaSymbol{(}\AgdaFunction{Algebra}\AgdaSpace{}%
\AgdaBound{𝓤}\AgdaSpace{}%
\AgdaBound{𝑆}\AgdaSymbol{)(}\AgdaFunction{ov}\AgdaSpace{}%
\AgdaBound{𝓤}\AgdaSymbol{)}\AgdaSpace{}%
\AgdaSymbol{→}\AgdaSpace{}%
\AgdaFunction{Algebra}\AgdaSpace{}%
\AgdaSymbol{(}\AgdaBound{𝓧}\AgdaSpace{}%
\AgdaOperator{\AgdaPrimitive{⊔}}\AgdaSpace{}%
\AgdaFunction{ov}\AgdaSpace{}%
\AgdaBound{𝓤}\AgdaSymbol{)}\AgdaSpace{}%
\AgdaBound{𝑆}\<%
\\
%
% \\[\AgdaEmptyExtraSkip]%
%
\>[1]\AgdaFunction{class-product}\AgdaSpace{}%
\AgdaBound{𝒦}\AgdaSpace{}%
\AgdaSymbol{=}\AgdaSpace{}%
\AgdaFunction{⨅}\AgdaSpace{}%
\AgdaSymbol{(}\AgdaSpace{}%
\AgdaFunction{𝔄}\AgdaSpace{}%
\AgdaBound{𝒦}\AgdaSpace{}%
\AgdaSymbol{)}\<%
\end{code}
\ccpad
% Alternatively, we could have defined the class product in a way that
% explicitly displays the index, like so.
% \ccpad
% \begin{code}%
% \>[0][@{}l@{\AgdaIndent{1}}]%
% \>[1]\AgdaFunction{class-product'}\AgdaSpace{}%
% \AgdaSymbol{:}\AgdaSpace{}%
% \AgdaFunction{Pred}\AgdaSpace{}%
% \AgdaSymbol{(}\AgdaFunction{Algebra}\AgdaSpace{}%
% \AgdaBound{𝓤}\AgdaSpace{}%
% \AgdaBound{𝑆}\AgdaSymbol{)(}\AgdaFunction{ov}\AgdaSpace{}%
% \AgdaBound{𝓤}\AgdaSymbol{)}\AgdaSpace{}%
% \AgdaSymbol{→}\AgdaSpace{}%
% \AgdaFunction{Algebra}\AgdaSpace{}%
% \AgdaSymbol{(}\AgdaBound{𝓧}\AgdaSpace{}%
% \AgdaOperator{\AgdaPrimitive{⊔}}\AgdaSpace{}%
% \AgdaFunction{ov}\AgdaSpace{}%
% \AgdaBound{𝓤}\AgdaSymbol{)}\AgdaSpace{}%
% \AgdaBound{𝑆}\<%
% \\
% %
% \\[\AgdaEmptyExtraSkip]%
% %
% \>[1]\AgdaFunction{class-product'}\AgdaSpace{}%
% \AgdaBound{𝒦}\AgdaSpace{}%
% \AgdaSymbol{=}\AgdaSpace{}%
% \AgdaFunction{⨅}\AgdaSpace{}%
% \AgdaSymbol{λ}\AgdaSpace{}%
% \AgdaSymbol{(}\AgdaBound{i}\AgdaSpace{}%
% \AgdaSymbol{:}\AgdaSpace{}%
% \AgdaSymbol{(}\AgdaFunction{Σ}\AgdaSpace{}%
% \AgdaBound{𝑨}\AgdaSpace{}%
% \AgdaFunction{꞉}\AgdaSpace{}%
% \AgdaSymbol{(}\AgdaFunction{Algebra}\AgdaSpace{}%
% \AgdaBound{𝓤}\AgdaSpace{}%
% \AgdaBound{𝑆}\AgdaSymbol{)}\AgdaSpace{}%
% \AgdaFunction{,}\AgdaSpace{}%
% \AgdaSymbol{(}\AgdaBound{𝑨}\AgdaSpace{}%
% \AgdaOperator{\AgdaFunction{∈}}\AgdaSpace{}%
% \AgdaBound{𝒦}\AgdaSymbol{)}\AgdaSpace{}%
% \AgdaOperator{\AgdaFunction{×}}\AgdaSpace{}%
% \AgdaSymbol{(}\AgdaBound{X}\AgdaSpace{}%
% \AgdaSymbol{→}\AgdaSpace{}%
% \AgdaOperator{\AgdaFunction{∣}}\AgdaSpace{}%
% \AgdaBound{𝑨}\AgdaSpace{}%
% \AgdaOperator{\AgdaFunction{∣}}\AgdaSymbol{)))}\AgdaSpace{}%
% \AgdaSymbol{→}\AgdaSpace{}%
% \AgdaOperator{\AgdaFunction{∣}}\AgdaSpace{}%
% \AgdaBound{i}\AgdaSpace{}%
% \AgdaOperator{\AgdaFunction{∣}}\<%
% \end{code}
% \ccpad
Observe that \af 𝔄 \ab 𝒦 (\ab 𝑨 , \ab p , \ab h) (which is simply \ab{𝑨}) may be viewed as the \defn{projection of the product} \af ⨅ \af 𝔄 \ab 𝒦 onto the ``(\ab 𝑨 , \ab p , \ab h)-th component'' of the product.

% If \ab{p} \as : \ab 𝑨 \af ∈ \ab 𝒦 and \ab{h} \as 
% : \ab X \as → \af ∣ \ab 𝑨 \af ∣, then we can think of the triple (\ab{𝑨} , \ab p , \ab h) \af ∈ \af ℑ \ab 𝒦 as an index over the class, and so we can think of \af 𝔄 (\ab 𝑨 , \ab p , \ab h) (which is simply \ab{𝑨}) as the projection of the product \af ⨅ ( \af 𝔄 \ab 𝒦 ) onto the (\ab 𝑨 , \ab p , \ab h)-th component.
