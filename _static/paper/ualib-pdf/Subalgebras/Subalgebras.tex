Here we define a \defn{subalgebra} of an algebra as well as the collection of all subalgebras of a given class of algebras.
\ccpad
\begin{code}%
\end{code}


\subsubsection{Subalgebra type}\label{sssec:subalgebra-type}
Given algebras \ab 𝑨 \as : \af{Algebra} \ab 𝓦 \ab 𝑆 and \ab 𝑩 \as : \af{Algebra} \ab 𝓤 \ab 𝑆, we say that \ab 𝑩 is a \textbf{subalgebra} of \ab 𝑨, and we write \ab 𝑩 \af{IsSubalgebraOf} \ab 𝑨 just in case \ab 𝑩 can be embedded in \ab 𝑨; in other terms, there exists a map \ab h \as : \af ∣ \ab 𝑨 \af ∣ \as → \af ∣ \ab 𝑩 \af ∣ from the universe of \ab 𝑨 to the universe of \ab 𝑩 such that \ab h is an embedding (i.e., \af{is-embedding} \ab h holds) and \ab h is a homomorphism from \ab 𝑨 to \ab 𝑩.
\ccpad
\begin{code}%
\end{code}

\subsubsection{Syntactic sugar}\label{syntactic-sugar}

We use the convenient \af{≤} notation for the subalgebra relation.
\ccpad
\begin{code}%
\end{code}

\subsubsection{Subalgebras of a class}\label{sssec:subalgebras-of-a-class}

\begin{code}%
\end{code}


\subsubsection{Subalgebra lemmas}\label{sssec:subalgebra-lemmas}

Here are a number of useful facts about subalgebras. Many of them seem redundant, and they are to some extent. However, each one differs slightly from the next, if only with respect to the explicitness or implicitness of their arguments. The aim is to make it as convenient as possible to apply the lemmas in different situations. (We're in the
UALib utility closet now, and elegance is not the priority.)
\ccpad
\begin{code}%
\end{code}
