To support formalization in type theory of research level mathematics in universal algebra and related fields, we present the Agda Universal Algebra Library (\agdaualib), a software library containing formal statements and proofs of the core definitions and results of universal algebra. The \ualib is written in \agda~\cite{Norell:2009}, a programming language and proof assistant based on \MLTT (\mltt) that supports dependent and inductive types.


\subsection{Motivation}\label{sec:motivation}
The seminal idea for the \agdaualib project was the observation that, on the one hand, a number of fundamental constructions in universal algebra can be defined recursively, and theorems about them proved by structural induction, while, on the other hand, inductive and dependent types make possible very precise formal representations of recursively defined objects, which often admit elegant constructive proofs of properties of such objects.  An important feature of such proofs in type theory is that they are total functional programs and, as such, they are computable, composable, and machine-verifiable.
% These observations suggested that there would be much to gain from implementing universal algebra in a language, such as Martin-L\"of type theory, that features dependent and inductive types.

Finally, our own research experience has taught us that a proof assistant and programming language (like Agda), when equipped with specialized libraries and domain-specific tactics to automate the proof idioms of our field, can be an extremely powerful and effective asset. As such we believe that proof assistants and their supporting libraries will eventually become indispensable tools in the working mathematician's toolkit.

\subsection{Attributions and Contributions}\label{sec:contributions}
The mathematical results described in this paper have well known \emph{informal} proofs. Our main contribution is the formalization, mechanization, and verification of the statements and proofs of these results in dependent type theory using Agda.

Unless explicitly stated otherwise, the Agda source code described in this paper is due to the author, with the following caveat: the \ualib depends on the \typetopology library of \MartinEscardo~\cite{MHE}. For convenience, we refer to Escard\'o's library as \typtop throughout the paper. For the sake of completeness and clarity, and to keep the paper mostly self-contained, we repeat some definitions from \typtop, but in each instance we cite the original source.\footnote{In the \ualib, such instances occur only inside hidden modules that are never actually used, followed immediately by a statement that imports the code in question from its original source.}
